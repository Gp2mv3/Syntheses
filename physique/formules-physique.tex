\documentclass[a4paper,10pt]{article}
\usepackage[utf8x]{inputenc}
\usepackage{amsmath}

%opening
\title{Formulaire de physique 1 FSA11}
\author{Nicolas Cognaux}

\begin{document}

\maketitle

\newpage
\part{Formules}

\section{Components of vectors}
\subsection{Vector decomposition}
\[ A = \sqrt{{A_x}^2 + {A_y}^2} \]
\[ \theta = \arctan{\frac{A_y}{A_x}} \]
\subsection{Scalar product}
\[ \vec{A}.\vec{B} = AB\cos(\theta) = |A||B|\cos{theta} \]
\[ \vec{A}.\vec{B} = A_x B_x + A_y B_y + A_z B_z \]


%%TOFIX !!!!!!!!!!!!!!!!!!!!!!!!!!!!!!!!!!!!!!!!!!!!!!! MATRIX
\subsection{Vector product}
\[ C = AB\sin{\theta} \]
\[ \vec{C} = \lgroup \begin{pmatrix}
	  A_y&A_z \\
	  B_y&B_z
          \end{pmatrix}, - \begin{pmatrix}
	  A_b&A_z \\
	  B_x&B_z
          \end{pmatrix}, \begin{pmatrix}
	  A_x&A_y \\
	  B_x&B_y
          \end{pmatrix},\rgroup
\]

\section{Specifics movements}
\subsection{Movment along a strait line MRU/MRUA}
\[ v_x = \displaystyle {\lim_{\Delta t \rightarrow 0 }} \frac{\Delta x}{\Delta t} = \frac{dx}{dt} \]
\[ a_x = \displaystyle {\lim_{\Delta t \rightarrow 0 }} \frac{v_x}{\Delta t} = \frac{dv_x}{dt} \]

\[ v_x = v_{0x} + a_{x}t \]
\[ x = x_0 + v_{0x} t + \frac{a t^2}{2}\]

\subsection{Projectile motion}

\[ x = (v_0 \cos{ \alpha_0 }) t \]
\[ y = (v_0 \sin{ \alpha_0 }) t - \frac{1}{2} g t^2 \]
\[ v_x = v_0 \cos{ \alpha_0 } \]
\[ v_y = v_0 \sin{ \alpha_0 } - gt\]

\subsection{Uniform circular motion (MCU)}
\[ a_{rad} = \frac{v^2}{R} \]
\[ T = \frac{2\pi R}{v}\]
\[ a_{rad} = \frac{4\pi^2 R}{T^2} \]
\[ a_{tan} = \frac{ d|\vec{v}| }{ dt }\]
\[ F_{net} = ma_{rad} = m\frac{v^2}{R}\]

\subsection{Rounding curve}
If the curve is flat:
\[ v_{max} = \sqrt{\mu_s g R} \]
If the curve is banked ($ \beta $ is the angle):
\[ \tan{\beta} = \frac{a_{rad}}{g}\]
\[ \tan{\beta} = \frac{v^2}{gR}\]

\subsection{Relative velocity}
In one dimension: 
\[ v_{P/A-x} = v_{P/B-x} + v_{B/A-x} \]
In two dimensions:
\[ \vec{v}_{P/A} = \vec{v}_{P/B} + \vec{v}_{B/A} \]

\subsection{Newtons laws of motion}
A body acted on by no net force moves with a constant velocity (may be zero) and a zero acceleration.
\[ \sum \vec{F} = 0 \]

If a net external force acts on a body, the body accelerates. The direction of acceleration is the same as the direction of the net force. The mass of the body times the acceleration of the body equals the net force vector.
\[ \sum \vec{F} = m \vec{a} \]

If body A exerts a force on body B (an ``action''), the body B exetc a force on body A (a ``reaction''). These two forces have the same magnitude but are opposite in direction. These two forces act on different bodies. 

\[ \vec{F}_{A on B} = -\vec{F}_{B on A} \]

\section{Fluid resistance and Terminal Speed}
Fluid resistance at low speed:
\[ f = kv \]
\[ v_t = \frac{mg}{k} \]

Fluid resistance at high speed:
\[ f = Dv^2 \]
\[ v_t = \sqrt{ \frac{mg}{D} }\]

\section{Kinetic and potential energy}
\[K = \frac{mv^2}{2}\]
\[U = mgy\]
$K$ and $U$ are defined with a possible constant (like the derivatives)

\subsection{Conservative work}
When no extern force interact with the body, the work is conservative (there are no energy-loss).
\[E = K_1 + U_1 = K_2 + U_2 = cste \]

\subsection{Non-conservative work}
If an external force acts on the body (i.e. air drag), the work is non conservative and the previous equation becomes:
\[K_1 + U_1 + W_{ext} = K_2 + U_2\]

And $W_{ext}$ is :
\[W_{ext} = (K_1 - K_2) + (U_1 - U_2) = F(y_1 - y_2)\]

\subsection{Elastic potential energy}
\[U_{el} = \frac{kx^2}{2}\]
$k$ depends on the spring

\section{Gravitation}
\[F_g = \frac{Gm_1m_2}{r^2}\]
\[G = 6.67x10^{-11} N.m^2/kg^2\]
The Weight of a body is the total gravitationnal force exerted on this body by all other bodies of the universe.

\[U = -\frac{Gm_Em}{r_e}\]

\subsection{Satellite in circular orbit}
\[ \frac{Gm_Em}{r_E} = \frac{mv^2}{2} \Leftrightarrow v = \sqrt{\frac{Gm_E}{r}} \]
\[T = \frac{2\pi r}{v} = 2\pi r \sqrt{\frac{r}{Gm_E}} = \frac{2\pi r^{3/2}}{\sqrt{Gm_E}}\]

\section{Momentum, impulse $\&$ collisions }
\[\vec{p} = m\vec{v}\]
\[\vec{J} = \sum{} \vec{F}\Delta t\]
The change in momentum of a particule during a time interval equals the impulseof the net force that acts on the particlue during that interval.
\[\vec{J} = \vec{p_2} - \vec{p_1}\]

\subsection{Conservation of Momentum}
If the vector sum of the external forces on a system is zero, the total momentul of the system is constant.
\[\vec{P} = \vec{p_1} + \vec{p_2} + ...\]
{ \bf Caution ! With two momentums of different direction, whe must use the vector addition !}

\end{document}
