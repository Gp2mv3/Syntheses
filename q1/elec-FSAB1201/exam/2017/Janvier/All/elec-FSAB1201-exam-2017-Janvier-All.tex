\documentclass[fr]{../../../../../../eplexam}
\usepackage{../../../../../../eplunits}
\usepackage[siunitx]{circuitikz}
\usepackage{enumitem}
\sisetup{per-mode=symbol}

%TODO corriger les nombreuses erreurs de l'environnement solution : impossibilité de mettre des figures à l'intérieur de ces environnement (float lost), et impossibilité de mettre des figures dans le document tout court (provoque des problèmes au niveau du rendu du document : si, avant de commencer une nouvelle page, il n'y a pas assez de place pour insérer la figure, alors celle-ci est insérée à la page d'après, qui est vide, et avec une petite bande grise de 1cm toute seule)

\hypertitle[']{Électricité}{1}{FSAB}{1201}{2017}{Janvier}
{Marius Dechamps \and Thomas Gasos \and Guillaume Prieur}% \footnote{Avec la relecture et les corrections de Jean-Martin Vlaeminck} }
{Roland Keunings et Jean-Didier Legat}

% Question 1
\section{}
Une sphère conductrice et non chargée, de rayon \SI{10}{\centi\meter} est centrée au point $(0,0,0)$. Deux cavités sphériques de \SI{1}{\centi\meter} sont creusées, la première a pour centre le point $(5,0,0)$ et contient une charge ponctuelle en son centre de \SI{8}{\nano\coulomb}. La seconde est centrée en $(-5,0,0)$ et contient une charge ponctuelle en son centre de \SI{-12}{\nano\coulomb}.
\begin{enumerate}[label=(\alph*)]
	\item Définir le comportement électrique de cette sphère.
	\item Calculer la valeur du champ électrique à la surface de la sphère.
\end{enumerate}

\begin{solution}
	\begin{enumerate}[label=(\alph*)]
		\item\label{q1a} Tout d'abord, dans une sphère conductrice les charges ne se répartissent pas uniformément dans tout son volume mais sont toutes concentrées à sa surface.

		Ceci dit, intéressons nous à la cavité centrée en $(5,0,0)$. Sa charge de \SI{8}{\nano\coulomb} va attirer des charges négatives qui se répartissent uniformément sur toute la surface de la cavité. Ces charges sont de \SI{-8}{\nano\coulomb}. De la même manière la seconde cavité provoque un déplacement de charges positives de \SI{12}{\nano\coulomb} réparties uniformément à la surface de la cavité.

		Ensuite, intéressons nous à la surface extérieure de la sphère. Celle-ci étant non chargée, le déplacement de charges vers la surface de la cavité entraîne un déplacement de charges vers la surface extérieure de la sphère de sorte que la somme des charges totales de la sphère vaut $0$. Comme $12 - 8 = \SI{4}{\nano\coulomb}$ se sont déplacés vers l'intérieur, \SI{-4}{\nano\coulomb} se répartissent uniformément à la surface de la sphère.

		\item Pour connaître le champ électrique à la surface de la sphère, nous utilisons la loi de Gauss. En considérant que le champ à la surface de la sphère est symétrique, le champ $E$ est identique sur toute la surface, et on peut écrire la loi de Gauss sous la forme
		\[ \oint E \, \dif{A} = E \cdot A = \frac{Q_\text{encl}}{\epsilon_0}\]
		Comme vu au point \ref{q1a}, la charge totale $Q$ vaut \SI{-4}{\nano\coulomb} et la surface d'une sphère est $A=4 \pi r^2$. Dès lors,
		\begin{align*}
		E &= \frac{-4 \cdot 10^{-9}}{4 \pi \epsilon_0 \cdot (0.1)^2} \\
		&= \SI{-3600}{\newton\per\coulomb}
		\end{align*}
	\end{enumerate}
\end{solution}

% Question 2
\section{}
Sachant que $V_S = \SI{8}{\volt}$ et que $a = \SI{2}{\milli\ampere/\volt}$, soit le circuit suivant :
\begin{center}
	\begin{circuitikz}[american currents, american voltages]
		\draw (0,2)
		to[V_=$V_S$] (0,0) % attention à la version de Circuitikz !!
		(0, 2) to[short] (4,2)
		to[R=\SI{12}{\kilo\ohm}] (6,2)
		to[short, -*] (9,2)
		(0,0) to[short, -*] (9,0)
		(2,0) to[R=\SI{20}{\kilo\ohm}] (2,2)
		(4,2) to[R, l_=\SI{5}{\kilo\ohm}, v^>=$V_x$] (4,0)
		(6,0) to[I, i_=$aV_x$] (6,2)
		(8,0) to[R, l_=\SI{2}{\kilo\ohm}] (8,2)
		(0,2) to[short] (0,3)
		to[R=\SI{6}{\kilo\ohm}] (6,3)
		to[short] (6,2)
		(9,2) node[anchor = south west] {$a$}
		(9,0) node[anchor = north west] {$b$};
	\end{circuitikz}
\end{center}

\begin{enumerate}[label=(\alph*)]
	\item Donner l'équivalent de Norton aux bornes $a$ et $b$ du circuit.
	\item Supposons que nous ajoutions une résistance de \SI{800}{\ohm} aux bornes $a$ et $b$, calculez la puissance dissipée par cette nouvelle résistance.
\end{enumerate}

\begin{solution}
	Plaçons la terre et relions les bornes $a$ et $b$, afin de déterminer le courant de Norton passant par ce fil.

	Notons les intensités utiles à la résolution du problème comme-ci :
	\begin{center}
		\begin{circuitikz}[american currents, american voltages]
			\draw (0,2)
			to[V_=8<\volt>] (0,0)
			(0,2) to[short] (4,2)
			to[R=\SI{12}{\kilo\ohm}, i = $I_2$] (6,2)
			to[short, -*, i = $I_N$] (9,2)
			to[short] (10,2)
			to[short] (10,0)
			to[short] (9,0)
			(0,0) to[short, -*] (9,0)
			(2,0) to[R=\SI{20}{\kilo\ohm}] (2,2)
			(4,0) to[R=\SI{5}{\kilo\ohm}] (4,2)
			(6,0) to[I=$aV_x$, i = $I_3$] (6,2)
			(8,0) to[R=\SI{2}{\kilo\ohm}] (8,2)
			(0,2) to[short] (0,3)
			to[R=\SI{6}{\kilo\ohm}, i = $I_1$] (6,3)
			to[short] (6,2)
			(9,2) node[anchor = south west] {$a$}
			(9,0) node[anchor = north west] {$b$}
			(0,0) node[ground] {} (-1,0);
		\end{circuitikz}
	\end{center}
	Nous savons qu'aucun courant ne passera dans la résistance de \SI{2}{\kilo\ohm} car sa tension aux bornes est nulle.
	Nous savons dès lors que
	\[I_1+I_2+I_3 = I_N\]
	La tension à l'autre borne des résistances de 6 et de \SI{12}{\kilo\ohm} est égale à \SI{0}{\volt}, par la liaison $a-b$. Dès lors,
	\[ I_N = \frac{8}{6}+\frac{8}{12}+16 = \SI{18}{\milli\ampere} \]
	Étant donné qu'il y a une source de courant commandée, nous ne pouvons pas annuler les sources directement pour déterminer $R_{eq}$. Nous allons donc calculer la tension équivalente de Thévenin $V_{T}$.

	Reprenons le circuit d'au dessus et considérons qu'entre les bornes $a$ et $b$, le circuit est ouvert. Dès lors, le courant $I_N$ passe dans la résistance de \SI{2}{\kilo\ohm} (et n'est plus le courant de Norton). Nommons $V_T$, la tension en $a$. Avec la loi de Kirkchoff sur le même n\oe{}ud qu'avant nous obtenons l'équation:
	\[\frac{V_S - V_T}{12} + \frac{V_S - V_T}{6} + I_3= \frac{V_T-0}{2}\]
	Avec $V_S=\SI{8}{\volt}$ et $I_3 = \SI{2}{\milli\ampere} \cdot 8 = \SI{16}{\milli\ampere}$, on obtient $V_T = \SI{24}{\volt}$

	Enfin, la résistance équivalente est
	\[ R_{eq} = \frac{V_T}{I_N} = \frac{24}{18} = \SI{1.333}{\kilo\ohm} \]
	Nous avons dès lors le circuit suivant, équivalent au premier :
	\begin{center}
		\begin{circuitikz}[american currents, american voltages]
			\draw (0,0)
			to[I = \SI{18}{\milli\ampere}] (0,2)
			to[short, -*] (3,2)
			(0,0) to[short, -*] (3,0)
			(2,0) to[R, l_= \SI{1.333}{\kilo\ohm}] (2,2)
			(3,2) node[anchor = south west] {$a$}
			(3,0) node[anchor = north west] {$b$};
		\end{circuitikz}
	\end{center}
	Plaçons maintenant notre résistance de \SI{800}{\ohm} aux bornes $a$ et $b$ comme ci :
	\begin{center}
		\begin{circuitikz}[american currents, american voltages]
			\draw (0,0)
			to[I = \SI{18}{\milli\ampere}] (0,2)
			to[short, -*] (3,2)
			to[short] (4,2)
			to[R = \SI{800}{\ohm}, i = $I_R$] (4,0)
			to[short] (3,0)
			(0,0) to[short, -*] (3,0)
			(2,0) to[R, l_= \SI{1.333}{\kilo\ohm}] (2,2)
			(3,2) node[anchor = south west] {$a$}
			(3,0) node[anchor = north west] {$b$};
		\end{circuitikz}
	\end{center}
	Nous savons que $P=RI^2$. Calculons donc le courant qui passe par cette nouvelle résistance grâce au diviseur de courant :
	\[ I_R = \frac{18\cdot 1.333}{1.333+0.8} = \SI{11.25}{\milli\ampere} \]
	Et donc nous pouvons calculer la puissance dissipée
	\[ P = 800\cdot (11.25 \cdot 10^{-3})^2 = \SI{0.101}{\watt} \]
\end{solution}

% Question 3
\section{}
Soit le circuit suivant, sachant que le switch se ferme en $t = 0$.
\begin{center}
	\begin{circuitikz}[american currents, american voltages]
		\draw (0,2)
		to[V, l_=24<\volt>] (0,0)
		(0, 2) to[short] (0,4.5)
		to[R = 26<\kilo\ohm>] (7,4.5)
		to[short] (7,2)
		to[V, l^= 9<\volt>] (7,0)
		to[short] (0, 0)
		(0,2) to[R = 6<\kilo\ohm>] (2,2)
		to[C = 4<\micro\farad>] (5,2)
		to[R = 15<\kilo\ohm>] (7,2)
		(0,3.5) to[R = 12<\kilo\ohm>] (2,3.5)
		to[switch] (3.5,3.5)
		to[R = 5<\kilo\ohm>] (5,3.5)
		to[R = 10<\kilo\ohm>] (7,3.5)
		(2,0) to[R = 12<\kilo\ohm>] (2,2)
		to[short] (2,3.5)
		(5,0) to[R = 3<\kilo\ohm>] (5,2)
		to[short] (5,3.5)
		(0,0) node[ground] {};
	\end{circuitikz}
\end{center}
\begin{itemize}
	\item Calculer les équations de la tension $V_C(t)$ et du courant $I_C(t)$ aux bornes de la capacité.
	\item Donner l'équivalent de Thévenin du circuit, vu par la capacité.
\end{itemize}

\begin{solution}
	Pour commencer, nous pouvons supprimer la résistance de \SI{26}{\kilo\ohm} de notre circuit : en effet, le courant qui la traverse est fixe et indépendante des autres composants à part les sources de tension, et elle n'influence donc pas les autres composants, dont la capacité.

	Regardons la tension aux bornes de la capacité quand $t<0$, avec l'interrupteur ouvert. Simplifions d'abord notre circuit. Nous mettons les résistances de \SI{12}{\kilo\ohm} et de \SI{6}{\kilo\ohm} en parallèle ainsi que les résistances de \SI{10}{\kilo\ohm} et de \SI{15}{\kilo\ohm}. Notre circuit devient alors
	\begin{center} % TODO corriger la ligne grise trop courte dans la marge de l'environnement solution.
		\begin{circuitikz}[american currents, american voltages]
			\draw (0,2)
			to[V, l_= \SI{24}{\volt}] (0,0)
			to[short] (7,0)
			(7, 2) to[V, l^= \SI{9}{\volt}] (7,0)
			(0,2) to[R = \SI{4}{\kilo\ohm}, -*] (2,2)
			to[C = \SI{4}{\micro\farad}, -*] (5,2)
			to[R = \SI{6}{\kilo\ohm}] (7,2)
			(2,0) to[R = \SI{12}{\kilo\ohm}] (2,2)
			(5,0) to[R = \SI{3}{\kilo\ohm}] (5,2)
			(2,2) node[anchor = south] {$V_1$}
			(5,2) node[anchor = south] {$V_2$}
			(0,0) node[ground] {};
		\end{circuitikz}
	\end{center}
	Par le diviseur résistif nous pouvons aisément calculer les tensions $V_1$ et $V_2$ aux bornes de la capacité.
	\[ V_1 = \frac{24\cdot 12}{4+12} = \SI{18}{\volt} \]
	\[ V_2  = \frac{9\cdot 3}{6+3} = \SI{3}{\volt} \]
	La différence de tension aux bornes de la capacité vaut donc
	\[ V_C(0^-) = \SI{18}{\volt} - \SI{3}{\volt} = \SI{15}{\volt} \]
	
	Calculons maintenant cette différence en $t \rightarrow \infty$.
	Nous pouvons garder les résistances de \SI{12}{\kilo\ohm} et de \SI{6}{\kilo\ohm} en parallèle ainsi que les résistances de \SI{10}{\kilo\ohm} et de \SI{15}{\kilo\ohm}. Et nous avons donc
	\begin{center}
		\begin{circuitikz}[american currents, american voltages]
			\draw (0,2)
			to[V, l_= \SI{24}{\volt}] (0, 0)
			to[short] (7,0)
			(7, 2) to[V, l^= \SI{9}{\volt}] (7,0)
			(0,2) to[R = \SI{4}{\kilo\ohm}, -*, i = $I_1$] (2,2)
			to[C = \SI{4}{\micro\farad}, -*] (5,2)
			to[R = \SI{6}{\kilo\ohm}, i<= $I_4$] (7,2)
			(2,2) to[R = \SI{12}{\kilo\ohm}, i =$I_2$] (2,0)
			(2,2) to[short] (2,3.5)
			to[R = \SI{5}{\kilo\ohm}, i = $I_3$] (5,3.5)
			to[short] (5,2)
			(5,2) to[R = \SI{3}{\kilo\ohm}, i = $I_5$] (5,0)
			(2,2) node[anchor = south west] {$V_1$}
			(5,2) node[anchor = south east] {$V_2$}
			(0,0) node[ground] {};
		\end{circuitikz}
	\end{center}
	Sachant que la capacité à l'équilibre peut être remplacée par un circuit ouvert, aucun courant ne passe par celle ci. Grâce à la méthode des n\oe{}uds, nous pouvons tirer deux équations : 
	\[ I_1=I_2+I_3 \]
	\[ I_3+I_4=I_5 \]
	Remplaçons par les valeurs de ces courants, à partir des tensions et des résistances :
	\[\frac{24-V_1}{4}=\frac{V_1}{12}+\frac{V_1-V_2}{5}\]
	\[\frac{V_2}{3}=\frac{9-V_2}{6}+\frac{V_1-V_2}{5}\]
	En résolvant nous trouvons que 
	\begin{align*}
	V_1 & = \SI{13.5}{\volt} \\
	V_2 & = \SI{6}{\volt} \\
	\end{align*}
	La différence de tension aux bornes de la capacité vaut
	\[ V_C(t\rightarrow\infty) = \SI{13.5}{\volt} - \SI{6}{\volt} = \SI{7.5}{\volt} \]
	Trouvons maintenant la résistance équivalente qui nous permettra de calculer la constante de temps $\tau$ du circuit. La résistance vue par la capacité peut être obtenue en annulant les sources :
	\begin{align*}
	R_{\text{éq}} & = ((\SI{12}{\kilo\ohm}//\SI{4}{\kilo\ohm})+(\SI{3}{\kilo\ohm}// \SI{6}{\kilo\ohm}))//\SI{5}{\kilo\ohm}\\
	& = \SI{2.5}{\kilo\ohm}
	\end{align*} %FIXME petit débat au niveau de la valeur : avant le 3k et le 4k étaient inversés
	La constante de temps vaut donc,
	\[ \tau = 2.5\cdot 10^3\cdot 4\cdot 10^{-6} = \SI{10}{\milli\second} \]
	En connaissant les valeurs initiale et finale aux bornes de la capacité nous pouvons trouver la valeur de $V_C(t)$
	\[ V_C(t) = 7.5 + 7.5\exp(-t/0.01) \]
	Dérivons pour trouver $I_C(t)$
	\[ I_C(t) = -3\cdot 10^{-3}\exp(-t/0.01) \]
	L'équivalent de Thévenin se trouve facile en connaissant le $R_{\text{éq}}$ et la tension aux bornes.
	\begin{center}
		\begin{circuitikz}[american currents, american voltages]
			\draw
			(0,0) to[short, -*]
			(3,0) to[short]
			(4,0) to[C = \SI{4}{\micro\farad}]
			(4,2) to[short]
			(3,2) to[R = \SI{2.5}{\kilo\ohm}, *-]
			(0,2) to[V = \SI{7.5}{\volt}] (0, 0)
			;
		\end{circuitikz}
	\end{center}
\end{solution}

% Question 4
\section{}
En vous basant sur des ampli op idéaux (courant d’entrée nul et gain infini), et en n'utilisant que des résistances de \SI{10}{\kilo\ohm}, donnez le schéma d’un circuit qui réalise la fonction suivante :
\[ V_{out} = - \frac{V_1(t)}{2} + \frac{V_2(t)}{4} + 5 \int_0^t V_3(t) \, \dif{t} \]
Expliquez chaque tension sortante des sous-circuits utilisés.
%FIXME je pense qu'on ne pouvait qu'utiliser des résistances de 10k, vu que personne ne proposait d'en simplifier, juste en mettre en parallèle pour en réduire. Néanmoins cet exercices est alors particulièrement chiant à coder et à répondre, et la capacité est strange. A confirmer.

\begin{solution}
	Nous construisons le schéma ci-après.

	Commençons par le montage intégrateur, tout au-dessus du schéma. Aucun courant ne passe dans l'ampli op, ce qui implique que tout le courant qui passe dans la résistance passe aussi dans la capacité. Comme les deux entrées de l'ampli op ont la même tension de \SI{0}{\volt} (vu que l'entrée + est directement connecté à une masse), le courant vaut donc : 
	\[ \frac{V_3(t)}{R_{i3}} = - C \fdif{V_B}{t} \]
	Et donc :
	\[ V_B = \frac{-1}{R_{i3} C} \int_0^t V_3(t) \, \dif{t} \]
	\begin{center}
		\begin{circuitikz}[american currents, american voltages]
			% La grosse majorité des distances sont hard-codées.
			\draw
			(6, 0) node[op amp] (opamp1) {} % inverseur
			(6, 4) node[op amp] (opamp2) {} % intégrateur
			(12, -0.5) node[op amp] (opamp3) {} % sommateur
			(opamp1.+) -| ++(-0.5, 0) node[ground] {}
			(opamp1.-) -- ++(-0.5, 0)
			           to[R=$R_{i2a}$] ++(-1.5, 0)
			           to[R=$R_{i2b}$] ++(-1.5, 0)
			           -| ++(-0.3, -0.3)
			           to[open, v=$V_2(t)$, o-o] ++ (0, -1.2) node[ground] {}
			(opamp1.-) |- ++(0.3, 1)
			           to[R=$R_{f2}$] ++(1.5, 0)
			           -| (opamp1.out) node[below] {$V_A$}
			           to[short, *-] ++(0.5, 0)
			           to[R=$R_{o2}$] ++(1.5, 0) coordinate(rassemblement)
			(rassemblement) to[short, *-] ++(0, -1.5)
			           to[R=$R_{i1}$] ++(-1.5, 0)
			           -| ++(-0.3, -0.3)
			           to[open, o-o, v=$V_1(t)$] ++(0, -1.2) node[ground] {}
			(opamp2.+) -| ++(-0.5, 0) node[ground] {}
			(opamp2.-) -- ++(-0.5, 0)
			           to[R=$R_{i3}$] ++(-1.5, 0)
			           -| ++ (-0.3, -0.3)
			           to[open, o-o, v=$V_3(t)$] ++(0, -1.2) node[ground] {}
			(opamp2.-) |- ++(0.3, 1)
			           to[C=10<\micro\farad>] ++(1.5, 0)
			           -| (opamp2.out) node[below] {$V_B$}
			           to[short] ++(0.5, 0)
			           to[R=$R_{o3}$] ++(1.5, 0)
			           -| (rassemblement)
			(opamp3.+) -| ++(-0.5, 0) node[ground] {}
			(rassemblement) -- (opamp3.-)
			           |- ++(0.3, 1)
			           to[R=$R_{fa}$] ++(1.5, 0)
			           -| (opamp3.out)
			           -| ++(0.3, -0.3)
			           to[open, o-o, v^=$V_\mathrm{out}(t)$] ++(0, -1.2) node[ground] {}
			(opamp3.-) ++ (0, 1) |- ++(0.3, 1)
			           to[R=$R_{fb}$] ++(1.5, 0) -| (opamp3.out)
			;
		\end{circuitikz}
	\end{center}

	Ensuite, nous regardons le montage inverseur, au milieu à gauche du schéma. Les mêmes conditions sur le courant et la tension aux entrées de l'ampli op que ci-dessus sont valables :
	\[ \frac{V_2(t)}{R_{i2a} + R_{i2b}} = \frac{-V_A}{R_{f2}} \]
	Et donc : 
	\[ V_A =-\frac{R_{f2}}{R_{i2a}+R_{i2b}} V_2(t) \]
	Enfin regardons le dernier montage inverseur tout à gauche. Par Kirchhoff, le courant traversant la résistance au-dessus de l'ampli op vaut la somme des courants rentrants.
	\[ -\frac{V_\mathrm{out}(t)}{R_{ia} // R_{ib}} = \frac{V_1(t)}{R_{i1}} + \frac{V_A}{R_{o2}} +\frac{V_B}{R_{o3}} \]
	En utilisant à présent le fait que toutes les résistances sont égales à $R$, on a
	\[ V_{out}(t) = \frac{1}{2 R C} \int_0^t V_3(t) dt+ \frac{1}{4} V_2(t)- \frac{1}{2} V_1(t) \]
	ce qui est bien la relation demandée (avec $2 R C = 0.2$).
\end{solution}

\section*{Partie mécanique}
La partie mécanique de l'examen se trouve dans le dossier dédié (Synthèses-EPL > q1 > meca-FSAB1201 > exam > 2017\_Janvier).

\end{document}
