\documentclass[fr]{../../../eplsummary}

\usepackage{tikz}
\usepackage{../../../eplunits}

\newcommand{\hati}{\hat{\imath}}
\newcommand{\hatj}{\hat{\jmath}}
\newcommand{\hatk}{\hat{k}}

\hypertitle{Mécanique}{1}{EPL}{1201}
{Nicolas Cognaux\and Mattéo Couplet\and Benoît Legat}
{Roland Keunings}
%    _
%   /A\
%  |   |
%  |   C
%  E   |
%      M

\part{Vecteurs}
% +----------+
% | Vecteurs |
% +----------+
\section{Décomposition des vecteurs}
\[ \| \vec{A} \| = \sqrt{{A_x}^2 + {A_y}^2} \]
\[ \theta = \arctan{\frac{A_y}{A_x}} \]

\section{Produit scalaire}
\[ \vec{A} \cdot \vec{B} = \|A\| \cdot \|B\| \cos{\theta} \]
\[ \vec{A} \cdot \vec{B} = A_x B_x + A_y B_y + A_z B_z \]

\section{Produit vectoriel}
\[ \| \vec{A} \times \vec{B} \| = \|\vec{A}\| \cdot \|\vec{B}\| \sin{\theta} \]

\begin{align*}
    \vec{A} \times \vec{B} &= 
      \begin{vmatrix}
        \hati&\hatj&\hatk\\
        A_x&A_y&A_z\\
        B_x&B_y&B_z
      \end{vmatrix} \\
      &= 
  \begin{vmatrix}
    A_y&A_z \\
    B_y&B_z
  \end{vmatrix} \hati  -
  \begin{vmatrix}
    A_b&A_z \\
    B_x&B_z
  \end{vmatrix} \hatj +
  \begin{vmatrix}
    A_x&A_y \\
    B_x&B_y
  \end{vmatrix} \hatk \\
  & = (A_yB_z - A_zB_y) \ \hati +  (A_zB_x - A_xB_z) \, \hatj + (A_xB_y - A_yB_x) \ \hatk
\end{align*}
On peut grâce à la règle de la main droite retrouver le produit des vecteurs de la base orthonormée 
\begin{align*}
  \hati \times \hatj & = \hatk\\
  \hatj \times \hatk & = \hati\\
  \hatk \times \hati & = \hatj
\end{align*}

\part{Dynamique : Lois de Newton}
% +----------------+
% | Lois de Newton |
% +----------------+


\section{Première Loi de Newton -- Loi d'inertie}
Si la somme des forces agissant sur un corps est nulle, alors il ne subit aucune
accélération et se déplace à vitesse constante.
\[ \sum \vec{F} = \vec{0} \Rightarrow \vec{a} = \vec{0} \text{ et } \vec{v} = \text{constante} \]

\paragraph{Référentiel inertiel}
Un référentiel inertiel ou galiléen est un référentiel dans lequel la première loi de Newton est d'application ; ainsi, tout corps libre est à l'équilibre. Un référentiel qui est accéléré par rapport à un référentiel inertiel n'est pas inertiel.

\section{Seconde Loi de Newton -- Loi du mouvement}
Soit un corps de masse $m$ : l'accélération subie par ce corps est
proportionnelle à la résultante des forces qu'il subit,
et inversement proportionnelle à sa masse m.
\[ \sum \vec{F} = m \vec{a} \]

\section{Troisième Loi de Newton -- Principe d'action-réaction}
Tout corps A exerçant une force sur un corps B
subit une force d'intensité égale,
de même direction mais de sens opposé, exercée par ce corps B.
\[ \vec{F}_{A\text{ sur }B} = -\vec{F}_{B\text{ sur }A} \]


\part{Statique}
% +-----------+
% | Équilibre |
% +-----------+
Un corps rigide est dit en équilibre s'il ne bouge ni ne tourne sur lui-même.
\section{Conditions}
Pour qu'il soit en équilibre, il faut que deux conditions soient respectées
\begin{itemize}
  \item La somme des forces est nulle.
      \[ \sum \vec{F} = \vec{0} \]
  \item La somme des moments de force en un point A est nulle.
      \[ \sum \vec{\tau_A} = \vec{0} \]
    avec
    $\vec{\tau_A} = \vec{AP} \times \vec{F}$
    où $P$ est le point où $F$ s'applique.

    Si la première condition est respectée,
    cette somme ne dépend pas du point $A$ choisi.
\end{itemize}

\part{Cinématique}
% +------------+
% | Mouvements |
% +------------+
\section{Mouvement rectiligne}
\[ v_x = \lim_{\Delta t \to 0} \frac{\Delta x}{\Delta t} = \frac{\dif x}{\dif t} \]
\[ a_x = \lim_{\Delta t \to 0} \frac{\Delta v_x}{\Delta t} = \frac{\dif v_x}{\dif t} \]

Dans le cas d'un mouvement rectiligne uniformément accéléré (MRUA), on a
\[ v_x = v_{0x} + a_{x}t \]
\[ x = x_0 + v_{0x} t + \frac{a t^2}{2} \]

\section{Mouvement à plusieurs dimensions}
\[ \vec{r} = x\hati + y\hatj + z\hatk \]
\[ \vec{v} = \frac{\dif \vec{r}}{\dif t} \qquad  \vec{a} = \frac{\dif \vec{v}}{\dif t} \]


\section{Ballistique}

\begin{figure}
  \begin{center}
    \begin{tikzpicture}[domain=1:4]
      \draw[->] (-0.2,0) -- (4.8,0) node[right] {$x$};
      \draw[->] (0,-0.2) -- (0,4.8) node[above] {$y$};
      \draw plot[id=sin] function{-x*x + 5.875*x - 4.375} node[right] {};
      \draw plot[id=vec] coordinates{(1,0.5)(2,4.375)}
      node[right] {$v_0$};
      \draw[fill=green!40] (1,0.5) -- (1,0.5)++(75:0.75)
      arc (75:0:0.75)-- (1,0.5);
      \draw (1,0.5)++(37.5:0.5) node {$\alpha_0$};
      \draw[dashed] (1,0.5) -- (1,0);
      \draw (1,0) -- (1,-0.1) node[below] {$x_0$};
      \draw[dashed] (1,0.5) -- (0,0.5);
      \draw (0,0.5) -- (-0.1,0.5) node[left] {$y_0$};
    \end{tikzpicture}
  \end{center}
  \caption{Trajectoire d'un projectile}
  \label{fig:proj}
\end{figure}
En utilisant les variables $v_0$ et $\alpha_0$
de la Figure~\ref{fig:proj}, on peut écrire
\begin{align}\label{eq:projvx}
  v_x & = v_0 \cos{ \alpha_0 }\\
  v_y & = v_0 \sin{ \alpha_0 } - gt \label{eq:projvy}
\end{align}
\[ x = \int v_x dt \stackrel{\eqref{eq:projvx}}{=}
x_0 + v_0 \cos{ \alpha_0 } t \]
\[ y = \int v_y dt \stackrel{\eqref{eq:projvy}}{=}
y_0 + v_0 \sin{ \alpha_0 } t - \frac{1}{2} g t^2 \]

\section{Mouvement circulaire}
Pour la démonstration voir \cite[p. 1]{notes}

\paragraph{Uniforme (MCU)}
\begin{align*}  
    \omega &= 2\pi f \\
    v &= \omega R \\
    a_\text{rad} &= \omega v = \omega^2 R = \frac{v^2}{R}
\end{align*}

\paragraph{Accéléré}
\[ a_\text{tan} = \frac{\dif v}{\dif t} \]

\section{Courbe}
\paragraph{Si la courbe est plate}
\[ m\frac{v^2}{R} \leq \mu_\text{s}mg \]
D'où
\[ v_\text{max} = \sqrt{\mu_\text{s} g R} \]
\paragraph{Si la courbe est incliné d'un angle $\beta$}
On utilise $n\cos\beta = mg$ et non comme certains auraient pu penser
$n = mg\cos\beta$ pour permettre à la voiture de suivre sa trajectoire,
voir \cite[p.~175]{young13} ou \cite[p.~177]{young14}.
On a donc aussi $n\sin\beta = ma_\text{rad}$.
\begin{align*}
  \tan{\beta} & = \frac{a_\text{rad}}{g}\\
\end{align*}

\section{Vitesse relative}
À une dimension
\[ v_{P/A-x} = v_{P/B-x} + v_{B/A-x} \]
À deux dimensions
\[ \vec{v}_{P/A} = \vec{v}_{P/B} + \vec{v}_{B/A} \]

\part{Frottement, résistance des fluides et vitesse terminale}
% +---------------------------------------------+
% | Résistance des fluides et vitesse terminale |
% +---------------------------------------------+

\section{Frottement}
\paragraph{Frottement cinétique}
\[ F_\mathrm{f} = \mu_\mathrm{k} n \]
\paragraph{Frottement statique}
\[ F_\mathrm{s} \le \mu_\mathrm{s} n \]

\section{À petite vitesse}
\[ F_\mathrm{f} = kv \]
\[ v_\mathrm{t} = \frac{mg}{k} \]
\section{À grande vitesse}
\[ F_\mathrm{f} = Dv^2 \]
\[ v_\mathrm{t} = \sqrt{ \frac{mg}{D} } \]

\part{Travail et énergies potentielle et cinétique}
% +----------------------------------+
% | Énergie potentielle et cinétique |
% +----------------------------------+
\section{Travail}
\[ W \eqdef \int_{P_1}^{P_2} \vec{F} \cdot \dif\vec{l} \]
Dans le cas d'un travail effectué par une force \textbf{constante} $\vec{F}$ sur un objet subissant un déplacement \textbf{rectiligne} $\vec{s}$, on a
\[ W = \vec{F} \cdot \vec{s} \]
\section{Énergie}
\[ K = \frac{mv^2}{2} + \mathrm{cste} \]
À des altitudes proches de la surface de la Terre, l'énergie potentielle gravitationnelle s'exprime
\[ U_\mathrm{grav} = mgy + \mathrm{cste} \]
Pour la démonstration voir \cite[p. 3]{notes}. \\
$K$ et $U$ sont définies à une constante près.

\section{Théorème travail-énergie}
Le travail effectué sur une particule équivaut au changement en énergie cinétique de cette particule 
\[ W_\mathrm{tot} = \Delta K \]

\section{Forces conservatives}
Lorsqu'aucune force extérieure n'agit sur le corps,
le travail est conservatif (il n'y a pas de perte d'énergie).
\[ E_\mathrm{méc} = K_1 + U_1 = K_2 + U_2 = \mathrm{cste} \]

\[ \vec{F} = -\vec{\grad{}} U \]

\section{Forces non-conservatives}
Lorsque des forces extérieures agissent sur le corps
(comme le frottement de l'air),
le travail est non-conservatif et l'équation précédente devient
\[ K_1 + U_1 + W_\mathrm{ext} = K_2 + U_2 \]
où $W_\mathrm{ext}$ vaut
\[ W_\mathrm{ext} = (K_2 - K_1) + (U_2 - U_1) \]
Si ce travail extérieur est fourni par une force constante sur un segment rectiligne (p. ex. le frottement du sol), on peut affirmer
\[ W_\mathrm{ext} = Fd = F(x_2 - x_1) \]

\section{Énergie potentielle élastique}
\[ F_{\text{ressort}} = kx \]
\[ U_\mathrm{él} = \frac{kx^2}{2} \]
$k$ est la constante de Hooke et indique la raideur du ressort.

\section{Puissance}
\[ P = \frac{\dif W}{\dif t} = \vec{F} \cdot \vec{v} \]

\part{Gravitation}
% +-------------+
% | Gravitation |
% +-------------+
\section{Force d'attraction d'un corps}
\[ F_\mathrm{grav} = \frac{Gm_1m_2}{r^2} \]
\[ G \approx \SI{6.67e-11}{Nm^2/kg^2} \]
Le poids d'un corps est la somme des forces gravitationnelles exercées
sur celui-ci par tous les autres corps de l'univers.
\[ W_\mathrm{grav} = \int_{r_1}^{r_2}F(r) \dif r \]
\[ U = -\frac{Gm_\mathrm{E}m}{r_\mathrm{E}} \]
Pour la démonstration voir \cite[p. 2]{notes}

\section{Satellite en orbite circulaire}
Pour la démonstration voir \cite[p. 3]{notes}
\[  v_\mathrm{orbitale}
= \sqrt{\frac{GM}{r}} 
\qquad T 
= \frac{2\pi r^{3/2}}{\sqrt{GM}} \]

\section{Vitesse de libération}
\[ v_{\text{libération}} = \sqrt{\frac{2GM}{r}} = \sqrt{2}\ v_\mathrm{orbitale} \]
Pour la démonstration voir \cite[p. 4]{notes}

\section{Rayon de Scharzschild}
Lorsqu'une étoile a un rayon inférieur au rayon de Schwarzschild, elle devient un trou noir, c'est-à-dire que son champ gravitationnel est tellement fort que la vitesse de libération est supérieure à celle de la lumière.
\[ R_\mathrm{S} = \frac{2GM}{c^2} \]
Pour la démonstration voir \cite[p. 4]{notes}

\part{Quantité de mouvement et impulsion}
% +------------------------------------+
% | Quantité de mouvement et impulsion |
% +------------------------------------+
\section{Définition par la deuxième loi de Newton}
Soit $\vec{p}$ la quantité de mouvement d'un corps
\[ \vec{p} = m\vec{v} \]
\[ \sum \vec{F} = m\vec{a} = m\frac{d\vec{v}}{dt} = \frac{d\vec{p}}{dt} \]
Soit l'impulsion $\vec{J}$ définie comme suit
\[ \vec{J} = \int_{t_1}^{t_2} \sum\vec{F} dt =
  \int_{t_1}^{t_2}\frac{d\vec{p}}{dt} dt =
\int_{\vec{p_1}}^{\vec{p_2}} d\vec{p} = \vec{p_2} - \vec{p_1} \]

\section{Quantité de mouvement et énergie cinétique}
\begin{itemize}
  \item L'énergie cinétique correspond au travail total effectué sur un corps
    pour accélérer celui-ci de l'état d'équilibre à sa vitesse actuelle;
  \item Le momentum équivaut à l'impulsion pour accélérer
    un corps de l'état d'équilibre à sa vitesse présente.
\end{itemize}

\section{Conservation de la quantité de mouvement}
Si la somme des forces extérieures est nulle,
alors la quantité de mouvement totale du système est contante.
\[ \vec{P} = \vec{p_1} + \vec{p_2} + ... \]
{\bf Attention ! Avec deux quantités de mouvement de directions différentes,
il faut utiliser l'addition vectorielle !}

\part{Collisions}
% +------------+
% | Collisions |
% +------------+
Pour toute collision,
la quantité de mouvement totale initiale et finale sont égales (en vertu de la conservation de la quantité de mouvement).
\section{Types de collisions}
\begin{description}
  \item [Élastique]
    Les forces intervenant dans la collision sont conservatrices
    et l'énergie totale du système reste donc
    la même avant et après la collision.
  \item [Inélastique]
    L'énergie totale du système est moindre après la collision.
  \item [Totalement inélastique]
    Après la collision,
    les deux corps \emph{collent} ensemble pour ne former qu'un.
\end{description}

\section{Collisions complètement inélastiques}
Par la conservation de la quantité de mouvement
\[ m_A\vec{v_{A1}} + m_B\vec{v_{B1}} = (m_A + m_B)\vec{v_2} \]
D'où
\[ \vec{v_{2}} = \frac{m_A\vec{v_{A1}} + m_B\vec{v_{B1}}}{m_A + m_B} \]

\section{Collisions élastiques}
Considérons que nous sommes dans un problème à une dimension.
Par la conservation de la quantité de mouvement
\begin{align*}
  m_1\vec{u_1} + m_2\vec{u_2} & = m_1\vec{v_1} +  m_2\vec{v_{2}}\\
  \frac{1}{2}m_1u_{1}^2 + \frac{1}{2}m_2u_{2}^2
  & = \frac{1}{2}m_1v_{1}^2 + \frac{1}{2}m_2v_{2}^2\\
\end{align*}
D'où
\begin{align*}
  v_1 & = \frac{u_1(m_1 - m_2) + 2m_2u_2}{m_1 + m_2}\\
  v_2 & = \frac{u_2(m_2 - m_1) + 2m_1u_1}{m_1 + m_2}
\end{align*}
La résolution algébrique nous donne aussi la solution triviale
$\vec{v_1} = \vec{u_1}$ et $\vec{v_2} = \vec{u_2}$ qui est à rejeter car ça suppose que
les corps soient passés l'un à travers l'autre sans provoquer de collision.

\part{Mouvement périodique}
Pour les démonstrations voir \cite[p. 5]{notes}
% +----------------------+
% | Mouvement périodique |
% +----------------------+
La position $x$ est en mouvement périodique
lorsqu'il existe une constante $k$ telle que
\[ F = kx \]
ça peut aussi marcher avec un angle
à la place de $x$ comme pour le cas du pendule.

\section{Formules}
\[ x = A\cos{(\omega{t} + \phi)} \]
\begin{description}
    \item [$A$] l'amplitude du mouvement;
    \item [$\omega$] la vitesse angulaire donnée par
        \[ \omega = \frac{v}{R} \mathrm{;} \]
    \item [$\phi$] le déphasage du mouvement.
\end{description}
\section{Fréquence, période et vitesse angulaire}
\[ T = \frac{1}{f} = \frac{2\pi}{\omega} \]
\[ \omega = 2\pi{f} = \frac{2\pi}{T} \]

\section{Oscillation d'un ressort}
Pour un ressort idéal, la loi suivante est d'application
\[ F_\text{rappel} = -kx \qquad \frac{\dif^2 x}{\dif t^2} = -\frac{k}{m} x \]

L'élongation en fonction du temps est alors donnée par
\[ x(t) = A\sin{(\omega{t} + \phi)} \]
Avec 
\[ \omega = \sqrt{\frac{k}{m}} \]

\section{Pendule simple}
\[ F_\mathrm{rappel} = -mg \sin\theta \approx -mg\theta = -mg\frac{x}{L} \Rightarrow k = \frac{mg}{L} \]
\[ \omega = \sqrt{\frac{g}{L}} \]

\section{Energie dans un mouvement harmonique}
\[ E = \frac{1}{2}mv^2_x + \frac{1}{2}kx^2 = \frac{1}{2}kA^2 = \mathrm{cste} \]

\begin{thebibliography}{9}
    \bibitem{notes}
        Mattéo Couplet,
        \emph{Notes de Mécanique},
        \url{https://www.dropbox.com/sh/mglnckwio1ug5x0/AACcqxTqPa0aTuGbKSaRYJZla/Q1/meca-FSAB1201/summary?dl=0}
  \bibitem{young13}
    Young and Freedman,
    \emph{University Physics},
    13th Edition
  \bibitem{young14}
    Young and Freedman,
    \emph{University Physics},
    14th Edition
\end{thebibliography}

\end{document}
