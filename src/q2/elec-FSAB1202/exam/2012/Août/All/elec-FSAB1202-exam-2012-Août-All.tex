\documentclass[fr]{../../../../../../eplexam}
\usepackage{../../../../../../eplunits}
\usepackage{../../../../../../eplelec}
\sisetup{per-mode=symbol}

\DeclareSIUnit\tour{tr}

\hypertitle{Élecricité et Magnétisme}{2}{FSAB}{1202}{2012}{Août}
{Mattéo Couplet}
{Claude Oestges}[
    \paragraph{Remarques} 
    \begin{enumerate}
        \item \textbf{Ne faites pas confiance à ce document}. Ce prétendu solutionnaire n'a été vérifié ni par des professeurs, ni par des tuteurs compétents et peut donc contenir des erreurs. Si vous en trouvez une, n'hésitez pas à le signaler dans la section \texttt{issues} du projet accessible par le lien plus haut.
        \item L'usage d'un formulaire récent (p. ex. 2015--2016) rend certaines questions triviales ; par conséquent, on considèra comme formules acquises uniquement les lois fondamentales et les définitions.
    \end{enumerate}
]

% \begin{document}

\section{}
Un disque diélectrique circulaire de centre $O$, de rayon $a = \SI{10}{\centi\meter}$ et situé dans le plan $xOy$, porte une charge $Q = \SI{e-6}{\coulomb}$ répartie uniformément sur toute sa surface. \\
Ce disque tourne à raison de \SI{3000}{\tour\per\minute} dans le sens positif (main droite autour de l'axe $Oz$). \\
Calculer le champ d'induction magétique $B$ apparaissant au centre de ce disque. Pour le calcul on considérera ce disque comme un dispositif plan sans épaisseur.

\begin{solution}
    Pour résoudre le problème, nous allons diviser le disque en une série d'anneaux concentriques. On peut tout d'abord affirmer
    \[ \B = \int \dif\B = \mu_0 \int \dif\H \]
    Aucune information sur la perméabilité relative n'est donnée ; on peut donc admettre que le système est dans le vide. \\
    Il s'agit d'exprimer $\dif\H$ pour chaque anneau de rayon $r$. Par Biot et Savart :
    \[
        \dif\H(r) = \frac{1}{4\pi} \int \frac{I \uvt \times \uvr}{r^2} \dif l = \frac{1}{4\pi} \int \frac{I \uvz}{r^2} \dif l
    \]
    Le courant et le rayon sont constants pour un même anneau ; on a donc
    \[ \dif\H(r) = \frac{I(r)}{4\pi r^2} \ 2\pi r \ \uvz = \frac{I(r)}{2r} \uvz \]
    Par définition, le courant est la variation de charge par rapport au temps. Si $\sigma = \frac{Q}{\pi a^2}$ est la charge par unité de surface, on a
    \[
        I(r) = \fdif{q}{t} = \frac{\sigma\dif r \dif x}{\dif t}
    \]
    où $\dif r$ est la différentielle de rayon et $\dif x$ la différentielle tangente à l'anneau. Sachant que $\fdif{x}{t} = v = \omega r$, on obtient 
    \[
        I(r) = \sigma\omega r \dif r
    \]
    On remplace et on intègre sur tout le rayon du disque (on se doute bien que le vecteur sera orienté selon $\uvz$):
    \[
        B = \mu_0 \int_0^a \frac{\sigma\omega r \dif r}{2r} = \frac{1}{2} \mu_0\sigma\omega a
    \]
    En remplaçant $\sigma$ et $\omega$, on obtient
    \[
        B = \frac{\mu_0 Q f}{a} \approx \SI{6.28e-10}{T}
    \]
\end{solution}

\section{}
Une bobine est constituée de $N = 1000$ spires de conducteur bobinées régulièrement sur un noyay de forme toroïdal (voir figure ci-dessous) dont le noyau magnétique possède une perméabilité magnétique relative $\relpmb = 1000$ constante. Soit $R = \SI{.5}{\meter}$ le rayon moyen du tore et $r = \SI{2.5}{\centi\meter}$ le rayon de sa section circulaire. La bobine est parcourue par un courant $I$. \\
\begin{center}
    \includegraphics[width=.5\textwidth]{img/q2.png}
\end{center}
\begin{enumerate}
    \item Que vaut le champ magnétisant $H$ le long du cercle pointillé de rayon $R$, lorsque le courant $I = \SI{1}{\ampere}$ ?
    \item Que vaut alors le champ magnétique $B$ pour le matériau considéré ?
    \item Calculez l'inductance $L$ de ce dispositif.
\end{enumerate}

\begin{solution}
    \begin{enumerate}
        \item 
            Appelons $C$ le cercle de rayon $R$ en question. Par la loi d'Ampère
            \[ \int_C \H \cdot \dif\vec{l} = I_\mathrm{tot}. \]
            Les spires étant bobinées régulièrement autour de ce cercle, le champ magnétisant est le même en tout point. Le courant total est la somme de chacun des courants traversant le contour fermé, c'est-à-dire $I_\mathrm{tot} = NI$. En remplaçant, on trouve
            \[ 2\pi R H = NI \Rightarrow H = \frac{NI}{2\pi R} \approx \SI{318.3}{\ampere\per\meter} \]
        \item 
            Puisqu'on considère le champ à l'intérieur du matériau, on multiplie le champ magnétisant par la perméabilité magnétique de ce matériau :
            \[ B = \pmb H = \vacpmb\relpmb H \approx \SI{.4}{\tesla} \]
        \item 
            Par définition, l'inductance du dispositif vaut
            \[ L \eqdef \frac{N\Phi}{I} \]
            où $\Phi$ est le flux magnétique d'une tranche arbitraire. Si $S$ est l'aire de cette tranche, on a
            \[ L = \frac{NBS}{I} = \frac{\pi NBr^2}{I} \approx \SI{.785}{\henry} \]
    \end{enumerate}
\end{solution}



\section{}


\end{document}
