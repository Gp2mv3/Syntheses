\documentclass[fr]{../../../../../../eplexam}

\usepackage{../../../../../../eplcommon}
\usepackage{../../../../../../eplunits}
\usepackage{physics}
\usepackage{tikz}

\newcommand{\elasticity}{\varepsilon}
\newcommand{\pcompt}{\pi}
\newcommand{\ppur}{\pi_{\textnormal{pur}}}
\newcommand{\surplus}{s}
\newcommand{\surpluscons}{S_{\textnormal{c}}}
\newcommand{\surplusprod}{S_{\textnormal{p}}}
\newcommand{\qi}{q_i}
\newcommand{\surpluscoll}{S}
\newcommand{\Lagr}{\mathcal{L}}
\usepackage{float}
\newcommand{\cmarg}{C_{\textnormal{m}}}
\newcommand{\cmoy}{C_{\textnormal{M}}}
\newcommand{\popt}{p^*}
\newcommand{\qopt}{q^*}
\newcommand{\mathsc}[1]{{\normalfont\textsc{#1}}}
\newcommand{\pmon}{p^{\mathsc{mon}}}

\newcounter{choice}
\renewcommand\thechoice{\textbf{\Alph{choice}}}
\newcommand\choicelabel{\thechoice$\quad$}

\newenvironment{choices}%
  {\list{\choicelabel}%
     {\usecounter{choice}\def\makelabel##1{\hss\llap{##1}}%
       \settowidth{\leftmargin}{W.\hskip\labelsep\hskip 2.5em}%
       \def\choice{%
         \item
       } % choice
       \labelwidth\leftmargin\advance\labelwidth-\labelsep
       \topsep=0pt
       \partopsep=0pt
     }%
  }%
  {\endlist}

\hypertitle{Sciences humaines - Économie de l'entreprise}{2}{EPL}{1803}{2019}{Septembre}{All}
{Stanislas Gorremans}
{Daxhelet Olivier \and Boucher Jacqueline }

\section*{Consignes et formules utiles}

\subsection*{Note}

Cette examen est celui donné au bac 2 et non celui au bac1.

Ces questions sont dans l'ordre du formulaire rose. 
Le formulaire bleue contenait les mêmes questions dans un ordre différent.

\subsection*{Consignes}

\begin{itemize}
     \item Chaque question a une bonne réponse unique.

     \item Seules les réponses cochées sur le formulaire adéquat seront prises
     en compte. Les consignes sur la façon de remplir ce formulaire sont
     reprises en fin d’examen.

     \item Le questionnaire peut être utilisé comme brouillon.

     \item L’examen est noté sur 20 points, attribués de la façon suivante
     (avant arrondi) :

     \begin{equation*}
          \begin{array}{l r}
               \textnormal{Bonne réponse}    & +1\phantom{.} \\
               \textnormal{Abstention}       & 0\phantom{.} \\
               \textnormal{Mauvaise réponse} & -\frac{1}{3}.
          \end{array}
     \end{equation*}
\end{itemize}

\subsection*{Formules utiles}

\begin{equation*}
     \phantom{,} \alpha + \alpha^2 + \alpha^3 + \cdots + \alpha^t =
     \frac{\alpha - \alpha^{t+1}}{1 - \alpha},
\end{equation*}

en particulier

\begin{equation}
     \label{r}
     \phantom{.} \frac{1}{1+r} + \left(\frac{1}{1+r}\right)^2 + \left(\frac{1}{1+r}\right)^3 + \cdots + \left(\frac{1}{1+r}\right)^t =
     \frac{1 - \left(\frac{1}{1+r}\right)^{t}}{r}.
\end{equation}

L'élasticité est donnée par

\begin{equation*}
     \phantom{.} \elasticity = \fdif{q}{p} \frac{p}{q}.
\end{equation*}

En absence d'arbitrage, on a

\begin{equation*}
     \phantom{,} 1 + r_{t,s} = (1 + r_{t}) (1 + r_{t+1}) \cdots (1 + r_{s-1}),
\end{equation*}

où $r_t = r_{t,t+1}$.

\newpage

\section{Question 1}
Le magazine LLN-News a reçu le premier juillet des bobines de papier destinées à son imprimerie pour un montant de 5 000 euros.
Le fournisseur consent un délai de paiement de deux mois maximum. Le 10 août, à court de liquidités, LLN-News  décide de régler sa dette en contractant un emprunt bancaire pour le même montant. 
La banque verse directement le montant dû sur le compte du fournisseur, de sorte  que cet argent ne transite pas par le compte de LLN-News. 

Sélectionner une suite correcte au début de phrase suivant : A la date du 10 août, le paiement de la dette au fournisseur moyennant un emprunt bancaire...

\begin{choices}
\choice ...fait baisser de 5 000 euros le passif de LLN-News, via une réduction de sa dette fournisseur.
\choice ...ne change rien à la valeur de l'actif ou du passif de LLN-News, mais modifie la répartition au sein des rubriques de son passif.
\choice ...ne change rien à la valeur de l'actif ou du passif de LLN-News, mais modifie la répartition entre rubriques de son actif.
\choice ...fait augmenter de 5 000 euros la valeur de l'actif de LLN-News, via une augmentation des stocks.
\end{choices}

%\nosolution

\section{Question 2}

Un marché ayant une fonction de demande\\
$$D(q) = 120 - q $$ \\
est desservi par une firme unique, ayant un coût total de production donné par
$$C(q) = 10 + 2q +2q^2.$$
Si la firme se comporte en monopole sur ce marché, elle détermine la quantité q qu'elle met sur le marché de façon à maximiser son profit.
Quelle sera alors la quantité à l'équilibre ? 

\begin{choices}
\choice 95.0
\choice 12.5
\choice 50/3
\choice $\sqrt{231}$ - 11
\end{choices} 

%\nosolution

\section{Question 3}
Dans un marché concurrentiel, laquelle de ces expressions est fausse ?

\begin{choices}
\choice Un producteur peut différencier son produit/service de celui des autres producteurs par une action publicitaire.
\choice La courbe d'offre du marché est croissante.
\choice Il y a libre entrée et sortie du marché. 
\choice La courbe de demande du marché est décroissante.
\end{choices}

%\nosolution

\section {Question 4}
Soit un marché en concurrence parfaite, sur lequel les courbes de l'offre et de la demande sont données respectivement par
$$O(q) = 6 + 2q \and D(q) = \begin{cases} (9-q)^2, & \mbox{si } q \le 9, \\ 0, & \mbox{si } q > 9 \end{cases} $$
de sorte que la quantité maximale échangée sur le marché est 9. Calculez le surplus des producteurs à l'équilibre. 

\begin{choices}
\choice $\frac{500}{3}$
\choice $\frac{425}{3}$
\choice 75
\choice 25
\end{choices}

%\nosolution

\section{Question 5}
Soient deux projets, A et B, dont Ies flux de trésorerie annuels sur trois années (t=0, 1, 2) sont donnés respectivement par 
$$F_A = (-540, 660, 0) \and F_B = (-360, 0, +484).$$
Le taux d'intérêt annuel vaut $r=10\%$, et on suppose I'absence de risque. Ces deux projets sont, exclusifs, de sorte qu'on ne peut choisir de réaliser les deux. Quel projet doit-on choisir ? 

\begin{choices}
\choice Aucun — les deux projets ne sont pas rentables.
\choice Le projet B. 
\choice Le projet A. 
\choice Impossible de trancher car les deux projets ont la meme valeur.
\end{choices}

%\nosolution

\section{Question 6}
Supposons qu'un monopole naturel, caractérisé par des rendements d'échelle
croissants, passe sous le contrôle d'un régulateur qui lui impose de vendre sa
production au coût marginal. Laquelle de ces affirmations est-elle fausse ?

\begin{choices}
\choice Le surplus social est maximisé.
\choice Le monopole a besoin de subsides pour continuer à fonctionner.
\choice Le monopole réduit sa production.
\choice Le monopole évolue d'une situation profitable à une perte.
\end{choices}

%\nosolution

\section{Question 7}
On considère un marché en duopole, avec deux firmes A et B. La fonction de
demande est $D(q) 150 - 2q$ et les firmes ont respectivement des fonctions de
coût

$$C_A(q_A) = 30q_A + 2q_A^2 \and C_B(q_B) = 30 q_B + 2q_B^2$$

On suppose que les deux firmes connaissent la fonction de demande et maximisent
leur profit, mais que la firme A se comporte en leader, tandis que la
firme B se comporte en follower. A l'équilibre, les productions des deux firmes
seront respectivement :

\begin{choices}
\choice $q_A = 16 \and q_B = 11$
\choice $q_A = 12 \and q_B = 12$
\choice $q_A = 90/7 \and q_B=165/14$
\choice $q_A = 15 \and q_B =15$
\end{choices}

%\nosolution

\section{Question 8}
Laquelle des affrmations suivantes est-elle fausse dans un marché oligopolistique ?

\begin{choices}
\choice Dans un duopole de Bertrand, les firmes décident le prix qu'elles pratiquent
alors que dans un équilibre de Cournot, elles décident des quantités à produire.
\choice Dans un duopole de Stackelberg, les deux firmes fixent simultanément leur
niveau de production.
\choice Le duopole asymétrique de Bowley n'est pas un équilibre économique.
\choice Dans le modèle de la firme dominante, certaines firmes se comportent comme
des producteurs en situation de concurrence parfaite.
\end{choices}

%\nosolution

\section{Question 9}
Pour assurer la maintenance d'une usine de taille $\lambda$, les frais fixes
s'élèvent à $20 + 4\lambda$. Ensuite, le cout additionnel pour produire
une quantité q est de $2q + 16$ de sorte que le total s'écrit 

$$C(q, \lambda) = 20 + 4\lambda + 2q + 16 \frac{q^2}{\lambda} $$

et l'on suppose que la firme peut, à long terme, ajuster sans frais la taille de
son usine pour l'adapter à son niveau de production. Quel sera le coût total de
production à long terme si la firme souhaite produire une quantité q = 10 ?

\begin{choices}
\choice 10
\choice 180
\choice 200
\choice 40
\end{choices}

%\nosolution

\section{Question 10}
Vous demandez à votre banque un emprunt de 10.000 euros sur un an. La banque
 estime que vous avez deux chances sur trois d'être capable de rembourser le
 capital et les intérêts au terme de l'emprunt. Pour se protéger du risque 
de défaut de paiement, la banque exige une garantie de 6.000 euros, qu'elle
 pourra saisir si vous n'êtes pas capable de rembourser à l'échéance.\\

Considérant que le taux sans risque est aujourd'hui de $0 \%$ et que
 la banque est neutre au risque, quel sera le taux d'intérêt à partir
 duquel la banque acceptera de vous consentir un tel prêt ?

\begin{choices}
\choice $0 \%$
\choice $33.333....... \%$
\choice $20 \%$
\choice D La banque n'accepte pas de vous prêter la somme demandée.
\end{choices}

%\nosolution

\section{Question 11}

Dans un marché en duopole avec deux firmes A et B, la fonction de demande
est donnée par
$$ D(q) = 100 - 2q.$$
Les deux firmes ont respectivement des fonctions de cont total de production
données par
$$C_A(q_A) 300 + 30 q_A + \frac{1}{2}q_A^2 et C_B(q_B) = 200 + 30 q_B + \frac{1}{2}q_B^2 $$
et ont la possibilité de se retirer du marche si elles le souhaitent, auquel cas
leur coût est nul. Si les deux firmes se comportent selon le modèle de Cournot,
quelle sera à l'équilibre la quantité offerte par la firme A ?

\begin{choices}
\choice $q_A = 14$
\choice $q_A = 0$
\choice $q_A = 10$
\choice $q_A = 70/3$
\end{choices}

%\nosolution

\section{Question 12}

Une firme dispose de plusieurs sites de production pour un même bien. Elle a
la possibilité de choisir comment elle répartit sa production entre deux sites,
dont les fonctions de coût respectives sont données par
$$ C_1(q_1) = 5 q_1 + q_1^2 \and C_2(q_2) = q_2 + \frac{1}{2} q_2^2 $$
Si la firme souhaite produire une quantité totale q = 10, comment choisira-t-
elle de répartir sa production entre les deux sites ?

\begin{choices}
\choice $q_1 = 0 et q_2 = 10$ car le coût de production du site 2 est toujours inférieur
celui du site 1.
\choice $q_1 = 5 \and q_2 =5$
\choice $q_1 = 2 \and q_2 = 8.$
\choice Aucune de ces propositions.
\end{choices}

%\nosolution

\section{Question 13}
Soit un marché financier sans arbitrage, dans lequel on propose
les produits sans risque suivants (années t = 0, 1, 2) :
\begin{itemize}
\item Produit A : investir 100.0 en t = 0 rapportera 115.5 en t = 2,
\item Produit B : investir 100.0 en t = 1 rapportera 105.0 en t = 2.
\end{itemize}
Quels sont les taux annuels $r_{01} et r_{12}$ ?

\begin{choices}
\choice $r_{01} = 15.5\% et r_{12} = 5\%$
\choice $r_{01} = 10\% et r_{12} = 5\%$
\choice $r_{01} = 5\% et r_{12} = 10\%$
\choice Il n'est pas possible de déterminer les taux annuels.
\end{choices}

\section{Question 14}
Une firme produit un bien en quantité Q à partir de deux facteurs
de production (intrants) en quantités $x_1 et x_2$. La quantité totale
de bien produit est donnée par la fonction de production

$$Q = f(x_1, x_2) = \sqrt{x_1} + \frac{1}{2}x_2,$$

et les prix unitaires respectifs de ces deux facteurs sont $p_1$ = 1 et 
$p_2$ = 4. On suppose que la firme minimise ses coûts de production. Quel
sera dans ce cas le coût de produire une quantité totale Q = 10 ?

\begin{choices}
\choice 100.0
\choice 68.0
\choice 80.0
\choice 64.0
\end{choices}

%\nosolution

\section{Question 15}
Laquelle de ces propositions est-elle fausse ?

\begin{choices}
\choice Un monopole discriminant au premier degré conduit à un
 état Pareto-optimal d'un point de vue de la société.
\choice Un monopole discriminant au troisième degré pratique un prix plus élevé sur
le segment ou la demande est la moins élastique.
\choice Donner aux consommateurs des rabais en fonction de la quantité commandée
est une discrimination au troisième degré.
\choice Discriminer n'est pas une pratique interdite par le droit de la concurrence.
\end{choices}

%\nosolution

\section{Question 16}
Soit un marché dont la fonction de demande est

$$D(q) = 240 - 2q.$$

On y trouve deux firmes dont Ies fonctions de coût sont

$$C_A(q_A, q_B) = 40 q_A +4 q_A (q_A + q_B) \and C_B(q_B, q_A) = 40 q_B - 4 q_B(q_A q_B),$$

donc impactées par des externalités. On sait que Ies deux firmes se comportent
en price-taker. Que peut-on dire du surplus collectif à l'équilibre sur ce marché ?

\begin{choices}
\choice Le surplus collectif est égal au surplus collectif maximal atteignable sur ce
marché.
\choice Le surplus collectif est inférieur de 125 au surplus collectif maximal atteignable
sur ce marché.
\choice Le surplus collectif est inférieur de 250 au surplus collectif maximal atteignable
sur ce marché.
\choice On ne peut pas calculer le surplus collectif en présence d'externalités.
\end{choices}

%\nosolution

\section{Question 17}
Soit une firme produisant un bien unique, qu'elle vend sur deux marchés A
et B, distincts et cloisonnés. La demande sur les deux marchés est donnée
respectivement par

$$ D_A(q_A) = 100 -q_A \and D_B(q_B) = 160 - 2q_B $$

Supposons que la fonction de coût total de production soit donnée par
et qu'il n'y ait pas de coût supplémentaire pour acheminer le bien vers les
deux marchés.

Si la firme se comporte en price-taker sur Ies deux marchés, quelles seront les
quantités vendues l'équilibre ?

\begin{choices}
\choice $q_A = 0 et q_B = 30.$
\choice $q_A = 40 et q_B = 50.$
\choice $q_A = 60 et q_B = 0.,$
\choice Aucune des propositions précédentes.
\end{choices}

%\nosolution

\section{Question 18}

Laquelle de ces propositions est-elle correcte, ? (complétez l'énoncé suivant)
Une élasticité-prix avec une valeur absolue de 2 pour un bien...

\begin{choices}
\choice est caractéristique d'une demande inélastique.
\choice mène à une augmentation des recettes de vente lorsque le prix du bien augmente.
\choice mène à une diminution des quantités demandées du bien de $2\%$ lorsque le prix
du bien augmente de $1\%$.
\choice mène à une augmentation des quantités demandées du bien de $2\%$ lorsque le
prix du bien augmente de $1\%$.
\end{choices}

%\nosolution

\section{Question 19}

Sur un marché en monopole, les fonctions de demande et de coût total sont
données respectivement par

$$D(q) = 150 - q \and C(q) = 1000 +30q +q2$$

Etant donné son caractère de monopole, l'entreprise est obligée de rester sur le
marché, mais l'état décide de contrôler les prix, tout en maximisant le surplus
collectif si un choix est possible.

Quel sera la quantité échangée si l'état contraint l'entreprise vendre son
coût moyen ?
\begin{choices}
\choice q = 30
\choice q = 50
\choice q = 10
\choice q = 40
\end{choices}

%\nosolution

\section{Question 20}

On dispose d'informations (partielles) sur les résultats économiques de la
Brutopie en 2018. On sait que la consommation totale de biens finaux y a
privés s'y sont élevés atteint un montant de 125 millions. Les investissements
privés s'y sont élevés à 75 millions et l'état a engagé pour 50 millions de dépenses publiques. Il y
a eu davantage d'exportations (20 millions) que d'importations (10 millions)
et la variation de stock n'a pas été significative (on peut la considérer comme
nulle). On sait encore que le niveau total des salaires s'y est monté à 120
millions, et que l'état a engrangé pour 90 millions d'impôts. On ne connaît
par contre ni le montant des bénéfices des entreprises, ni des subventions
accordées, ni la valeur ajoutée totale produite par l'économie du pays.

Moyennant ces informations, que peut-on en déduire sur le PIB nominal de la
Brutopie en 2018 ?

\begin{choices}
\choice Le PIB nominal de la Brutopie en 2018 vaut 240 millions.
\choice Le PIB nominal de la Brutopie en 2018 vaut 260 millions.
\choice Le PIB nominal de la Brutopie en 2018 vaut 350 millions.
\choice On ne peut pas calculer le PIB de Ia Brutopie.
\end{choices}

%\nosolution

RAPPEL: une abstention n'est prise en compte que si la case abstention est cochée.




\end{document}
