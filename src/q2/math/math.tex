\documentclass[11pt,a4paper]{article}

% French
\usepackage[utf8x]{inputenc}
\usepackage[T1]{fontenc}
\usepackage{lmodern}

\usepackage{ifthen}
\usepackage{url}


\usepackage{multirow}
%%% SECTION TITLE APPEARANCE
\usepackage{sectsty}
\allsectionsfont{\sffamily\mdseries\upshape} % (See the fntguide.pdf for font help)
% (This matches ConTeXt defaults)

% Color
% cfr http://en.wikibooks.org/wiki/LaTeX/Colors
\usepackage{color}
\usepackage[usenames,dvipsnames,svgnames,table]{xcolor}
\definecolor{dkgreen}{rgb}{0.25,0.7,0.35}
\definecolor{dkred}{rgb}{0.7,0,0}

\newcommand{\matlab}{\textsc{Matlab}}
\newcommand{\octave}{\textsc{GNU/Octave}}
\newcommand{\qtoctave}{\textsc{QtOctave}}
\newcommand{\oz}{\textsc{Oz}}
\newcommand{\java}{\textsc{Java}}
\newcommand{\clang}{\textsc{C}}

% Math symbols
\usepackage{amsmath}
\usepackage{amssymb}
\usepackage{amsthm}
\DeclareMathOperator*{\argmin}{arg\,min}
\DeclareMathOperator*{\argmax}{arg\,max}

% Sets
\newcommand{\Z}{\mathbb{Z}}
\newcommand{\R}{\mathbb{R}}
\newcommand{\Rn}{\R^n}
\newcommand{\Rnn}{\R^{n \times n}}
\newcommand{\C}{\mathbb{C}}
\newcommand{\K}{\mathbb{K}}
\newcommand{\Kn}{\K^n}
\newcommand{\Knn}{\K^{n \times n}}

% Chemistry
\newcommand{\std}{\ensuremath{^{\circ}}}
\newcommand\ph{\ensuremath{\mathrm{pH}}}

% Unit vectors
\usepackage{esint}
\usepackage{esvect}
\newcommand{\kmath}{k}
\newcommand{\xunit}{\hat{\imath}}
\newcommand{\yunit}{\hat{\jmath}}
\newcommand{\zunit}{\hat{\kmath}}

% rot & div & grad & lap
\DeclareMathOperator{\newdiv}{div}
\newcommand{\divn}[1]{\nabla \cdot #1}
\newcommand{\rotn}[1]{\nabla \times #1}
\newcommand{\grad}[1]{\nabla #1}
\newcommand{\gradn}[1]{\nabla #1}
\newcommand{\lap}[1]{\nabla^2 #1}


% Elec
\newcommand{\B}{\vec B}
\newcommand{\E}{\vec E}
\newcommand{\EMF}{\mathcal{E}}
\newcommand{\perm}{\varepsilon} % permittivity

\newcommand{\bigoh}{\mathcal{O}}
\newcommand\eqdef{\triangleq}

\DeclareMathOperator{\newdiff}{d} % use \dif instead
\newcommand{\dif}{\newdiff\!}
\newcommand{\fpart}[2]{\frac{\partial #1}{\partial #2}}
\newcommand{\ffpart}[2]{\frac{\partial^2 #1}{\partial #2^2}}
\newcommand{\fdpart}[3]{\frac{\partial^2 #1}{\partial #2\partial #3}}
\newcommand{\fdif}[2]{\frac{\dif #1}{\dif #2}}
\newcommand{\ffdif}[2]{\frac{\dif^2 #1}{\dif #2^2}}
\newcommand{\constant}{\ensuremath{\mathrm{cst}}}

% Numbers and units
\usepackage[squaren, Gray]{SIunits}
\usepackage{sistyle}
\usepackage[autolanguage]{numprint}
%\usepackage{numprint}
\newcommand\si[2]{\numprint[#2]{#1}}
\newcommand\np[1]{\numprint{#1}}

\newcommand\strong[1]{\textbf{#1}}
\newcommand{\annexe}{\part{Annexes}\appendix}

% Bibliography
\newcommand{\biblio}{\bibliographystyle{plain}\bibliography{biblio}}

\usepackage{fullpage}

% Logic
\DeclareUnicodeCharacter{22A8}{\tautologie}
\DeclareUnicodeCharacter{22AD}{\contradiction}


% le `[e ]' rend le premier argument (#1) optionnel
% avec comme valeur par défaut `e `
\newcommand{\hypertitle}[8][e ]{
\ifthenelse{\equal{#2}{en}}
{\usepackage[english]{babel}}
{\usepackage[frenchb]{babel}}

% Listing
% always put it after babel
% http://tex.stackexchange.com/questions/100717/code-in-lstlisting-breaks-document-compile-error
\usepackage{listings}
\lstset{
  numbers=left,
  numberstyle=\tiny\color{gray},
  basicstyle=\rm\small\ttfamily,
  keywordstyle=\bfseries\color{dkred},
  frame=single,
  commentstyle=\color{gray}=small,
  stringstyle=\color{dkgreen},
  %backgroundcolor=\color{gray!10},
  %tabsize=2,
  rulecolor=\color{black!30},
  %title=\lstname,
  breaklines=true,
  framextopmargin=2pt,
  framexbottommargin=2pt,
  extendedchars=true,
  inputencoding=utf8x
}



\usepackage{iflang}
\usepackage{tikz}
\usepackage{multicol}
\usetikzlibrary{positioning}

% Theorem and definitions
\theoremstyle{definition}
\IfLanguageName{english}
{
\newtheorem{mydef}{Definition}
\newtheorem{mynota}[mydef]{Notation}
\newtheorem{myprop}[mydef]{Property}
\newtheorem{myrem}[mydef]{Remark}
\newtheorem{myform}[mydef]{Formula}
\newtheorem{mycorr}[mydef]{Corollary}
\newtheorem{mytheo}[mydef]{Theorem}
\newtheorem{mylem}[mydef]{Lemma}
\newtheorem{myexem}[mydef]{Example}
\newtheorem{myineg}[mydef]{Inequality}
\newtheorem{mycipher}[mydef]{Cipher}
\newtheorem{myatk}[mydef]{Attack}
}
{
\newtheorem{mydef}{Définition}
\newtheorem{mynota}[mydef]{Notation}
\newtheorem{myprop}[mydef]{Propriété}
\newtheorem{myrem}[mydef]{Remarque}
\newtheorem{myform}[mydef]{Formules}
\newtheorem{mycorr}[mydef]{Corrolaire}
\newtheorem{mytheo}[mydef]{Théorème}
\newtheorem{mylem}[mydef]{Lemme}
\newtheorem{myexem}[mydef]{Exemple}
\newtheorem{myineg}[mydef]{Inégalité}
\newtheorem{mycipher}[mydef]{Cipher}
\newtheorem{myatk}[mydef]{Attaque}
}

\IfLanguageName{english}
{
\newcommand{\keyword}{mot clef}
\newcommand{\keywords}{mots clefs}
}
{
\newcommand{\keyword}{keyword}
\newcommand{\keywords}{keywords}
}

% Floats and referencing
\newcommand{\sectionref}[1]{section~\ref{sec:##1}}
\IfLanguageName{english}
{\newcommand{\annexeref}[1]{appendix~\ref{ann:##1}}}
{\newcommand{\annexeref}[1]{annexe~\ref{ann:##1}}}
\newcommand{\figuref}[1]{figure~\ref{fig:##1}}
\newcommand{\tabref}[1]{table~\ref{tab:##1}}
\usepackage{xparse}
\NewDocumentEnvironment{myfig}{mm}
{\begin{figure}[!ht]\centering}
{\caption{##2}\label{fig:##1}\end{figure}}

\usepackage{hyperref}
{\renewcommand{\and}{\unskip, }
\hypersetup{pdfauthor={#7},
            pdftitle={Summary of #3 Q#4 - L#5#6},
            pdfsubject={#3}}
}

\usepackage{pdfpages}

\title{\IfLanguageName{english}{Summary of }{Synth\`ese d#1}#3 Q#4 - L#5#6}
\author{#7}

\begin{document}

\ifthenelse{\isundefined{\skiptitlepage}}{
\begin{titlepage}
\maketitle

  \paragraph{\IfLanguageName{english}{Important Information}{Informations importantes}}
   \IfLanguageName{english}{This document is largely inspired from the excellent course given by}{Ce document est grandement inspiré de l'excellent cours donné par}
   #8
   \IfLanguageName{english}{at the }{à l'}%
   EPL (École Polytechnique de Louvain),
   \IfLanguageName{english}{faculty of the }{faculté de l'}%
   UCL (Université Catholique de Louvain).
   \IfLanguageName{english}
   {It has been written by the aforementioned authors with the help of all other students,
   yours is therefore welcome as well.
   It is always possible to improve it,
   It is even more true of the course has change because the summary must be updated accordingly.
   The source code can be found at the following address}
   {Il est écrit par les auteurs susnommés avec l'aide de tous
   les autres étudiants, la vôtre est donc la bienvenue.
   Il y a toujours moyen de l'améliorer, surtout si le cours change car la synthèse doit alors être mise à jour en conséquence.
   On peut retrouver le code source à l'adresse suivante}
   \begin{center}
     \url{https://github.com/Gp2mv3/Syntheses}.
   \end{center}
   \IfLanguageName{english}
   {You can also find there the content of the \texttt{README} file which contains
   more information, you are invited to read it.}
   {On y trouve aussi le contenu du \texttt{README} qui contient de plus
   amples informations, vous êtes invité à le lire.}

   \IfLanguageName{english}
   {It is written on it that questions, error reports,
   improvement suggestions or any discussion about the project
   are to be submitted at the following address}
   {Il y est indiqué que les questions, signalements d'erreurs,
   suggestions d'améliorations ou quelque discussion que ce soit
   relative au projet
   sont à spécifier de préférence à l'adresse suivante}
   \begin{center}
     \url{https://github.com/Gp2mv3/Syntheses/issues}.
   \end{center}
   \IfLanguageName{english}
   {It allows everyone to see them, comment and act accordingly.
   You are invited to join the discussions.}
   {Ça permet à tout le monde de les voir, les commenter et agir
   en conséquence.
   Vous êtes d'ailleurs invité à participer aux discussions.}

   \IfLanguageName{english}
   {You can also find informations on the wiki}
   {Vous trouverez aussi des informations dans le wiki}
   \begin{center}
     \url{https://github.com/Gp2mv3/Syntheses/wiki}
   \end{center}
   \IfLanguageName{english}
   {like the status of the summaries for each course}
   {comme le statut des synthèses pour chaque cours}
   \begin{center}
     \url{https://github.com/Gp2mv3/Syntheses/wiki/Status}
   \end{center}
   \IfLanguageName{english}
   {you can notice that there is still a lot missing,
   your help is welcome.}
   {vous pouvez d'ailleurs remarquer qu'il en manque encore beaucoup,
   votre aide est la bienvenue.}

   \IfLanguageName{english}
   {to contribute to the bug tracker of the wiki,
   you just have to create an account on GitHub.
   To interact with the source code of the summaries,
   you will have to install}
   {Pour contribuer au bug tracker et au wiki, il vous suffira de
   créer un compte sur GitHub.
   Pour interagir avec le code des synthèses,
   il vous faudra installer}
   \LaTeX.
   \IfLanguageName{english}
   {To directly interact with the source code on GitHub,
   you will have to use}
   {Pour interagir directement avec le code sur GitHub,
   vous devrez utiliser}
   \texttt{git}.
   \IfLanguageName{english}
   {If it is a problem,
   we are of course open to contributions sent by mail
   or any other mean.}
   {Si cela pose problème,
   nous sommes évidemment ouverts à des contributeurs envoyant leurs
   changements par mail ou par n'importe quel autre moyen.}
\end{titlepage}
}{}

\ifthenelse{\isundefined{\skiptableofcontents}}{
\tableofcontents
}{}
}


\usepackage{tensor}
\usepackage{array}
\usepackage{subfigure}
\usepackage{cellspace}
\usepackage{tabularx}
\usepackage{graphicx}
\addparagraphcolumntypes{X}

\DeclareMathOperator{\dist}{dist}
\DeclareMathOperator{\asin}{asin}
\DeclareMathOperator{\acos}{acos}
\DeclareMathOperator{\atan}{atan}
\DeclareMathOperator{\acot}{acot}
\DeclareMathOperator{\cis}{cis}
\DeclareMathOperator{\adj}{adj}
\DeclareMathOperator{\trace}{trace}
\DeclareMathOperator{\ind}{ind}
\DeclareMathOperator{\newdet}{det}
\DeclareMathOperator{\newker}{Ker}
\DeclareMathOperator{\newim}{Im}
\DeclareMathOperator{\newdim}{dim}
\DeclareMathOperator{\newrang}{rang}
\DeclareMathOperator{\newnull}{null}
\DeclareMathOperator{\newint}{int}
\DeclareMathOperator{\newfr}{fr}
\DeclareMathOperator{\newgrad}{grad}

\newcommand{\rot}{\vv{\mathrm{rot}}\;}
\newcommand{\sbt}{\,\begin{picture}(-1,1)(-1,-3)\circle*{2.5}\end{picture}\ }
\newcommand{\pa}{\partial}

%% Pour liste à la ligne
\newcommand*\InsertTheoremBreak{%
	\begingroup % keep changes local
		\setlength\itemsep{0pt}%
		\setlength\parsep{0pt}%
		\item[\vbox{\null}]%
	\endgroup%
}%

\hypertitle{Math\'ematique}{2}{FSAB}{1102}
{Nicolas Cognaux\and L\'eopold Cambier\and Beno\^it Legat}
{François Glineur, Roland Keunings et Enrico Vitale}

\part{Algèbre}
%       + <- x
%      /|
%     /è|
%    /gb|
%   /l r| <- x - P_V(x) \in V^\perp
%  /A  e|
% +-----+ <- P_V(x)

\section{Espaces euclidiens}
\begin{mydef}[Espace euclidien]
	Un espace euclidien $E$ est un espace vectoriel réel équipé d'un produit scalaire, c'est à dire d'une application
	$(-|-) : E \times E \to \R : (x, y) \mapsto (x|y)$ qui est
	\begin{description}
			\item[Bilinéaire]
			$\forall x, y, z \in E, \alpha, \beta \in \R$
			\begin{eqnarray*}
				(\alpha x + \beta y | z) & = & \alpha (x | z) + \beta (y | z)\\
				(x | \alpha y + \beta z) & = & \alpha (x | y) + \beta (x | z)
			\end{eqnarray*}
			\item[Symétrique]
			$\forall x,y \in E$
			$$(x|y) = (y|x)$$
			\item[Défini positif]
			$\forall x \in E \setminus \{0\}$
			$$(x|x) > 0$$
	\end{description}
\end{mydef}

\begin{myrem}
	Notons que le fait d'être symétrique et linéaire à gauche (resp. à droite) entraine automatiquement la linéarité à droite (resp. à gauche).
\end{myrem}

\subsection{Produit scalaire}

\begin{mydef}
	Soient $E$ un espace euclidien, $V \subseteq E$ et $x,y \in E$.
	La norme de $x$ et la distance entre $x$ et $y$ sont définies respectivement comme suit
	\begin{eqnarray*}
		||x|| & \eqdef & \sqrt{(x|x)}\footnotemark\\
		\dist(x, y) & \eqdef & ||x - y||\\
	\end{eqnarray*}
	\footnotetext{On comprend ici pourquoi le produit scalaire doit être défini positif.}
\end{mydef}

\begin{myprop}
	Soient $E$ un espace euclidien, $x,y \in E$ et $\alpha \in \R$.
	\begin{itemize}
		\item $(x|0) = 0 = (0|x)$;
		\item $||x|| \geq 0$;
		\item $\dist(x, y) \geq 0$;
		\item $x \neq 0 \iff ||x|| > 0$;
		\item $x \neq y \iff \dist(x, y) > 0$;
		\item $||\alpha x|| = |\alpha|||x||$;
		\item $|(x | y)| \leq ||x||\cdot||y||$ (Inégalite de Cauchy)
			\footnote{Le cas d'égalité se fait si et seulement si $x$ est parallèle à $y$.};
		\item $||x + y|| \leq ||x|| + ||y||$ (Inégalite triangulaire).
	\end{itemize}
\end{myprop}

\subsection{Orthogonalité}

\begin{mydef}
	Soient $E$ un espace euclidien, $V \subseteq E$ et $x,y \in E$.
	L'orthogonalité entre deux vecteurs et l'espace orthogonal sont définis respectivement comme suit
	\begin{eqnarray*}
		x \perp y & \stackrel{\Delta}{\Leftrightarrow} & (x|y) = 0\\
		V^{\perp} & \eqdef & \left\{x \in E | x \perp v, \forall v \in V\right\}
	\end{eqnarray*}
\end{mydef}

\begin{myprop}
	Soient $E$ un espace euclidien et $V \subseteq E$.
	\begin{itemize}
		\item $V \cap V^{\perp} \subseteq \{0\}$ avec égalite $\iff 0 \in V$;
		\item $E^{\perp} = \{0\}$;
		\item $\{0\}^{\perp} = E$;
		\item $V^{\perp}$ est un sev de $E$;
		\item $V_1 \subseteq V_2 \Rightarrow V_1^{\perp} \supseteq V_2^{\perp}$.
		\item $V \subseteq \left( V^{\perp} \right) ^{\perp}$;
	\end{itemize}
	Si $V$ est un sev de $E$
	\begin{itemize}
		\item $V = \left(V^\perp \right)^\perp$;
		\item $E = V \oplus V^{\perp}$ et cette somme est une somme directe;
			\item
			Si $E$ est de dimension finie, alors
			$\dim E = \dim V + \dim V^{\perp}$.
	\end{itemize}
\end{myprop}


\begin{mydef}
	Soient $E$ un espace euclidien et $x_1, x_2,... ,x_n \in E$,
	\begin{enumerate}
		\item La famille $x_1, x_2,... ,x_n$ est une famille orthogonale si
			\begin{itemize}
				\item $x_i \neq 0, \forall i$ 
				\footnote{Plus tard, on parlera de base orthogonale, or 0 ne peut
				pas faire partie d'une base, d'où cette condition.};
				\item $x_i \perp x_j, \forall i \neq j$.
			\end{itemize}

		\item La famille $x_1, x_2,... ,x_n$ est une famille orthonormée si
			\begin{itemize}
				\item $||x_i|| = 1, \forall i$;
				\item $x_i \perp x_j, \forall i \neq j$.
			\end{itemize}
	\end{enumerate}
\end{mydef}

\begin{myprop}
	Soient $E$ un espace euclidien, et $u_1, \ldots, u_n$ une base orthonormée de $V$
	\footnote{Cette propriété est utilisée dans le cours LFSAB1202 pour dire que ${\bf u}.{\bf v} = uv$
	où $u$ et $v$ sont leurs coordonnées respectives dans la base orthonormée $[{\bf\hat{I}}]$.}.
	$\forall \alpha_i, \beta_i \in \R,$
	$$(\alpha_1u_1 + \ldots + \alpha_nu_n | \beta_1u_1 + \ldots + \beta_nu_n) = \alpha_1\beta_1 + \ldots + \alpha_n\beta_n$$
\end{myprop}

\begin{myprop}\InsertTheoremBreak
	\begin{itemize}
		\item Une famille orthonormée est une famille orthogonale;
		\item Une famille orthogonale est une famille libre.
	\end{itemize}
\end{myprop}

\begin{myprop}
	Soient $E$ un espace euclidien, $V$ un sev de $E$ de dimension finie tel que $V \neq \{0\}$. $V$ admet une base orthonormée.
\end{myprop}

\begin{mycorr}
	Tout espace euclidien de dimension finie admet une base orthonormée.
\end{mycorr}

\begin{myprop}
	Soit $L : E \to F$ une application linéaire entre espaces euclidiens, avec $E$ de dimension finie.
	Il existe
	\footnote{On peut le démontrer sans trop de mal avec l'inégalité de Cauchy.}
	un réel $K \geq 0$ tel que, pour tout $x \in E$,
	\[ ||L(x)|| \leq K \cdot ||x|| \]
\end{myprop}

\subsection{Projection orthogonale}
\begin{mytheo}
	Soient $E$ un espace euclidien, $V$ un sev de dimension finie de $E$ et $x \in E$.
	Il existe un unique vecteur $P_V(x)$ tel que
	\[
	\left\{
	\begin{array}{l}
		P_V(x) \in V\\
		x - P_V(x) \in V^{\perp}
	\end{array}
	\right.
	\]
	On appelle ce vecteur $P_V(x)$ la projection orthogonale de $x$ sur $V$.

	Si $u_1, \ldots, u_n$ est une base orthonormée de $V$, alors
	\[ P_V(x) = (x|u_1)u_1 + \ldots + (x|u_n)u_n \]
\end{mytheo}

\begin{myrem}
On constate que le calcul de la projection orthogonale nécessite une somme de $n$ termes 
(où $n$ est la dimension de l'espace sur lequel on projette). En pratique, si $V$ est un
sev, il est parfois plus intéressant de projeter plutôt sur $V^{\perp}$ que sur $V$. En 
effet, si $\dim V^{\perp} < \dim V$, le calcul de la projection orthogonale sera plus rapide
sur $V^{\perp}$ que sur $V$ (le calcul d'une base orthonormée via la méthode de Gram-Schmidt
sera aussi plus rapide). Il est ensuite facile de retrouver la projection sur $V$ via 
la relation suivante :

\[ P_V(x) = x - P_{V^{\perp}}(x) \]
\end{myrem}

\begin{myprop}
	Soient $E$ un espace euclidien, $V$ un sev de $E$ et $x \in E$.
	\begin{itemize}
		\item $y \neq P_V(x) \Rightarrow \dist(x, P_V(x)) < \dist(x, y)$;
		\item $P_V : E \to E$ est une application linéaire;
		\item $\newker P_V = V^{\perp}$;
		\item $\newim P_V = V = \{x \in E | P_V(x) = x\}$;
		\item $P_V \circ P_V \circ \cdots \circ P_V = P_V$.
	\end{itemize}
\end{myprop}

\subsection{Méthode de Gram-Schmidt}

Supposons qu'on possède une base quelconque $(e_1, \dots , e_n)$ de $V$.
On peut construire une base orthonormée $(u_1, \dots, u_n)$ de la manière suivante
\begin{eqnarray*}
	u_1 &=& \frac{e_1}{||e_1||}\\
	u_2 &=& \frac{e_2 - u_1 \cdot (e_2|u_1)}{||e_2 - u_1 \cdot (e_2|u_1)||}\\
	u_3 &=& \frac{e_3 - u_1 \cdot (e_3|u_1) - u_2 \cdot (e_3|u_2)}{||e_3 - u_1 \cdot (e_3|u_1) - u_2 \cdot (e_3|u_2)||}\\
	\vdots &=& \vdots\\
	u_n &=& \frac{e_n - \sum_{i=1}^{n-1} u_i \cdot (e_n|u_i) }{ || e_n - \sum_{i=1}^{n-1} u_i \cdot (e_n|u_i) || }
\end{eqnarray*}

\subsection{Application aux systèmes d'équations linéaires}
Soit
\[ S : A \cdot x = b, A \in \R^{m \times n}, x \in \Rn, b \in \R^m \]
un système d'équation linéaire.
Posons $b' \eqdef P_{\mathcal{C}(A)}(b)$.
Les solutions du système $S' : A \cdot x = b'$ sont dites solutions approchées de $S$.

Si $b \in \mathcal{C}(A)$, on aura $b' = b$ et les solutions seront les solutions exactes.

%%%%%%%%%%%%%%%%%%%%%%%%%%%%%%%%%%%%%%%%%%%%%%%%%%%%%%%%%%%%%%%%%%%%%%%%%%%%%%%%%%%%%%%%%%%%%
%%%%%%%%%%%%%%%%%%%%%%%%%%%%%%%%%%%%%%%%%%%%%%%%%%%%%%%%%%%%%%%%%%%%%%%%%%%%%%%%%%%%%%%%%%%%%

\section{Diagonalisation de matrices et d'opérateurs linéaires}

\subsection{Définitions}

\begin{mynota}
	Pour la suite de cette synthèse, $\K$ dénote aussi bien $\R$ que $\C$.
\end{mynota}

\begin{mydef}[Opérateur linéaire]
	Soit $E$ un espace vectoriel sur un corps $\K$.
	Un opérateur linéaire sur $E$ est une application linéaire $L: E \to E$.
\end{mydef}

\begin{mydef}[Matrice diagonalisable]
	Une matrice $A \in \Knn$ est diagonalisable si $\exists P, D \in \Knn$, avec $P$ inversible et $D$ diagonale, telles que $A = P \cdot D \cdot P^{-1}$.
\end{mydef}

\begin{mydef}[Valeur propre, vecteur propre et espace propre] Soit $L : V \rightarrow V$ un opérateur linéaire
	\begin{itemize}
		\item $\lambda \in \K$ est \emph{valeur propre} si $\exists x \in V, x \neq 0$ tel que $L(x) = \lambda \cdot x$ ;
		\item un tel $x$ est dit \emph{vecteur propre} de $L$ associé à $\lambda$ ;
		\item si $\lambda \in \K$ est valeur propre de $L$, l'\emph{espace propre} associé à $\lambda$ est l'ensemble
			\[ E(\lambda) = \{ x \in V | L(x) = \lambda \cdot x \} \]
	\end{itemize}
\end{mydef}

\begin{myrem}
	\InsertTheoremBreak
	\begin{itemize}
		\item $E(\lambda) = \newker (L - \lambda \cdot I)$ donc $E(\lambda)$ est un sous-espace vectoriel de V ;
		\item $\lambda$ est valeur propre de L $\Leftrightarrow \newdim E(\lambda) > 0$.
	\end{itemize}
\end{myrem}

\begin{mydef}[Opérateur diagonalisable]
	Un opérateur linéaire $L$ est \emph{diagonalisable} si $V$ admet une base $f$ formée par les vecteurs propres de $L$. Dans ce cas, on a la relation
	\[ \tensor*[_e]{(L)}{_e}
	= \tensor*[_e]{(I)}{_f} \cdot \tensor*[_f]{(L)}{_f} \cdot \tensor*[_f]{(I)}{_e}
	= P \cdot D \cdot P^{-1} \]
\end{mydef}

\begin{mydef}[Multiplicité algébrique]
	Soit un opérateur linéaire $L$.
	La multiplicité algébrique d'une valeur propre $\lambda_i$ est la multiplicité de $\lambda_i$ en tant que racine du polynôme caractéristique en $\lambda$
	\[ P(\lambda) = \det \left( \tensor*[_e]{L}{_e} - \lambda I \right) \]
	On la note $m_a(\lambda_i)$.
\end{mydef}

\begin{mydef}[Multiplicité géométrique]
	La multiplicité géométrique d'une valeur propre $\lambda$ est définie ainsi
	\[ m_g(\lambda) \eqdef \newdim E(\lambda) \]
\end{mydef}

\begin{mydef}
	Soit $A \in \Knn$,
	la trace de $A$ notée $\trace(A)$ est la somme des éléments de sa diagonale.
\end{mydef}

\subsection{Diagonalisation}

\begin{myprop}
	Soient $L : V \to V$ un opérateur linéaire et $e = (e_1, \dots, e_n)$ une base de $V$:
	$L$ est diagonalisable $\Leftrightarrow \tensor[_e]{(L)}{_e}$ est diagonalisable.
\end{myprop}

\begin{myrem}
	\InsertTheoremBreak
	\begin{itemize}
		\item Les valeurs propres et les vecteurs propres de $A \in \Knn$ sont aussi les valeurs propres et les vecteurs propres de $L_A : \Kn \rightarrow \Kn : L_A(x) = A \cdot x$ ;
		\item Si $A$ est diagonalisable, alors
			$A = P\cdot D \cdot P^{-1}$ où
			\[
			D = \begin{pmatrix} \lambda_1 &  &  &  \\
				& \lambda_2 &  & \\
				& & \ddots & \\
				& & & \lambda_n \\
			\end{pmatrix}
			\]
			est la matrice diagonale des valeurs propres de $A$
			et $P$ est une matrice inversible dont les colonnes sont une base de $\Kn$ formée par les vecteurs propres de $A$.
			Les vecteurs propres en colonne dans $P$ doivent impérativement être dans le même ordre que leur valeur propre respective dans $D$.
		\item $\lambda_i \in \K$ est valeur propre de $A$ si et seulement si $\lambda_i$ est racine du polynôme caractéristique
			\[ \newdet (A - \lambda \cdot I) \in \K[\lambda]_n \]
	\end{itemize}
\end{myrem}

\begin{myprop}
	Soient $A$ une matrice \textbf{symétrique},
	$\lambda_1, \dots, \lambda_r$ des valeurs propres de $A$ \textbf{distinctes}
	et $x_1, \dots, x_r$ des vecteur propres relatifs, respectivement à $\lambda_1, \dots, \lambda_r$.
	La famille formée par les vecteurs $x_1, \dots, x_r$ est une famille \textbf{orthogonale} et donc aussi une famille \textbf{libre}.
\end{myprop}

\begin{myprop}[Multiplicité]
	Si $\lambda$ est l'une des valeurs propres d'un opérateur linéaire $L$, alors
	\[ 1 \leq m_g (\lambda) \leq m_a (\lambda) \]
\end{myprop}

\begin{myprop}[Conditions nécessaires et suffisantes pour qu'une matrice soit diagonalisable]
	Soit $\lambda_1 , \dots , \lambda_r$ l'ensemble des valeurs propres distinctes de $L : V \rightarrow V$ ($\newdim V = n$).
	Les conditions suivantes sont équivalentes:
	\begin{itemize}
		\item $L$ est diagonalisable
		\item $V = E(\lambda_1) \oplus \cdots \oplus E(\lambda_r)$ (c'est d'office en somme directe);
		\item $m_g(\lambda_1) + \cdots + m_g(\lambda_r) = n$ ;
		\item $m_a (\lambda_1) + \cdots + m_a (\lambda_r) = n$ et $m_g(\lambda_i) = m_a (\lambda_i)$ $\forall i = 1, \dots, r$.
	\end{itemize}
	\subparagraph{Remarque}
	Si $\K = \C$, la condition $m_a(\lambda_1) + \dots + m_a(\lambda_r)$ est tout le temps vraie.
	Sinon, la seule manière qu'elle soit fausse est que le polynôme caractéristique ait une racine complexe.
\end{myprop}

\begin{myprop}
	Soient $A \in \Knn$ et $\lambda_1, \dots, \lambda_r \in \K$ les valeurs propres distinctes de $a$.
	\begin{eqnarray*}
		\det(A) &=& \prod_{i = 1}^{r} \lambda_i^{m_a(\lambda_i)}\\
		\trace(A) &=& \sum_{i = 1}^{r} \lambda_i \cdot m_a(\lambda_i)
	\end{eqnarray*}
\end{myprop}

\begin{myrem}
	Notons que
	\footnote{Rappel ({\bf Le théorème de la nullité et du rang}):
	Soit $L:A \to B$ une application linéaire, on a $\newnull L + \newrang L = \dim A$}
	\begin{align*}
		m_g(\lambda) & = \newnull (L - \lambda \cdot I) \\
		& = n - \newrang (L - \lambda \cdot I)
	\end{align*}
	Une conséquence importante de cette relation est qu'on a toujours :
	\[ m_g(0) = m_a(0) \]
\end{myrem}

\subsection{Diagonalisation en pratique}
Soit une application linéaire $L$ avec $A = \tensor*[_e]{L}{_e}$.
Calculez le polynôme
\[ P(\lambda) = \det \left( A - \lambda I \right) \]
Calculez les $r$ différentes racines $\lambda_i$ de multiplicité algébrique $m_a(\lambda_i)$.
Vérifiez ici que $\sum_{i=1}^r m_a(\lambda_i) = n$. Si ce n'est pas le cas, $L$ et $A$ ne sont pas diagonalisables.
%TODO: cas où sum m_a != n ?

Pour chaque $\lambda_i$, vérifiez que $m_g(\lambda_i) = m_a(\lambda_i)$. Si ce n'est pas le cas, $L$ et $A$ ne sont pas diagonalisables.

Calculez une base de $E(\lambda_i)$ pour chaque $\lambda_i$ et ça devrait vous donner $n$ vecteurs $x_j$.
Prenez
\begin{eqnarray*}
	P &=& \begin{pmatrix}x_1 & \cdots & x_n\end{pmatrix}\\
	D &=&
	\begin{pmatrix}
		\lambda_1 & &\\
		&\ddots&\\
		&&\lambda_r
	\end{pmatrix}
\end{eqnarray*}
En plaçant $\lambda_i$ $m_g(\lambda_i) = m_a(\lambda_i)$
fois dans la diagonale de $D$.

Vous pouvez alors dire
\[ A = PDP^{-1} \]

\paragraph{Remarque}
Dans le cadre d'une évaluation, le calcul de l'inversion de $P$ ne sera sûrement pas requis, à moins que $n \leq 2$ ou que $P$ soit orthogonale.

En effet, rappelez vous que lorsque $A$ est \textbf{symétrique}, si $x_j$ et $x_k$ sont associé à les $\lambda_i$ différents, alors ils sont nécessairement orthogonaux.
Dès lors, si vous avec $m_g(\lambda_i) = 1$ pour tout $\lambda_i$, tous vos $x_i$ sont orthogonaux entre eux.
Vous aurez donc
\[
  P^{-1} =
  \begin{pmatrix}
    \frac{x_1}{\|x_1\|^2} & \cdots & \frac{x_n}{\|x_n\|^2}
  \end{pmatrix}^T
\]
Bien sûr si $P$ était orthonormée, $||x_i|| = 1$ pour tout $i$ et
donc $P^{-1} = P^T$.

%%%%%%%%%%%%%%%%%%%%%%%%%%%%%%%%%%%%%%%%%%%%%%%%%%%%%%%%%%%%%%%%%%%%%%%%%%%%%%%%%%
%%%%%%%%%%%%%%%%%%%%%%%%%%%%%%%%%%%%%%%%%%%%%%%%%%%%%%%%%%%%%%%%%%%%%%%%%%%%%%%%%%

\section{Formes quadratiques}

\subsection{Théorème spectral}

\begin{mydef}\InsertTheoremBreak
	\begin{enumerate}
		\item Une matrice carrée $A \in \Rnn$ est symétrique si $A^T = A$.
		\item Une matrice carrée $Q \in \Rnn$ est orthogonale si $Q^T\cdot Q = I$.
	\end{enumerate}
\end{mydef}

\begin{myrem}\InsertTheoremBreak
	\begin{enumerate}
		\item $Q$ est orthogonale précisément si ses colonnes sont une base orthonormée de $\R^n$ par rapport au produit scalaire canonique.
		\item Si $Q$ est orthogonale, alors elle est inversible et $Q^{-1} = Q^T$.
	\end{enumerate}
\end{myrem}

\begin{mytheo}[Théorème spectral]
	Soit $A \in \Rnn$.
	Les conditions suivantes sont équivalentes:
	\begin{itemize}
		\item $A$ est symétrique ;
		\item $\R^n$ admet une base orthonormée de vecteurs propres de $A$ ;
		\item $\exists Q \in \Rnn$ orthogonale, $D \in \Rnn$ diagonale, telles que
			\[ A = Q \cdot D \cdot Q^{-1} = Q \cdot D \cdot Q^T. \]
	\end{itemize}
	\subparagraph{Remarque}
	Le théorème spectral ne s'applique qu'au matrices réelles.
\end{mytheo}

\subsection{Définitions}

\begin{mynota}
	Soit $V$ un espace vectoriel réel de dimension finie $n$,
	et soit $e = \{e_1, \dots, e_n\}$ une base de $V$.
	Pour un vecteur $v = x_1e_1 + \dots + x_ne_n \in V$, nous écrivons
	\[ \tensor[_e]{v}{} = \begin{pmatrix}x_1\\\vdots\\x_n\end{pmatrix}. \]
\end{mynota}

Voici deux définitions équivalentes de ``Forme quadratique''.

\begin{mydef}[Forme quadratique]
	Soit $V$ un espace vectoriel \emph{réel} de dimension finie $n$,
	et soit $e = \{e_1, \dots, e_n\}$ une base de $V$.
	Une fonction $q: V \to \R$ est une forme quadratique s'il existe
	une matrice \emph{symétrique} $A \in \Rnn$ telle que, pour tout $v \in V$,
	on ait $q(v) = \tensor[_e]{v}{^T} \cdot A \cdot \tensor[_e]{v}{}$.
	Explicitement, si $v = x_1e_1 + \dots + x_ne_n$,
	\[ q(v) = \begin{pmatrix}x_1 & \dots & x_n\end{pmatrix} \cdot A \cdot \begin{pmatrix}x_1\\\vdots\\x_n\end{pmatrix} \]
	est un polynôme homogène de degré 2 par rapport aux $n$ indéterminées $x_1, \dots, x_n$.
	Nous notons $A = q(e)$ et nous l'appelons matrice associée à la forme $q$ par rapport à la base $e$.
	\subparagraph{Remarque}
	\begin{itemize}
		\item Une fois la base $e$ fixée, si la matrice $A$ existe, elle est unique;
		\item Le fait que $q: V \to \R$ soit une forme quadratique ne dépend pas du choix de la base.
	\end{itemize}
\end{mydef}

\begin{mydef}[Forme quadratique]
	Soit $V$ un espace vectoriel \emph{réel}.
	Une forme quadratique est une fonction
	\[ q : V \rightarrow \R \]
	satisfaisant à deux conditions :
	\begin{itemize}
		\item $\forall \lambda \in \R, \forall x \in V : q(\lambda x) = \lambda^2 q(x)$ ;
		\item $\bar{q} : V \times V \rightarrow \R$ est bilinéaire.
	\end{itemize}
	avec $\bar{q}(x,y) = \frac 12 (q(x+y) - q(x) - q(y) )$.
	\subparagraph{Remarque}
	$\bar{q}$ est \emph{symétrique} : $\bar{q}(x,y) = \bar{q}(y,x)$.
\end{mydef}

\subsection{Lien avec les matrices symétriques}

\begin{mynota}
	Soient $V$ un espace vectoriel réel et $e = (e_1, \dots , e_n)$ une base de $V$.
	On définit la notation suivante
	\[ q(e) =
	\begin{pmatrix}
		\bar{q}(e_1,e_1) & \ldots & \bar{q}(e_1,e_n)\\
		\vdots & \ddots & \vdots\\
		\bar{q}(e_n,e_1) & \ldots & \bar{q}(e_n,e_n)
	\end{pmatrix}
	\]
\end{mynota}

\begin{myprop}
	Soient $V$ un espace vectoriel réel et $e = (e_1, \dots , e_n)$ une base de $V$.
	Il y a une bijection entre
	\begin{itemize}
		\item Les formes quadratiques sur $V$ et
		\item les matrices symétriques $n \times n$ à coefficients réels.
	\end{itemize}
	De fait, par définition :
	\[ q(e)_{i,j} = \bar{q}(e_i, e_j) \]
	\[ q_{A,e} : V \rightarrow \R : q(x)_{A,e} = \tensor[_e]{x}{^T} \cdot A \cdot \tensor[_e]{x}{} \]
\end{myprop}

\begin{mycorr}
	De là, on tire que $q(e)$ est l'\emph{unique} matrice symétrique ($q(e) \in \Rnn$) telle que
	\[ q(x) = \tensor[_e]{x}{^T} \cdot q(e) \cdot \tensor[_e]{x}{} \]
\end{mycorr}

\begin{mycorr} Soit $V$ un espace vectoriel réel, $q$ une forme quadratique, $e = (e_1, \dots , e_n)$ et $f = (f_1, \dots, f_n)$ deux bases de $V$, alors
	\[ q(f) = (\tensor[_e]{I}{_f})^T \cdot q(e) \cdot \tensor[_e]{I}{_f} \]
	Réciproquement, si $B = P^T \cdot q(e) \cdot P$, il existe une base $f$ telle que $q(f) = B$.
	Il suffit de prendre la base $f$ déterminée par $P = \tensor[_e]{I}{_f}$.
\end{mycorr}

\begin{myrem}
	Deux matrices symétriques $A, B \in \Rnn$ sont associée à une même forme quadratique $q: V \to \R$ si il existe une matrice inversible $P$ telle que $B = P^T \cdot A \cdot P$.
\end{myrem}

\subsection{Caractère}

\begin{mydef}
	Soit $V$ un espace vectoriel réel, $q : V \rightarrow \R$ forme quadratique. La forme quadratique est
	\begin{itemize}
		\item définie positive si $q(x) >0 \; \forall x \in V, x \neq 0$ ;
		\item semi-définie positive si $q(x) \geq 0 \; \forall x \in V$ ;
		\item définie négative si $q(x) < 0 \; \forall x \in V, x \neq 0$ ;
		\item semi-définie négative si $q(x) \leq 0 \; \forall x \in V$ ;
		\item indéfinie si $\exists x \in V$ tel que $q(x) > 0$ et $\exists y \in V$ tel que $q(x) < 0$.
	\end{itemize}
\end{mydef}

\begin{myprop}
	\InsertTheoremBreak
	\begin{itemize}
		\item Le caractère d'une matrice $A$ est par définition celui de la forme quadratique associée
			$q_A: \Rn \to \R: q(x) = \tensor[_e]{x}{^T} \cdot A \cdot \tensor[_e]{x}{}$;
		\item Le caractère de $A$ ne dépend pas de cette base $e$ de $\R^n$.
	\end{itemize}
\end{myprop}

\subsection{Caractère et valeurs propres}

\begin{mydef}
	Soient $V$ un espace vectoriel réel, $q : V \rightarrow \R$ une forme quadratique et $e$ une base de $V$.
	On définit
	\[ \newrang q = \newrang q(e) \]
\end{mydef}

\begin{myprop}[Caractère et valeurs propres]
	Soient $q: V \to \R$ une forme quadratique et $e$ une base de $V$.
	Comme $q(e)$ est symétrique, par le théorème spectral, elle est donc diagonalisable,
	le caractère de $q$ peut être déterminé par les valeurs propres de $q(e)$. $q$ est
	\begin{itemize}
		\item définie positive si et seulement si $\lambda_1, \dots, \lambda_n > 0$;
		\item semi-définie positive si et seulement si $\lambda_1, \dots, \lambda_n \geq 0$ ;
		\item définie négative si et seulement si $\lambda_1, \dots, \lambda_n < 0$ ;
		\item semi-définie négative si et seulement si $\lambda_1, \dots, \lambda_n \leq 0$ ;
		\item indéfinie si $\exists  \lambda_i > 0$ et $\exists \lambda_j < 0$.
	\end{itemize}
\end{myprop}

\begin{myprop}[Loi d'inertie de Sylvester]
	Soit $q: V \to \R$ une forme quadratique.
	\begin{itemize}
		\item L'indice de positivité de la forme $q$ est le nombre de valeurs propres strictement positives de $q(e)$.
			Il est noté $\ind_+q$;
		\item L'indice de négativité de la forme $q$ est le nombre de valeurs propres strictement négatives de $q(e)$.
			Il est noté $\ind_-q$.
	\end{itemize}
	On a aussi
	\[ \ind_+q + \ind_-q = \newrang q \]
\end{myprop}

\begin{myrem}
	On voit que le caractère de $q$ dépend des valeurs propres de $q(e)$.
	Seulement, $q(e)$ et $q(f)$ n'ont pas spécialement les mêmes valeurs propres.
	Mais ils sont lié par la loi d'inertie de Sylvester.
	Ils ont le même nombre de valeurs propres strictement positives,
	le même nombre de valeurs proptres strictement négatives
	et 0 a la même multiplicité pour $q(e)$ et $q(f)$ qui vaut
	\[ n - \newrang q = n - \ind_+q - \ind_-q \]
\end{myrem}

\part{Équations différentielles}
% +--------------------------------------------------------------------+
% | d Équations différentielles                                        |
% | --------------------------- + Équations différentielles = e^{-x}   |
% |            dx                                                      |
% +--------------------------------------------------------------------+

\section{Définition}

\begin{mynota}
	Soit un corps $\K$.
	L'ensemble des polynômes à coefficients constants de $\K$ et de degré inférieur où égal à $n$ se note $\K[x]_n$.
\end{mynota}

\begin{mydef}[Problème de Cauchy]
	Un problème de Cauchy est défini de la manière suivante
	\[ \mathcal{P} = \left\{
	\begin{array}{l}
		y^{(n)}(t) + a_{n-1}y^{(n-1)}(t) + \cdots + a_1 y'(t) + a_0 y(t) = f(t) \\
		y(t_0) = \alpha_1, y'(t_0) = \alpha_2, \dots , y^{(n-1)}(t_0) = \alpha_n
	\end{array} \right.
	\]
	où $a_0, \dots, a_{n-1}, \alpha_1, \dots, \alpha_n \in \K, t_0 \in \R$ et $f: \R \to \K$ sont donnés,
	et il faut déterminer une fonction $y: \R \to \K$ dérivable $n$ fois
	telle que le système d'équation précédent soit vérifié pour tout $t \in \R$.
\end{mydef}

\begin{mytheo}[Existence et unicité]\InsertTheoremBreak
	\begin{itemize}
		\item S'il existe une solution au problème de Cauchy, alors elle est unique.
		\item Si $f$ est une combinaison linéaire de $(\mathrm{polynôme})\cdot(\mathrm{exponentielle})$,
			alors il existe une solution au problème de Cauchy.
	\end{itemize}
\end{mytheo}

\section{Solution}

Soit un problème de Cauchy $\mathcal{P}$. On étudie l'équation homogène et non-homogène séparément.

\subsection{Solution de l'équation homogène}
\begin{mytheo}[Structure de l'ensemble des solutions d'une équation différentielle homogène]
	Considérons l'équation homogène
	\[ y^{(n)}(t) + a_{n-1}y^{(n-1)} + \dots + a_1y' + a_0y = 0 \]
	Soient $r_1, \dots, r_q \in \C$ les racines du polynôme caractéristique
	\[ r^n + a_{n-1}r^{n-1} + \dots + a_1r + a_0 \]
	avec multiplicité respective $m_1, \dots, m_q$ (et donc $m_1 + \dots + m_q = n$).

	L'ensemble des solutions de l'équation homogène est l'ensemble est fonctions de la forme
    \[ \sum_{j=1}^{q} P_j(t) \cdot e^{r_j t} =
    P_1(t)e^{r_1 t} + P_2(t)e^{r_2 t} + \ldots + P_q(t)e^{r_q t} \]
	où $P_j \in \C[x]_{m_j -1}$, c'est à dire que $P_j$ est un polynôme
    de degré $m_j-1$ à coefficients complexes donc par exemple,
    si $r_1=42$ est une racine double du polynôme caractéristique,
    $P_1$ est un polynôme du premier degré à coefficients complexes.

	Autrement dit,
	\[ \begin{Bmatrix}
		e^{r_1t}, & te^{r_1t}, & \dots, & t^{m_1-1}e^{r_1t}\\
		\vdots & \vdots & \vdots & \vdots\\
		e^{r_qt}, & te^{r_qt}, & \dots, & t^{m_q-1}e^{r_qt}\\
	\end{Bmatrix} \]
	est une base de l'ensemble des solutions de l'équation homogène.

	Il est intéressant de remarquer que comme $m_1 + \dots + m_q = n$,
	la dimension de l'espace des solutions de l'équation homogène est $n$.
\end{mytheo}

\subsection{Solution particulière de l'équation non-homogène}
\begin{myprop}[Principe de superposition]
	Si $y_1(t)$ est solution de l'équation $\sum_0^n a_j y^{(j)} = f_1(t)$ et $y_2$ de l'équation $\sum_0^n a_j y^{(j)} = f_2(t)$,
	alors $y_1(t) + y_2(t)$ est solution de l'équation $\sum_0^n a_j y^{(j)} = f_1(t) + f_2(t)$.
\end{myprop}

\begin{myprop}
	\label{prop:part}
	Soient $r \in \C$ et $Q_p \in \C[x]_p$.
	L'équation non-homogène
	\[ \sum_{j=0}^n a_j y^{(j)}(t) = Q_p(t) \cdot e^{rt} \]
	admet toujours une solution de la forme
	\[ v(t) = t^m \cdot R_p(t) \cdot e^{rt} \]
	où $m$ est la multiplicité de $r$ en tant que racine du polynôme caractéristique,
	et $R_p \in \C[x]_p$ est un polynôme à déterminer.
\end{myprop}

\paragraph{En pratique}
Utilisez le principle de superposition et traitez séparément chaque terme de $f(t)$.
Pour chaque terme, trouvez la solution particulière ainsi
\begin{itemize}
	\item Si votre terme est de la forme $Q_p(t) \cdot e^{rt}$ avec $r \in \C$ et $Q_p \in \C[x]_p$,
		utilisez la Propriété~\ref{prop:part} et déterminez les coefficient de $R_p$ en injectant
		$t^m \cdot R_p(t) \cdot e^{rt}$ dans l'équation non-homogène avec juste $Q_p(t) \cdot e^{rt}$
		dans la partie non-homogène.
		Vous devriez pouvoir exprimez ainsi les coefficients par des expressions indépendantes de $t$.
		Si votre expression dépend de $t$, vos calculs sont faux.
	\item Si votre terme est de la forme $Q_p(t) \cdot \cos(rt)$ (resp. $Q_p(t) \cdot \sin(rt)$),
		prenez comme solution particulière la partie réelle (resp. imaginaire) d'une solution particulière de
		$Q_p(t) \cdot e^{irt}$.
		Malheureusement, cela ne marche \textbf{que si $a_1, \dots, a_n \in \R$}.
		Si ce n'est pas le cas, regardez s'il n'y a pas moyen de contourner le problème.
		Par exemple, si tous les coefficients sont imaginaires purs, essayez de tout diviser par $i$.
\end{itemize}
Ensuite, sommez toutes vos solutions particulières et vous aurez une solution particulière de l'équation non-homogène.

\subsection{Solution de l'équation non-homogène}
\begin{mytheo}[Structure de l'ensemble des solutions d'une équation différentielle linéaire]
	Soit $v: \R \to \C$ une solution particulière de l'équation non-homogène.
	Une fonction $y: \R \to \C$ est solution de l'équation non-homogène si et seulement si elle peut s'écrire sous la forme
	\[ y(t) = v(t) + u(t) \]
	où $u(t)$ est une solution arbitraire de l'équation homogène.
\end{mytheo}

\paragraph{En pratique}
Trouvez l'expression des solutions de l'équation homogène et une solution particulière de l'équation non-homogène.
La somme vous donne l'expression des solutions de l'équation non-homogène.

Il vous reste alors $n$ coefficient qui peuvent varier.
Si vous résolvez un problème de Cauchy, vous pouvez les déterminer à l'aide des conditions initiales.
\[ y(t_0) = \alpha_1, y'(t_0) = \alpha_2, \dots , y^{(n-1)}(t_0) = \alpha_n \]

\section{Solutions réelles}
Imaginons qu'on veuille se limiter aux solutions réelles.  Rappelons nous que $y$ est défini ainsi: $y: \R \to \C$.
On veut que $\newim y \subseteq \R$, c'est à dire $y(t) \in \R$ pour tout $t \in \R$.

Il y a 3 cas de figure à envisager avant d'attaquer la méthode la plus longue
\begin{itemize}
	\item Si on veut résoudre un problème de Cauchy, on trouvera une solution unique,
		il suffira donc de vérifier si $\newim y \subseteq \R$
	\item Si $a_1, \dots, a_n \in \R$,
		il y a une méthode efficace pour trouver les solutions réelles de l'équation homogène.
		Seulement, c'est la solution de la non-homogène qui doit être réelle.
		Dès lors, il se pourrait qu'une solution de l'équation homogène soit complexe,
		mais qu'additionnée avec une solution particulière, ça fasse une solution réelle.
		La méthode ne nous donnerait donc pas toutes les solutions réelles.
		Il vaut mieux donc appliquer cette méthode uniquement quand on a aussi $\newim f \subseteq \R$.

		Pour appliquer la méthode, on sépare les racines réelles des non réelles du polynôme caractéristique
		\[ r_1, \dots, r_p \in \R, \qquad s_1, \dots, s_t \in \C \setminus \R, \qquad
		\bar{s_1}, \dots, \bar{s_t} \in \C \setminus \R \]
		avec multiplicité respective $m_1, \dots, m_p$, $n_1, \dots, n_t$, $n_1, \dots, n_t$.
		Le fait de dire que si $s_j$ est racine, alors $\bar{s_j}$ l'est aussi vient du fait que $a_1, \dots, a_n \in \R$.
		C'est le seul moment où on utilise cette hypothèse.
		Posons $s_j = b_j + ic_j$.
		Les solutions réelles de l'équation homogène sont les fonctions du type
		\[ y(t) = \sum_{j = 1}^p P_j(t)e^{r_jt} + \sum_{k=1}^t(B_k(t)\cos(c_kt)+C_k(t)\sin(c_kt))e^{b_kt} \]
		avec $P_j \in \R[x]_{m_j-1}$, $B_k, C_k \in \R[x]_{n_k-1}$ arbitraires.
	\item Sinon, vous pouvez utiliser la méthode suivante,
		elle marche même si pour un certain $j$, $a_j \in \C$ ou $\exists t \in \R$ tel que $f(t) \in \C$.

		Pour appliquer la méthode, on trouve la solution complexe de l'équation non-homogène.
		On sépare, pour chaque coefficient, partie imaginaire et partie réelle
		(e.g. si vous avez un coefficient $A \in \C$, posez $A = A_x + iA_y$ avec $A_x, A_y \in \R$).
		N'oublions pas que les coefficients des polynômes $P_j$ dans l'équation homogène sont complexe.
		Transformez vos exponentielles $e^{a + bi}$ en $e^a(\cos(b) + i\sin(b))$.
		Développez votre expression pour isolez la partie réelle et imaginaire de la solution complexe de l'équation non-homogène.
		Imposez alors que la partie imaginaire est nulle.
		Vous aurez des conditions supplémentaires sur vos coefficients.
		Ça vous donnera votre expression finale de la solution réelle de l'équation non-homogène.
\end{itemize}

%%%%%%%%%%%%%%%%%%%%%%%%%%%%%%%%%%%%%%%%%%%%%%%%%%%%%%%%%%%%%%%%%%%%%%%%%%%%%%%%%%%%%%%
%%%%%%%%%%%%%%%%%%%%%%%%%%%%%%%%%%%%%%%%%%%%%%%%%%%%%%%%%%%%%%%%%%%%%%%%%%%%%%%%%%%%%%

\part{Calcul différentiel à plusieurs variables}

\section{Définition}


\begin{mydef}[Fonction à plusieurs variables]
	Une fonction
	\[ f : A \subseteq \R^n \rightarrow \R \]
	est définie par l'ensemble des paires $(a,f(a))$ pour chaque $a \in A$. Ces paires forment un sous-ensemble de $\R^n \times \R$ appelé graphe de $f$. Ces fonctions sont des fonctions scalaires.

	Une fonction
	\[ f : A \subseteq \R^n \rightarrow \R^m \]
	est définie par l'ensemble des paires $(a,f(a))$ pour chaque $a \in A$. Ces paires forment un sous-ensemble de $\R^n \times \R^m$ appelé graphe de $f$. Ces fonctions sont des fonctions vectorielles. On peut toujours décomposer ces fonctions en plusieurs fonctions scalaires. En d'autres mots, on peut toujours écrire
	\[ f = (f_1, f_2, \dots, f_n) \]
	où chaque $f_i$ est scalaire.
\end{mydef}

%%%%%%%%%%%%%%%%%%%%%%%%%%%%%%%%%%%%%%%%%%%%%%%%%%%%%%%%%%%%%%%%%%%%%%%%%%%%%%%%%%%%%%%
%%%%%%%%%%%%%%%%%%%%%%%%%%%%%%%%%%%%%%%%%%%%%%%%%%%%%%%%%%%%%%%%%%%%%%%%%%%%%%%%%%%%%%

\section{Concepts de base}

\subsection{Ensembles élémentaires}
Soit
\[ f : A \subseteq \R^n \rightarrow \R^m \]
Une fonctions est caractérisée par trois ensembles principaux :
\begin{mydef} [Graphe de $f$] Le graphe de $f$ est défini comme
	\[ \mathrm{graph} f = \{(a,f(a))| a \in A\} \subseteq \R^n \times \R^m \]
\end{mydef}

\begin{mydef} [Image de $f$] L'image de $f$ est l'ensemble
	\[ \mathrm{Im} f = \{ f(a) | a \in A\} \subseteq \R^m \]
\end{mydef}

\begin{mydef} [Ensemble de niveau k]
	L'ensemble de niveau $k$ de $f$ est
	\[ \text{ensemble de niveau } k\text{ de }f = \{ a | a \in A \;\text{et}\; f(a) = k \} \subseteq \R^n \]
\end{mydef}

%TODO: Restriction

\subsection{Points et ensembles particuliers}

\begin{mydef} [Boules] Les voisinages se définissent à partir des notions de boules ouvertes et fermées.
	\begin{itemize}
		\item Une boule \textbf{ouverte} centrée autour du point $a \in \R^n$:
			\[ B(a;r) = \{ x \in \R^n |\,||x-a|| < r \} \]
		\item Et une boule \textbf{fermée} centrée autour du point $a \in \R^n$:
			\[ B[a;r] = \{ x \in \R^n |\,||x-a|| \leq r \} \]
	\end{itemize}
\end{mydef}

\begin{mydef}[Point intérieur]
	$a$ est un point \textbf{intérieur} de $A$ si et seulement si il existe une boule ouverte centrée en $a$ qui est contenue dans $A$.
	C'est à dire, $\exists r > 0$ tel que
	\[ B(a; r) \subseteq A \]
\end{mydef}

\begin{mydef} [Point limite]
	$a$ est un point \textbf{limite} de $A$ si et seulement si toute boule ouverte centrée en $a$ intersecte $A$ autre part qu'en $a$.
	C'est à dire, $\forall r > 0$, on a
	\[ (A \setminus \{a\}) \cap B(a; r) \neq \emptyset \]
\end{mydef}

\begin{mydef} [Point isolé]
	$a$ est un point \textbf{isolé} de $A$ si et seulement si il existe une boule ouverte centrée en $a$ qui n'intersecte $A$ qu'en $a$.
	C'est à dire, $\exists r > 0$ tel que
	\[ A \cap B(a; r) = a \]
\end{mydef}

\begin{mydef}[Intérieur]
	L'intérieur de $A$ est l'ensemble des points intérieurs à $A$.
	Il se note $\newint A$.
\end{mydef}

\begin{mydef} [Fermeture]
	C'est l'ensemble des points limites et isolés.
	Elle se note $\bar{A}$.
\end{mydef}

\begin{mydef} [Frontière] La frontière de $A$ est définie par
	\[ \newfr A \eqdef \bar{A} \setminus \newint A \]
\end{mydef}

\subsubsection{Propriétés}

\begin{myprop}
	\[ \newint A \subseteq A \subseteq \bar{A} \]
\end{myprop}

\begin{myprop}
	\[ \newint B(a; r) = \newint B[a; r] = B(a; r) \]
\end{myprop}

\subsubsection{Résumé sous forme de schémas}

La figure \ref{fig:ensembles} présente un résumé des différents types de points et d'ensembles. Un trait plein signifie un bord fermé, des pointillés un bord ouvert, des hachures un ensemble <<~remplis~>>, et un point un point isolé.

\begin{figure}[!ht]
	\begin{center}
		\subfigure[$A$]{\includegraphics[height=1.8cm]{images/figuresasy-1}}\qquad
		\subfigure[int($A$) --- l'intérieur de $A$]{\includegraphics[height=1.8cm]{images/figuresasy-2}}\qquad
		\subfigure[$\bar{A}$ --- la fermeture de $A$]{\includegraphics[height=1.8cm]{images/figuresasy-3}}\\
		\subfigure[fr($A$) --- la frontière de $A$]{\includegraphics[height=1.8cm]{images/figuresasy-4}}\qquad
		\subfigure[Les points limites ou d'accumulation de $A$]{\includegraphics[height=1.8cm]{images/figuresasy-5}}\qquad
		\subfigure[Les points isolés de $A$]{\includegraphics[height=1.8cm]{images/figuresasy-6}}\\
		\caption{Les divers ensembles}
		\label{fig:ensembles}
	\end{center}
\end{figure}


\subsection{Ensembles ouverts, bornés et complémentaires}

\begin{mydef}
	$A$ est un ensemble \textbf{borné} s'il existe une boule ouverte qui contient $A$;
	sinon, l'ensemble $A$ est dit \textbf{non-borné}.
\end{mydef}

\begin{mydef}
	Une ensemble est \textbf{ouvert} si et seulement si
	\[ A = \newint A \]
	Un ensemble est \textbf{fermé} si et seulement si
	\[ A = \bar{A} \]
\end{mydef}

\begin{myprop}
	L'ensemble vide $\emptyset$ et l'espace complet $\Rn$ sont à la fois ouverts et fermés.
\end{myprop}

\begin{myprop}\InsertTheoremBreak
	\begin{itemize}
		\item La fermeture d'un ensemble est le plus petit ensemble fermé contenant cet ensemble,
		\item L'intérieur d'un ensemble est le plus grand ensemble ouvert contenu dans cet ensemble.
	\end{itemize}
\end{myprop}

\begin{mydef}
	Le complémentaire d'une ensemble $A \subseteq \Rn$ est l'ensemble des points n'appartenant pas à $A$.
	C'est à dire $\Rn \setminus A$.
\end{mydef}

\begin{mytheo}
	Le complémentaire d'un ensemble ouvert est toujours un ensemble fermé;
	celui d'une ensemble fermé est toujours ouvert.
\end{mytheo}

\subsection{Limites}

Dans cette partie, on considère $f$, une fonction telle que
\[ f : A \subseteq \R^n \to \R \]

\begin{mydef}[Limite d'une fonction à plusieurs variables]
	On peut définir la limite de $f$ en tout point de la \emph{fermeture} de son domaine (éventuellement hors de $A$)
	Si $a$ est un point isolé de $A$, alors
	\[ \lim_{x \to a} f(x) = f(a) \]
	Si $a$ est un point limite de $A$, alors
	\[ \lim_{x \to a} f(x) = L \]
	si et seulement si $\forall \epsilon >0, \exists \delta > 0$ tel que $x \in A$ et
	\[ 0 < || x - a || < \delta \Rightarrow |f(x) - L| < \epsilon \]
	Si la fonction est vectorielle et que $a$ est un point limite de $A$, alors
	\[ \lim_{x \to a} f(x) = L \]
	si et seulement si $\forall \epsilon >0, \exists \delta > 0$ tel que $x \in A$ et
	\[ 0 < || x - a || < \delta \Rightarrow ||f(x) - L|| < \epsilon \]
	(comme on dit $0 < ||x - a||$, la valeur de $f(a)$, si elle existe, n'influe en rien la limite).

	Si on écrit $f$ sous la forme $f = (f_1 , \dots , f_n)$ où chaque $f_i$ est une fonction scalaire, alors la limite de $f$ existe si et seulement si la limite de chaque $f_i$ existe
	\[ \lim_{x \to a} f(x) = \left( \lim_{x \to a} f_1(x), \lim_{x \to a} f_2(x), \dots , \lim_{x \to a} f_n(x) \right) \]
\end{mydef}

\begin{myprop}[Condition nécessaire et suffisante à l'existence d'une limite]
	Soit $a \in \bar{A}$ est $B = B(a; r)$ une boule ouverte centrée en $a$.
	Soit $f_B$ une restriction de la fonction $f$ à $B$
	\( \lim_{x \to a}f(x) \)
	existe si et seulement si
	\( \lim_{x \to a}f_B(x) \)
	existe et si elles existent elles sont égales.
\end{myprop}

\begin{myprop}[Condition nécessaire et \textbf{pas} suffisante à l'existence d'une limite]
	\label{prop:path}
	Soient $B$ un sous ensemble de $A$ et $a \in \bar{B}$.
	Soit $f_B$ une restriction de la fonction $f$ à un $B$.
	Si
	\( \lim_{x \to a}f(x) \)
	existe, alors
	\( \lim_{x \to a}f_B(x) \)
	existe et elles sont égales.
	\subparagraph{Remarque}
	\begin{itemize}
		\item On fait souvent ces restrictions le long de chemins: des (demi-)droites passant par $a$ ou d'autres courbes;
		\item Certaines fonctions n'ont pas de limite en un point alors que toutes les restrictions à des droites passant par
			ce point admettent une même limite;
		\item Si $\lim_{x \to a}f_B(x)$ n'exite pas, alors $\lim_{x \to a} f(x)$ n'existe pas non plus.
		\item Si $\lim_{x \to a}f_{B_1}(x)$ et $\lim_{x \to a}f_{B_2}(x)$ existent et sont différentes, alors
			$\lim_{x \to a} f(x)$ n'existe pas.
	\end{itemize}
\end{myprop}

\begin{myprop}[Recollement (condition nécessaire et suffisante)]
	\label{prop:rec}
	Soit $A = A_1 \cup A_2$ ; soit $f_1$, $f_2$ les restrictions respectives de $f$ à $A_1$ et $A_2$. Si $a \in \bar{A_1}$ et $a \in \bar{A_2}$, alors
	\[ \lim_{x \to a}f(x) = L \Leftrightarrow \left\{
	\begin{array}{l}
		\lim_{x \to a} f_1(x) = L \\
		\lim_{x \to a}f_2(x) = L
	\end{array} \right. \]
\end{myprop}

\begin{myrem}
	La différence entre la Propriété~\ref{prop:path} et la Propriété~\ref{prop:rec} c'est que
	si vous coupez le domaine $A$ en un nombre fini de restrictions dont l'union redonne $A$,
	alors vous avez la condition suffisante, sinon, vous ne l'avez pas.
\end{myrem}

\begin{myform}[Combinaisons algébriques]
	Les quatre opérations classiques ($+$, $-$, $\cdot$, $\newdiv$) sont possibles, mais \textbf{pas} la composition de limites.
\end{myform}

\begin{myprop}[Positivité] Si
	\begin{itemize}
		\item $f(x) \geq 0 \Rightarrow \lim_{x \to a} f(x) \geq 0$ ;
		\item $f(x) > 0 \Rightarrow \lim_{x \to a} f(x) \geq 0$ ;
		\item $f(x) \leq g(x) \Rightarrow \lim_{x \to a} f(x) \leq \lim_{x \to a} g(x)$.
	\end{itemize}
	NB : ces propriétés sont aussi valables dans un voisinage de $a$.
\end{myprop}

\begin{myprop}[Étau (condition suffisante)]
	Si $g \leq f \leq h$ et que $\lim_{x \to a} g(x) = \lim_{x \to a} h(x) = L$ alors
	\[ \lim_{x \to a} f(x) = L \]
	\begin{itemize}
		\item Fonctionne aussi dans un voisinage de $a$.
	\end{itemize}
\end{myprop}

\begin{myform}[Application de l'étau]\InsertTheoremBreak
	\begin{itemize}
		\item si $f \rightarrow 0$, alors $|f| \rightarrow 0$ ;
		\item si $f \to 0$, et que $g$ est bornée, alors $\lim_{x \to a} (g \cdot f) = 0$.
	\end{itemize}
\end{myform}

\begin{mydef}[Continuité]
	Une fonction $f : A \subseteq \R^n \rightarrow \R^m$ est continue en $a \in A$ si et seulement si
	\[ \lim_{x \rightarrow a}f(x) = f(a) \]
\end{mydef}

\begin{myprop}\InsertTheoremBreak
	\begin{itemize}
		\item Si $f$ et $g$ sont continues en $a$, alors
			$f \pm g$ et $f \cdot g$ sont continues en $a$.
			Si $g(a) \neq 0$, alors $\frac{f}{g}$ est continue en $a$.

		\item Si $f$ est continue en $a$ et $g$ en $f(a)$,
			$g \circ f$ est continue en $a$.

		\item Une fonction vectorielle est continue si et seulement si chaque composante est continue.
	\end{itemize}
\end{myprop}

\begin{myrem}
	Une fonction est toujours continue en un point isolé de $A$.
\end{myrem}

\begin{myrem}
	Les quatres opérations classiques \emph{ainsi que la composition} préservent la continuité.
\end{myrem}

%%%%%%%%%%%%%%%%%%%%%%%%%%%%%%%%%%%%%%%%%%%%%%%%%%%%%%%%%%%%%%%%%%%%%%%%%%%%%%%%%%%%%%%
%%%%%%%%%%%%%%%%%%%%%%%%%%%%%%%%%%%%%%%%%%%%%%%%%%%%%%%%%%%%%%%%%%%%%%%%%%%%%%%%%%%%%%


\section{Dérivées directionnelles et partielles}

\begin{mydef}
	Une \textbf{direction} de $\Rn$ est un vecteur $d \in \Rn$ tel que $||d|| = 1$.
\end{mydef}

\begin{mydef}[Dérivée directionnelle] Soit $f : A \subseteq \R^n \to \R^m$, $a$ un point intérieur de $A$, $d \in \R^n$ une direction. La dérivée directionnelle de $f$ en $a$ dans la direction $d$ est notée $D_df(x)$ et vaut, si elle existe, la limite suivante
	\[ D_df(a) = \lim_{t \to 0} \frac{f(a+td) - f(a)}{t} \]
\end{mydef}

\begin{myform}[Approximation linéaire dans une direction]
	\[ f(a+td) \approx f(a) + t D_df(a) \]
\end{myform}

\begin{myform}
	\begin{eqnarray*}
		D_d(g+h)(a) &=& D_dg(a) + D_d h(a)\\
		D_d(gh)(a) &=& D_dg(a)h(a) + g(a)D_dh(a)\\
		D_d(g\circ h)(a) &=& g'(h(a))D_dh(a)\\
		D_df(a) &=& (D_df_1(a), D_df_2(a), \dots, D_df_m(a))
	\end{eqnarray*}
\end{myform}

\begin{mydef}[Dérivée partielle]
	C'est la dérivée directionnelle le long d'un axe.
	Soit une fonction $f : A\subseteq \R^n \to \R^m$ et $a$ un point intérieur de $A$. La dérivée partielle de $f$ par raport à une de ses variables $x_i$ est la dérivée directionelle de $f$ au point $a$ dans la direction
	\[ e^{(i)} = (\underbrace{0; \dots; 0}_{i-1\text{ fois }0}; 1; 0; \dots; 0) \]
	et se note
	\[ \frac{\pa f}{\pa x_i}(a) = D_{e^{(i)}} f(a) \]
\end{mydef}

\begin{myrem}[Continuité et dérivées directionnelles]
	L'existence de toute les dérivées directionnelles (et donc à fortiori partielle) en un point n'implique pas la continuité.
\end{myrem}

\begin{myform}[Calcul des dérivées partielles] Pour calculer les dérivées partielles de $f$ selon une de ses variables $x_i$, on fixe toutes les autres variables et on calcule la dérivée partielle comme la dérivée d'une fonction à une seule variable ($x_i$).
\end{myform}

\begin{mydef}[Dérivées partielles d'ordre supérieur]
	L'existence des $i$\ieme{} dérivées partielles défini une fonction
	\[ \frac{\pa^i f}{\pa x^i} : \newint A \subseteq \R^n \to \R : x \to \frac{\pa^i f}{\pa x^i}(x) \]
\end{mydef}

%%%%%%%%%%%%%%%%%%%%%%%%%%%%%%%%%%%%%%%%%%%%%%%%%%%%%%%%%%%%%%%%%%%%%%%%%%%%%%%%%%%%%%%
%%%%%%%%%%%%%%%%%%%%%%%%%%%%%%%%%%%%%%%%%%%%%%%%%%%%%%%%%%%%%%%%%%%%%%%%%%%%%%%%%%%%%%

\section{Différentielle}

\subsection{Définitions}

\begin{mydef}[Différentielle]
	Soit une fonction $f : A \subseteq \R^n \to \R^m$ et $a \in \R^n$ un point intérieur de $A$. La fonction $f$ est différentiable en $a$ si et seulement si il existe une application linéaire $L : \R^n \to \R^m$ telle que
	\[ \lim_{h \to 0} \frac{f(a+h) - f(a) - L(h)}{||h||} = 0 \]

	Si cette application linéaire $L$ existe, alors on l'appelle différentielle de $f$ en $a$ on elle se note
	\[ L = \dif f_a \]
\end{mydef}

\begin{myprop}
	Soit $f : A \subseteq \R^n \to \R^m$ et $a$ un point intérieur à $A$. $f$ est différentiable si et seulement si il existe une application linéaire $L : \R^n \to \R^m$ et une fonction $r : \R^n \to \R^m$ telles que
	\[ f(a+h) = f(a) + L(h) + ||h|| r(h) \]
	et
	\[ \lim_{h \to 0}r(h) = 0 \]
\end{myprop}

\begin{myprop}[Continuité]
	La différentiabilité garanti la continuité.
\end{myprop}

\begin{myform}[Lien avec les dérivées partielles] On a les formules
	\[ \dif f_a : (\dif x_1 , \dots , \dif x_n) \to \sum_{i=1}^n \frac{\pa f}{\pa x_i}(a) \dif x_i \]
	\[ \frac{\pa f}{\pa x_i}(a) = \dif f_a(e^{(i)}) \]
\end{myform}

\begin{myform}
	Si $f$ est \textbf{différentiable} au point $a$, on a
	\[ Df_d(a) = \dif f_a(d) \]
	Si $f$ n'est \textbf{pas différentiable} au point $a$, on a \textbf{pas} spécialement
	\[ Df_d(a) = \sum_{i = 1}^n \frac{\partial f}{\partial x_i}d_i \]
	Si vous utilisez cette formule, il ne faut pas oublier de mentionner le fait que $f$ est différentiable au point $a$
	et de le montrer si ce n'est pas déjà fait.
\end{myform}

\begin{myprop}[Condition suffisante pour la différentiabilité] Une fonction $f : A \subseteq \R^n \to \R^m$ et $a$ un point intérieur à $A$. $f$ est différentiable si
	\begin{itemize}
		\item toutes les dérivées partielles de $f$ \emph{existent} en $a$ ;
		\item toutes les dérivées partielles \emph{sauf au plus une} sont continues dans un voisinage de $a$ (dans \emph{une} boule ouverte centrée en $a$)
	\end{itemize}
\end{myprop}

\subsection{Procédure de calcul de la différentielle}
\begin{enumerate}
	\item Calculer toutes les dérivées partielles de $f$ au point $a$ (règle usuelle ou définition);
		si l'une d'entre elles n'existe pas, $f$ n'est pas différentiable au point $a$
	\item Écrire une expression candidate pour la différentielle
		\[ \dif f_a(\dif x_1, \dif x_2, \dots, \dif x_n) = \sum_{i = 1}^n \frac{\partial f}{\partial x_i}(a) \dif x_i \]
	\item Vérifier que cette expression est correcte
		\begin{enumerate}
			\item Soit en prouvant séparément que la fonction est différentiable en $a$
				\begin{itemize}
					\item[$\bullet$] en montrant la continuité des dérivées partielles dans un voisinage $a$;
					\item[$\bullet$] ou en combinant des fonctions connues différentiables pour obtenir $f$.
				\end{itemize}
				\begin{itemize}
					\item Si c'est possible, l'expression candidate est bien la différentielle;
					\item mais si ce n'est pas possible, on ne peut rien en conclure.
				\end{itemize}
			\item Soit en remplaçant $L(h)$ par l'expression candidate dans la définition
				\begin{itemize}
					\item Si définition vérifiée, l'expression candidate est bien la différentielle;
					\item Si définition pas vérifiée (limite différente de 0 ou inexistante),
						on est bien sûr que $f$ n'est pas différentiable au point $a$.
				\end{itemize}
		\end{enumerate}
\end{enumerate}

\subsection{Formules de calcul}

\begin{myprop}
	Les opérations usuelles (somme, produit, composition, etc.) préservent la différentiabilité
	(à l'intérieur du domaine pertinent).

	Ceci permet de prouver la différentiabilité de la plupart des fonctions classiques,
	à l'exception des points délicats
	(dénominateur nul, fonctions définies à l'aide de cas par morceaux),
	à l'aide de combinaisons de fonctions simples dont la différentiabilité est prouvée:
	\begin{enumerate}
		\item les fonctions linéaires et affines
		\item les fonctions dérivables à une variable
	\end{enumerate}
\end{myprop}

\noindent
\begin{tabularx}{1.0\textwidth}{|S X|S X|}
	\hline
	Fonction & Différentielle (au point $a$, évaluée en $h$)\\
	\hline
	$ \displaystyle M \cdot x \;\text{avec}\; M \in \R^{m \times n} $ & $\displaystyle M \cdot h $\\
	\hline
	$ \displaystyle M \cdot x + p $ & $\displaystyle  M \cdot h $ \\
	\hline
	$ \lambda \cdot f(x) \;\text{avec}\; \lambda \in \R $ & $\displaystyle \lambda \cdot \dif f_a (h) $ \\
	\hline
	$ \displaystyle M \cdot f(x) \;\text{avec}\; M \in \R^{p \times m}$ & $\displaystyle M \cdot \dif f_a (h)$ \\
	\hline
	$ \displaystyle f(x) + g(x) $ & $ \displaystyle \dif f_a (h) + \dif g_a (h) $ \\
	\hline
	$ \displaystyle f(x)\cdot g(x) $ & $\displaystyle  g(a) \cdot \dif f_a(h) + f(a) \cdot \dif g_a(h) $ \\
	\hline
	$ \displaystyle \frac{f(x)}{g(x)} $ & $\displaystyle  \frac{g(a) \cdot \dif f_a(h) - f(a) \cdot \dif g_a(h) }{g(a)^2} $ \\
	\hline
	$ \displaystyle (f(x)|g(x)) $ & $\displaystyle  (\dif f_a(h) | g(a) ) + (f(a)|\dif g_a(h)) $ \\
	\hline
	$ \displaystyle (f \circ g)(x) $ & $\displaystyle  \dif f_{g(a)} \circ \dif g_a(h) $ \\
	\hline
\end{tabularx}

%%%%%%%%%%%%%%%%%%%%%%%%%%%%%%%%%%%%%%%%%%%%%%%%%%%%%%%%%%%%%%%%%%%%%%%%%%%%%%%%%%%%%%%
%%%%%%%%%%%%%%%%%%%%%%%%%%%%%%%%%%%%%%%%%%%%%%%%%%%%%%%%%%%%%%%%%%%%%%%%%%%%%%%%%%%%%%

\section{Plans et vecteurs tangents, gradient}

\subsection{Courbes et surfaces paramétrées}

\begin{mydef}[Courbe paramétrée]
	Soit une courbe paramétrée dans $\R^m$ décrite par $f : A \subseteq \R \to \R^m$
	\[ \mathcal{C} = \{ f(t) | t \in A \subseteq \R \} \subseteq \R^m \]
\end{mydef}

\begin{myform}[Droite tangente]
	Si $f$ est différentiable, alors la droite tangente à la courbe $\mathcal{C}$ en $f(a)$ a pour équation paramétrique
	\[ t \to f(a) + t f'(a) \]
	Où $f'(a)$ est le vecteur tangent à $\mathcal{C}$ en $a$
	Dans ce cas,
	\[ f(a + t) \approx f(a) + \dif f_a(t) = f(a) + t f'(a) \]
\end{myform}

\begin{mydef}[Surface paramétrée]
	Une surface paramétrée de $\R^m$ décrite par $f : A \subseteq \R^2 \to \R^m$ est
	\[ \mathcal{S} = \{ f(u,v) | (u,v) \in A \subseteq \R^2 \} \subseteq \R^m \]
\end{mydef}

\begin{myform}[Plan tangent]
	Si $f$ est différentiable en $a$, le plan tangent à $f$ au point $f(a,b)$ est paramétrée par
	\[ (u,v) \to f(a,b) + u \frac{\pa f}{\pa x_1} (a,b) + v \frac{\pa f}{\pa x_2}(a,b) \]
\end{myform}

\subsection{Gradient}

\begin{mydef}[Gradient] Soit $f : A \subseteq \R^n \to \R$ une fonction scalaire différentiable en $a$, le gradient est un vecteur ligne à $n$ composantes
	\[ (\newgrad f)(a) = \nabla f(a) = \left( \frac{\pa f}{\pa x_1}(a) , \dots, \frac{\pa f}{\pa x_n}(a) \right) \]
\end{mydef}

\begin{myform} On a que
	\[ D_df(a) = \dif f_a(d) = \nabla f(a) d \]
\end{myform}

\begin{myprop} Le gradient est parallèle à la direction de plus forte pente,
	celle où $D_df(a)$ est maximal
	\footnote{\c{C}a peut se prouver aisément en majorant $D_df(a)$ par l'inégalité de Cauchy et en remarquant qu'on doit être dans le cas d'égalité pour être maximum.}.
	Il est aussi orthogonal à tout vecteur tangent à la courbe de niveau de $f$ passant par le point $a$.
\end{myprop}

\subsection{Courbes et surfaces définies par un graphe}

\begin{myform}[Plan tangent à un graphe]
	Soit une fonction scalaire $f : A \subseteq \R^n \to \R$ différentiable en $a$ un point intérieur de $A$. Le plan tangent $z$ au point $(a,f(a))$ est
	\begin{align*}
		z & = f(a) + \frac{\pa f}{\pa x_1}(a)(x_1 - a_1) + \cdots + \frac{\pa f}{\pa x_n}(a) (x_n - a_n) \\
		& = f(a) + \dif f_a(x-a) \\
		& = f(a) + \nabla f(a)(x-a)
	\end{align*}

	La normale de ce plan tangent est
	\[
	\begin{pmatrix}
		\nabla f(a) & -1
	\end{pmatrix}
	\]
\end{myform}

%%%%%%%%%%%%%%%%%%%%%%%%%%%%%%%%%%%%%%%%%%%%%%%%%%%%%%%%%%%%%%%%%%%%%%%%%%%%%%%%%%%%%%%
%%%%%%%%%%%%%%%%%%%%%%%%%%%%%%%%%%%%%%%%%%%%%%%%%%%%%%%%%%%%%%%%%%%%%%%%%%%%%%%%%%%%%%

\section{Théorème des accroissement finis}

\begin{mytheo}[Théorème des accroissements finis] Soit une fonction scalaire $f : A \subseteq \R^n \to \R$ continue sur l'intervalle fermé $[a,b]$ et différentiable sur l'intervalle $]a,b[$, $\exists c \in ]a,b[$ tel que
	\[ f(b) - f(a) = \dif f_c(b-a) \]
\end{mytheo}

\begin{mytheo}[Théorème des accroissements finis (formulation équivalente)] Soit $f$ une fonction scalaire. $\exists 0 < \theta < 1$ tel que
	\[ f(a+h) = f(a) + \dif f_{a + \theta h}(h) = f(a) + \nabla f(a + \theta h)h \]
\end{mytheo}

\begin{myprop}[Pour les fonctions vectorielles]
	On a pas le théorème des accroissements finis, mais on a que
	$\exists c \in ]a; b[$ tel que
	\[ || f(b) - f(a) || \leq || \dif f_c(b-a) || \]
\end{myprop}

%%%%%%%%%%%%%%%%%%%%%%%%%%%%%%%%%%%%%%%%%%%%%%%%%%%%%%%%%%%%%%%%%%%%%%%%%%%%%%%%%%%%%%%
%%%%%%%%%%%%%%%%%%%%%%%%%%%%%%%%%%%%%%%%%%%%%%%%%%%%%%%%%%%%%%%%%%%%%%%%%%%%%%%%%%%%%%

\section{Classe de fonction}

\begin{mydef}[Classe de fonctions $\mathcal{C}^k$]
	Une fonction $f$ est dite de classe $\mathcal{C}^k$ sur un ensemble ouvert $A$ si et seulement si toutes ses dérivées partielles $k$\ieme{} existent et sont continues sur $A$. On écrit alors
	\[ f \in \mathcal{C}^k(A) \]
\end{mydef}

\begin{myprop}[Classe de fonction] On a les inclusions suivantes
	\[ \mathcal{C}^{\infty}(A) \subset \dots \subset \mathcal{C}^3(A) \subset \mathcal{C}^2(A) \subset \mathcal{C}^1(A) \subset \mathcal{C}^0(A) \]
\end{myprop}

%%%%%%%%%%%%%%%%%%%%%%%%%%%%%%%%%%%%%%%%%%%%%%%%%%%%%%%%%%%%%%%%%%%%%%%%%%%%%%%%%%%%%%%
%%%%%%%%%%%%%%%%%%%%%%%%%%%%%%%%%%%%%%%%%%%%%%%%%%%%%%%%%%%%%%%%%%%%%%%%%%%%%%%%%%%%%%

\section{Matrices hessiennes et jacobiennes}

\begin{mydef}[Matrice hessienne]
	Soit $f : A \subseteq \R^n \to \R$ et $a$ un point intérieur de $A$. Si toutes les dérivées partielles secondes existent en $a$, la matrice hessienne est la matrice carrée de dimension $n$ suivante
	\[ Hf(a) = \begin{pmatrix}
		\frac{\pa^2 f}{\pa x_1^2}(a) & \frac{\pa^2 f}{\pa x_2 x_1}(a) & \cdots & \frac{\pa^2 f}{\pa x_n \pa x_1}(a) \\
		\frac{\pa^2 f}{\pa x_1 \pa x_2}(a) & \frac{\pa^2 f}{\pa x_2^2}(a) & \cdots & \frac{\pa^2 f}{\pa x_n \pa x_2}(a) \\
		\vdots & \vdots & \ddots & \vdots \\
		\frac{\pa^2 f}{\pa x_1 \pa x_n}(a) & \frac{\pa^2 f}{\pa x_2 \pa x_n}(a) & \cdots & \frac{\pa^2 f}{\pa x_n^2}(a) \\
	\end{pmatrix} \]
\end{mydef}

\begin{myprop}[Matrice symétrique]
	Soit $f : A \subseteq \R^n \to \R$.
	Si $f \in \mathcal{C}^2(A)$, alors
	\[ \frac{\pa^2 f}{\pa x_j \pa x_i}(a) = \frac{\pa^2 f}{\pa x_i \pa x_j}(a) \]
	pour tout $a \in A$ et pour tous $1 \leq i,j \leq n$. Dans ce cas, la matrice hessienne est symétrique.
\end{myprop}

\begin{myprop}[Ordres supérieurs]
	Si une fonction est de classe $\mathcal{C}^k$ alors on peut permuter l'ordre de dérivation des variables pour les dérivées partielles d'ordre $k$.
\end{myprop}

\begin{mydef}[Matrice jacobienne]
	Soit une fonction $f : A \subseteq \R^n \to \R^m$ différentiable en un point $a$ intérieur à $A$. La matrice jacobienne est une matrice $m \times n$ :
	\[ Jf(a) =
	\begin{pmatrix}
		\frac{\pa f_1}{\pa x_1}(a) & \frac{\pa f_1}{\pa x_2}(a) & \cdots & \frac{\pa f_1}{\pa x_n}(a) \\
		\frac{\pa f_2}{\pa x_1}(a) & \frac{\pa f_2}{\pa x_2}(a) & \cdots & \frac{\pa f_2}{\pa x_n}(a) \\
		\vdots & \vdots & \ddots & \vdots \\
		\frac{\pa f_m}{\pa x_1}(a) & \frac{\pa f_m}{\pa x_2}(a) & \cdots & \frac{\pa f_m}{\pa x_n}(a) \\
	\end{pmatrix} \text{avec } f =
	\begin{pmatrix}
		f_1 \\ f_2 \\ \vdots \\ f_m
	\end{pmatrix} \]
\end{mydef}

\begin{myform}[Lien avec gradient, dérivées partielles, ...]
	On a les formules suivantes :
	\[ Jf(a) =
	\begin{pmatrix} \nabla f_1(a) \\ \nabla f_2(a) \\ \vdots \\ \nabla f_m(a)
	\end{pmatrix} \]
	\[ Jf(a) =
	\begin{pmatrix} \frac{\pa f}{\pa x_1}(a) & \frac{\pa f}{\pa x_2}(a) & \cdots & \frac{\pa f}{\pa x_n}(a)
	\end{pmatrix} \]
	Pour chaque composante scalaire d'une fonction vectorielle
	\[ \dif f_a(h) = Jf(a) h \]
	La matrice jacobienne est donc la matrice associée à l'application linéaire $\dif f_a$.
	Pour une fonction à une seule variable représentant une courbe paramétrée de $\R^n$ :
	\[ Jf(a) = f'(a) \]
\end{myform}

%%%%%%%%%%%%%%%%%%%%%%%%%%%%%%%%%%%%%%%%%%%%%%%%%%%%%%%%%%%%%%%%%%%%%%%%%%%%%%%%%%%%%%%
%%%%%%%%%%%%%%%%%%%%%%%%%%%%%%%%%%%%%%%%%%%%%%%%%%%%%%%%%%%%%%%%%%%%%%%%%%%%%%%%%%%%%%

\section{Développement de Taylor}

\begin{myform}[Développement avec reste d'ordre 1]
	Soit $A \subseteq \R^n$ un ensemble ouvert, $f : A \subseteq \R^n \to \R$ une fonction scalaire de classe $\mathcal{C}^1$, $a \in \R^n$ un point de $A$ et $h$ un vecteur de $\R^n$ tel que $[a, a+h] \subseteq A$. $\exists 0 < \theta < 1$ tel que
	\begin{align*}
		f(a+h) & = f(a) + \nabla f(a+\theta h)h \\
		& = f(a) + \sum_{i =1}^n \frac{\pa f}{\pa x_i} (a+\theta h)h_i
	\end{align*}
	C'est le théorème des accroissement finis.
\end{myform}

\begin{myform}[Développement avec reste d'ordre 2]
	Soit $f$ une fonction de classe $\mathcal{C}^2$.
	$\exists 0 < \theta < 1$ tel que
	\begin{align*} f(a+h) & = f(a) + \nabla f(a) h + \frac12 h^T Hf(a+\theta h) h \\
		& = f(a) + \sum_{i=1}^n \frac{\pa f}{\pa x_i}(a)h_i + \frac12 \sum_{i=1}^n \sum_{j=1}^n \frac{\pa^2 f}{\pa x_i \pa x_j} (a+ \theta h) h_i h_j
	\end{align*}
\end{myform}

\begin{myform}[Développement avec reste d'ordre 3]
	Soit $f$ une fonction de classe $\mathcal{C}^3$.
	$\exists 0 < \theta < 1$ tel que
	\begin{eqnarray*}
		f(a+h) &=& f(a) + \sum_{i=1}^n \frac{\pa f}{\pa x_i}(a)h_i
		+ \frac 12 \sum_{i=1}^n \sum_{j=1}^n \frac{\pa^2 f}{\pa x_i \pa x_j} (a) h_i h_j \\
		&& + \frac{1}{3!} \sum_{i=1}^n \sum_{j=1}^n \sum_{k=1}^n \frac{\pa^3 f}{\pa x_i \pa x_j \pa x_k}(a+\theta h)h_i h_j h_k
	\end{eqnarray*}
	Dans le cas d'une fonction $f$ à 2 variables $x, y$, le reste du développement de Taylor s'écrit
	\[
	\frac{1}{6}
	\left(
	\frac{\pa^3 f}{\pa x^3}(c)h_1^3
	+ 3\frac{\pa^3 f}{\pa x^2\pa y}(c)h_1^2h_2\\
	+ 3\frac{\pa^3 f}{\pa x\pa y^3}(c)h_1h_2^2
	+ \frac{\pa^3 f}{\pa y^3}(c)h_2^3
	\right)
	\]
	où $a + \theta h$ a été remplacé par $c$ pour plus de concision.
\end{myform}

%%%%%%%%%%%%%%%%%%%%%%%%%%%%%%%%%%%%%%%%%%%%%%%%%%%%%%%%%%%%%%%%%%%%%%%%%%%%%%%%%%%%%%%
%%%%%%%%%%%%%%%%%%%%%%%%%%%%%%%%%%%%%%%%%%%%%%%%%%%%%%%%%%%%%%%%%%%%%%%%%%%%%%%%%%%%%%

\section{Composition (\emph{chain rule})}

\begin{myform}[\emph{Chain rule}]
	On peut écrire la matrice jacobienne d'une composée comme un produit matriciel:
	\[ J(g \circ f)(a) = Jg(f(a)) \cdot Jf(a) \]
	Si on explicite les composantes $f_1, f_2, \dots , f_m$ de $f$ avec
	\[ g(f(x)) = g( f_1(x_1, \dots, x_n), f_2(x_1, \dots, x_n), \dots, f_m(x_1, \dots, x_n)) \]
	et $g$ définie de sorte qu'on ait $g : (y_1, y_2, \dots , y_m) \rightarrow g(y_1, y_2, \dots, y_m)$. Donc $x$ sont les variables naturelles de $f$ et $y$ celles de $g$.
	Alors on a la formule\footnote{Notez la différence par rapport aux slides du cours, où les $y_1, y_2, \dots, y_n$ remplacent les $x_1, x_2, \dots, x_n$ pour ne pas les confondre avec les variables de $f$}
	\[ \frac{\pa (g \circ f)}{\pa x_i}(a) = \frac{ \pa g}{\pa y_1}(f(a)) \frac{\pa f_1}{\pa x_i}(a) + \frac{ \pa g}{\pa y_2}(f(a)) \frac{\pa f_2}{\pa x_i}(a) + \cdots + \frac{ \pa g}{\pa y_m}(f(a)) \frac{\pa f_m}{\pa x_i}(a) \]
\end{myform}

%%%%%%%%%%%%%%%%%%%%%%%%%%%%%%%%%%%%%%%%%%%%%%%%%%%%%%%%%%%%%%%%%%%%%%%%%%%%%%%%%%%%%%%
%%%%%%%%%%%%%%%%%%%%%%%%%%%%%%%%%%%%%%%%%%%%%%%%%%%%%%%%%%%%%%%%%%%%%%%%%%%%%%%%%%%%%%

\section{Optimisation}

On ne parlera ici que de fonction scalaire car il n'y a pas d'ordre dans $\Rn$ avec $n > 1$.
On ne peut pas dire $a < b$ si $a, b \in \Rn$, ce n'est pas défini.

\subsection{Définitions}

\begin{mydef}[Extrémum global] Un extremum global est un point où $f$ prend une valeur extrême ; c'est à dire un point $a$ tel que
	\[ f(a) \begin{array}{l} \geq \\ \leq \end{array} f(x) \; \forall x \in A \begin{array}{l} \text{(maximum)} \\ \text{(minimum)}
	\end{array} \]
\end{mydef}

\begin{mydef}[Extrémum local] Un extremum local est un point où $f$ prend une valeur extrême sur un voisinage; c'est à dire un point $a$ tel que
	\[ \exists r >0 \; \text{tel que} \; f(a) \begin{array}{l} \geq \\ \leq \end{array} f(x) \; \forall x \in B(a;r) \cap A \begin{array}{l} \text{(maximum)} \\ \text{(minimum)}
	\end{array} \]
\end{mydef}

\begin{mydef}[Extrémum local libre]
	Un extremum local libre de $f$ est un extremum local $a \in \newint A$.
\end{mydef}

\begin{mydef}[Extrémum local lié]
	Un extremum local lié de $f$ est un extremum local $a \in \newfr A \cap A$.
\end{mydef}

\begin{mydef}[Point selle]
	Un point selle (parfois aussi appelé un ``col'') est un point $a \in \newint A$ où $\forall r > 0, \exists x_+ \in B(a;r) \; \text{et} \; x_- \in B(a;r)$ tels que
	\[ f(x_-) < f(a) < f(x_+) \]
\end{mydef}

\begin{mydef}[Point critique]
	Un point critique est un point $a \in \newint A$ où le gradient existe et est nul.
\end{mydef}

\begin{mydef}[Point singulier]
	Un point singulier est un point du domaine où le gradient n'existe pas.
\end{mydef}

\begin{mydef}[Ensemble compact]
	Un  ensemble est compact s'il est \textbf{fermé} et \textbf{borné},
	c'est à dire qu'il ne ``tend pas vers l'infini'' et qu'il contient tous ses points frontières.
\end{mydef}

\subsection{Conditions pour trouver des extrema}

\begin{myprop}[Propriété de Fermat]
	Soit $a \in \newint A$, extrémum local libre. Soit $f$ une direction de $\R^n$ ; si $D_df(a)$ existe, elle est nulle. Donc, $\frac{\pa f}{\pa x_i}(a) = 0$, $\nabla f(a) = 0$ et $\dif f_a = 0$ (s'ils existent).
\end{myprop}
Cette propriété nous permet d'énoncer une condition nécessaire
\begin{myprop}[Condition nécessaire]
	Soit $f \in \mathcal{C}^2$, $a \in \newint A$. Pour que $a$ soit un extremum, il est nécessaire que
	\[ \nabla f(a) = 0 \]
\end{myprop}

\begin{myrem}
	Cette condition n'est pas applicable aux points frontières où le calcul différentiel n'apporte pas d'informations.
\end{myrem}

\begin{myprop}[Conditions nécessaires à la détermination des minima et maxima]
	Soit $a$ un point critique. Pour que $a$ soit un minimum (resp. un maximum),
    il faut que $Hf(a)$ soit semi-définie positive (resp. négative).
\end{myprop}

\begin{myprop}[Conditions suffisantes à la déterminations des minima, maxima et points de selle]
	Si la hessienne est (à un point critique)
	\begin{itemize}
		\item définie positive, ceci garanti un minimum ;
		\item définie négative, ceci garanti un maximum ;
		\item indéfinie, ceci garanti un point de selle.
	\end{itemize}
\end{myprop}

\begin{myprop}[Point singulier]
	Dans le cas d'un point singulier, la détermination doit se faire à la main. Il n'y a aucun outil particulier.
\end{myprop}

\begin{mytheo}[Bornes atteintes]
	Si $f$ est \textbf{continue} et que $A$ est \textbf{compact}, alors $f(A)$ est \textbf{compact}. Dans ce cas, $f$ admet un maximum et un minimum sur le domaine $A$.
\end{mytheo}

\subsection{Optimisation sous contrainte}

Soit $f : A \subseteq \R^n \to \R$ une fonction scalaire et $g : A \subseteq \R^n \to \R$ les contraintes ($g(x) = 0$). On défini l'ensemble admissible

\begin{mydef}[Ensemble admissible] L'ensemble admissible est
	\[ \Phi = \{ x \in A | g(x) = 0 \} \]
\end{mydef}

\begin{mydef}[Lagrangien] Le Lagrangien est la fonction
	\[ \mathcal{L} : A \times \R \to \R : (x, \lambda) \to L(x, \lambda) = f(x) - \lambda g(x) \]
\end{mydef}

\begin{myprop}[Condition nécessaire]
	Si $a$ est un extremum local de $f$ sur $\Phi$ et si
	\begin{itemize}
		\item $\exists r >0$ tel que $f, g$ sont différentiables
			sur $B(a;r)$ ;
		\item les dérivées partielles de $f$ et $g$ existent et sont continues en $a$ ;
		\item $\nabla g(a) \neq 0$ ;
	\end{itemize}
	alors $\exists \lambda$ tel que $(a,\lambda)$ est un point critique de $\mathcal{L}(x,\lambda)$, c'est à dire
	\[ \nabla \mathcal{L}(a,\lambda) = (\nabla f(a) - \lambda\nabla g(a), g(a)) = 0 .\]
\end{myprop}

\begin{myrem}[Hessienne et Lagrangien]
	Attention on ne peut absolument pas utiliser la hessienne pour conclure à propos des optimisations sous contraintes résolues avec l'aide du Lagrangien.
\end{myrem}

\begin{myprop}[Contraintes multiples]
	Dans ce cas, on applique plusieurs fois le Lagrangien d'affilé :
	\[ \mathcal{L} : A \times \R^m \to \R : (x, \lambda) \to L(x, \lambda) = f(x) - \lambda_1 g_1(x) - \lambda_2 g_2(x) - \cdots - \lambda_m g_m(x) \]
	Dans ce cas, la condition nécessaire pour être un extremum n'est plus
	que le gradient (de la contrainte) soit non-nul, mais que
	\[ \newrang Jg(a) = m \]
	c'est à dire que la Jacobienne des contraintes doit être de rang plein (ou, intuitivement, que les contraintes doivent être indépendantes\footnote{\emph{cf.} cours de mécanique avec la technique des multiplicateurs de Lagrange}).
\end{myprop}

\begin{myrem}
	Les multiplicateurs de Lagrange nous donne une condition nécessaire, pas suffisante.
\end{myrem}

\subsection{Méthode}

\subsubsection{Calcul d'extrema locaux libres}
Un \textbf{extrema local libre} est un extrema local à l'intérieur du domaine.
C'est soit un point \textbf{critique}, soit un point \textbf{singulier}.
\begin{itemize}
	\item Si le \textbf{gradient} n'existe pas, c'est un point \textbf{singulier}.
		Il faut élaborer un raisonnement \emph{ad hoc} pour déterminer si c'est un extrema local libre ou pas.
	\item Si le \textbf{gradient} existe, il est nécessaire qu'il soit nul.
		C'est alors un point \textbf{critique}.
		C'est donc soit un point \textbf{selle}, soit un \textbf{extrema local}.
		On calcule alors la \textbf{hessienne} en ce point.
		\begin{itemize}
			\item Si elle est indéfinie, on a un point \textbf{selle}.
			\item Si elle est définie positive, on a un \textbf{minima local}.
			\item Si elle est définie négative, on a un \textbf{maxima local}.
			\item Si elle est semi-définie positive, on a un point \textbf{selle} ou un \textbf{minima local}.
			\item Si elle est semi-définie négative, on a un point \textbf{selle} ou un \textbf{maxima local}.
		\end{itemize}
		Si on est dans un des deux derniers cas, on peut utiliser \textbf{Taylor} d'ordre supérieur, ou tenter un raisonnement \emph{ad hoc} pour déterminer si c'est un point \textbf{selle} ou un \textbf{extrema local}.
\end{itemize}

\subsubsection{Calcul d'extrema locaux liés}
Un \textbf{extrema local lié} est un extrema local à l'intersection entre la frontière du domaine et le domaine.
C'est
\begin{itemize}
	\item soit un point qui ne répond pas aux conditions de différentiabilité/continuité;
	\item soit un point qui ne répond pas à la condition d'indépendance linéaire des gradients des $g_i$;
	\item soit un point situé sur la frontière du domaine de $f$ ou de $g$;
	\item soit un point \textbf{critique} du Lagrangien.
\end{itemize}
Soit vous décidez de re-paramétrisez la frontière puis de calculer ses extrema.
Soit vous analysez cas par cas
\begin{itemize}
	\item Dans les deux premiers cas, vérifiez qu'ils respectent bien les contraintes
		\footnote{Par exemple, si $g(x, y) = x^2 + y^2 - 1$, $\nabla g = 0$ lorsque $(x, y) = (0, 0)$ mais $g(0, 0) \neq 0$ donc ce n'est pas un extremum local lié.}
		et qu'ils appartient au domaine de $f$.
		Sinon, ce ne sont pas pas des extrema locaux lié.
		Dans le cas favorable, il nous faut un raisonnement \emph{ad hoc} pour déterminer si ce sont des exterma locaux ou pas.
	\item Dans le troisième cas, il faut les analyser en particulier en rajoutant une contrainte ou en les paramétrisant.
	\item Dans le dernier cas, vous imposez que $\exists \lambda \in \R^m$ tel que $\nabla \mathcal{L}(a, \lambda) = 0$.
		C'est une condition nécessaire donc ça vous limitera le nombre d'extrema locaux potentiels mais il vous faudra un raisonnement \emph{ad hoc} pour trancher si ces derniers sont des extrema locaux ou pas.
\end{itemize}

\subsubsection{Calcul d'extrema globaux}
L'existence d'un extrema global peut nous être garantie par le théorème des \textbf{bornes atteintes} ($f$ continue, $A$ compacte).
Un extremum global est soit un extremum \textbf{local libre}, soit un extremum \textbf{local lié} (sur la frontière du domaine).

Il suffit d'\textbf{énumérer} tous ces points et \textbf{sélectionner} les valeurs extrêmes
\textbf{mais} il est également possible qu'il n'\textbf{existe pas} d'extremum global.

\subsubsection{Trouver un raisonnement \emph{ad hoc}}
On vous demande souvent de trouver un raisonnement \emph{ad hoc}, en voici quelques un qui peuvent en constituer un dans la plupart des cas.

Les raisonnements suivant sont appliqués pour les minima, ils s'appliquent bien entendu aussi pour les maxima de manière légèrement adaptée.

\paragraph{Énoncé}
Si pour certains points, vous n'arrivez toujours pas à décider si c'est un minima local ou pas,
vérifiez que vous avez vraiment besoin de le savoir.
Comme on va voir plus loin, s'il ne faut que savoir le minimum global, ce n'est pas toujours nécessaire.

\paragraph{Voisinage}
Si dans un voisinage du point, vous remarquez que tous les points sont plus grand, alors c'est un minimum local.
Si vous prouvez que pour tout voisinage, il y en a un plus grand strictement et un plus petit strictement qui sont dans le domaine,
alors c'est un point selle.

\paragraph{Par les minima globaux}
Si vous avez \textbf{montré} qu'il \textbf{existait} un \textbf{minimum global}
et qu'il vous reste des points dont vous n'êtes pas sur si ce sont des minima locaux ou pas.
Comparez les quand même avec les minima locaux pour lesquels vous l'avez montré.

Si le plus petit de ces points est un point pour lequel vous ne saviez pas s'il était un minimum local ou pas,
alors vous avez la certitude que c'était un minimum local car c'est le minimum global.
En effet, vous savez qu'il y en a un
et il est le plus petit de tous les points satisfaisant des conditions nécessaires pour être un minimum global.

Pour les autres minima locaux potentiels, vous ne pouvez pas en conclure que ce ne sont pas des minima locaux.
Par contre, vous pouvez restreindre le domaine à un ensemble compacte.
Si vous vous débrouillez pour que $f$ soit continu dans cet restriction, par le théorème des \textbf{bornes atteintes},
elle aura un minimum global.
Ca rajoutera probablement des frontières, donc des minima locaux liés supplémentaire propres à cette restriction.
Rajoutez ces derniers aux candidats et retirez ceux qui étaient dans le domaine mais qui ne sont pas dans la restriction.
Si vous trouvez comme minimum global pour cettre restriction un point intérieur,
vous pouvez en conclure que c'est un minimum local libre.

%%%%%%%%%%%%%%%%%%%%%%%%%%%%%%%%%%%%%%%%%%%%%%%%%%%%%%%%%%%%%%%%%%%%%%%%%%%%%%%%%%%%%%%
%%%%%%%%%%%%%%%%%%%%%%%%%%%%%%%%%%%%%%%%%%%%%%%%%%%%%%%%%%%%%%%%%%%%%%%%%%%%%%%%%%%%%%

\part{Calcul intégral et intégrales multiples}

\section{Généralités}

\begin{mydef}[Fonction intégrable] $f$ est intégrable sur le rectangle $D$ et a pour intégrale double la valeur $I$
	\[ I = \iint_D f(x,y) \dif A \]
	si et seulement si $\forall \epsilon > 0, \exists \delta > 0$ tel que pour toute partition $P \subset D$ tel que $||P|| < \delta$ et pour tous choix des points $(x_{ij}^{*},y_{ij}^{*})$ dans $R_{ij} \subset P$ on a
	\[ |R(f,C,P) - I| < \epsilon \]
\end{mydef}

\begin{myrem}
	\[ \dif A = \dif x \dif y = \dif y \dif x \]
\end{myrem}

\begin{mytheo}[Fonction intégrable et continue]
	Si $f$ est continue sur $D$ alors $f$ est intégrable sur $D$
\end{mytheo}

\begin{myprop} $f$ est intégrable sur un domaine quelconque $D$ si et seulement si $\hat{f}$ est intégrable sur un rectangle $R$ avec
	\[ \hat{f} : (x,y) \to \hat{f}(x,y) = \left\{
	\begin{array}{lll} f(x,y) & \text{si} & (x,y) \in D \\ 0 & \text{si} & (x,y) \notin D  \; \text{et} \; (x,y) \in R
	\end{array}
	\right.
	\]
	et dans ce cas
	\[ \iint_D f(x,y) \dif A = \iint_R \hat{f}(x,y) \dif A \]
\end{myprop}

\begin{myrem}
	Ces propriétés sont <<~extensibles~>> à $n$ dimensions.
\end{myrem}

\begin{myform}[Dépendance linéaire de l'intégrant]
	$\forall L, M \in \R$ on a
	\[ \iint_D (L f(x,y) + M g(x,y)) \dif A = L \iint_D f(x,y) \dif A + M \iint_D g(x,y) \dif A \]
\end{myform}

\begin{myform}[Autres formules] On a, si $f \leq g$ sur $D$ :
	\[ \iint_D f(x,y) \dif A \leq \iint_D g(x,y) \dif A \]

	Si $D = D_1 \cup D_2 \cup \dots \cup D_k$ avec $D_i \cap D_j = \varnothing \; \forall i \neq j$, si $f$ est intégrale sur $D_j$ $\forall j = 1, \dots, k$ alors
	\[ \iint_D f(x,y) \dif A = \sum_{j=1}^k \iint_{D_j} f(x,y) \dif A \]

	On a également que
	\[ \left| \iint_D f(x,y) \dif A \right| \leq \iint \left| f(x,y) \right| \dif A \]
\end{myform}

\begin{mytheo}[Théorème de Fubini] Soit $f$ définie sur un pavé $D = [a_1,b_1] \times [a_2,b_2] \times \cdots \times [a_n,b_n] \subset \R^n$, $f$ continue sur $D$. Alors pour toute permutation $\sigma$ de la suite des nombres entiers $(1,2,\dots, n)$, on a
	\[ \int_D f = \int_{\sigma(1)}^{\sigma(1)} \left[ \int_{\sigma(2)}^{\sigma(2)}   \left[ \dots \left[ \int_{\sigma(n)}^{\sigma(n)} f(x_1,x_2,\dots,x_m) \dif x_{\sigma(n)} \right]   \dots \right] \dif x_{\sigma(2)} \right] \dif x_{\sigma(1)} \]
\end{mytheo}

\section{Méthodes de calcul}

\subsection{Par inspection} Si la fonction présente une symétrie, soit sur elle même soit sur son domaine, on peut travailler par <<~morceaux~>>.

\subsection{Par itération}

\begin{mydef}[Domaine x ou y-simple]
	Le domaine d'une fonction est
	\begin{itemize}
		\item y-simple si $x \in [a;b]$ et $y \in [c(x);d(x)]$ ;
		\item x-simple si $y \in [c;d]$ et $x \in [a(y);b(y)]$.
	\end{itemize}
\end{mydef}

\begin{myprop} Si $f$ est définie sur un domaine x-simple ou y-simple, alors on peut calculer l'intégrale comme ceci : soit $D$ un domaine y-simple,
	\begin{align*} \iint_D f(x,y)\dif A & = \int_a^b \left[ \int_{c(x)}^{d(x)} f(x,y) \dif y \right] \dif x \\
		& = \int_a^b \dif x \int_{c(x)}^{d(x)} f(x,y) \dif y
	\end{align*}
	De la même manière, si le domaine $D$ est x-simple, on a
	\[ \iint_D f(x,y)\dif A  = \int_c^d \dif y \int_{a(y)}^{b(y)} f(x,y) \dif x \]
\end{myprop}

\subsection{Par changement de variable}
Soit une intégrale
\[ \iint_D f(x,y) \dif A \]
à calculer. On va faire effectuer une transformation \emph{bijective} (ou \emph{mapping}) et ce en deux étapes.
\begin{enumerate}
	\item on pose
		\[ f(x,y) = f(x(u,v),y(u,v)) = g(u,v) \]
	\item on exprime ensuite $\dif A$ en fonction de $\dif u$ et $\dif v$. Pour ce faire, on a la formule
		\[ \dif A = |\newdet(J)| \dif u \dif v \]
		avec $J$ la jacobienne suivante :
		\[ J =
		\begin{pmatrix}
			\frac{\pa x}{\pa u} & \frac{\pa x}{\pa v} \\ \frac{\pa y}{\pa u} & \frac{\pa y}{\pa v}
		\end{pmatrix}
		\overset{\Delta}{=} \frac{\pa(x,y)}{\pa (u,v)} \]
\end{enumerate}

\section{Intégrale de lignes et de surfaces}

\subsection{Définitions et théorèmes généraux}

\begin{mydef}[Champ de vecteurs conservatifs] $\vv{F}$ est un champ de vecteurs conservatifs si et seulement si il existe $\Phi$, une fonction scalaire, telle que
	\[ \vv{F} = \nabla \Phi \]
	On dit alors que $\Phi$ est un potentiel scalaire et que $F$ dérive d'un potentiel.
	Dans ce cas, on a que, si $C$ est une courbe dont les extrémités sont $P_1$ et $P_2$,
	\[ \int_C F \cdot \dif r = \Phi(P_2) - \Phi(P_1) \]
	De plus il y a indépendance des chemins choisis.
\end{mydef}

\begin{myprop}[Champ de vecteurs conservatifs]
	Si $P_1 = P_2$, alors
	\[ \oint F \cdot \dif r = 0 \]

	Si $C = C_1 - C_2$, alors
	\[ \int_C F \cdot \dif r = 0 \Leftrightarrow \int_{C_1} F \cdot \dif r - \int_{C_2} F \cdot \dif r = 0
	\Leftrightarrow \int_{C_1} F \cdot \dif r = \int_{C_2} F \cdot \dif r \]
\end{myprop}

\begin{mytheo} Soit $D$ ouvert et connexe de $\R^3$ et $\vv{F}$ un champ vectoriel défini sur $D$. Il y a équivalence des trois énoncés suivants :
	\begin{itemize}
		\item $\vv{F}$ est conservatif sur $D$, $\exists \Phi$ tel que $\nabla \Phi = \vv{F}$ ;
		\item $\oint_C \vv{F} \cdot \dif r = 0$, pour toute courbe fermée dans $D$ ;
		\item pour tout points $P_0$, $P_1$ dans $D$, $\int_C F \dif r$ a la même valeur quelque soit le chemin $C$ entre $P_0$ et $P_1$.
	\end{itemize}
\end{mytheo}

\begin{mydef}[Ensemble simplement connexe] C'est un ensemble dans lequel toute courbe fermée qui ne s'intersecte pas peut être transformée de manière continue en un seul point de $D$ sans quitter $D$ (par exemple, $\R^2$ est simplement connexe, mais $\R^2 \setminus \{(0;0)\}$ pas).
\end{mydef}

\begin{mytheo}[Théorème de Poincaré] Soit $D$ \emph{simplement} connexe, $F$ un champ vectoriel défini sur $D$, alors on a une 4\ieme{} équivalence aux trois précédentes
	\[ \frac{\pa F_i}{\pa x_j} = \frac{\pa F_j}{\pa x_i} \]
	pour $i,j = 1,2,3$.
\end{mytheo}

\begin{mydef}[Surface orientée]
	Une surface orientée est une surface à laquelle on définit un côté positif et un côté négatif.
	Pour ce faire, on définit la normale qui pointe, perpendiculairement à la surface, vers le côté positif.
	Le sens de parcourt du bord se fait soit en appliquant la règle de la main droite avec la normale,
	soit avec la convention suivante:
	Lorsque on regarde la surface du côté positif, si on marche le long du parcourt, la surface est à notre gauche.
\end{mydef}

\subsection{Intégrale de ligne d'un champ scalaire}

\begin{mytheo}[Existence de l'intégrale de ligne]
	Soit $C$ une courbe de classe $\mathcal{C}^1$ et $f$ une fonction continue, alors
	\[ \int_C f \dif r \]
	existe.
\end{mytheo}

\subsubsection{Méthode de calcul}

On paramétrise la courbe $C$ par une fonction $r : t \to r(t)$ avec $t \in [a;b]$ selon les limites du problème. On calcule ensuite l'intégrale par la formule
\[ \int_C f(x,y,z) \dif r = \int_a^b f(r(t)) \left|\left| \frac{\dif r}{\dif t} \right|\right| \dif t \]

\subsection{Intégrale de surface d'un champ scalaire}

On veut donc calculer
\[ \iint_S f(x,y,z) \dif S \]

\subsubsection{Méthode de calcul}

On procède en deux étapes.
\begin{enumerate}
	\item Pour cela on doit paramétrer la surface : soit $r : (u,v) \to r(u,v)$ avec $(u,v) \in R$, représentation paramétrique de la surface. On a donc
		\[ r(u,v) = x(u,v) \xunit + y(u,v) \yunit + z(u,v) \zunit \]
	\item On exprime ensuite $\dif S$ en fonction de $\dif u$ et $\dif v$ par la formule
		\begin{align*} \dif S & = \left| \left| \frac{\pa r}{\pa u} \dif u \times \frac{\pa r}{\pa v} \dif v \right| \right| \\
			& = \left| \left| \frac{\pa r}{\pa u} \times \frac{\pa r}{\pa v} \right| \right|  \dif u \dif v
		\end{align*}
\end{enumerate}
On a donc finalement la formule suivante :
\[ \iint_S f(x,y,z) \dif S = \iint_R f(r(u,v)) \left| \left| \frac{\pa r}{\pa u} \times \frac{\pa r}{\pa v} \right| \right|  \dif u \dif v \]

\subsection{Intégrale de ligne de vecteur le long d'une courbe}

On veut donc calculer la composante tangentielle de vecteur le long d'une courbe
\[ \int_C \vv{F} \cdot \dif \vv{r} \]

\subsubsection{Méthode de calcul}

Comme auparavant, on paramétrise la courbe par une fonction $r : t \to r(t) = x(t) \xunit + y(t) \yunit + z(t) \zunit$ avec $t \in [a;b]$
Et on a donc à calculer
\[ \int_C \vv{F} \cdot \dif \vv{r} = \int_a^b \vv{F}(r(t)) \cdot \left(  \frac{\dif \vv{r}}{\dif t} \right) \dif t \]

\subsection{Intégrale de champ de vecteur à travers une surface}

On veut calculer l'intégrale de la composante normale d'un champ vectoriel à une surface, soit
\[ \iint_S \vv{F} \cdot \hat{N} \dif S = \iint_S \vv{F} \cdot \dif \vv{S} \]

\subsubsection{Méthode de calcul}

On paramétrise la surface par une fonction $r : (u,v) \to r(u,v) = x(u,v) \xunit + y(u,v) \yunit + z(u,v) \zunit$ avec $(u,v) \in D$. Le flux est alors donné par
\[ \iint_S \vv{F} \cdot \dif \vv{S} = \pm \iint_D \vv{F}(r(u,v)) \cdot \left( \frac{\pa r}{\pa u} \times \frac{\pa r}{\pa v} \right) \dif u \dif v \]

\subsection{Résumé}

\begin{center}
	\begin{tabular}{|S c|S c|}
		\hline
		\textbf{Type} & \textbf{Formule} \\
		\hline
		Scalaire sur une ligne & $\displaystyle \int_C f(x,y,z) \dif r = \int_a^b f(r(t)) \left|\left| \frac{\dif r}{\dif t} \right| \right| \dif t$ \\
		\hline
		Scalaire sur une surface & $\displaystyle \iint_S f(x,y,z) \dif S = \iint_R f(r(u,v)) \left| \left| \frac{\pa r}{\pa u} \times \frac{\pa r}{\pa v} \right| \right| \dif u \dif v $ \\
		\hline
		Vectoriel sur une ligne & $ \displaystyle \int_C \vv{F} \cdot \dif \vv{r} = \int_a^b \vv{F}(r(t)) \cdot \left(  \frac{\dif \vv{r}}{\dif t} \right) \dif t $ \\
		\hline
		Vectoriel sur une surface & $ \displaystyle \iint_S \vv{F} \cdot \dif \vv{S} = \pm \iint_D \vv{F}(r(u,v)) \cdot \left( \frac{\pa r}{\pa u} \times \frac{\pa r}{\pa v} \right) \dif u \dif v $ \\
		\hline
	\end{tabular}
\end{center}

\section{Analyse vectorielle}

\begin{mydef}[Gradient] On défini le gradient de $f$ comme
	\[ \nabla f = \left( \frac{\pa f}{\pa x} , \frac{\pa f}{\pa y} , \frac{\pa f}{\pa z} \right) \]
	On peut aussi le définir comme opérateur
	\[ \nabla \sbt =
	\frac{\pa \sbt}{\pa x} \xunit + \frac{\pa \sbt}{\pa y} \yunit + \frac{\pa \sbt}{\pa z} \zunit \]
\end{mydef}

\begin{mydef}[Divergence] On défini la divergence de $\vv{F}$ comme
	\begin{align*} \newdiv \vv{F} & = \nabla \cdot \vv{F} \\
		& = \left( \frac{\pa \sbt}{\pa x} \xunit + \frac{\pa \sbt}{\pa y} \yunit + \frac{\pa \sbt}{\pa z} \zunit \right) \cdot \left( F_1 \xunit + F_2 \yunit + F_3 \zunit \right) \\
		& = \frac{\pa F_1}{\pa x} + \frac{\pa F_2}{\pa y} + \frac{\pa F_3}{\pa z} \\
	\end{align*}
\end{mydef}

\begin{myrem}
	\[ \vv{F} \cdot \nabla \sbt = F_1 \frac{\pa \sbt}{\pa x} + F_2 \frac{\pa \sbt}{\pa y} + F_3 \frac{\pa \sbt}{\pa z}
	\neq \nabla \cdot \vv{F} \]
\end{myrem}

\begin{mydef}[Rotationnel] On défini le rotationnel de $\vv{F}$ comme
	\begin{align*} \rot \vv{F} & = \nabla \times \vv{F} \\
		& = \left( \frac{\pa \sbt}{\pa x} \xunit + \frac{\pa \sbt}{\pa y} \yunit + \frac{\pa \sbt}{\pa z} \zunit \right) \times \left( F_1 \xunit + F_2 \yunit + F_3 \zunit \right) \\
		& = \left( \frac{\pa F_3}{\pa y} -  \frac{\pa F_2}{\pa z} \right) \xunit + \left( \frac{\pa F_1}{\pa z} -  \frac{\pa F_3}{\pa x} \right) \yunit + \left( \frac{\pa F_2}{\pa x} -  \frac{\pa F_1}{\pa y} \right) \zunit
	\end{align*}
\end{mydef}

\begin{myrem}[Produit vectoriel]
	\[ \left( \frac{\pa F_3}{\pa y} -  \frac{\pa F_2}{\pa z} \right) \xunit + \left( \frac{\pa F_1}{\pa z} -  \frac{\pa F_3}{\pa x} \right) \yunit + \left( \frac{\pa F_2}{\pa x} -  \frac{\pa F_1}{\pa y} \right) \zunit =
	\left|
	\begin{array}{ccc}
		\xunit & \yunit & \zunit \\
		\frac{\pa \sbt}{\pa x} & \frac{\pa \sbt}{\pa y} & \frac{\pa \sbt}{\pa z} \\
		F_1 & F_2 & F_3
	\end{array}
	\right| \]
\end{myrem}

\begin{myrem}[En 2 dimensions]
	En deux dimensions, on a
	\[ \newdiv \vv{F} = \frac{\pa F_1}{\pa x} + \frac{\pa F_2}{\pa y} \]
	\[ \rot \vv{F} = \left( \frac{\pa F_2}{\pa x} - \frac{\pa F_1}{\pa y} \right) \hat{k} \]
\end{myrem}

\begin{myform}[Formules générales]
	\[ \nabla (\vv{F} \times \vv{G}) = (\nabla \times \vv{F})\vv{G} - \vv{F} \cdot (\nabla \times \vv{G}) \]
	\[ \newdiv ( \rot \vv{F}) = \nabla (\nabla \times \vv{F}) = 0 \]
	\[ \rot (\nabla \Phi) = \nabla \times (\nabla \Phi) = 0 \]
\end{myform}

\begin{mydef}[Champ de vecteurs solénoïdal ou incompressible] Soit $\vv{F}$ un champ de vecteurs tel que
	\[ \newdiv \vv{F} = 0 \]
	dans $D$ est dit incompressible ou solénoïdal.
\end{mydef}

\begin{mydef}[Champ de vecteurs irrotationnel] Soit $\vv{F}$ un champ de vecteurs tel que
	\[ \rot \vv{F} = 0 \]
	dans $D$ est dit irrotationnel.
\end{mydef}

\begin{myprop}[Champ de vecteurs solénoïdal et irrotationnel] On a les implications suivantes :
	\begin{itemize}
		\item $ \vv{F} \; \text{irrotationnel} \Rightarrow \vv{F} \; \text{conservatif (sur un domaine simplement connexe et ouvert)} $ ;
		\item $ \vv{F} \;\text{conservatif} \Rightarrow \vv{F} \;\text{irrotationnel} $ ;
		\item $ \newdiv (\rot \vv{G}) = 0  \Rightarrow \text{Le rotationnel d'un champ de vecteur est solénoïdal} $ ;
		\item $ \vv{F} \;\text{est un champ vect. solénoïdal} \Rightarrow \newdiv (\rot \vv{G}) = 0 \;\text{(sur un domaine étoilé)} $, ou encore : il existe $\vv{G}$ tel que $\vv{F} = \rot \vv{G}$.
	\end{itemize}
\end{myprop}

\begin{mydef}[Domaine étoilé]
	Un domaine est dit étoile si il existe $P_0$ dans $D$ tel que $P_0 + t(P - P_0)$, reliant $P_0$ à tout $P \in D$ est entièrement contenue dans $D$.
\end{mydef}

\begin{myprop}[Potentiel vecteur]
	Si $\vv{F}$ est un champ vectoriel solénoïdal sur $D$, domaine étoilé, alors il existe un potentiel vecteur $\vv{G}$ tel que $\vv{F} = \rot \vv{G}$.
\end{myprop}

\begin{myform}[Fonction gradient]
	Une fonction $\vv{f}$ est dit être un gradient si il existe $\Phi$ telle que
	\[ \vv{f} = \nabla \Phi \]
	Si $\vv{f}$ est définie sur un domaine simplement connexe et que $\rot \vv{f} = 0$, alors c'est un gradient.
	Si c'est le cas, et si en plus $\vv{f}$ est définie sur un domaine étoilé, on peut déterminer $\Phi$ par
	\[ \Phi = \int_0^1 \left( x F_1(xt,yt,zt) + y F_2(xt,yt,zt) + z F_3(xt,yt,zt) \right) \dif t \]
\end{myform}

\section{Théorèmes intégraux}

\begin{mytheo}[Théorème de Green]
	On travaille dans $\R^2$. Soit $R$ une région du plan avec un bord $C$ orienté. Soit $\vv{F}$ un champ de vecteurs. On a l'égalité
	\[ \iint_R \rot \vv{F} \cdot \hat{k} \dif A = \oint_C \vv{F} \cdot \dif \vv{r} \]
\end{mytheo}

\begin{myrem}[Domaines adjacents]
	Si on a deux domaines adjacents, $R_1$ et $R_2$, alors le théorème est valable sur leur union, le bord adjacent étant <<~annulé~>>.
\end{myrem}

\begin{mytheo}[Théorème de Stokes]
	Soit $S$ une surface de $\R^3$ et soit $C$ le contour orienté de cette surface. Alors
	\[ \iint_S \rot \vv{F} \cdot \hat{N} \dif S = \oint_C \vv{F} \dif \vv{r} \]
\end{mytheo}

\begin{myprop}[Application du théorème de Stokes]
	On peut calculer l'intégrale compliquée du rotationnel d'un champ de vecteur sur une surface en changeant cette surface. Soit $S_1$ et $S_2$ deux surfaces possédant le même contour fermé $C$. Alors on a
	\[ \iint_{S_1} \rot \vv{F} \cdot \hat{N} \dif S = \oint_C \vv{F} \cdot \dif \vv{r} = \iint_{S_2} \rot \vv{F} \cdot \hat{N} \dif S \]
\end{myprop}

\begin{mytheo}[Théorème de la divergence]
	Soit $D$ un volume de $\R^3$, $S$ (surface fermée) le bord de $D$, et $\hat{N}$ la normale à cette surface fermée. Alors
	\[ \iiint_D \newdiv \vv{F} \dif V = \oiint_S \vv{F} \cdot \hat{N} \dif S \]
\end{mytheo}

\section{Rotationnel et divergence}

\begin{myprop}[Rotationnel]
	Le rotationnel c'est la circulation d'un vecteur par unité de surface. C'est donc la densité de circulation. On a
	\[ \rot \vv{P} \cdot \vv{N} = \lim_{\epsilon \to 0} \frac 1{\pi \epsilon^2} \oint_{C_\epsilon} \vv{F} \dif \vv{r} \]
	avec $C_\epsilon$ un cercle de rayon $\epsilon$ centré en $P$.
\end{myprop}

\begin{myprop}[Divergence]
	La divergence, c'est la densité de flux par unité de volume. Soit $D_{\epsilon}$ la boule de rayon $\epsilon$ et $S_{\epsilon}$ la surface de cette boule. On a
	\[ \newdiv \vv{F}(P) = \lim_{\epsilon \to 0} \frac{3}{4 \pi \epsilon^3} \oiint_{S_{\epsilon}} \vv{F} \cdot \hat{N} \dif S \]
\end{myprop}

\end{document}
