\part{Cours Magistral 1 -- Les intégrales}
\section{Introdution}
\emph{Livre de référence : Calculus Adams}
Les théorèmes ne sont pas démontrés par manque de temps mais ils sont tout de même intéressants.
\section{Intégrales multiples}

Cette section se réfère au chapitre 14 du livre de référence.
\subsection{Introduction}
\[\int_{a}^{b} {f(x) dx}\]


\begin{enumerate}

\item Partition P du domaine $D=[a,b]$
$$ P = \{x_0,x_1,...,x_n\}$$
$$a=x_0<x_1<\cdots<x_n=b$$
\item On définit la norme de P
$$||P|| = \underset{1\le i \le n}{max}(\overbrace{(x_i-x_{i-1})}^{\Delta x_i}=1$$

\item Dans chaque $[{x_{i-1},x_i}]$ : choisit un point $c_i$ avec $c=(c_1,c_2,\cdots,c_n)$

\item  Somme de Riemann : $$R(f,P,c)=\sum_{i=1}^{n} \overbrace{f(c_i)\cdot \Delta  x_i}^{\textbf{aire du rectangle i}} $$

\item Passage à la limite $$\lim\limits_{\substack{||P|| \to 0 \\ n \to \infty}} R(f,P,c) =\int_{a}^{b} {f(x) dx}$$

\end{enumerate}
Si cette limite existe, quelque soit les points $c_i$, alors on dira que cette fonction est intégrable.
\begin{myrem}
$$\int_{a}^{b} {f(x) dx} = \int_{a}^{b} {f(t) dt} = \int_{a}^{b} {f(\alpha) d\alpha}$$ où $f(x)$ représente l'intégrant et $dx$ représete la varibale d'intégration \textit{`` dumming variable ''}
\end{myrem}

\begin{myrem}

Les courbes sous l'axe sont négatives et au dessus sont positives. On additionne le tout. On peut donc avoir une somme d'aire négative.

$$\int_{a}^{b} {f(x) dx} \textbf{        } D=[a,b] \subset \mathbb{R}^1$$
$$\int_D f(x_1,x_2,...,x_n)dx_1 \cdots dx_n \textbf{      } D \in \mathbb{R}^n$$
\begin{itemize}


\item $n=1 \to n=2$ C'est pas évident !
\item $n=2 \to n= \infty $ C'est plus simple !

\end{itemize}
\end{myrem}





\subsection{Intégrale Double}
\includegraphics[scale=1]{image3.png}






Région solide $ S \subset \mathbb{R}^3$
\[I=\int \limits_{D}f\overset{ici}{=}\iint_D f (x,y) dA\]
$dA$ est un élément d'aire infinitésimale
\\
$(f \ge 0) \to I = \text{Volume de S}$


\begin{enumerate}
\item

$$D= \text{``pavé'' dans }\mathbb{R}^2 = [a,b]\times[c,d]$$

Chaque rectangle est apellée $R_{ij}$
\[(1 \le i \le m)\]
\[(1 \le j \le n)\]

$$R_{ij}=[x_{i-1},x_i]\times[y_{j-1},y_j]$$
Aire $R_{ij}$ vaut donc
$$R_{ij}=[x_{i-1},x_i] . [y_{j-1},y_j]= \Delta x_i \Delta y_j$$

\item{Norme de P}


diag = diametre $R_{ij}= \sqrt{\Delta x_i^2+\Delta y_j^2}$


\item


\textit{Dans chaque $R_{ij}$, on choisit un point}
$$\{c_{ij}^* = (x_{ij}^*,y_{ij}^*)\} = C^*$$

\item
$$R(f, P,c^*)=\sum_{i=1}^{m} \sum_{j=1}^{n} f(x_{ij}^*,y_{ij}^*)\Delta A_{ij}$$
Avec $\Delta A_{ij}$ qui vaut $\Delta x_i \Delta y_j$


\item

$$\lim\limits_{\substack{||P|| \to 0 \\ n \to \infty\\ m \to \infty}} R(f,P,c^*)=\int_D f$$

\end{enumerate}



\begin{mytheo}

 Si f est continue sur $D$, alors f est intégrale sur D. C'est vrai aussi pour toutes les dimensions ! C'est une condition suffisante ( l'inverse n'est pas vrai).

C'est-à-dire une fonction non-continue peut être intégrable.

\end{mytheo}






\subsection{ Généralisation à $D$ borné }

\subsubsection{ dans $ \mathbb{R}^2$ }






\includegraphics[scale=1]{image4.png}
\\Rectangle qui contient le domaine D et dont les côtés sont parralèles aux axes.

$\hat{f} = \text{ extension f de D à} \mathbb{R}$


\[
\hat{f}(x,y)\left\{
\begin{array}{l r}
f(x,y)\text{ si }(x,y)\in D\\
O\text{ sinon}\\
\end{array}
\right.
\]

f intégrable sur D $ \Longleftrightarrow \hat{f}$ intégrable sur $\mathbb{R}$ et alors
$\int_D f \overset{def}{=}\int_R \hat{f} $


On peut appliquer maintenant pour $n$
\\
$ D \subset \mathbb{R}^n$
$$\int_D f = \overbrace{\iint \cdots}^{n-uple} \int_D f(x_1,x_2,\cdots,x_n)dx_1 dx_2 \cdots dx_n $$

Le produit des $dx_1 \cdots dx_n $ qui sont des éléments infinitésimaux de Volume






\subsubsection{dans $\mathbb{R}^3$}





D = boite de $\mathbb{R}^3$ avec les faces parallèles aux plans de coordonées\\

Il faut faire un découpage en sous boites $ R_{ijk} $ de taille $\Delta x_i \Delta y_j \Delta z_k$

$\left\{
\begin{array}{l}
1 \le i \le m \\
1 \le j \le n \\
1 \le k \le l \\
\end{array}
\right. $

 Il faut choisir \[C_{ijk}^* \in R_{ijk} ( x_{ijk}^*,y_{ijk}^*,z_{ijk}^*)\]

\[R(f,P,c^*)=\sum_{i=1}^m \sum_{j=1}^n\sum_{k=1}^l f (x_{ijk}^*,y_{ijk}^*,z_{ijk}^*) D V_{ijk}\]

On aura donc que

$$\lim\limits_{\substack{l \to \infty \\ n \to \infty\\ m \to \infty}} R=\int_D f$$

Solide borné de $\mathbb{R}^3$
\[\hat{f}=
\left\{
\begin{array}{rr}
f & \text{si}(x,y,z)\in D \\
0 & \text{sinon}

\end{array}
\right.
\]

\[\int_D f \overset{def}{=} \int_R \hat{f}\]
\emph{
Donc, lorsque le domaine de définition d'une intégrale est borné dans $\mathbb{R}^2$ ou $\mathbb{R}^3$, on peut étendre ce domaine lorsque l'on pose que les points qui n'appartiennent pas à ce domaine valent 0.}
\subsection{Propriétés}

\[\int_D f \text{ avec D borné sur } \mathbb{R}^n\]

\begin{myprop}
$f\equiv 1 $
\[\text{D borné de }\mathbb{R}^n\]
\[n=1 : D=[a,b] \int_a^b 1 = b-a = \text{Longueur de }D\]
\[n=2 : D=[a,b]\times[c,d] \int_a^b \int_c^d 1 = (b-a) \cdot (d-c)= \text{Aire de }D\]

\[n=3 : \int_D 1 = \text{Volume de }D\]

Et ainsi de suite. Il suffit d'intégrer la fonction f(x)=1 pour connaître la longueure, la surface, le volume,... du domaine.
\end{myprop}


\begin{myprop}
$\forall L,M \in \mathbb{R} ,\forall f,g \text{ intégrable sur } D \subset \mathbb{R}^n$


\[\int_D(Lf+Mg)\overset{?}{=}L\int_D f + M \int_D g\]
On effectue un déplacement linéaire de l'intégrale. L'intégrales étant une fonction linéaire, cette égalité est juste !

\end{myprop}


\begin{myprop}
$$\text{Si } f \le  g  \forall \text{point de }D\subset \mathbb{R}^n \text{alors} \int_D f \le \int_D g $$
Cette propriétés se vérifie très facilement de manière géométrique.
\end{myprop}

\begin{myprop}
$D= D_1 \cup D_2 \cup \cdots \cup D_k$ et $D_i \cup D_j = \phi$ et $i\neq j$ non-overlopping

On aura donc que $f$ intégrale sur $D_i$
\[\int_D f=\sum_{j=1}^k \int_{D_j} f\]

On peut découper le domaine D en plusieurs sous-domaines distints et calculer la somme des intégrales sur chacun des sous-domaines.
\end{myprop}

\begin{myprop}
\[\left|\int_D f \right| \le \int_D|f|\]

La valeure absolue de l'intégrale de f sera toujours plus petite que l'intégrale de la valeure absolue de f. Encore une fois, on voit très rapidement géométriquement.
\end{myprop}

\begin{myprop}[Théorème de Fubini]
\textit{Ce théorème ne se trouve pas dans le livre !}
Il permet de \emph{réduire} le calcul d'une intégrale n-uple à une \emph{succession} ( dans l'ordre qui convient au mieux ) de n intégrales \emph{unidimensionelles}.

\textit{\textbf{Enoncé}} \\Soit f définie sur un pavé D de $\mathbb{R}^n$\\
$D=[a_1,b_1]\times[a_2,b_2]\times \cdots\ \times[a_n,b_n]$


\[f(x_1,x_2,\cdots,x_n) : D\to\mathbb{R}\]
Si f est continue sur D, alors : $(\sigma(1),\sigma(2),...,\sigma(n)) $= une permutation quelconque de (1,2,...,n)

$$
\int_D f= \int_{a_{\sigma(1)}}^{b{\sigma(1)}}
\left[
\int_{a_{\sigma(2)}}^{b{\sigma(2)}}
\left[ \cdots\ \left[
\int_{a_{\sigma(n)}}^{b{\sigma(n)}}
f(x_1,x_2,\cdots,x_n )dx_{\sigma(n)}
\right]\cdots\right]
dx_{\sigma(2)} \right] dx_{\sigma(1)}
$$

Si la fonction f est \textbf{continue}, on peut calculer les intégrales les unes après les autres dans l'ordre de notre choix en considérant les autres variables comme des constantes.
\end{myprop}




\subsubsection{Illustration du Théorème de Fubini pour $n=2$}




$ \left\{
\begin{array}{l}
D = [0,2]\times[1,3]\\
f = x^2+y
\end{array}
\right.
$

$
 \left\{
\begin{array}{l}
x_1=x\\
x_2=y
\end{array}
\right.
$
$(1,2)=(\sigma(1),\sigma(2))$

\[\int_D f = \int_0^2 \left[ \overbrace{\int_1^3 (x^2+y) dy}^{\text{Intégrale interne}}\right] dx\]

A l'intérieur, c'est `` \textit{y} '' qui est la variable d'intégration et `` \textit{x} '' est considéré comme constant.

\[=x^2\int_1^3 dy+\int _1^3 y dy = x^2 \cdot 2+\left.\frac{y^2}{2}\right]_1^3 = 2x^2+4\]

Au finale, l'intégral vaudra donc,
\[  \int_0^2(2x^2+4)dx = 2\frac{x^3}{3}+4x \left]_{(2,0)} \right.= 40/3\]
On va maintenant vérifier si, en intégrant dans l'ordre inverse, on obtient le même résultat.
\[I = \int_1^3 \left[ \int_0^2 (x^2+y)dx \right] dy\]

A l'intérieur, on a \[\frac{x^3}{3}+2y = 8/3 + 2y\]

Au final, on aura donc \[I = \int_1^3(\frac{8}{3}+2y)dy = 40/3\]

Le théorème de Fubini fonctionne !

\begin{myrem}
Notation des physiciens\\
Exemple :
$$\left\{
\begin{array}{l}
\int_0^2dx\int_1^3dy (x^2+y)\\
\int_1^3dy\int_0^2 dx ( x^2+y)
\end{array}
\right.
$$
On lit à l'envers, c'est-à-dire de droite à gauche.
\[\int_D f = \int_{a_{\sigma(1)}}^{b{\sigma(1)}} dx_{\sigma_1} \cdots \int_{a_{\sigma(n)}}^{b{\sigma(n)}} dx_{\sigma_n} f(x_1 \cdots x_n)\]

\end{myrem}

\subsection{Méthodes de calcul de l'intégrale double}
\subsubsection{Par inspection : $(D,f)$}
Exemple 1 :

\[I=\int_D 3 \text{ sur le domaine } D=[a,b]\times[c,d]\]
\[I=3\int_D 1 = 3 \cdot \text{Surface de }D = 3\overbrace{(d-c)(b-a)}^{\text{Surface de }D} \]

Autre exemple :
\[I = \int_D ( \sin(x) +y^3 + 4 )\text{ sur le domaine } D=x^2+y^2\le 1\]
\[I = \iint_D(\sin(x)+y^3+4)dxdy = \int_D \sin(x) + \int_D y^3 + \int_D 4 = I_1 + I_2 + I_3\]
\begin{itemize}

\item \textbf{I1}




\[\sin(-x) = -\sin(x) \]
C'est une fonction impaire !
Donc on aura \emph{$I_1=0$}
Mais attention, le domaine doit être symetrique par rapport à l'axe des y.
\item \textbf{I2}
Pour la même raison, nous avons $I_2=0$ parce que $(-y^3) = -(y^3)$

\item \textbf{I3}

L'intégrale devient donc
\[\int_D 4 = 4(\pi r^2) = 4\pi \]

\end{itemize}
\subsubsection{Utiliser/généraliser Fubini}

Le domaine D est "y-simple" ou alors " x-simple"\\
\includegraphics[scale=0.7]{image2.png}
\\
Les bords de la fonction pour y-simple sont $y=c(x)$ et $y=d(x)$

Pour $x$-simple, c'est la même chose mais dans l'autre sens. Les bornes sont donc $x=a(y)$ et $x=b(y)$
\begin{myrem}
$D$ peut être à la fois $x$ et $y$-simple
\end{myrem}
\begin{myrem}
Contre-exemple : Lorsqu'une des deux bornes n'est pas une fonction $y=d(x)$ ou $y=c(x)$, le domaine n'est pas y-simple. C'est la même chose pour x-simple.

\end{myrem}

\begin{myrem}
D est ``régulier'' lorsqu'on peut trouver une union finie de domaines simples.


Par exemple, dans ce cas-ci, chacun des 4 sous-domaines est un domaine simple\\
\includegraphics[scale=1]{image1.png}

\end{myrem}
