\documentclass[11pt,a4paper]{article}

% French
\usepackage[utf8x]{inputenc}
\usepackage[frenchb]{babel}
\usepackage[T1]{fontenc}
\usepackage{lmodern}
\usepackage{ifthen}

% Color
% cfr http://en.wikibooks.org/wiki/LaTeX/Colors
\usepackage{color}
\usepackage[usenames,dvipsnames,svgnames,table]{xcolor}
\definecolor{dkgreen}{rgb}{0.25,0.7,0.35}
\definecolor{dkred}{rgb}{0.7,0,0}

% Floats and referencing
\newcommand{\sectionref}[1]{section~\ref{sec:#1}}
\newcommand{\annexeref}[1]{annexe~\ref{ann:#1}}
\newcommand{\figuref}[1]{figure~\ref{fig:#1}}
\newcommand{\tabref}[1]{table~\ref{tab:#1}}

% Listing
\usepackage{listings}
\lstset{
  numbers=left,
  numberstyle=\tiny\color{gray},
  basicstyle=\rm\small\ttfamily,
  keywordstyle=\bfseries\color{dkred},
  frame=single,
  commentstyle=\color{gray}=small,
  stringstyle=\color{dkgreen},
  %backgroundcolor=\color{gray!10},
  %tabsize=2,
  rulecolor=\color{black!30},
  %title=\lstname,
  breaklines=true,
  framextopmargin=2pt,
  framexbottommargin=2pt,
  extendedchars=true,
  inputencoding=utf8x
}
\newcommand{\oz}{\textsc{Oz}}
\newcommand{\java}{\textsc{Java}}
\newcommand{\clang}{\textsc{C}}
\newcommand{\keyword}{mot clef}

% Math symbols
\usepackage{amsmath}
\usepackage{amssymb}
\usepackage{amsthm}

% Sets
\newcommand{\R}{\mathbb{R}}
\newcommand{\Rn}{\R^n}
\newcommand{\Rnn}{\R^{n \times n}}
\newcommand{\C}{\mathbb{C}}
\newcommand{\K}{\mathbb{K}}
\newcommand{\Kn}{\K^n}
\newcommand{\Knn}{\K^{n \times n}}

% Chemistry
\newcommand{\std}{\ensuremath{^{\circ}}}
\newcommand\ph{\ensuremath{\mathrm{pH}}}

% Theorem and definitions
\theoremstyle{definition}
\newtheorem{mydef}{Définition}
\newtheorem{mynota}[mydef]{Notation}
\newtheorem{myprop}[mydef]{Propriétés}
\newtheorem{myrem}[mydef]{Remarque}
\newtheorem{myform}[mydef]{Formules}
\newtheorem{mycorr}[mydef]{Corrolaire}
\newtheorem{mytheo}[mydef]{Théorème}
\newtheorem{mylem}[mydef]{Lemme}
\newtheorem{myexem}[mydef]{Exemple}
\newtheorem{myineg}[mydef]{Inégalité}

% Unit vectors
\usepackage{esint}
\usepackage{esvect}
\newcommand{\kmath}{k}
\newcommand{\xunit}{\hat{\imath}}
\newcommand{\yunit}{\hat{\jmath}}
\newcommand{\zunit}{\hat{\kmath}}

% rot & div & grad & lap
\DeclareMathOperator{\newdiv}{div}
\newcommand{\divn}[1]{\nabla \cdot #1}
\newcommand{\rotn}[1]{\nabla \times #1}
\newcommand{\grad}[1]{\nabla #1}
\newcommand{\gradn}[1]{\nabla #1}
\newcommand{\lap}[1]{\nabla^2 #1}


% Elec
\newcommand{\B}{\vec B}
\newcommand{\E}{\vec E}
\newcommand{\EMF}{\mathcal{E}}
\newcommand{\perm}{\varepsilon} % permittivity

\newcommand{\bigoh}{\mathcal{O}}
\newcommand\eqdef{\triangleq}

\DeclareMathOperator{\newdiff}{d} % use \dif instead
\newcommand{\dif}{\newdiff\!}
\newcommand{\fpart}[2]{\frac{\partial #1}{\partial #2}}
\newcommand{\ffpart}[2]{\frac{\partial^2 #1}{\partial #2^2}}
\newcommand{\fdpart}[3]{\frac{\partial^2 #1}{\partial #2\partial #3}}
\newcommand{\fdif}[2]{\frac{\dif #1}{\dif #2}}
\newcommand{\ffdif}[2]{\frac{\dif^2 #1}{\dif #2^2}}
\newcommand{\constant}{\ensuremath{\mathrm{cst}}}

% Numbers and units
\usepackage[squaren, Gray]{SIunits}
\usepackage{sistyle}
\usepackage[autolanguage]{numprint}
%\usepackage{numprint}
\newcommand\si[2]{\numprint[#2]{#1}}
\newcommand\np[1]{\numprint{#1}}

\newcommand\strong[1]{\textbf{#1}}
\newcommand{\annexe}{\part{Annexes}\appendix}

% Bibliography
\newcommand{\biblio}{\bibliographystyle{plain}\bibliography{biblio}}

\usepackage{fullpage}
% le `[e ]' rend le premier argument (#1) optionnel
% avec comme valeur par défaut `e `
\newcommand{\hypertitle}[7][e ]{
\usepackage{hyperref}
{\renewcommand{\and}{\unskip, }
\hypersetup{pdfauthor={#6},
            pdftitle={Synth\`ese d#1#2 Q#3 - L#4#5},
            pdfsubject={#2}}
}

\title{Synth\`ese d#1#2 Q#3 - L#4#5}
\author{#6}

\begin{document}

\ifthenelse{\isundefined{\skiptitlepage}}{
\begin{titlepage}
\maketitle

 \paragraph{Informations importantes}
   Ce document est grandement inspiré de l'excellent cours
   donné par #7 à l'EPL (École Polytechnique de Louvain),
   faculté de l'UCL (Université Catholique de Louvain).
   Il est écrit par les auteurs susnommés avec l'aide de tous
   les autres étudiants et votre aide est la bienvenue.
   Il y a toujours moyen de l'améliorer surtout que si le cours
   change, la synthèse doit être changée en conséquence.
   On peut retrouver le code source à l'adresse suivante
   \begin{center}
     \url{https://github.com/Gp2mv3/Syntheses}.
   \end{center}
   On y trouve aussi le contenu du \texttt{README} qui contient de plus
   amples informations, vous êtes invité à le lire.

   Il y est indiqué que les questions, signalements d'erreurs,
   suggestions d'améliorations ou quelque discussion que ce soit
   relative au projet
   % signalisations ou signalements ?
   sont à spécifier de préférence à l'adresse suivante
   \begin{center}
     \url{https://github.com/Gp2mv3/Syntheses/issues}.
   \end{center}
   Ça permet à tout le monde de les voir, les commenter et agir
   en conséquence.
   Vous êtes d'ailleurs invité à participer aux discussions.

   Vous trouverez aussi des informations sur le wiki
   \begin{center}
     \url{https://github.com/Gp2mv3/Syntheses/wiki}.
   \end{center}
   comme le status des synthèses pour chaque cours
   \begin{center}
     \url{https://github.com/Gp2mv3/Syntheses/wiki/Status}.
   \end{center}
   vous pouvez d'ailleurs remarquer qu'il en manque encore beaucoup,
   votre aide est la bienvenue.

   Pour contribuer au bug tracker et au wiki, il vous suffira de
   créer un compte sur Github.
   Pour interagir avec le code des synthèses,
   il vous faudra installer \LaTeX.
   Pour interagir directement avec le code sur Github,
   vous devez utiliser \texttt{git}.
   Si cela pose problème,
   nous sommes évidemment ouvert à des contributeurs envoyant leur
   changement par mail ou n'importe quel autre moyen.
\end{titlepage}
}{}

\ifthenelse{\isundefined{\skiptableofcontents}}{
\tableofcontents
}{}
}


% TODO
% Garbage collection in semantique
% terminal in proc

\hypertitle{Informatique}{3}{1402}
{Benoît Legat et Lucas Nyssens}
{Benoît Legat \and Lucas Nyssens}

\lstset{language={Oz}}

\newcommand{\st}{\mathrm{ST}}
\newcommand{\ce}{\mathrm{CE}}
\newcommand{\mozart}{Mozart}

\part{Programmation en \oz{}}
\section{Programmation déclarative}
La programmation déclarative est une programmation
dans laquelle les variables sont à affectation unique et
dans laquelle on écrit plutôt des fonctions récursives à la place
de boucles.

Le grand avantage de ce paradigme est qu'il reste déterministe,
même avec de la concurrence.

\section{Syntaxe et sémantique}
\subsection{Syntaxe et language noyau}
La syntaxe, c'est la définition de ce qui peut être écrit.
Lors de la création d'un language, on a envie que la syntaxe soit
stricte car c'est plus simple d'implémenter le language ainsi mais
pour l'utilisateur, c'est plus rapide d'écrire avec une syntaxe
moins stricte.

Une manière de faire pour palier ce problème est la création
d'un language noyau dont la syntaxe est un sous-ensemble de celle
du language.

La définition du language est plus simple via ce language noyau et
tous les éléments du language peuvent être défini à partir du language noyau.

\subsection{Sémantique}
La sémantique, c'est la description de ce que fait chaque élément
du language.
Comme tout le language peut être défini à partir du language noyau,
il suffit de définir la sémantique du language noyau et on défini toute
la sémantique.
Par contre, il faut traduire son programme en language noyau
si on veut analyser sa sémantique.

En \oz{}, la sémantique se fait à l'aide d'une machine virtuelle.
Cette machine virtuelle est un tuple constitué de l'ensemble des
threads, de la mémoire à affectation unique $\sigma$ et de la mémoire
à affectation multiple $\mu$
\[ (\{\st_1, \ldots, \st_n\}, \sigma, \mu). \]
Chaque thread est représenté par une pile $\st_i$ consitué de couples
$(\verb|<s>|, \ce)$ où $\ce$ est l'environnement contextuel
de l'instruction \lstinline|<s>|
\[ \st_i = [(\verb|<s>|_1, \ce_1), \ldots, (\verb|<s>|_n, \ce_n)]. \]

La machine abstraite répète indéfiniment la même action jusqu'à ce
que toutes les piles soient vides ou qu'elle soit bloquée à cause du
dataflow (voir section~\ref{sec:dataflow}).
Durant cette action, elle choisit une pile.
Ce choix n'est pas arbitraire mais dépend de l'OS (Operating System) dans
lequel \mozart{} s'exécute.
Elle prend l'instruction au sommet de la pile $\verb|<s>|_1$ et l'exécute
avec comme environnement contextuel $\ce_1$.
Cette exécution dépend de l'instruction en question.
La sémantique des différentes instruction est donnée dans la suite de cette
synthèse.

Si on fait de la programmation déclarative, il y a plus de mémoire à
affectation multiple
\[ (\{\st_1, \ldots, \st_n\}, \sigma) \]
et si on ne fait pas de concurrence, il n'y a qu'une seule pile
\[ (\st, \sigma, \mu). \]
La programmation déclarative non concurrente a donc une machine abstraite
plus simple à utiliser
\[ (\st, \sigma). \]

\section{Instructions et expressions}
\label{sec:sv}
\oz{} fait une différence fondamentale entre expression et instruction.
D'ailleurs, en language noyau, les seules ``lignes''
qu'il accepte sont les instructions.
Si on prend le language complet,
il y a une exception car
la dernière instruction d'une fonction doit être une expression.

Par exemple, si on essaie d'exécuter la ligne
\begin{lstlisting}
3
\end{lstlisting}
\oz{} nous dit
\begin{lstlisting}
%** expression at statement position
\end{lstlisting}
car il s'attend à ce que 3 soit une instruction.

Si on essaie d'appeler \lstinline{Foo} défini comme suit
\begin{lstlisting}
proc {Hello}
  {Browse 'Hello world'}
end
fun {Foo}
  {Hello}
end
\end{lstlisting}
\oz{} nous dit
\begin{lstlisting}
%** illegal arity in application
%**
%** Arity found:          1
%** Expected:             0
%** Application (names):  {Hello _}
%** Application (values): {<P/0> _<optimized>}
\end{lstlisting}
car comme \lstinline|Foo| est une fonction,
et que \lstinline|{Hello}| est la dernière ligne qu'il exécute,
il s'attend à ce que \lstinline|Hello| soit une fonction
et non une procédure.
En language noyau, les fonction sont des procédures avec un argument
en plus donc il veut donner un argument en plus à \lstinline|Hello|
mais \lstinline{Hello} ne prend pas d'argument d'où le
``\lstinline{illegal arity}''.

Comme vu plus haut,
\oz{} n'a aucune exception à cela seulement en language noyau.
Par exemple,
\begin{lstlisting}
A = 1
\end{lstlisting}
peut-être utilisé comme expression et comme instruction.
En effet, du sucre syntaxique nous permet d'écrire
\begin{lstlisting}
B = A = 1
\end{lstlisting}
Seulement, en language noyau, ça doit être écrit
\begin{lstlisting}
B = A
A = 1
\end{lstlisting}

\subsection{Composition séquentielle}
Dans le language noyau, on parle toujours d'\emph{une} instruction
alors qu'en soit, parfois, on pourrait en mettre plusieurs.
Pour permettre celà, on rajoute la définition de composition séquentielle.
On dit qu'une instruction, ça peut être deux instruction.
Pour celà, on rajoute dans la syntaxe du language noyau
\begin{lstlisting}
<s> ::= <s>_1 <s>_2
       | ...
\end{lstlisting}
où \lstinline|...| est constitué des autres syntaxes que peut prendre
une instruction.

Évidemment, 3 instructions par la composition séquentielle, c'est
aussi une seule instruction mais il faut effectuer deux compositions
séquentielle pour cela.

Cette définition est utile pour la sémantique.
Analysons la sémantique d'une composition séquentielle de l'instruction
\lstinline|<s>_1 <s>_2| d'environnement contextuel $\ce$.
La machine virtuelle va ajouté \lstinline|<s>_2| dans la pile
avec l'environnement contextuel $\ce$ puis il va ajouter
\lstinline|<s>_1| avec l'environnement contextuel $\ce$.

Par exemple, la machine virtuelle transformera
\begin{lstlisting}
([(A = 1
   B = A + 1
   C = A + B, CE), ...], s)
\end{lstlisting}
en
\begin{lstlisting}
([(A = 1, CE),
  (B = A + 1
   C = A + B, CE), ...], s)
\end{lstlisting}
puis en
\begin{lstlisting}
([(A = 1, CE),
  (B = A + 1, CE),
  (C = A + B, CE), ...], s)
\end{lstlisting}

\subsection{L'instruction qui ne fait rien}
La syntaxe de \oz{} est plutôt stricte sur certains points.
Quand elle dit qu'elle veut une instruction, elle veut
soit une seule instruction, soit plusieurs car on peut considérer
que c'est une seule instruction avec la composition séquentielle mais
pas zéro instructions !

Par exemple
\begin{lstlisting}
if D \= 0 then
  F = N / D
else
end
\end{lstlisting}
n'est pas syntaxiquement valide.
Ça pose réellement problème car en language noyau,
on ne peut pas omettre le \lstinline|else| (voir section~\ref{sec:if}).

\oz{} a donc une instruction qui ne fait rien, le \lstinline|skip|.
\begin{lstlisting}
if D \= 0 then
  F = N / D
end
\end{lstlisting}
s'écrit donc en language noyau
\begin{lstlisting}
local NotZero in
  NotZero = D \= 0
  if NotZero then
    F = N / D
  else
    skip
  end
end
\end{lstlisting}
La sémantique du \lstinline|skip| est assez simple.
On le retire tout simplement de la pile sans rien faire.

\section{Variables}
Les variables en \oz{} sont à affectation unique,
quand elle sont à affectation multiple, on ne les appelle plus des variables
mais des cellules.

\subsection{Déclaration}
\label{sec:dec}
Pour créer une variable ou une cellule,
il faut d'abord créer un identificateur avec \lstinline|declare| ou un
\lstinline|local|.

\subsubsection{Le \keyword{} declare}
Il y a deux manière d'utiliser un \lstinline|declare|.
\begin{itemize}
  \item
    La première
    \begin{lstlisting}
declare <v>_1 <v>_2 ... <v>_n in
<s>
    \end{lstlisting}
    où les identificateurs \lstinline|<v>_i| sont créés
    avec une variable en mémoire associée.

  \item
    La seconde est un sucre syntaxique.
    \begin{lstlisting}
declare
<s>
    \end{lstlisting}
    \oz{} va lister tous les identificateurs utilisés dans \lstinline|<s>|
    qui ne sont pas dans l'environnement contextuel
    et va réécrire le code en language noyau en mettant un \lstinline|in|
    avec ces identificateurs avant.
    Par exemple,
    \begin{lstlisting}
declare
A = 1
B = A + 1
    \end{lstlisting}
    va se réécrire
    \begin{lstlisting}
declare A B in
A = 1
B = A + 1
    \end{lstlisting}
    en language noyau.
\end{itemize}
On peut remarquer que le \lstinline|declare| ne termine
pas par un \lstinline|end| mais par la fin du fichier (ou de la région
sélectionnée dans \mozart{}).
En effet, il est soit terminé par un autre \lstinline|declare|, soit par
une fin de fichier.

\subsubsection{Le \keyword{} local}
\lstinline|local| permet, tout comme le \lstinline|declare| de créer
des identificateurs mais lui peut-être terminé par un \lstinline|end|.
Du coup, le \lstinline|declare| ne s'utilise souvent qu'une seule fois en
début de fichier alors que le \lstinline|local| s'utilise un peu partout.

Sa syntaxe dans le language noyau est la suivante
\begin{lstlisting}
local <x> in <s> end
\end{lstlisting}
Pour définir sa sémantique, supposons que l'environnement du \lstinline|local|
était $\ce_1$.
Il ajoute \lstinline|(<s>, CE_2)| au dessus de sa pile d'instructions avec
$\ce_2 = \ce_1 + \{\verb|<x>| \to x\}$ où
$x$ est la variable en mémoire créée avec
\lstinline|<x>|.
On ajoute d'ailleurs $x$ à la mémoire, $\sigma' = \sigma \cup \{x\}$.

\subsection{Assignation}
Une variable peut avoir deux états (rien à voir avec les cellules)
\begin{itemize}
  \item Non assignée, c'est quand la variable en mémoire n'a pas encore
    de valeur, on peut alors lui assigner n'importe quelle valeur;
  \item Assignée, c'est quand la variable en mémoire a déjà une valeur.
    Dans ce cas, on peut lui assigner une valeur si et seulement si
    c'est la même valeur que celle qu'il a déjà.
\end{itemize}

La syntaxe dans le language noyau d'une assignation est la suivante
\begin{lstlisting}
<x> = <v>
\end{lstlisting}
mais le language rend l'assignation symétrique.

Sémantiquement, la machine virtuelle assigne juste \lstinline|<v>| à
la variable en mémoire associée à \lstinline|<x>| dans
l'environnement contextuel.

Par exemple, dans
\begin{lstlisting}
local X Y in
  X = 1
  Y = 2
  local X in
    X = Y
  end
end
\end{lstlisting}
à la ligne 5, on a
\[ ([(X=Y, \{X\to x', Y\to y\})], \{x=1, x', y=2\}) \]
et la machine virtuelle transforme ça en
\[ ([], \{x=1, x'=2, y=2\}). \]

\section{Cellules}
Le concept d'objet est basé sur le fait qu'une fonction,
qu'on appelle alors \emph{méthode},
s'exécute avec un certain contexte.
Ce concept serait bien pauvre si ce contexte était constitué de constantes.

Tout comme en \java{}, en \oz{}, on sait créer des variables qui ne sont
pas constantes. On appelle ça des cellules.

\subsection{Création des cellules}
Pour créer des cellules,
on utilise la fonction \lstinline|NewCell|.
Par exemple,
\begin{lstlisting}
A = {NewCell 0}
\end{lstlisting}
Crée une nouvelle cellule avec comme valeur
\lstinline|0| et l'assigne à \lstinline|A|.
En language noyau, ça donne
\begin{lstlisting}
local B in
  B = 0
  {NewCell B A}
end
\end{lstlisting}
Au niveau de la sémantique, il devrait déjà y avoir \lstinline|A->a|
dans $\ce{}$ et \lstinline|a| dans $\sigma$.
Après le \lstinline|local|, la composition séquentielle
et le \lstinline|B = 0|, on arrive à
(en considérant qu'on utilise pas de concurrence)
\[ ([(\verb|{NewCell B A}|, \{A\to a, B\to b, \ldots\}), \ldots],
\{a, b = 0, \ldots\}, \mu). \]
Après \lstinline|{NewCell B A}|, on aura
\[ \sigma = \{a = \xi, b = 0, \ldots\} \]
et
\[ \mu = \{a:b, \ldots\}. \]
$\xi$ est appelé le \emph{nom} de \lstinline|A|.

\subsection{Lecture d'une cellule}
L'instruction
\lstinline|A = B|
où \lstinline|B| est une cellule,
n'affecte pas à \lstinline|A| le contenu de \lstinline|B| mais leur donne
le même nom donc si on fait après \lstinline|A := 1|,
le contenu de \lstinline|B| sera changé aussi.

Pour lire le contenu d'une cellule, il faut utiliser \lstinline|@|
Dans l'exemple précédent, l'expression \lstinline|@A| vaut \lstinline|b|,
c'est à dire 0.

\subsection{Écriture d'une cellule}
Pour écrire dans une cellule, il faut utiliser \lstinline|:=|.
Par exemple, à la ligne 4 de
\begin{lstlisting}
A = 0
B = 2
C = {NewCell A}
C := B
C := @C + 1
\end{lstlisting}
la machine virtuelle change la mémoire à affectation multiple
de \lstinline|{c:a,...}| à \lstinline|{c:b,...}|.

La ligne 5 s'écrirait en language noyau en % FIXME
\begin{lstlisting}
local D E F in
  D = @C
  E = 1
  F = D + E
  C := F
end
\end{lstlisting}
Pour la ligne 5, la machine virtuelle changerait la mémoire
à affectation multiple $\mu$ de
\lstinline|{c:b, ...}| à \lstinline|{c:f, ...}| où la mémoire
à affectation unique $\sigma$ vaut
\lstinline|{a=0, b=2, c=|$\xi$\lstinline|, d = 2, e = 1, f = 3}|.


\section{Enregistrements, Tuples et Listes}
%      '|'
%      / \
%    42   nil
\subsection{Enregistrements}
Un enregistrement est constitué d'une étiquette, de champs et
d'une variable associée à chaque champ.
La syntaxe est la suivante
\begin{lstlisting}
<lit>(<f>_1:<x>_1 ... <f>_n:<x>_n)
\end{lstlisting}
où \lstinline|<lit>| est l'étiquette,
les \lstinline|<f>_i| sont les champs et
les \lstinline|<x>_i| sont les variables associées.

Lors de la création d'un enregistrement,
on peut ne pas donner de champ à une variable.
\oz{} donnera $i$ comme champ à la $i\ieme{}$ variable sans champ.
Si il existe déjà un champ $i$, \oz{} ne voudra pas fonctionner.

\subsection{Tuples}
Un tuple est un enregistrement dont les champ sont tous les entiers de
1 à $n$ pour un certain $n$.

\subsection{Listes}
Les listes ont une définition récursive.
C'est soit l'atome \lstinline|nil|,
soit un tuple d'étiquette \lstinline$'|'$ et de deux champ dont le
deuxième vaut une liste.

Par exemple,
\begin{lstlisting}
'|'(chien(nom:'Milou') '|'(chient(nom:'Idefix') nil))
\end{lstlisting}
est une liste.
On peut l'écrire de façon plus condensée en \oz{} de la manière suivante
\begin{lstlisting}
chien(nom:'Milou')|(chient(nom:'Idefix')|nil)
\end{lstlisting}
on peut même l'écrire
\begin{lstlisting}
[chien(nom:'Milou') chient(nom:'Idefix')]
\end{lstlisting}

\section{Structures conditionnelles}
%
%  /\  _____
% /??\/true
% \??/\_____
%  \/  false
Il y a deux types de structures conditionnelles en \oz{},
le \lstinline|if|, et le \lstinline|case|.

\subsection{Structures conditionnelles par conditions}
\label{sec:if}
La syntaxe en language noyau du \lstinline|if| est la
suivante
\begin{lstlisting}
if <x> then
  <s>_1
else
  <s>_2
end
\end{lstlisting}
où \lstinline|<x>| doit être de type \lstinline|boolean|.
D'un point de vue sémantique, supposons qu'on soit à
\[ ([(1-5,\ce),...], \sigma) \]
la machine abstraite va regarder la valeur de \lstinline|<x>|,
si elle vaut \lstinline|true|, alors, ça devient
\[ ([(2,\ce),...], \sigma) \]
sinon, ça devient
\[ ([(4,\ce),...], \sigma). \]

Au point de vue du language,
on peut donner une expression comme condition,
on est pas obligé d'avoir un \lstinline|else|
et on peut même utiliser des \lstinline|elseif|.
Par exemple,
\begin{lstlisting}
if L == oz orelse L == haskell then
  Declaratif = true
  ObjetOriented = true
elseif L == java orelse L == 'C++' then
  Declaratif = false
  ObjetOriented = true
elseif L == 'C' then
  Declaratif = false
  ObjetOriented = false
end
\end{lstlisting}
est syntaxiquement correct.

\subsection{Manipulation de boolean}
Le type \lstinline|boolean| est constitué d'\lstinline|atom| qui valent
soit \lstinline|true|, soit \lstinline|false|.
Pour les manipuler, on a les 3 méthodes
\lstinline|Not|, \lstinline|And| et \lstinline|Or| ainsi que les deux
\keyword s \lstinline|orelse| et \lstinline|andthen|.
La méthode \lstinline|Not| est simple, elle inverse le boolean.
C'est l'équivalent de l'opérateur logique $\lnot$.

Les méthodes \lstinline|And| et \lstinline|Or| semblent faire
la même chose que \lstinline|orelse| et \lstinline|andthen| mais il y a
une différence importante.

Prenons comme exemple, un fonction qui détermine
si \lstinline|B| est un diviseur de \lstinline|A|.
En utilisant \lstinline|andthen|, ça donne
\begin{lstlisting}
fun {DivisorOf B A}
  B \= 0 andthen (A mod B) == 0
end
\end{lstlisting}
On sait que si \lstinline|B| vaut 0,
\lstinline|A mod B| donne une erreur.
Seulement, avec \lstinline|andthen|,
\oz{} est malin et si \lstinline|B \= 0| vaut \lstinline|false|,
il se dit que quelque soit la valeur de \lstinline|(A mod B) == 0|,
l'expression totale vaudra \lstinline|false| et donc il ne l'évaluera
même pas.
En fait, en s'approchant du language noyau, ça devient
\begin{lstlisting}
fun {DivisorOf B A}
  if B \= 0 then
    (A mod B) == 0
  else
    false
  end
end
\end{lstlisting}

Ce n'est pas le cas de la méthode \lstinline|And| qui est simplement
un appel de méthode.
Comme les deux expressions sont des arguments,
il faut les évaluer avant de les passer en argument.
\begin{lstlisting}
fun {DivisorOf B A}
  {And B \= 0 (A mod B) == 0}
end
\end{lstlisting}
devient donc, en s'approchant du language noyau,
\begin{lstlisting}
fun {DivisorOf B A}
  local C D in
    C = B \= 0
    D = (A mod B) == 0
    {And C D}
  end
end
\end{lstlisting}
où \lstinline|(A mod B) == 0| est évaluée dans tous les cas.

Il y a une différence semblable entre l'expression
\lstinline|A orelse B| et l'expression \lstinline|{Or A B}|.
En effet, si \lstinline|A| vaut \lstinline|true|, dans le cas
du \lstinline|orelse|, \lstinline|B| n'est même pas évalué
alors que dans le cas du \lstinline|Or|, il est évalué.

\subsection{Structures conditionnelles par pattern matching}
Les \lstinline|if| sont bien pratiques pour comparer des nombres
mais quand il faut comparer des enregistrements, des tuples ou des listes,
c'est bien plus pratique de faire du pattern matching.

Pour faire du pattern matching,
il faut proposer un expression en partie construite et laisser
des variables pour les parties qu'on veut savoir
Par exemple, si on veut savoir le nom des personnes ayant comme prénom Jean,
il faut utiliser une clause avec \lstinline|personne(nom:Name prenom:Jean)|.

On rassemble les clause dans un case qui a la syntaxe suivante
\begin{lstlisting}
case <x>
of <clause>_1 then
  <s>_1
[] <clause>_2 then
  <s>_2
...
[] <clause>_n then
  <s>_n
else
  <s>_{n+1}
end
\end{lstlisting}
Notons, qu'ici le \lstinline|else| n'est pas obligatoire même dans
le language noyau.
Du point de vue sémantique, soit $\ce$ l'environnement contextuel
du \lstinline|case|, la machine abstraite va trouver
la première clause qui correspond à \lstinline|<x>|.
\begin{itemize}
  \item Si elle n'en
    trouve pas, elle va juste ajouter \lstinline|<s>_{n+1}| avec $\ce{}$ comme
    environnement contextuel;
  \item Sinon, elle créer une variable pour chaque variable
    de la clause qu'il ne connaissait pas (disons $X$ et $Y$)
    et va rajouter l'instruction correspondante \lstinline|<s>_i| au sommet
    de la pile avec comme environnement contextuel
    $\ce + \{X\to x, Y\to y\}$ et il va rajouter $x$ et $y$ dans la mémoire
    la valeur associée qu'il y avait dans \lstinline|<x>|.
\end{itemize}

Prenons un exemple.
\begin{lstlisting}
case P
of personne(nom:Name prenom:'Jean') then
   {Browse Name}
[] personne(nom:'Dupont' prenom:Fname) then
   {Browse Fname}
[] chien(nom:'Milou') then
   {Browse 'Bonjour Milou, comment va Tintin ?'}
end
\end{lstlisting}
Ici, si \lstinline|P| vaut
\lstinline|personne(nom:'Dupont' prenom:'Jean')|.
Il n'y a que la première clause qui sera exécutée.
On passera de
\[ ([(1-8,\ce), \ldots], \sigma) \]
en
\[ ([(3,\ce+\{\verb|Name|\to\verb|name|\}), \ldots],
\sigma\cup\{\verb|name|=\verb|'Dupont'|\}). \]

\section{Fonctions et procédures}
%   __               _ __  _ __ ___   ___
%  / _|_   _ _ __   | '_ \| '__/ _ \ / __|
% | |_| | | | '_ \  | |_) | | | (_) | (__
% |  _| |_| | | | | | .__/|_|  \___/ \___|
% |_|  \__,_|_| |_| |_|
\subsection{Procédures}
Une procédure en \oz{},
c'est une instruction
(qui peut être constituée de plusieurs instructions
par la composition séquentielle)
qui dépend de paramètres.

\subsubsection{Définition d'une procédure}
La syntaxe en language noyau est la suivante
\begin{lstlisting}
proc {<x> <x>_1 <x>_2 ... <x>_n}
  <s>
end
\end{lstlisting}
mais ça peut aussi être écrit
\begin{lstlisting}
<x> = proc {$ <x>_1 <x>_2 ... <x>_n}
  <s>
end
\end{lstlisting}
mais cette écriture ne fait pas partie du language noyau.

La sémantique est la même que pour l'assignation de variable sauf qu'ici,
la valeur est \lstinline|(proc {$ <x>_1 <x>_2 ... <x>_n} <s> end,CE)|
où $\ce$ est l'environnement auxquels on ne garde que les identificateurs
libres, c'est à dire les identificateurs qui sont déclarés avant la
procédure et qu'on utilise dans la procédure.

Il faut bien évidemment déclarer la variable \lstinline|<x>|
avec un \lstinline|declare| ou un \lstinline|local|
(voir section~\ref{sec:dec})
tout comme une variable à laquelle on assigne un entier.

Donc, par exemple, avec les lignes de 3 à 5 de
\begin{lstlisting}
local Answer in
  Answer = 42
  proc {GetAnswer X}
    X = Answer
  end
end
\end{lstlisting}
on passe de
\[
  ([(3-5, \{\verb|GetAnswer|\to\verb|getAnswer|,
  \verb|Answer|\to\verb|42|,\ldots\}),\ldots],\newline
  \{\verb|getAnswer|,\ldots\})
\]
en
\[ ([\ldots],
\{\verb|getAnswer|=(\verb|proc {$ X} 4 end|,
\{\verb|Answer|\to\verb|42|\}),
\ldots\}). \]
On remarque que dans l'environnement contextuel de la cellule en mémoire,
on a gardé que \lstinline|Answer|, on a pas gardé les $\ldots$ car on utilisait
que \lstinline|Answer| ni \lstinline|getAnswer| car on ne faisait pas
de récursion ni même \lstinline|X|.

\subsection{Fonctions}
Les fonctions ne sont pas définies dans le language noyau.
Ce sont juste des procédures qui renvoient une valeur.
La syntaxe est la suivante
\begin{lstlisting}
fun {<x> <x>_1 <x>_2 ... <x>_n)
  <s> <v>
| <v>
end
\end{lstlisting}
C'est à dire que le corps de la fonction est soit constitué par une
instruction (ou plusieurs par la composition séquentielle)
suivie d'une expression, soit par une expression seule.
\footnote{Je n'aurais pas pu juste mettre \lstinline|<v>| en supposant
qu'il existe une composition
séquentielle \lstinline|<v> = <s> <v>'| car c'est faux car en language
noyau, il n'y a que des intructions, les expressions sont obligées de faire
partie d'une intruction (voir section~\ref{sec:sv}).}

Sa traduction en language noyau est
\begin{lstlisting}
proc {<x> <x>_1 <x>_2 ... <x>_n <x>_{n+1})
  <s> <x>_{n+1} = <v>
| <x>_{n+1} = <v>
end
\end{lstlisting}

\subsection{Boucle for}
La boucle \lstinline|for|, c'est du pur sucre syntaxique,
elle a deux formes.
La première,
\begin{lstlisting}
for <x> in <v> do <s> end
\end{lstlisting}
où \lstinline|<v>| doit être une liste.
La traduction en procédure est
\begin{lstlisting}
local
  proc {Loop L}
    case L of <x>|T then
      <s>
      {Loop T}
    else skip end
  end
in
  {Loop <v>}
end
\end{lstlisting}
La deuxième,
\begin{lstlisting}
for <x> in <int>_1...<int>_2 do <s> end
\end{lstlisting}
La traduction en procédure est
\begin{lstlisting}
local
  N = <int>_2
  proc {Loop I}
    if I =< N
      <s>
      {Loop I+1}
    else skip end
  end
in
  {Loop <int>_1}
end
\end{lstlisting}


\section{Concurrence}
\subsection{Concurrence}
%TODO
\subsection{Dataflow}
\label{sec:dataflow}
%TODO

\section{Programmation orientée objet}
%TODO

\part{Définitions} % FIXME

\begin{description}
  \item[Identificateur]
    Nom introduit sur le clavier. Commence avec une majuscule.

  \item[Variable en mémoire]
    Partie de la mémoire dans le système, caché au programmeur.
    Représenté avec des minuscules.

  \item[Environnement]
    Fonction qui prend un identificateur et renvoie une variable.
    Il fait correspondre chaque identificateur
    à une variable en mémoire (et sa valeur).
    Un même identificateur peut correspondre
    à différentes variables en différents endroits du programme.

  \item[Portée lexicale]
    Partie d'un programme pour laquelle l'identificateur
    correspond à la même variable en mémoire.

  \item[Identificateur libre]
    C'est l'identificateur compris
    dans l'environnement contextuel d'une procédure,
    c'est-à-dire les identificateurs qui ne sont pas dans la définition
    de la procédure et qui ne sont pas déclarés dans la procédure.

  \item[Invariant]
    C'est une formule logique qui est vraie à chaque appel récursif.
    Il contient des informations globales
    (qui ne changent pas avec l'exécution ($n$))
    et locales (qui changent à chaque appel ($i$ et $a$)).
    Exemple:
    fonction qui calcule la $n\ieme{}$ puissance de $x$.
    Un invariant est $x^n = x^ia$.

  \item[Accumulateur]
    Se référer au principe des vases communiquant.
    Dans l'exemple précédent, l'accumulateur est $a$.

  \item[Récursion terminale]
    Une fonction est récursive terminale,
    si l'appel récursif est la dernière instruction à exécuter
    dans le corps de la fonction.
    Ainsi la taille de la pile d'instruction
    (cfr Sémantique) reste constante au fil des appels récursifs.

  \item[Spécification]
    C'est une formule mathématique,
    ça n'a rien à voir avec le langage de programmation.
    Exemple:
    définition de $x^n$. $x^0=1$; $x^{2n+1}=x\cdot x^{2n}$; $x^{2n} = (x^n)^2$.

  \item[Liste]
    Structure composée soit d'une liste vide (\lstinline|nil|),
    soit d'un élément suivi d'une autre liste.

  \item[Pattern matching]
    C'est la correspondance des formes.
    Il s'agit d'une succession de clauses.
    Une clause est exécutée,
    si la forme de l'élément dans l'instruction \lstinline|case| correspond à
    ce qui est compris dans la clause.
    La forme correspond si l'étiquette (label) et les arguments correspondent.
    Si c'est le cas,
    les identificateurs de la forme sont affectés aux parties correspondantes.
    Les clauses sont testées dans l'ordre textuel.
    La première à correspondre est exécutée et pas les autres.

  \item[Tuple]
    Structure composée d'une étiquette (label) et d'un nombre fini de champs.
    Les champs sont numérotés à partir de 1.

  \item[Enregistrement ou Record]
    Semblable à un Tuple, mais dont on peut définir le nom des champs.
    Les champs dont on n'a pas défini le nom sont numérotés à partir de 1.
    Deux champs ne peuvent pas avoir le même nom.

    Liste $\subset$ Tuple $\subset$ Record

  \item[Arbre]
    Il s'agit soit d'un nœud (Record) ou d'une feuille (\lstinline|leaf|).
    Chaque nœud comporte un ou plusieurs sous arbres (branches).
    Un arbre binaire est un arbre dont chaque nœud a deux branches.
    Un arbre est ordonné si, pour chaque nœud,
    la clé de leurs sous arbres de gauche (resp. droite)
    sont inférieures (supérieures) à la clé du nœud.

  \item[Sémantique]
    C'est ce que fait l'implémentation du langage de l'ordinateur.
    Il s'agit du modèle précis de l'exécution du langage de programmation.
    On l'utilise pour vérifier que le programme répond
    bien aux spécifications (souvent par preuve inductive).

  \item[Machine abstraite]
    Construction mathématique qui modélise l'exécution
    (ce sont les concepts utilisés pour effectuer
    la sémantique sur un programme).
    \begin{itemize}
      \item mémoire à affectation unique : variable et valeur (la valeur d'une variable ne change pas)
        $\sigma = \{ x=10, y=2, c=\xi\}$

      \item mémoire à affectation multiple : contenu des cellules
        $\mu = \{c:x\}$

      \item environnement:
        $E= \{ X\rightarrow x, Y \rightarrow y\}$

      \item instruction sémantique: instruction + son environnement
        $(<s>,E)$

      \item pile sémantique: pile d'instructions sémantiques
        $\st=[(<s_1>,E_1), \ldots, (<s_n>,E_n)]$

      \item exécution : séquence d'états d'exécution
        $(\st_1,\sigma_1)\rightarrow(\st_2,\sigma_2)\rightarrow
        \ldots \rightarrow(\st_n,\sigma_n)$
    \end{itemize}

  \item[Quelques règles de sémantique]

    Etat initial: $\st_0 =([(<s>,E=\emptyset)],\sigma_0=\emptyset)$

    A la fin: $ST_n =([],\sigma_n)$

    Environnement: adjonction $E_2 = E_1 \cup \{X\rightarrow x\}$
    sauf pour les appels de fonction: restriction à l'environnement contextuel.

    Composition séquentielle: $([(<s>,E)],\sigma)
    \rightarrow ([(<s_1>,E),(<s_2>,E)],\sigma)$

    Procédure en mémoire:
    $p = ($\lstinline|proc(|\$ \lstinline|X1 X2 \ldots) <s> end|$, \ce)$

  \item[Argument formel]
    Arguments dans la définition d'une procédure.
    Exemple:
    \lstinline|P = proc(|\$ \lstinline|X1 X2)|
    \lstinline|X1| et \lstinline|X2| sont des arguments formels.

  \item[Environnement contextuel]
    Environnement propre à une certaine procédure.
    Il fait le lien entre, uniquement,
    les identificateurs libres et leur variable en mémoire.

  \item[Cellule]
    C'est un conteneur avec une identité constante et un contenu variable.
    L'identité est le nom ou adresse de la cellule.
    Le contenu est associé à une variable qui est constante,
    mais la variable peut être remplacée par une autre.
    Il y a égalité entre deux cellules si ce sont les mêmes.

    Exemple : $\sigma = \{x=10, c=\xi\}$ et $\mu = \{c:x\}$
    Nous avons que $\xi$ est l'identité de \lstinline|C|, son contenu est $x$.

    En langage noyau, la construction d'une cellule est de la manière suivante:
    \lstinline|{NewCell I C}| ($\Leftrightarrow$ \lstinline|C = {NewCell I}|)

  \item[Modularité]
    On dit qu'un système (ou programme) est \emph{modulaire} si des mises
    à jour dans une partie du système n'obligent pas de changer le reste.
    L'état est bénéfique pour la modularité.

  \item[Encapsulation]
    A compléter. % TODO

  \item[Abstraction]
    A compléter. % TODO

  \item[Polymorphisme]
    Une méthode est polymorphe si elle peut prendre
    des arguments différents et faire la même chose dans chacun des cas.
    Une telle méthode est reprise dans la définition d'une interface en java.

  \item[Héritage]
    Lorsqu'une classe \lstinline|A| hérite d'une classe \lstinline|B|,
    \lstinline|A| prend la définition de \lstinline|B|
    comme base (méthodes et attributs)
    et peut réécrire les méthodes de \lstinline|B| ou en ajouter.

  \item[Super-Classe]
    \lstinline|B| est une super-classe de \lstinline|A|
    si \lstinline|B| est dans la
    déclaration \lstinline|from| de \lstinline|A|
    ou est une super-classe d'une classe
    qui est dans la déclaration \lstinline|from| de \lstinline|A|.

  \item[Objet]
    Ensemble des procédures visibles (\lstinline|methode|) qui
    ont accès à un état caché dans une cellule (\lstinline|attribut|).

  \item[Classe]
    Séparation entre définition d'objet et création.

  \item[Lien Dynamique]
    \lstinline|self| A utiliser par défaut,
    fait référence à l'objet qui appelle la méthode.

  \item[Lien Statique]
    Un appel
    \lstinline|SuperClasse,methode|
    dans une classe \lstinline|A|
    fait appel à la méthode décrite dans la classe \lstinline|SuperClasse|
    qui doit être une super-classe de \lstinline|A|.
    À utiliser uniquement pour la redéfinition d'une méthode
    déjà définie dans une super-classe.

  \item[Concurrence]
    Exécution de plusieurs activités au même temps.
    Elles peuvent communiquer (passer des informations de l'une à l'autre)
    et synchroniser (une activité attend une autre).

  \item[Exécution en Dataflow]
    Si on utilise une variable non initialisée,
    l'exécution attend qu'elle le soit.
    \lstinline|Browse| utilise un \lstinline|thread|,
    il actualise l'affichage quand un identificateur est lié.

  \item[Fil Thread]
    C'est une activité, une séquence d'instructions en exécution.
    Chaque fil est indépendant des autres,
    c'est-à-dire qu'il n'y a pas d'ordre entre eux.
    Deux fils communiquent s'ils partagent les mêmes variables.

  \item[Ordre total]
    Tous les états d'exécution d'un même fil font un ordre total.
    Dans un programme séquentiel (non concurrent), il existe un seul fil,
    l'ordre total décrit l'ordre entre chaque paire d'états.

  \item[Ordre partiel]
    Les états d'exécution du programme complet (s'il est concurrent)
    sont dans un ordre partiel.
    On ne connaît pas l'ordre d'exécution entre un état d'un fil
    et un autre d'un autre fil, ce pour chaque paire d'états du programme.

  \item[Non déterminisme]
    Ça apparait lorsqu'on utilise la concurrence avec les cellules.
    Absolument à éviter à cause de l'ordre partiel.
    Un programme concurrent est non déterministe
    si le résultat final n'est pas déterministe.

  \item[Flot]
    C'est une liste dont l'extrémité est une variable non liée.
    Ça sert comme canal de communication entre deux fils (agents).

  \item[Agent]
    Activité concurrente qui utilise un ou
    plusieurs canaux de communication (flots).

  \item[Producteur]
    Agent qui génère un flot de données.
    (Fonction utile: \lstinline|{Delay 1000}|
    met l'exécution en pause pendant une seconde).

  \item[Consommateur]
    Agent qui lit un flot de données. Généralement les affichent.
\end{description}


\annexe
\section{Installation et utilisation}
\subsection{Installation}
\subsubsection{Sous GNU/Linux}
Sous Debian/Ubuntu, voir le wiki communautaire d'Ubuntu
\begin{itemize}
  \item
    \url{https://help.ubuntu.com/community/Mozart}.
\end{itemize}

Pour les autres distribution, voir les liens suivant sur le site officiel
\begin{itemize}
  \item
    \url{http://www.mozart-oz.org/download/view.cgi?action=rpm&version=1.4.0};
  \item
    \url{http://www.mozart-oz.org/documentation/install/node2.html#chapter.rpm}.
\end{itemize}

\subsection{Utilisation}
\mozart{} est juste un plugin pour l'éditeur de texte GNU/Emacs.
Ce dernier est une éditeur de texte libre extrèmement puissant et complet.
Vous ne connaitrez sans doute jamais toutes ses fonctionnalités.

Pour pouvez par exemple essayer de taper \verb|M-x tetris| ou
\verb|M-x doctor| (\verb|M-x| signifie \verb|Alt + X|).

Néanmoins, pour développer en Oz, les fonctions utiles se trouvent
dans le sous-menu \emph{oz} de GNU/Emacs.

Voici les principales, données pour les régions mais elles sont pareilles
pour une ligne ou un buffer.
\begin{description}
  \item[Indent] Indente automatiquement la région sélectionnée.
  \item[Comment] Met la région sélectionnée en commentaire.
  \item[Core Syntax] Traduit la région sélectionnée en language noyau.
  \item[Feed Region] Exécute la région sélectionnée.
  \item[Halt Oz] Interrompt l'exécution en cours.
\end{description}

\end{document}
