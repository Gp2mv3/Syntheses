\documentclass[11pt,a4paper]{article}

% French
\usepackage[utf8x]{inputenc}
\usepackage[frenchb]{babel}
\usepackage[T1]{fontenc}
\usepackage{lmodern}
\usepackage{ifthen}

% Color
% cfr http://en.wikibooks.org/wiki/LaTeX/Colors
\usepackage{color}
\usepackage[usenames,dvipsnames,svgnames,table]{xcolor}
\definecolor{dkgreen}{rgb}{0.25,0.7,0.35}
\definecolor{dkred}{rgb}{0.7,0,0}

% Floats and referencing
\newcommand{\sectionref}[1]{section~\ref{sec:#1}}
\newcommand{\annexeref}[1]{annexe~\ref{ann:#1}}
\newcommand{\figuref}[1]{figure~\ref{fig:#1}}
\newcommand{\tabref}[1]{table~\ref{tab:#1}}

% Listing
\usepackage{listings}
\lstset{
  numbers=left,
  numberstyle=\tiny\color{gray},
  basicstyle=\rm\small\ttfamily,
  keywordstyle=\bfseries\color{dkred},
  frame=single,
  commentstyle=\color{gray}=small,
  stringstyle=\color{dkgreen},
  %backgroundcolor=\color{gray!10},
  %tabsize=2,
  rulecolor=\color{black!30},
  %title=\lstname,
  breaklines=true,
  framextopmargin=2pt,
  framexbottommargin=2pt,
  extendedchars=true,
  inputencoding=utf8x
}
\newcommand{\oz}{\textsc{Oz}}
\newcommand{\java}{\textsc{Java}}
\newcommand{\clang}{\textsc{C}}
\newcommand{\keyword}{mot clef}

% Math symbols
\usepackage{amsmath}
\usepackage{amssymb}
\usepackage{amsthm}

% Sets
\newcommand{\R}{\mathbb{R}}
\newcommand{\Rn}{\R^n}
\newcommand{\Rnn}{\R^{n \times n}}
\newcommand{\C}{\mathbb{C}}
\newcommand{\K}{\mathbb{K}}
\newcommand{\Kn}{\K^n}
\newcommand{\Knn}{\K^{n \times n}}

% Chemistry
\newcommand{\std}{\ensuremath{^{\circ}}}
\newcommand\ph{\ensuremath{\mathrm{pH}}}

% Theorem and definitions
\theoremstyle{definition}
\newtheorem{mydef}{Définition}
\newtheorem{mynota}[mydef]{Notation}
\newtheorem{myprop}[mydef]{Propriétés}
\newtheorem{myrem}[mydef]{Remarque}
\newtheorem{myform}[mydef]{Formules}
\newtheorem{mycorr}[mydef]{Corrolaire}
\newtheorem{mytheo}[mydef]{Théorème}
\newtheorem{mylem}[mydef]{Lemme}
\newtheorem{myexem}[mydef]{Exemple}
\newtheorem{myineg}[mydef]{Inégalité}

% Unit vectors
\usepackage{esint}
\usepackage{esvect}
\newcommand{\kmath}{k}
\newcommand{\xunit}{\hat{\imath}}
\newcommand{\yunit}{\hat{\jmath}}
\newcommand{\zunit}{\hat{\kmath}}

% rot & div & grad & lap
\DeclareMathOperator{\newdiv}{div}
\newcommand{\divn}[1]{\nabla \cdot #1}
\newcommand{\rotn}[1]{\nabla \times #1}
\newcommand{\grad}[1]{\nabla #1}
\newcommand{\gradn}[1]{\nabla #1}
\newcommand{\lap}[1]{\nabla^2 #1}


% Elec
\newcommand{\B}{\vec B}
\newcommand{\E}{\vec E}
\newcommand{\EMF}{\mathcal{E}}
\newcommand{\perm}{\varepsilon} % permittivity

\newcommand{\bigoh}{\mathcal{O}}
\newcommand\eqdef{\triangleq}

\DeclareMathOperator{\newdiff}{d} % use \dif instead
\newcommand{\dif}{\newdiff\!}
\newcommand{\fpart}[2]{\frac{\partial #1}{\partial #2}}
\newcommand{\ffpart}[2]{\frac{\partial^2 #1}{\partial #2^2}}
\newcommand{\fdpart}[3]{\frac{\partial^2 #1}{\partial #2\partial #3}}
\newcommand{\fdif}[2]{\frac{\dif #1}{\dif #2}}
\newcommand{\ffdif}[2]{\frac{\dif^2 #1}{\dif #2^2}}
\newcommand{\constant}{\ensuremath{\mathrm{cst}}}

% Numbers and units
\usepackage[squaren, Gray]{SIunits}
\usepackage{sistyle}
\usepackage[autolanguage]{numprint}
%\usepackage{numprint}
\newcommand\si[2]{\numprint[#2]{#1}}
\newcommand\np[1]{\numprint{#1}}

\newcommand\strong[1]{\textbf{#1}}
\newcommand{\annexe}{\part{Annexes}\appendix}

% Bibliography
\newcommand{\biblio}{\bibliographystyle{plain}\bibliography{biblio}}

\usepackage{fullpage}
% le `[e ]' rend le premier argument (#1) optionnel
% avec comme valeur par défaut `e `
\newcommand{\hypertitle}[7][e ]{
\usepackage{hyperref}
{\renewcommand{\and}{\unskip, }
\hypersetup{pdfauthor={#6},
            pdftitle={Synth\`ese d#1#2 Q#3 - L#4#5},
            pdfsubject={#2}}
}

\title{Synth\`ese d#1#2 Q#3 - L#4#5}
\author{#6}

\begin{document}

\ifthenelse{\isundefined{\skiptitlepage}}{
\begin{titlepage}
\maketitle

 \paragraph{Informations importantes}
   Ce document est grandement inspiré de l'excellent cours
   donné par #7 à l'EPL (École Polytechnique de Louvain),
   faculté de l'UCL (Université Catholique de Louvain).
   Il est écrit par les auteurs susnommés avec l'aide de tous
   les autres étudiants et votre aide est la bienvenue.
   Il y a toujours moyen de l'améliorer surtout que si le cours
   change, la synthèse doit être changée en conséquence.
   On peut retrouver le code source à l'adresse suivante
   \begin{center}
     \url{https://github.com/Gp2mv3/Syntheses}.
   \end{center}
   On y trouve aussi le contenu du \texttt{README} qui contient de plus
   amples informations, vous êtes invité à le lire.

   Il y est indiqué que les questions, signalements d'erreurs,
   suggestions d'améliorations ou quelque discussion que ce soit
   relative au projet
   % signalisations ou signalements ?
   sont à spécifier de préférence à l'adresse suivante
   \begin{center}
     \url{https://github.com/Gp2mv3/Syntheses/issues}.
   \end{center}
   Ça permet à tout le monde de les voir, les commenter et agir
   en conséquence.
   Vous êtes d'ailleurs invité à participer aux discussions.

   Vous trouverez aussi des informations sur le wiki
   \begin{center}
     \url{https://github.com/Gp2mv3/Syntheses/wiki}.
   \end{center}
   comme le status des synthèses pour chaque cours
   \begin{center}
     \url{https://github.com/Gp2mv3/Syntheses/wiki/Status}.
   \end{center}
   vous pouvez d'ailleurs remarquer qu'il en manque encore beaucoup,
   votre aide est la bienvenue.

   Pour contribuer au bug tracker et au wiki, il vous suffira de
   créer un compte sur Github.
   Pour interagir avec le code des synthèses,
   il vous faudra installer \LaTeX.
   Pour interagir directement avec le code sur Github,
   vous devez utiliser \texttt{git}.
   Si cela pose problème,
   nous sommes évidemment ouvert à des contributeurs envoyant leur
   changement par mail ou n'importe quel autre moyen.
\end{titlepage}
}{}

\ifthenelse{\isundefined{\skiptableofcontents}}{
\tableofcontents
}{}
}


\usepackage{pgfplots}
%\pgfplotsset{compat=1.11} % FIXME why is it here ? 1.11 is too recent (August 2014) to be enforced :P
\usepackage{caption}
\usepackage{subcaption}
\usepackage{xspace}

\hypertitle{Physique}{3}{FSAB}{1203}
{Ga\"{e}tan Cassiers\and Nicolas Cognaux\and Benoît Legat\and Antoine Legat\and Antoine Paris}
{Jérôme Louveaux, Claude Oestges et Jean-Christophe Charlier}

\part{Ondes}
%   _     _     _     _     _     _     _...
%  / \   / \   / \   / \   / \   / \   / ...
% /   \_/   \_/   \_/   \_/   \_/   \_/  ...
\section{Courant de déplacement}
La loi d'ampère
\[ \oint \vec{B} \cdot \dif \vec{l} = \mu_0 \int \vec{J} \cdot \dif \vec{A} \]
est incohérente dans certains cas (exemple du condensateur).

Il faut rajouter un terme appelé \emph{courant de déplacement}
pour qu'elle soit correcte dans tous les cas.
Ce courant est déterminé par
\[ \vec{J}_\mathrm{D} = \perm_0\fpart{\vec{E}}{t} \]
et l'équation d'Ampère correcte est alors
\[ \oint \vec{B} \cdot \dif \vec{l} =
\mu_0 \int \vec{J} \cdot \dif \vec{A} +
\mu_0\perm_0 \int\fpart{\vec{E}}{t} \cdot \dif \vec{A}. \]

Le courant de déplacement n'est pas vraiment un courant (déplacement de charges),
mais il est équivalent : il permet de représenter le \og courant\fg qui
circule dans un condensateur (et donc de vérifier la loi des n\oe{}uds
(loi de Kirchoff) dans tous les cas).

\section{Équations de Maxwell}
On peut résumer toutes les lois qui gouvernent l'électromagnétisme
par 4 lois que les théorèmes intégraux nous permettent de réécrire
de 2 manière différentes
\begin{align*}
  &\text{Gauss} & \oint \E \cdot \dif \vec{A} & = \frac{Q}{\perm}
  & \divn{\E} & = \frac{\rho}{\perm}\\
  &\text{No monopôle magn.} & \oint \B \cdot \dif \vec{A} & = 0
  & \divn{\B} & = 0\\
  &\text{Lenz-Faraday} & \oint \E \cdot \dif \vec{l} & = -\fpart{}{t} \int \B \cdot \dif \vec{A}~~~\text{\footnotemark}
  & \rotn{\E} & = -\fpart{\B}{t}\\
  &\text{Maxwell-Ampère} & \oint \B \cdot \dif \vec{l} & = \mu \int \vec{J} \cdot \dif \vec{A}
  + \mu\perm\fpart{}{t} \int \E \dif \vec{A}
  & \rotn{\B} & = \mu\vec{J} + \mu\perm\fpart{\E}{t}
\end{align*}
\footnotetext{Les physiciens qui aiment la symétrie cherchent un terme supplémentaire ici.}

avec
\begin{align*}
  \vec{E} & = -\gradn{V}\\
  Q & = \int \rho \dif V.
\end{align*}

\section{Ondes}

\subsection{Équation d'onde et équations de propagation}

Une onde est un phénomène de propagation d'une quantité \(\vec{u}\)
 qui vérifie l'\emph{équation d'onde}
\begin{equation*}\begin{split}
\frac{\partial^2 u}{\partial t^2} &= v^2 \nabla^2 u \\
 &= v^2 \left(
 \frac{\partial^2 u}{\partial x^2} +
 \frac{\partial^2 u}{\partial y^2} +
 \frac{\partial^2 u}{\partial z^2}\right)
\end{split}\end{equation*}

Lorsque deux grandeurs sont liées par les \emph{équations de propagation}
\begin{equation}\begin{split}\label{eq:propagation}
    \nabla^2 A & = -a \frac{\partial B}{\partial t} \\
    \frac{\partial A}{\partial t} & = -b \nabla^2 B
\end{split}\end{equation}
Elles obéissent toutes les deux à la même équation d'onde
\begin{align*}
\dfrac{\partial^2 A}{\partial t^2} &= v^2 \nabla^2 A
& \dfrac{\partial^2 B}{\partial t^2} &= v^2 \nabla^2 B
\end{align*}
avec \[v = \sqrt{\frac{b}{a}}\]

Il existe deux types d'ondes,
\begin{itemize}
  \item les ondes transverses où $\vec{u} \perp \vec{v}$ (par exemple une corde);
  \item les ondes longitudinales où $\vec{u} \parallel \vec{v}$ (ex: le son).
\end{itemize}

\subsection[Cas particuliers]{Solution de l'équation d'onde dans des cas particuliers}
\subsubsection{Onde 1D}

On peut se demander s'il existe une fonction à une variable $C(w)$ vérifiant
l'équation d'onde à une dimension, c'est-à-dire $C(w) = u(x,t)$,
et si oui quelle est l'expression de $w(x,t)$.
La réponse est oui, si on prend $w(x,t) = x \pm\footnote{En fait, le signe importe peu, vu que $v$ peut être positif ou négatif (car il y a seulement $v^2$ dans l'équation d'onde} vt$\footnote{ou tout multiple de $w$},
on vérifie\footnote{à l'aide de la \textit{Chain rule}}
\[ \dfrac{\partial^2 C(w)}{\partial t^2} =
v^2 \dfrac{\partial^2 C(w)}{\partial x^2}\]

Cela correspond bien à l'intuition. En effet, c'est comme si on avait une fonction $C(x)$
(p. ex. une sinusoïde ou un gaussienne) qui se déplaçait dans le temps à une vitesse $v$.

Si $v$ est positif, l'équation $w(x,t) = x - vt$ correspond à une onde se déplaçant vers
les $x$ croissants, tandis que $w(x,t) = x + vt$ correspond à une onde qui se déplace vers
les $x$ décroissants.

L'équation d'onde en 1D est donc
\begin{equation}u(x, t) = f(x - vt)\label{eq:onde1D}\end{equation}
Pour les ondes périodiques, on écrit souvent cette équation $u(x, t) = f(kx-\omega t)$
(voir \ref{sec:onde_sin}).

En l'absence de constante d'intégration\footnote{
    Le champ magnétique de la Terre dans le cas d'une onde EM,
    la hauteur de la corde au repos dans le cas d'une corde vibrante ...},
le rapport de $A$ sur $B$ (variables des équations de propagation \eqref{eq:propagation})
vaut une constante particulière dénotée $Z$ et appelée l'\emph{impédance caractéristique}
\footnote{Par exemple, l'impédance caractéristique d'un milieu pour une onde acoustique
représente la résistance du milieu au passage de cette onde.}
\[\dfrac{A(x-vt)}{B(x-vt)} = Z = \sqrt{ab}\]


\subsubsection{Onde 3D : cas symétriques}\label{sec:cas_symetriques}
Dans certains cas symétriques, on se ramène au cas 1D:
\begin{itemize}
\item Onde plane :
\begin{equation}f(\vec{k}\cdot\vec{x}\pm\omega t)\label{eq:expr_onde_plane}\end{equation} avec
$\vec{x}$ le vecteur position du point et $\vec{k}$ le vecteur d'onde.
\item Onde cylindrique : \(f(k\rho\pm\omega t)\) avec \(\rho = \sqrt{x^2+y^2}\)
\item Onde sphérique : \(f(kr\pm\omega t)\) avec \(r = \sqrt{x^2+y^2+z^2}\)
\end{itemize}


\subsubsection{Onde sinusoïdale}\label{sec:onde_sin}
Une onde sinusoïdale 1D a l'équation suivante
\begin{equation}\label{eq:onde_sin}
A(x, t) = A_0 \sin(kx - \omega t + \phi)
\footnote{Cette expression est donnée dans le cas 1D
(c'est un cas particulier de l'équation \eqref{eq:onde1D}) mais on peut
bien sûr généraliser aux cas 3D symétriques (\ref{sec:cas_symetriques})}
\end{equation}
et les caractéristiques suivantes
\begin{itemize}
  \item $A_0$, son amplitude, c'est à dire son intensité maximale;
  \item $x$, la position du point;
  \item $\omega$, sa vitesse angulaire;
  \item $f$, sa fréquence d'oscillation;
  \item $T$, sa période d'oscillation;
  \item $k$, le nombre d'onde;
  \item $\lambda$, la longueur d'onde;
  \item $v$, sa vitesse de propagation;
  \item $\phi$, une phase.
\end{itemize}
Elles sont liées entre elles par les relations suivantes
\begin{align*}
  T & = \frac{1}{f}\\
  \omega & = 2\pi f\\
  v & = \lambda f\\
  k & = \frac{2\pi}{\lambda}
\end{align*}

% TODO : front d'onde

\subsection{Ondes électromagnétiques}

\subsubsection{Propriétés}
Les ondes électromagnétiques sont des ondes transverses
de $\vec{E}$ et $\vec{B}$.

Leur vitesse de propagation est $c$ (vitesse de la lumière).

On a les relations supplémentaires\footnote{Ces relations peuvent
être déduites pour un cas particulier d'onde
électromagnétique relativement simple (les ondes électromagnétiques
planes) en vérifiant la validité des lois d'Ampère et de Faraday.
Elles sont cependant valables pour \emph{toutes} les ondes
électromagnétiques.}
\begin{align*}
  \vec{E} & \perp \vec{B}\\
  c & = \frac{1}{\sqrt{\perm\mu}}\\
  \frac{E}{B} & = c.
\end{align*}

\subsubsection{Équations de propagation et équation d'onde}
En émettant les hypothèses que $\vec{E}$ est suivant $\hat{y}$ et $\vec{H}$ suivant $\hat{z}$ et à l'aide des équations de Maxwell
\footnote{Sous l'hypothèse que $\vec{J}=0$. Dans le vide ce sera toujours le cas.}, on trouve que $\vec{E}$ et $\vec{H}$ ne dépendent
que de leur abscisse $x$.Les champs électrique et magnétique sont donc constants le long d'un plan perpendiculaire à l'axe des abscisses.
On obtient également les équations de propagation d'une onde électromagnétique
\begin{align*}
\fpart{E_y}{x} &= -\mu \fpart{H_z}{t} & \fpart{H_z}{x} &= -\perm \fpart{E_y}{t}.
\end{align*}

En appliquant $\fpart{}{t}$ et $\fpart{}{x}$ à ces relations, on obtient l'équation d'onde de $H_z$ et $E_y$ respectivement
\begin{align*}
\dfrac{\partial^2 H_z}{\partial t^2} &=
c^2 \dfrac{\partial^2 H_z}{\partial x^2}
& \dfrac{\partial^2 E_y}{\partial t^2} &=
c^2 \dfrac{\partial^2 E_y}{\partial x^2}.
\end{align*}

Cela permet de calculer l'impédance caractéristique,
\[
Z = \sqrt{\dfrac{\mu}{\perm}} \overset{\text{si dans le vide}}{=} \sqrt{\dfrac{\mu_0}{\perm_0}} = \unit{120\pi}{\ohm},
\]
et dépend donc du milieu.

On a montré qu'une champ électromagnétique peut se propager dans le vide
(ou dans la matière) sous la forme d'une onde. Cette onde peut avoir différentes
caractéristiques selon l'émetteur (antenne...).
Un cas fréquent est l'onde plane sinusoïdale (\eqref{eq:expr_onde_plane} et
\eqref{eq:onde_sin}).

\subsubsection{Densité d'énergie d'une onde électromagnétique}
En se rappelant de la densité d'énergie d'un champ électrique et
d'un champ magnétique, on peut écrire

$$u = u_E + u_M = \frac{\perm E^2}{2} + \frac{H^2}{2\mu}.$$

Or on sait que $B = \frac{E}{c} = E\sqrt{\perm\mu}$, on a donc

$$u = \unit{\perm E^2}{\joule\per\meter\cubed}$$

\subsubsection{Intensité d'une onde électromagnétique}
L'intensité d'une onde électromagnétique, c'est à
dire la densité de puissance véhiculée par l'onde, est calculée
en multipliant la densité d'énergie par la vitesse de propagation

$$I = cu = \unit{c\perm E^2}{\watt\per\meter\squared}$$

De manière plus générale, on défini le \emph{vecteur de Poynting}

\[\vec{S} = \vec{E} \times \vec{H} = \frac{1}{\mu} \vec{E} \times \vec{B}\]

parallèle à la directions de propagation de l'onde électromagnétique
et la norme en est l'intensité.

\paragraph{Onde sinusoïdale} L'intensité moyenne d'une onde EM sinusoïdale vaut
\[I_m = \frac{\epsilon c E^2_{max}}{2}\]

\subsection{Ondes mécaniques}
Il est important, pour commencer, d'insister sur le fait
que les ondes transportent de l'énergie et de la quantité de mouvement,
et non de la matière d'un point à un autre.

\subsubsection{Corde}
On peut appliquer une onde transverse dans une corde.
Si la corde est tendue avec une force $F$ et que
sa masse linéique $[\kilogram\per\meter]$ est $m_L$,
la vitesse de propagation d'une telle onde est
\[ v = \sqrt{\frac{F}{m_L}}. \]

\subsubsection{Son}
Le son est une onde longitudinale de compression de l'air.
La vitesse de propagation du son dans l'air à \si{298}{K} et \si{1}{atm}
$v = \si{344}{\meter\per\second}$.

La vitesse du son varie avec la température, suivant la loi des gaz parfaits.
Nous avons donc la relation $v = \sqrt{\frac{B}{\rho_0}} \simeq 20.1\sqrt{T}$

Pour mesurer l'intensité d'une onde, on utilise une échelle logarithmique
\[ I[\deci\bel] = 10 \log_{10}\frac{I[\watt\per\meter\squared]}{I_0} \]
où $I_0 = \si{1e-12}{\watt\per\meter\squared}$.

\paragraph{Décibel}

Permet d'exprimer un rapport de puissances ou d'intensités
(pas uniquement du son !)
\[\text{rapport [dB] } = 10 \log_{10}\frac{I_1}{I_2}\]
Pour un rapport d'amplitudes, tensions\dots \(A\) tel que \(I \propto A^2\) :
\[\text{rapport [dB] } = \strong{20} \log_{10}\frac{A_1}{A_2}\]

\subsection{Effet Doppler}
Soit une source se déplaçant à une vitesse $v_s$ et émettant
une onde de vitesse $v$ et de fréquence $f_s$.
Soit une observateur se déplaçant à une vitesse $v_o$ et
observant cette onde à une fréquence $f_o$.

On suppose que les trois vitesses sont parallèles.

\subsubsection{Pour une onde non-électromagnétique}
Si on a $v \ll c$, on peut dire que
\[ f_o = \frac{v \pm v_o}{v \pm v_s} f_s \]

Il faut utiliser le bon sens pour savoir si c'est un plus ou un moins.
Par exemple, si la source se rapproche de l'observateur, et que
l'observateur se rapproche de la source, les deux vitesses
tendent à augmenter la fréquence perçue par l'observateur donc
\[ f_o = \frac{v + v_o}{v - v_s} f_s. \]

\subsubsection{Pour une onde électromagnétique}
Lorsqu'on applique l'effet Doppler aux ondes électromagnétiques,
on ne s'intéresse plus à la vitesse de l'observateur et de la source séparément
mais à leur vitesse relative $u$ avec $u$ \emph{positif} s'ils se rapprochent
et \emph{négatif} s'ils s'éloignent.

La relativité nous permet de montrer que
\[ f_o = \sqrt{\frac{c + u}{c - u}}f_s. \]
Si $u \ll c$, on a
\[ \frac{f_o - f_s}{f_s} \approx \frac{u}{c}. \]

\paragraph{Attention} Parfois, il faut appliquer l'effet Doppler deux fois
comme pour un radar où l'objet est d'abord observateur et le radar la source
avant que les rôles ne s'inversent.
On a alors
\[ f_o = \frac{c + u}{c - u}f_s \]
et si $u \ll c$
\[ \frac{f_o - f_s}{f_s} \approx \frac{2u}{c}. \]

\section{Polarisation, réflexion et réfraction}

\subsection{Polarisation}
Le type de polarisation indique la forme du lieu parcouru
par l'extrémité du vecteur (souvent champ électrique) associé à un
point de l'espace.

Supposons que l'onde se déplace selon $\vec{z}$,
une polarisation générale serait
\[ \vec{E}(\vec{r}, t) = A_x \sin(\vec{k}\cdot\vec{r} - \omega t + \phi_1)
  \xunit
+ A_y \sin(\vec{k} \cdot \vec{r} - \omega t + \phi_2) \yunit \]
C'est ce qu'on appelle on polarisation elliptique.
Il y a deux cas dégénérés:
\begin{itemize}
  \item Si $\phi_2-\phi_1 = \pm \frac{\pi}{2}$
    et $A_x = A_y$, c'est une polarisation circulaire;
  \item Si $\phi_2-\phi_1 = 0$ ou $\pm\pi$ ou si $A_x = 0$ ou $A_y = 0$,
    c'est une polarisation linéaire.
\end{itemize}

\subsection{Réflexion et réfraction}
Lorsqu'une onde passe d'un milieu à un autre,
elle se réfléchit en une onde \emph{réfléchie} et se réfracte en une onde
\emph{transmise} à la surface de séparation des deux milieu.

\paragraph{Attention}
Une onde électromagnétique ne passe pas à travers une membrane métallique,
elle y est réfléchie.

\subsubsection{Réflexion}

Lors de la réflexion, les paramètres $\omega$, $f$, $v$ et $\lambda$
sont identiques à ceux de l'onde incidente.
L'angle de réflexion est égal à l'angle d'incidence.

L'onde réfléchie n'est pas toujours en phase avec l'onde incidente.
Elle est soit en phase, soit déphasée de $\pi$.

En fait, si le coefficient donné par l'équation de Fresnel est positif,
elle est en phase, sinon elle est déphasée (c.-à-d. en opposition de phase).
C'est à dire que:
\begin{itemize}
  \item Si $n_1 < n_2$, la réflexion de la partie de l'onde
    qui est perpendiculaire au plan
    d'incidence est déphasée de $\pi$ avec l'onde incidente
    et celle de la partie parallèle est déphasée de $\pi$
    pour $\theta < \theta_b$;
  \item Si $n_1 > n_2$, la réflexion de la partie perpendiculaire est
    en phase et celle de la partie parallèle est déphasée de $\pi$
    pour $\theta_b < \theta$.
\end{itemize}

% What's this ?
%\begin{figure}
  %\centering
  %\begin{subfigure}[b]{0.45\textwidth}
    %\begin{tikzpicture}%[x=2cm,y=3cm]
      %\begin{axis}
        %\addplot[smooth, color=blue, domain=0:90]
        %{(cos(sin(2*sin(x))) - 2*cos(x))/
        %(cos(asin(2*sin(x))) + 2*cos(x))};
      %\end{axis}
    %\end{tikzpicture}
  %\end{subfigure}
  %\begin{subfigure}[b]{0.45\textwidth}
  %\end{subfigure}
%\end{figure}

\subsubsection{Réfraction}
L'onde transmise a la même vitesse angulaire $\omega$
et la même fréquence $\nu$ que l'onde incidente
mais pas la même vitesse $v$ ni la même longueur d'onde $\lambda$.

Elle est en phase avec l'onde incidente.

\paragraph{Indice de réfraction}
On définit les indices de réfraction comme suit
\[ \frac{n_1}{n_2} \eqdef \frac{v_2}{v_1} = \frac{\lambda_2}{\lambda_1} \]

Dans le cas d'une onde électromagnétique,
$v = c = \frac{1}{\sqrt{\perm\mu}}$.
En définissant $n_\mathrm{vide} = 1$,
on a alors
$n = \sqrt{\perm_r\mu_r}$.

\paragraph{Loi de Snell-Descartes}
L'angle transmis $\theta_2$ par rapport à l'angle d'incidence
$\theta_1$ nous est donné par la loi de \emph{Snell-Descartes}
\begin{equation}
  \label{eq:snell}
  n_1 \sin(\theta_1) = n_2 \sin(\theta_2).
\end{equation}

\subsubsection{Équations de Fresnel}
Pour trouver l'intensité de l'onde réfléchie et transmise, il nous
faut décomposer son intensité en une composante parallèle au plan d'incidence
$E^\parallel$ et une perpendiculaire $E^\perp$.

Soit $E_{1r}$ l'intensité du champ réfléchi
et $E_{2}$ l'intensité du champ transmis.
On a les formules suivantes
\begin{align*}
  E_{1r}^\parallel & = \frac{n_1\cos\theta_2 - n_2\cos\theta_1}
  {n_1\cos\theta_2 + n_2\cos\theta_1}E_{1}^\parallel
  \stackrel{\eqref{eq:snell}}{=}
  \frac{\tan(\theta_2 - \theta_1)}{\tan(\theta_2 + \theta_1)}E_1^\parallel
  & E_2^\parallel & = \frac{2n_1\cos(\theta_1)}
  {n_1\cos\theta_2 - n_2\cos\theta_1}E_{1}^\parallel\\
  E_{1r}^\perp & = \frac{n_1\cos\theta_1 - n_2\cos\theta_2}
  {n_1\cos\theta_1 + n_2\cos\theta_2}E_{1}^\perp
  \stackrel{\eqref{eq:snell}}{=}
  \frac{\sin(\theta_2 - \theta_1)}{\sin(\theta_2 + \theta_1)}E_1^\perp
  & E_2^\perp & = \frac{2n_1\cos(\theta_1)}
  {n_1\cos\theta_1 - n_2\cos\theta_2}E_{1}^\perp\\
\end{align*}

\paragraph{Angle de Brewster}
Il existe un angle pour lequel $E_{1r}^\parallel = 0$,
on l'appelle l'angle de Brewster et c'est l'angle $\theta_B$ tel que
\[ \tan\theta_B = \frac{n_2}{n_1}. \]

\paragraph{Angle critique}
Si $n_1 > n_2$, il existe un angle critique $\theta_{1c}$ tel que
$E_{2} = 0$ pour tout $\theta_1 \geq \theta_{1c}$.
C'est l'angle qui respecte
\[ \sin\theta_{1c} = \frac{n_2}{n_1}. \]

On voit bien ici pourquoi, si $n_1 < n_2$, cet angle n'existe pas.

\section{Interférence et diffraction}
Quand une onde traverse une fente ou un objet, elle subit une diffraction.
Si c'est un objet et non une fente, l'effet est le même,
on ne traitera donc que le cas de fentes.
On considère aussi que le point où on mesure l'onde est loin des fentes
par rapport à leur distance entre elles.

\subsection{Approximation de Fraunhofer}
\label{sec:fraunhofer}
Lorsqu'il faut estimer la différence de distance entre différentes
fentes ou antennes situées aux points $P_i$ émettant des ondes
et un point $P$ loin d'elles,
il est nécessaire de faire l'approximation de Fraunhofer qui
consiste à considérer que tous les vecteurs $\vec{P_iP}$
sont parallèles.
La distance supplémentaire d'une fente à l'autre est donc $d\sin\theta$
où $d$ est la distance entre deux fentes et $\theta$ l'angle
formé avec la perpendiculaire aux deux fentes.

\subsection{Interférence}
Si une onde arrive sur deux fentes de largeur négligeable séparées par
une distance $d$,
il y aura interférence constructive si $\exists n \in \mathbb{N}$ tel que
\[ d\sin\theta = n \lambda \]
et interférence destructive si $\exists n \in \mathbb{N}$ tel que
\[ d\sin\theta = \left(n+\frac{1}{2}\right) \lambda \].

Si on considère $N$ fentes de largeur négligeable séparées par une
distance $d$,
\[ I(P) \propto \frac{A^2}{R^2} \cdot
  \frac{\sin^2\left(\frac{N \pi d \sin\theta}{\lambda}\right)}
{\sin^2\left(\frac{\pi d \sin\theta}{\lambda}\right)} \]
où $R$ est la distance entre $P$ et la fente la plus proche.

Ce qui donne des minimas si
$\exists n \in \mathbb{N}_0$ non divisibles par $N$ tel que
\[ N d \sin \theta = n \lambda \]
et des maxima si $\exists n \in \mathbb{N}$ tel que
\[ d \sin \theta = n \lambda. \]

\subsubsection{Interférence sur un film mince}
Soit un film mince d'épaisseur $d$ illuminé par une onde
plane. Il y a ici 3 milieux à considèrer :

\begin{enumerate}
	\item Le milieu 0 (situé au dessus du film) d'indice
	de réfraction $n_0$, dans ce milieu l'onde a une longueur
	d'onde $\lambda_0$ ;
	\item Le milieu 1 (dans le film) d'indice de réfraction
	$n_1$, dans ce milieu l'onde a une longueur d'onde $\lambda_1$ ;
	\item Le milieu 2 (en dessous du film) d'indice de réfraction
	$n_2$, dans ce milieu l'onde a une longueur d'onde $\lambda_2$ ;
\end{enumerate}

Les différents angles d'incidences sont représentés sur la
figure \ref{fig:film-mince}.

%TODO figure à refaire si quelqu'un a le temps...
\begin{figure}[ht!]
	\centering
		\includegraphics[scale=1.0]{film-mince.png}
		\caption{Un film mince illuminé par une onde plane.}
		\label{fig:film-mince}
\end{figure}

Le déphasage $\Delta \Phi$ entre deux rayons
réfléchis (le rouge et le bleu sur la figure \ref{fig:film-mince})
est donné par

$$\Delta \Phi = \frac{4\pi dn_1}{n_0\lambda_0}\cdot\cos(\theta_1) + \alpha_1$$

où $\alpha_1$ vaut 0 ou $\pi$ selon le déphasage dû à la
réflexion ou à la réfraction aux interfaces (voir signes
des coéfficients des équations de Fresnel).
La couleur associé à $\lambda_0$ sera plus perçue si

$$\Delta \Phi = 2m\pi$$

ou sera moins perçue si

$$\Delta \Phi = (2m+1)\pi.$$

C'est en se basant sur ce principe que l'on peut
fabriquer des couches anti-reflets ou, au contraire,
des couches réflectrices.

\subsubsection{Figure d'interférences}
La superposition d'ondes donne lieu à une \emph{figure
d'interférences}. Sur ces figures d'interférence, on
peut distinguer les lignes nodales et les lignes anti-nodales.
Les lignes nodales (en jaune sur la figure \ref{fig:nodale})
représentent le lieu des interférences destructives tandis que
les lignes anti-nodales (en bleu sur la figure \ref{fig:antinodale})
représentent le lieu des interférences constructives.

\begin{figure}[ht]
	\centering
	\begin{subfigure}[b]{0.45\textwidth}
		\includegraphics[scale=0.8]{fig_interferences_nodales.png}
		\caption{Lignes nodales.}
		\label{fig:nodale}
	\end{subfigure}
	\begin{subfigure}[b]{0.45\textwidth}
		\includegraphics[scale=0.8]{fig_interferences_antinodales.png}
	\caption{Lignes anti-nodales.}
	\label{fig:antinodale}
	\end{subfigure}
	\caption{Figure d'interférence entre deux sources de même fréquence.}
\end{figure}

\subsubsection{Applications}
Les interférences permettent de redistribuer spatialement l'énergie,
et donc de diriger l'énergie vers certaines régions de l'espace.
Les interférences sont donc utilisées, par exemple pour les antennes
GSM.

%TODO : diagramme polaire (vraiment utile?),représentation complexe d'une onde,
%interféromètre de Michelson

\subsection{Diffraction}
Supposons maintenant que l'épaisseur $a$ des fentes n'est plus
négligeable.
\begin{itemize}
  \item Pour une fente rectangulaire (voir figure \ref{fig:diffraction1}),
    \[ I(P) \propto I_0 \times
      \frac{\sin^2\left(\frac{\pi a \sin\theta}{\lambda}\right)}
    {\left(\frac{\pi a \sin\theta}{\lambda}\right)^2}; \]
  \item et pour $N$ fentes (voir figure \ref{fig:diffractionN}),
    \[ I(P) \propto I_0 \times
      \frac{\sin^2\left(\frac{\pi a \sin\theta}{\lambda}\right)}
      {\left(\frac{\pi a \sin\theta}{\lambda}\right)^2} \times
      \frac{\sin^2\left(\frac{N \pi d \sin\theta}{\lambda}\right)}
    {\sin^2\left(\frac{\pi d \sin\theta}{\lambda}\right)}. \]
\end{itemize}
On a des minima si $\exists n \in \mathbb{N}_0$ tel que
\[ a \sin \theta = n\lambda. \]

Si $N \neq 1$, on a aussi les minimas et maximas qu'avait
l'interférence à $N$ fentes.
Sauf bien sûr, les maximas pour lesquels $a\sin\theta = n\lambda$.
Par exemple, si $a = \frac{d}{2}$, tous les maximas d'ordre pairs
seront \emph{éteints}.

\begin{figure}[ht!]
	\centering
	\includegraphics[scale=0.6]{diffraction1.jpg}
	\caption{Intensité d'une onde diffractée par une fente.}
	\label{fig:diffraction1}
\end{figure}

\begin{figure}[ht!]
	\centering
	\includegraphics[scale=0.6]{diffraction3.jpg}
	\caption{Intensité d'une onde diffractée par $N$ fentes.}
	\label{fig:diffractionN}
\end{figure}

%TODO application (mesure d'un objet de très petite taille),
%pouvoir de résolution

\section{Ondes stationnaire}
Une onde stationnaire est une onde pour laquelle ses nœuds ne bougent pas.
Elle est obtenue en fixant des nœuds ou des ventres à ses extrémités.
Son équation est la suivante
\[ \xi(x, t) = A \sin(kx) \cos(\omega t). \]

En général, un système supporte plusieurs modes\footnote{Dans un système
oscillatoire, un mode propre d'oscillation est une des fréquences auxquelles
un système excitable peut osciller après avoir été perturbé au voisinage
de son état d'équilibre stable.}, on a donc

$$\xi(x, t) = \sum_{m=1}^{\infty} A_m \sin(k_mx) \cos(\omega_m t).$$

%TODO modulation d'amplitude

\subsection{Battements}
La somme de deux ondes de fréquences différentes (mais proches)
$\omega_1$ et $\omega_2$ donne des battements (interférences se
déplaçant dans l'espace) La fréquence des battements
est donnée par

$$\frac{|\omega_1 - \omega_2|}{2}$$

Le principe de battements est illustré à la figure \ref{fig:battements}.
La somme des deux ondes (de \unit{30}{\hertz} et \unit{32}{\hertz})
est représentée en bleue, l'\emph{enveloppe} de leur somme est representée en rouge.
Sur la figure, on peut remarquer que la fréquence de l'enveloppe (qui est la
fréquence de battements), est bien $\frac{32-30}{2} = \unit{1}{\hertz}$.

\begin{figure}[ht!]
	\centering
	\includegraphics[scale=0.8]{battements.png}
	\caption{Illustration du principe de battements.}
	\label{fig:battements}
\end{figure}

\section{Émission}
L'émission d'une onde EM est obtenue en accélérant deux charges.
Lorsque ces charges changent de vitesse, elles émettent une onde, qui sera
parallèle à leur accélération.

Ce principe est utilisé dans les antennes élémentaires.

\subsection{Antennes}
Les antennes sont modélisées par des segments dans lequel un courant
alternatif circule.
Supposons que le segment soit aligné avec l'axe $z$.
Il produit alors une onde électromagnétique partout dans le plan $xy$ mais
dont l'amplitude est inversement proportionnelle à $R$ et l'intensité
est inversement proportionnelles à $R^2$.

Ce type d'antenne est une antenne dite demi longueur d'onde, elle émet
donc une onde dont $\lambda = h/2$.

Pour analyser l'interférence de plusieurs antennes,
on fait l'approximation de Fraunhofer, voir section~\ref{sec:fraunhofer}.

%TODO CM8

%%%%%%%%%%%%%%%%%%%%%%%%%%%%
%	    QUANTIQUE          %
%%%%%%%%%%%%%%%%%%%%%%%%%%%%

\part{Physique quantique}
\section{Aspects corpusculaires des ondes électromagnétiques}
% TODO : rayonnement du corps noir
\subsection{L'effet photo-électrique}
Une surface absorbant une lumière peut ``éjecter'' des électrons,
on appelle ça l'effet \emph{photo-électrique}.
Pour observer l'effet photo-électrique, considérons l'expérience
suivante. On place un système cathode/anode dans le vide. Ce système
est relié à un circuit électrique via lequel on peut mesurer le
photo-courant. La différence de potentiel entre l'anode et la
cathode est notée $V_{AC}$. En éclairant la cathode, celle-ci émet
des électrons avec différentes énergies cinétiques $\leq K_{\text{max}}$.
On peut chercher à annuler ce photo-courant en applicant un potentiel
d'arrêt $V_0 = -V_{AC}$ au système anode/cathode. En faisant de la sorte
on annule l'énergie cinétique des électrons, $K_{\text{max}} = eV_0$.
Comparons maintenant les prédictions du modèle ondulatoire de la lumière
avec les observations expérimentales.

\begin{tabular}{p{0.45\textwidth}|p{0.45\textwidth}}
	\textbf{Prédictions du modèle ondulatoire} & \textbf{Observations expérimentales} \\
	\hline
	L'intensité d'une onde EM ne dépend pas de sa fréquence. Le photo-courant
	ne devrait donc pas dépendre de la fréquence de la lumière.
	& Le photo-courant dépend de la fréquence de la lumière. Pour un
	matériau donné, on observe le photo-courant qu'à partir d'une
	certaine \textbf{fréquence seuil}. \\
	\hline
	Comme l'énergie reçue par la cathode dépend de l'intensité de la lumière,
	on s'attend à ce que le potentiel d'arrêt augmente avec l'intensité
	de la lumière. De plus, comme l'intensité ne dépend pas de la fréquence,
	on s'attend à ce que le potentiel d'arrêt n'en dépende pas non plus.
	& Le potentiel d'arrêt ne dépend pas de l'intensité mais bien de la
	fréquence. Plus la fréquence de la lumière est élevée, plus le potentiel
	d'arrêt augmente (cela signifie que l'énergie du photo-électron éjecté
	est plus grande). En augmentant l'intensité de la lumière, on augmente juste
	le nombre d'électrons par seconde et donc le photo-courant.
\end{tabular}

\paragraph{L'explication d'Einstein}
La lumière est composée de photons.
Chaque photon voyage à la vitesse de la lumière $c$.
Son énergie est appelée quanta et vaut
\[ E = hf = \frac{hc}{\lambda} \]
où $h = \unit{6.626\cdot10^{-34}}{\joule\second}$ est
la constante de Planck.
Sa quantité de mouvement vaut
\[ p = mv = \frac{h}{\lambda} = \hbar k. \]

Ce postulat explique les observations expérimentales
faites précédemment :
\begin{enumerate}
	\item	Pour qu'un électron soit éjecté du matériau,
	l'énergie $hf$ apportée par le photon à
	l'électron doit être supérieure au travail d'extraction
	$\phi$ de l'électron dans ce matériau. On comprend donc
	l'existence d'une fréquence seuil $f = \frac{\phi}{h}$.
	\item Une plus grande intensité de la lumière signifie
	un plus grand nombre de photons par seconde absorbé,
	donc un plus grand nombre d'électrons par seconde éjecté,
	et donc un plus grand photo-courant.
	\item Enfin, on peut relier le potentiel d'arrêt à la
	fréquence en appliquant la conservation de l'énergie
	\[ K_{\text{max}} = eV_0 = hf - \phi .\]
\end{enumerate}

\subsection{La diffusion Compton}
Compton étudie la diffusion inélastique des rayons X
par la matière. On envoie de la lumière de longueur d'onde
$\lambda$ sur un électron au repos. On observe la longueur
d'onde $\lambda'$ de la lumière diffusée.

\begin{tabular}{p{0.45\textwidth}|p{0.45\textwidth}}
	\textbf{Prédictions du modèle ondulatoire} & \textbf{Observations expérimentales} \\
	\hline
	D'après la physique classique, la diffusion devrait être élastique, $\lambda = \lambda'$.
	& La lumière diffusée a une longueur d'onde différente, $\lambda' > \lambda$. On observe
	également que $\lambda'$ varie avec l'angle de de diffusion.
\end{tabular}

Le modèle des photons explique également ce caractère corpusculaire
de la lumière. En effet, en considérant les photons comme des
particules et en appliquant la conservation de la quantité
de mouvement, on trouve
\[ \lambda' -\lambda = \frac{h}{mc}(1-\cos\theta),\]
ce qui correspond à la formule trouvée expérimentalement.

\subsection{Le photon}
\subsubsection{Nature corpusculaire du photon}
\paragraph{Un photon est indivible}
Pour le démontrer, considérons
l'expérience illustrée à la figure \ref{fig:exp-photon1}.

\begin{figure}[ht!]
	\centering
	\includegraphics[scale=0.5]{exp_photon_1.jpg}
	\caption{Une source envoie de la lumière sur un miroir
	à travers une fente percée.}
	\label{fig:exp-photon1}
\end{figure}

En considérant les photons comme une onde, on peut prédire
que le faisceau se divisera en deux faisceaux d'intensité diminuée
de moitié et donc que l'énergie reçue à chaque cellule sera $\frac{1}{2}hf$.
En sachant que le seuil de sensibilité des cellules est $\approx hf$,
on ne devrait donc pas observer de courant. L'expérience
montre pourtant qu'un courant est créé sur les deux cellules mais
avec un nombre de photons par seconde diminué de moitié (par rapport
au cas ou tout le faisceau lumineux incident serait dirigé vers une seule
cellule). Ceci montre bien que le photon est indivisible.

\paragraph{Un photon conserve son énergie}
Pour le démontrer, considérons l'expérience illustrée
à la figure \ref{fig:exp-photon2}.

\begin{figure}[ht!]
	\centering
	\includegraphics[scale=0.5]{exp_photon_2.jpg}
	\caption{Un atome (donc une source de rayonnement
	électromagnétique) situé à une distance $R$ d'une cellule
	photo-électrique.}
	\label{fig:exp-photon2}
\end{figure}

L'expérience consiste à mesurer le courant
sur la cellule photo-électrique en fonction de $R$.
En considérant le photon comme une onde électromagnétique,
on peut prédire que son énergie diminue en $\frac{1}{R^2}$. On s'attend
donc à ce qu'on ne puisse plus mesurer de courant sur
la cellule à partir d'une certaine distance.
L'expérience montre, même pour des distances énormes,
qu'un courant est toujours produit sur la cellule.
Le nombre de photons par seconde arrivant sur la cellule
diminue cependant en $\frac{1}{R^2}$.

\subsubsection{Nature ondulatoire du photon}
Bien qu'il se comporte parfois comme une particule,
le photon se comporte aussi comme une onde. On peut
en effet observer des phénomènes de diffraction de photons
et d'interférences entre photons.

\section{Aspects ondulatoires des particules}
\subsection{Ondes de ``de Broglie''}
Chaque particule est caractérisée par une masse au repos $m > 0$
et une vitesse $v$.

Une onde de Broglie lui est associée et vaut
\[ \lambda = \frac{h}{mv} \]
Son énergie vaut
\[ E = hf = \hbar \omega \]

\subsection{Diffraction des électrons}
Les électrons ayant un caractère ondulatoire, il
est possible d'étudier la diffractions des électrons
ou encore les interférences entre électrons. C'est l'objectif
de l'expérience des fentes de Young. L'expérience de
Young et le mystère quantique qui en découle sont
expliqués dans la vidéo ``Double Slit Experiment explained
by Jim Al-Khalili'' dont l'adresse est disponible dans
l'annexe \ref{sec:videos}.

Une des applications de la diffraction des électrons est
le microscope électronique, qui peut offrir une résolution
de \unit{0.1}{\nano\meter} à \unit{10}{\nano\meter}.
% TODO : expliquer le fonctionnement des microscopes électroniques

\subsection{Fonction d'onde}
Une particule est liée à une fonction d'onde.
L'amplitude de cette fonction d'onde, i.e. $|\Psi|^2$, peut être interprété
comme la \emph{densité} de probabilité par unité de volume de trouver
la particule à l'endroit de l'espace et au moment où cette fonction
d'onde est calculée.
C'est à dire que la probabilité de trouver une particule
au temps $t$ dans un parallélépipède rectangle de côtés
$\dif x$, $\dif y$ et $\dif z$ au point de coordonnée $(x, y, z)$ est
\[ |\Psi(x, y, z, t)|^2 \dif x \dif y \dif z .\]
Il est important de se rappeler que
\[ |z|^2 = z^{*} \cdot z \]
où $z^{*}$ est le conjugué de $z$.

On peut effectuer une séparation de variable sur la fonction d'onde,
\[ \Psi(\vec{r}, t) = \psi_1(\vec{r}) \cdot \psi_2(t). \]

\paragraph{Particule non-localisée}
Dans le cas d'une particule non-localisée, on a
\begin{align*}
\psi_1(\vec{r}) &= C_1(k)\exp(i\vec{k}\cdot\vec{r})
& \psi_2(t) &= C_2(E)\exp(-i\frac{E}{\hbar}t)
\end{align*}
où $C_1(k)$ et $C_2(E)$ sont des constantes.
Il y a plusieurs façons de voir que ces fonctions
correspondent à une particule non-localisée :
\begin{itemize}
	\item $\psi_1$ et $\psi_2$ sont des sinusoïdes
	pures. On peut donc définir, respectivement,
	leur nombre d'onde $k$ et leur fréquence $f$.
	La quantité de mouvement (resp. l'énergie) de
	la particule est donc bien définie
	et on a $\Delta p = 0$\footnote{En effet,
	comme $p = \hbar k$, $\Delta k = 0$ implique $\Delta p = 0$.}
	(resp. $\Delta E = 0$). Par le
	principe d'incertitude d'Heisenberg, on a alors
	$\Delta x = \infty$ (resp. $\Delta t = \infty$) ;
	\item $|\psi_1|^2 = C_1(k)$ et $|\psi_2|^2 = C_2(E)$. Ce
	qui signifie que la particule peut se trouver
	n'importe où dans l'espace (resp. dans le temps). ;
	\item $\psi_1$ et $\psi_2$ sont des sinusoïdes
	s'étendant de $-\infty$ à $+\infty$.
\end{itemize}
En pratique, on a toujours une idée de la
position d'une particule, elle se trouve dans
un espace connu.

\paragraph{Particule localisée}
Dans le cas d'une particule localisée, on a
\begin{align*}
\psi_1(\vec{r}) &= \int_{-\infty}^{+\infty} C_1(k)\exp(i\vec{k}\cdot\vec{r})
& \psi_2(t) &= \int_{-\infty}^{+\infty} C_2(E)\exp(-i\frac{E}{\hbar}t).
\end{align*}
Dans ces deux expressions, l'intégrale exprime ce qu'on
appelle une transformée de Fourier. C'est à dire que
$\psi_1$ et $\psi_2$ sont des sommes infinies de sinusoïdes
d'amplitude $C_1(k)$ et $C_2(E)$ de tout nombres d'onde et
de toute fréquences (et donc d'énergie).
En additionnant plusieurs sinusoïdes de nombres d'onde
différents (resp. de fréquences différentes),
on localise la fonction d'onde dans l'espace (resp.
dans le temps). Ce concept est illustré à la figure
\ref{fig:wave-packets} pour $\psi_1$.

\begin{figure}[ht!]
	\centering
	\includegraphics[scale=0.85]{wave_packets.jpg}
	\caption{Explication du concept de paquet d'ondes.}
	\label{fig:wave-packets}
\end{figure}

$C_1(k)$ peut être interprété comme la densité de probabilité
de présence d'une onde de nombre d'onde $k$. De même $C_2(E)$
peut être interpreté comme la densité de probabilité de
présence d'une onde d'énergie $E$. Ces distributions
peuvent être représentées par des fonctions gausiennes.
La largeur des ces fonctions gausiennes représente la
gamme de valeurs que peut prendre $k$ ou $E$. Plus la
gausienne sera large, plus l'imprécision sur $k$ (resp.
sur $E$) sera grande, plus la particule sera localisée
dans l'espace (resp. dans le temps).
Dans le cas d'une particule non-localisée, la largeur
de ces fonctions gausiennes est infinement petite. Ainsi,
$C_1(k)$ et $C_2(E)$ sont nulles partout sauf en un seul
point, ce qui signifie que $k$
et $E$ ne peuvent prendre qu'une seule valeur.
Les intégrales de la transformée de Fourier
redeviennent alors bien les expressions du paragraphe
précédent sur les particules non-localisées.

\section{Principe d'incertitude d'Heisenberg}
\begin{mynota}
  Posons l'opérateur $\Delta$ comme
  la grandeur de l’intervalle d'incertitude
  d'une grandeur physique.
\end{mynota}

On remarque que lorsqu'on a la fonction d'onde $\Psi$,
on a toutes les grandeur physiques intéressantes de la particule.
Mais on remarque aussi qu'une même particule,
en plus d'avoir une incertitude sur la position et le temps,
peut avoir une énergie $E$, une fréquence $f$,
une longueur d'onde $\lambda$, une vitesse $v$,
une quantité de mouvement $p$ et un nombre d'onde $k$ incertains.

\subsection{Le principe d'incertitude d'Heisenberg pour la position}
En effet, si tous les $C_1(k)$ sont non nuls,
$k$ peut valoir la valeur qu'il veut,
c'est à dire que
$\Delta k = \Delta p = \Delta v = \infty$.
Par contre, si $C_1(k)$ est non nul seulement
pour $k \in [-1;1]$, on a que
$\Delta k = 2$, $\Delta p = 2\hbar$ et $\Delta v = \frac{2\hbar}{m}$.

En fait, si $\Delta k = 0$, on remarque que
\[ |\psi_1(\vec{r})|
= \left|C_1(k) \exp\left(i\vec{r} \cdot \vec{k}\right)\right|
= \left|C_1(k)\right|\cdot\left|\exp\left(i\vec{r} \cdot \vec{k}\right)\right|
= C_1(k) \]
c'est à dire que la probabilité de trouver la particule est la même
quelle que soit la position.
On a donc $\Delta x = \infty$, $\Delta y = \infty$ et $\Delta z = \infty$.

De même, si on impose $\Delta x = 0$, il faut que $\psi_1(\vec{r})$ soit nul
partout sauf en un point (la position de la particule). Comme $\psi_1(\vec{r})$
est une transformée de Fourier, il faudra intégrer les sinus
sur un intervalle de longueur infinie pour obtenir
$\psi_1$ tel que $\psi_1(\vec{r})$ vaille 0 partout sauf en un seul point
donc $\Delta p_x = \infty$.

Le principe d'incertitude d'Heisenberg pour la position nous dit que
\begin{align*}
  \Delta x \cdot \Delta p_x \geq \frac{\hbar}{2}\\
  \Delta y \cdot \Delta p_y \geq \frac{\hbar}{2}\\
  \Delta z \cdot \Delta p_z \geq \frac{\hbar}{2}.
\end{align*}

Ce n'est donc pas un hasard qu'on ne puisse pas connaître en même temps
la position de la particule avec précision ainsi que sa quantité de mouvement.
Il est important de bien comprendre que cette inégalité n'est pas due
à des défauts des appareils de mesures mais est liée à la nature
ondulatoire des particules.

\subsection{Le principe d'incertitude d'Heisenberg pour le temps}
Le même raisonnement marche aussi pour $\psi_2$.
Si on connaît $f$, c'est à dire
que $\Delta E = \Delta f = 0$, on a
\[ |\psi_2(\vec{r})|
= \left|C_2(E) \exp\left(-i\frac{E}{\hbar}t\right)\right|
= \left|C_2(E)\right|\cdot\left|\exp\left(-i\frac{E}{\hbar}t\right)\right|
= C_2(E). \]
C'est à dire que la particule n'est pas localisée dans le temps, $\Delta t = \infty$.

Si on veut que la probabilité que la particule soit là qu'à un temps précis,
c'est à dire que $\Delta t = 0$,
comme $\psi_2$ est une transformée de Fourier, il faudra intégrer les sinus
sur un intervalle de longueur infinie
pour obtenir $\psi_2$ tel qu'il soit non nul que pour un certain $t$.
Donc, on aura $\Delta E = \infty$.

Le principe d'incertitude d'Heisenberg pour le temps nous dit que
\begin{align*}
  \Delta t \cdot \Delta E \geq \frac{\hbar}{2}.
\end{align*}

La plupart du temps, on travaille dans le cas où
$\Delta t = \infty$ et l'énergie est dans un état défini $\Delta E = 0$
(état stationaire)\footnote{C'est le
cas d'un électron autour d'un noyau d'atome par exemple.}.
On a alors
\[ \Psi(\vec{r}, t) = \psi_1(\vec{r}) \exp\left(-i\frac{E}{\hbar}t\right), \]

\section{L'équation de Schrödinger à une dimension}
Soit une particule de masse $m$ et d'énergie $E$ dans un potentiel $V(\vec{r})$,
on a l'équation suivante
\begin{align*}
	& \left[-\frac{\hbar^2\lap}{2m} + V(\vec{r},t)\right]\psi(\vec{r},t) = i\hbar\fpart{\psi(\vec{r},t)}{t}
	& \text{(dépendante du temps)}.
\end{align*}
La valeur de $|\psi(\vec{r},t)|^2$ varie donc en fonction du temps.
Comme expliqué précedemment, dans un grand nombre de cas, l'énergie
est dans un état défini, c'est à dire qu'on a
\[ \Psi(\vec{r},t) = \psi_1(\vec{r})\exp{-\frac{iEt}{\hbar}}.\]
Dans ce cas, on constate assez facilement que
\[ |\psi(\vec{r},t)|^2 = |\psi_1(\vec{r})|^2.\]
En substituant l'équation d'onde stationnaire dans
l'équation de Schrödinger dépendante du temps, on obtient l'équation
de Schrödinger indépendante du temps
\begin{align*}
	& \left[-\frac{\hbar^2\lap}{2m} + V(\vec{r})\right]\psi_1(\vec{r}) = i\hbar\fpart{\psi_1(\vec{r})}{t}
	& \text{(indépendante du temps)}.
\end{align*}

Une dérivation simple de ces équations est donnée en annexe \ref{sec:deriv-eq}

\paragraph{Paradoxe du chat de Schrödinger}
La vidéo ``What Is the Wave Function? - Instant Egghead \#50'' dont
l'adresse est donnée dans l'annexe \ref{sec:videos} explique
ce paradoxe.

% TODO : particule libre non-localisée/localisée
\subsection{Puits de potentiel 1D infini}
\label{sec:puits-1d}
On va résoudre l'équation de Schrödinger dans un puits infini,
c'est à dire avec
\[ U(x) = \left\{
  \begin{aligned}
    0 & \text{ si } x \in [0; L]\\
    \infty & \text{ sinon}.
  \end{aligned}
\right. \]
Dans l'intervalle $[0; L]$, $\psi_1(x)$ satisfait
l'équation
\[ -\frac{\hbar^2}{2m}\ffdif{\psi_1(x)}{x} = E\psi_1(x).\]
On a donc $\psi_1(x) = Ae^{ikx}+Be^{-ikx}$ où $A$ et $B$
sont définies par les conditons aux limites
\[ \left\{
  \begin{aligned}
    \psi_1(0) &=& 0 & \Rightarrow B = -A \Rightarrow \psi_1(x) = C\sin(kx)\\
    \psi_1(L) &=& 0 & \Rightarrow k = \frac{n\pi}{L}
  \end{aligned}
\right. \]
avec $n = 1, 2, \ldots$. En réinjectant $\psi_1(x) = C\sin(kx)$
dans l'équation de départ, on trouve $E = \frac{\hbar^2k^2}{2m}$
et donc
\begin{align*}
  E_n & = \frac{n^2h^2}{8mL^2} & n = 1, 2, \ldots.
\end{align*}
Cela signifie que le spectre d'énergie est \emph{discret}.
Il ne nous reste plus qu'à éliminer la constante $C$.
Pour cela, on va utiliser le fait que la particule doit
bien se trouver quelque part et donc imposer
\[ \int_{-\infty}^{\infty} |\psi_1(x)|^2 \dif x = 1.\]
Cette propriété de \emph{normalisation} permet de trouver
$C$. On trouve donc finalement
\begin{align*}
  {\psi_1}_n(x) & =
  \sqrt{\frac{2}{L}}\sin\left(\frac{n\pi x}{L}\right) & n = 1, 2, \ldots.
\end{align*}

\begin{myrem}
	On remarque que le niveau fondamental d'énergie (correspondant
	à $n=1$) n'est pas nul, contrairement à celui d'une particule
	classique dans le même potentiel. On peut comprendre cela en
	se servant du principe d'incertitude d'Heinsenberg. En effet,
	comme la particule est confinée, on a une certaine précision
	sur sa position : $\Delta x$ n'est donc pas infini. $\Delta p$
	ne peut donc pas être nul, et la particule a une certaine énergie.
	On comprend aussi pourquoi l'énergie augmente quand la longueur
	du puits diminue. De manière générale, tout confinement
	spatial d'une particule quantique conduit à la quantification
	de son spectre d'énergie.
	Comme la fonction d'onde est sinusoïdale, on remarque aussi
	que la probabilité de présence $|{\psi_1}_n(x)|^2$ peut valoir
	0 à certains endroits, ce qui est contraire à la physique
	classique.
\end{myrem}

\subsection{Puits de potentiel fini}
% TODO : ajouter la résolution ? Sans doute un peu long
% et peut-être pas vraiment utile.
On résolvant l'équation de Schrödinger dans un puits
fini, c'est à dire avec
\[ U(x) = \left\{
  \begin{aligned}
    0 & \text{ si } x \in [0; L]\\
    U_0 & \text{ sinon}
  \end{aligned}
\right. \]
on remarque que $\psi_1(x)$ est non nul pour $x < 0$ et $x > L$
même s'il tend vers 0 tel une exponentielle, même si l'énergie de
la particule est inférieure à $U_0$. Ce phénomène,
appelée \emph{pénétration de barrière} est totalement
contraire à la physique classique!
L'énergie est à nouveau quantifiée et vaut
\[
  E_n = -\frac{\hbar^2\alpha_n^2}{2m}
\]
où $\alpha_n$ ne peut être déterminé que numériquement.

\begin{myrem}
	Cette situation correspond à un électron
	situé sur la surface d'un métal et qui a besoin
	d'une énergie $U_0$ pour s'en extraire.
\end{myrem}

\subsection{Puits de potentiel 3D infini}
On peut étendre la résolution de l'équation de Schrödinger
de la sous-section \ref{sec:puits-1d} à un cas 3D.
Soit une particule dans une boite de volume $V = L^3$,
avec
\[ U(x,y,z) = \left\{
  \begin{aligned}
    0 & \text{ si } x,y,z \in [0; L]\\
    \infty & \text{ sinon}.
  \end{aligned}
\right. \]
On trouve à nouveau que l'énergie est quantifiée
\begin{align*}
  E & = \frac{h^2(n_1^2+n_2^2+n_3^2)}{8mL^2} & n_1,n_2,n_3 = 1, 2, \ldots.
\end{align*}
La fonction d'onde générale est donnée par
\begin{align*}
  {\psi_1}_{(n_1,n_2,n_3)}(x) & =
  \sqrt{\frac{8}{L^3}}\sin\left(\frac{n_1\pi x}{L}\right)
											\sin\left(\frac{n_2\pi x}{L}\right)
											\sin\left(\frac{n_3\pi x}{L}\right)& n_1,n_2,n_3 = 1, 2, \ldots.
\end{align*}

On remarque qu'une particule peut avoir plusieurs états
quantiques distincts pour une même énergie. Par exemple,
les états $(n_1,n_2,n_3) = (2,1,1)$, $(1,2,1)$ et $(1,1,2)$
ont la même énergie mais des fonctions d'ondes différentes.
On appelle ça la \emph{dégénérescence}.
\subsection{Barrière de potentiel}
En résolvant l'équation de Schrödinger
aux alentour d'une barrière de potentiel,
c'est à dire avec
\[ U(x) = \left\{
  \begin{aligned}
    U_0 & \text{ si } x \in [0; L]\\
    0 & \text{ sinon}
  \end{aligned}
\right. \]
on remarque que même si $E < U_0$,
$\psi_1(x)$ est non-nul pour $x > 0$ et aussi pour $x > L$.
C'est à dire que la particule peut passer la barrière.
On appelle ça l'\emph{effet tunnel}.

En analysant le comportement de $\Psi$ en fonction du temps,
on remarque qu'il y a
une partie qui est réfléchie par la barrière de potentiel et
une partie qui passe à travers la barrière par l'effet tunnel.
Bien évidemment, plus $\frac{U_0}{E}$ est grand,
moins il y a de probabilité que l'électron passe à travers la barrière
et plus il y en a qu'il soit réfléchi.
A partir de la résolution de l'équation de Schrödinger, on
peut également calculer un coéfficient de transmission
\[ T = \frac{4E(U_0-E)}{4E(U_0-E)+U_0^2\sinh^2(\frac{L}{h}\sqrt{2m(U_0-E)})}.\]

\paragraph{Microscope à effet tunnel}
Une des applications de l'effet tunnel est le
microscope à effet tunnel. Son fonctionnement est expliqué dans
la vidéo ``Au-delà des nuages. Le microscope à effet tunnel''
dont l'adresse est donnée dans l'annexe \ref{sec:videos}.

\subsection{Oscillateur harmonique}
Le potentiel autour d'un atome en fonction de la distance
par rapport au noyau ressemble à une parabole.
C'est le potentiel d'un oscillateur harmonique.

En résolvant l'équation de Schrödinger avec ce potentiel,
on trouve que

\begin{align*}
  E_n & = \left(n+\frac{1}{2}\right)\hbar\omega & n = 0, 1, 2, \ldots.
\end{align*}

où $\omega = \sqrt\frac{k'}{m}$ et
$k'$ est la constante de raideur.

On peut remarquer que ces niveaux d'énergie sont équidistants et que
le niveau zéro est d'énergie non nulle.

Les fonctions de probabilité de présence sont symétriques pour les
niveaux impaires et antisymétriques pour les niveaux pairs.

\section{Atome}
Tout électron dans un atome voyage dans une orbitale caractérisée par
4 nombres quantiques: $n$, $l$, $m_l$ et $m_s$.
Ces nombres quantiques respectent
\begin{align*}
  n, l, m_l & \in \mathbb{Z}\\
  n & \geq 1\\
  0 & \leq l < n\\
  |m_l| & \leq l\\
  m_s & = \pm \frac12
\end{align*}

L'énergie d'un électron est donc
\[ E_n = -\frac{Z_\mathrm{eff}^2}{(4\pi\perm_0)^2}\frac{m_ee^4}{2n^2\hbar^2}
= -Z_\mathrm{eff}^2\frac{\si{13.6}{\electronvolt}}{n^2} \]
où $m_e$ est la masse d'un électron, $e$ sa charge et
$Z_\mathrm{eff}$ est le nombre de charge vue par un électron.
Par exemple si c'est le 12\ieme{} électron d'un atome de 15 protons,
$Z_\mathrm{eff} = 15 - 11 = 4$.

Le moment angulaire de son orbite vaut
\[ L = \sqrt{l(l+1)} \hbar \]
avec une composante en $z$ valant
\[ L_z = m_l \hbar. \]

Son moment angulaire vaut
\[ S = \sqrt\frac34 \hbar \]
avec une composante en $z$ valant
\[ S_z = m_s \hbar. \]

\subsection{Probabilité de présence}
Pour trouver la probabilité de trouver un électron dans un atome
à une distance $r$ quand on connaît $\Psi(\vec{r})$,
il ne faut \emph{pas} faire $|\Psi(\vec{r})|^2 \dif r$
car comme on calcule maintenant $\Psi(\vec{r})$ en fonction des 3 dimension,
c'est la densité par volume.
Comme on est dans le cas d'une sphère, la formule de la probabilité
$P(r)$ de trouver l'électron à une distance entre $r$ et $r+\dif r$ est
\[ P(r) \dif r = |\Psi(\vec{r})|^2 \dif V
= |\Psi(\vec{r})|^2 4\pi r^2 \dif r. \]

\subsection{Rayon de l'orbitale}
Le rayon de l'orbitale est obtenu par la formule suivante
\[ r = \frac{4\pi\perm_0\hbar^2}{m_ee^2}n^2. \]
où $m_e$ est la masse d'un électron et $e$ sa charge.

\subsection{Principe d'exclusion de Pauli}
Deux électrons ne peuvent pas avoir les mêmes 4 nombres quantiques
$n$, $l$, $m_l$ et $m_s$.
Ce principe n'est donc utile que dans le cas de plusieurs électrons.

\subsection{Effet photoélectrique}
Un électron peut changer d'orbite à deux conditions
\begin{itemize}
  \item Le principe d'exclusion de Pauli le permet;
  \item $n$ et $l$ changent exactement de 1.
    C'est à dire que $\Delta n = \pm 1$ et $\Delta l = \pm 1$.
\end{itemize}

Si $\Delta n = 1$, c'est qu'un photon de fréquence
$\nu = \frac{\Delta E_n}{h}$ a été absorbé par l'électron.

Si $\Delta n = -1$, l'électron émet un photon de fréquence
$\nu = \frac{\Delta E_n}{h}$.

\section{Liaisons atomiques}
Lorsque deux atomes se lient,
ils vibrent et tournent.
Ça crée une énergie vibrationnelle
\[ E_n = \left(n + \frac{1}{2}\right)\hbar\sqrt\frac{k'}{m_r} \]
et une énergie rotationnelle
\[ E_l = l(l+1) \frac{\hbar}{2m_rr_0^2} \]
où $r_0$ est la distance entre les deux atomes,
$k'$ est la constante de raideur de la vibration et
$m_r$ est la masse réduite des masses des deux atomes $m_1$ et $m_2$ valant
\[ m_r = \frac{m_1m_2}{m_1+m_2}. \]

\section{Solides}
Dans un solides,
les électrons sont mis en communs entre les différents atomes.
On remarque que les électrons ne peuvent avoir leur énergie que dans certains
intervalles qu'on appelle bande.
Le principe d'exclusion de Pauli s'applique ici aussi et donc certaines
bandes peuvent être remplies et ne plus accepter d'électrons.
La dernière bande remplie est appelée la bande de valence et la suivante
bande de conduction.
La longueur de la bande interdite entre les deux est appelée
band gap et est notée $E_g$.

S'il n'y a pas exactement le bon nombre d'électron pour
que la bande de valence soit remplie et que la bande de conduction soit
vide, il y en a dans la bande de conduction.
La séparation entre les bandes pleines et vides, et égale à $E_F$,
l'énergie de Fermi.
Les électrons présents dans une bande partiellement remplie sont libres de se
déplacer facilement dans cette bande. Le courant peut donc passer facilement.
On dit que le solide est un \emph{conducteur}.

Sinon, si $E_g$ est élevé, de l'ordre de 2 à \si{6}{\electronvolt},
on dit que c'est un \emph{isolant}.
Si $E_g$ est plus faible, on dit que c'est un \emph{semi-conducteur}.

\subsection{Semi-conducteur}
Pour les semi-conducteurs,
si la température passe au dessus de \si{0}{\kelvin},
il y a une probabilité que les électrons
se trouvent sur la bande de conduction,
un petit courant peut donc passer.

La distribution des électrons est donnée par la distribution de Fermi-Dirac:
\[ f(E) = \frac{1}{1+\exp\left(\frac{E-E_F}{k_BT}\right)}. \]
où $E_F$ est l'énergie de Fermi et est défini telle que
$f(E_F) = \frac{1}{2}$.
%TODO graph

\appendix
\section{Unités}
En physique quantique, il y a deux unités souvent utilisées:
\begin{itemize}
\item le Hartree ($1Ha =  4,36\cdot 10^{-18} J$)
\item le Alchtreum ($1 \dot{A} = 10^{-10} m$)
\end{itemize}

\section{Dérivation simple des équations de Schrödinger}
\label{sec:deriv-eq}
\paragraph{Indépendante du temps}
Premièrement, on calcule l'énergie de la particule
\[ E = E_{\text{cin}} + E_{\text{pot}} = \frac{mv^2}{2} + U = \frac{p^2}{2m} + U.\]
On considère ensuite une fonction d'onde quelconque $\psi = e^{i(kx-\omega t)}$
que l'on va dériver deux fois par rapport à $x$ :
\[ \fpart{\psi}{x} = ik\psi ,\]
\[ \ffpart{\psi}{x} = (ik)^2\psi. \]
Or, on a $p = \frac{h}{\lambda} = \hbar k$ et donc $k = \frac{p}{\hbar}$.
En substituant dans $\ffpart{\psi(x)}{x}$, on obtient
\[ \ffpart{\psi}{x} = -\frac{p^2}{\hbar^2} \Rightarrow p^2 =
-\frac{\hbar^2}{\psi}\ffpart{\psi}{x}.\]
Enfin, en substituant cette valeur de $p^2$ dans l'équation de l'énergie
multiplié par $\psi$, on obtient bien
\[ -\frac{\hbar^2}{2m}\ffdif{\psi}{x} + U(x)\psi = E\psi.\]

\paragraph{Dépendante du temps}
Premièrement, on calcule l'énergie $E = \hbar\omega = hf$ et on
considère une fonction d'onde quelconque $\psi = e^{i(kx-\omega t)}$
que l'on va dériver par rapport à $t$ :
\[ \fpart{\psi}{t} = -i\omega\psi \Rightarrow \omega = \fpart{\psi}{t}\frac{i}{\psi}.\]
En isolant $\omega$ dans l'équation de l'énergie, on obtient $\omega = \frac{E}{\hbar}$
que l'on peut égaler à l'autre valeur de $\omega$ trouvant en dérivant $\psi$ :
\[ E\psi = i\hbar\fpart{\psi}{t}.\]
Or, on a déjà trouvé une équation pour $E\psi$ lors de la dérivation de l'équation
de Schrödinger indépendante du temps, on obtient donc finalement
\[ i\hbar\fpart{\psi}{t} = -\frac{\hbar^2}{2m}\ffdif{\psi}{x} + U(x)\psi.\]

\section{Vidéos des CM de physique quantique}
\label{sec:videos}
\begin{itemize}
	\item Double Slit Experiment explained by Jim Al-Khalili :
	\url{http://youtu.be/A9tKncAdlHQ} ;
	\item Le yin et le yang (La dualité onde-particule) :
	\url{http://youtu.be/N968DgSVLkg} ;
	\item L'électron dans tous ses états :
	\url{http://youtu.be/9Uf_LNULgeo} ;
	\item Au-delà des nuages. Le microscope à effet tunnel :
	\url{http://youtu.be/HaqNSbQA0hs} ;
	\item What Is the Wave Function? - Instant Egghead \#50
	(Schrödinger's cat) : \url{http://youtu.be/aowYf44gDRY} ;
\end{itemize}
\end{document}
