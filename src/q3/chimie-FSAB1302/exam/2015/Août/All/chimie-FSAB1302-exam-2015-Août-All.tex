\documentclass[fr]{../../../../../../eplexam}

\usepackage{../../../../../../eplchem}
\usepackage{../../../../../../eplunits}

\newcommand{\potstd}[1]{E\std_{\text{éq},\text{#1}}}
\newcommand{\potstdce}{\potstd{cell}}
\newcommand{\potstdca}{\potstd{cathode}}
\newcommand{\potstdan}{\potstd{anode}}

\hypertitle{Chimie et chimie physique}{3}{FSAB}{1302}{2015}{Août}
{Louis Devillez}
{Hervé Jeanmart et Joris Proost}

\section{Question 1}
Écrivez pour la corrosion en milieu aqueux de l'argent(\ce{Ag}) en son ion \ce{Ag+}
\begin{enumerate}
	\item Toutes les demi-réactions cathodiques possibles 
	\item Toutes les réactions globales possibles
	\item Calculez en se basant sur l'échelle redox donnée, les forces électromotrices de ces réactions de corrosion en milieu acide sous conditions standards.
\end{enumerate}

\begin{solution}
	La réduction se déroule à la cathode donc voici l'ensemble des équations de réductions possibles:
	
	\begin{align*}
		\ce{O2(g) + 4H+(aq)} + 4e^- & = 2\ce{H2O(l)}\\
		\ce{O2(g) + 2H2O(l)} + 4e^- & = 4\ce{OH^-(aq)}
	\end{align*}
	
	L'équation d'oxydation du \ce{Ag} est la suivante:
	
	$$\ce{Ag} = \ce{Ag+} + e^- $$
	
	Les réactions complètes sont donc:
	\begin{align*}
		\ce{O2(g) + 4H+(aq)} + 4\ce{Ag} & = 2\ce{H2O(l)} + 4 \ce{Ag+}\\
		\ce{O2(g) + 2H2O(l)} + 4\ce{Ag} & = 4\ce{OH^-(aq)} + 4\ce{Ag+}
	\end{align*}
	
	    En utilisant l'équation d'oxydation de l'\ce{Ag} on arrive à
	    \begin{align*}
	    	\potstdce & = 1.23 - 0.8 = 0.43\\
	    	\potstdce & = 0.40 - 0.8 = -0.40.
	    \end{align*}
	    
	    On remarque donc qu'on ne peut utiliser que la première équation

\end{solution}
\section{Question 2}
Des expériences en laboratoire ont permis de démontrer le mécanisme réactionnel suivant en milieu aqueux:
\begin{enumerate}
	\item \ce{ClO-} + \ce{H2O} = \ce{HClO} + \ce{OH-} (équilibre rapide)
	
	\item \ce{HClO} + \ce{I-} $\rightarrow$ \ce{HIO} + \ce{Cl-} (très lente)
	\item \ce{HIO} + \ce{OH-} = \ce{IO-} + \ce{H2O} (équilibre rapide)
\end{enumerate}
\begin{itemize}
	\item Quel est la réaction globale ?
	\item Écrivez la loi de vitesse de la réaction globale fondée sur ce mécanisme réactionnel
	\item Comment change la vitesse de réaction si on augmente le pH ? Vérifiez votre réponse avec le principe de Le Chatelier.
\end{itemize}

\begin{solution}
	la réaction globale est la suivante:
	
	$$\ce{ClO-} + \ce{I-} \rightarrow \ce{Cl-} + \ce{IO-} $$
	
	\begin{align*}
		r_{réaction} &= r_2\\
		& =k_2[\ce{HClO}][\ce{I-}]\\
		& =\frac{k_2}{K_1}[\ce{I-}]\frac{[\ce{ClO-}]}{[\ce{OH-}]}
	\end{align*}
	
	si on augmente le ph la concentration en $[\ce{OH-}]$ augmente donc la vitesse doit diminuer ce qui est bien confirmé par Le Chatelier car si on augmente la concentration en \ce{OH-} on diminue celle de \ce{HClO} et donc on diminue la vitesse de réaction
\end{solution}

\section{Question 3}
Le rendement des moteurs à combustion interne croit avec la pression maximale du cycle. Pour augmenter celle-ci, on envisage de réaliser la compression dans 2 cylindres différents, un basse pression et un haute pression. La détente serait également en 2 étapes, d'abords dans le cylindre haute pression et ensuite dans le cylindre basse pression. Ce concept porte le nom de de ``split cycle engine''. On vous propose de calculer le rendement thermique d'un tel cycle qui, dans une forme simplifiée et idéalisée, est composée des transformations suivantes:
\begin{itemize}
	\item 1-2 compression isentropique
	\item 2-3 refroidissement intermédiaire isobare
	\item 3-4 compression isentropique
	\item 4-5 apport de chaleur isobare
	\item 5-6 détente isentropique
	\item 6-1 refroidissement isochore
\end{itemize}

Le rapport volumétrique de compression du premier cylindre $V_1 / V_2 = 5$. Celui du deuxième cycle est $V_3/V_4=11$. Entre les deux, le refroidissement intermédiaire permet de ramener la température à $\SI{350}{\kelvin}$ avant la seconde compression. L'apport de chaleur de la source chaude est équivalente à une combustion à charge diluée, soit $\SI{2200}{\kilo\joule\per\kilogram}$. La détente, 5-6, se fait dans les deux cylindres successivement. Le gaz qui parcourt le cycle est assimilé à de l'air (qui se compote comme un gaz parfais) dont les propriétés thermiques sont telles que $\gamma$ est constant et vaux 1.3
\begin{enumerate}
	\item Complétez le tableau suivant
	
	    \begin{center}
	    	\begin{tabular}{|c|c|c|c|c|}
	    		\hline
	    		&p$[\si{\bar}]$&$T[\si{\kelvin}]$&$V[\si{\meter^3}]$&$S-S_1 [\SI{}{\joule\per\kelvin}]$\\
	    		\hline
	    		1&1&300&0.001&\\
	    		\hline
	    		2&&&&\\
	    		\hline
	    		3&&&&\\
	    		\hline
	    		4&&&&\\
	    		\hline
	    		5&&&&\\
	    		\hline
	    		6&&&&\\
	    		\hline
	    	\end{tabular}
	    \end{center}
	
	\item Donenz le travail net du cycle
	\item donnez le rendement du cycle
\end{enumerate}

\begin{solution}
	Étant donné que l'on nous donne $\gamma$ il faut recalculer $c_v$ et $c_p$
	
	
	$$\gamma = \frac{c_p}{c_v} = \frac{c_v + R^*}{c_v} $$
	$$c_v = \frac{R^*}{\gamma - 1} = \SI{956}{\joule\per\kelvin\per\kilogram} $$
	$$ c_p = c_v + R^* = \SI{1244}{\joule\per\kelvin\per\kilogram} $$
	
	calculons notre masse
	
	$$ m = \frac{P V}{R^*T} = \SI{0.0012}{\gram} $$
	
	Ensuite remplissons notre tableau avec ce que l'on connait:
	
\begin{center}
    	\begin{tabular}{|c|c|c|c|c|}
    		\hline
    		&p$[\si{\bar}]$&$T[\si{\kelvin}]$&$V[\si{\meter^3}]$&$S-S_1 [\SI{}{\joule\per\kelvin}]$\\
    		\hline
    		1&1&300&0.001&0\\
    		\hline
    		2&$P_2$&&0.0002&0\\
    		\hline
    		3&$P_2$&350&11*$V_4$&\\
    		\hline
    		4&$P_4$&&$V_4$&\\
    		\hline
    		5&$P_4$&&&\\
    		\hline
    		6&&&0.001&\\
    		\hline
    	\end{tabular}
\end{center}

Avec la transformation 1-2 étant isentropique
$$P_2 = P_1 \left(\frac{V_1}{V_2}\right){\gamma} = 9.52 $$

Avec la transformation 3-4 étant isentropique
$$P_4 = P_2 \left(\frac{11*V_4}{V_4}\right)^{\gamma} = 273.21 $$

$$V_4 = \frac{mR^*T}{11*P} =  1.11 * 10^{-5}$$

Avec la transformation 4-5 étant un apport de chaleur isobare

$$T_5 = T_4 + dT = T_4 + \frac{Q}{c_p} =  2517$$

Avec la transformation 1-2 étant isentropique
$$P_6 = P_5 \left(\frac{V_5}{V_6}\right){\gamma} = 2.08$$

Nous avons maintenant 2 états sur 3 pour chaque ligne nous pouvons donc finir de compléter notre tableau:

\begin{center}
	\begin{tabular}{|c|c|c|c|c|}
		\hline
		&p$[\si{\bar}]$&$T[\si{\kelvin}]$&$V[\si{\meter^3}]$&$S-S_1 [\SI{}{\joule\per\kelvin}]$\\
		\hline
		1&1&300&0.001&0\\
		\hline
		2&9.52&571.2&0.0002&0\\
		\hline
		3&9.52&350&$1.23 * 10^{-4}$&$S_3$\\
		\hline
		4&273.21&910&$1.11 * 10^{-5}$&$S_3$\\
		\hline
		5&273.21&2517&$3.07 * 10^{-5}$&$S_5$\\
		\hline
		6&2.08&624.6&0.001&$S_5$\\
		\hline
	\end{tabular}
\end{center}

$$S_3 = mc_p \ln\left(\frac{T_f}{T_i}\right)) = -0.707$$
$$S_5 = S_3 mc_p \ln\left(\frac{T_f}{T_i}\right)) = -0.762$$
$$Q_1 = m*c_p \Delta T = -312$$
$$Q_2 = m*c_p \Delta T = 2321$$
$$Q_3 = m*cv \Delta T = -360$$
$$W_{tot} = -(Q_1 + Q_2 + Q_3) = 1648$$

$$\eta = \frac{W_{tot}}{Q_c} = 0.71$$
	
\end{solution}

\section{Question 4}
Le travail effectué ou reçu par les fluides s'exprime de manière différente suivant que le système est fermé ou ouvert. Dans le cas d'un système fermé, on a:
$$W =  -\int p \text{d}V$$

Tandis que pour  les systèmes ouverts, l'expression est

$$W_m = \int v\text{d}P$$
où il a été fait abstraction d'éventues autres termes dans les deux expressions ci-dessus.
\begin{enumerate}
	\item Justifiez physiquement la forme prise par l'expression du travail dans le cas des systèmes fermés. Quelles sont les hypothèses nécessaire pour que la définition soit correcte ?
	\item Réalisez le développement mathématique nécessaire pour établir l'expression du $W_m$ pour système ouvert à partir de celle développée pour le travail dans les systèmes fermés.
	\item Il n'y a pas de terme du type $p\text{d}V$ ou $v\text{d}P$ dans l'expression énergétique du travail moteur. Pourquoi ? Justifiez votre réponse sur base physique et non mathématique
\end{enumerate}

\begin{solution}
	\begin{itemize}
		\item Le travail correspond à l'intégrale de F selon le chemin parcouru. Pour que la définition soit correcte, il faut que la pression interne soit égale à la pression extérieure.
		\item Il faut ajouter du travail $\int p dV$ les travaux d'extraction et d'insertion
		$$w = p_1v_1 -p_2v_2 + \int pdV$$ 
		$$p_1v_1 -p_2v_2 = - \int d(pV) = - \int p dV = - \int V dp$$
		
		$$W_{tot} =  \int V dp$$
	\end{itemize}
\end{solution}


\end{document}
