\documentclass[fr]{../../../../../../eplexam}

\hypertitle{Chimie et chimie physique}{3}{FSAB}{1302}{2015}{Août}
{Martin Braquet}
{Hervé Jeanmart et Joris Proost}

\section{}

A) Trier les hydrocarbures ci-dessous selon la valeur absolue de leur chaleur standard de combustion:
$$C_2H_4-C_3H_8-C_2H_6-CH_4-C_3H_6$$
B) Même question pour les composées suivants:
$$ C_2H_6-CH_3COCH_3-CH_3CH_2OH-CH_4-CH_3COOH-CH_3OH  $$

\begin{solution}
$$ CH_4<C_2H_4<C_2H_6<C_3H_6<C_3H_8 $$
$$ |\Delta H^{\circ}_{comb}|=890<1411<1560<2058<2220 \quad [KJ/mol]$$
$$ CH_3OH<CH_3COOH<CH_4<CH_3CH_2OH<C_2H_6<CH_3COCH_3 $$
$$ |\Delta H^{\circ}_{comb}|=726<875<890<1368<1560<1790 \quad [KJ/mol]$$
Les valeurs sont obtenues en connaissant les enthalpies de formation de ces composés (ainsi que celle de $H_2O$ et de $CO_2$). En effet, voici le raisonnement pour la combustion de $CH_4$: 
$$CH_{4(g)}+2O_{2(g)} \rightarrow 2H_2O_{(l)}+CO_{2(g)}$$
avec 
$$\Delta H^{CO_2}_f=-393\quad[KJ/mol], \quad
\Delta H^{H_2O}_f=-286 \quad[KJ/mol] , \quad
\Delta H^{CH_4}_f=-75\quad[KJ/mol]$$
Cela donne ainsi
$$\Delta H^{\circ}_{comb}=2\Delta H^{H_2O}_f+\Delta H^{CO_2}_f-\Delta H^{CH_4}_f=-890\:[KJ/mol]$$
puisque l'enthalpie de formation de $O_{2(g)}$ (composé de base) est nulle.
\end{solution}
\section{}
Écrivez pour la corrosion en milieu aqueux de Ag en son ion $Ag^+$
\begin{itemize}
    \item 

A) \begin{enumerate}
    \item Toutes les demi-réactions cathodiques possibles
    \item Toutes les réactions globales possibles
    
\end{enumerate}
\item
B) Calculez en se basant sur l'échelle rédox donnée, les forces électromotrices de ces réactions de corrosion en milieu acide sous des conditions standards. 
\end{itemize}
\begin{solution}
La réaction anodique de l'argent est
$$ Ag^++e^- \Rightarrow Ag \qquad E^{\circ}_{éq}=0,8V $$
Les réactions cathodiques possibles sont toutes celles avec un $E^{\circ}_{éq}$ supérieur à celui de l'argent.
Par exemple pour l'eau,
$$O_2+4H^++4e^-\Rightarrow2H_2O \qquad E^{\circ}_{éq}=1,23V$$
La réaction est alors
$$  4Ag+O_2+4H^+\Rightarrow2H_2O+4Ag^+ $$
On obtient finalement le potentiel électrochimique de la cellule sous conditions standards:
$$ E^{\circ}_{éq,cell}= E^{\circ}_{éq,cath}-E^{\circ}_{éq,an}=1,23-0,8=0,43V$$
\end{solution}

\section{}


Des expériences en laboratoire ont permis de démontrer le mécanisme réactionnel suivant en milieu aqueux:
\begin{enumerate}
    \item $ClO^-+H_2O=HClO+OH^-$ (équilibre rapide)
    \item $HClO+I^- \Rightarrow HIO+Cl^-$ (très lente)
    \item $HIO+OH^-=IO^-+H_2O$ (équilibre rapide)
\end{enumerate}
\begin{itemize}
    \item Quelle est la réaction globale?
    \item Ecrivez la loi de vitesse de la réaction globale fondée sur ce mécanisme réactionnel
    \item Comment change la vitesse de réaction si on augmente le pH? Vérifiez votre réponse avec le principe de Le Chatelier.
\end{itemize}
\begin{solution}
Réaction globale:
$$ ClO^-+I^- \Rightarrow Cl^-+IO^-$$
La vitesse globale est donnée par la vitesse de réaction la plus lente:
$$ r=r_2=k_2[HClO][I^-]$$
On a
$$r_1^{->}=k_1[ClO^-][H_2O] \quad et \quad r_1^{<-}=k_1^{'}[HClO][OH^-]$$
$[H2O]$ n'intervient pas reéllement dans le mécanisme réctionnel puisque sa concentration est nettement supérieure à la concentration des autres réactifs, on parle de dégénérescence d'ordre. On peut donc poser $$k^*_1=k_1\:[H_2O]=55,5\:k_1$$
avec $k_1$ la constante cinétique de la réaction (1).\\ On a alors
$$ r_1^{->}=k_1[ClO^-][H2O]=k^*_1[ClO^-] $$
Puisque $r_1^{->}=r_1^{<-}$, $$[HClO]=\frac{k^*_1}{k_1^{'}}\frac{[ClO^-]}{[OH^-]}$$
La vitesse globale est donc
$$r=\frac{k_2k^*_1}{k_1^{'}}\frac{[ClO^-][I^-]}{[OH^-]}$$
Une augmentation du PH induit une augmentation des ions $OH^-$, et donc une diminution de la vitesse de réaction.\\
Une augmentation des produits dans la réaction 1 déplace l'équilibre vers la gauche (le principe de le Chatelier stipule qu'un système s'oppose toujours à une perturbation) et diminue donc la vitesse de la 2e réaction puisqu'il y aura moins de $HClO^-$ produit. 

\end{solution}

\section{}

Le rendement des moteurs à combustion interne croit avec la pression maximale du cycle. Pour augmenter celle-ci, on envisage de réaliser la compression dans 2 cylindres différents, un à basse pression et un à haute pression. La détente serait également en deux étapes, d'abord dans le cylindre haute pression et ensuite dans le cylindre basse pression. Ce concept porte le nom de "Split cycle engine". On vous propose de calculer le rendement thermique d'un tel cycle qui, dans une forme simplifiée et idéalisée, est composée des transformations suivantes:
\begin{center}
1-2) Compression isentropique\\
2-3) Refroidissement intermédiaire isobare\\
3-4) Compression isentropique\\
4-5) Apport de chaleur isobare \\
5-6) Détente isentropique\\
6-1) Refroidissement isochore
\end{center}
Le rapport volumétrique de compression du premier cylindre $V_1/V_2$   vaut 5. Celui du deuxième cycle $V_3/V_4$ vaut 11. Entre les deux, le refroidissement intermédiaire permet de ramener la température à 350K avant la seconde compression. L'apport de chaleur de la source chaude est équivalent à une combustion à charge diluée, soit 2200 KJ/Kg. La détente 5-6 se fait dans les deux cylindres successivement. Le gaz qui parcourt le cycle est assimilé à de l'air (se comportant comme un gaz parfait) dont les propriétés thermiques sont telles que $\gamma$ est constant et vaut 1,3.
\begin{enumerate}
    \item Complétez le tableau suivant
    \begin{center}
\begin{tabular}{|c|c|c|c|c|}
  \hline
  &$p [bar]$ & $T [K]$ & $V [m^3]$ & $S-S_1 [J/K]$\\
  \hline
  1 & 1 & 300 & $10^{-3}$ & \\
  2 &  &  &  & \\
  3 &  &  & & \\
  4 & & & &\\
  5 & & & &\\
  6 & & & &\\
  \hline
\end{tabular}
\end{center}
    \item Donnez le travail net du cycle
    \item Donnez le rendement du cycle
\end{enumerate}

\begin{solution}
\begin{center}
\begin{tabular}{|c|c|c|c|c|}
  \hline
  &$p [bar]$ & $T [K]$ & $V [m^3]$ & $S-S_1 [J/K]$\\
  \hline
  1 & 1 & 300 & $10^{-3}$ & 0\\
  2 & 8.1 & 487,1 & $2.10^{-4}$ & 0\\
  3 & 8.1 & 350 & $1,437.10^{-4}$ & -0,476\\
  4 & 183 & 718,6 & $1,306.10^{-5}$ & -0,476\\
  5 & 183 & 2487 & $4,52.10^{-5}$ & 1,31\\
  6 & 3,267 & 973,3 & $10^{-3}$ & 1,31\\
  \hline
\end{tabular}
\end{center}

\noindent Le nombre de moles parcourant le cycle est $n=0,04$ moles.\\
On cherche les capacités calorifiques:
$$ \frac{c_p}{c_v}=\gamma=1,3 \qquad c_p-c_v=R$$
$$ \Rightarrow \quad c_p=36 \: [J/mol.K] \qquad c_v=27,69\: [J/mol.K]$$
La ligne 2 est obtenue par la loi de Poisson:
$$ p_2=p_1(\frac{V_1}{V_2})^{\gamma}$$
La 3e ligne est obtenue uniquement par la loi des gaz parfaits puisque T et p sont donnés.\\ 
On connait $V_4$ par l'énoncé et on obtient $p_4$ par la loi de Poisson:
 $ p_4=p_3(\frac{V_3}{V_4})^{\gamma}$.\\
 On a 
 $$Q=2,2.10^6M_mn=2548,48\:J$$
 Pour la ligne 5, la chaleur fournie à pression constante équivaut au $\Delta H$:
 $$d H=d U+pd V+Vd p=\delta Q +\delta W+pd V+Vd p=\delta Q+Vd p=\delta Q$$
 puisque $d p=0$.
 Ainsi
 $$ Q_{p=cst}=\Delta H=nc_p\Delta T$$
 $$ \Delta T =T_5-T_4=\frac{Q}{nc_p}= 1768,5K$$
 On a donc:
 $$T_5=T_4+1768,5=2487K$$
 Enfin, on connait $V_6=V_1$, et $ p_6=p_5(\frac{V_5}{V_6})^{\gamma}$.
 On calcule les variations d'entropie:
 $$dS=\frac{\delta Q}{T}=\frac{dU+pdV}{T}=nc_v\frac{dT}{T}+nR\frac{dV}{V}$$
 $$\Delta S=nc_v\ln\frac{T_f}{T_i}+nR\frac{V_f}{V_i}$$
 On obtient ainsi $S_3$ et $S_5$.
 On calcule les travaux (les isentropiques sont des transformations avec $Q=0$):
 $$W_{1-2}=\Delta U=nc_v (T_2-T_1)=206,89J$$
 $$W_{2-3}=-p\Delta V=45,6J$$
 $$W_{3-4}=\Delta U=nc_v(T_4-T_3)=408,67J$$
 $$W_{4-5}=-p\Delta V=-588J$$
 $$W_{5-6}=\Delta U=nc_v(T_6-T_5)=-1668,3J$$
 $$W_{6-1}=0J$$
 Donc,
 $$W_{tot}=-1595J$$
 Finalement,
 $$\eta=\frac{|W_{tot}|}{Q_{4-5}}=\frac{1595}{2548,5}=62,8\%$$
\end{solution}

\section{}

Le travail effectué ou reçu par les fluides s'exprime de manière différente suivant que le système est fermé ou ouvert. Dans le cas d'un système fermé, on a:
$$W=-\int pdV$$
Tandis que pour les systèmes ouverts, l'expression est 
$$w_m=\int^2_1vdp$$
où il a été fait abstraction d'éventuels autres termes dans les deux expressions ci-dessus.
\begin{enumerate}
    \item Justifiez physiquement la forme prise par l'expression du travail dans le cas des systèmes fermés. Quelles sont les hypothèses nécessaires pour que la définition soit correcte?
    \item Réalisez le développement mathématique nécessaire pour établir l'expression du $w_m$ pour un système ouvert à partir de celle développée pour le travail dans les systèmes fermés.
    \item Il n'y a pas de terme du type $pdv$ ou $vdp$ dans l'expression du travail moteur. Pourquoi? Justifiez votre réponse sur base physique et non mathématique.
\end{enumerate}

\begin{solution}
A) L'intégrale signale qu'une augmentation de volume induit un travail. Le signe - signifie que le travail effectué sur le système est négatif s'il    est en expansion. Cela veut donc dire que le système effectue lui-même un travail sur l'extérieur. On considère que la transformation est réversible et par conséquent, que le pression externe est identique à la pression interne.\\ \\
B) Voir les pages 29 et 30 du syllabus de thermodynamique.\\ \\
C) Ces termes n'apparaissent pas car ils sont pris en compte dans le terme de variation d'enthalpie $\Delta h$. En plus, ces termes de travaux sont repris dans la chaleur échangée, q, qui affecte directement le contenu énergétique total du fluide.
\end{solution}

\section{QCM}

\begin{enumerate}
    \item A 300K et à pression atmosphérique, quelle est la vitesse quadratique moyenne des molécules de dioxyde de carbone?
    $$ 505[m/s]\qquad 412[m/s]\qquad 337[m/s]\qquad380[m/s]\qquad11[m/s] $$
    \item Une machine frigorifique travaille selon un cycle de Carnot inversé entre une température extérieure de 25°C et une température intérieure de -10°C. Quel le coefficient de performance de cette machine?
    $$0,12 \qquad 2,5 \qquad7,5 \qquad0,6 \qquad8,5$$
    \item Selon le second principe, pour une transformation quelconque, il est impossible de convertir totalement de la chaleur en travail.
    $$ vrai  \qquad faux$$
    \item Afin de mesurer le débit d'eau dans une conduite de 100 mm de diamètre, on réalise localement une contraction régulière avec un rapport de diamètre de 4. La chute de pression est de 2000 Pa. Quel est le débit dans la conduite?
    $$1[kg/s] \qquad 16 [kg/s] \qquad 2[kg/s] \qquad 200[L/s] \qquad 1000[L/s]$$
    \item Quelle est la définition du coefficient de Joule-Thomson?
\end{enumerate}

\begin{solution}
\begin{enumerate}
    \item 412 [m/s]
    $$ \overline{c^2}=\sqrt{\frac{3RT}{M_m}}\qquad \mbox{avec      } M_m=44*10^{-3} \quad [Kg/mol]$$
    \item 7,5
    $$COP_{frigo}=\frac{1}{\frac{T_H}{T_C}-1}$$
    \item Vrai
    \item 1 [kg/s]
   $$ c_1^2=\bigg(\frac{A_2}{A_1}\bigg)^2c^2_2=\bigg(\frac{R_2}{R_1}\bigg)^4c^2_2$$
   $$ \Delta p=-\frac{\rho}{2} \Delta c^2=-\frac{\rho}{2}\bigg(1-\Big(\frac{R_2}{R_1}\Big)^4\bigg)c^2_2 \Rightarrow c_2^2=\frac{-2\Delta p}{\rho \bigg(1-\Big(\frac{R_2}{R_1}\Big)^4\bigg)}=4,0159$$
   $$ \dot V=\pi R_2^2c_2=\pi*(0,05/4)^2*\sqrt{4,0159}=10^{-3}\:[m^3/s] $$
    \item $\left(\frac{\partial p}{\partial T}\right)_H$
\end{enumerate}
\end{solution}

\end{document}
