\documentclass[fr]{../../../../../../eplexam}

\usepackage{../../../math-FSAB1103-exam}

\hypertitle{Math\'ematique}{3}{FSAB}{1103}{2016}{Ao\^ut}
{William André}
{Jean-François Remacle et Grégoire Winckelmans}

\section{Question 1 : caractéristiques}
Soit l'EDP suivante :
\[ P\fpart{u}{x} + Q\fpart{u}{y} = R\]
et la courbe $\Gamma(s)$ définie par $x(s)=r_0\cos(s)$ et $y(s)=r_0\sin(s)$ \\
On sait que l'équation des caractéristiques est de la forme $r = L\theta + B(r_0,s)$ avec $x = r\cos(\theta)$ et $y=r\sin(\theta)$
\begin{enumerate}
	\item Déterminez $B(r_0,s)$. Dessinez le réseau des caractéritiques pour $0\leq r \leq 2r_0$ et $s=-\frac{\pi}{4};0;\frac{\pi}{4};\frac{\pi}{2}$
	\item Différentiez l'équation des caractéristiques pour obtenir la forme de $P(x,y)$ et $Q(x,y)$ \\ Rappel : $\frac{\mathrm d}{\mathrm dv} \arctan(v)=\frac{1}{(1+v^2)}$.
	\item En supposant $R=0$, déterminer la valeur de $u(2r_0,2r_0)$. En quelle(s) valeur(s) la solution n'est elle pas définie ?
\end{enumerate}


\begin{solution}
\end{solution}

\section{Question 2 : séparation de variables}
Soit l'EDP
\[ \fpart{u}{t} = \alpha\ffpart{u}{x} + Q_0\]
avec $\alpha > 0$ constant. Le domaine est borné : $0 \leq x \leq L$. La condition initiale est $u(x,0) = u_0(\frac{2x}{L}-1)$ avec $u_0 > 0$ constant. Pour $t>0$, les conditions aux limites sont $\dfrac{\partial u }{\partial x}(0,t)=\frac{2u_0}{L}$ et $u(L,t)=u_0$
\begin{enumerate}
	\item
	De quel type d'EDP s'agit il, physiquement et mathématiquement ? Qu'est ce qu'un temps ``court'' pour le problème présent, $0 < t \ll ...$ ?

	\item
	Obtenez ensuite, mathématiquement, la solution $u(x,t)$ du problème. $u(x,t) = R(x) + \Theta (x,t)$.
	\begin{itemize}
		\item Obtenez d'abord la solution en régime. ( $R(x)$ )
		\item Obtenez ensuite la solution transitoire ( $\Theta (x,t)$ ) grace a la méthode de séparation des variables.
		\item Soit $Q_0 = \dfrac{2\alpha \cdot u_0}{L^2}$, Esquissez le graphe attendu de $\frac{u}{u_0}$ en fonction de $\frac{x}{L}$ en des temps différents : $t=0$ (CI), temps court, temps moyen, temps long, temps très long (solution de régime).
	\end{itemize}
\end{enumerate}

\begin{solution}
\end{solution}


\section{Question 3 : théorie complexe}
On définit la fonction
\[ w = \frac{1}{i} \log{\left(\frac{\sqrt{z^2-1}+i}{z}\right)}\]
\begin{enumerate}
  \item Obtenez une expression pour $\sin(w)$ en fonction de $z$ (indice : l'expression est simple)
  \item Trouver les points de branchement ( indice: $\eta =\sqrt{z^2-1}+i$ est toujours positif). Effectuer les coupure de manière à ce que la fonction soit bien définie pour z= x avec x réel et |x|>1. Définir la branche principale , nous n'utiliserons plus que celle ci pour la suite.
  \item Obtenez l'expression de $w$ pour $z=x$ pour $x>1$. Obtenez d'abord l'expression de $\eta$.
  \item Obtenez l'expression complexe de $w$ pour $z=iy$ pour $y>0$. Obtenez d'abord l'expression de $\eta$.
  \item On peut montrer (ne le faites pas ici !) que $w = \int_0^Z{\frac{1}{1+\zeta^2}d\zeta}$ avec $Z=\frac{1}{z}$ \\Obtenez le développement en série de $w$ autour de $z_0 = 0$. Quel est le rayon de convergence de la série ?
    (\textbf{Aide}: $\frac{1}{\sqrt{1+z}} = 1 - \frac{1}{2}z^1 + \frac{1.3}{2.4}z^2 - \frac{1.3.5}{2.4.6}z^3 + ...$)
\end{enumerate}

\begin{solution}
\begin{enumerate}
	\item $\sin(w) = \frac{e^{wi}-e^{-wi}}{2i} = \frac{\frac{\sqrt{z^2-1}+1}{z}-\frac{z}{\sqrt{z^2-1}+1}}{2i} = \frac{z^2-1+2i\sqrt{z^2-1}-1-z^2}{2iz(\sqrt{z^2-1}+i)}=\frac{2i\sqrt{z^2-i}-2}{2i\sqrt{z^2-i}-2}\frac{1}{z}=\frac{1}{z}$
	\item
	\item \[\eta(x) = \sqrt{x^2-1}+i=xe^{i\tan\left(\frac{1}{\sqrt{x^2-1}}\right)}\]
		\[w(x) = \frac{1}{i} \log\left(\frac{xe^{i\tan\left(\frac{1}{\sqrt{x^2-1}}\right)}}{x}\right) = \tan\left(\frac{1}{\sqrt{x^2-1}}\right)\]
	\item \[\eta(iy) = \sqrt{(iy)^2-1}+i = i(\sqrt{y^2+1}+1)\]
		\[w(iy) = \frac{1}{i} \log\left(\frac{i(\sqrt{y^2+1}+1)}{iy}\right) = -i \ln\left(\frac{\sqrt{y^2+1}+1}{y	}\right)\]
	\item \[\frac{1}{\sqrt{1+\zeta^2}} = 1 - \frac{1}{2}\zeta^2 + \frac{1.3}{2.4}\zeta^4 - \frac{1.3.5}{2.4.6}\zeta^6 + ...\]
		\[w = \int_0^Z{\frac{1}{\sqrt{1+\zeta^2}}d\zeta} = \left[\zeta - \frac{1}{2}\frac{\zeta^3}{3} + \frac{1.3}{2.4}\frac{\zeta^5}{5} - \frac{1.3.5}{2.4.6}\frac{\zeta^7}{7} + ...\right]_0^{1/z}
		= \frac{1}{z} - \frac{1}{2}\frac{1}{3z^3} + \frac{1.3}{2.4}\frac{1}{5z^5} - \frac{1.3.5}{2.4.6}\frac{1}{7z^7} + ...\]
\end{enumerate}
\end{solution}

\section{Question 4 : Intégration complexe}
Obtenez la valeur de l'intégrale suivante
\[\int_0^\infty{\frac{x^2\log(x)}{(x^2+1)^2}dx}\] en utilisant les lemmes de Jordan (donnés à l'examen)

\begin{solution}
	Il y a deux poles d'ordre 2 en $\pm i$ et un en $0$ d'ordre 1. Nous choisissons donc un contour qui correspond à un demi cercle dans le plan supérieur avec une encoche autour de $0$.\\
	On peut montrer par les lemmes de Jordan que l'intégrale sur les demis arcs de cercle vaut 0. (à détailler)\\
	Il reste donc $\sum Res = \int_C{\frac{z^2\log(z)}{(z^2+1)^2}dz} = \int_0^\infty{\frac{x^2\log(x)}{(x^2+1)^2}dx} + \int_{-\infty}^0{\frac{x^2\log(x)}{(x^2+1)^2}dx}$\\
	Or, en posant $x=-y$ on trouve $\int_{-\infty}^0{\frac{x^2\log(x)}{(x^2+1)^2}dx} = \int_0^\infty{\frac{y^2(\ln(y)+i\pi)}{(y^2+1)^2}dy}$\\
	L'équation peut maintenant se réécrire \[2\int_0^\infty{\frac{x^2\log(x)}{(x^2+1)^2}dx} + i\int_0^\infty{\frac{\pi}{(x^2+1)^2}dx} = \sum Res\]
	Le seul résidu dans le contour est $i$. Il ne reste plus qu'à calculer $\sum Res = \frac{(2i\log(i)+1)(2i)^2-i^2\log(i).8i}{(2i)^4} = \frac{-8i\frac{i\pi}{2}-4i+8i\frac{i\pi}{2}}{16} = \frac{\pi}{4} - \frac{i}{4}$\\
	Par identification des parties réelles et imaginaires, on trouve \[\int_0^\infty{\frac{x^2\log(x)}{(x^2+1)^2}dx} = \frac{\pi}{4}\]
	  
\end{solution}

\end{document}
