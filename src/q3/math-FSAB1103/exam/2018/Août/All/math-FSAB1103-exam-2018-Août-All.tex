\documentclass[fr]{../../../../../../eplexam}

\hypertitle{Mathématiques}{3}{FSAB}{1103}{2018}{Août}{All}
{Jean-Martin Vlaeminck}
{Jean-François Remacle, Grégoire Winckelmans et Roland Keunings}

\section{EDP-caractéristiques}
On donne l'équation suivante sur le domaine \(\R^2\):
\[ y^{2n-1} \fpart{u}{x} - x^{2n-1} \fpart{u}{y} = U \frac{y^{4n-1}}{x^{2n+1}} \]
où \(n\) est un entier, \(U\) est une constante de même dimension que \(u\).

La caractéristique est telle que \(x(s) = s\), \(y(s) = s\) et \(u(s, 0) = 0\), pour \(s \ge 0\).

\begin{itemize}
	\item Obtenez les équations des caractéristiques, et tracez-les pour \(n=1\) et \(n=\infty\) pour les \(s\) suivants: \(0, 1\).
	\item Résolvez l'équation sur l'ensemble du domaine.
\end{itemize}

Petite note amusante: vous pouvez vérifier que les dimensions de chaque membre de cette équation sont cohérentes entre elles.

\begin{solution}
 TODO
\end{solution}

\section{EDP-séparation des variables}
On donne l'équation de Laplace sur un arc de cercle (secteur d'anneau, \( a \le r \le b \) et \( 0 \le \theta \le \beta \)):
\[ \nabla^2 u(r, \theta) = \frac{1}{r} \fpart{}{r} \left( r \fpart{u}{r} \right) + \frac{1}{r^2} \ffpart{u}{\theta} = 0 \]
Les conditions limites sont \(u(a, \theta) = u(r, 0) = 0 \),
\( \fpart{u}{\theta} (r, \beta) = \frac{U}{\beta} \log\left( \frac{r}{a} \right) \)
et \( \fpart{u}{r} (b, \theta) = h(\theta) \), avec \(h(\theta) = \constant g(\theta) \);
la valeur de la constante et la fonction \(g(\theta)\) sont en attente de confirmation.

\begin{itemize}
	\item Faites un dessin du domaine dans le plan \( (x, y) \) et \( (r, \theta) \)
	\item Résolvez cette équation par séparation des variables. Énoncez clairement les intégrales à évaluer en fonction de \(g(\theta) \)
\end{itemize}

\begin{solution}
TODO
\end{solution}

\section{Analyse complexe 1}
Soit la fonction
\[ f(z) = \frac{z}{e^{z^2} -1} \]
Calculez le résidu de cette fonction en \( z = 0 \).

Alternative: Calculez les résidus de cette fonction.

\begin{solution}
TODO
\end{solution}

\section{Analyse complexe 2}
Calculez l'intégrale suivante
\[ \int_{0}^{2\pi} \frac{\cos(x)}{5+4\cos(x)} \dif{x} \]

\begin{solution}
TODO
\end{solution}

\end{document}
