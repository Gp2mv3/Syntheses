\documentclass{article}

\usepackage[latin1]{inputenc} 
\usepackage[T1]{fontenc}      
\usepackage[francais]{babel} 

\title{Juin 2013 - Solution Q4}
\author{R�mi Dekimpe}
\date{30.12.2013}

\begin{document}
\maketitle

\section*{Question 4}
Calculez l'int�grale suivante
\[
\int_{0}^{\infty} \frac{cos(x)}{x^2+a^2} dx
\]
o� � est un r�el strictement positif par la m�thode des r�sidus.

\subsection*{Solution}
On va consid�rer pour cela l'int�grale
\[
\oint_C f(z) dz =\oint_C \frac{e^{iz}}{z^2+a^2} dz
\]
que l'on va int�grer sur un demi-cercle C dans la moiti� sup�rieur du plan complexe, dont on va faire tendre le rayon R vers l'infini.
\paragraph{} Le d�nominateur de $f(z)$ poss�de 2 racines: $ia$ et $-ia$. La racine $ia$ se trouve � l'int�rieur du contour ($a<R$ car on fait tendre R vers l'infini) et c'est donc un p�le d'ordre 1. La m�thode des r�sidus nous donne donc
\[
\oint f(z) dz=2\pi i \cdot Res(f,ia)=2\pi i\cdot g(ia) = \frac{\pi}{ae^{a}}
\]
o�
\[
g(z)=(z-ia)\cdot f(z)=\frac{e^{iz}}{z+ia}
\]

\paragraph{} On peut d�composer l'int�grale sur le contour en une somme d'int�grales pour faire appara�tre l'int�grale que nous devons calculer
\[
\oint_C f(z) dz=\int_0^{R} f(x)dx + \int_0^{\pi} f(Re^{i\theta})Re^{i\theta}id\theta+ \int_{-R}^0 f(x)dx
\]
\paragraph{} On peut annuler la deuxi�me int�grale gr�ce au deuxi�me Lemme de Jordan lorsque R tend vers l'infini. En effet,
\[
f(z)=\frac{1}{z^2+a^2} e^{iz}=h(z)e^{iz}
\]
Or
\[
\lim\limits_{z \to \infty} h(z)=\lim\limits_{z \to \infty} \frac{1}{z^2+a^2}=0
\]
Donc
\[
\int_0^{\pi} f(Re^{i\theta})Re^{i\theta}id\theta=0
\]
\paragraph{} Il nous reste les 2 int�grales le long de l'axe r�el.
\[
\int_0^{\infty} \frac{e^{ix}}{x^2+a^2} dx + \int_{-\infty}^0 \frac{e^{ix}}{x^2+a^2}dx
\]
\[
=\int_0^{\infty} \frac{e^{ix}}{x^2+a^2} dx+ \int_0^{\infty}\frac{e^{-ix}}{x^2+a^2} dx
\]
\[
=\int_0^{\infty} \frac{e^{ix}+e^{-ix}}{x^2+a^2} dx
\]
\[
=2\int_0^{\infty} \frac{cos(x)}{x^2+a^2} dx
\]
Ce qui nous donne bien l'int�grale voulue.
\paragraph{} On obtient donc
\[
\frac{1}{2}\int_0^{\infty} \frac{cos(x)}{x^2+a^2} dx=\frac{1}{2}\oint_C \frac{e^{iz}}{z^2+a^2} dz=\frac{\pi}{2ae^{a}}
\]
\end{document}