\documentclass[fr]{../../../../../../eplexam}

\usepackage{../../../../../../eplunits}
\usepackage{../../../../../../eplchem}
\DeclareSIUnit\molar{\textsc{m}}
\newcommand*\diff{\mathop{}\!\mathrm{d}}

\hypertitle{Projet FSAB1503}{3}{FSAB}{1503}{2017}{Janvier}
{Martin Braquet \and Gilles Peiffer}
{Juray de Wilde}

\section{}

Une usine pétrochimique dégage du dioxyde de carbone, composé excédent suite à un ensemble de réactions chimiques. Pour limiter son dégagement dans l'air, une colonne d'absorption est installée. Elle mesure $\SI{8}{\metre}$ de hauteur et \SI{2}{\metre} de diamètre. Le liquide présent dans la colonne est une solution de $\ce{NaOH} (\SI{1.1}{\molar})$. Son débit est de $\SI{3500}{\cubic\metre\per\hour}$. Le débit des gaz entrant dans la colonne est de $\SI{40 000}{\cubic\metre\per\hour}$. La concentration de la solution à la sortie de la colonne n'est plus que de $\SI{1.0}{\molar}$. La surface interfaciale de contact entre la phase gaz-liquide est de $\SI{440}{\meter\squared\per\cubic\meter}$ de volume. La réaction d'absorption se fait à température ambiante. Le coefficient de transfert de masse en phase gazeuse $k_{\textnormal{A}_\textnormal{g}}$ vaut $\SI{181}{\mol\per\hour\per\square\metre\per\bar}$. On vous propose un raisonnement par étapes pour déterminer si le débit de sortie de $\ce{CO2}$ est suffisamment faible pour respecter les normes environnementales (max. $\SI{1000}{\cubic\metre\per\hour}$).

Vous pouvez faire les hypothèses que 
\begin{itemize}
    \item $C_{\ce{NaOH}}\ll C_{\textnormal{u}}$ et $p_{\ce{CO2}}\ll p_{\textnormal{u}}$ (u étant un composé inerte)
    \item La réaction est contrôlée par la résistance en phase gazeuse: $$(-r_{\textnormal{A}})=k_{\textnormal{A}_{\textnormal{g}}}p_{\textnormal{A}} \mbox{   avec   } A = \ce{CO2}$$
\end{itemize}
\begin{enumerate}
    \item 
        Calculez:
        \begin{itemize}
            \item $G$: le flux molaire de gaz inertes par surface de section de la colonne ($\si{\mol\per\hour\per\meter\squared}$);
            \item $L$: le flux molaire de composés inertes dans la phase liquide par surface de section de la colonne ($\si{\mol\per\hour\per\meter\squared}$);
            \item $C_{\textnormal{u}}$: la concentration des composés inertes dans le liquide ($\si{\mol\per\litre}$).
        \end{itemize}
        
    \item 
    Que vaut $p_{\ce{CO2}}^{\textnormal{in}}-p_{\ce{CO2}}^{\textnormal{out}}$?
    \item 
    Que vaut $p_{\ce{CO2}}^{\textnormal{out}} / p_{\ce{CO2}}^{\textnormal{in}}$?
    \item 
    À partir de ce système de 2 équations à 2 inconnues, calculez $p_{\ce{CO2}}^{\textnormal{out}}$.
    \item 
    Que vaut la pression maximale à l'entrée, $p_{\ce{CO2}\textnormal{,max}}^{\textnormal{in}}$?
    \item 
    L'approximation de la vitesse réactionnelle est-elle correcte dans toute la colonne?
    \item 
    L'industrie respecte-t-elle les normes environnementales?

\end{enumerate}

\begin{solution}

\begin{enumerate}
    \item 
        
        Le flux molaire de gaz inertes vaut
        $$G=\frac{\dot V_\textnormal{g} \:p}{RT} \cdot \frac{1}{S}=\frac{\num{40000e5}}{\num{8.3145} \cdot \num{298} \cdot \pi R^2} = \SI{5.139e5}{\mol\per\hour\per\metre\squared}$$
        
        La concentration des composés inertes dans le liquide correspond à la concentration d'$H_2O$ dans l'eau:
        $$C_{\textnormal{u}} = \SI{55.56}{\molar}$$
        
        Le flux molaire de composés inertes dans la phase liquide vaut
        $$L = \frac{C_{\textnormal{u}} \dot V_\textnormal{l}}{S} = \frac{\num{55.56} \cdot \num{3500e3}}{\pi R^2} = \SI{6.189e7}{\mol\per\hour\per\metre\squared}$$
    
    \item 
    
        Puisque la réaction consomme une mole de $\ce{CO2}$ pour $\num{2}$ moles de $\ce{OH-}$, on a
        $$ \num{2}(\dot n_{\ce{CO2}}^{\textnormal{in}} - \dot n_{\ce{CO2}}^{\textnormal{out}})= \dot n_{\ce{OH-}}^{\textnormal{in}} - \dot n_{\ce{OH-}}^{\textnormal{out}} $$
        $$ \frac{\dot V_\textnormal{g}}{RT}(p_{\ce{CO2}}^{\textnormal{in}} - p_{\ce{CO2}}^{\textnormal{out}}) = \frac{\dot V_\textnormal{l}}{\num{2}}(C_{\ce{OH-}}^{\textnormal{in}} - C_{\ce{OH-}}^{\textnormal{out}}) $$
        
        
        \begin{eqnarray*}
\implies p_{\ce{CO2}}^{\textnormal{in}} - p_{\ce{CO2}}^{\textnormal{out}} & = & \frac{RT\dot V_\textnormal{l}}{\num{2} \dot V_\textnormal{g}}(C_{\ce{OH-}}^{\textnormal{in}} - C_{\ce{OH-}}^{\textnormal{out}}) \\
& = & \frac{\num{8.3145} \cdot \num{298} \cdot \num{3500e3}}{\num{2} \cdot \num{40000}} \cdot (\num{1.1} - \num{1})\\
& = & \SI{0.1084}{\bar}
\end{eqnarray*}
    
    \item 
    
        On utilise la formule (fournie en annexe) donnant la hauteur de la colonne pour des concentrations données. Il faut la manipuler pour exprimer le rapport des pressions en fonction de la hauteur.
        \begin{eqnarray*}
h & = & \int_{Y_{\textnormal{A,out}}}^{Y_{\textnormal{{A,in}}}} \frac{G \diff{Y_\textnormal{A}}}{(-r_{\textnormal{A}})\: a} \\
& = & \int_{Y_{\textnormal{A,out}}}^{Y_{\textnormal{{A,in}}}} \frac{G \diff{Y_\textnormal{A}}}{k_{\textnormal{A}_{\textnormal{g}}} p_{\textnormal{A}}\: a}\\
& = & \frac{G}{k_{\textnormal{A}_{\textnormal{g}}} p_\textnormal{u}\: a} \int_{Y_{\textnormal{A,out}}}^{Y_{\textnormal{{A,in}}}} \frac{\diff{Y_{\textnormal{A}}}}{Y_{\textnormal{A}}}\\
& = &  \frac{G}{k_{\textnormal{A}_{\textnormal{g}}}p_{\textnormal{u}}\: a}\ln\frac{Y_{\textnormal{A,in}}}{Y_{\textnormal{A,out}}}
\end{eqnarray*}

\begin{eqnarray*}
\implies \frac{Y_{\textnormal{A,in}}}{Y_{\textnormal{A,out}}} & = & \exp\Big(\frac{-h k_{\textnormal{A}_{\textnormal{g}}} p_{\textnormal{u}}\: a}{G}\Big) \\
& = & \exp\Big(\frac{\num{-8} \cdot \num{181} \cdot \num{1} \cdot \num{440}}{\num{5.139e5}}\Big)\\
& = &  \num{0.2894}
\end{eqnarray*}
    
    \item 
    
        On arrive au système:
        $$
\left\{
\begin{array}{ccl}
p_{\ce{CO2}}^{\textnormal{in}}-p_{\ce{CO2}}^{\textnormal{out}} & = & \SI{0.1084}{\bar}\\
p_{\ce{CO2}}^{\textnormal{out}} / p_{\ce{CO2}}^{\textnormal{in}} & = & \num{0.2894}
\end{array}
\right.
$$
       $$
\Rightarrow \left\{
\begin{array}{ccl}
p_{\ce{CO2}}^{\textnormal{in}} & = & \SI{0.1525}{\bar}\\
p_{\ce{CO2}}^{\textnormal{out}} & = & \SI{0.04415}{\bar}
\end{array}
\right.
$$

    \item 
        $p_{\ce{CO2}\textnormal{,max}}^{\textnormal{in}} = \SI{0.1525}{\bar}$
    \item
        L'approximation faite est $k_{\textnormal{A}_{\textnormal{g}}}\:p_{\textnormal{A}} < k_\textnormal{l}\: C_{\textnormal{B}}$. Puisque cette inégalité est vérifiée au bas de la colonne, quand la pression des gaz est la plus élevée et la concentration en $\ce{NaOH}$, elle est vérifiée dans toute la colonne. En montant dans la colonne, la concentration de la solution ($C_\textnormal{B}$) augmente et la pression de $\ce{CO2}$ diminue.
    \item
        On a, par hypothèse que l'on travaille avec des gaz parfaits:
        $$\dot V_{\ce{CO2}}^{\textnormal{out}} / \dot V_{\textnormal{tot,g}} = p_{\ce{CO2}}^{\textnormal{out}} / p = \num{0.04415}$$
        $$\implies \dot V_{\ce{CO2}}^{\textnormal{out}} = \num{0.04415} \cdot \dot V_{\textnormal{tot,g}} = \num{0.04415} \cdot \num{40000} = \SI{1766}{\cubic\metre\per\hour} > \SI{1000}{\cubic\metre\per\hour}$$
        Ce dégagement de $\ce{CO2}$ est donc trop élevé.
        
\end{enumerate}

\end{solution}

\end{document}
