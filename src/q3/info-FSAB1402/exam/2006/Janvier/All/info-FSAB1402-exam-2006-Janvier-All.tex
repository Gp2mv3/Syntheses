\documentclass[fr]{../../../../../../eplexam}
\usepackage{../../../info-FSAB1402-exam}
\usepackage{../../../../../../eplcode}

\hypertitle[']{Informatique}{3}{FSAB}{1402}{2006}{Janvier}
{Martin Braquet}
{Peter Van Roy}

\section{(5 pt)}

Pour cette question, définissez la fonction déclarative \lstinline|Scie| qui prend deux listes d’entiers, 

Xs=$[x_1\:x_2 \ldots x_{kn}]$ et Ys=$[y1 \:y2 \ldots y_n]$ avec $n \geqslant 0$ et $k\geqslant 0$, et qui renvoie la liste 
Zs=$[(x_1y_1) \:(x_2y_2)\ldots $ $(x_ny_n) (x_{n+1}y_1) (x_{n+2}y_2) \ldots (x_{2n}y_n) (x_{2n+1}y_1) (x_{2n+2}y_2) \ldots (x_{kn}y_n)]$. Le nombre d’éléments
dans Xs est \textit{kn}, ce qui est \textit{k} fois le nombre d’éléments \textit{n} dans Ys. Voici un exemple d’exécution :
si Xs=[1 2 3 4 5 6] et Ys=[10 100 1000] alors Zs=[10 200 3000 40 500 6000] . Dans
votre réponse il faut définir chaque fonction dont vous avez besoin et chaque fonction récursive
doit faire une récursion terminale. Attention aux détails de syntaxe !

\begin{solution}

\lstinputlisting{Janvier-2006-Q1.oz}

\end{solution}

\section{(5 pt)}
Voici un petit programme:

\lstinputlisting{Janvier-2006-Q2.oz}

\begin{itemize} 

	\item Qu'est-ce qui est affiché quand on exécute ce programme ? 
 
	\item Donnez la traduction de ce programme en langage noyau. Attention à donner une traduction complète ! 
 
	\item Donnez les environnements contextuels de F et G. 
	
	\item Donnez un pas d'exécution de la machine abstraite pour l'affectation de C. 
 
  \item Donnez quelques pas d'exécution de la machine abstraite pour montrer la définition de F et G.
	
	\item Donnez quelques pas d'exécution de la machine abstraite pour montrer l'appel de F et G.

\end{itemize} 

\begin{solution}

\begin{itemize} 

	\item Il affiche 11. 
 
	\item \lstinputlisting{Janvier-2006-Q2(sol).oz} 
 
	\item $$CE_{F}=\{ C \rightarrow c \}$$
				$$CE_{G}=\{ F \rightarrow f \}$$
		
	\item 
	\begin{enumerate}
	
	\item 
				$$\bigg( [(<l1-l27>,\{ Browse \rightarrow browse \})],\{browse=(proc\{\$\: X\}\ldots end,CE_{Browse})\}, \emptyset \bigg)$$
				
	\item Création de variables et assignations + création d'une cellule
				$$\bigg( [(<l16-l26>,E_1\cup\{ F\rightarrow f,G\rightarrow g,C\rightarrow c,Tmp1\rightarrow tmp1,Tmp4\rightarrow tmp4,Tmp5\rightarrow tmp5,$$ $$Tmp6\rightarrow tmp6 \})],\sigma_1\cup\{tmp1=5, f=(proc\{\$\: N\:R1\}<l6-l8>end,CE_{F}=\{ C \rightarrow c \}) ,$$
				$$g=(proc\{\$\: K\:R2\}<l11-l13> end,CE_{G}=\{ F \rightarrow f \}), c=cellc, tmp4,tmp5,tmp6 \}, \{ cellc:tmp1\} \bigg)$$
				
	\item Lecture et assignations sur une cellule
				$$\bigg( [(<l20-l26>,E_2)],\sigma_2\cup\{tmp4=5, tmp5=2,tmp6=7 \}, \{ cellc:tmp6\} \bigg)$$
					
	\item Création de variables et assignations + création d'une cellule
				$$\bigg( [(<l24-l25>,E_3\cup\{ C_2\rightarrow c_2,Tmp7\rightarrow tmp7,Tmp8\rightarrow tmp8 \})],$$ $$\sigma_3\cup\{c_2=cellc_2,tmp7=10, tmp8=2, tmp9\}, \{ cellc:tmp6,cellc_2:tmp7\} \bigg)$$
	
	\item Appel de la procédure G
			$$	\bigg( [(<l11-l13>,E_{5a}=\{ F\rightarrow f,K\rightarrow tmp8,R2 \rightarrow tmp9,Tmp3\rightarrow tmp3 \}),(<l25>,E_4)],$$
			$$\sigma_3\cup\{c_2=cellc_2,tmp7=10, tmp8=2, tmp9\}, \{ cellc:tmp6,cellc_2:tmp7\} \bigg)$$
				
	\item Appel de la procédure F
				$$\bigg( [(<l6-l7>,\{ C\rightarrow cellc,N\rightarrow tmp8,R1 \rightarrow tmp3,Tmp2\rightarrow tmp2 \}),(<l13>,E_{5a}),$$
				$$(<l25>,E_4)],\sigma_5\cup\{tmp2\}, \{ cellc:tmp6,cellc_2:tmp7\} \bigg)$$
				
	\item Exécution du corps de F
				$$\bigg( [(<l13>,E_{5a}),(<l25>,E_4)],\sigma_6\cup\{tmp2=7,tmp3=9\}, \{ cellc:tmp6,cellc_2:tmp7\} \bigg)$$
				
	\item Exécution du corps de G
			$$	\bigg( [(<l25>,E_4)],\sigma_7\cup\{tmp9=11\}, \{ cellc:tmp6,cellc_2:tmp7\} \bigg)$$
				
	\item Affiche la valeur de la variable pointée par Tmp9 via le "`Browser"'. ($Tmp9\rightarrow tmp9$, et puisque $tmp9=11$, il affiche 11)
				$$\bigg( [],\sigma_8, \{ cellc:tmp6,cellc_2:tmp7\} \bigg)$$
	
	\end{enumerate}

\end{itemize} 

\end{solution}

\section{(5 pt)}
Définissez les concepts suivants avec précision. Pour chaque concept sauf la concurrence donnez
un fragment de code pour illustrer le concept.

\begin{itemize}
\item Identificateur
\item Concurrence
\item Thread
\item Flot
\item Agent
\end{itemize}

\end{document}
