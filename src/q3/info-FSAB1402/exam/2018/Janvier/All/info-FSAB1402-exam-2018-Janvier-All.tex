\documentclass[fr]{../../../../../../eplexam}
\usepackage{../../../../../../eplcode}
\lstset{language=Oz}

\hypertitle{Informatique % Enter the right title here
}{3}{FSAB}{1402}{2018}{Janvier}{All}
{Jean-Martin Vlaeminck\thanks{Merci à Antonio Gasos, Romain Rezs\"ohazy et Fred Verhaegen pour avoir transmis les questions}}
{Peter Van Roy}

\section{Programmation}
\begin{enumerate}
	\item Définissez une fonction \lstinline|Premiers| qui prend en argument
	une liste de listes $[L_1, L_2,\dots,L_n]$ et qui renvoie une liste de listes
	$[M_1,M_2,\dots,M_m]$ contenant dans $M_1$ les premiers éléments des listes
	$L_i$, dans $M_2$ les deuxièmes éléments, \dots, dans $M_m$ les $m$-ièmes
	et derniers éléments des listes.
	\item Définissez une fonction \lstinline|Derniers| qui prend en argument
	une liste de listes $[L_1,L_2,\dots,L_n]$ et qui renvoie une liste de listes
	$[M_1,M_2,\dots,M_m]$ telle que $M_1$ contient les derniers éléments
	des listes $L_i$, $M_2$ les avant-derniers éléments, \dots, et $M_m$
	les premiers éléments des listes.
\end{enumerate}

\nosolution

\section{Langage noyau et sémantique}

Soit le programme suivant:
\lstinputlisting{src/q2.oz}

\begin{enumerate}
	\item Qu'affiche ce programme?
	\item Traduisez ce programme en langage noyau. Attention à donner une traduction complète!
	\item Donnez les environnements contextuels de toutes les procédures.
	\item Donnez un pas d'exécution de la machine abstraite pour montrer
	\begin{itemize}
		\item la création d'une cellule,
		\item l'affectation d'une cellule,
		\item la lecture d'une cellule,
		\item la définition d'une procédure,
		\item l'appel d'une procédure.
	\end{itemize}
\end{enumerate}

\begin{solution}
\begin{enumerate}
	\item Il affiche $8$.
	\item Le code suivant donne une traduction.
	\lstinputlisting{src/q2-sol.oz}
	\item \nosubsolution
	\item \nosubsolution
\end{enumerate}
\end{solution}

\section{Définitions}
Définissez les termes suivants:
\begin{enumerate}
	\item portée statique d'une occurrence d'un identificateur;
	\item lien dynamique dans un objet; % wat ?
	\item non-déterminisme;
	\item problème NP-complet;
	\item notation $\bigoh$.
\end{enumerate}

\begin{solution}
Consultez les nombreux documents donnant des définitions de ces termes.
\end{solution}

\end{document}
