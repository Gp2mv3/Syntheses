\documentclass[fr]{../../../../../../eplexam}
\usepackage{../../../info-FSAB1402-exam}
\usepackage{../../../../../../eplcode}
\usepackage{listings}
\lstset{
    escapeinside={\%*}{*)},
    basicstyle=\footnotesize\ttfamily,
}

\hypertitle{info-FSAB1402}{3}{FSAB}{1402}{2017}{Janvier}
{Martin Braquet\and Jean-Martin Vlaeminck}
{Peter Van Roy}


\section{(5 pt)}

Écrivez une fonction \lstinline|{Incorpore L1 L2}| qui implémente les spécifications suivantes.
Si $L1=[a_0, \ldots, a_{m-1}]$ et $L2=[b_0, \ldots, b_{n-1}]$,
la fonction renvoie \lstinline|true| ssi
$\exists i_0<i_1<\ldots<i_{m-1}$ tels que
$b_{i_k}=a_k \quad \forall 0 \le k \le m-1$.
Autrement dit, la liste $L1$ est \og incorporée \fg{} dans $L2$ :
chaque élément de $L1$ est présent dans $L2$, dans le même ordre. Cette fonction peut être écrite en utilisant une seule fonction auxiliaire.

Écrivez ensuite une fonction \lstinline|{Apex LL}| qui renvoie l'apex d'une liste de liste. Si $LL=[L_0, L_1,\ldots,L_{n-1}]$, l'apex existe s'il existe un unique naturel $q$ tel que $0\le q \le \min(LL)-1$, et $\forall k$ tel que $0\le k \le q$, on a \lstinline[mathescape]|{Incorpore $L_k$ $L_{k-1}$}| et $\forall k$ tel que $q<k\le n-1$, on a \lstinline[mathescape]|{Incorpore $L_{k-1}$ $L_k$}|. S'il existe un unique $q$, alors \lstinline|{Apex LL}| renvoie $L_q$. Sinon, la fonction renvoie \lstinline|false|. On peut interpréter cette définition de la manière suivante : chaque liste, en partant d'une liste au centre (l'apex), est incluse dans la liste adjacente la plus éloignée de ce centre.


Dans votre réponse, il faut définir chaque fonction dont vous avez besoin et chaque fonction récursive
doit faire une récursion terminale. Attention aux détails de syntaxe !
\begin{solution}

\lstinputlisting{Janvier-2017-Q1.oz}

\end{solution}

\section{(5 pt)}
Voici un petit programme:

\lstinputlisting{Janvier-2017-Q2.oz}

\begin{itemize} 

	\item Qu'est-ce qui est affiché quand on exécute ce programme ? 
 
	\item Donnez la traduction de ce programme en langage noyau. Attention à donner une traduction complète ! 
 
	\item Donnez les environnements contextuels de toutes les procédures. 
	
	\item Donnez un pas d'exécution de la machine abstraite pour montrer la création d'une cellule. 
	
	\item Donnez un pas d'exécution de la machine abstraite pour montrer l'affectation d'une cellule. 
	
	\item Donnez un pas d'exécution de la machine abstraite pour montrer la lecture d'une cellule. 
 
  \item Donnez un pas d'exécution de la machine abstraite pour montrer la définition d'une procédure.
	
	\item Donnez un pas d'exécution de la machine abstraite pour montrer l'appel d'une procédure.

\end{itemize} 

\begin{solution}

\begin{itemize} 

	\item Il affiche 2. 
 
	\item \lstinputlisting{Janvier-2017-Q2(sol).oz} 
 
	\item $$\mbox{CE}_{F}=\{ NewCell \rightarrow newcell \}$$
				$$\mbox{CE}_{R4}=\{ C \rightarrow c, A\rightarrow i8 \}$$
				$$\mbox{CE}_{P}= \emptyset $$ 
		
	\item 
	Toute l'exécution de la machine abstraite est effectuée. Les états d'exécution à fournir à l'examen sont signalés ci-dessous.
	
		\begin{enumerate}
			\item 
					$$\bigg( \big[(<l1-l29>,E_1=\{Browse\rightarrow browse,NewCell\rightarrow newcell \}) \big],$$
					$$\{ browse=(proc\{\$\: X\}\ldots end,CE_{Browse}),newcell=(proc\{\$ \:X\: Y\}\ldots end,CE_{NewCell}) \}, \emptyset \bigg)$$
					
			\item 
					$$\bigg( \big[(<l2-l28>,E_2=E_1\cup\{F\rightarrow f,P\rightarrow p,R4\rightarrow r4,R5\rightarrow r5,I8\rightarrow i8 \}) \big],
					\sigma_1\cup\{ f,p,r4,r5,i8 \}, \emptyset \bigg)$$
					
			\item Avant la définition d'une procédure
					$$\bigg( \big[(<l2-l15>,E_2),(<l16-l28>,E_2) \big],
					\sigma_2, \emptyset \bigg)$$
					
			\item Après la définition d'une procédure
					$$\bigg( \big[(<l16-l28>,E_2) \big],
					\sigma_3\cup\{ f=(proc\{\$ \:A\: R\}<l3-l14>end,CE_{F}=\{NewCell\rightarrow newcell \})  \}, \emptyset \bigg)$$
					
			\item 
					$$\bigg( \big[(<l26-l28>,E_2) \big],
					\sigma_4\cup\{ i8=1,p=(proc\{\$ \:C\: R3\}<l17-l23>end,CE_{P}=\emptyset)  \}, \emptyset \bigg)$$
					
			\item Avant l'appel d'une procédure
					$$\bigg( \big[(<l26>,E_2),(<l27-l28>,E_2) \big],
					\sigma_5, \emptyset \bigg)$$
					
			\item Après l'appel d'une procédure
					$$\bigg( \big[(<l3-l14>,\{NewCell\rightarrow newcell,A\rightarrow i8,R\rightarrow r4 \}) \big],(<l27-l28>,E_2) \big],
					\sigma_5, \emptyset \bigg)$$
					
			\item Avant la création d'une cellule
					$$\bigg( \big[(<l5>,E_3=\{NewCell\rightarrow newcell,A\rightarrow i8,R\rightarrow r4,C\rightarrow c,I1\rightarrow i1 \}) \big],$$
					$$(<l6-l13>	,E_3),(<l27-l28>,E_2) \big],
					\sigma_5\cup\{ i1=0  \}, \emptyset \bigg)$$
					
			\item Après la création d'une cellule
					$$\bigg( \big[(<l6-l13>	,E_3),(<l27-l28>,E_2) \big],
					\sigma_8\cup\{ c=\xi  \}, \{c:i1\} \bigg)$$
							
			\item 
					$$\bigg( \big[(<l27-l28>,E_2) \big],
					\sigma_9\cup\{ r4=(proc\{\$ \:B\: R2\}<l7-l12>end,CE_{R4}=\{A\rightarrow i8,C\rightarrow c  \})  \}, \{c:i1\} \bigg)$$
									
			\item 
					$$\bigg( \big[(<l27>,E_2),(<l28>,E_2) \big],
					\sigma_{10}, \{c:i1\} \bigg)$$
					
			\item 
					$$\bigg( \big[(<l7-l12>,\{A\rightarrow i8,C\rightarrow c,B\rightarrow p,R2\rightarrow r5  \}),(<l28>,E_2) \big],
					\sigma_{11}, \{c:i1\} \bigg)$$
					
			\item 
					$$\bigg( \big[(<l8-l11>,E_4=\{A\rightarrow i8,C\rightarrow c,B\rightarrow p,R2\rightarrow r5,I2\rightarrow i2,I3\rightarrow i3,
						I4\rightarrow i4\}),(<l28>,E_2) \big],$$
						$$\sigma_{12}\cup\{ i2,i3,i4 \}, \{c:i1\} \bigg)$$
					
			\item 
					$$\bigg( \big[(<l17-l23>,\{C\rightarrow c,R3\rightarrow i2 \}),(<l9-l11>,E_4),(<l28>,E_2) \big],
					\sigma_{13}, \{c:i1\} \bigg)$$
							
			\item Avant la lecture d'une cellule
					$$\bigg( \big[(<l18>,E_5=\{C\rightarrow c,R3\rightarrow i2,I5\rightarrow i5,I6\rightarrow i6,I7\rightarrow i7 \}),(<l19-l22>,E_5),$$
					$$(<l9-l11>,E_4),(<l28>,E_2) \big],
					\sigma_{13}, \{c:i1\} \bigg)$$
					
			\item Après la lecture d'une cellule
					$$\bigg( \big[(<l19-l22>,E_5),(<l9-l11>,E_4),(<l28>,E_2) \big],
					\sigma_{13}\cup\{ i5=0 \}, \{c:i1\} \bigg)$$
					
			\item Avant l'affectation d'une cellule
					$$\bigg( \big[(<l21>,E_5),(<l22>,E_5),(<l9-l11>,E_4),(<l28>,E_2) \big],
					\sigma_{16}\cup\{ i6=1,i7=1 \}, \{c:i1\} \bigg)$$
					
			\item Après l'affectation d'une cellule
					$$\bigg( \big[(<l22>,E_5),(<l9-l11>,E_4),(<l28>,E_2) \big],
					\sigma_{17}, \{c:i7\} \bigg)$$
							
			\item 
					$$\bigg( \big[(<l9-l11>,E_4),(<l28>,E_2) \big],
					\sigma_{18}\cup\{ i2=1 \}, \{c:i7\} \bigg)$$
					
			\item 
					$$\bigg( \big[(<l10-l11>,E_4),(<l28>,E_2) \big],
					\sigma_{19}\cup\{ i4=0 \}, \{c:i7\} \bigg)$$
							
			\item 
					$$\bigg( \big[(<l17-l23>,\{C\rightarrow c,R3\rightarrow i3 \}),(<l11>,E_4),(<l28>,E_2) \big],
					\sigma_{20}, \{c:i7\} \bigg)$$
									
			\item 
					$$\bigg( \big[(<l21-l22>,\{C\rightarrow c,R3\rightarrow i3,I5\rightarrow i5_b,I6\rightarrow i6_b,I7\rightarrow i7_b \}),
					(<l11>,E_4),(<l28>,E_2) \big],$$
					$$\sigma_{21}\cup\{ i5_b=1,i6_b=1,i7_b=2 \}, \{c:i7\} \bigg)$$
					
			\item 
					$$\bigg( \big[(<l11>,E_4),(<l28>,E_2) \big],
					\sigma_{22}\cup\{ i3=2 \}, \{c:i7_b\} \bigg)$$
					
			\item 
					$$\bigg( \big[(<l28>,E_2) \big],
					\sigma_{23}\cup\{ r5=2 \}, \{c:i7_b\} \bigg)$$
					
			\item Affiche $R5\rightarrow r5=2$
					$$\bigg( \big[\big],
					\sigma_{24}, \{c:i7_b\} \bigg)$$		
		\end{enumerate}

La pile d'instructions est vide, le programme est terminé.

\end{itemize} 

\end{solution}

\section{(5 pt)}
Définissez les concepts suivants avec précision. Pour chaque concept sauf NP-complet, donnez un fragment de code pour illustrer.

\begin{itemize}
\item Non-déterminisme
\item Problème NP-complet
\item Modularité
\item Polymorphisme
\item Portée statique d'un identificateur
\end{itemize}

\begin{solution}
Une grande partie des concepts demandés aux examens se trouve dans les définitions des synthèses sur le drive EPL.
\end{solution}

\end{document}
