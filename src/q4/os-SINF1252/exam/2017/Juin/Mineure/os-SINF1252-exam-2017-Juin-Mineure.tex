\documentclass[fr, license=none]{../../../../../../eplexam}

\usepackage{../../../../../../eplcode}

\lstset{language={C}}

\hypertitle{Systèmes informatiques}{4}{SINF}{1252}{2017}{Juin}{Mineure}
{Gilles Peiffer \and Thomas Reniers}
{Olivier Bonaventure}

% TODO : Partie théorique, vient avant INGInious (where are the questions though...).

Le syllabus est accessible depuis l'URL \url{http://sites.uclouvain.be/SystInfo}.

\textbf{Les pages de manuel sont accessibles depuis les URLs suivants :}
\begin{itemize}
     \item \url{http://sites.uclouvain.be/SystInfo/manpages/man1} (commandes);
     \item \url{http://sites.uclouvain.be/SystInfo/manpages/man2} (appels systèmes);
     \item \url{http://sites.uclouvain.be/SystInfo/manpages/man3} (fonctions librairies).
\end{itemize}

\textbf{Attention:} veuillez utiliser la version \textbf{\html{}} du syllabus.

\section{Traduction de code assembleur}

La fonction suivante a été écrite en assembleur.
Traduisez-la en une version équivalente en \clang{}.
Votre fonction doit nécessairement avoir comme nom \texttt{f}.

Commencez par réfléchir à la signature de cette fonction et les types d'arguments qu'elle reçoit.

\begin{lstlisting}[language={[x86masm]Assembler}, emph={\$,\%,(,),movl,cmpl,addl},emphstyle={\color{blue}\bfseries}]
f:
    movl 4(%esp), %eax
    movl 8(%esp), %ebx
    movl (%ebx), %ecx
    cmpl %eax, %ecx
    jl end
    ret
end:
    movl %ecx,%eax
    ret

\end{lstlisting}

\begin{solution}

\begin{lstlisting}
int f(int a, int*b)
{
    if(*b < a)
    {
        return *b;
    }
    return a;
}

\end{lstlisting}

\end{solution}

% TODO Other questions...

\end{document}
