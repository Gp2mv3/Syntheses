\documentclass[fr]{../../../../../../eplexam}

\usepackage{../../../../../../eplcode}
\lstset{language={C}}

\hypertitle{Systèmes informatiques}{4}{SINF}{1252}{2019}{Juin}{Majeure}
{Gauthier de Moffarts}
{Olivier Bonaventure}

\part{Partie théorique}
\section{Question 1 [2 points]}
Un système de fichier utilise des inodes et répertoires. Où sont stockés les informations suivantes?

\begin{itemize}
\item le nom d'un fichier
\item la taille d'un fichier
\item les permissions d'un fichier
\item les zones de la mémoire où est stocké le fichier
\end{itemize}

\begin{solution}
Un inode contient les informations suivantes:
\begin{lstlisting}
struct simple_inode {
	uint16 mode;
	uid_t uid;
	gid_t gid;
	uint32 size;
	unit32 atime;
	unit32 mtime;
	unit32 ctime;
	uint16 nlinks;
	uint16 zone[10];
};
\end{lstlisting}
La réponse devient maintenant plus évidente, les taille, les permissions (mode) et les zones mémoires sont stockées dans les inodes.

Enfin, on peut justifier le fait que le nom est dans un répertoire en se rappelant ces phrases du syllabus: \og Un répertoire est un fichier qui a un format spécial. Il contient une suite d’entrées qui contiennent chacune un nom (de fichier ou de répertoire), une indication de type qui permet notamment de distinguer les fichiers des répertoires et le numéro de l’inode qui contient les méta-données du fichier ou répertoire \fg{}.
\end{solution}


\section{Question 2 [4 points]}
Écrivez l'algorithme de Peterson (ou variante) en pseudo-code.
\begin{solution}
\lstinputlisting{../../../../summary/code/peterson.c}%autant ne pas dupliquer les codes
\end{solution}

\section{Question 3 [2 points]}
Pour un processus, quelles sont les pages qui ont les permissions RW- et quels sont celles dans les permissions sont R-X?

\nosolution



\section{Question 4 [4 points]}
Dans quelle zone de la mémoire se trouvent les variables \lstinline|a, b, c, r, g, k|, le poiteur \lstinline|ptr| (ainsi que la zone vers laquelle il pointe) et les codes des fonctions f, main, printf? Donnez aussi l'état de la stack lors de l'exécution de la fonction f, à l'endroit du macro ---ICI---. 
\lstinputlisting{Code/zoneMemoire.c}

\begin{solution}
\textbf{---TO COMPLETE---}

\lstinline|a, b, r, ptr, g, k| sont sur la stack

Le contenu de \lstinline|ptr| est sur le heap.

c est avec les variables globales.

La fonction f et la main sont dans le segment texte

La fonction standard printf est dans les données dynamiques, entre la stack et le heap.

Afin de représenter la stack, il faut se rappeler que la stack dédiée à la fonction f conserve celle de la fonction main. En fait, à l'appel de la fonction f, on pousse sur la stack les arguments de f (en commençant par le dernier) ainsi que son adresse de retour. Seulement ensuite, viennent les valeurs telles que \og r \fg{}, provenant de l'exécution de f.
\end{solution}

\section{Question 5 [2 points]}
Expliquez les fonctionnement de la fonction pthread\_create notamment en expliquant ce qu'il se passe dans la table des pages.


\nosolution


\part{Partie pratique}
\section{Question 1 [10\% de la partie inginious]}
Traduction d'une fonction en assembleur
\begin{lstlisting}[language={[x86masm]Assembler}]
f : 	movl 4(%esp), %eax
	movl 8(%esp), %ebx
	movl (%ebx), %ecx
	addl %ecx ,%eax
	movl 12(%esp), %ecx
	addl %ecx, eax
	ret
\end{lstlisting}
Ce code n'est pas le code original mais s'en rapproche, si un de vous a passé l'examen et s'en rappelle... N'hésitez pas à l'éditer!
\begin{solution}
Le point important de cette exercice est l'utilisation d'un pointeur pour le deuxième paramètre! En effet, la présence de parenthèse indique le mode d'adressage indirect.
\end{solution}


\section{Question 2 [45\% de la partie inginious]}
Ecrire la fonction « add » d’une liste simplement chaînée.


\nosolution

\section{Question 3 [45\% de la partie inginious]}
Ecrire une fonction qui réalise la même chose que la commande « cp » d’un shell.
bash (réalise une copie d’un fichier) et qui conserve les mêmes permissions sur les 2 fichiers, en utilisant uniquement les appels systèmes open, read, write, stat et close. Il est interdit d'utiliser un buffer de taille supérieure à 100 bytes ainsi que de faire appel à la fonction \og system \fg{}.

\nosolution

\paragraph{Note}
Merci à Nicolas K. pour avoir posté les questions, Ben D. et Stan. G pour avoir apporté des précisions lors de la rédaction de ce document.

\end{document}
