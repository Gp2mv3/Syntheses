\documentclass[fr, license=none]{../../../../../../eplexam}

\usepackage{../../../../../../eplcode}

\lstset{language={C}}

\hypertitle{Systèmes informatiques}{4}{SINF}{1252}{2016}{Juin}{Majeure}
{Gilles Peiffer}
{Olivier Bonaventure}

% TODO : Partie théorique, vient avant INGInious (where are the questions though...).

Le syllabus est accessible depuis l'URL \url{http://sites.uclouvain.be/SystInfo}.

\textbf{Les pages de manuel sont accessibles depuis les URLs suivants :}
\begin{itemize}
     \item \url{http://sites.uclouvain.be/SystInfo/manpages/man1} (commandes);
     \item \url{http://sites.uclouvain.be/SystInfo/manpages/man2} (appels systèmes);
     \item \url{http://sites.uclouvain.be/SystInfo/manpages/man3} (fonctions librairies).
\end{itemize}

\textbf{Attention:} veuillez utiliser la version \textbf{\html{}} du syllabus.

\section{Traduction de code assembleur}

Traduisez en langage \clang{} la fonction assembleur suivante.

\begin{lstlisting}[language={[x86masm]Assembler}, emph={\$,\%,(,),movl,cmpl,addl},emphstyle={\color{blue}\bfseries}]
fct:
    movl 4(%esp), %eax
    cmpl $0,%eax
    jg n
    movl $-1,%eax
    ret
n:  movl $0,%ebx
    movl %eax, %ecx
    movl $0, %eax
    movl $0, %edx
l:  addl $2, %ebx
    addl %ebx, %eax
    addl $1, %edx
    cmpl %ecx, %edx
    jl l
    ret

\end{lstlisting}

Votre fonction doit nécessairement s'appeler \texttt{fct}.

\begin{solution}

\begin{lstlisting}
int fct(int a)
{
    if (a > 0)
    {
        int b = 0;
        int c = a;
        int a = 0;
        int d = 0;
        do
        {
            b += 2;
            a += b;
            d++;
        } while (d < c);
        return a;
    }
    else
    {
        return -1;
    }
}

\end{lstlisting}

\end{solution}

\section{Déterminer la valeur de retour d'un programme exécutable}

Dans un programme similaire à un shell, on vous demande d'écrire une fonction
qui permet de lancer un exécutable et d'indiquer si :

\begin{itemize}
     \item le programme ne s'est pas exécuté ou a retourné une valeur de retour positive;
     \item le programme s'est exécuté correctement et a retourné une valeur de retour $=0$;
     \item le programme a été interrompu par un signal.
\end{itemize}

Insérez ici le code de la fonction \texttt{run}.

\begin{lstlisting}
#include <unistd.h>
#include <sys/wait.h>
#include <stdlib.h>
#include <stdio.h>
#include <string.h>
#include <errno.h>

/*
 * @pre : prog!=null, arg!=null
 * @post: exécute le programme prog en lui passant
 *       arg comme arguments et sans environnement.
 *       Retourne
 *       -1 si le programme n'est pas exécutable
 *       ou a retourné une valeur de retour >0
 *       0 si il s'exécute correctement et retourne une valeur de retour = 0
 *       -2 si il s'exécute et est interrompu par la réception d'un signal
 */
int run(char *prog, char *arg[]) {

\end{lstlisting}

\begin{solution}

\begin{lstlisting}
pid_t pid=fork();
int status;

if (pid==0)
{ /* child process */
    int err = execv(prog,arg);
    if (err != 0)
    {
        exit(127);
    }
    exit(-1);
}
else
{ /* pid != 0; parent process */
    int fils = waitpid(pid,&status,0); /* wait for child to exit */
    if (fils == -1)
    {
        exit(EXIT_FAILURE);
    }
    if (WIFEXITED(status))
    {
        if (WEXITSTATUS(status) == 0)
        {
            return 0;
        }
        return -1;
    }
    else
    {
        if(WIFSIGNALED(status))
        {
            return -2;
        }
    }
    return EXIT_SUCCESS;
}

\end{lstlisting}

\end{solution}

\section{Modification de fichier}

La fonction \texttt{reverse}, dont les spécifications sont reprises ci-dessous,
permet de manipuler les données dans un fichier.

\begin{lstlisting}
/*
 * @pre filename!=NULL
 * @post Modifie le contenu du fichier filename en échangeant :
 *  - le premier byte du fichier avec le dernier
 *  - le deuxième byte avec l'avant dernier
 *  - ...
 * Apres modification, le fichier est fermé.
 * retourne 0 en cas de succès, -1 en cas d'erreur.
 * L'implémentation ne peut pas utiliser read, write, fread, fwrite, fgetc, fgets, ... Ce qui implique que mmap est obligatoire.
 */

\end{lstlisting}

Insérez ci-dessous le corps de votre fonction \texttt{reverse}.

\begin{lstlisting}
#include <sys/types.h>
#include <sys/stat.h>
#include <sys/mman.h>
#include <fcntl.h>
#include <stdlib.h>
#include <stdio.h>
#include <unistd.h>
#include <string.h>

/*
 * @pre filename!=NULL
 * @post Modifie le contenu du fichier filename en echangeant :
 *  - le premier byte du fichier avec le dernier
 *  - le deuxieme byte avec l'avant dernier
 *  - ...
 * Apres modification, le fichier est ferme.
 * retourne 0 en cas de succes, -1 en cas d'erreur.
 * L'implementation ne peut pas utiliser, fopen, read, write, fread, fwrite, fgetc, fgets, ... open et mmap sont obligatoires.
 */

int reverse(char *filename) {

\end{lstlisting}

\begin{solution}

\begin{lstlisting}
struct stat file_stat;
int file1;

char *dst;

if ((file1 = open(filename, O_RDWR)) < 0)
{
    perror("open");
    return -1;
}

if (fstat(file1,&file_stat) < 0)
{
    return -1;
}

if ((dst = mmap(NULL, file_stat.st_size, PROT_READ | PROT_WRITE, MAP_SHARED, file1, 0)) == NULL)
{
    return -1;
}

char *copy = malloc(file_stat.st_size);

int i;
memcpy(copy, dst, file_stat.st_size);

for(i=0; i < file_stat.st_size; ++i)
{
    dst[i] = copy[file_stat.st_size-i-1];
}

free(copy);

if (munmap(dst, file_stat.st_size) < 0)
{
    return -1;
}

if (close(file1) < 0)
{
    return -1;
}

return 0;

\end{lstlisting}

\end{solution}

\section{\texttt{Arraylist}}

Vous devez modifier une librairie qui implémente une \texttt{ArrayList}
en y ajoutant une fonction. Cette \texttt{ArrayList} s'utilise comme suit

\begin{lstlisting}
int main(void) {
     struct array_list *head = arraylist_init((size_t) 2, (size_t) sizeof(int));
     int first = 5;
     int second = 6;
     int third = 7;
     int tmp;

     int ret;

     if (!head)
             return 0;

     set_element(head, 0, (void *)&first);
     set_element(head, 1, (void *)&second);

     get_element(head, 1, (void *)&tmp);
     // tmp contient 6
     add_tail(head, (void *)&third);
     get_element(head, 2, (void *)&tmp);
     // tmp contient 7
     printf("array-list size: %d element-size %d\n", get_size(head), get_elem_size(head));
     // affiche array-list size: 3 element-size 4
     array_list_destroy(head);
     return 0;
}

\end{lstlisting}

\begin{lstlisting}
#include <errno.h>
#include <stdio.h>
#include <stdlib.h>
#include <string.h>
#include <semaphore.h>

struct array_list {
     char *content;
     size_t size;
     size_t elem_size;
     sem_t mutex;
};

/* Initialize an array-list
 *
 * @return: Pointer to array_list on   success, NULL on error and errno is set.
 */
struct array_list *arraylist_init(size_t n_elem, size_t elem_size)
{
     struct array_list *alist= (struct array_list *)malloc(sizeof(*alist));

     if (!alist)
             return NULL;

     alist->content = malloc(n_elem * elem_size);
     if (!alist->content) {
             free(alist);
             return NULL;
     }

     if (sem_init(&alist->mutex, 0, 1)) {
             free(alist->content);
             free(alist);
             return NULL;
     }

     alist->size = n_elem;
     alist->elem_size = elem_size;

     return alist;
}

/* Set the element 'elem' at position 'index' (indexes start at 0) inside the
 * array-list
 *
 * @return: -1 on error, 0 on success.
 */
int set_element(struct array_list *alist, int index, void *elem)

{
     char *ptr;

     if (index >= alist->size)
             return -1;

     if (sem_wait(&alist->mutex))
             return -1;

     ptr = alist->content + (index * alist->elem_size);
     memcpy(ptr, elem, alist->elem_size);

     if (sem_post(&alist->mutex))
             return -1;

     return 0;
}


/* Get the element at position 'index' of the array-list and copy it into 'elem'.
 * It is exepected that 'elem' has a sufficiently large memory-region to
 * hold the data.
 *
 * @return: -1 on error, 0 on success.
 */
int get_element(struct array_list *alist, int index, void *elem)
// code non fourni

/* Get the current size of the array-list */
size_t get_size(struct array_list *alist)
{
       return alist->size;
}

/* Get the size of the elements of the   array-list */
size_t get_elem_size(struct array_list *alist)
{
       return alist->elem_size;
}

/* Fully destroys the array-list. All elements will be lost */
int array_list_destroy(struct array_list *alist)
// code non fourni

/* Expand the array-list by one element and put 'elem' at this place.
 *
 * @return: -1 on error, 0 on success
 *
 * Vous *devez* utiliser realloc dans cette fonction
 */
int add_tail(struct array_list *alist, void *elem)
// question

\end{lstlisting}

\begin{solution}

\begin{lstlisting}
if (sem_wait(&alist->mutex))
{
    return -1;
}

(alist->size) = (alist->size)+1;
char* tmp = realloc(alist->content, alist->size * alist->elem_size);

if (!tmp)
{ // si erreur de realloc
    sem_post(&alist->mutex);
    return -1;
}
else
{
    alist->content = tmp;
}

char *ptr;

size_t inc = ((alist->size)-1) * (alist->elem_size);
ptr = (alist->content+inc);

memcpy(ptr, elem, alist->elem_size);

if (sem_post(&alist->mutex))
{
     return -1;
}

return 0;

\end{lstlisting}

\end{solution}

\end{document}
