\section{Méthode du simplexe}

\subsection{Algorithme}

Soit un problème d'optimisation linéaire sous la forme
\[
\renewcommand{\arraystretch}{1.5}
\begin{array}{c@{\quad}rcr@{\quad}l}
	\max\limits_{x \in \Rn} & c^T x &     &      &                       \\
	\textnormal{s.t.}       &   A x & =   & b\,, & A \in \R^{m \times n}\,, \\
	                        &     x & \ge & 0\,. &                       \\
\end{array}
\]

Reconstituons l'algorithme.

\begin{enumerate}
	\item \textbf{Recherche d'un sommet.}

	On a $m$ variables de base (dans $\xb$)
	et $n-m$ variables hors base (dans $\xn$).
	On a un sommet en $\begin{bmatrix} \xb & \xn \end{bmatrix}^T$,
	à condition que $\xn = 0$ et $\xb = B^{-1} b \ge 0$,
	avec $A = \begin{bmatrix} B & N \end{bmatrix}$.
	Cela veut donc dire que la matrice $B$ doit être inversible.
	\item \textbf{Recherche d'un sommet adjacent meilleur.}
	\begin{enumerate}
		\item \textbf{Regarder les sommets adjacents.}

		Un sommet adjacent est un sommet ayant
		une variable de différence dans la base $B$.
		\item Éventuellement, \textbf{choisir un sommet meilleur.}
	\end{enumerate}
	\item \textbf{Arrêt.}

	Si le programme s'arrête, on est à l'optimum.
\end{enumerate}

% TODO : Would've liked to include this but it seems to be a lot more complicated than it looks :(.
%
%\begin{algorithm}
%	\caption{Finding the vertex of the optimum solution to a linear programming problem}
%	\begin{algorithmic}[1]
%		\Require{$B$ is an invertible matrix}
%		\StateX
%		\Function{Simplex}{$A, b, c, x$}
%			\Let{$\xn$}{$0$}
%			\Let{$\xb$}{$B^{-1} b$} \Comment{$x$ is now a vertex}
%			\For{$x_{\textnormal{adj}}$ \gets every vertex adjacent to $x$}
%				\If{$c^T x_{\textnormal{adj}} > c^T x$}
%					\Let{$x$}{$x_{\textnormal{adj}}$}
%				\EndIf
%			\EndFor
%			\State \Return{$x$} \Comment{$x$ is now the vertex of the optimum solution}
%		\EndFunction
%	\end{algorithmic}
%\end{algorithm}

\subsection{Tableau simplexe}

\begin{myexem}
	Considérons le problème suivant:
	\[
	\renewcommand{\arraystretch}{1.5}
	\begin{array}{c@{\quad}rcrcrcrcr@{\quad}l}
		\min\limits_{x} &&& x_2 & - & 5 x_3 & + & 5 x_4 &&& \\
		\textnormal{s.t.} & x_1 & + & x_2 & - & 11 x_3 & + & 7 x_4 & = & 10\phantom{\,,} & \\
		&&& x_2 & - & 8 x_3 & + & 4 x_4 & = & 4\phantom{\,,} & \\
		&&&&&&& x_i & \ge & 0\,, & i \in \{1,2,3,4\}\,.
	\end{array}
	\]
	Construisons le tableau simplexe pour ce problème.
	\[
	\renewcommand{\arraystretch}{1.5}
	\begin{array}{rrrr|r}
		x_1 & x_2 &  x_3 & \multicolumn{1}{r}{x_4} &    \\
		  0 &   1 &   -5 &                       5 &  z \\
		\hline
		  1 &   1 &  -11 &                       7 & 10 \\
		  0 &   1 &   -8 &                       4 &  4
	\end{array}
	\]
	Choisissons la base $\left\{1, 2\right\}$.
	On transforme alors le tableau pour obtenir
	\[
	\renewcommand{\arraystretch}{1.5}
	\begin{array}{cccc|c}
		0 & 0 &  3 & 1 & z-4 \\
		\hline
		1 & 0 & -3 & 3 &   6 \\
		0 & 1 & -8 & 4 &   4
	\end{array}
	\]
	On appelle les valeurs au-dessus de la ligne horizontale
	les \emph{coûts réduits}.
	Le dernier tableau est appelé
	tableau canonique pour la base $\left\{1, 2\right\}$.
	Pour trouver le coût de la solution,
	le terme encadré en haut à droite
	($z-4$ en l'occurrence) doit être égalé à zéro.
	La valeur de $z$ ainsi obtenue est le coût.
	La solution optimale est donc
	\[
	x^* = (x_1, x_2, x_3, x_4) = (6,4,0,0)\,,\quad z^* = 4\,.
	\]
	On sait qu'elle est optimale
	car les coûts réduits sont tous positifs ou nuls.

	Le tableau simplexe trouvé ci-dessus est en fait
	un cas spécial du tableau général ci-dessous.
	\[
	\renewcommand{\arraystretch}{1.5}
	\begin{array}{cccc|c}
		0 & 0 & c_3 & c_4 & z-4 \\
		\hline
		1 & 0 &  -3 &   3 &   6 \\
		0 & 1 &  -8 &   4 &   4
	\end{array}
	\]
	Comme expliqué ci-dessus, le cas $c_3, c_4 \ge 0$
	indique que le sommet est optimal.
	Une autre possibilité serait le tableau
	\[
	\renewcommand{\arraystretch}{1.5}
	\begin{array}{cccc|c}
		0 & 0 & -3 & 1 & z-4 \\
		\hline
		1 & 0 & -3 &  3 &   6 \\
		0 & 1 & -8 &  4 &   4
	\end{array}
	\]
	Ici, le coût réduit associé à $x_3$ est négatif.
	Conservons $x_4 = 0$, et posons $x_3 = \lambda$.
	On obtient alors $-3 \lambda = z - 3$.
	Plus $\lambda$ augmente, plus $z$ diminue.
	Cependant, plus $\lambda$ augmente,
	plus $x_1$ et $x_2$ augmentent également.
	Les conditions $x_1,x_2 \ge 0$ sont toujours satisfaites,
	et le coût est décroissant le long de la demi-droite
	\[
	(6+3\lambda,4+8\lambda,\lambda,0)\,,\quad \lambda \ge 0\,,
	\]
	totalement contenue dans le polyèdre.
	Le coût n'est pas borné.

	Finalement, prenons un cas
	où les deux coûts réduits sont négatifs.
	\[
	\renewcommand{\arraystretch}{1.5}
	\begin{array}{cccc|c}
		0 & 0 & -3 & -1 & z-4 \\
		\hline
		1 & 0 &  3 &   3 &  6 \\
		0 & 1 &  6 &   4 &  4
	\end{array}
	\]
	Introduisons $x_3$ dans la base.
	On aura alors $x_3 = \lambda$, en gardant $x_4 = 0$.
	Pour trouver un nouveau sommet,
	trouvons la variable qui doit sortir de la base.
	Les contraintes deviennent
	\[
	\renewcommand{\arraystretch}{1.5}
	\left\{
	\begin{array}{rcrcrcr}
		x_1 & + &     &   & 3 \lambda & = & 6\,,\\
		    &   & x_2 & + & 6 \lambda & = & 4\,.
	\end{array}
	\right.
	\]
	Lorsque $\lambda$ augmente,
	la contrainte $x_2 + 6 \lambda = 4$ est
	la première à devenir critique.
	C'est donc $x_2$ qui sort de la base.
	On obtient $\lambda = 2/3$
	et on passe au sommet $(4,0,2/3,0)$.
	Le coût optimal a diminué de deux unités en changeant de base.
\end{myexem}

On a donc trois cas possibles,
et le tableau simplexe permet de rapidement identifier
dans lequel on se trouve:
\begin{itemize}
	\item si tous les coûts réduits sont plus grands ou égals à zéro,
	le sommet est un sommet optimal;
	\item si il existe un coût réduit strictement négatif
	avec une colonne dont tous les éléments sont
	plus petits ou égals à zéro,
	le coût optimal est non borné;
	\item sinon, il y a un sommet adjacent de coût moindre.
	On choisit un des coûts réduits strictement négatifs,
	on calcule $\lambda$ par la règle du ratio
	et on en déduit la variable à faire sortir de la base.
\end{itemize}

\subsection{Cyclage}

Il peut arriver qu'un problème tourne à l'infini.
On parle de ``cyclage''.
Il est possible d'éviter ce phénomène
en utilisant la \emph{règle de Bland}\footnote{Découverte par Robert Bland
alors qu'il était en visite à l'UCLouvain \dSmiley.}:
parmi les variables candidates à l’entrée dans (et à la sortie de) la base,
choisir celle de plus petit indice.

\subsection{Initialisation de l'algorithme}

Afin de lancer l'algorithme du simplexe,
il faut un sommet initial.
La recherche d'un sommet d'un polyèdre peut se faire
lors d'une phase d'initialisation (phase I)
en résolvant un problème d'optimisation linéaire annexe
pour lequel on dispose d'un sommet initial.

Soit le polyèdre
\[
\mathcal{P} = \set{x \in \Rn \suchthat Ax = b\,,\quad x \ge 0}
\]
pour lequel nous cherchons un sommet.
Supposons sans perte de généralité $b \ge 0$.
On introduit autant de variables $y_i$ que d'équations
et on construit le problème annexe
\[
\renewcommand{\arraystretch}{1.5}
\begin{array}{c@{\quad}rcl}
	\min\limits_{y} & \sum y_i &&\\
	\textnormal{s.t.} & Ax + y & = & b\\
	& x & \ge & 0\\
	& y & \ge & 0\,.
\end{array}
\]
La solution $x = 0, y = b$ est solution admissible de ce problème.
Si le coût optimal du problème annexe
est supérieur à zéro,
c’est qu’il n’y a pas de solution admissible pour laquelle $y = 0$,
et donc le polyèdre initial $\mathcal{P}$ est vide.
Si le coût optimal du problème annexe est égal à zéro,
on considère une solution admissible de base optimale $(x^*,0)$.
La solution $x^*$ est un sommet du polyèdre initial $\mathcal{P}$.

Au niveau de l'efficacité,
on n'a pas toujours besoin de tous les $y_i$.
