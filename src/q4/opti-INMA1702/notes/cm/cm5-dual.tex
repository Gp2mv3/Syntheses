\section{Dualité}

\subsection{Complexité de l'algorithme du simplexe}

Prenons le cas d'un problème à $n$ variables.
Une base de ce problème a donc $m$ indices
parmi ces $n$ variables.

Dans le pire des cas, le temps de résolution est
asymptotiquement exponentiel en $m$.

Cependant, pour la plupart des problèmes,
on trouve en pratique que la résolution est polynomiale
en $m$ et $n$.

Il existe aussi des algorithmes avec
une complexité polynomiale dans le pire des cas:
les méthodes de points intérieurs
(comme par exemple l'algorithme de Karmarkar).

\subsection{Dual}

Un problème d’optimisation linéaire
possède toujours un problème companion, un problème dual,
pour lequel le rôle des variables et des contraintes est inversé.
Le dual du dual est le problème initial
que l’on appelle problème primal.
Pour chaque variable dans le problème primal,
il y a une contrainte dans le dual
et pour chaque contrainte dans le problème primal,
il y a une variable dans le problème dual.

Ce problème dual permet notamment de trouver
la sensibilité par rapport au membre de droite.
C'est-à-dire que l'on peut estimer facilement
l'effet d'un changement de $b$ dans l'équation
$Ax = b$.

\subsection{Dualité faible/forte}

Soit $(P)$ un problème primal de minimisation.
Soit $(D)$ le problème dual de maximisation correspondant.

\begin{mytheo}[Théorème de la dualité faible]
	Si $x$ est admissible pour $(P)$ et a comme coût associé $p$
	et $y$ est admissible pour $(D)$ et a comme coût associé $d$,
	alors $p \ge d$.
	\begin{proof}
		Preuve par construction du dual.
	\end{proof}

	\begin{center}
		\begin{tikzpicture}
		\draw[->,thick] (0,0)--(0,6) node[above]{Objectif $\in \R$};
		\coordinate (A) at (0,0);
		\coordinate (B) at (0,6);

		\begin{scope}[>={Stealth[black]}, every node/.style={blue}]
		\node[label=0:$d_1$] (D1) at (0,0.5) {$\times$};
		\node[label=0:$d_2$] (D2) at (0,1.5) {$\times$};
		\node[label=0:$d^*$] (Dopt) at (0,2.5) {$\times$};
		\end{scope}

		\begin{scope}[>={Stealth[black]}, every node/.style={red}]
		\node[label=180:$p_1$] (P1) at (0,5.5) {$\times$};
		\node[label=180:$p_2$] (P2) at (0,4.5) {$\times$};
		\node[label=180:$p^*$] (Popt) at (0,3.5) {$\times$};

		\draw [decorate,decoration={brace,amplitude=10pt},xshift=-15pt,yshift=0pt]
		(-0.5,0.3) -- (-0.5,2.7) node [blue,midway,xshift=-1.8cm]
		{solutions duales};
		\draw [decorate,decoration={brace,amplitude=10pt,mirror, raise=15pt},yshift=0pt]
		(0.5,3.3) -- (0.5,5.7) node [red,midway,xshift=2.5cm]
		{solutions primales};
		\end{scope}
		\end{tikzpicture}
	\end{center}
\end{mytheo}

\begin{mycorr}
	Si $p^*$ est le coût optimal pour $(P)$
	et $d^*$ est le coût optimal pour $(D)$,
	alors $p^* \ge d^*$.
\end{mycorr}

\begin{mycorr}
	Si $x$ et $y$ sont tous les deux admissibles
	et vérifient $p = d$,
	alors ils sont tous les deux optimaux:
	\[
	(x,y,p,d) = (x^*,y^*,p^*,d^*)\,.
	\]
\end{mycorr}

\begin{mycorr}
	Si le primal est non borné,
	alors le dual est impossible.
	Si le dual est non borné,
	alors le primal est impossible.
\end{mycorr}

\begin{mytheo}[Théorème de la dualité forte]
	Si $(P)$ admet une solution optimale $x^*$,
	alors $(D)$ admet une solution optimale $y^*$
	telle que $p^* = d^*$.

	\begin{center}
		\begin{tikzpicture}
		\draw[->,thick] (0,0)--(0,6) node[above]{Objectif $\in \R$};
		\coordinate (A) at (0,0);
		\coordinate (B) at (0,6);

		\begin{scope}[>={Stealth[black]}, every node/.style={blue}]
		\node[label=0:$d_1$] (D1) at (0,1) {$\times$};
		\node[label=0:$d_2$] (D2) at (0,2) {$\times$};
		\end{scope}

		\begin{scope}[>={Stealth[black]}, every node/.style={red}]
		\node[label=180:$p_1$] (P1) at (0,5) {$\times$};
		\node[label=180:$p_2$] (P2) at (0,4) {$\times$};
		\end{scope}

		\begin{scope}[>={Stealth[black]}, every node/.style={magenta}]
		\node[label=180:$p^*$,label=0:$d^*$] (Popt) at (0,3) {$\times$};
		\end{scope}

		\draw [decorate,decoration={brace,amplitude=10pt},xshift=-15pt,yshift=0pt]
		(-0.5,0.8) -- (-0.5,3.2) node [blue,midway,xshift=-1.8cm]
		{solutions duales};

		\draw [decorate,decoration={brace,amplitude=10pt,mirror, raise=15pt},yshift=0pt]
		(0.5,2.8) -- (0.5,5.2) node [red,midway,xshift=2.5cm]
		{solutions primales};
		\end{tikzpicture}
	\end{center}
\end{mytheo}

\subsection{Classification}

\begin{table}[H]
	\centering
	\renewcommand{\arraystretch}{1.2}
	\begin{tabular}{|c|ccc|}
		\hline
		\diagbox[width=3.5cm]{$(P)$}{$(D)$} & Non borné & Solution optimale & Impossible\\
		\hline
		Non borné & \xmark & \xmark & \checkmark\\
		Solution optimale & \xmark & \checkmark & \xmark\\
		Impossible & \checkmark & \xmark & \checkmark\tablefootnote{Existe, mais numériquement instable.}\\
		\hline
	\end{tabular}
\end{table}

Pour une contrainte $a_i^T x \le b_i$
et la variable duale $y_i$,
on a une relation d'exclusion.
\begin{table}[H]
	\centering
	\begin{tabular}{c|cc}
		& $a_i^T x = b_i$ & $a_i^T x < b_i$\\
		\hline
		$y_i = 0$ & \checkmark & \checkmark\tablefootnote{Par exemple, imaginons avoir $\SI{24}{\hour}$ de disponible,
		mais une solution optimale avec $t \le \SI{24}{\hour}$.}\\
		$y_i > 0$ & \checkmark & \xmark
	\end{tabular}
\end{table}
