\documentclass[fr]{../../../../../../eplexam}
\usepackage{listings}

\definecolor{codegreen}{rgb}{0,0.6,0}
\definecolor{codegray}{rgb}{0.5,0.5,0.5}
\definecolor{codepurple}{rgb}{0.58,0,0.82}
\definecolor{backcolour}{rgb}{0.95,0.95,0.92}

\lstdefinestyle{mystyle}{
    backgroundcolor=\color{backcolour},   
    commentstyle=\color{codegreen},
    keywordstyle=\color{magenta},
    numberstyle=\tiny\color{codegray},
    stringstyle=\color{codepurple},
    basicstyle=\ttfamily\footnotesize,
    breakatwhitespace=false,         
    breaklines=true,                 
    captionpos=b,                    
    keepspaces=true,                 
    numbers=left,                    
    numbersep=5pt,                  
    showspaces=false,                
    showstringspaces=false,
    showtabs=false,                  
    tabsize=2
}
\lstset{style=mystyle}
\lstset{language=C}

\hypertitle{Projet P3}{4}{EPL}{1503}{2023}{Juin}{All}
{Thomas Debelle}
{Olivier Bonaventure \& Axel Legay}

\section{Étudiants en échec}
L'EPL a développé un petit logiciel en C qui stocke les points de tous les étudiants dans un tableau contenant des structures record définies comme suit :
\begin{lstlisting}[escapechar=µ]
struct record {    char *student; // nom de l'µ\textcolor{gray}{é}µtudiant (toujours diffµ\textcolor{gray}{é}µrent de NULL)
    int n; // nombre de cours suivis (strictement positif)
    int *points; // points obtenus pour chaque cours (les points sont toujours en 0 et 20)
};
\end{lstlisting}
% µ\textcolor{gray}{é}µ

Ecrivez le corps de la fonction \textbf{echecs} dont la spécification est la suivante :

\begin{lstlisting}[escapechar=µ]
/*
* @pre n>0 correspond au nombre d'µ\textcolor{gray}{é}µtudiants dans la classe, *classe!=NULL
* @post retourne le nombre de cours en µ\textcolor{gray}{é}µchecs pour l'ensemble des µ\textcolor{gray}{é}µtudiants de la classe
*/
int echecs(int n, struct record *classe) {
\end{lstlisting}
A titre d'exemple, considérons la classe contenant trois étudiants : - Marie qui a suivi 3 cours et obtenu 18, 10 et 11 - Jean qui a suivi 5 cours et obtenu 9, 7, 13, 20 et 10 - Luc qui a suivi 4 cours et obtenu 10, 11, 6 et 0\\

Lorsque l'on applique la fonction échecs à un tableau contenant les 3 records de ces étudiants, elle retourne la valeur 4.

\begin{solution}
\begin{lstlisting}[escapechar=µ]
if(classe == NULL){
    return 0;
}

int count = 0;
for(int i = 0; i < n; i++){
    for(int j = 0; j < classe[i].n; j++){
        if(classe[i].points[j] < 10){
            count++;
        }
    }
}

return count;
\end{lstlisting}
\end{solution}

\section{Nombre de caractères différents}
Un chaîne de caractères contient différents caractères. Vous devez écrire le corps de la fonction ndiff qui retourne le nombre de caractères différents dans la chaîne passée en argument.\\

Complétez le corps de la fonction ndiff sur base de la spécification ci-dessous:

\begin{lstlisting}[escapechar=µ]
#include <stdio.h>
#include <stdlib.h>
#include <string.h>

/*
 * @pre -
 * @post retourne le nombre de caractµ\textcolor{gray}{è}µres diffµ\textcolor{gray}{é}µrents dans la chaµ\textcolor{gray}{î}µne pointµ\textcolor{gray}{é}µe par str
 * Exemples:
 *  - ndiff("ABA") retourne 2
 *  - ndiff("") retourne 0
 *  - ndiff(NULL) retourne 0
 *  - ndiff("AabBAb") retourne 4
 */
int ndiff(char *str) {
\end{lstlisting}

\begin{solution}
\begin{lstlisting}[escapechar=µ]
if(str == NULL){
    return 0;
}
if(strlen(str) == 0){
    return 0;
}

char* found = (char*) malloc(sizeof(char)*(strlen(str) + 1)); // create a string with all the different character of str
int size = 0; int flag = 0; // the size of the found array and a flag to check if we found something

for(int i = 0; i < strlen(str); i++){
    if(size == 0){
        found[size] = str[i];
        size++;
    }else{
        flag = 1;
        for(int j = 0; j < size; j++){
            if(found[j] == str[i]){
                flag = 0;
                break;
            }
        }
        if(flag){
            found[size] = str[i];
            size++;
        }
    }
}

return size;
\end{lstlisting}
\end{solution}

\section{Liste doublement chaînée}

Une liste doublement chaînée est implémentée comme suit. Cette structure de données permet d'enregistrer un ensemble d'éléments, en liant pour chacun une clé à une valeur. On va faire l'hypothèse dans cette question qu'il n'y a qu'un seul exemplaire de chaque clé dans la liste.

\begin{lstlisting}[escapechar=µ]
typedef struct node {
    int key;
    double value;
    struct node* next;
    struct node* prev;
} node_t;

typedef struct list {
    node_t* head;
    node_t* tail;
} list_t;

list_t* create_list() {
    list_t* list = (list_t*) malloc(sizeof(list_t));
    if (list == NULL) {
        return NULL;
    }

    list->head = NULL;
    list->tail = NULL

    return list;
}

int insert(list_t* list, int key, double value) {
    node_t* node = (node_t*) malloc(sizeof(node_t));
    if (node == NULL) {
        return -1;
    }

    node->key = key;
    node->value = value;
    node->next = NULL;

    if (list->tail == NULL) {
        list->head = node;
        node->prev = NULL;
    } else {
        list->tail->next = node;
        node->prev = list->tail;
    }

    list->tail = node
    return 0;
}

void free_list(list_t* list) {
    node_t* current = list->tail;

    if (current == NULL) {
        free(list);
        return;
    }

    while (current->prev != NULL) {
        current = current->prev;
        free(current->next);
    }

    free(current);
    free(list);
}
\end{lstlisting}
Vous devez implémenter la fonction delete dont les spécifications vous sont fournies.\\
Insérez ici le corps de la fonction delete dont les spécifications vous sont données ci-dessous :

\begin{solution}
\begin{lstlisting}[escapechar=µ]
if(list == NULL){
    return 0;
}

if(list->tail == NULL && list->head == NULL){
    return 0;
}

node_t* current = list->head; node_t* prev = current;

// Check if the list is made of 1 element
if(current->next == NULL){
    if(current->key == key){
        free_list(list); // Free the whole list
        return key;
    }else{
        return 0;
    }
}

while(current != NULL){    
    if(current->key == key){
        if(current->prev == NULL){ // If we remove the head of the list
            printf("We are removing a head\n");
            list->head = current->next;
            current->next->prev = NULL;
            free(current); // Because we used malloc to create a node
            return key;
        }
        if(current->next == NULL){ // If we remove the tail of the list
            list->tail = current->prev;
            current->prev->next = NULL;
            free(current);
            return key;
        }

        prev->next = current->next;
        current->next->prev = prev;
        
        free(current);

        return key;
    }
    
    prev = current;
    current = current->next;
}

return 0;
\end{lstlisting}
\end{solution}

\section{Copie de fichier}
Ecrivez une fonction permettant de copier un fichier quelconque. Votre fonction ne peut bien entendu pas modifier le fichier source.\\

Vous pouvez uniquement utiliser les fonctions open(2), mmap(2), munmap(2), msync(2), ftruncate(3), fstat(2), memcpy(3) and close(2). Vous ne pouvez appeler memcpy(3) qu'une seule fois.\\

Astuce : msync(2) est une fonction qui vous permet de garantir que les modifications faites en mémoire (notamment celles à une adresse pointée par mmap) seront aussi sauvées sur disque. Lisez la documentation pour voir comment l'utiliser correctement.\\

Astuce : La fonction ftruncate(3) est une fonction que l'on utilise assez rarement, mais elle est nécessaire pour cette question.

\begin{lstlisting}[escapechar=µ]
/*
 * @pre file_name != NULL, nom du fichier source
 *      new_file_name != NULL, nom du fichier destination (la copie)
 *
 * @post copie le contenu de  {file_name} vers {new_file_name}.
 *       return 0 si la fonction se termine avec succµ\textcolor{gray}{è}µs, -1 en cas d'erreur.
 */
int copy(char *file_name, char *new_file_name) {
\end{lstlisting}

\begin{solution}
J'ai eu 90.91\% mais une version un peu différente existe dans le syllabus de théorie à la partie gestion des utilisateurs et des systèmes de fichiers. Mon erreur serait une erreur avec les fichiers vide.
\begin{lstlisting}[escapechar=µ]
if(file_name == NULL || new_file_name == NULL){
    return -1;
}

int fdOld = open(file_name, O_RDONLY);
if(fdOld == -1){
    return -1;
}

struct stat* buffStatOld = (struct stat*) malloc(sizeof(struct stat));
if(buffStatOld == NULL){
    close(fdOld);
    return -1;
}

if(fstat(fdOld, buffStatOld)){
    free(buffStatOld); close(fdOld);
    return -1;
}

if(buffStatOld->st_size == 0){ // If the file is empty
    return 0;
}

int fdNew = open(new_file_name, buffStatOld->st_mode | O_CREAT | O_RDWR); 
// we have all the permission from the other files + we can create a new file if needed and we can write
if(fdNew == -1){
    free(buffStatOld); close(fdOld);
    return -1;
}

void* oldMap = mmap(NULL, buffStatOld->st_size, PROT_READ, MAP_SHARED, fdOld, 0);
if(oldMap == MAP_FAILED){
    free(buffStatOld); close(fdOld); close(fdNew);
    return -1;
}

if(ftruncate(fdNew, buffStatOld->st_size)){
    free(buffStatOld); close(fdOld); close(fdNew);
    return -1;
}

void* newMap = mmap(NULL, buffStatOld->st_size, PROT_READ | PROT_WRITE, MAP_SHARED, fdNew, 0);
if(newMap == MAP_FAILED){
    free(buffStatOld); close(fdOld); close(fdNew); 
    munmap(oldMap, buffStatOld->st_size);
    return -1;
}

// The actual interesting code
if(memcpy(newMap, oldMap, buffStatOld->st_size) == NULL){ // We copy the file in memory from the old to the new and check if it fails
    free(buffStatOld); close(fdOld); close(fdNew); 
    munmap(oldMap, buffStatOld->st_size); munmap(newMap, buffStatOld->st_size); 
    return -1;
}

if(msync(newMap, buffStatOld->st_size, MS_SYNC)){ // MS_SYNC can slow down the program but it's a small trade off and we check if the syncing fails
    free(buffStatOld); close(fdOld); close(fdNew); 
    munmap(oldMap, buffStatOld->st_size); munmap(newMap, buffStatOld->st_size); 
    return -1;
}
// The end of it TT

if(munmap(oldMap, buffStatOld->st_size)){
    free(buffStatOld); close(fdOld); close(fdNew); 
    munmap(newMap, buffStatOld->st_size); 
    return -1;
}


if(munmap(newMap, buffStatOld->st_size)){
    free(buffStatOld); close(fdOld); close(fdNew); 
    return -1;
}

if(close(fdNew)){
    free(buffStatOld); close(fdOld);
    return -1;
}

if(close(fdOld)){
    free(buffStatOld);
    return -1;
}

free(buffStatOld);
return 0;
\end{lstlisting}
\end{solution}

\section{Comparaison de fonctions}
Parfois il est intéressant de comparer le résultat de fonctions sans nécessairement savoir quel est le code de ces fonctions. Dans cet exercice, vous devez écrire une fonction compare qui reçoit comme arguments un intervalle fermé et deux pointeurs vers des fonctions qui retournent chacune un entier.\\

Pour illustrer cette fonction compare, considérons les fonctions identite et abs:

\begin{lstlisting}[escapechar=µ]
int identite(int x) { return x; }
int abs(int x) {
  if (x>=0)
      return x;
  else
     return -x;
}
\end{lstlisting}

Dans l'intervalle $[0,2]$ $(min=0, max=2)$, compare retourne la valeur 3 lorsque elle est appliquée aux fonctions identité et abs. Dans l'intervalle $[-1,1]$ $(min=-1, max=1)$, elle retourne $2$ lorsqu'elle est appliquée aux mêmes fonctions.\\

Proposez le corps de la fonction compare dont la spécification est reprise ci-dessous:
\begin{lstlisting}[escapechar=µ]
/*
 * @pre min<max, f1!=NULL, f2!=NULL
 * @post applique les fonctions f1 et f2 µ\textcolor{gray}{à}µ tous les entiers dans l'intervalle fermµ\textcolor{gray}{é}µ [min,max]
 *       et retourne le nombre de fois oµ\textcolor{gray}{ù}µ ces deux fonctions donnent le mµ\textcolor{gray}{ê}µme rµ\textcolor{gray}{é}µsultat
 */
int compare(int min, int max, int f1(int), int f2(int)) {
\end{lstlisting}

\begin{solution}
\begin{lstlisting}[escapechar=µ]
int resultat = 0;
for(int i = min; i <= max; i++){
    if(f1(i) == f2(i)){
        resultat++;
    }
}

return resultat;
\end{lstlisting}
\end{solution}


\end{document}
