\documentclass[fr]{../../../../../../eplexam}

\usepackage{pgfplots}

\hypertitle{Signaux et Systèmes}{4}{FSAB}{1106}{2018}{Juin}{All}
{Martin Braquet}
{Luc Vandendorpe et Vincent Wertz}

\section{LV2}

On considère la séquence $x[n]=\cos(n\Omega_0)$ avec $\Omega_0=2\pi K/N$ où $K$ et $N$ sont des entiers non nuls.

\begin{enumerate}
 \item Calculez et donnez la transformée de Fourier $X\left(\mathrm{e}^{j\Omega}\right)$de $x[n]$.
 \item On calcule ensuite la TFD de $x[n]$ en $N$ points. Donnez les valeurs $X\left(\mathrm{e}^{j\Omega_l}\right)$ ainsi obtenues. Expliquez.
 \item Aux $N$ valeurs de $x[n]$ utilisées pour le calcul de la TFD au point précédent, on ajoute $N$ valeurs nulles. On dispose donc d'un vecteur de taille $2N$ dont les $N$ dernières valeurs sont nulles. On calcule une TFD de taille $2N$ à partir de ce vecteur. Les valeurs obtenues sont notées $Y[l]$ pour $0\leqslant l \leqslant 2N-1$. Calculez les valeurs $Y[l]$. Expliquez.
 \item Pour les valeurs $K=3$ et $N=8$, représentez (approximativement) sur la figure \ref{graphe}, \textbf{dans l'intervalle} $\mathbf{0-2}\boldsymbol{\pi}$, les valeurs $X\left(\mathrm{e}^{j\Omega_l}\right)$ et $Y[l]$ obtenues.
\end{enumerate}

\begin{figure}[!h]
 \centering
 \begin{tikzpicture}
  \draw [<->,thick] (0,3) node (yaxis) [above] {}
        |- (6,0) node (xaxis) [left, below] {};
  \draw (-1,0) -- (0,0) -- (0,-1);
  \foreach \x in {1,...,10}
     		\draw (\x/2,1pt) -- (\x/2,-3pt);
  \node[below=0.2cm] at (5,0) {$2\pi$};
 \end{tikzpicture}
 \caption{}
 \label{graphe}
\end{figure}

\end{document}
