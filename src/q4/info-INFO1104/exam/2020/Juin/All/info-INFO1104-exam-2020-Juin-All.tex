\documentclass[fr]{../../../../../../eplexam}
\usepackage{listings}
\definecolor{codegreen}{rgb}{0,0.6,0}
\definecolor{codegray}{rgb}{0.5,0.5,0.5}
\definecolor{codepurple}{rgb}{0.58,0,0.82}
\definecolor{backcolour}{rgb}{1.0,1.0,1.0}
\definecolor{codeblue}{rgb}{0,0,0.8}

\lstdefinestyle{mystyle}{
    backgroundcolor=\color{backcolour},   
    commentstyle=\color{codegray},
    keywordstyle=\color{codeblue},
    numberstyle=\tiny\color{codegray},
    stringstyle=\color{codeblue},
    basicstyle=\ttfamily\footnotesize,
    breakatwhitespace=false,         
    breaklines=true,                 
    captionpos=b,                    
    keepspaces=true,                 
    numbers=left,                    
    numbersep=5pt,                  
    showspaces=false,                
    showstringspaces=false,
    showtabs=false,                  
    tabsize=2,
    frame=shadowbox
}
\lstset{style=mystyle}
\lstset{language=Oz}

\hypertitle{Concepts des langages de programmation}{4}{LINFO}{1104}{2020}{Juin}{All}
{Norah Habets \and Thomas Debelle}
{P. Van Roy}

\section{Programming paradigms}
This question concerns the programming paradigms we have seen in the
course. Which of the following programming paradigms are deterministic (no observable
nondeterminism)? Select all that apply, multiple responses are possible
% Insérer ci-dessous la solution à la question

\begin{solution}
 \begin{itemize}
    \item Functional programming: Oui car pas de concurrence.
    \item Object-oriented programming: Non car permet un état mutable.
    \item Functional dataflow: Oui car concurrent mais le résultat est toujours le même quelqu'en soit l'ordre d'exéctuion.
    \item Actor dataflow: Non car ports.
    \item active object: Non car ports.
 \end{itemize}
\end{solution}

\section{Free identifiers}
This question concerns free identifiers in a program. Given the following program fragment
What are the free identifiers in this fragment ? Select all that applyn multiple responses are possible.

\begin{lstlisting}
local Z in
  fun {F X} Z+X end
  fun {G X} fun {$ Y} X+Y end end
end
\end{lstlisting}


\begin{solution}
    Identificateur libre correspond à un élément déclaré en dehors d'une fonction/procédure mais utilisé dedans. Dans notre cas \textbf{X} et \textbf{Z}.
\end{solution}

\section{List data structure}
This question concerns the list, a recursive data structure we have defined in the
course. Which of the following terms are lists? Select all that apply, multiple responses are possible.

\begin{solution}
    \begin{tabular}{c c}
        nil & (a|nil)|b\\
        null & a|(nil|b)\\
        leaf & a|b\\
        (a|nil)|(b|nil) & nil|a|b\\
        a|(b|nil) & nil|nil
    \end{tabular}
\end{solution}

\begin{lstlisting}
 abc
\end{lstlisting}

\section{Ordered binary trees}
This question concerns ordered binary trees. Given the three tree operations defined in the course
to operate on ordered binary trees, namely \{Lookup K T\}, \{Insert K V T\}, and \{Delete K T\}. Assume they are defined correctly
in the same way we defined them in the course, where the trees have the following syntactic definition:
<obtree V> ::= leaf | tree(key:V value:V <obtree V> <obtree V>)
Execute the following code on paper (not on a computer):
A=\{Delete dog \{Insert mouse souris \{Insert cat chat \{Insert dog chien leaf\}\}\}\}

\begin{solution}
\begin{lstlisting}[escapechar=µ]
A=tree(key:mouse value:souris tree(key:cat value:chat leaf leaf) leaf)
A=tree(key:cat value:chat leaf tree(key:mouse value:souris leaf leaf))
\end{lstlisting}
\end{solution}

\section{Abstract machine semantics}
The following two questions concern the abstract machine semantics we have seen in the course. 
We will investigate the following program's semantics.

\begin{lstlisting}
local F G A B C in
  A=5
  fun {G X} A*X end
  fun {F X}
    local A in
      A={G X}
      {G A}
    end
  end
  B=10
  C={F B}
  {Browse C}
end
\end{lstlisting}
Assume the program executes until just before the call C=\{F B\}. At this point, the execution state is the following:
\begin{lstlisting}
([(C={F B},E1), ({Browse C},E1)],
 {f=(proc {$ X R} local A in A={G X} R={G A} end end,{G->g)),
 g=(proc {$ X R} R=A*X end,{A->a}), a=5, b=10, c}
\end{lstlisting}
Here E1 has a particular value that we do not specify further, and variables f and g are bound to procedure values. Now do one
execution step according to the abstract machine semantics. This starts the execution of procedure F.\\

Which of the following choices best describes the execution state after this one step? Select only one
choice.

\begin{solution}
    On met le corps de F en haute de la pile. Réponse E ou F. Ce n'est pas encore calculé à ce moment là. E2=\{X $\rightarrow$ b; R $\rightarrow$ c; G $\rightarrow$ g\}
\end{solution}

\section{Lambda Calculus}
Reduce the expression FIRST (PAIR 1 NIL) to its final form. Show all reductions. Make sure to keep the
abbreviations for 1 and NIL during these reductions, since they will not be reduced. Please put one
reduction on each line of your answer (use a bullet list)

\begin{solution}
    $(FIRST (PAIR \wedge NIL)) \longrightarrow (\lambda p ( p TRUE)(PAIR \wedge NIL)) \longrightarrow ((PAIR \wedge NIL) TRUE) \longrightarrow ((\lambda x.\lambda y . \lambda f.$(f x y) 1 NIL) TRUE) $\longrightarrow (TRUE 1 NIL) \longrightarrow (\lambda x. \lambda y . x \wedge NIL) \longrightarrow 1$
\end{solution}

\section{Partial values}
This question is about using unbound variables when building data structures. Consider this program fragment

\begin{solution}
    true $\longrightarrow$ quand Oz rencontre une variable non liée, il met l'instruction en pause et continue jusqu'à ce que la variable soit liée.
\end{solution}

\section{Data abstraction}
The following two questions are about a data abstraction that is defined as follows.
We define a counter with five operations:
\begin{enumerate}
    \item Create a new counter instance with initial count equal to integer 0.
    \item Increment the counter (value N becomes N+1)
    \item Decrement the counter (value N becomes N-1)
    \item Test if the counter is zero (return true if the value is zero and false if it is nonzero).
    \item Get the counter's value (return the value N).
\end{enumerate}
We will implement this counter in two different ways.\\ 
Give the implementation of the counter abstraction as an abstract data type (Oz code). You can assume
that any useful helper operation defined in the course exists. Try to make the code look nice (use a
numbered bullet list)

\begin{solution}
\begin{lstlisting}[escapechar=µ]
local Wrap Unwrap in
  {NewWrapper Wrap Unwrap}
  fun {Newcounter} {Wrap 0} end
  fun {Inc C} {Wrap {Unwrap C}+1} end
  fun {Dec C} {Wrap {Unwrap C}-1} end
  fun {IsZero C} {Unwrap C}==0 end
  fun {Get C} {Unwrap C} end
end
\end{lstlisting}
\end{solution}

Give the implementation for the counter abstraction as an object (Oz code). You should group the object
operations in a record (do not use class syntax). Try to make the code look nice (use a numbered bullet
list).

\begin{solution}
\begin{lstlisting}[escapechar=µ]
fun {NewConter}
  C={NewCell 0}
  proc {Inc} C:=@C+1 end
  proc {Dec} C:=@C-1 end
  fun {IsZero} @C==0 end
  fun {Get} @C end
in
  counter(inc:Inc dec:Dec isZero:IsZero get:Get)
end
\end{lstlisting}
\end{solution}

\section{Limits of deterministic dataflow}
This question is about client/servers and the limits of deterministic dataflow. Given a
client/server application with two clients and one server. Each client puts its messages on a stream. The server reads both streams
and services both clients. Whenever a client sends a message to the server, the server must respond to the message and send a
reply back to the client in a reasonable time that depends only on the message travel time between client and server, and the
server's computation time. Assume the client/server is written in deterministic dataflow, and the server code is as follows. In this
code, S1 is a stream of requests from client 1 and S2 is a stream of requests from client 2.
\begin{lstlisting}
proc {Server S1 S2}
  case (S1|S2) of (M1|T1)|(M2|T2) then
    ... (handle M1)
    ... (handle M2)
   {Server T1 T2}
  end
end
\end{lstlisting}

Explain in detail why this server code does not satisfy the specification of the client/server, as it is given
above. Give a precise sequence of operations (client requests and server actions) to show what can go
wrong.

\begin{solution}
    Si un utilisateur envoie un message au serveur et pas l'autre, alors le case ne sera pas satsifait et le message ne sera pas traité tant que l'autre utilisateur n'envoie pas de message.
    Dans ce cas, le temps de réponse ne dépend pas du temps de voyage et du temps de calcul, ce qui ne répond pas aux \textbf{spécifications}.
\end{solution}

\section{Message protocols}
The following two questions are about message protocols. We will do an RMI with callback between a client
object and a server object. We will use the following implementation. You can assume that NewPortObject2 is defined in the same
way as in the course, namely to create port objects with no internal state.

\begin{center}
\begin{minipage}{0.45\linewidth}
\begin{lstlisting}
declare
proc {ServerProc Msg}
  case Msg of calc(X Y Client) 
    then D in
    {Send Client delta(D)}
    Y=X*X+2.0*D*X+D*D+23.0
  end
end

Server={NewPortObject2 ServerProc}
\end{lstlisting}
\end{minipage}
%
\begin{minipage}{0.45\linewidth}
\begin{lstlisting}
declare
proc {ClientProc Msg}
  case Msg of work(Z) then Y in
    {Send Server calc(10.0 Y Client)}
    Z=Y+10.0
    [] delta(D) then
    D=0.1
  end
end
Client={NewPortObject2 ClientProc}
\end{lstlisting}
\end{minipage}
\end{center}

Explain what an "RMI with callback" is supposed to do: what is the precise sequence of messages that
should be exchanged between the client and server.
\begin{solution}
    Un RMI avec callback est un protocole de message dans lequel le client va demander au serveur de faire une action. Mais pour pouvoir compléter celle-ci, le serveur a besoin 
    d'une info de la part du client, donc il va envoyer un message au client en demandant cette info. Une fois que le client a répondu, le serveur peut terminer 
    ses calculs et renvoyer le résultat de l'action au client. Donc, ici, le client demande au serveur de lier $Y$ avec \{Send Server calc(10.0 Y Client)\} mais le serveur a besoin de $D$ pour calculer $Y$.
    Le serveur envoie donc un message au client pour demander de lier $D$ avec \{Send Client delta($D$)\}. Le client va donc lier $D$ et le serveur pourra donc terminer son calcul et lier $Y$. 
\end{solution}

Explain why the code given above is incorrect. Give a precise execution scenario to show the problem.
Explain how to fix the problem (explain in words what is going on and how to correct it), and explain how
to fix the code (you can give a code fragment in your answer).

\begin{solution}
    La ligne $Z = Y + 10.0$ devrait être dans le thread. Quand cette ligne est exécutée, $Y$ n'est pas lié. Donc, le programme reste bloquée sur cette ligne. Mettre cette
    instruction dans un thread permet au programe de continuer même si $Y$ n'est pas lié.
\end{solution}

\section{Memory management}

\begin{center}
\begin{minipage}{0.45\linewidth}
\begin{lstlisting}
%Program 1:
declare S P Loop in
P={NewPort S}
proc {Loop N} 
  {Send P N} {Loop N+1} 
end
{Loop 1}
\end{lstlisting}
\end{minipage}
%
\begin{minipage}{0.45\linewidth}
\begin{lstlisting}
%Program 2:
declare P Loop in
local S in
P={NewPort S}
end
proc {Loop N} {Send P N} {Loop N+1} end
{Loop 1}
\end{lstlisting}
\end{minipage}
\end{center}

One of these programs has constant active memory size and the other has a memory leak where the active memory size grows
without bounds (which may crash the computer eventually if the process is not terminated)\\

For both programs, explain what happens to its active memory during its execution and why. Your
explanation must justify how the behavior is a consequence of the port semantics and of the difference
between the declare and local statements.

\begin{solution}
    C'est le programme 1 qui a une faute de mémoire. \\
    La sémantique des ports est la suivante: pour la création d'un port: P=\{NewPort S\} donne l'environnement \{P $\rightarrow$ p, S $\rightarrow$ s\} et dnas la mémoire
    des ports, la paire $p:s$ est créée avec, $s$, une variable non liée. Quand on effectue l'opération \{Send P X\}, on crée 
    une nouvelle variable non liée $s'$, on lie $s$ à $x|s'$ et on met à jour la paire $p:s'$. \\

    Quand le port est dans un \textit{declare}, la variable $s$ sera toujours atteignable et ne sera donc pas éliminé par le garbage
    collector. Tandis que si le stream du port est dans un local, les variables $s$ de ce stream ne seront accessibles que depuis ce local. Donc une fois que le local est exécuté
    , chaque fois qu'un message est envoyé au port, la variable $s$ du stream devient \textit{inatteignable}, et est donc éliminée, gardant ainsi la consommation
    de mémoire active constante.
\end{solution}

\section{Deterministic dataflow with ports}
This question is about concurrent programming with deterministic dataflow and ports. Given
the following program which does dynamic concurrent composition. It defines a thread creation abstraction in which threads can create threads recursively, and the top level call to NewThread terminates when all subthreads have terminated.

\begin{lstlisting}
proc {NewThread P SubThread}
  proc {ZeroExit N S}
    case S of X|S2 then
      if N+X\=0 then
        {ZeroExit N+X S2}
      end
    end
  end
  S Pt={NewPort S}
in
  proc {SubThread P}
    thread
      {Send Pt 1} % This should be done before the thread creation
      {P}
      {Send Pt ~1}
    end
  end
  {SubThread P}
  {ZeroExit 0 S}
end
\end{lstlisting}

This program deliberately has a bug: the {Send Pt 1} call is done inside the new thread (to avoid a bug, it should be done before the
thread creation).\\

Explain this buggy behavior. You should give a precise scenario of multiple thread creations that shows
the buggy behavior, such that the NewThread call exits despite that there is still at least one active thread.
In your explanation, you should make explicit all the scheduler decisions that result in the buggy behavior.

\begin{solution}
    Le problème avec ce programme, c'est qu'il peut s'arrêter avant que tous les threads soient finis. En effet, le programme s'arrête quand le compteur de threads est égal à $0$.
    À chaque fois qu'un thread se termine, sa dernière instruction est de décrémenter le compteur de thread. Quand on crée un thread il faut incrémenter le compteur. Le problème avec
    le code proposé est le suivant: le scheduler décide quel thread s'exécute quand et pour combien de temps. En incrémentant le compteur de threads à l'intérieur d'un thread,
    la situation suivante peut se produire.\\
    Imaginons le compteur est à 1 et le thread en cours d'exécution est presque terminé. Le scheduler commence un nouveau thread et crée donc une nouvelle stack avec son environnement,
    mais n'exécute encore aucune instruction sur cette stack, il a juste démarré ce nouveau thread. Puis, le scheduler change de thread et termine l'exécution du thread qui se terminait.
    Le compteur de threads passe donc à 0 avant que le nouveau thread n'ait eu l'occasion d'incrémenter ce compteur. Ce derneier étant à 0, l'exécution du programme se termine, malgré
    le fait qu'il n'y ait toujours un thread en cours d'exécution.\\ 

    En mettant l'incrémentation du compteur avant le thread éviterait ce problème.
\end{solution}

\section{Erlang/OTP}
  This question is about building resilient distributed applications with Erlang/OTP. The
Erlang/OTP platform has many abstractions to construct resilient distributed applications. Assume you are
building a resilient distributed collaborative graphic editor, in which many users can work simultaneously
on the same document. This editor should always maintain a consistent document view for all users, it
should be completely robust against any kind of failure, and it should have high performance with low
perceived latency. Explain as best you can how you would structure this editor using the Erlang/OTP
abstractions we have seen in the course. Try to explain the role of as many Erlang/OTP abstractions as
possible in your answer. You will be graded on the completeness of your answer (mentioning as many as
possible the relevant Erlang abstractions among the ones we have seen in the course) and on the
coherence of your answer (how clearly you can explain the overall architecture of your editor, how all the
Erlang abstractions work together to achieve the overall goals, and what is the role of each abstraction in
achieving the overall goals). To achieve coherence, think about your answer before writing it down and
reread it before submitting the final version.
\nosolution

% Ajouter \nosolution s'il n'y a pas de solution disponible pour cette question

% \nosolution

\end{document}
