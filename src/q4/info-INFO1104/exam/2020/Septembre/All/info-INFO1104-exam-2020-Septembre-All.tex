\documentclass[fr]{../../../../../../eplexam}
\usepackage{listings}
\definecolor{codegreen}{rgb}{0,0.6,0}
\definecolor{codegray}{rgb}{0.5,0.5,0.5}
\definecolor{codepurple}{rgb}{0.58,0,0.82}
\definecolor{backcolour}{rgb}{1.0,1.0,1.0}
\definecolor{codeblue}{rgb}{0,0,0.8}

\lstdefinestyle{mystyle}{
    backgroundcolor=\color{backcolour},   
    commentstyle=\color{codegray},
    keywordstyle=\color{codeblue},
    numberstyle=\tiny\color{codegray},
    stringstyle=\color{codeblue},
    basicstyle=\ttfamily\footnotesize,
    breakatwhitespace=false,         
    breaklines=true,                 
    captionpos=b,                    
    keepspaces=true,                 
    numbers=left,                    
    numbersep=5pt,                  
    showspaces=false,                
    showstringspaces=false,
    showtabs=false,                  
    tabsize=2,
    frame=shadowbox
}
\lstset{style=mystyle}
\lstset{language=Oz}



\hypertitle{Concepts des langages de programmation}{4}{INFO}{1104}{2020}{Septembre}{All}
{Norah Habets \and Thomas Debelle}
{P. Van Roy}

\section{List data structure}
This question is about the list, an important recursive data structure that we have seen
in the course. Which of the following terms are lists? Select all that apply, multiple responses are possible.

\begin{solution}
\begin{tabular}{c c}
nil|nil|nil & a|b|nil\\
0|nil & a|b\\
nil|b|a & (a|nil)|(b|nil)\\
nil|nil & b|(a|nil)\\
(a|nil)|b & 0 \\
 & nil|0
\end{tabular}
\end{solution}

\section{Functional programming}
Define the function \{SumOdd L A\} that calculates the sum of all elements of L
in odd positions starting from the first. If L=[a1 a2 ... am] then \{SumOdd L A\} returns A+a1+a3+... (up to
the end of L). Your definition must be declarative and tail-recursive and it must use pattern matching. You
will be judged on the precise correctness of your answer: any error (including syntax errors) will be penalized.
\begin{lstlisting}
declare
fun {SumOdd L A}
  case L
  of nil then A
  [] H|nil then A+H
  [] H|T then {SumOdd T.2 A+H}
  end 
end
\end{lstlisting}

\section{Contextual environment}
\begin{lstlisting}
fun {F X}
  fun {$ Y}
    fun {$ Z}
      X+Y+Z
    end
  end
end
\end{lstlisting}
What is the contextual environment of the outermost anonymous function in this definition, i.e., the function with header fun \{\$ Y\}?

\begin{solution}
  L'environnement contextuel d'une fonction (ou d'une procédure) contient tous les identificateurs qui sont utilisés à l'intérieur de la fonction mais déclarés en dehors
	\begin{lstlisting}[escapechar=µ]
	Ec = {X µ$\rightarrow$µ x}    de {$ Y}
	\end{lstlisting}
\end{solution}

\section{Lambda calculus}
Here we use the capital letter L to represent the lambda symbol, and =DEF to represent a definition. For this question,
reduce the expression AND TRUE (NOT FALSE) to its final form. Show all intermediate reductions. For readability, keep the
abbreviations TRUE and FALSE in your reductions when they are not applied as functions (write them in full when they are
used as functions). Please put one reduction on each line of your answer.

\begin{solution}
	(AND TRUE (NOT FALSE)) $\longrightarrow$ ($\lambda p. \lambda q. p qp$ TRUE (NOT FALSE))$\longrightarrow$ (TRUE (NOT FALSE) TRUE) $\longrightarrow$ ($\lambda x.\lambda y.x$(NOT FALSE) TRUE) $\longrightarrow$ (NOT FALSE) $\longrightarrow$ ($\lambda p.p$ FALSE TRUE FALSE) $\longrightarrow$ (FALSE FALSE TRUE) $\longrightarrow$ ($\lambda x.\lambda y.y$ FALSE TRUE) $\longrightarrow$ TRUE
\end{solution}

For this question, reduce the same expression as in the previous
question, but do the reduction in a different order. The result should be the same as before. This illustrates the Church-Rosser theorem?

\begin{solution}
	(AND TRUE (NOT FALSE)) $\longrightarrow$ (AND TRUE ($\lambda p.p$ FALSE TRUE FALSE)) $\longrightarrow$ (AND TRUE (FALSE FALSE TRUE)) $\longrightarrow$ (AND TRUE ($\lambda x. \lambda y.y$ FALSE TRUE)) $\longrightarrow$ (AND TRUE TRUE) $\longrightarrow$ ($\lambda p. \lambda q.pqp$ TRUE TRUE) $\longrightarrow$ (TRUE TRUE TRUE) $\longrightarrow$ ($\lambda x. \lambda y.x$ TRUE TRUE) $\longrightarrow$ TRUE
\end{solution}


\section{Abstract machine semantics}
\begin{lstlisting}[escapechar=µ]
([({P A B},{Pµ$\rightarrow$µp,Aµ$\rightarrow$µa,Bµ$\rightarrow$µb})], {a=5,n=10,b,p=(proc {$ X Y} Y=X+N end,{Nµ$\rightarrow$µn})} 
\end{lstlisting}
Which of the following choices best describes the execution state after doing exactly one step in the abstract machine?
Select only one choice.\par 

In case the above answer was A or B, which of the following choices best describes the value of E1?
Select only one possibility.

\begin{solution}
	\begin{lstlisting}[escapechar=µ]
	([({P A B}, {P µ$\rightarrow$µ p, A µ$\rightarrow$µ a, B µ$\rightarrow$µ b})],{a=5,n=10,b,p=(proc {$ X Y} Y=X+N end, {N µ$\rightarrow$µ n})})
	µ$\longrightarrow$µ ([(Y=X+N, µ$E_1$µ)], {a=5, n=10, b, p=proc {$ X Y} Y=X+N end, {N µ$\rightarrow$µ n}})
	µ$\longrightarrow$µ µ$E_1$µ={X µ$\rightarrow$µ a, Y µ$\rightarrow$µ b, N µ$\rightarrow$µ n} 
	\end{lstlisting}
\end{solution}

\section{List data structures}
Given the following instruction:

\begin{lstlisting}
{Browse [nil]==nil|nil}
\end{lstlisting}
What does this display?

\begin{solution}
	\begin{lstlisting}[escapechar=µ]
	{Browse [nil] == nil|nil} µ$\longrightarrow$µ true 
	\end{lstlisting}
\end{solution}

\section{Limitation of deterministic dataflow}
The deterministic dataflow paradigm is very useful when writing concurrent programs
because it has no race conditions. However, it has a strong limitation which means that in some cases it cannot be used. Consider
a client/server application with two clients and one server. Each client puts its requests on a stream. The server reads both
streams and handles the requests. There is an important fairness requirement: whenever a client sends a message to the server,
the server must respond in a reasonable time that depends only on the message travel time between client and server, and its own
computations. There must be no other dependency. Assume the client/server is written in deterministic dataflow, and the server
code is as follows. In this code, S1 is a stream of requests from client 1 and S2 is a stream of requests from client 2.

\begin{lstlisting}
proc {Server S1 S2}
  case (S1|S2) of (M1|T1)|S2 then
  ... (handle M1)
  {Server T1 S2}
  [] S1|(M2|T2) then
  ... (handle M2)
  {Server S1 T2}
  end
end
\end{lstlisting}

Explain precisely why the server code given above does not satisfy the fairness requirement. Give a
concrete scenario of client and server operations to show what can go wrong.

\begin{solution}
Cela ne fonctionne pas si le client 2 envoie un message et que le client 1 n'envoie rien. C'est la sémantique du case qui fait que ça ne fonctionne pas. Le deuxième pattern est pris quand on ne peut pas faire réussir le premier. Cette procédure va donc toujours attendre le premier pattern (que le client 1 envoie un message) et ne va jamais passer au deuxième. Si on échangeait les patterns, ça sera l'inverse. De plus, le terme de "\textit{fairness}" est dit d'un scheduler, si chaque runnable thread sera exécuté dans un temps fini, ce qui n'est donc pas le cas ici.
\end{solution}

One might think that writing the server in a different way (but still in deterministic dataflow) would solve the
problem. But this is not possible. For this question, give the general reason why it is not possible to
satisfy the fairness requirement in deterministic dataflow.

\begin{solution}
En dataflow déterministe, on ne peut écrire que des programmes sans non déterminisme \textit{observable}. Or, dans le cas de l'application \textbf{serveur|client}, le non déterminisme est inséparable de l'application. En effet, il n'est pas possible de prévoir quand un client va appeler le serveur. Il faut donc être capable d'écrire un programme qui puisse attendre sur plusieurs pattern, ce qui n'est pas possible en dataflow déterministe.
\end{solution}


Explain how to write a client/server that does satisfy the fairness requirement. You first need to introduce
a new concept, since pure deterministic dataflow cannot solve the problem. Give a simple code
implementation of a server and a simple code implementation of a client. Your code can be extremely
simple, as long as it shows how to satisfy the fairness requirement.

\begin{solution}
Pour pouvoir écrire une application du style client serveur, il faut introduire le concept de \textbf{port}, c'est à dire un stream \textit{nommé}.
\begin{lstlisting}[escapechar=µ]
P={NewPort S}
proc {Server S}
  case S
  of M|T then
    (handle M)
    {Server T}
  end 
end
\end{lstlisting}
On envoie un message au serveur via \{Send P M\} où M est le message et P notre port/stream nommé.
\end{solution}

\section{Nondeterminism} 
Is the following program nondeterministic ?
\begin{lstlisting}
declare X={NewCell 1}
thread X:=2 end
thread X:=2 end
\end{lstlisting}

\begin{solution}
Oui et est donc dangereux. On ne sait pas quel X:=2 sera exécuté en premier et X:= n'est pas une opération \textbf{atomique} donc il est possible qu'un thread "sélectionné" X puis soit arrêté par le scheduler qui va donner du temps de calcul à l'autre thread. Donc cela va modifier X et le X "tenu" par le premier thread ne sera plus le bon.
\end{solution}

\section{Port objects}
In the course we have seen stateful port objects. It is remarkable that they can be defined as follows with the
FoldL operation:

\begin{lstlisting}
fun {NewPortObject Init F}
P Out in
  thread S in
    P={NewPort S} Out={FoldL S F Init}
  end
  P
end
\end{lstlisting}

Instead of using a local function Loop to read the message stream and update the state, we use FoldL. For this question,
explain how this definition works. Your answer should start with a specification of the FoldL operation and show how this
specification can be used to do the work of the port object's state transition function.

\begin{solution}
La fonction FoldL permet d'appliquer une fonction sur tous les éléments d'une liste. Ici, pour passer d'un état à un autre, il faut appliquer une fonction sur les éléments du stream de l'objet. Puisqu'un stream est  une liste avec la dernière variable \textit{non liée}, la fonction FoldL est parfaite.
\end{solution}

\section{Mutable state}
In the course, we introduced mutable state in the form of cells, which are a form of
variable that can be updated multiple times. This is no longer compatible with functional programming,
but it enables new abilities that are not possible in functional programming. For this question, give and
explain an example of a new ability that becomes available with mutable state that was not available
without mutable state.

\begin{solution}
L'ajout de cellules est nécessaire pour ajouter de la modularité dans un programme c'est à dire changer des fonctions sans pour autant changer leur interface. Si, par exemple, on veut ajouter un compteur dans une fonction pour garder une trace de combien de fois elle est appelée lors de l'exécution d'un programme, en programmation fonctionnelle, il faudrait rajouter un argument à cette fonction et donc modifier le code partout où cette fonction est appelée. Avec les cellules ce n'est pas nécessaire. En programmation fonctionnelle, il n'y a pas d'état, le programme ne se souvient pas de ce qui s'est passé avant. Avec les cellules, il est possible de créer des programmes avec l'état, qui ont donc une mémoire interne. Un programme peut changer son comportement en fonction de ses états passés.
\end{solution}

\section{Memory management}
This question is about memory management. You are given the following two programs.
\begin{center}
\begin{minipage}{0.45\linewidth}
\begin{lstlisting}
%Program 1:
declare S P Loop in
P={NewPort S}
proc {Loop N} 
  {Send P N} {Loop N+1} 
  end
{Loop 1}
\end{lstlisting}
\end{minipage}
%
\begin{minipage}{0.45\linewidth}
\begin{lstlisting}
%Program 2:
declare P Loop in
local S in
  P={NewPort S}
end
proc [Loop N} {Send P N} {Loop N+1} end
{Loop 1
\end{lstlisting}
\end{minipage}
\end{center}
One of these programs has constant active memory size and the other has a memory leak where the active memory size grows
without bounds (which may crash the computer eventually if the process is not terminated).\\
To understand the behavior of these programs, it is essential to understand the port semantics. Given the
port semantics, explain for both program what happens to its active memory during its execution and
why. Your answer must show how the behavior is a consequence of port semantics and of the difference
between the declare and local statements.

\begin{solution}
    C'est le programme 1 qui a une faute de mémoire. \\
    La sémantique des ports est la suivante: pour la création d'un port: P=\{NewPort S\} donne l'environnement \{P $\rightarrow$ p, S $\rightarrow$ s\} et dnas la mémoire
    des ports, la paire $p:s$ est créée avec, $s$, une variable non liée. Quand on effectue l'opération \{Send P X\}, on crée 
    une nouvelle variable non liée $s'$, on lie $s$ à $x|s'$ et on met à jour la paire $p:s'$. \\

    Quand le port est dans un \textit{declare}, la variable $s$ sera toujours atteignable et ne sera donc pas éliminé par le garbage
    collector. Tandis que si le stream du port est dans un local, les variables $s$ de ce stream ne seront accessibles que depuis ce local. Donc une fois que le local est exécuté
    , chaque fois qu'un message est envoyé au port, la variable $s$ du stream devient \textit{inatteignable}, et est donc éliminée, gardant ainsi la consommation
    de mémoire active constante.
\end{solution}

\section{Erlang/OTP}
This question is about building robust distributed applications using the Erlang/OTP
platform. One of the key concepts in Erlang/OTP is the supervisor tree, where one process is able to
observe and manage a set of other processes\par

Explain the one\_for\_one and one\_for\_all restart strategies, and give for each strategy a simple application
scenario where it would be useful.

\begin{solution}
ces stratégies sont utilisées quand un superviseur supervise plusieurs processus:
\begin{itemize}
\item One\_for\_one: quand un des enfants crash, il est simplement redémarré par le superviseur sans que ça n'impacte les autres processus (\textit{ex}: client serveur: pour pas impacter les autres clients si il y a un crash)
\item One\_for\_all: si un des enfants crash, tous les processus liés par le même superviseur sont arrêtés et relancés, même s'ils n'ont aucun problème (\textit{ex}: processus synchronisé: nécessaire quand les processus sont dépendants, pour qu'ils restent synchronisés)
\end{itemize}
\end{solution}

% Ajouter \nosolution s'il n'y a pas de solution disponible pour cette question

% \nosolution

\end{document}
