\documentclass[fr]{../../../../../../eplexam}
\usepackage{listings}
\definecolor{codegreen}{rgb}{0,0.6,0}
\definecolor{codegray}{rgb}{0.5,0.5,0.5}
\definecolor{codepurple}{rgb}{0.58,0,0.82}
\definecolor{backcolour}{rgb}{1.0,1.0,1.0}
\definecolor{codeblue}{rgb}{0,0,0.8}

\lstdefinestyle{mystyle}{
    backgroundcolor=\color{backcolour},   
    commentstyle=\color{codegray},
    keywordstyle=\color{codeblue},
    numberstyle=\tiny\color{codegray},
    stringstyle=\color{codeblue},
    basicstyle=\ttfamily\footnotesize,
    breakatwhitespace=false,         
    breaklines=true,                 
    captionpos=b,                    
    keepspaces=true,                 
    numbers=left,                    
    numbersep=5pt,                  
    showspaces=false,                
    showstringspaces=false,
    showtabs=false,                  
    tabsize=2,
    frame=shadowbox
}
\lstset{style=mystyle}
\lstset{language=Oz}
\hypertitle{}{4}{INFO}{1104}{2022}{Septembre}{All}
{Norah Habets \and Thomas Debelle}
{P. Van Roy}

\section{Kernel Language}
\begin{center}
\begin{minipage}{0.45\linewidth}
Translate the following program into kernel language. Be careful to do a complete translation that uses only kernel language instructions !
\end{minipage}
%
\begin{minipage}{0.45\linewidth}
\begin{lstlisting}
local F C in
  C={NewCell 1}
  fun {F A}
    @A
  end
  C:={F C}+1
end
\end{lstlisting}
\end{minipage}
\end{center}

\begin{solution}
\begin{lstlisting}[escapechar=µ]
local F C One C1 in One=1
  {NewCell One C}
  F=proc {$ A ?R1}
    R1=@A
  end
  {F C ?R2}
  C1=R2+One
  C:=C1
end
\end{lstlisting}
\end{solution}

\section{Basic concepts}
\subsection{Scope}
Define "scope" in a programming language. Give an example.

\begin{solution}
La portée d'un identificateur est la partie où l'occurrence se réfère à la même déclaration de variable
\begin{lstlisting}[escapechar=µ]
local
  X
in
  X=42 {Browse X} %Affiche 42
  local
    X
  in
    X=11 {Browse X} %Affiche 11
  end 
  {Browse X} %Affiche 42
end
\end{lstlisting}
\end{solution}

\subsection{Contextual environment}
Define "contextual environment" of a procedure. Give an example

\begin{solution}
L'environnement contextuel une fonction (ou procédure) contient tous les identificateurs qui sont utilisés dans la fonction mais déclarés en dehors.
\begin{lstlisting}[escapechar=µ]
declare
fun {Compose F G}
  fun {$ X}{F {G X}} end
end
\end{lstlisting}
Où E = \{F $\rightarrow$ f, G $\rightarrow$ g\}
\end{solution}

\section{Lambda calculus}
Reduce the expression (PLUS 1 (IFTHENELSE (ISZERO 1) O I)) to its simplest form. Show all reduction steps. To make your answer easier to understand, you must only replace a function by its definition when you need it for a reduction step. If you do not do this, there wüll be a penalty.

\begin{solution}
\begin{itemize}
\item (ISZERO 1) $\longrightarrow$ ($\lambda n.(n(\lambda x.$FALSE) TRUE 1) $\longrightarrow$ (1 ($\lambda x.$FALSE) TRUE) $\longrightarrow$ ($\lambda f.(\lambda x$(f x))($\lambda x.$FALSE) TRUE) $\longrightarrow$ (($\lambda x.$ FALSE) TRUE) $\longrightarrow$ (($\lambda x. \lambda x. \lambda y.y$) TRUE) $\longrightarrow$ $\lambda x. \lambda y.y$  $\longrightarrow$ FALSE
\item (IFTHENELSE (ISZERO 1) 0 1) $\longrightarrow$ (IFTHENELSE FALSE 0 1) $\longrightarrow$ ($\lambda p. \lambda a. \lambda b.$(p a b) FALSE 0 1) $\longrightarrow$ (FALSE 0 1) $\longrightarrow$ ($\lambda x. \lambda y.y$ 0 1) $\longrightarrow$ 1
\item (PLUS 1 (IFTHENELSE (ISZERO 1) 0 1)) $\longrightarrow$ (PLUS 1 1) $\longrightarrow$ ($\lambda m. \lambda n. \lambda f. \lambda x.$(m f)((n f) x) 1 1) $\longrightarrow$ ($\lambda f. \lambda x.$(1 f)((1 f) x)) $\longrightarrow$ ($\lambda f. \lambda x.$(f (f x))) $\longrightarrow$ 2
\end{itemize}
\end{solution}

\section{Functional object}
Give a code example of a functional Object. Calling a functional Object returns another Object with a new value inside. For example, creates the counter object Cl with initial count 0. Then returns a new counter Object C2 with count 1, and returns a new counter Object C3 with count 2. Hints: A functional Object is purely declarative; cells are not allowed. You need to define an auxiliary function \{CounterObject I\} that returns a counter Object with count I. A counter Object is a record with an 'inc' field that contains an increment function.

\begin{solution}
\begin{lstlisting}[escapechar=µ]
declare
fun {NewCounter}
  fun {CounterObject I}
    fun {Inc} {CounterObject I+1} end
    fun {Get} I end %Optionnel
  in
    counter(inc:Inc get:Get)
  end
in
  {CounterObject 0}
end
\end{lstlisting}
\end{solution}


\section{Nondeterminism}
\subsection{Definition of nondeterminism}
Define the concept of nondeterminism in a programming language. Be careful to give a precise definition! It is a bit subtle. Give an example of a program that shows nondeterminism.

\begin{solution}
Non-déterminisme correspond à la capacité d'un système à faire des  décisions indépendamment du développeur.
\begin{lstlisting}
declare X1 X2
X1=1 X2=2
thread {Browse X1} end
thread {Browse X2} end
\end{lstlisting}
On ne sait pas si X1 va être affiché en premier ou non.
\end{solution}


\subsection{Overcoming the limitation of deterministic dataflow}
Deterministic dataflow has the strong property that it has no observable nondeterminism. For this question, first give a specification of a client/server application, second use this specification to explain Why the client/server be written as a deterministic dataflow program. Third, briefly define the concept of a port and fourth show in a few lines of code how to write a client/server with a port.

\begin{solution}
Sans non-déterminisme observable, le résultat est toujours le même. Or, dans une application Client|Serveur, le résultat dépend de quand on reçoit le message. On reçoit un message, on compile et on répond. Il y a donc d'office du non-déterminisme. Le résultat est choisi par le scheduler et non par le développeur.\par
Un port est un stream:
\begin{lstlisting}[escapechar=µ]
P={NewPort S} %Crµ\textcolor{gray}{é}µe un port avec un stream
{Send P N} %Ajoute N au stream
\end{lstlisting}
Client Serveur avec un port. Les clients envoient et reçoivent des message. Les clients sont des agents concurrents. Le serveur reçoit un message, fait un calcul en local et envoie une réponse immédiate.
\end{solution}

\section{Erlang mechanisms for long-lived programs}

\subsection{Process linking}
An Erlang process is similar to a port Object. An important primitive operation to support fault tolerance in Erlang is linking, where processes are linked together. For this question, define linking and explain what it does. How does it work when there are more than processes?

\begin{solution}
Link crée un lien entre 2 processus qui est bidirectionnel dans les 2 sens. Lorsqu'un process s'arrête il envoie le signal "normal" à tous les processus auxquels il est lié, sinon il envoie la cause de son arrêt \textit{anormal}. Les processus qui reçoivent un message "anormal" crash aussi et renvoient le message causant le crash.\par 
Un process peut être configuré pour piéger les signaux de sorte en appelant \textit{process\_flag}(trap\_exit, true)\par 
Un superviseur peut recevoir dans sa mailbox le message sans s'arrêter et prendre des mesures pour restaurer le système. \par 
Ainsi, tous les processus liés au crash en suivant la chaine jusqu'à arriver au superviseur. Si aucun process ne démarre, le programme s'arrête. On peut outrepasser un superviseur avec \textbf{exit(kill, Pid)}.
\end{solution}

\subsection{Dynamic code change}
In Erlangitis possible to update the code without stopping the system. Erlang has two basic mechanisms to allow this. For this question, explain the two mechanisms and give small code fragments in Erlang to illustrate them (as we did in the course). Hint: one of the mechanisms needs to be supported by the implementation, whereas the other mechanism is a consequence of functional programming.

\begin{solution}
\begin{minipage}[t]{0.45\linewidth}
\begin{lstlisting}[language=erlang, caption={use new version}]
-module(m). 

loop(Data, F) -> 
  receive
  (From,Q} ->
    {Reply,Data1}=F(Q,Data), 
    m:loop(Data1, F)
end.
\end{lstlisting}
\end{minipage}
%
\begin{minipage}[t]{0.45\linewidth}
\begin{lstlisting}[language=erlang, caption={use old version}]
-module(m). 

loop(Data, F) -> 
  receive
  {From,Q} ->
     {Reply,Data1}=F(Q,Data), 
     loop(Data1, F)
end.
\end{lstlisting}
\end{minipage}
\end{solution}

\end{document}
