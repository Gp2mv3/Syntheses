\documentclass[en]{../../../../../../epltest}
\usepackage{../../../../../../eplunits}
\usepackage{circuitikz}
\usepackage{steinmetz}

\hypertitle{circmes-ELEC1370}{4}{ELEC}{1370}{2017}{Mars}{All}
{Louis Devillez}
{Christophe Craeye \and Bruno Dehez \and Oestges Claude}
\section{Test A}
\nosolution

\section{Test B}
On considère le circuit suivant, opérant à $\SI{2}{\kilo\hertz}$.

\begin{center}
	\begin{circuitikz}[american]
		\draw (0,2) to [R,l_= 2R] (0,0) -- (2,0) to[I=$2\phase{0^\circ}$,*-*] (2,2) -- (0,2);
		\draw (2,2) to[V=$12\phase{0^\circ}$] (4,2) to[C=$\SI{-j}{\ohm}$,*-*] (4,0) -- (2,0);
		\draw (4,2) to[L = $\SI{j}{\ohm}$] (6,2) to[R = R,*-*] (6,0) -- (4,0); 
		\draw (6,2) to[short,-*] (8,2) node[below]{+};
		\draw (6,0) to[short,-*] (8,0) node[above]{-};
		\draw (8,1)node[]{$V_0$};
		\draw [dashed](7,-0.5) rectangle(-1,3);
	\end{circuitikz}
\end{center}
\begin{itemize}
	\item (2 points) Si $R = \SI{0.5}{\ohm}$, quel est l'équivalent de Norton du circuit encadré vu à ses bornes ?
	\item (8 points) Si on mesure une tension $V_0 = 5.06\angle{-71.56^\circ}V$ (amplitude et phase), calculer:
	\subitem le courant (amplitude et phase) dans la capacité;
	\subitem la valeur de la résistance R;
	\subitem la tension (amplitude et phase) aux bornes de la source de courant.
\end{itemize}

\begin{solution}
	\begin{itemize}
		\item Pour trouver $R_{eq}$, on annule les sources.
		
		\begin{center}
			\begin{circuitikz}
				\draw (2,0) -- (0,0) to[R,l_= $\SI{1}{\ohm}$] (0,2) -- (2,2) to[C=$\SI{-j}{\ohm}$,*-*] (2,0) -- (4,0)  to[R,l_= $\SI{0.5}{\ohm}$,*-*] (4,2);
				\draw (2,2) to[L = $\SI{j}{\ohm}$,] (4,2) to[short,-*] (6,2)node[below]{+};
				\draw (4,0) to[short,-*] (6,0)node[above]{-};
				\draw (6,1)node[]{$V_0$};			\end{circuitikz}
		\end{center}
		
		La résistance de gauche est en $\parallel$ avec la capacité
		$$Z_1 = \cfrac{1}{\dfrac{1}{1}+ j} = 0.707\angle-45^\circ $$
		
		$Z_1$ est en série avec l'inductance
		$$Z_2 = Z_1 + j = 0.707\angle\ang{45} $$
		
		$Z_2$ est en $\parallel$ avec la résistance de droite
		$$R_{eq} = \frac{1}{\cfrac{1}{Z_2} + \cfrac{1}{0.5}} = 0.316\angle\ang{18.43}$$
		
		On va maintenant chercher $V_0$
		
		soit
		$$Z = \cfrac{1}{j\omega C+\dfrac{1}{R+j\omega L}} = 2.23\angle\ang{-26.56} $$
		
		$$ I_{source} = I_{2R} + I_Z$$
		$$2\angle\ang{0} = \frac{V_1}{1} + \frac{V_1 + 12\angle\ang{0}}{Z} $$
		$$V_1(1 + \frac{1}{Z}) = 2\angle\ang{0} - \frac{12\angle\ang{0}}{Z}$$
		$$V_1 = 2.61\angle\ang{-147} $$
		$$V_0 = \frac{V_1 + 12\angle\ang{0}}{j \omega L + R}R = 4.43\angle\ang{-71.57}$$
		
		Il ne nous reste plus qu'à trouver $I_N$ avec $V_0$ ($V_T$) et $R_{eq}$
		$$I_N = \frac{V_0}{R_{eq}} = 14\angle\ang{-90} $$
		
		\begin{center}
			\begin{circuitikz}[american]
				\draw (3,0) to[short,*-] (0,0) to[I=$14\angle\ang{-90}$] (0,2) to[short,-*] (3,2);
				\draw (2,2) to [generic =$ 0.316\angle\ang{18.43}$,*-*] (2,0);
			\end{circuitikz}
		\end{center}
		\qquad
		
		\item Écrivons nos différentes équations:
		\subitem $I_L = \cfrac{V_0}{R}$
		\subitem $I_S = 2\angle\ang{0}$
		\subitem $V_1 = V_2 - 12\angle\ang{0}$
		\subitem $V_2 = \cfrac{V_0 (R+j)}{R}$
		\subitem $I_C = \cfrac{V_2}{-j}$
		\subitem $I_R = \cfrac{V_1}{2R}$
		
		$$I_S = I_L + I_C + I_R$$
		$$2\angle\ang{0} =\dfrac{V_0}{R} + \cfrac{\dfrac{V_0 (R+j)}{R}}{-j} + \cfrac{\dfrac{V_0 (R+j)}{R} - 12\angle\ang{0}}{2R}$$
		$$-2R^2j * 2\angle\ang{0} = -2RV_0 j+2R V_0(R+j)-jV_0(R+j)+Rj* 12\angle\ang{0} $$
		$$(4\angle\ang{90} + 2V_0)R^2+ (12\angle\ang{90} -jV_0)R +V_0 = 0$$
		
		On tombe ainsi sur une équation du second degré en R. On trouve alors $R \approx 1$
		
		On peut calculer la tension après la source de tension:
		$$ V_2 = 7.156\angle\ang{-26.56}$$
		
		
		On peut ensuite trouver les autres inconnues. On trouve alors $I_C=7.16\angle\ang{63.44}$ et $V_1 = 6.50\angle\ang{-150}$
		
	\end{itemize}
\end{solution}

\end{document}
