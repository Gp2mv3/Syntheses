\documentclass[fr]{../../../../../../eplexam}
\usepackage[oldvoltagedirection]{circuitikz}
\usepackage{bodegraph}
\usepackage{pgfplots}
\usepackage{SIunits}
\usepackage{amsmath}
\usepackage{enumitem}
\usepackage{xcolor}


\pgfplotsset{compat=newest}
\tikzset{meter/.style={draw,thick,circle,fill=white,minimum size =0.75cm,inner sep=0pt}}
\hypertitle{circmes-ELEC1370}{4}{ELEC}{1370}{2018}{Juin}{Majeure}
{Brieuc Balon}
{Claude Oestges, Bruno Dehez and Christophe Craeye}
\section{Question Oestges : phaseurs}
Soit le circuit suivant fonctionnant à la fréquence de $\SI{1}{\kilo\hertz}$:
\begin{center}
	\begin{circuitikz} 
	\draw
	(0,0)--(6,0)
	(0,0) to [american controlled current source,l=$4I_x$](0,3)
	to[R,l_=$\SI{1}{\ohm}$,v^>=$V_x$](0,6)
	(-1,6) to[short,o-](6,6)
	(-1,3) to[short,o-](0,3)
	to[R,l_=$\SI{1}{\ohm}$](3,3)
	to[R,l_=$\SI{1}{\ohm}$,i>^=$I_o$](6,3)
	(3,6) to [C,l^=$\SI{-\imagj}{\ohm}$](3,3)
	(6,6)to[R,l_=$\SI{1}{\ohm}$,i>^=$I_x$](6,3)
	(3,0) to[american voltage source, l_=$\SI{4\angle \ang{0}{\volt}}$](3,3)
	(6,0) to[american controlled voltage source,l_=$2V_x$](6,3);
	\end{circuitikz}
\end{center}
On demande:
\begin{enumerate}
    \item Calculer le courant $I_o$ (amplitude et phase)
    \item La puissance délivrée par la source indépendante
    \item L'équivalent de Norton aux bornes de la résistance en haut à gauche (Question indépendante des autres)
\end{enumerate}
\begin{solution}
Pour la suite des calculs nous utiliserons les notations telles qu’illustrées :
\begin{center}
	\begin{circuitikz} 
	\draw
	(0,0)--(6,0)
	(0,0) to [american controlled current source,l=$4I_x$](0,3)
	to[R,l_=$\SI{1}{\ohm}$,v^>=$V_x$,i_<=$I_1$](0,6)
	(-1,6) to[short,o-](6,6)
	(-1,3) to[short,o-*](0,3)
	to[R,l_=$\SI{1}{\ohm}$,i>_=$I_3$,v^=$V_3$](3,3)
	to[R,l_=$\SI{1}{\ohm}$,i_>=$I_o$,v^<=$V_o$](6,3)
	(3,6) to [C,l^=$\SI{-\imagj}{\ohm}$,i<^=$I_2$,v>=$V_2$,*-](3,3)
	(6,6)to[R,l_=$\SI{1}{\ohm}$,i>_=$I_x$,v^<=$V_{x^{'}}$](6,3)
	(3,0) to[american voltage source, l_=$\SI{4\angle \ang{0}{\volt}}$,i>_=$I_S$](3,3)
	(6,0) to[american controlled voltage source,l_=$2V_x$](6,3)
	(3,6) node[above,color = orange]{$A$}{}
	(0,3)node[below left,color = teal]{$B$}{}
	%Flèches
    (4.7,4.5) node[scale=2,color=brown]{$\circlearrowright$}{}
    (1.3,4.5)node[scale=2,color=red]{$\circlearrowright$}{}
    (4.8,1.5)node[scale=2,color=blue]{$\circlearrowright$}{};
	\end{circuitikz}
\end{center}
\begin{enumerate}
    \item Nous pouvons effectuer 3 lois des mailles sur les mailles indiquées et 2 lois des noeuds sur les 2 noeuds indiqués :
\begin{equation*}
    \begin{dcases}
            \textcolor{teal}{0 = I_1-I_3+4I_x}\\
            \textcolor{orange}{I_2 = I_1+I_x}\\
            \textcolor{red}{\imagj I_2 = I_1+I_3}\\
            \textcolor{brown}{I_x = I_o+\imagj I_2}\\
            \textcolor{blue}{4 = I_o +2V_x}\\
            V_x = I_1\\
    \end{dcases}
    \Leftrightarrow 
    \begin{bmatrix}
         0 & 1 & 0 &-1 & 4 \\
         0 & 1 & -1 & 0 & 1\\
         0 & 1 & -\imagj & 1 & 0\\
         1 & 0 & \imagj  & 0 & -1\\
         1 & 2 & 0 & 0 & 0\\
    \end{bmatrix}
    \begin{bmatrix}
        I_o\\
        I_1\\
        I_2\\
        I_3\\
        I_x\\
    \end{bmatrix}
    =
    \begin{bmatrix}
        0\\
        0\\
        0\\
        0\\
        4\\
    \end{bmatrix}
    \Leftrightarrow
    \begin{dcases}
        I_o = \frac{4}{3}\angle\ang{-126.869} [A]\\
        I_1 = 2.45\angle\ang{12.529} [A]\\
        I_2 = 1.1925\angle\ang{26.565} [A]\\
        I_3 = 2.9814\angle\ang{169.695} [A]\\
        I_x = \frac{4}{3} \angle\ang{180} [A]\\
    \end{dcases}
\end{equation*}
Le courant $I_o=\frac{4}{3}\angle\ang{-126.869}$ [A]
\item La puissance\footnote{Les phaseurs expriment la valeur efficace de la tension/courant qu'ils représentent.} délivrée par la source indépendante peux s'écrire comme : 
\begin{align*}
    P &= \Re(V\cdot I_S^\ast) ~~\mbox{avec}~~I_S = I_0+I_2-I_3 = 3.373\angle\ang{-18.43} [A]\\
    &=\Re(4\angle 0^\circ \cdot 3.373\angle\ang{18.43}) = 12.8[W]
\end{align*}

\item Pour calculer l'équivalent de Norton aux bornes de la résistance, nous allons calculer le courant $I_N$ et la tension $V_{Th}$\footnote{Étant donné que nous avons des sources dépendantes nous ne pouvons pas calculer la résistance équivalente $Z_{eq}$.}. \\
Pour calculer le courant $I_N$, le circuit peut être représenté comme : 
\begin{center}
	\begin{circuitikz} 
	\draw
	(0,0)--(6,0)
	(0,0) to [american controlled current source,l=$4I_x$](0,3)
	to[short,i<_=$I_N$](0,6)
	(-1,6) to[short,o-](6,6)
	(-1,3) to[short,o-*](0,3)
	to[R,l_=$\SI{1}{\ohm}$,i>_=$I_3$,v^=$V_3$](3,3)
	to[R,l_=$\SI{1}{\ohm}$,i_>=$I_o$,v^<=$V_o$](6,3)
	(3,6) to [C,l^=$\SI{-\imagj}{\ohm}$,i<^=$I_2$,v>=$V_2$,*-](3,3)
	(6,6)to[R,l_=$\SI{1}{\ohm}$,i>_=$I_x$,v^<=$V_x$](6,3)
	(3,0) to[american voltage source, l^=$\SI{4\angle\ang{0}}{\volt}$,i>_=$I_S$](3,3)
	(6,0) --(6,3)
	(3,6) node[above,color = orange]{$A$}{}
	(0,3)node[below left,color = teal]{$B$}{}
	%Flèches
    (4.7,4.5) node[scale=2,color=brown]{$\circlearrowright$}{}
    (1.3,4.5)node[scale=2,color=red]{$\circlearrowright$}{}
    (4.5,1.5)node[scale=2,color=blue]{$\circlearrowright$}{};
	\end{circuitikz}
\end{center}
Nous pouvons effectuer 3 lois des mailles sur les mailles indiquées et 2 lois des noeuds sur les 2 noeuds indiqués :
\begin{equation*}
    \begin{dcases}
            \textcolor{teal}{I_3 = I_N+4I_x}\\
            \textcolor{orange}{I_2 = I_N+I_x}\\
            \textcolor{red}{I_3=\imagj I_2}\\
            \textcolor{brown}{I_o =I_x-\imagj I_2}\\
            \textcolor{blue}{I_o = 4\angle\ang{0}}\\
    \end{dcases}
    \Leftrightarrow 
    \begin{bmatrix}
         0 & 1 & -4 & -1 \\
         1 & 0 & -1 & -1 \\
         -\imagj  & 1 & 0 & 0 \\
         -\imagj  & 0 & 1 & 0 \\
    \end{bmatrix}
    \begin{bmatrix}
        I_2\\
        I_3\\
        I_x\\
        I_N\\
    \end{bmatrix}
    =
    \begin{bmatrix}
        0\\
        0\\
        0\\
        4\\
    \end{bmatrix}
    \Leftrightarrow
    \begin{dcases}
        I_2 = 5.366\angle\ang{116.565} [A]\\
        I_3 = 5.366\angle\ang{-153.435} [A]\\
        I_x = 2.529\angle\ang{-108.435} [A]\\
        I_N = 7.375\angle\ang{102.529} [A]\\
    \end{dcases}
\end{equation*}
\begin{equation*}
    I_N = 7.375\angle\ang{102.529} = -\frac{8}{5}+\frac{36}{5}\imagj  [A]
\end{equation*}
Pour calculer la tension $V_{Th}$, le circuit peut être représenté comme :
\begin{center}
	\begin{circuitikz} 
	\draw
	(0,0)--(3,0)to[short,i=$I_x$](6,0)
	(0,0) to [american controlled current source,l=$4I_x$](0,3)
	(-1,3) to[open,v^>=$V_{Th}$](-1,6)
	(-1,6) to[short,o-](6,6)
	(-1,3) to[short,o-*](0,3)
	to[R,l_=$\SI{1}{\ohm}$,v^=$V_3$](3,3)
	to[R,l_=$\SI{1}{\ohm}$,i_>=$I_o$,v^<=$V_o$](6,3)
	(3,6) to [C,l^=$\SI{-\imagj}{\ohm}$,v>=$V_2$,*-](3,3)
	(6,6)to[R,l_=$\SI{1}{\ohm}$,v^<=$V_{x^{'}}$](6,3)
	(3,0) to[american voltage source, l^=$\SI{4\angle\ang{0}}{\volt}$,i>_=$I_S$](3,3)
	(6,0) to[american controlled voltage source,l_=$2V_{Th}$](6,3)
	%Flèches
    (4.7,4.5) node[scale=2,color=brown]{$\circlearrowright$}{}
    (4.5,1.5)node[scale=2,color=blue]{$\circlearrowright$}{};
	\end{circuitikz}
\end{center}
On peut remarquer que :
\begin{equation*}
    V_{Th} = -(4I_x -\imagj I_x)=I_x(\imagj -4)
\end{equation*}
Nous pouvons effectuer 2 lois des mailles sur les mailles indiquées : 
\begin{equation*}
    \begin{dcases}
            \textcolor{brown}{I_o = I_x(1-\imagj )}\\
            \textcolor{blue}{4=I_o+2V_{Th}}\\
    \end{dcases}
    \Leftrightarrow 
    \begin{dcases}
            \textcolor{brown}{0 = I_o +I_x (\imagj -1)}\\
            \textcolor{blue}{4= I_o+2(\imagj -4)I_x}\\
    \end{dcases}
    \Leftrightarrow
    \begin{bmatrix}
         1 & \imagj -1 \\
         1 & 2(\imagj -4)\\
    \end{bmatrix}
    \begin{bmatrix}
        I_o\\
        I_x\\
    \end{bmatrix}
    =
    \begin{bmatrix}
        0\\
        4\\
    \end{bmatrix}
    \Leftrightarrow
    \begin{dcases}
        I_o = \frac{4}{5}\angle\ang{143.13} [A]\\
        I_x = 0.565\angle\ang{-171.87} [A]\\
    \end{dcases}
\end{equation*}
\begin{equation*}
    V_{Th} =I_x(\imagj -4) = 2.33\angle\ang{-5.906} = \frac{58}{25}-\frac{6}{25}\imagj [V]
\end{equation*}
Nous pouvons maintenant déterminer l'impédance équivalente $Z_{eq}$ :
\begin{equation*}
    Z_{eq} = \frac{V_{Th}}{I_N} = \frac{\frac{58}{25}-\frac{6}{25}\imagj }{ -\frac{8}{5}+\frac{36}{5}\imagj } =-(\frac{1}{10}+\frac{3}{10}\imagj )=0.316\angle\ang{-108.435} [\Omega]
\end{equation*}
Le circuit équivalent de Norton aux bornes de la résistance peut-être représenté comme :
\begin{center}
    \begin{circuitikz} 
    \draw 
    (0,0) to [american current source,l=$I_N$] (0,3)--(1,3) 
    (1,3) to[european resistor,l=$Z_{eq}$](1,0)
    (0,0) to[short,-o](2,0)
    (1,3)to[short,-o](2,3);
    \end{circuitikz} 
\end{center}
\end{enumerate}
\end{solution}
\newpage
\section{Question Oestges : quadripôles} 
Soit le circuit suivant avec l'amplificateur opérationnel idéal:
\begin{center}
	\begin{circuitikz} 
	\draw
	(6,-0.5) node[op amp] (opamp) {}
	(0,0) node[above]{$V_i$}{} to[R, l_=$\SI{1}{\kilo\ohm}$,o-](3,0)
	to[R,l_=$\SI{500}{\ohm}$] (3,-2) node[ground]{}
	(3,0) to [C,l=$\SI{0.1}{\micro\farad}$](3,3)
	(3,0) to [C,l_=$\SI{0.1}{\micro\farad}$](opamp.-)
	(opamp.+) to[short] (4.5,-1)--(4.5,-1.5)node[ground]{}
	(4.5,0) to[R,l_=$\SI{10}{\kilo\ohm}$] (4.5,3)
	(3,3)--(7,3)--(7,-0.5)--(opamp.out)
	(7,-0.5) to[short,-o] (8,-0.5)node[above]{$V_o$}{};
	\end{circuitikz}
\end{center}
Déterminer $Z_{in}$ si on met un charge $z_L=\SI{1}{\kilo\ohm}$ à la sortie de l'amplificateur opérationnel avec : $R_1 = \SI{1}{\kilo\ohm}$, $R_2=\SI{500}{\ohm}$, $R_3=\SI{10}{\kilo\ohm}$ et $C=\SI{0.1}{\mico\farad}$.
\begin{solution}
Pour déterminer $Z_{in}$, nous allons utiliser la représentation quadripôle Z. La matrice $Z$ est calculée grâce à : 
\begin{equation*}
    \begin{bmatrix}
        V_i \\
        V_o\\
    \end{bmatrix} = 
    \begin{bmatrix}
        z_i & z_r \\
        z_f & z_o\\
    \end{bmatrix}
    \begin{bmatrix}
        i_i \\
        i_o\\
    \end{bmatrix}
\end{equation*}
En annulant le courant $i_i$ le circuit devient :
\begin{center}
	\begin{circuitikz} 
	\draw
	(6,-0.5) node[op amp] (opamp) {}
	(0,0) node[above]{$V_i$}{} to[R, l_=$\SI{1}{\kilo\ohm}$,o-](3,0)
	to[R,l_=$\SI{500}{\ohm}$] (3,-2) node[ground]{}
	(3,0) to [C,l=$\SI{0.1}{\micro\farad}$](3,3)
	(3,0) to [C,l_=$\SI{0.1}{\micro\farad}$](opamp.-)
	(opamp.+) to[short] (4.5,-1)--(4.5,-1.5)node[ground]{}
	(4.5,0) to[R,l_=$\SI{10}{\kilo\ohm}$] (4.5,3)
	(3,3)--(7,3)--(7,-0.5)--(opamp.out)
	(7,-0.5) to[short,i<_=$i_o$,-o] (8,-0.5)node[above]{$V_o$}{};
	\end{circuitikz}
\end{center}
Comme nous avons un amplificateur opérationnel que l'on considère comme idéal, son impédance d'entrée vaut $\SI{0}{\ohm}$. Le courant $i_0$ passe donc entièrement dans l'amplificateur opérationnel et aucun courant ne passe dans le reste du circuit. Dès lors tout le circuit est à $\SI{0}{\volt}$. Nous obtenons donc :
\begin{equation*}
\left \{
    \begin{array}{rcl}
    V_i &=& 0 \cdot i_i \\
    V_o &=& 0 \cdot i_o \\
\end{array}
\right.
\Leftrightarrow 
\left \{
    \begin{array}{rcl}
    z_r &=& 0 [\Omega]\\
    z_o &=& 0 [\Omega]\\
\end{array}
\right.
\end{equation*}
En annulant le courant $i_o$ le circuit devient :
\begin{center}
	\begin{circuitikz} 
	\draw
	(6,-0.5) node[op amp] (opamp) {}
	(0,0) node[above]{$V_i$}{} to[R, l_=$R_1$,i>_=$i_i$,v^=$ $,o-*](3,0) node[above left]{$V_x$}{}
	to[R,l_=$R_2$,v^=$ $] (3,-2) node[ground]{}
	(3,0) to [C,l=$C$,v_=$ $](3,3)
	(3,0) to [C,l_=$C$,v^=$ $](opamp.-)
	(opamp.+) to[short] (4.5,-1)--(4.5,-1.5)node[ground]{}
	(4.5,0)node[below]{$V^-$}{} to[R,l_=$R_3$,v^=$ $,*-] (4.5,3)
	(3,3)--(7,3)--(7,-0.5)--(opamp.out)
	(7,-0.5) to[short,-o] (8,-0.5)node[above]{$V_o$}{};
	\end{circuitikz}
\end{center}
Comme l'amplificateur opérationnel est idéal et en rétroaction négative, on sait que $V^+=V^-=0$V. Dès lors:
\begin{equation*}
    (V_x-0)\imagj \omega C=\frac{0-V_{o}}{R_3C\imagj \omega} \Leftrightarrow V_x=-\frac{V_{o}}{R_3C\imagj \omega}
\end{equation*}
Nous pouvons faire une loi des noeuds au niveau du noeud $V_x$ :
\begin{align*}
    i_i &= (V_x-V_o)\imagj\omega C+(V_x-0)\imagj \omega C+\frac{V_x-0}{R_2}\\
    \Leftrightarrow i_i &= V_x(2\imagj \omega C+\frac{1}{R_2})-V_o\imagj \omega C\\
    \Leftrightarrow i_i &= -V_o(\frac{1}{R_3C\imagj \omega}(2\imagj \omega C+\frac{1}{R_2})+\imagj \omega C)\\
    \Leftrightarrow i_i &= - V_o \left[ \frac{1+2\imagj \omega CR_2+R_2R_3C^2(\imagj \omega)^2}{R_2R_3C\imagj \omega}\right]\\
    \Leftrightarrow V_o &= -\frac{CR_2R_3\imagj \omega}{1+2R_2C\imagj \omega+R_2R_3C^2(\imagj \omega)^2}i_i \\
    \Rightarrow z_f &= -\frac{CR_2R_3\imagj \omega}{1+2R_2C\imagj \omega+R_2R_3C^2(\imagj \omega)^2} [\Omega]
\end{align*}
Nous pouvons calculer $V_i$ selon :
\begin{align*}
    V_i &= R_1i_i+V_x\\
    \Leftrightarrow V_i &= R_1i_i -\frac{V_o}{R_3C\imagj \omega}\\
    \Leftrightarrow V_i &= R_1i_i+ i_i \frac{R_2}{1+2R_2C\imagj \omega+R_2R_3C^2(\imagj \omega)^2}\\
    \Leftrightarrow V_i &= i_i \frac{R_1+R_2+2R_1R_2C\imagj \omega+R_1R_2R_3C^2(\imagj \omega)^2}{1+2R_2C\imagj \omega+R_2R_3C^2(\imagj \omega)^2}\\
    \Rightarrow z_i &= \frac{R_1+R_2+2R_1R_2C\imagj \omega+R_1R_2R_3C^2(\imagj \omega)^2}{1+2R_2C\imagj \omega+R_2R_3C^2(\imagj \omega)^2}[\Omega]
\end{align*}
La matrice Z est donc :
\begin{equation*}
    Z = \begin{bmatrix} 
          \frac{R_1+R_2+2R_1R_2C\imagj \omega+R_1R_2R_3C^2(\imagj \omega)^2}{1+2R_2C\imagj \omega+R_2R_3C^2(\imagj \omega)^2} & 0 \\
         -\frac{CR_2R_3\imagj \omega}{1+2R_2C\imagj \omega+R_2R_3C^2(\imagj \omega)^2} & 0\\
        \end{bmatrix}
\end{equation*}
Grâce à la table des caractéristiques externes nous pouvons calculer l'impédance d'entrée $Z_{in}$ :
\begin{align*}
    Z_{in} &= z_i\left(1+\frac{z_r}{z_i}A_{if,o}\right)~~\mbox{avec}~A_{if,o} =-\frac{z_f}{z_o+z_L}\\
    &= z_i(1+\frac{0}{z_i}\cdot-\frac{z_f}{z_l+0})\\
    &=z_i\\
    \Rightarrow Z_{in} &= \frac{R_1+R_2+2R_1R_2C\imagj \omega+R_1R_2R_3C^2(\imagj \omega)^2}{1+2R_2C\imagj \omega+R_2R_3C^2(\imagj \omega)^2} [\Omega]\\
\end{align*}
\end{solution}
\newpage
\section{Question Dehez : circuit magnétique couplé} 
Soit le circuit suivant :
\begin{center}
    \begin{circuitikz} 
    \draw
    (0,0.45) to [american voltage source,l^=$\SI{12\angle\ang{0}}{\volt}$](0,2.55)
    to[R,l_=$\SI{3}{\ohm}$](3,2.55) 
    (4,1.5) node[transformer] (T) {}
    (T.outer dot A1) node[circ]{}
    (T.outer dot B1) node[circ]{}
    (T-L1.south) node[left]{$\SI{3\imagj}{\ohm}$}{}
    (T-L2.south) node[right]{$\SI{3\imagj}{\ohm}$}{}
    (3,0.45) --(0,0.45)
    (4.5,0.45)--(9.5,0.45)
    (7,0.45)to[C,l=$\SI{-6\imagj}{\ohm}$](7,1)
    (5.5,1)--(8,1)
    to[R,l_=$\SI{6}{\ohm}$] (8,2.55)
    (5.5,1) to [R,l_=$\SI{6}{\ohm}$] (5.5,2.55)--(5,2.55)
    (5.5,2.55) to[R,l_=$\SI{6}{\ohm}$] (8,2.55) --(9.5,2.55)to[C,l=$\SI{-6\imagj}{\ohm}$](9.5,0.45);
    \draw[<->,>=stealth] ($(3.6,2.7)$)  to [bend left] node[pos=0.5,above] {$\SI{3\imagj}{\ohm}$} ($(4.4,2.7)$);
    \end{circuitikz}
\end{center}
On demande :
\begin{enumerate}
    \item Calculer le facteur de dispersion. Expliquer ce que signifie ce résultat.
    \item Calculé le courant délivré par la source de tension.
\end{enumerate}
On souhaite mesure le courant délivré par la source à l'aide d'un ampèremètre. Cependant, l'ampèremètre possède un fond d'échelle de 100[mA]. On utilise un transformateur afin de pallier à ce problème.
\begin{center}
    \begin{circuitikz} 
    \draw
    (-3,0.45) to [american voltage source,l^=$\SI{12\angle\ang{0}}{\volt}$](-3,2.55) to[L](0,2.55)
    to[R,l_=$\SI{3}{\ohm}$](3,2.55) 
    (4,1.5) node[transformer] (T) {}
    (T.outer dot A1) node[circ]{}
    (T.outer dot B1) node[circ]{}
    (T-L1.south) node[left]{$\SI{3\imagj}{\ohm}$}{}
    (T-L2.south) node[right]{$\SI{3\imagj}{\ohm}$}{}
    (3,0.45) --(-3,0.45)
    (4.5,0.45)--(9.5,0.45)
    (7,0.45)to[C,l=$\SI{-6\imagj}{\ohm}$](7,1)
    (5.5,1)--(8,1)
    to[R,l_=$\SI{6}{\ohm}$] (8,2.55)
    (5.5,1) to [R,l_=$\SI{6}{\ohm}$] (5.5,2.55)--(5,2.55)
    (5.5,2.55) to[R,l_=$\SI{6}{\ohm}$] (8,2.55) --(9.5,2.55)to[C,l=$\SI{-6\imagj}{\ohm}$](9.5,0.45)
    (-3,3.5)to[L](0,3.5)--(0,4.5) to node[meter]{\small A}(-3,4.5)--(-3,3.5)
    (-2,3)--(-1,3)
    (-2,3.1)--(-1,3.1);
    \draw (-2.3,2.5)node[below right]{$\bullet$};
     \draw (-2.3,4)node[below right]{$\bullet$};
    \draw[white] (-3,3)  to[] node[black,pos=0.5,rotate = 90] {$n:1$} (-3,3.1);
    \draw[<->,>=stealth] ($(3.6,2.7)$)  to [bend left] node[pos=0.5,above] {$\SI{3\imagj}{\ohm}$} ($(4.4,2.7)$);
    \end{circuitikz}
\end{center}
\begin{enumerate}[resume]
    \item Calculer la valeur idéale que devrait prendre le rapport de transformation $n$ de ce transformateur pour avoir une mesure correspondant au fond d’échelle de l'ampèremètre.
\end{enumerate}
\begin{solution}
\begin{enumerate}
\item Le facteur de dispersion s’écrit comme 
\begin{equation*}
        \sigma = 1 -\frac{M^2}{L_1L_2} = 1 - \frac{(3\imagj )^2}{3\imagj \cdot 3\imagj } = 0
\end{equation*}
Le couplage magnétique est parfait, il n'y a aucun flux de fuite.
\item Nous allons modifier le circuit en utilisant la dualité étoile-triangle :
\begin{center}
    \begin{circuitikz} 
    \draw
    (0,0.45) to [american voltage source,l^=$12\angle 0^\circ$](0,2.55)
    to[R,l_=$\SI{3}{\ohm}$](3,2.55) 
    (4,1.5) node[transformer] (T) {}
    (T.outer dot A1) node[circ]{}
    (T.outer dot B1) node[circ]{}
    (T-L1.south) node[left]{$\SI{3\imagj}{\ohm}$}{}
    (T-L2.south) node[right]{$\SI{3\imagj}{\ohm}$}{}
    (3,0.45) --(0,0.45)
    (5,2.55)--(8,2.55)
    to[R,l^=$R_a$](8,1)
    to[R,l_=$R_b$](7,0)to[C,l_=$\SI{-6\imagj}{\ohm}$](6,-1)
    (8,1) to[R,l^=$R_c$](9,0) to[C,l^=$\SI{-6\imagj}{\ohm}$](10,-1)--(5.05,-1)--(5.05,0.45);
    \draw[<->,>=stealth] ($(3.6,2.7)$)  to [bend left] node[pos=0.5,above] {$\SI{3\imagj}{\ohm}$} ($(4.4,2.7)$);
    \end{circuitikz}
\end{center}
\begin{equation*}
    R_a=R_b=R_c=\frac{6\cdot6}{6+6+6}=2[\Omega]
\end{equation*}
On remarque que les 2 branches du bas de l'étoile sont en parallèle. Dès lors:
\begin{equation*}
    Z=(2-6\imagj )||(2-6\imagj )=\left(\frac{1}{2-6\imagj }+\frac{1}{2-6\imagj }\right)^{-1}=1-3\imagj [\Omega]
\end{equation*}
Le circuit devient :
\begin{center}
    \begin{circuitikz} 
    \draw
    (0,0.45) to [american voltage source,l^=$12\angle 0^\circ$](0,2.55)
    to[R,l_=$\SI{3}{\ohm}$,i>_=$I_1$,v^<=$ $](3,2.55) 
    (4,1.5) node[transformer] (T) {}
    (T.outer dot A1) node[circ]{}
    (T.outer dot B1) node[circ]{}
    (T-L1.south) node[left]{$\SI{3\imagj}{\ohm}$}{}
    (T-L2.south) node[right]{$\SI{3\imagj}{\ohm}$}{}
    (3,0.45) --(0,0.45)
    (5.05,0.45) to[european resistor,l_=$\SI{1-3\imagj}{\ohm}$,i>_=$I_2$,v^<=$ $](7,0.45)
    to[R,l_=$\SI{2}{\ohm}$,v^<=$ $](7,2.55)--(5.05,2.55);
     \draw[<->,>=stealth] ($(3.6,2.7)$)  to [bend left] node[pos=0.5,above] {$\SI{3\imagj}{\ohm}$} ($(4.4,2.7)$);
    \end{circuitikz}
\end{center}
Nous pouvons effectuer 2 lois des mailles sur les 2 uniques mailles du circuit :
\begin{equation*}
\left \{
    \begin{array}{rcl}
    12 &=& (3+3\imagj )I_1 +3\imagj I_2 \\
    0 &=& 3\imagj I_1 +3I_2 \\
\end{array}
\right.
\Leftrightarrow 
\left \{
    \begin{array}{rcl}
    I_1 &=& \frac{8}{5} -\frac{4}{5}\imagj  =1.788\angle\ang{-26.565} [A]\\
    I_2 &=& -\frac{4}{5} -\frac{8}{5}\imagj  =1.788\angle\ang{-116.565} [A]\\
\end{array}
\right.
\end{equation*}
\item Le fond d'échelle correspond à la valeur maximale que peut 'lire' l'appareil de mesure. Il faut donc que le courant mesurée par l'ampèremètre ne dépasse pas $\SI{100}{\milli\ampere}$. La formule qui lie le courant et le nombre de spires est donnée par :
\begin{equation*}
        \frac{I_{primaire}}{I_{secondaire}} = \pm \frac{N_{secondaire}}{N_{primaire}}
        \Leftrightarrow \frac{I_1}{I_{ampèremètre}}= \frac{1}{n} ~~\mbox{avec}~~I_{ampèremètre}< 0.1[A]
    \end{equation*}
    \begin{equation*}
        \Leftrightarrow I_{1} \cdot n< 0.1 = I_{ampèremètre} \Leftrightarrow n <\frac{0.1}{I_{1}} = \frac{0.1}{1.788} = 0.056 
    \end{equation*}
Il faut que le nombre de spires soit inférieur à 0.056. Il est donc impossible d'avoir un courant inférieur à 0.1[A] dans le circuit secondaire. Si l'on avait inversé le rapport c'est-à-dire $N_1=1$ et $N_2=n$ nous aurions pu trouvé un nombre minimum de spires égale à 17.8.
\end{enumerate}
\end{solution}
\newpage
\section{Question Craeye : transitoire} 
Soit le circuit suivant dont l'interrupteur change de position en $t=0$.
\begin{center}
    \begin{circuitikz} 
        \draw (0,3) node[ground]{} to[american voltage source,l=$\SI{2}{\volt}$]  (0,6) to[short,-o] (1,6) coordinate (B);
        \draw (2,3) node[ground]{}to[american voltage source,l=$\SI{1}{\volt}$] (2,4) to[short,-o] (2,5) coordinate (A);
        \draw (1.4,5.4) to[short,o-] (2,6);
        \draw 
        (8,5.5) node[op amp] (opamp) {}
        (2,6) to[L,l_=$L$] ++(3,0) to[R,l_=$R$](opamp.-)
        (opamp.-) to[short,*-] ++(0,2.5)
        to[R,l^=$R$]++(3,0) to [short]++(0,-3)
        (opamp.+)--(6.8,4.5)node[ground]{}
        (6.8,7)to[C,l_=$C$](9.8,7)
        (9.8,5.5) to [R,l^=$R$](9.8,3)node[ground]{}
        (opamp.out)to[short,-*](9.8,5.5) to [short,-o] (11,5.5)node[above]{$V_o$}{};
        \draw [->,>=stealth] ($ (A) + (-0.2,0) $)  to [bend left] node[pos=0.5,below left] {$t=0$} ($ (B) + (0,-0.2) $);
        \draw (A) node[right]{$A$};
        \draw (B) node[above]{$B$};
    \end{circuitikz}
\end{center}
On demande la tension $V_o(t)$ avec les données numériques suivantes:
$R=\SI{1}{\kilo\ohm}$, $L=\SI{1}{\milli\henry}$, $C =\SI{1}{\nano\farad}$
\begin{solution}
En $t<0$, on peut réécrire le circuit comme :
\begin{center}
    \begin{circuitikz} 
        \draw (2,3) node[ground]{}to[american voltage source,l=$\SI{1}{\volt}$] (2,6)
        (8,5.5) node[op amp] (opamp) {}
        (2,6) to[short,i>_=$I_L(0^-)$] ++(3,0) to[R,l_=$R$,v^=$ $](opamp.-)
        (opamp.-) to[short,*-] ++(0,2.5)
        to[R,l^=$R$,v_=$ $]++(3,0) to [short]++(0,-3)
        (opamp.+)--(6.8,4.5)node[ground]{}
        (6.8,7)to[open,v^=$V_C(0^-)$,*-*](9.8,7)
        (9.8,5.5) to [R,l^=$R$](9.8,3)node[ground]{}
        (opamp.out)to[short,-*](9.8,5.5) to [short,-o] (11,5.5)node[above]{$V_o$}{};
    \end{circuitikz}
\end{center}
Étant donner que l'amplificateur opérationnel est considéré comme parfait et est en rétroaction négative ($V^+=V^-=0[V]$) nous obtenons:
\begin{equation*}
    \left \{
\begin{array}{rcl}
I_L(0^-) &=& \frac{1}{R}=10^{-3} [A]\\
V_{C}(0^-) &=& R I_L(0^-)=1[V]\\
\end{array}
\right.
\end{equation*}
En $t>0$, on peut réécrire le circuit comme :
\begin{center}
    \begin{circuitikz} 
        \draw 
        (0,3) node[ground]{} to[american voltage source,l=$\SI{2}{\volt}$]  (0,6) to[short] (2,6)
        (8,5.5) node[op amp] (opamp) {}
        (2,6) to[L,l_=$L$,i>_=$I_1(s)$,v^=$V_L(s)$] ++(3,0) to[R,l_=$R$,v^=$V_R(s)$](opamp.-)
        (opamp.-) to[short,*-] ++(0,3.5)
        to[R,l_=$R$,i>_=$I_2(s)$,v^=$V_{R^{'}}(s)$]++(3,0) to [short]++(0,-4)
        (opamp.+)--(6.8,4.5)node[ground]{}
        (6.8,7)to[C,l_=$C$,v^=$V_C(s)$](9.8,7)
        (9.8,5.5) to [R,l^=$R$](9.8,3)node[ground]{}
        (opamp.out)to[short,-*](9.8,5.5) to [short,-o] (11,5.5)node[above]{$V_o(s)$}{};
    \end{circuitikz}
\end{center}
On peut remarquer que la résistance en bas à droite n'a pas d'impact sur la tension $V_o(s)$.
Nous pouvons effectuer une loi des mailles sur la maille de gauche:
\begin{align*}
    \frac{2}{s} &= V_L(s) + V_R(s) ~~\mbox{avec}~~V_L(s) = sLI_1(s) -LI_L(0^-)\\
    \Leftrightarrow \frac{2}{s} &= sLI_1(s) -LI_L(0^-) + RI_1(s)\\
    \Leftrightarrow \frac{2}{s} &= I_1(s)(R+sL) -LI_L(0^-)
\end{align*}
\begin{equation*}
     \Rightarrow I_1(s) = \frac{2+sLI_L(0^-)}{s(R+sL)}
\end{equation*}
Nous pouvons effectuer une loi des mailles sur la maille tout en haut :
\begin{align*}
    V_C(s)&=V_{R{'}}\\
    \Leftrightarrow V_C(s) &= RI_2(s)\\
    \Leftrightarrow I_2(s) &= \frac{V_C(s)}{R}
\end{align*}
Sachant que $ V_C(s) = \frac{I_1(s)-I_2(s)}{Cs}+\frac{V_C(0^-)}{s}$ nous pouvons calculer :
\begin{align*}
    V_C(s) &=  \frac{I_1(s)-I_2(s)}{Cs}+\frac{V_C(0^-)}{s} \\
   \Leftrightarrow V_C(s) &= \frac{2+sLI_L(0^-)}{s^2C(R+sL)}-\frac{V_C(s)}{sRC}+V_C(0^-)\\
   \Leftrightarrow V_C(s) (1+\frac{1}{sRC}) &= \frac{2+sLI_L(0^-)+sC(R+sL)V_C(0^-)}{s^2C(R+sL)}\\
   \Leftrightarrow V_C(s) &= \frac{2+s(LI_L(0^-)+CRV_C(0^-))+s^2LRCV_C(0^-)}{s^2C(R+sL)}\frac{RsC}{1+sRC}\\
   \Leftrightarrow V_C(s) &= \frac{2R+Rs(LI_L(0^-)+CRV_C(0^-))+s^2LRCV_C(0^-)}{s(R+s(R^2C+L)+s^2RCL)}\\
\end{align*}
Sachant que $V_C(s) = -V_o(s)$ :
\begin{equation*}
    V_o(s) =- \frac{2+s(LI_L(0^-)+RCV_C(0^-))+s^2LCV_C(0^-)}{s(1+s(RC+\frac{L}{R})+s^2LC)}
\end{equation*}
En substituant les paramètres par leurs valeurs nous obtenons :
\begin{equation*}
    V_o(s) = -\frac{2+2\cdot 10^{-3}s+10^{-12}s^2}{s(1+2\cdot 10^{-6}+10^{-12}s^2)}=-\frac{2\cdot 10^{12}+2\cdot 10^6+s^2}{s(10^{12}+2\cdot 10^6+s^2)}=-\frac{2\cdot 10^{12}+2\cdot 10^6+s^2}{s(s+10^6)^2}
\end{equation*}
En effectuant la décomposition en fraction simple nous avons :
\begin{equation*}
    V_C(s) = \frac{A}{s} + \frac{B}{s+10^6}+\frac{C}{(s+10^{6})^2} \Leftrightarrow 
\begin{dcases}
 A +B =-1\\
 2\cdot 10^6 A+ 10^6B+C = -2\cdot 10^6\\
 10^{12}A = -2 \cdot 10^{12}\\
\end{dcases}
\Leftrightarrow
\begin{dcases}
 A= -2\\
 B= 1\\
 C=10^{6}\\
\end{dcases}
\end{equation*}
Nous obtenons donc dans le domaine de Laplace :
\begin{equation*}
    V_o(s) = -\frac{2}{s} +\frac{1}{s+10^6}+ \frac{10^6}{(s+\cdot 10^{6})^2}
\end{equation*}
Cela correspond dans le domaine temporel à : 
\begin{equation*}
    V_o(t) =  \left[-2+e^{-10^6t}(1+10^{6}t)\right]u(t)
\end{equation*}
Le graphe ressemble à
\begin{center}
    \begin{tikzpicture}
        \begin{axis}[enlargelimits=true,grid=major,ylabel=$V_o(t)$,xlabel=$t(s)$]
            \addplot [blue,domain=0:0.00001,samples=200]{-2+e^(-1*10^6*x)*(1+10^6*x)};
        \end{axis}
    \end{tikzpicture}
\end{center}
\end{solution}
\end{document}