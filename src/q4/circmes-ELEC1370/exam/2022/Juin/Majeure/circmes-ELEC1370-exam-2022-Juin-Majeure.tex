\documentclass[fr]{../../../../../../eplexam}

\hypertitle{circmes}{4}{ELEC}{1370}{2022}{Juin}{Majeure}
{Quentin De Laet}
{Christophe Craeye, Bruno Dehez and Claude Oestges}


\usepackage[oldvoltagedirection]{circuitikz}
\usepackage{amsmath}

\begin{document}

\section{Question Oestges : Bode et quadripôles}

On considère le montage suivant : 

\begin{center}
\begin{circuitikz}
    \draw (0,0) node[above]{$v_i$} to[short, o-*] ++(1,0) coordinate(SPL1) -- ++(0,-2) to[R=$R_H$] ++(2,0) to[C=$C_H$] ++(2,0) 
    node[op amp, anchor=-](OAH){} to[short,*-] ++(0,1.25) coordinate(OAHm) to[R=$R_H$] (OAHm -| OAH.out) to[short,-*] (OAH.out) 
    to[R=$R_i$] ++(2,0) coordinate(EndH)
    (OAH.+) node[ground]{}
    (SPL1) -- ++(0,1.5) to[R=$R_L$] ++(4,0) node[op amp, anchor=-](OAL){} to[short,*-] ++(0,1.25) coordinate(OALm) 
    to[R=$R_L$, *-*] (OALm -| OAL.out) 
    (OALm) -- ++(0,1.25) coordinate(OALm2) to[C=$C_L$] (OALm2 -| OAL.out) -- (OAL.out) to[R=$R_i$, *-] ++(2,0)
    (OAL.+) node[ground]{}
    (EndH) -- ++(0,2) coordinate(OAf_in) -- (OAf_in |- OAL.out) 
    (OAf_in) to[short, *-*] ++(1,0) node[op amp, anchor=-](OAf){} -- ++(0,1.25) coordinate(OAfm) to[R=$R_f$] (OAfm -| OAf.out) -- (OAf.out) 
    to[short,*-o] ++(1,0) node[above]{$v_o$}
    (OAf.+) node[ground]{}
    ;
\end{circuitikz}
\end{center}

On demande :
\begin{enumerate}
	\item le gain $g_f(\omega) = \frac{v_o}{v_i}$ du montage
	\item le diagramme de Bode de ce montage (en dB et en phase) avec les valeurs suivantes (unités omises) :
	\begin{itemize}
	    \item $R_f = 2R_i = 2000$
	    \item $R_L C_L = 10^{-5}$
	    \item $R_H C_H = 3\cdot10^{-6}$
	\end{itemize}
	\item l'utilité de ce montage
	\item les valeurs $Z_{in}$ et $Z_{out}$ du quadripôle lorsqu'une charge de valeur $Z_L$ est placée à sa sortie, 
    sachant que la source de tension connectée en entrée possède une résistance interne de valeur $R_S$.
\end{enumerate}

\nosolution

\newpage
\section{Question Dehez : circuits magnétiques couplés et puissance}

Soit le montage suivant, fonctionnant à une fréquence de $50Hz$ :

\begin{center}
\begin{circuitikz}[american]
    \draw (0,0) to[V, l=$24V\angle 0$] ++(0,2) -- node[flowarrow]{$\Bar{I_a}$} ++(1,0) -- ++(2,0) to[R=$1\Omega$] ++(2,0) coordinate(SPLIT) 
    to[american inductor=$j2\Omega$] node[below](a){$\bullet$} ++(2,0) to[C=$-j2\Omega$] ++(2,0) node[below](b){$\bullet$} 
    to[american inductor=$j2\Omega$] ++(2,0) to[R=$1\Omega$] ++(0,-2) -- (0,0);
    \draw (SPLIT) to[C=$-j1\Omega$,*-*] ++(0,-2);
    \draw (a) to[open] ++(0,1.1) [ultra thick] [stealth-stealth, scale=1.02] to[bend left] node[above,pos=0.1]{$j1\Omega$} ++(2,0);
\end{circuitikz}
\end{center}

On demande :
\begin{enumerate}
    \item le facteur de dispersion entre les inductances couplées
    \item la valeur du courant $\bar{I_a}$ (amplitude et phase)
    \item la puissance active, réactive et apparente fournie par la source de tension
    \item la valeur de l'impédance $Z$ à placer (voir circuit ci-dessous) afin d'annuler la puissance réactive fournie par la source de tension
\end{enumerate}

\begin{center}
\begin{circuitikz}[american]
    \draw (0,0) to[V, l=$24V\angle 0$] ++(0,2) -- node[flowarrow]{$\Bar{I_a}$} ++(1,0) to[generic=$Z$] ++(2,0) to[R=$1\Omega$] ++(2,0) 
    coordinate(SPLIT) to[american inductor=$j2\Omega$] node[below](a){$\bullet$} ++(2,0) to[C=$-j2\Omega$] ++(2,0) node[below](b){$\bullet$} 
    to[american inductor=$j2\Omega$] ++(2,0) to[R=$1\Omega$] ++(0,-2) -- (0,0);
    \draw (SPLIT) to[C=$-j1\Omega$,*-*] ++(0,-2);
    \draw (a) to[open] ++(0,1.1) [ultra thick] [stealth-stealth, scale=1.02] to[bend left] node[above,pos=0.1]{$j1\Omega$} ++(2,0);
\end{circuitikz}
\end{center}

\nosolution

\newpage
\section{Question Craeye : transitoire}

Soit le circuit suivant, avec l'interrupteur passant de A à B en $t=0$. \\
Sachant que $\omega = 2\cdot10^6$ $rad/s$, $R = 1$ $k\Omega$, $L = 1$ $mH$ et $C = 1/6$ $nF$, calculer la valeur de $V_o(t)$ pour $t > 0$.\newline
Exprimez le terme harmonique de la réponse sous forme paramétrique.\newline
\textbf{Point bonus :} trouvez la valeur des coefficients du terme harmonique.

\begin{center}
\begin{circuitikz}[american]
    \draw (0,0) to[sV, v^<=$\cos{(\omega t)}$] ++(0,2) to[R=$R$] ++(0,2.5) -- ++(2,0) node[above](B){\textbf{B}} 
    node[spdt, scale=-1, anchor=out 2](Sw){};
    \draw (0,0) -- (0,0 -| Sw.out 1) coordinate(a) to[V=$2V$] ++(0,2) to[R=$R$] (Sw.out 1) node[left]{\textbf{A}};
    \draw (Sw.in) -- ++(1,0) coordinate(b) to[C, a_=$C$] (b |- a) coordinate(c) -- (a);
    \draw (b) to[L=$L$] ++(2,0) coordinate(d) to[R, a_=$R$,v^=$V_o$] (d |- c) -- (c);
\end{circuitikz}
\end{center}

\nosolution

\end{document}
