\documentclass[fr]{../../../../../../eplexam}
\usepackage{../../../../../../eplunits}
\usepackage[oldvoltagedirection]{circuitikz}
\pgfplotsset{compat=newest}
\usepackage[oldvoltagedirection]{circuitikz}
\usepackage{bodegraph}
\usepackage{SIunits}
\usepackage{pgfplots}
\usepackage{amsmath}
\usepackage{enumitem}


\pgfplotsset{compat=newest}
\tikzset{meter/.style={draw,thick,circle,fill=white,minimum size =0.75cm,inner sep=0pt}}
\hypertitle{circmes-ELEC1370}{4}{ELEC}{1370}{2016}{Juin}{Majeure}
{Nicolas Verbeek\and Adrien Couplet\and Martin Van Essche\and Guillaume Gilson\and Guillaume Colinet \and Brieuc Balon}
{Claude Oestges, Bruno Dehez and Christophe Craeye}
\section{Question Oestges : phaseurs}
Soit le circuit suivant avec $I_o = \SI{10\angle\ang{350}}{\ampere}$.
\begin{center}
	\begin{circuitikz} \draw
		to[european resistor,l=$Z_\alpha$, i_>=$I_x$, -*] (0,2.5) to[american current source,l=$\SI{2\angle\ang{0}}{\ampere}$] (2.5,2.5)
 		(2.5,0) to[american voltage source, l=$\SI{12\angle\ang{0}}{\volt}$,-*] (2.5,2.5)
 		(2.5,2.5) to[R, l=$\SI{1}{\ohm}$] (5,2.5)
  		to[R,l=$\SI{1}{\ohm}$, i_>=$I_o$,*-] (5,0)
 		(0,0) -- (5,0)
 		(0,2.5) to[R,l=$\SI{1}{\ohm}$] (0,5) -- (2.5,5) to[L,l=$\SI{1\imagj}{\ohm}$,*-] (2.5,2.5)
 		(2.5,5) -- (5,5) to[C,l=$\SI{-1\imagj}{\ohm}$] (5,2.5);
	\end{circuitikz}
\end{center}
On demande:
\begin{enumerate}
    \item Le courant $I_x$
    \item L'impédance $Z_\alpha$
    \item La tension dans la source de courant
    \item La puissance de la source de tension
\end{enumerate}
\begin{solution}
Pour la suite des calculs nous utiliserons les notations telles qu’illustrées :
\begin{center}
	\begin{circuitikz} \draw
		to[european resistor,l_=$Z_\alpha$, i>_=$I_x$,v^=$V_Z$, -*] (0,2.5) node [left]{$A$}{} to[american current source,v^>=$V_C$,l_=$\SI{2\angle\ang{0}}{\ampere}$] (2.5,2.5) 
 		(2.5,0) to[american voltage source, l_=$\SI{12\angle\ang{0}}{\volt}$,i=$I_S$,-*] (2.5,2.5)
 		(2.5,2.5) to[R, l=$\SI{1}{\ohm}$,v_=$V_2$,i=$I_2$] (5,2.5)node [right]{$B$}{}
  		to[R,l_=$\SI{1}{\ohm}$, i_>=$I_o$,v^=$V_o$,*-] (5,0)
 		(0,0) to [short, -*] (2.5,0) node[below]{$D$}{} -- (5,0)
 		(0,2.5) to[R,l=$\SI{1}{\ohm}$,v_>=$V_5$,i<=$I_5$] (0,5) -- (2.5,5) node [above]{$C$}{}to[L,l=$\SI{1\imagj}{\ohm}$,i<=$I_4$,v>=$V_4$,*-] (2.5,2.5)
 		(2.5,5) -- (5,5) to[C,l_=$\SI{-1\imagj}{\ohm}$,i<_=$I_3$,v^>=$V_3$] (5,2.5);
	\end{circuitikz}
\end{center}
\begin{enumerate}
\item Sur base de la loi d'Ohm (V = ZI) nous pouvons écrire :
\begin{equation*}
    V_o = I_o \cdot 1\Omega =  \SI{10\angle\ang{350}}{\ampere}
\end{equation*}
Nous pouvons effectuer une loi des mailles sur la maille en bas à droite :
\begin{equation*}
    V_2 = 12\angle\ang{0} - V_0 = \SI{2.765\angle\ang{38.9}}{\volt}
\end{equation*}
Sur base de la loi d'Ohm (V = ZI) nous pouvons écrire :
\begin{equation*}
    I_2 = \frac{V_2}{\SI{1}{\ohm}} = 2.765\SI{2.765\angle\ang{38.9}}{\ampere}
\end{equation*}
Nous pouvons effectuer une loi des noeuds sur le noeud $B$ :
\begin{equation*}
    I_3 = I_2 - I_o = \SI{8.44\angle\ang{155.71}}{\ampere} 
\end{equation*}
Sur base de la loi d'Ohm (V = ZI) nous pouvons écrire :
\begin{equation*}
    V_3 = -\imagj  \cdot I_3 = \SI{8.44\angle\ang{65.71}}{\volt}
\end{equation*}
Nous pouvons effectuer une loi des mailles sur la maille en haut à droite :
\begin{equation*}
    V_4 = V_2 +V_3 = \SI{10.98\angle\ang{59.19}}{\volt}
\end{equation*}
Sur base de la loi d'Ohm (V = ZI) nous pouvons écrire :
\begin{equation*}
    I_4 = \frac{V_4}{\imagj } = \SI{10.98\angle\ang{-30.81}}{\ampere}
\end{equation*}
Nous pouvons effectuer une loi des noeuds sur le noeud centrale :
\begin{equation*}
    I_S = I_2 + I_4 - 2\angle\ang{0} = \SI{10.34\angle\ang{-22.08}}{\ampere}
\end{equation*}
Nous pouvons effectuer une loi des noeuds sur le noeud $D$ :
\begin{equation*}
    I_x = I_o - I_S = \SI{2.168\angle\ang{83.01}}{\ampere}
\end{equation*}
\item Nous pouvons effectuer une loi des noeuds sur le noeud $C$ :
\begin{equation*}
    I_5 = I_4+I_3 = \SI{2.766\angle\ang{-51.08}}{\ampere}
\end{equation*}
Sur base de la loi d'Ohm (V = ZI) nous pouvons écrire :
\begin{equation*}
    V_5 = \SI{1}{\ohm} \cdot I_5 = \SI{2.766\angle\ang{-51.08}}{\volt}
\end{equation*}
Nous pouvons effectuer une loi des mailles sur la grande maille:
\begin{equation*}
    V_Z = V_3 + V_5 -V_o = \SI{8.63\angle\ang{122.51}}{\volt}
\end{equation*}
Sur base de la loi d'Ohm (V = ZI) nous pouvons écrire :
\begin{equation*}
    Z_\alpha = \frac{V_Z}{I_x} = \SI{3.98\angle\ang{39.5}}{\ohm}= \SI{3.07+2.53\imagj}{\ohm}
\end{equation*}
\item Nous pouvons effectuer une loi des mailles sur la maille en bas à gauche :
\begin{equation*}
    V_C = V_4+V_5 = \SI{10.352\angle\ang{44.67}}{\volt}
\end{equation*}
\item La puissance\footnote{Nous travaillons avec des phaseurs, ils représentent la valeur éfficace de la tension ou du courant. S'il n'est pas spécifié dans l'énoncé si la valeur de la tension $V_o$ représente la valeur de crête ou la valeur efficace de la tension, n'hésitez pas à le spécifier dans votre exercice. Nous considérons que $V_o$ correspond à la valeur efficace dans la suite des calculs. } active délivrée par la source est : 
\begin{equation*}
    P = \Re(S) = \Re(V\cdot I_S^\ast) = \Re(12\angle\ang{0} \cdot  10.34\angle\ang{22.08}) = 114.98[W]
\end{equation*}
\end{enumerate}
\end{solution}
\newpage
\section{Question Oestges : Bode et quadripôles} 
Soit le circuit avec amplificateurs opérationnels idéaux
\begin{center}
    \begin{circuitikz}[scale = 0.9, transform shape]
		\draw
		(-1,1) node [above]{$V_\text{in}$} to [short,o-] (-0.5,1) 
		to [R, l=$R_1$] (2,1)
		(6,0.5) node[op amp] (opamp) {}
		(opamp.-) to [C,-*,l_=$C$] (2,1) 
		(opamp.+) to [short] (4,0)
		(opamp.out) to [short] (8,0.5)
		(4.5,1) to [short,*-] (4.5,2.5) to [R,l(=$R_2$,-*] (8,2.5)
		(4,0) -- (4,-0.5) node [ground] {}
		(2,1) -- (2,4) to [C,l=$C$] (8,4) to [short,-*] (8,0.5) to [R,l=$R_3$,-*] (11,0.5) -- (12,0.5)
		(-0.5,1) to [short,*-] (-0.5,-2) to [R,l=$R$] (11,-2) -- (11,0.5)

		(14,0) node[op amp] (opamp2) {}
		(opamp2.-) to [short,-*] (12,0.5) 
		(opamp2.+) to [short] (12,-0.5)
		(12,-0.5) -- (12,-1) node [ground] {}
		(opamp2.out) to [short,-*] (15.5,0)

		(12,0.5) -- (12,2) to [R, l=$R$] (15.5,2) -- (15.5,0) to [short,-o] (16,0) node [above]{$V_\text{out}$};
	\end{circuitikz}
\end{center}

On demande:
\begin{enumerate}
    \item La fonction de transfert $\frac{V_\text{out}}{V_\text{in}}$
    \item Le diagramme de Bode de la fonction de transfert trouvée au point précédent
    \item L'impédance de sortie $Z_\text{out}$ en sachant que $Z_{in} = 50\Omega$.
\end{enumerate}
En vous aidant des données numériques suivantes : $R_1 = \SI{21.6}{\kilo\ohm}$, $R_2 =\SI{28.8}{\kilo\ohm}$, $R_3 = \SI{10}{\kilo\ohm}$, $R = \SI{30}{\kilo\ohm}$, $C =\SI{1}{\nano\farad}$.
\begin{solution}
    \begin{enumerate}
        Pour la suite des calculs nous utiliserons les notations telles qu’illustrées :
        \begin{center}
    \begin{circuitikz}[scale = 0.9, transform shape]
		\draw
		(-1,1) node [above]{$V_\text{in}$} to [short,o-] (-0.5,1) 
		to [R, l=$R_1$,v =$ $] (2,1) node[label ={[font=\footnotesize]below:$V_x$}]{}
		(6,0.5) node[op amp] (opamp) {}
		 (2,1)to [C,l^=$C$,v=$ $] (opamp.-) 
		(opamp.+) to [short] (4,0)
		(opamp.out) to [short] (8,0.5)
		(4.5,1) node[label ={[font=\footnotesize]below:$B$}]{} to [short,*-] (4.5,2.5) to [R,l=$R_2$,v=$ $,-*] (8,2.5)
		(4,0) -- (4,-0.5) node [ground] {}
		(2,1) -- (2,4) to [C,l=$C$,v=$ $] (8,4) to [short,-*] (8,0.5) node[label ={[font=\footnotesize]below:$V_{out1}$}]{} to [R,l=$R_3$,v=$ $,-*] (11,0.5) node[label ={[font=\footnotesize]above:$A$}]{} -- (12,0.5)
		(-0.5,1) to [short,*-] (-0.5,-2) to [R,l=$R$,v=$ $] (11,-2) -- (11,0.5)

		(14,0) node[op amp] (opamp2) {}
		(opamp2.-) to [short,-*] (12,0.5) 
		(opamp2.+) to [short] (12,-0.5)
		(12,-0.5) -- (12,-1) node [ground] {}
		(opamp2.out) to [short,-*] (15.5,0)

		(12,0.5) -- (12,2) to [R, l=$R$,v=$ $] (15.5,2) -- (15.5,0) to [short,-o] (16,0) node [above]{$V_\text{out}$};
	\end{circuitikz}
\end{center}
Étant donné que les amplificateurs opérationnels sont en rétroaction négative et qu'ils sont considérés comme idéaux, nous pouvons dire que la tensions aux bornes positives ($A$ et $B$) et négatives de chacun des amplificateurs est nulle.\\
\item Nous pouvons effectuer une loi des noeuds sur le noeud $B$ sachant que le courant $i^- =0$: 
\begin{equation*}
    V_x \imagj \omega C = -\frac{V_{out1}}{R_2} \Leftrightarrow V_x = -\frac{V_{out1}}{R_2\imagj \omega C}
\end{equation*}
Nous pouvons effectuer une loi des noeuds sur le noeud $V_x$ :
\begin{align*}
    & &\frac{V_{in}-V_x}{R_1} &= (V_x -V_{out1})\imagj \omega C +V_x \imagj \omega C\\
    &\Leftrightarrow &\frac{V_{in}}{R_1} &= V_x (\frac{1}{R_1} + 2\imagj \omega C) -V_{out1} \imagj \omega C\\
    &\Leftrightarrow &V_{in} &= V_x (1 + 2R_1\imagj \omega C) -V_{out1} \imagj \omega CR_1\\
    &\Leftrightarrow &V_{in} &= -V_{out1}(\frac{1+2R_1C\imagj \omega}{R_2\imagj \omega C}+R_1\imagj \omega C)
\end{align*}
\begin{equation*}
    \Rightarrow V_{out1} = -\frac{R_2\imagj \omega C}{1+2R_1C\imagj \omega +R_1R_2C^2(\imagj \omega)^2} V_{in}
\end{equation*}
Nous pouvons effectuer une loi des noeuds sur le noeud $A$ sachant que le courant $i^- =\SI{0}{\ampere}$:
\begin{align*}
    & &-\frac{V_{out}}{R} &=\frac{V_{out1}}{R_3} +\frac{V_{in}}{R} \\
    &\Leftrightarrow &-V_{out}R_3 &= R_3 V_{in} + R V_{out1}\\
    &\Leftrightarrow &-V_{out} R_3 &= R_3 V_{in} - \frac{RR_2\imagj \omega C}{1+2R_1C\imagj \omega+R_1R_2C^2(\imagj \omega)^2} V_{in} \\
    &\Leftrightarrow &-V_{out}R_3 &= V_{in}(\frac{R_3+2R_1R_3C\imagj \omega - RR_2\imagj \omega C + R_1R_2R_3C^2(\imagj \omega)^2}{1+2R_1C\imagj \omega+R_1R_2C^2(\imagj \omega)^2}
\end{align*}
\begin{equation*}
    \Rightarrow \frac{V_{out}}{V_{in}}= -(\frac{1+C\imagj \omega(2R_1-\frac{RR_2}{R_3})+R_1R_2C^2(\imagj \omega)^2}{1+2R_1C\imagj \omega +R_1R_2C^2(\imagj \omega)^2}
\end{equation*}
La fonction de transfert peut être réécrite sous la forme générale : 
\begin{equation*}
H(\imagj \omega) = K \frac{1 + 2\xi_0 \imagj \frac{\omega}{\omega_0}+ (\imagj \frac{\omega}{\omega_0})^2}{1+2\xi_1 \imagj \frac{\omega}{\omega_1}+ (\imagj \frac{\omega}{\omega_1})^2}~~
\mbox{avec}~~
\left\{\begin{matrix*}[l]
K= -1\\
\omega_0 = \frac{1}{\sqrt{R_1R_2C^2}} =40093 [\mbox{rad/s}]\\
\omega_1 = \frac{1}{\sqrt{R_1R_2C^2}} =40093 [\mbox{rad/s}] \\
\xi_0 = \frac{\omega_0}{2}C (2R_1-\frac{RR_2}{R_3})=-0.866\text{\footnotemark}\\
\xi_1 = \frac{\omega_1}{2}2R_1C = 0.866
\end{matrix*}\right.
\end{equation*}
\footnotetext{$^1$Et oui les $\xi$ peuvent prendre des valeurs négatives!}
\item Le diagramme de Bode de la fonction de transfert est représentée ci-dessous :
\begin{center}
     \begin{tikzpicture}[
    gnuplot def/.append style={prefix={}},
]
% Grid Style
\tikzset{
    semilog lines/.style={black},
    semilog lines 2/.style={gray,dotted},
    semilog half lines/.style={gray, dotted},
    semilog label x/.style={below,font=\tiny},
    semilog label y/.style={above,font=\tiny} }
% Magnitude Plot
\begin{scope}[xscale=7/5, yscale=3/50]
    \UnitedB
    \semilog{0}{8}{-30}{30} 
    %Asymp
     \BodeGraph[green,samples=1000]{0:5.35}{\SOAmpAsymp{1}{0.866}{40093}}
     \BodeGraph[blue,samples=1000]{0:5.35}{-\SOAmpAsymp{1}{-0.866}{40093}}
    %Real
    \BodeGraph[red,samples=1000]{0:8}{\SOAmp{1}{0.866}{40093}-\SOAmp{-1}{-0.866}{40093}}
\end{scope}
% Phase plot
\begin{scope}[yshift=-9cm,xscale=7/5,yscale=3/180]
    \UniteDegre
    \OrdBode{30}
    \semilog{0}{8}{-180}{360}
    %Asymp
     \BodeGraph[green,samples=1000]{0:8}{\SOArgAsymp{1}{0.866}{40093}}
     %Pas possible d'avoir la phase d'une constante donc on bidouille un peu
     \BodeGraph[blue,samples=1000]{0:8}{-\SOArgAsymp{1}{-0.866}{40093}-\IntArg{-100000}-\IntArg{-100000}}
    %Real
    \BodeGraph[red,samples=1000]{0:8}{\SOArg{1}{0.866}{40093}-\SOArg{-1}{-0.866}{40093}}
\end{scope}
\end{tikzpicture}
\end{center}
C'est un filtre passe-tout qui engendre un déphasage au niveau de sa fréquence de coupure. Le bleu correspond au tracé asymptotique du numérateur, le vert correspond au tracé asymptotique du dénominateur et le rouge correspond au tracé de la fonction de transfert.
    \end{enumerate}
\end{solution}
\newpage
\section{Question Dehez : pont de Wheatstone} 
Soit le pont de Wheatstone un peu modifié. 
\begin{center}
    \begin{circuitikz}[european resistors]
        \def\x{6}
        \def\y{6}
        % Size of the bridge
        \def\dx{3}
        \def\dy{3}
        % Voltage source
        \draw (0,0) to [sinusoidal voltage source, l=$V_s\; \omega_0$]
        (0, \y) to (\x, \y)
        % Left half bridge
        to [american resistor, l_=$R$, *-*] (\x-\dx,\y-\dy) node[label ={[font=\footnotesize]left:$A$}]{}% Top left resistor
        to [vR, l_=$Z_m$, -*] (\x,\y-2*\dy);  % Bottom left resistor
        % Right half bridge
        \draw (\x,\y)
        to [american resistor, l=$R_x$] (7.5, 4.5) % Top right resistor
        to [L, l=$L_x$,-*] (\x+\dx, \y-\dy)node[label ={[font=\footnotesize]right:$B$}]{}
        to [american resistor, l_=$R$, -*] (\x,\y-2*\dy)  % Bottom right resistor
        % Draw connection to (-) terminal of voltage source
        to (\x, 0) to (0,0);
        % Draw voltmeter
        \draw (\x-\dx, \y-\dy) to [short, -o] (5.5,3) 
        (5.5,3.5) to [open, v^=$V_{ab}$] (6.5,3.5)
        (6.5,3) to [short,o-] (\x+\dx, \y-\dy);
    \end{circuitikz}
\end{center}

Avec l'impédance $Z_m$ une mise en parallèle d'une capacité $C_m$ et d'une résistance $R_m$. On demande
\begin{enumerate}
    \item L'expression de $L_x$ et $R_x$ en fonction de $C_m$, $R_m$ et $R$ quand le pont est équilibré (c'est-à-dire quand $V_{ab}=0$).
    \item Lorsque cet équilibre est atteint, la relation dépend-t-elle de $\omega$ ?
    \item Quel appareil faut-il utiliser pour mesurer $V_{ab}$? Pourquoi?
\end{enumerate}

\begin{solution}
\begin{enumerate}
    \item Ce circuit est un cas généralisé du pont de Wheatstone appelé pont de Maxwell\footnote{\url{https://fr.wikipedia.org/wiki/Pont_de_Maxwell}}. Celui-ci permet de mesurer la valeur d'une inductance inconnue grâce à une résistance et un condensateur étalonné ($Z_m$). Lorsque le pont est à l'équilibre, on peut établir la relation :
    \begin{equation*}
        \frac{R}{Z_m}=\frac{R_x+\imagj \omega L_x}{R}
    \end{equation*}
    Cette relation peut être trouvée en calculer la tension $V_{ab}$. Sur base des 2 diviseurs de tensions nous avons :
    \begin{align*}
        \begin{dcases}
            V_A= \frac{R}{Z_m+R}V_s\\
            V_B= \frac{R}{R+R_x+\imagj \omega L_x}V_s\\
        \end{dcases}
        \Rightarrow V_{ab} = V_A - V_B = (\frac{R}{Z_m+R} - \frac{R}{R+R_x+\imagj \omega L_x})V_s
    \end{align*}
    Quand le pont est équilibré, $V_{ab} =0$ et donc :
    \begin{equation*}
        \frac{R}{Z_m+R} =\frac{R}{R+R_x+\imagj \omega L_x} \Leftrightarrow \frac{Z_m+R}{R} =\frac{R+R_x+\imagj \omega L_x}{R} \Leftrightarrow \frac{R}{Z_m}=\frac{R_x+\imagj \omega L_x}{R}
    \end{equation*}
    Sachant que $Z_m$ est la mise en parallèle d'une capacité $C_m$ et d'une résistance $R_m$ peut réécrire la relation comme : 
    \begin{align*}
        &\Rightarrow \frac{R}{(\frac{1}{R_m}+\imagj \omega C_m)^{-1}} = \frac{R_x+\imagj \omega L_x}{R}\\
        &\Leftrightarrow \frac{R(1+\imagj \omega C_mR_m)}{R_m} =\frac{R_x+\imagj \omega L_x}{R}\\
        &\Leftrightarrow R^2+\imagj \omega R^2C_mR_m =R_m R_x +\imagj \omega R_mL_x
    \end{align*}
    \begin{equation*}
        \Rightarrow
        \begin{dcases}
            R^2= R_mR_x\\
            R^2C_mR_m= R_mL_x\\
        \end{dcases}
        \Leftrightarrow
        \begin{dcases}
            R_x= \frac{R^2}{R_m}    \\
            L_x = R^2C_m\\
        \end{dcases}
    \end{equation*}
    \item La relation ne dépend pas de $\omega$
    \item Il faut utiliser un galvanomètre pour mesurer $V_{ab}$ car le galvanomètre permet de voir les fluctuations du courant. Quand le pont est à l'équilibre, il n'y a aucun courant qui passe dans le galvanomètre donc son aiguille est immobile. 
 \end{enumerate}
\end{solution}
\newpage
\section{Question Craeye : transitoire} 
Soit le circuit suivant dont l'interrupteur 1V s'ouvre et l'interrupteur 2V se ferme en $t=0$.
\begin{center}
    \begin{circuitikz} 
    	\draw
		(5,2.5) to[R,l=$R$] (5,0)
	 	(0,0) -- (5,0)
	 	(-1.5,0) -- (0,0)
	 	(-1.5,0) to[american voltage source,l=$SI{2}{\volt}$] (-1.5,2.5) to [closing switch,l=$t{=}0$] (-1.5,5) -- (0,5)
		(0,0) to[american voltage source,l=$SI{1}{\volt}$] (0,2.5) to [opening switch,l=$t{=}0$] (0,5) to [C,l=$C$,*-] (5,5)
		(5,5) to[L,l=$L$,*-] (5,2.5)
		(0,5) -- (0,6.5) to [R,l=$R$] (5,6.5) -- (5,5)
		(6,5) to [open, v^=$v_o$] (6,0);
	\end{circuitikz}
\end{center}
On demande la tension $V_o$ en $t=0^+$ avec les données numériques suivantes:
$R=\SI{1}{\kilo\ohm}$, $L=\SI{1}{\milli\henry}$, $C =\SI{1}{\nano\farad}$
\begin{solution}
En $t<0$, on peut réécrire le circuit comme :
\begin{center}
    \begin{circuitikz} 
    	\draw
		(5,2.5) to[R,l=$R$] (5,0)
	 	(0,0) -- (5,0)
		(0,0) to[american voltage source,l=$SI{1}{\volt}$] (0,5) to [open,v<=$V_C(0^-)$,*-*] (5,5) to[short,i=$I_L(0^-)$] (5,2.5)
		(0,5) -- (0,6.5) to [R,l=$R$] (5,6.5) -- (5,5);
	\end{circuitikz}
\end{center}
Sur base du diviseur de tension et la loi des mailles sur l’unique maille nous obtenons :
\begin{equation*}
    \left \{
\begin{array}{rcl}
V_{C}(0^-) &=& \frac{R}{R+R} \cdot 1[V] = 0.5[V]\\
I_L(0^-) &=& \frac{1}{R+R} = 5\cdot 10^{-4} [A]
\end{array}
\right.
\end{equation*}
En $t>0$, on peut réécrire le circuit comme :
\begin{center}
    \begin{circuitikz} 
    	\draw
		(5,2.5) to[R,l=$R$,v=$V_{R^{'}}(s)$] (5,0)
	 	(-1.5,0) -- (5,0)
	 	(-1.5,0) to[american voltage source,l=2V] (-1.5,5) 
		 --(0,5) to [C,l=$C$,*-,v_<=$V_C(s)$] (5,5)
		(5,5) to[L,l=$L$,i=$I_2(s)$,v_=$V_L(s)$,*-] (5,2.5)
		(0,5) -- (0,6.5) to [R,l_=$R$,i_=$I_1(s)$,v^=$V_R(s)$] (5,6.5) -- (5,5)
		(6,5) to [open, v^=$V_o(s)$] (6,0);
	\end{circuitikz}
\end{center}
Sur base de la loi des mailles sur la maille du haut nous pouvons écrire :
\begin{align*}
   & \Rightarrow &RI_1(s) &= V_C(s)~~\mbox{avec}~~ V_C(s) = \frac{I_2(s)-I_1(s)}{Cs}+\frac{V_C(0^-)}{s}\\
   & \Leftrightarrow &RI_1(s) &= \frac{I_2(s)-I_1(s)}{Cs}+\frac{V_C(0^-)}{s}\\
    & \Leftrightarrow &I_1(s)(R+\frac{1}{Cs}) &= \frac{V_C(0^-)}{s}+\frac{I_2(s)}{Cs}\\
    & \Leftrightarrow &I_1(s) &= \frac{I_2(s)+CV_C(0^-)}{Cs}\frac{Cs}{1+sRC}
\end{align*}
\begin{equation*}
   \Rightarrow  I_1(s)= \frac{I_2(s)+CV_C(0^-)}{1+sRC}
\end{equation*}
En réécrivant $V_o(s)$ :
\begin{align*}
   & \Rightarrow &V_o(s) &= V_{R^{'}}(s) + V_L(s)~~\mbox{avec}~~ V_L(s) = sLI_2(s)-LI_L(0^-)\\
   & \Leftrightarrow &V_o(s) &= RI_2(s) +sLI_2(s)-LI_L(0^-)\\
    & \Leftrightarrow &V_o(s) &= I_2(s)(R+sL) - LI_L(0^-) \\
    & \Leftrightarrow &I_2(R+sL) &= V_o(s) + LI_L(0^-)
\end{align*}
\begin{equation*}
   \Rightarrow  I_2(s)= \frac{V_o(s)+LI_L(0^-)}{R+sL}
\end{equation*}
Exprimons $I_1(s)$ en fonction de $V_o(s)$ :
\begin{equation*}
    I_1(s) = \frac{V_o(s)+LI_L(0^-)+(R+sL)CV_C(0^-)}{(R+sL)(1+sRC)}
\end{equation*}
En effectuant une loi des mailles sur la maille globale nous pouvons écrire:
\begin{align*}
   & \Rightarrow &\frac{2}{s} &= V_R(s) + V_o(s)\\
   & \Leftrightarrow &\frac{2}{s} &= RI_1(s)+V_o(s)\\
    & \Leftrightarrow &\frac{2}{s} &= R(\frac{V_o(s)+LI_L(0^-)+RCV_C(0^-)+sLCV_C(0^-)}{R+sR^2C+sL+s^2RLC})+V_o(s)\\
    & \Leftrightarrow &2R+2R^2Cs+2Ls+2RCLs^2 &= Rs(V_o(s)+LI_L(0^-)+RCV_C(0^-))+s^2RLCV_C(0^-)\\
     &&& +s(R+s(R^2C+L)+s^2RLC)V_o(s)\\
    & \Leftrightarrow &V_o(s)(2R+s(R^2C+L)+s^2RLC) &= 2R+s(2R^2C+2L-I_L(0^-)RL\\
     &&& -R^2CV_C(0^-))+s^2(2RLC-RLCV_C(0^-)) 
\end{align*}
\begin{equation*}
   \Rightarrow  V_o(s)= \frac{2R+s(2R^2C+2L-RLI_L(0^-)-R^2CV_C(0^-))+s^2(2RCL-RLCV_C(0^-))}{s(2R+s(R^2C+L)+s^2RCL)}
\end{equation*}
En substituant les paramètres par leurs valeurs nous obtenons :
\begin{equation*}
    V_o(s) = \frac{2000+3\cdot 10^{-3}s+ 1.5\cdot 10^{-9}s^2}{s(2000+2\cdot 10^{-3}s+10^{-9}s^2)}= 1.5\cdot \left[\frac{\frac{4}{3}10^{12}+2\cdot 10^{-6}s+s^2}{s(2\cdot 10^{12}+2\cdot 10^{6}s+s^2)}\right]
\end{equation*}
En effectuant la décomposition en fraction simple nous avons :
\begin{equation*}
    V_C(s) = \frac{A}{s} + \frac{B\cdot 10^{6}}{(s+10^6)^2+10^{12}}+\frac{C(10^6+s)}{(s+10^{6})^2+10^{12}} \Leftrightarrow 
\begin{dcases}
A+C =1.5\\
 2\cdot 10^6 A+ 10^6B+10^6C = 3\cdot 10^6\\
 2\cdot 10^{12}A = 2\cdot 10^{12}\\
\end{dcases}
\Leftrightarrow
\begin{dcases}
 A= 1\\
 B= \frac{1}{2}\\
 C=\frac{1}{2}\\
\end{dcases}
\end{equation*}
Nous obtenons donc dans le domaine de Laplace :
\begin{equation*}
    V_o(s) = \frac{1}{s} + \frac{1}{2} (\frac{10^6}{(s+10^6)^2+10^{12}}+ \frac{10^6+s}{(s+\cdot 10^{6})^2+10^{12}})
\end{equation*}
Cela correspond dans le domaine temporel à : 
\begin{equation*}
    V_o(t) =  \left[1+\frac{1}{2}\cdot e^{-10^6t}(\cos(10^6t)+\sin(10^6t))\right]u(t)
\end{equation*}
La solution équivalente avec seulement un cosinus est :
\begin{equation*}
    V_o(t)= \left[1 + \frac{1}{2\sqrt{2}}\cdot e^{-10^6t}\cdot\cos\left(10^6t + 135^\circ \right)\right]u(t)
\end{equation*}
Le graphe ressemble à
\begin{center}
    \begin{tikzpicture}
        \begin{axis}[enlargelimits=true,grid=major,ylabel=$V_o(t)$,xlabel=$t(s)$]
            \addplot [blue,domain=0:0.00001,samples=200]{1+(1/2)*e^(-(10^6)*x)*cos(10^6*x)+(1/2)*e^(-10^6*x)*sin(10^6*x)};
        \end{axis}
    \end{tikzpicture}
\end{center}
\end{solution}
\end{document}