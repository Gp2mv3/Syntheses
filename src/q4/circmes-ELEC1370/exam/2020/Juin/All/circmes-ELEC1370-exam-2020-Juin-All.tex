\documentclass[fr]{../../../../../../eplexam}
\usepackage{../../../../../../eplunits}
\usepackage[oldvoltagedirection]{circuitikz}
\usepackage{bodegraph}
\usepackage{pgfplots}
\usepackage{amsmath}
\usepackage{enumitem}
\pgfplotsset{compat=newest}
\tikzset{meter/.style={draw,thick,circle,fill=white,minimum size =0.75cm,inner sep=0pt}}
\hypertitle{circmes-ELEC1370}{4}{ELEC}{1370}{2020}{Juin}{Maj}
{Brieuc Balon \and Ottilie Bonfanti}
{Claude Oestges, Bruno Dehez and Christophe Craeye}

\section{Question Oestges : Bode}
On considère le circuit de filtrage suivant où l'amplificateur opérationnel est considéré idéal. Les composants ont les valeurs suivantes : $R_1 = 10k\Omega$, $R_2 = 20k\Omega$, $R_3=20k\Omega$, $C_1=10nF$.
\begin{center}
    \begin{circuitikz} 
    	\draw
  		(5,3) node[op amp] (opamp) {}
  		(opamp.-) --(3.8,4.5) 
  		to[R,l^=$R_3$](7.5,4.5)--(7.5,3)
  		(opamp.out) to [short,-o](8,3) node[right]{$V_{out}$}{}
  		(1,4.5) node[left]{$V_{in}$} to [R,l^=$R_2$,o-](4.5,4.5)
  		(1.5,4.5) --(1.5,2.5)
  		to[C,l_=$C_1$] (opamp.+)
  		(3.5,2.5) to [R,l^=$R_1$](3.5,0)node[ground]{};
 	\end{circuitikz}
\end{center}
\begin{enumerate}
    \item On veut déterminer l'expression analytique (sans remplacer par les valeurs) du gain en tension $G(\omega)=\frac{V_{in}}{V_{out}}$. Dans cette première sous-question, indiquez les équations des mailles et des noeuds nécessaires à la résolution du problème.
    \item Quelle est l'expression analytique (sans remplacer par les valeurs) du gain $G(\omega)=\frac{V_{in}}{V_{out}}$ ?
    \item Tracer le diagramme de Bode (amplitude) de $G(\omega)$ pour les valeurs données des éléments.
    \item Tracer le diagramme de Bode (phase) de $G(\omega)$ pour les valeurs données des éléments. Quelle est la fonction de ce circuit ?
    \item Quelle est l'expression analytique de l'impédance d'entrée $Z_{in}$ si l'on considère que la charge est une impédance de $10k\Omega$ (limiter les calculs : il n'est pas nécessaire de calculer les matrices du quadripôles pour répondre).
    \item Quelle est l'expression analytique de l'impédance de sortie $Z_{out}$ si l'on considère que la source en entrée a une résistance interne égale à $75\Omega$ (limiter les calculs : il n'est pas nécessaire de calculer les matrices du quadripôles pour répondre).
\end{enumerate}
\begin{solution}
\begin{enumerate}
    \item Comme l'amplificateur opérationnel est en rétroaction négative et est supposé idéal ($i^-=i^+=0[A]$) nous obtenons:
    \begin{equation*}
        \begin{dcases}
         V^+ = V^-\\
         \frac{V_{in}-V^-}{R_2} = \frac{V^- - V_{out}}{R_3}\\
         (V_{in} -V^+)j\omega C_1 = \frac{V^+}{R_1}
        \end{dcases}
    \end{equation*}
    \item En prenant la dernière équation:
    \begin{equation*}
        V^+(\frac{1}{R_1}+j\omega C_1)= V_{in}j\omega C_1 \Leftrightarrow V^+ = \frac{R_1j\omega C_1}{1+j\omega C_1R_1}
    \end{equation*}
    En prenant la deuxième équation:
    \begin{align*}
        -\frac{V_{out}}{R_3} &= \frac{V_{in}}{R_2}-V^-(\frac{1}{R_2}+\frac{1}{R_3})\\
        -\frac{V_{out}}{R_3} &= V_{in} (\frac{1}{R_2}-\frac{R_1C_1j\omega}{1+R_1C_1j\omega}\frac{R_2+R_3}{R_2R_3})\\
        -\frac{V_{out}}{R_3} &= V_{in} (\frac{R_3+j\omega C_1R_1R_3 -(R_2+R_3)R_1j\omega C_1}{R_2R_3(1+j\omega C_1R1)})\\
        -\frac{V_{out}}{R_3} &=V_{in}\frac{R_3-j\omega C_1R_1R_2}{R_2R_3(1+j\omega C_1R_1)}\\
        \Rightarrow G(\omega) &= -\frac{R_3}{R_2}(\frac{1-j\omega C_1\frac{R_1R_2}{R_3}}{1+j\omega C_1R_1})
    \end{align*}
    \item La fonction de transfert du gain $g_f$ peut être réécrite sous la forme générale : 
    \begin{equation*}
    H(j\omega) = K \frac{1-j\frac{\omega}{\omega_0}}{1+j\frac{\omega}{\omega_1}}~~
    \mbox{avec}~~
    \left\{\begin{matrix*}[l]
    K= -\frac{R_3}{R_2} =-1\\
    \omega_0 = \frac{R_3}{R_1R_2C_1}= 10^{4}[rad/s]\\
    \omega_1 = \frac{1}{R_1C_1}= 10^4[rad/s] \\
    \end{matrix*}\right.
    \end{equation*}
    \begin{center}
     \begin{tikzpicture}[
       gnuplot def/.append style={prefix={}},
       ]
    % Grid Style
    \tikzset{
     semilog lines/.style={black},
      semilog lines 2/.style={gray,dotted},
      semilog half lines/.style={gray, dotted},
      semilog label x/.style={below,font=\tiny},
      semilog label y/.style={above,font=\tiny} }
    % Magnitude Plot
    \begin{scope}[xscale=7/5, yscale=3/50]
        \UnitedB
        \semilog{0}{8}{-30}{30} 
        %Asymp
         \BodeGraph[green,samples=1000]{0:5.5}{\POAmpAsymp{1}{0.0001}}
         \BodeGraph[blue,samples=1000]{0:5.5}{-\POAmpAsymp{-1}{-0.0001}}
        %Real
        \BodeGraph[red,samples=1000]{0:8}{\POAmp{1}{0.0001}-\POAmp{-1}{-0.0001}}
    \end{scope}
    % Phase plot
    \begin{scope}[yshift=-7cm,xscale=7/5,yscale=3/180]
        \UniteDegre
        \OrdBode{30}
        \semilog{0}{8}{-90}{270}
        %Asymp
         \BodeGraph[green,samples=1000]{0:8}{\POArgAsymp{1}{0.0001}}
         \BodeGraph[blue,samples=1000]{0:8}{-\POArgAsymp{1}{-0.0001}-\IntArg{-100000}-\IntArg{-100000}}
        %Real
        \BodeGraph[red,samples=1000]{0:8}{\POArg{1}{0.0001}-\POArg{-1}{-0.0001}-\IntArg{-100000}-\IntArg{-100000}}
    \end{scope}
    \end{tikzpicture}
   \end{center}
   \item C'est un filtre passe-tout qui déphase de $180^\circ$ avant la fréquence de coupure. 
   \item Pour déterminer l'impédance d'entrée $Z_{in}$, nous allons utiliser la représentation Z du quadripôle.
   \begin{equation*}
        \begin{bmatrix}
        V_i \\
        V_o
        \end{bmatrix} = 
     \begin{bmatrix}
        z_i & z_r \\
        z_f & z_o
     \end{bmatrix}
     \begin{bmatrix}
         i_i \\
         i_o
     \end{bmatrix}
   \end{equation*} 
   En annulant le courant d'entrée $i_i$, le circuit devient :
   \begin{center}
    \begin{circuitikz} 
    	\draw
  		(5,3) node[op amp] (opamp) {}
  		(opamp.-) --(3.8,4.5) 
  		to[R,l^=$R_3$](7.5,4.5)--(7.5,3)
  		(opamp.out) to [short,-o](8,3)
  		(7.5,3) to[short, i_<=$i_o$](8,3)
  		(1,4.5) to [R,l^=$R_2$,o-](4.5,4.5)
  		(1.5,4.5) --(1.5,2.5)
  		to[C,l_=$C_1$] (opamp.+)
  		(3.5,2.5) to [R,l^=$R_1$](3.5,0)node[ground]{}
  		(3.5,0) to[short,-o](8,0)
  		to [open,v>=$V_{out}$](8,3) 
  		(3.5,0)to[short,-o](1,0)
  		to [open,v^>=$V_{in}$](1,4.5);
 	\end{circuitikz}
   \end{center}
   Comme nous avons un amplificateur opérationnel que l'on considère comme idéal, son impédance d'entrée vaut 0$\Omega$. Le courant $i_0$ passe donc entièrement dans l'amplificateur opérationnel et aucun courant ne passe dans le reste du circuit. Dès lors tout le circuit est à 0V. Nous obtenons donc :
   \begin{equation*}
   \left \{
    \begin{array}{rcl}
      V_i &=& 0\cdot i_o \\
      V_o &=& 0 \cdot i_o\\
    \end{array}
    \right.
    \Leftrightarrow 
    \left \{
      \begin{array}{rcl}
       z_r &=& 0 [\Omega]\\
       z_o &=&  0 [\Omega]\\
      \end{array}
    \right.
    \end{equation*}
    On remarque que les termes de droite de la matrice permettent de simplifier le reste des calculs. Calculons $Z_{in}$ :
    \begin{align*}
        Z_{in} &= Z_i (1+\frac{z_r}{z_i}A_{if,o}) ~~\mbox{avec}~~A_{if,o} = -\frac{z_f}{z_o+z_L}=-\frac{z_f}{z_L}\\
        &=z_i(1+\frac{0}{z_i}\frac{-z_f}{z_L})\\
        &=z_i
    \end{align*}
    L'impédance d'entrée ne dépend que de $z_i$. Pour calculer cela nous annulons le courant d'entrée $i_o$ et le circuit devient :
    \begin{center}
    \begin{circuitikz} 
    	\draw
  		(5,3) node[op amp] (opamp) {}
  		(opamp.-) --(3.8,4.5) 
  		to[R,l^=$R_3$](7.5,4.5)--(7.5,3)
  		(opamp.out) to [short,-o](8,3)
  		(7.5,3) to[short](8,3)
  		(1,4.5) to [short,i=$i_i$,o-](1.5,4.5) to[R,l^=$R_2$,i>_=$i_2$,*-](4.5,4.5)
  		(1.5,4.5) to[short,i>=$i_1$](1.5,2.5)
  		to[C,l_=$C_1$] (opamp.+)
  		(3.5,2.5) to [R,l^=$R_1$](3.5,0)node[ground]{}
  		(3.5,0) to[short,-o](8,0)
  		to [open,v>=$V_{out}$](8,3) 
  		(3.5,0)to[short,-o](1,0)
  		to [open,v^>=$V_{in}$](1,4.5);
 	\end{circuitikz}
    \end{center}
    Nous pouvons effectuer une loi des mailles sur la maille tout à gauche, nous obtenons :
    \begin{equation*}
        V_{in} =i_1(R_1+\frac{1}{j\omega C_1}\Leftrightarrow i_1 = \frac{j\omega C_1}{1+R_1C_1j\omega}V_{in}
    \end{equation*}
    Nous pouvons exprimer le courant $i_2$ avec :
    \begin{align*}
        i_2 &= \frac{V_{in}-V^-}{R_2}~~\mbox{avec}~~V^-=V^+=R_1i_1\\
        &=\frac{V_{in}-R_1i_1}{R_2}\\
        &=\frac{V_{in}}{R_2}(1-\frac{j\omega R_1C_1}{1+R_1C_1j\omega})\\
        &=\frac{V_{in}}{R_2}(\frac{1}{1+R_1C_1j\omega})
    \end{align*}
    Nous avons exprimé $i_1$ et $i_2$ au moyen de $V_{in}$. Nous allons maintenant lier $V_{in}$ à $i_i$ au moyen de la relation :
    \begin{align*}
        i_i &= i_1 +i_2\\
        &=V_{in}(\frac{1+R_2C_1j\omega}{R_2(1+R_1C_1j\omega)})\\
        \Leftrightarrow V_{in} &= \frac{R_2(1+R_1C_1j\omega)}{1+R_2C_1j\omega} i_i
    \end{align*}
    \begin{equation*}
        \Rightarrow Z_{in}=z_i = \frac{R_2(1+R_1C_1j\omega)}{1+R_2C_1j\omega} [\Omega]
    \end{equation*}
    \item Comme nous avons un amplificateur opérationnel que l’on considère comme idéal, nous pouvons supposer que $Z_{out}=$ 0 $\Omega$. Si cela ne vous apparaît pas vous pouvez toujours calculer l'impédance de sortie au moyen de la table des caractéristiques externes: 
    \begin{align*}
         Z_{out} &= z_o-z_r\frac{z_f}{z_G+z_i}~~ \mbox{avec}~~ z_G = 75[\Omega]\\
         &= 0-0\cdot \frac{z_f}{z_G+z_i}\\
         &=0
    \end{align*}
    \begin{equation*}
       \Rightarrow Z_{out} = 0 [\Omega]
    \end{equation*}
\end{enumerate}
\end{solution}
\newpage
\section{Question Craeye : transitoire}
On considère le circuit suivant où $L=2mH$, $C=0.25nF$ et $R=6k\Omega$ et où l'interrupteur passe de A à B au temps $t=0$.
\begin{center}
    \begin{circuitikz} 
        \draw (0,2) node[ground]{} to[american voltage source,l=$1V$] (0,4) to [R,l=$R$] (0,6) to[short,-o] (1,6) coordinate (B);
        \draw (2,2) node[ground]{}to[american voltage source,l=$3V$] (2,3) to[R,l=$R$,-o] (2,5) coordinate (A)
        (1.4,5.4) to[short,o-] (2,6)
        (8,5.5) node[op amp] (opamp) {}
        (2,6) to [L,l_=$L$](opamp.-)
        (opamp.-) to[short,*-] ++(0,2.5)
        to[R,l^=$R$]++(3,0) to [short]++(0,-3)
        (opamp.+)to[R,l^=$R$](6.8,3)node[ground]{}
        (6.8,7)to[C,l_=$C$](9.8,7)
        (opamp.out)to[short,-*](9.8,5.5) to [short,-o] (11,5.5)node[above]{$V_o$}{};
        \draw [->,>=stealth] ($ (A) + (-0.2,0) $)  to [bend left] node[pos=0.5,below left] {$t=0$} ($ (B) + (0,-0.2) $);
        \draw (A) node[right]{$A$};
        \draw (B) node[above]{$B$};
    \end{circuitikz}
\end{center}
\begin{enumerate}
    \item Dans le domaine de Laplace, donner les relations qui lient la tension de source, la tension $V_o$ et le courant dans l'inductance.
    \item Donner l'expression de la tension $V_o$ en fonction du temps pour $t>0$
\end{enumerate}
\begin{solution}
\begin{enumerate}
    \item En $t<0$, on peut réécrire le circuit comme :
\begin{center}
    \begin{circuitikz} 
        \draw (2,3) node[ground]{}to[american voltage source,l=$3V$] (2,6)
        (8,5.5) node[op amp] (opamp) {}
        (2,6) to[R,l_=$R$,v^=$ $] ++(3,0) to [short,i>_=$I_L(0^-)$](opamp.-)
        (opamp.-) to[short,*-] ++(0,2.5)
        to[R,l^=$R$,v_=$ $]++(3,0) to [short]++(0,-3)
        (opamp.+)to[R,l^=$R$](6.8,3)node[ground]{}
        (6.8,7)to[open,v^=$V_C(0^-)$,*-*](9.8,7)
        (opamp.out)to[short,-*](9.8,5.5) to [short,-o] (11,5.5)node[above]{$V_o$}{};
    \end{circuitikz}
\end{center}
Étant donner que l'amplificateur opérationnel est considéré comme parfait ($i^+=i^-=0A$) et est en rétroaction négative ($V^+=V^-=0V$) car la masse ne délivre aucune tension et aucun courant, nous obtenons:
\begin{equation*}
    \left \{
\begin{array}{rcl}
I_L(0^-) &=& \frac{3}{R}=5\cdot 10^{-4} [A]\\
V_{C}(0^-) &=& 0-V_o = R I_L(0^-)=3[V]\\
\end{array}
\right.
\end{equation*}
En $t>0$, on peut réécrire le circuit comme :
\begin{center}
    \begin{circuitikz} 
        \draw (2,3) node[ground]{}to[american voltage source,l=$1V$] (2,6)
        (8,5.5) node[op amp] (opamp) {}
        (2,6) to[R,l_=$R$,v^=$V_R(s)$] ++(3,0) to [L,l_=$L$,i>_=$I_1(s)$,v^=$V_L(s)$](opamp.-)
        (opamp.-) to[short,*-] ++(0,3.5)
        to[R,l_=$R$,v^=$V_{R^{'}}(s)$,i>_=$I_2(s)$]++(3,0) to [short]++(0,-4)
        (opamp.+)to[R,l^=$R$](6.8,3)node[ground]{}
        (6.8,7)to[C,l_=$C$,v^=$V_C(s)$](9.8,7)
        (opamp.out)to[short,-*](9.8,5.5) to [short,-o] (11,5.5)node[above]{$V_o(s)$}{};
    \end{circuitikz}
\end{center}
Nous pouvons effectuer une loi des mailles sur la maille de gauche:
\begin{align*}
    \frac{1}{s} &= RI_1(s)+V_L(s)~~\mbox{avec}~~V_L(s) = LsI_1(s)-LI_L(0^-)\\
    \Leftrightarrow \frac{1+sLI_L(0^-)}{s} &= I_1(s)(R+sL)\\
    \Rightarrow I_1(s) &= \frac{1+sLI_L(0^-)}{s(R+sL)}
\end{align*}
Nous pouvons effectuer une loi des mailles sur la maille supérieure : 
\begin{align*}
    V_C(s) &= V_{R^{'}}(s) ~~\mbox{avec}~~V_C(s) =-V_o(s)\\
    \Leftrightarrow V_C(s) &= -RI_2(s) \\
    \Rightarrow I_2(s) &= -\frac{V_o(s)}{R}
\end{align*}
Le courant dans la capacité peut s'écrire : 
\begin{align*}
    I_1(s) -I_2(s) &= sCV_C(s) -CV_C(0^-) ~~\mbox{avec}~~V_C(s) =-V_o(s)\\
    \Leftrightarrow \frac{1+sLI_L(0^-)}{s(R+sL)}+\frac{V_o(s)}{R} &= -scV_o(s) -cV_c(0^-)\\
    \Leftrightarrow -V_o(s) (\frac{1}{R}+sC) &= \frac{1+sLI_L(0^-)+sRCV_C(0^-)+s^2LCV_C(0^-)}{s(R+sL)}\\
    \Leftrightarrow V_o(s) &= -\frac{R+sR(LI_L(0^-)+RCV_C(0^-))+s^2RLCV_C(0^-)}{s(R+sL)(1+sRC)}
\end{align*}
\begin{equation*}
    \Rightarrow V_o(s) = -\frac{\frac{1}{LC}+\frac{s}{LC}(LI_L(0^-)+RCV_C(0^-))+s^2V_C(0^-)}{s(s+\frac{R}{L})(s+\frac{1}{RC})}
\end{equation*}
\item En substituant les paramètres par leurs valeurs nous obtenons :
\begin{equation*}
    V_o(s) = -\frac{2\cdot 10^{12}+ 11\cdot 10^{6}s+3s^2}{s(s+3\cdot 10^{6})(s+\frac{2\cdot10^6 }{3})}
\end{equation*}
En effectuant la décomposition en fraction simple nous avons :
\begin{equation*}
    V_C(s) = \frac{A}{s} + \frac{B}{s+3\cdot10^6}+\frac{C}{s+\frac{2\cdot10^6 }{3}} \Leftrightarrow 
\begin{dcases}
 A + B + C = -3\\
 (3+\frac{2}{3}) 10^6 A+ \frac{2}{3}10^6B+ 3\cdot 10^6C = -11\cdot 10^6\\
 2\cdot 10^{12}A = -2 \cdot 10^{12}\\
\end{dcases}
\Leftrightarrow
\begin{dcases}
 A= -1\\
 B= \frac{4}{7}\\
 C=-\frac{18}{7}\\
\end{dcases}
\end{equation*}
Nous obtenons donc dans le domaine de Laplace :
\begin{equation*}
    V_o(s) = -\frac{1}{s} +\frac{4}{7} \cdot \frac{1}{s+3\cdot10^6}- \frac{18}{7}\cdot \frac{1}{s+\frac{2\cdot10^6}{3}}
\end{equation*}
Cela correspond dans le domaine temporel à : 
\begin{equation*}
    V_o(t) =  \left[-1+\frac{4}{7}\cdot e^{-3\cdot 10^6t}-\frac{18}{7}\cdot e^{-\frac{2}{3}\cdot 10^6t}\right]u(t)
\end{equation*}
Le graphe ressemble à
\begin{center}
    \begin{tikzpicture}
        \begin{axis}[enlargelimits=true,grid=major,ylabel=$V_o(t)$,xlabel=$t(s)$]
            \addplot [blue,domain=0:0.00001,samples=200]{-1+(4/7)*e^(-3*x*10^6)-(18/7)*e^(-(2/3)*x*10^6};
        \end{axis}
    \end{tikzpicture}
\end{center}
\end{enumerate}
\end{solution}
\newpage 
\section{Question Dehez : triphasé}
On considère le circuit suivant, alimenté par une source de tension triphasée dont la fréquence est de 50Hz.
\begin{center}
    \begin{circuitikz}[scale=0.9]
        \coordinate (s1) at (0,0);
        \coordinate (s2) at ($ (s1) + ({6*cos(60)},{-6*sin(60)}) $);
        \coordinate (s3) at ($ (s1) + ({-6*cos(60)},{-6*sin(60)}) $);
        \coordinate (c3) at ($ (s1) + (0,{-sqrt(9)}) $);
        
        %sources
        \draw (c3) to[V,l=$220\angle 0^\circ \,V$,i=$I_1$,*-] (s1);
        \draw (c3) to [V,l=$220\angle 120^\circ \,V$,i_=$I_2$] (s2); 
        \draw (c3) to [V,l_=$220\angle 240^\circ \,V$,i=$I_3$] (s3);
        
        \draw (s1) to [R,l=$1\Omega$] ++(9,0) coordinate (l11);
        \draw (s3) to[short] ++(0,-2) coordinate (p1);
        
        \coordinate (l12) at ($ (l11) + ({6*cos(60)},{-6*sin(60)}) $);
        \coordinate (l13) at ($ (l11) + ({-6*cos(60)},{-6*sin(60)}) $);
        \coordinate (c1) at ($ (l11) + (0,{-sqrt(10)}) $);
        
        %inductances
        \draw (l11) to[L,l_=$1j\Omega$] (9,-1.5) coordinate (l111);
        \draw (l13) to[L,l_=$1j\Omega$] (7,-4.5) coordinate (l133)  ;
        \draw  (l12)to[L,l_=$1j\Omega$]  (11,-4.5) coordinate (l122) ;
        
        %résistances
        \draw (l13) to[R,l=$1\Omega$] (s2);
        \draw (l12) to[short] ++(0,-2) to[R,l=$1\Omega$] (p1);
        
        %Résistances triangle
        \draw (l111) to [R,l^=$60\Omega$](l122);
        \draw (l122) to [R,l^=$60\Omega$](l133);
        \draw (l111) to [R,l_=$60\Omega$](l133);
        
        %bullets inductances
        \draw (l11) node[below left]{$\bullet$};
        \draw (l12) node[above]{$\bullet$};
        \draw (l13) node[below right]{$\bullet$};
        
         %inductances mutuelles
        \draw [<->,>=stealth] ($ (l11) + (0.3,-0.2) $)  to [bend left] node[pos=0.5,above right] {$j\Omega$} ($ (l12) + (0.2,0.2) $);
        \draw [<->,>=stealth] ($ (l12) + (-0.2,-0.2) $)  to [bend left] node[pos=0.5,above left] {$j\Omega$} ($ (l13) + (0,-0.4) $);
        \draw [<->,>=stealth] ($ (l13) + (-0.2,0.2) $)  to [bend left] node[pos=0.5,above] {$j\Omega$} ($ (l11) + (-0.4,-0.2) $);
    
    \end{circuitikz}
\end{center}
\begin{enumerate}
    \item Calculer le facteur de dispersion entre les inductances couplées magnétiquement.
    \item Calculer le module et la phase du courant $I_1$.
    \item Calculer la valeur de la puissance active fournie par la source triphasée.
    \item Calculer la valeur de la puissance réactive fournie par la source triphasée. 
    \item Calculer la valeur la valeur de la capacité C et de ces condensateurs qui permettra d'annuler la puissance réactive fournie par la source.\footnote{Ce n'est pas un noeud là où la capacité coupe le câble avec le courant $I_2$}
    
\begin{center}
    \begin{circuitikz}[scale=0.9]
        \coordinate (s1) at (0,0);
        \coordinate (s2) at ($ (s1) + ({6*cos(60)},{-6*sin(60)}) $);
        \coordinate (s3) at ($ (s1) + ({-6*cos(60)},{-6*sin(60)}) $);
        \coordinate (c3) at ($ (s1) + (0,{-sqrt(9)}) $);
        
        %sources
        \draw (c3) to[V,l=$220\angle 0^\circ \,V$,i=$I_1$,*-] (s1);
        \draw (c3) to [V,l=$220\angle 120^\circ \,V$,i_=$I_2$] (s2); 
        \draw (c3) to [V,l_=$220\angle 240^\circ \,V$,i=$I_3$] (s3);
        
        \draw (s1) to [R,l=$1\Omega$] ++(6,0) coordinate (c11) to [short] ++(5,0)coordinate (l11);
        \draw (s3) to[short] ++(0,-2) coordinate (p1);
        
        \coordinate (l12) at ($ (l11) + ({6*cos(60)},{-6*sin(60)}) $);
        \coordinate (l13) at ($ (l11) + ({-6*cos(60)},{-6*sin(60)}) $);
        \coordinate (c1) at ($ (l11) + (0,{-sqrt(10)}) $);
        
        %inductances
        \draw (l11) to[L,l_=$1j\Omega$] (11,-1.5) coordinate (l111);
        \draw (l13) to[L,l_=$1j\Omega$] (9,-4.5) coordinate (l133)  ;
        \draw  (l12)to[L,l_=$1j\Omega$]  (13,-4.5) coordinate (l122) ;
        
        %résistances
        \draw (l13) to [short] ++(-3,0) coordinate(c22) to[R,l=$1\Omega$] (s2);
        \draw (l12) to[short] ++(0,-2) to[short] ++(-6,0) coordinate (c33) to[R,l=$1\Omega$] (p1);
        
        %Résistances triangle
        \draw (l111) to [R,l^=$60\Omega$](l122);
        \draw (l122) to [R,l^=$60\Omega$](l133);
        \draw (l111) to [R,l_=$60\Omega$](l133);
        
        %bullets inductances
        \draw (l11) node[below left]{$\bullet$};
        \draw (l12) node[above]{$\bullet$};
        \draw (l13) node[below right]{$\bullet$};
        
         %inductances mutuelles
        \draw [<->,>=stealth] ($ (l11) + (0.3,-0.2) $)  to [bend left] node[pos=0.5,above right] {$j\Omega$} ($ (l12) + (0.2,0.2) $);
        \draw [<->,>=stealth] ($ (l12) + (-0.2,-0.2) $)  to [bend left] node[pos=0.5,above left] {$j\Omega$} ($ (l13) + (0,-0.4) $);
        \draw [<->,>=stealth] ($ (l13) + (-0.2,0.2) $)  to [bend left] node[pos=0.5,above] {$j\Omega$} ($ (l11) + (-0.4,-0.2) $);
        %Capacités
        \draw (c11) to [C,l=$C$,*-*]++(0,-3)coordinate (c00);
        \draw (c00) to [C,l=$C$,-*] (c22);
        \draw (c33) to [C,l=$C$,*-]  ++(-1,2) coordinate (c000);
        \draw (c000) to [short] (c00);
    \end{circuitikz}
\end{center}
\end{enumerate}
\begin{solution}
\begin{enumerate}
    \item Le facteur de dispersion s’écrit comme 
     \begin{equation*}
        \sigma = 1 -\frac{M^2}{L_1L_2} = 1 - \frac{(j)^2}{2j*2j} = 0.75
    \end{equation*}
    Nous obtenons bien un facteur de dispersion $\in$ [0,1].
    \item Nous allons commencer par changer la charge triangle formée par les résistance en une charge étoile en sachant que $R_{étoile}$ = $\frac{R_{triangle}}{3}$.
    On obtient le schéma simplifié ci-dessous
    
\begin{center}
    \begin{circuitikz}[scale=0.9]
        \coordinate (s1) at (0,0);
        \coordinate (s2) at ($ (s1) + ({6*cos(60)},{-6*sin(60)}) $);
        \coordinate (s3) at ($ (s1) + ({-6*cos(60)},{-6*sin(60)}) $);
        \coordinate (c3) at ($ (s1) + (0,{-sqrt(9)}) $);
        
        %sources
        \draw (c3) to[V,l=$220\angle 0^\circ \,V$,i=$I_1$,*-] (s1);
        \draw (c3) to [V,l=$220\angle 120^\circ \,V$,i=$I_2$] (s2); 
        \draw (c3) to [V,l=$220\angle 240^\circ \,V$,i=$I_3$] (s3);
        
        \draw (s1) to [R,l=$1\Omega$] ++(9,0) coordinate (l11);
        \draw (s3) to[short] ++(0,-2) coordinate (p1);
        
        \coordinate (l12) at ($ (l11) + ({6*cos(60)},{-6*sin(60)}) $);
        \coordinate (l13) at ($ (l11) + ({-6*cos(60)},{-6*sin(60)}) $);
        \coordinate (c1) at ($ (l11) + (0,{-sqrt(10)}) $);
        
        %inductances
        \draw (l11) to[L,l_=$1j\Omega$] (9,-1.5);
        \draw (9,-1.5) to[R,l=$20\Omega$,-*] (c1);
        \draw (c1) to[R,l=$20\Omega$] (7,-4.5);
        \draw (7,-4.5) to[L,l_=$1j\Omega$] (l13);
        \draw (c1) to[R,l=$20\Omega$] (10.5,-4.3);
        \draw (10.5,-4.3) to[L,l_=$1j\Omega$] (l12);
        
        %résistances
        \draw (l13) to[R,l=$1\Omega$] (s2);
        \draw (l12) to[short] ++(0,-2) to[R,l=$1\Omega$] (p1);

        %bullets inductances
        \draw (l11) node[below left]{$\bullet$};
        \draw (l12) node[above]{$\bullet$};
        \draw (l13) node[below right]{$\bullet$};
        
         %inductances mutuelles
        \draw [<->,>=stealth] ($ (l11) + (0.3,-0.2) $)  to [bend left] node[pos=0.5,above right] {$j\Omega$} ($ (l12) + (0.2,0.2) $);
        \draw [<->,>=stealth] ($ (l12) + (-0.2,-0.2) $)  to [bend left] node[pos=0.5,above left] {$j\Omega$} ($ (l13) + (0,-0.4) $);
        \draw [<->,>=stealth] ($ (l13) + (-0.2,0.2) $)  to [bend left] node[pos=0.5,above] {$j\Omega$} ($ (l11) + (-0.4,-0.2) $);
    
    \end{circuitikz}
\end{center}

    Nous allons ensuite enlever l'inductance mutuelle entre chaque pair d'inductance en sachant que la somme des courants est nulle. On a donc 
    \begin{align*}
        2jI_1 + 1jI_2 + 1jI_3  &= 2jI_1 + 1j(I_2 + I_3) \\
        &= 2jI_1 + 1j(-I_1) \\
        &= 1jI_1
    \end{align*}
        
    
    
On peut dès lors travailler sur l'équivalent monophasé de ce circuit et calculer le courant $I_1$.

\begin{center}
    \begin{circuitikz}[scale=0.9]
        \coordinate (s1) at (0,0);
        \coordinate (c3) at ($ (s1) + (0,{-sqrt(10)}) $);
        
        %sources
        \draw (c3) to[V,l=$220\angle 0^\circ \,V$,i=$I_1$] (s1);
        
        \draw (s1) to [R,l=$1\Omega$] ++(9,0) coordinate (l11);
        \coordinate (c1) at ($ (l11) + (0,{-sqrt(10)}) $);
        
        %inductances
        \draw (l11) to[L,l_=$1j\Omega$] (9,-1.5);
        \draw (9,-1.5) to[R,l=$20\Omega$] (c1);

        %bullets inductances
        \draw (l11) node[below left]{$\bullet$};
        \draw (c1) to (c3);
    
    \end{circuitikz}
\end{center}
\begin{align*}
        220\angle 0^\circ \ &= (21 + j)I_1\\
        \Leftrightarrow I_1 &= 10.4643\angle -2.7263^\circ [A] 
\end{align*}  
\item La puissance active de la source est la contribution des puissances actives des 3 circuits équivalents monophasés.
\begin{align*}
    P = 3\cdot \Re(V \cdot I_1^*) = 3\cdot \Re( 220\angle 0^\circ \ \cdot 10,4643\angle 2,7263^\circ \ ) = 6898.64 [W]
\end{align*}
\item La puissance réactive de la source est la contribution des puissances réactives des 3 circuits équivalents monophasés.
\begin{align*}
    P = 3 \cdot \Im(V \cdot I_1^*) = 3 \cdot \Im( 220\angle 0^\circ \ \cdot 10.4643\angle 2.7263^\circ \ ) = 328.506 [VAR]
\end{align*}
\item L'équivalent monophasé peut être réécris comme suit 
\begin{center}
    \begin{circuitikz}[scale=0.9]
        \coordinate (s1) at (0,0);
        \coordinate (c3) at ($ (s1) + (0,{-sqrt(10)}) $);
        
        \draw (6,0) to [C,l_=$\frac{1}{j\omega C}$] (6,-{sqrt(10)});
        
        %sources
        \draw (c3) to[V,l=$220\angle 0^\circ \,V$,i=$I_1$] (s1);
        
        \draw (s1) to [R,l=$1\Omega$] ++(9,0) coordinate (l11);
        \coordinate (c1) at ($ (l11) + (0,{-sqrt(10)}) $);
        
        %inductances
        \draw (l11) to[L,l_=$1j\Omega$] (9,-1.5);
        \draw (9,-1.5) to[R,l=$20\Omega$] (c1);

        %bullets inductances
        \draw (l11) node[below left]{$\bullet$};
        \draw (c1) to (c3);
    
    \end{circuitikz}
\end{center}
ou encore,
\begin{center}
    \begin{circuitikz}[scale=0.9]
        \coordinate (s1) at (0,0);
        \coordinate (c3) at ($ (s1) + (0,{-sqrt(10)}) $);
        %sources
        \draw (c3) to[V,l=$220\angle 0^\circ \,V$,i=$I_1$] (s1);
        \draw (s1) to [R,l=$1\Omega$] ++(9,0) coordinate (l11);
        \coordinate (c1) at ($ (l11) + (0,{-sqrt(10)}) $);
        %inductances
        \draw (l11) to[european resistor,l=Z] (c1);
        \draw (c1) to (c3);
    \end{circuitikz}
\end{center}
\begin{align*}
     Z &= (20 + j) || \frac{1}{j\omega C}\\
    &= \frac{20 + j}{1+20j\omega C - \omega C} [\Omega]
\end{align*}
Pour que la puissance réactive soit nulle, il faut que la partie imaginaire de Z soit nulle. 
\begin{align*}
    \Rightarrow Z &= \frac{20 + j}{1-\omega C+20j\omega C} \cdot  \frac{(1-\omega C) - 20j\omega C}{(1-\omega C)-20j\omega C}\\
    &= \frac{20+j-20\omega C-j\omega C-400j\omega C+20\omega C}{(1-\omega C)^2+400\omega ^2C^2}\\
    &= \frac{20+j(1-401\omega C)}{(1-\omega C)^2+400\omega ^2C^2}\\
    \Rightarrow \Im(Z) &= 0\\
    \Leftrightarrow 0 &= (1-401\omega C)\\
    \Leftrightarrow C &= \frac{1}{401\omega }
    = \frac{1}{401 \cdot 2 \cdot \pi \cdot 50}
    = 7.938 \mu F
\end{align*}

\end{enumerate}
\end{solution}
\end{document}