\documentclass[fr]{../../../../../../eplexam}
\usepackage{../../../../../../eplunits}
\usepackage[oldvoltagedirection]{circuitikz}
\usepackage{bodegraph}
\usepackage{pgfplots}
\usepackage{amsmath}
\usepackage{enumitem}
\pgfplotsset{compat=newest}
\tikzset{meter/.style={draw,thick,circle,fill=white,minimum size =0.75cm,inner sep=0pt}}
\hypertitle{circmes-ELEC1370}{4}{ELEC}{1370}{2014}{Août}{All}
{Brieuc Balon}
{Claude Oestges, Bruno Dehez and Christophe Craeye}
\section{Question Oestges : phaseurs}
On considère le circuit suivant
\begin{center}
    \begin{circuitikz} 
    	\draw
		to[european resistor,l=$jX$, -*] (0,2.5) 
  		(2.5,2.5) to[american current source,l=$2\angle 0^\circ$A] (0,2.5)
  		(2.5,2.5) to [american voltage source,l=$12\angle 0^\circ$V] (5,2.5)
  		(2.5,0) to[L,l_=j$4\Omega$,-*] (2.5,2.5)
 		(5,2.5) to[R,v=$V_o$,l=$2\Omega$,*-] (5,0)
 		(0,0) -- (5,0)
 		(0,2.5) -- (0,4) to[R,l=$R$] (5,4) -- (5,2.5);
 	\end{circuitikz}
\end{center}
On mesure une tension $V_o=5.5\angle -101.1^\circ$V. On demande de calculer
\begin{enumerate}
    \item Les valeurs de $R$ et de $X$ (ces valeurs sont réelles)
    \item la tension (amplitude et phase) aux bornes de la source de courant de 2A, ainsi que sa valeur efficace.
    \item la puissance délivrée par la source de tension
\end{enumerate}
\begin{solution}
Pour la suite des calculs nous utiliserons les notations telles qu’illustrées :
\begin{center}
    \begin{circuitikz} 
    	\draw
		to[european resistor,l=$jX$,v=$V_X$,i=$I_X$,-*] (0,2.5) 
  		(2.5,2.5) node [above]{$C$}{} to[american current source,l=$2\angle 0^\circ$A,v>=$V_C$] (0,2.5) node [left]{$B$}{}
  		(2.5,2.5) to [american voltage source,l=$12\angle 0^\circ$V,i=$I_S$] (5,2.5)
  		(2.5,0) node[below]{$A$}{} to[L,l^=j$4\Omega$,i<=$I_L$,v>=$V_L$,-*] (2.5,2.5)
 		(5,2.5) to[R,v^=$V_o$,l_=$2\Omega$,i=$I_o$,*-] (5,0)
 		(0,0) -- (5,0)
 		(0,2.5) -- (0,4) to[R,l_=$R$,v^=$V_R$,i=$I_R$] (5,4) -- (5,2.5);
 	\end{circuitikz}
\end{center}
\begin{enumerate}
    \item Sur base de la relation élémentaire V = ZI nous pouvons écrire :
    \begin{equation*}
        I_o = \frac{V_o}{2} = 2.75\angle -101.1^\circ [A]
    \end{equation*}
    Nous pouvons effectuer une loi des mailles sur la maille en bas à droite :
    \begin{equation*}
        V_L = V_o - 12\angle 0^\circ = 14.13\angle -157.54^\circ [V]
    \end{equation*}
    Sur base de la relation élémentaire V = ZI nous pouvons écrire :
    \begin{equation*}
        I_L = \frac{V_L}{4j} = 3.532\angle 112.455^\circ [A]
    \end{equation*}
    Nous pouvons effectuer une loi des noeuds sur le noeud $A$ :
    \begin{equation*}
        I_X = I_L +I_o = 1.962\angle 163.23^\circ [A]
    \end{equation*}
    Nous pouvons effectuer une loi des noeuds sur le noeud $B$ :
    \begin{equation*}
        I_R = I_X + 2\angle 0^\circ = 0.58\angle 77.91^\circ [A]
    \end{equation*}
     Nous pouvons effectuer une loi des mailles sur la maille globale :
    \begin{equation*}
        V_o + V_R +V_X = 0 \Leftrightarrow 5.5\angle -101.1^\circ + RI_R + jXI_X = 0
    \end{equation*}
    Nous avons une équation pour deux inconnues. Seulement il faut se rappeler que ces équations sont des équations complexes et donc que nous avons 2 équations pour 2 inconnues. Le système donne donc : 
    \begin{equation*}
        \left\{\begin{matrix*}[l]
        -1.059+0.121R -0.566X = 0\\
        -5.3971+0.567R - 1.878X = 0
        \end{matrix*}\right.
        \Leftrightarrow
         \left\{\begin{matrix*}[l]
        R = 11.38  \\
        X = 0.561
        \end{matrix*}\right.
    \end{equation*}
    \item Nous pouvons effectuer une loi des mailles sur la maille en bas à gauche :
    \begin{equation*}
        V_C = V_R + 12\angle 0^\circ = R I_R + 12\angle 0^\circ = 14.86\angle 25.75^\circ [V] \Rightarrow V_{eff} = |\frac{V_C}{\sqrt{2}}| = 10.5 [V] 
    \end{equation*}
     Nous pouvons effectuer une loi des noeuds sur le noeud $C$ :
     \begin{equation*}
         I_S = -(2\angle 0^\circ + I_L) = 3.33\angle -101.28^\circ [A]
     \end{equation*}
     La puissance\footnote{Nous travaillons avec des phaseurs, ils représentent la valeur efficace de la tension ou du courant. S'il n'est pas spécifié dans l'énonce si la valeur de la tension $V_o$ représente la valeur de crête ou la valeur efficace de la tension, n'hésitez pas à le spécifier dans votre exercice. Nous considérons que $V_o$ correspond à la valeur efficace dans la suite des calculs. } active délivrée par la source est : 
\begin{equation*}
    P = \Re(S) = \Re(V\cdot I_S^\ast) = \Re(12\angle 0^\circ \cdot  3.33\angle 101.28^\circ) = -7.816[W]
\end{equation*}
\end{enumerate}
\end{solution}
\section{Question Oestges : Bode et quadripôles}
On considère le circuit de filtrage suivant
\begin{center}\centering
	\begin{circuitikz}
		\draw
		(0,1) node [above]{$V_\text{in}$} to [short,o-] (1,1) 
		to [R, l=$R_1$] (2,1)
		(6,0.5) node[op amp] (opamp) {}
		(opamp.-) to [C,l_=$C_1$] (2,1) 
		(opamp.+) to [short] (4,0)
		(opamp.out) to [short] (8,0.5)
		(4.5,1) to [short,*-*] (4.5,2.5) to [R,l=$R_2$,-*] (8,2.5) -- (8,0.5)
		(4.5,2.5) -- (4.5,4) to [C,l=$C_2$] (8,4) -- (8,2.5)
		(4,0) -- (4,-0.5) node [ground] {}
		(8,0.5) to [short,*-o] (9,0.5) node [above]{$V_\text{out}$};
	\end{circuitikz}
\end{center}
où l'amplificateur opérationnel est considéré comme idéal et $C_1 = C_2 = 10nF$, $R_1 = 5k\Omega$ et $R_2 = 1k\Omega$. On demande de
\begin{enumerate}
    \item Déterminer l'expression analytique (sans remplacer par les valeurs) de la matrice \textbf{G} du quadripôle ainsi formé.
    \item Tracer le diagramme de Bode (amplitude et phase) du gain $g_f$ (pour les valeurs données des éléments). Quelle est la fonction de ce circuit (passe-haut, passe-bas, passe-bande) ? Quelle est la bande passante du filtre ?
    \item Si la charge est une impédance de $1k\Omega$, déterminer la valeur de l’impédance d’entrée $Z_\text{in}$, respectivement à 1, 10 et $100kHz$.
\end{enumerate}

\begin{solution}
\begin{enumerate}
    \item La matrice $G$ est calculée grâce à : 
    \begin{equation*}
        \begin{bmatrix}
        I_i \\
        V_o
        \end{bmatrix} = 
\begin{bmatrix}
g_i & g_r \\
g_f & g_o
\end{bmatrix}
\begin{bmatrix}
V_i \\
I_o
\end{bmatrix}
\end{equation*}
En annulant la tension $V_i$ le circuit devient :
\begin{center}\centering
	\begin{circuitikz}
		\draw
		(9,-1) to [short,-*] (4,-1)--(0,-1) to[short,i=$I_i$](0,1)
		to [R, l=$R_1$] (2,1)
		(6,0.5) node[op amp] (opamp) {}
		(opamp.-) to [C,l_=$C_1$] (2,1) 
		(opamp.+) to [short] (4,0)
		(opamp.out) to [short] (8,0.5)
		(4.5,1) to [short,*-*] (4.5,2.5) to [R,l=$R_2$,-*] (8,2.5) -- (8,0.5)
		(4.5,2.5) -- (4.5,4) to [C,l=$C_2$] (8,4) -- (8,2.5)
		(4,0) -- (4,-1.5) node [ground] {}
		(8,0.5) to [short,i<=$I_o$,*-o] (9,0.5)  to [open,v^=$V_o$,-o] (9,-1);
	\end{circuitikz}
\end{center}
Comme nous avons un amplificateur opérationnel que l'on considère comme idéal, son impédance d'entrée vaut $0\Omega$. Le courant $I_0$ passe donc entièrement dans l'amplificateur opérationnel et aucun courant ne passe dans le reste du circuit. Dès lors tout le circuit est à 0V. Nous obtenons donc :
\begin{equation*}
\left \{
    \begin{array}{rcl}
    I_i &=& 0\cdot I_o \\
    V_o &=& 0 \cdot I_o
\end{array}
\right.
\Leftrightarrow 
\left \{
    \begin{array}{rcl}
    g_r &=& 0 \\
    g_o &=&  0 [\Omega]
\end{array}
\right.
\end{equation*}
En annulant le courant $I_o$ le circuit devient :
\begin{center}\centering
	\begin{circuitikz}
		\draw
		(0,-1) to [short,o-*] (4,-1) to  [short,-o] (9,-1)
		(0,1)  to [short,o-] (1,1) 
		(0,-1) to [open,v^>=$V_i$,-o](0,1)
		(0.5,1)to [R, l=$R_1$,i=$I_i$] (2.5,1)
		(6,0.5) node[op amp] (opamp) {}
		(opamp.-) to [C,l_=$C_1$] (2.5,1) 
		(opamp.+) to [short] (4,0)
		(opamp.out) to [short] (8,0.5)
		(4.5,1) to [short,*-*] (4.5,2.5) to [R,l=$R_2$,-*] (8,2.5) -- (8,0.5)
		(4.5,2.5) -- (4.5,4) to [C,l=$C_2$] (8,4) -- (8,2.5)
		(4,0) -- (4,-1.5) node [ground] {}
		(8,0.5) to [short,*-o] (9,0.5) to [open,v^<=$V_o$,-o](9,-1);
	\end{circuitikz}
\end{center}
Étant donné que l'amplificateur opérationnel est en rétroaction négative ($V^+=V^-=0V$). Dès lors, sur base de la relation élémentaire $V = ZI$ nous pouvons écrire :
\begin{equation*}
\left \{
    \begin{array}{rcl}
    V_i &=& (r_1+\frac{1}{j\omega C_1}I_i \\
    V_o &=&  -(j\omega C_2 +\frac{1}{R_2})^{-1} I_i
\end{array}
\right.
\Leftrightarrow 
\left \{
    \begin{array}{rcl}
    I_i &=& \frac{j\omega C_1}{1+R_1C_1j\omega} V_i\\
    V_o &=& - \frac{R_2C_1j\omega}{(1+R_2C_2j\omega)(1+R_1C_1j\omega)}V_i
\end{array}
\right.
\end{equation*}
\begin{equation*}
    \Rightarrow
\left \{
    \begin{array}{rcl}
    g_i &=& \frac{j\omega C_1}{1+R_1C_1j\omega} [\Omega^{-1}]\\
    g_f &=& -\frac{R_2C_1j\omega}{(1+R_2C_2j\omega)(1+R_1C_1j\omega)}
\end{array}
\right.
\end{equation*}
La matrice G est donc :
\begin{equation*}
    G = \begin{bmatrix} \frac{j\omega C_1}{1+R_1C_1j\omega} & 0 \\ - \frac{R_2C_1j\omega}{(1+R_2C_2j\omega)(1+R_1C_1j\omega)} & 0 \end{bmatrix}
\end{equation*}
\item La fonction de transfert du gain $g_f$ peut être réécrite sous la forme générale : 
\begin{equation*}
H(j\omega) = K \frac{j\frac{\omega}{\omega_0}}{1+2\xi_1 j\frac{\omega}{\omega_1}+ (j\frac{\omega}{\omega_1})^2}~~
\mbox{avec}~~
\left\{\begin{matrix*}[l]
K= -1\\
\omega_0 = \frac{1}{R_2C_1} = 10^5[\mbox{rad/s}]\\
\omega_1 = \frac{1}{\sqrt{R_1R_2C_1C_2}} =44721 [\mbox{rad/s}] \\
\xi_1 = \frac{\omega_1}{2}(R_1C_1+R_2C_2)=1.3416\\
\end{matrix*}\right.
\end{equation*}
Le diagramme de Bode de la fonction de transfert du gain $g_f$ est représentée ci-dessous :
\begin{center}
     \begin{tikzpicture}[
    gnuplot def/.append style={prefix={}},
]
% Grid Style
\tikzset{
    semilog lines/.style={black},
    semilog lines 2/.style={gray,dotted},
    semilog half lines/.style={gray, dotted},
    semilog label x/.style={below,font=\tiny},
    semilog label y/.style={above,font=\tiny} }
% Magnitude Plot
\begin{scope}[xscale=7/5, yscale=3/50]
    \UnitedB
    \semilog{0}{8}{-50}{30} 
    %Asymp
     \BodeGraph[green,samples=1000]{0:5.9}{\SOAmpAsymp{1}{1.3416}{44721}}
     \BodeGraph[blue,samples=1000]{3.5:6.5}{-\IntAmp{100000}}
    %Real
    \BodeGraph[red,samples=1000]{2.5:6.8}{\SOAmp{1}{1.3416}{44721} - \IntAmp{100000}}
\end{scope}
% Phase plot
\begin{scope}[yshift=-4cm,xscale=7/5,yscale=3/180]
    \UniteDegre
    \OrdBode{30}
    \semilog{0}{8}{-270}{0}
    %Asymp
     \BodeGraph[green,samples=1000]{0:8}{\SOArgAsymp{1}{1.3416}{44721}}
     \BodeGraph[blue,samples=1000]{0:8}{ \IntArg{-100000} }
    %Real
    \BodeGraph[red,samples=1000]{0:8}{\SOArg{1}{1.3416}{44721} + \IntArg{-100000}}
\end{scope}
\end{tikzpicture}
\end{center}
C'est un filtre passe bande dont la bande passante est donnée par :
\begin{equation*}
    BDW = 2\xi \omega_1 = 119852[rad/s]
\end{equation*}
    \item L'impédance d'entrée s'écrit :
    \begin{equation*}
        Z_{in} = \frac{1}{g_i}\frac{1}{1+g_r\cdot A_{if,o}} ~~ \mbox{avec}~~A_{if,o} = -\frac{g_f}{g_i}\frac{1}{z_L+g_o}
    \end{equation*}
     \begin{equation*}
       \Leftrightarrow Z_{in}= \frac{1}{g_i}\frac{1}{1-0\cdot \frac{g_f}{g_i}\frac{1}{z_L+0}}= \frac{1}{g_i}=\frac{1+R_1C_1j\omega}{C_1j\omega} [\Omega]
    \end{equation*}
    \begin{equation*}
        Z_{in} = 
        \left\{\begin{matrix*}[l]
        5000-10^5j[\Omega] ~~\mbox{si}~~\omega = 1kHz\\
         5000-10^4j[\Omega] ~~\mbox{si}~~\omega = 10kHz\\ 5000-10^3j[\Omega] ~~\mbox{si}~~\omega = 100kHz
\end{matrix*}\right.
    \end{equation*}
\end{enumerate}
\end{solution}

\section{Question Dehez : circuit magnétiques couplés}
On considère le circuit suivant 
\begin{center}
    \begin{circuitikz}
        \draw (0,0) to[R,l=\SI{12}{\ohm},*-] ++(-2,0) to[american voltage source,l=$12\angle 0^\circ$V] ++(0,2) to[R,l=$2\Omega$] ++(2,0) to[R,l=$4\Omega$] ++(0,-2);
        \draw (0,0) to[R,l=$4\Omega$,-*] ++(0,-2) to[short] ++(-2,0) to[american voltage source,l=$24\angle 0^\circ$V,-*] ++(0,2);
        \draw (0,0) to[C,l=$-2j\Omega$] ++(3,0) coordinate (b1) to[L,l_=$4j\Omega$] ++(0,-2) to[short] ++(-3,0);
        \draw (b1) node[below left]{$\bullet$};
        \draw ($ (b1) + (1,-2) $) coordinate (b4) to[L,l_=$3j\Omega$] ++(0,2) coordinate (b2) to[C,l=$-1j\Omega$,-*] ++(3,0) to[short, -o] ++(1,0) coordinate (v1) node[right]{$+$};
        \draw (b4) to[short,-*] ++(3,0) coordinate (c) to[R,l=$2\Omega$] ++(0,2);
        \draw (c) to[short,-o] ++(1,0) coordinate (v2) node[right]{$-$};
        \draw (b4) node[above right]{$\bullet$};
        
        \draw [<->,>=stealth] ($ (b1) + (-0.2,0.2) $)  to [bend left] node[pos=0.5,above] {$-1j\Omega$} ($ (b2) + (0.2,0.2) $);
        \draw [<-,>=stealth] ($ (v1) + (0,-0.2) $)  to [bend left] node[pos=0.5,right] {$V_o$} ($ (v2) + (0,0.2) $);
    \end{circuitikz}
\end{center}
\begin{enumerate}
    \item Calculer le facteur de dispersion entre les deux inductances couplées magnétiquement.
    \item Calculer l'amplitude et la phase de la tension $V_o$.
\end{enumerate}
On souhaite mesurer l’amplitude de cette tension au moyen d’un voltmètre idéal, mais dont le fond d’échelle, de $100mV$, n’est à priori pas adapté. L’usage d’un transformateur (supposé idéal) est alors envisagé pour adapter la tension mesurée par l’ampèremètre.
\begin{center}
    \begin{circuitikz}
        \draw (0,0) to[R,l=$12\Omega$,*-] ++(-2,0) to[american voltage source,l=$12\angle 0^\circ$V] ++(0,2) to[R,l=$2\Omega$] ++(2,0) to[R,l=$4\Omega$] ++(0,-2);
        \draw (0,0) to[R,l=$4\Omega$,-*] ++(0,-2) to[short] ++(-2,0) to[american voltage source,l=$24\angle 0^\circ$V,-*] ++(0,2);
        \draw (0,0) to[C,l=$-2j\Omega$] ++(3,0) coordinate (b1) to[L,l_=$4j\Omega$] ++(0,-2) to[short] ++(-3,0);
        \draw (b1) node[below left]{$\bullet$};
        \draw ($ (b1) + (1,-2) $) coordinate (b4) to[L,l_=$3j\Omega$] ++(0,2) coordinate (b2) to[C,l=$-1j\Omega$,-*] ++(3,0) to[short, -o] ++(1,0) coordinate (v1);
        \draw (b4) to[short,-*] ++(3,0) coordinate (c) to[R,l=$2\Omega$] ++(0,2);
        \draw (c) to[short,-o] ++(1,0) coordinate (v2);
        \draw (b4) node[above right]{$\bullet$};
        
        \draw [<->,>=stealth] ($ (b1) + (-0.2,0.2) $)  to [bend left] node[pos=0.5,above] {$-1j\Omega$} ($ (b2) + (0.2,0.2) $);
        \draw [<-,>=stealth] ($ (v1) + (0,-0.2) $)  to[] node[pos=0.5,right] {$V_o$} ($ (v2) + (0,0.2) $);
        
        \draw (v1) to[short,o-] ++(1,0) coordinate (t1) to[L] ++(0,-2) coordinate (t3) to[short,-o] ++(-1,0);
        \draw ($ (t1) + (1,-2) $)    coordinate (t4) to[L] ++(0, 2) coordinate (t2) to[short] ++(1,0) to[short] ++(0,-1) node[meter]{\small V} to[short] ++(0,-1) to[short] ++(-1,0);
        
        \draw (t2) node[below right]{$\bullet$};
        \draw (t1) node[below left]{$\bullet$};
        \draw[thick] ($ (t1) + (0.4,-0.2) $)  -- ($ (t3) + (0.4,0.2) $);
        \draw[thick] ($ (t2) + (-0.4,-0.2) $) -- ($ (t4) + (-0.4,0.2) $);
        
        \draw[white] ($ (t1) + (-0.2,0.2) $)  to[] node[black,pos=0.5,above] {$n:1$} ($ (t2) + (0.2,0.2) $);
    \end{circuitikz}
\end{center}
\begin{enumerate}[resume]
    \item Calculer la valeur idéale que devrait prendre le rapport de transformation $n$ de ce transformateur pour avoir une mesure correspondant au fond d’échelle du voltmètre.
\end{enumerate}
\begin{solution}
	\begin{enumerate}
	\item Le facteur de dispersion s’écrit comme 
     \begin{equation*}
        \sigma = 1 -\frac{M^2}{L_1L_2} = 1 - \frac{(-j)^2}{3j\cdot 4j} = \frac{11}{12} 
    \end{equation*}
    \item Pour la suite des calculs nous utiliserons les notations telles qu’illustrées :
    \begin{center}
            \begin{circuitikz}
                \draw (0,0) to[R,i<=$I_2$,l=$12\Omega$,*-] ++(-2,0) to[american voltage     source,l=$12\angle 0^\circ$V] ++(0,2) to[R,l_=$6\Omega$,i=$I_1$] ++(2,0) -- ++(0,-2);
                \draw (0,0) to[R,l=$4\Omega$,-*] ++(0,-2) to[short] ++(-2,0) to[american voltage source,l=$24\angle 0^\circ$V,-*] ++(0,2);
                \draw (0,0) to[C,l=$-2j\Omega$,i=$I_3$] ++(3,0) coordinate (b1) to[L,l_=$4j\Omega$] ++(0,-2) to[short] ++(-3,0);
                \draw (b1) node[below left]{$\bullet$};
                \draw ($ (b1) + (1,-2) $) coordinate (b4) to[L,l_=$3j\Omega$] ++(0,2) coordinate (b2) to[C,l=$-j\Omega$,i>_=$I_4$,-*] ++(3,0) to[short, -o] ++(1,0) coordinate (v1) node[right]{$+$};
                \draw (b4) to[short,-*] ++(3,0) coordinate (c) to[R,l=$2\Omega$] ++(0,2);
                \draw (c) to[short,-o] ++(1,0) coordinate (v2) node[right]{$-$};
                \draw (b4) node[above right]{$\bullet$};
                \draw [<->,>=stealth] ($ (b1) + (-0.2,0.2) $)  to [bend left] node[pos=0.5,above] {$-j\Omega$} ($ (b2) + (0.2,0.2) $);
                \draw [<-,>=stealth] ($ (v1) + (0,-0.2) $)  to [bend left] node[pos=0.5,right] {$V_o$} ($ (v2) + (0,0.2) $);
                \draw (-1,1) node[scale=2]{$\circlearrowright$};
                \draw (-1,-1) node[scale=2]{$\circlearrowright$};
                \draw (1.5,-1) node[scale=2]{$\circlearrowright$};
                \draw (5.5,-1) node[scale=2]{$\circlearrowright$};
            \end{circuitikz}
        \end{center}
        Nous pouvons effectuer quatre lois des mailles sur les quatre mailles indiquées:
        \begin{equation*}
        \begin{dcases}
         12 = 6I_1 - 12I_2 \\
         24 = 12I_2 + 4(I_1+I_2-I_3) \\
         2jI_3 -jI_4 = 4(I_1+I_2-I_3)\\
         -jI_3+I_4(2+2j) = 0\\
        \end{dcases}
        \Leftrightarrow
        \begin{dcases}
         I_1 =4.945\angle-13.43^\circ [A] \\
         I_2 = 1.518^\circ [A] \\
         I_3 = 5.61\angle-37.875^\circ [A]\\
         I_4 = 1.984\angle 7.125^\circ [A]\\
        \end{dcases}
        \end{equation*}
        Sur base de la relation élémentaire $V=ZI$ nous pouvons écrire :
        \begin{equation*}
            V_o = 2\Omega \cdot I_4 = 3.968\angle 7.125^\circ [V]
        \end{equation*}
    \item Le fond d'échelle correspond à la valeur maximale que peut 'lire' l'appareil de mesure. Il faut donc que la tension mesurée par le voltmètre ne dépasse pas 100mV. La formule qui lie les tensions et le nombre de spires est donnée par :
    \begin{equation*}
        \frac{V_{primaire}}{V_{secondaire}} = \frac{N_{primaire}}{N_{secondaire}}
        \Leftrightarrow \frac{V_o}{V_{voltmètre}}= \frac{n}{1} ~~\mbox{avec}~~V_{voltmètre}< 0.1[V]
    \end{equation*}
    \begin{equation*}
        \Leftrightarrow \frac{V_o}{n}< 0.1 = V_{voltmètre} \Leftrightarrow n >\frac{V_o}{0.1} = \frac{3.968}{0.1} = 39.68 
    \end{equation*}
    Il faut que le nombre de spires soit au moins supérieur à 40.
\end{enumerate}
\end{solution}

\section{Question Craeye : transitoire} 
Soit le circuit suivant dont l'interrupteur 2V s'ouvre et l'interrupteur 1V se ferme en $t=0$.
\begin{center}
    \begin{circuitikz} 
    	\draw
 		(0,0) -- (4,0)
 		(-1.5,0) -- (0,0)
 		(-1.5,0) to[american voltage source,l=1V] (-1.5,2.5) to [closing switch,l=$t{=}0$] (-1.5,5) -- (0,5)
 		(0,0) to[american voltage source,l=2V] (0,2.5) to [opening switch,l=$t{=}0$] (0,5) to [R,l=$R$,*-] (1.8,5)
 		(1.8,5) to [L,l=$L$] (4,5)
		(5,3.6) to [open, v^=$V_o$] (5,1)
		(4,5) to [C,l=$C$] (4,0);
	\end{circuitikz}
\end{center}
On demande la tension $V_o$ en $t>0$ avec les données numériques suivantes :
$R=2.5k\Omega$, $L=0.5mH$, $C = 0.5nF$

\begin{solution}
En $t<0$, on peut réécrire le circuit comme :
\begin{center}
    \begin{circuitikz} 
    	\draw
 		(0,0) to[american voltage source,l=2V,i=$I_L(0^-)$]   (0,5) to [R,l=$R$] (1.8,5)
 		(1.8,5) to [L,l=$L$] (4,5)
		(5,3.6) to [open, v^=$V_C(0^-)$] (5,1)
		(0,0) --(4,0) --(4,2) 
		(4,3)--(4,5);
	\end{circuitikz}
\end{center}
Le condensateur est chargé et nous obtenons : 
\begin{equation*}
    \left \{
\begin{array}{rcl}
V_{C}(0^-) &=& 2[V]\\
I_L(0^-) &=& 0 [A]
\end{array}
\right.
\end{equation*}
En $t>0$, on peut réécrire le circuit comme :
\begin{center}
    \begin{circuitikz} 
    	\draw
 		(0,0) -- (4,0)
 		(-1.5,0) -- (0,0)
 		(-1.5,0) to[american voltage source,l=1V,i=$I(s)$]  (-1.5,5)
 		 to [R,l=$R$,v=$V_R(s)$] (1.8,5)
 		(1.8,5) to [L,l=$L$,v=$V_L(s)$] (4,5)
		(5,3.6) to [open, v^=$V_o(s)$] (5,1)
		(4,5) to [C,l=$C$] (4,0);
	\end{circuitikz}
\end{center}
Sur base de la loi des mailles sur la maille globale nous pouvons écrire :
\begin{align*}
   & \Rightarrow &\frac{1}{s} &= V_R(s)+V_L(s)+V_o(s)\\
   & \Leftrightarrow &\frac{1}{s} &= RI(s) +V_L(s)+V_o(s) ~~\mbox{avec}~~ V_L(s) = sLI(s)-LI_L(0^-)\\
    & \Leftrightarrow &\frac{1}{s} &= (R+sL)I(s)-LI_L(0^-) +V_o(s) ~~\mbox{avec}~~I(s) = I_C(s) = sCV_o(s)-CV_C(0^-)\\
    & \Leftrightarrow & \frac{1}{s} &= (R+sL)(sCV_o(s)-CV_C(0^-))+V_o(s)-LI_L(0^-) \\
    &   \Leftrightarrow &\frac{1}{s} &= RsCV_o(s) +s^2CLV_o(s)-RCV_C(0^-)-sCLV_C(0^-)+V_o(s)-LI_L(0^-)\\
    & \Leftrightarrow & 1 &= sV_o(s) (sRC+s^2LC +1)-sRCV_C(0^-)-s^2CLV_C(0^-)-sLI_L(0^-)
\end{align*}
\begin{equation*}
   \Rightarrow  V_o(s)= \frac{1+s(RCV_C(0^-)-LI_L(0^-))+s^2LCV_C(0^-)}{s(1+sRC+s^2LC)}
\end{equation*}
En substituant les valeurs nous obtenons :
\begin{equation*}
    V_o(s) = \frac{1+2.5\cdot 10^{-6}s+5\cdot 10^{-13}s^2}{s(1+1.25\cdot 10^{-6}s +2.5\cdot 10^{-13}s^2)} = 2\frac{s^2+5\cdot10^6s+2\cdot 10^{12}}{s(s^2+5\cdot 10^6s+4\cdot 10^{12})}
\end{equation*}
En effectuant la décomposition en fraction simple nous avons :
\begin{equation*}
    V_C(s) = \frac{A}{s} + \frac{B}{s+10^6}+\frac{C}{s+4\cdot 10^{6}} \Leftrightarrow 
\begin{dcases}
A+B+C =2\\
 5\cdot 10^6 A+ 4\cdot 10^6B+10^6C = 10^7\\
 4\cdot 10^{12}A = 4\cdot 10^{12}\\
\end{dcases}
\Leftrightarrow
\begin{dcases}
 A= 1\\
 B= \frac{4}{3}\\
 C=-\frac{1}{3}\\
\end{dcases}
\end{equation*}
Nous obtenons donc dans le domaine de Laplace :
\begin{equation*}
    V_o(s) = \frac{1}{s} + \frac{4}{3} \frac{1}{s+10^6}-\frac{1}{3} \frac{1}{s+4\cdot 10^{6}}
\end{equation*}
Cela correspond dans le domaine temporel à : 
\begin{equation*}
    V_o(t) =  \left[1+\frac{4}{3} e^{-10^6 t}-\frac{1}{3} e^{-4\cdot10^6 t}  \right]u(t)
\end{equation*}
Le graphe ressemble à
\begin{center}
    \begin{tikzpicture}
        \begin{axis}[enlargelimits=true,grid=major,ylabel=$V_o(t)$,xlabel=$t(s)$]
            \addplot [blue,domain=0:0.000012,samples=200]{1-(1/3)*e^(-4*(10^6)*x)+(4/3)*e^(-10^6*x)};
        \end{axis}
    \end{tikzpicture}
\end{center}
\end{solution}
\end{document}