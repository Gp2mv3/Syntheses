\documentclass[fr]{../../../../../../eplexam}
\usepackage{../../../../../../eplunits}
\usepackage[oldvoltagedirection]{circuitikz}
\usepackage{bodegraph}
\usepackage{pgfplots}
\usepackage{amsmath}
\usepackage{enumitem}
\pgfplotsset{compat=newest}
\tikzset{meter/.style={draw,thick,circle,fill=white,minimum size =0.75cm,inner sep=0pt}}

\hypertitle{circmes-ELEC1370}{4}{ELEC}{1370}{2014}{Juin}{Majeure}
{Brieuc Balon}
{Claude Oestges, Bruno Dehez and Christophe Craeye}

\section{Question Oestges : phaseurs}
On considère le circuit suivant
\begin{center}
    \begin{circuitikz} 
    	\draw
		to[american voltage source,l=$12\angle 0^\circ $] (0,2.5) 
  		(2.5,2.5) to[R,l=$2\Omega$] (0,2.5)
  		(2.5,2.5) to [C,l=-j$1\Omega$] (5,2.5)
  		(2.5,0) to[european resistor,l=$Z$] (2.5,2.5)
 		(5,2.5) to[R,v=$V_o$,l=$1\Omega$] (5,0)
 		(0,0) -- (5,0);
 	\end{circuitikz}
\end{center}
On mesure une tension $V_o=3\angle 20^\circ$V. On demande de calculer
\begin{enumerate}
    \item La valeur de l'impédance $Z$
    \item La valeur de l'inductance ou capacité mise en série avec une résistance de l'impédance $Z$
    \item La puissance active et la puissance réactive fournie par la source
    \item La valeur de $Z = R + Xj$ (avec $R>0$) pour que la puissance réactive fournie par la source soit nulle
\end{enumerate}

\begin{solution}
Pour la résolution du circuit nous utiliserons les notations : 
\begin{center}
    \begin{circuitikz} 
    	\draw
		to[american voltage source,l=$12\angle 0^\circ $] (0,2.5) 
  		 (0,2.5)to[R,l=$2\Omega$,i=$I_R$,v=$V_R$] (2.5,2.5)
  		(2.5,2.5) to [C,l=-j$1\Omega$,v=$V_C$] (5,2.5)
  		(2.5,2.5)to[european resistor,l=$Z$,i=$I_Z$,v=$V_Z$,*-] (2.5,0)
 		(5,2.5) to[R,v=$V_o$,l=$1\Omega$,i=$I_o$] (5,0)
 		(2.5,2.5) node[label={[font=\footnotesize]above:A}]{}
 		(0,0) -- (5,0);
 	\end{circuitikz}
\end{center}
Sur base de la relation élémentaire $V=ZI$ nous pouvons écrire :
\begin{equation*}
    I_o = \frac{V_o}{1} = 3\angle 70^\circ [A]
\end{equation*}
\begin{equation*}
    V_C = -j *I_o = 3\angle 70^\circ [V]
\end{equation*}
Nous pouvons effectuer une loi des mailles sur la maille de droite:
\begin{equation*}
    V_Z = V_o + V_C = 3\sqrt{2}\angle -25^\circ [V]
\end{equation*}
Nous pouvons effectuer une loi des mailles sur grande maille:
\begin{equation*}
    V_R = 12 - V_o - V_C = 12 - V_Z = 8.349\angle 12.4^\circ [V]
\end{equation*}
Sur base de la relation élémentaire $V=ZI$ nous pouvons écrire :
\begin{equation*}
    I_R = \frac{V_R}{2} = 4.175\angle 12.4^\circ [A]
\end{equation*}
Nous pouvons effectuer une loi des noeuds sur le noeud A :
\begin{equation*}
    I_Z = I_R - I_o = 1.265\angle -5.878 ^\circ [A]
\end{equation*}
\begin{enumerate}
    \item Sur base de la relation élémentaire $V=ZI$ nous pouvons écrire :
    \begin{equation*}
    Z = \frac{V_Z}{I_Z} = 3.354\angle -19.122^\circ = 3.169-1.096j [\Omega]
    \end{equation*}
   \item Nous avons que :
   \begin{equation*}
       R= 3.17 [\Omega] et C = \frac{1}{1.096\omega}[F]
   \end{equation*}
   \item La puissance\footnote{Nous travaillons avec des phaseurs, ils représentent la valeur efficace de la tension ou du courant. S’il n’est pas spécifié dans l’énoncé si la valeur de la tension $V_o$ représente la valeur de crête ou la valeur efficace de la tension, n’hésitez pas à le spécifier dans votre exercice. Nous considérons que $V_o$ correspond à la valeur efficace dans la suite des calculs.} complexe de la source est : 
   \begin{equation*}
       S = 12\angle 0^\circ * I_R^\ast = 12\angle 0^\circ * 4.175\angle -12.4^\circ = 50.1\angle -12.4^\circ =48.93 -10.76j [VA]
   \end{equation*}
   Donc P = 48.93 [W] et Q = 10.76 [VAR]
   
   \item Il existe plusieurs méthodes de résolution pour cette question. Celle qui suit est la plus générale. Pour annuler la puissance réactive, il faut que l'impédance équivalent vue par la source soit purement réelle. On remarque assez aisement que : 
   \begin{equation*}
       Z_{eq} = ((1-j)||Z)+2 \Rightarrow \Im( Z_{eq})= 0 
   \end{equation*}
   \begin{equation*}
       \Im( Z_{eq})= \Im((( \frac{1}{1-j}+\frac{R}{+Xj})^{-1})+2) = \Im(( \frac{1}{1-j}+\frac{R}{+Xj})^{-1}) = \Im((\frac{1-j+R+Xj}{(1-j)(R+Xj)})^{-1}) =0
   \end{equation*}
   \begin{equation*}
       \Leftrightarrow \Im(\frac{(1-j)(R+Xj)}{1+R+j(X-1)})=\Im(\frac{(1-j)(R+Xj)}{1+R+j(X-1)} \frac{(1+R)-j(X-1)}{(1+R)-j(X-1)} )= \Im((1-j)(R+Xj)(1+R-j(X-1)))=0
   \end{equation*}
   \begin{equation*}
        \Leftrightarrow \Im(R+Xj-Rj+X+R^2+RXj-R^2j+RX+Rj-X+R+jX-RXj+X^2-RX-X^2)=0
   \end{equation*}
   \begin{equation*}
       -R^2+2X-X^2=0 ~~\mbox{avec}~~[(R\ne -1,X\ne 1),R>0]
   \end{equation*}
   Les solution de cette équations sont :
   \begin{equation*}
             [(R=\pm 1,X=1),(R=0,X=0),(R=0,X=2)]
   \end{equation*}
   Sur base des conditions, la seule solution pour que la puissance réactive fournie par la source soit égale à 0 est $R=X=1$. Il faut que Z = 1+j.
\end{enumerate}
\end{solution}

\section{Question Oestges : Bode et quadripôles}
On considère le circuit suivant
\begin{center}
	\begin{circuitikz}
		\draw
		(0,1) node [above]{$V_\text{in}$} to [short,o-] (1,1) 
		 to [C,l=$C_1$,*-] (4,1)
		(1,1) -- (1,2.5) to [R,l=$R_2$] (4,2.5) -- (4,1) to [C,l=$C_2$] (4,-3)
		(8,0.5) node[op amp, yscale=-1] (opamp) {}
		(opamp.-) -- (5.5,0) -- (5.5,-1.2) to [R,l=$R_4$,*-] (10,-1.2) -- (10,0.5)
		(5.5,-1.2) to [R,l=$R_3$] (5.5,-3)
		(opamp.+) to [short,-*] (4,1)
		(opamp.out) -- (10,0.5) to [short,*-o] (11,0.5) node [above]{$V_\text{out}$}
		(0,-3) -- (11,-3)(4.5,-3) --(4.5,-3.4)node [ground] {};
	\end{circuitikz}
\end{center}
On demande de calculer
\begin{enumerate}
    \item La matrice $G$ associée au quadripôle de ce circuit
    \item Tracer le diagramme de Bode (amplitude et phase) du coefficient $g_f$ et en déduire la fonction de ce circuit.
    \item Déterminer $Z_\text{out}$ si $Z_\text{in} = 100\Omega$.
\end{enumerate}
PS: il manque les données pour tracer le diagramme de Bode correctement
\begin{solution}
	\begin{enumerate}
\item La matrice $G$ est calculée grâce à : 
    \begin{equation*}
        \begin{bmatrix}
        I_i \\
        V_o
        \end{bmatrix} = 
     \begin{bmatrix}
        g_i & g_r \\
        g_f & g_o
     \end{bmatrix}
     \begin{bmatrix}
         V_i \\
         I_o
     \end{bmatrix}
    \end{equation*} 
En annulant la tension $V_i$ le circuit devient :
\begin{center}
	\begin{circuitikz}
		\draw
		(0,-3)to[short,i=$I_i$]
		(0,1) -- (1,1) 
		 to [C,l=$C_1$,*-] (4,1)
		(1,1) -- (1,2.5) to [R,l=$R_2$] (4,2.5) -- (4,1) to [C,l=$C_2$] (4,-3)
		(8,0.5) node[op amp, yscale=-1] (opamp) {}
		(opamp.-) -- (5.5,0) -- (5.5,-1.2) to [R,l=$R_4$,*-] (10,-1.2) -- (10,0.5)
		(5.5,-1.2) to [R,l=$R_3$] (5.5,-3)
		(opamp.+) to [short,-*] (4,1)
		(opamp.out) -- (10,0.5) to [short,i<=$I_o$,*-o] (11,0.5)
		to [open,v^=$V_o$,-o] (11,-3)--
		(0,-3) (4.5,-3) --(4.5,-3.4)node [ground] {};
	\end{circuitikz}
\end{center}
Comme nous avons un amplificateur opérationnel que l'on considère comme idéal, son impédance d'entrée vaut $0\Omega$. Le courant $I_0$ passe donc entièrement dans l'amplificateur opérationnel et aucun courant ne passe dans le reste du circuit. Dès lors tout le circuit est à 0V. Nous obtenons donc :
\begin{equation*}
\left \{
    \begin{array}{rcl}
    I_i &=& 0\cdot I_o \\
    V_o &=& 0 \cdot I_o
\end{array}
\right.
\Leftrightarrow 
\left \{
    \begin{array}{rcl}
    g_r &=& 0 \\
    g_o &=&  0 [\Omega]
\end{array}
\right.
\end{equation*}
En annulant le courant $I_o$ le circuit devient :
\begin{center}
	\begin{circuitikz}
		\draw
		(0,-3)to[open,v^>=$V_i$,o-o]
		(0,1) -- (1,1) 
		 to [C,l=$C_1$,*-] (4,1)
		(1,1) -- (1,2.5) to [R,l=$R_2$] (4,2.5) -- (4,1) to [C,l=$C_2$] (4,-3)
		(8,0.5) node[op amp, yscale=-1] (opamp) {}
		(opamp.-) -- (5.5,0) -- (5.5,-1.2) to [R,l=$R_4$,*-] (10,-1.2) -- (10,0.5)
		(5.5,-1.2) to [R,l=$R_3$] (5.5,-3)
		(opamp.+) to [short,-*] (4,1)
		(opamp.out) -- (10,0.5) to [short,*-o] (11,0.5)
		to [open,v^=$V_o$,-o] (11,-3)--
		(0,-3) (4.5,-3) --(4.5,-3.4)node [ground] {};
	\end{circuitikz}
\end{center}
Étant donné que l'amplificateur opérationnel est en rétroaction négative ($V^+=V^-$) et qu'il est considéré comme parfait ($i^-=i^+=0A$) nous obtenons: 
\begin{equation*}
\left \{
    \begin{array}{rcl}
    V_i &=& (\frac{R_2}{1+R_2C_1j\omega}+\frac{1}{j\omega C_2})I_i \\
    V_o &=& \frac{R_4+R_3}{R_3} V^- ~~\mbox{avec}~~V^-=V^+ = \frac{C_2}{(C_1||R_2)+C_2} V_i
\end{array}
\right.
\Leftrightarrow 
\left \{
    \begin{array}{rcl}
    I_i &=& j\omega C_2 \cdot \frac{1+R_2C_1j\omega}{1+R_2(C_1+C_2)j\omega} V_i\\
    V_o &=& \frac{R_4+R_3}{R_3} \frac{1+R_2C_1j\omega}{1+R_2(C_1+C_2)j\omega} V_i
\end{array}
\right.
\end{equation*}
\begin{equation*}
    \Rightarrow
\left \{
    \begin{array}{rcl}
    g_i &=& j\omega C_2 \cdot \frac{1+R_2C_1j\omega}{1+R_2(C_1+C_2)j\omega} [\Omega^{-1}]\\
    g_f &=& \frac{R_4+R_3}{R_3} \frac{1+R_2C_1j\omega}{1+R_2(C_1+C_2)j\omega} 
\end{array}
\right.
\end{equation*}
La matrice G est donc :
\begin{equation*}
    G = \begin{bmatrix}j\omega C_2 \cdot \frac{1+R_2C_1j\omega}{1+R_2(C_1+C_2)j\omega} & 0 \\ \frac{R_4+R_3}{R_3} \frac{1+R_2C_1j\omega}{1+R_2(C_1+C_2)j\omega}  & 0 \end{bmatrix}
\end{equation*}
\item La fonction de transfert du gain $g_f$ peut être réécrite sous la forme générale : 
\begin{equation*}
H(j\omega) = K \frac{j\frac{\omega}{\omega_0}}{1+j\frac{\omega}{\omega_1}}~~
\mbox{avec}~~
\left\{\begin{matrix*}[l]
K= \frac{R_3+R_4}{R_4}\\
\omega_0 = \frac{1}{R_2C_1}\\
\omega_1 = \frac{1}{R_2(C_1+C_2)} \\
\end{matrix*}\right.
\end{equation*}
Nous allons effectuer le diagramme de Bode pour les valeurs de composants $C_1=1nF$,$C_2=100nF$,$R_2=R_4=10k\Omega$,$R_3=1k\Omega$. Cependant on peut remarque assez simplement pour n'importe quelle valeur de composant que comme $\omega_0>\omega_1$ on aura donc : 
\begin{itemize}
    \item Au niveau du gain : un gain égale à $20\log(K)$ avant $\omega_1$, gain qui décroit de $-20dB/dec$ entre $\omega_1$ et $\omega_0$ et un gain qui croit de $0dB/dec$ au-delà de $\omega_0$.
    \item Au niveau du déphasage : un déphasage de $0^\circ$ avant $\omega_1$, un déphasage asymptotique\footnotetext{Qui sera atteint ou non selon les valeurs des capacités.} entre $\omega_1$ et $\omega_0$ et un déphasage de $0^\circ$ au-delà de $\omega_0$.
\end{itemize}
Le diagramme de Bode de la fonction de transfert du gain $g_f$ avec les paramètre définie précédemment est représentée ci-dessous :
\begin{center}
     \begin{tikzpicture}[
    gnuplot def/.append style={prefix={}},
]
% Grid Style
\tikzset{
    semilog lines/.style={black},
    semilog lines 2/.style={gray,dotted},
    semilog half lines/.style={gray, dotted},
    semilog label x/.style={below,font=\tiny},
    semilog label y/.style={above,font=\tiny} }
% Magnitude Plot
\begin{scope}[xscale=7/5, yscale=3/50]
    \UnitedB
    \semilog{0}{8}{-30}{40} 
    %Asymp
     \BodeGraph[green,samples=1000]{0:4.5}{\POAmpAsymp{1}{0.00101}}
     \BodeGraph[blue,samples=1000]{0:6}{-\POAmpAsymp{0.09091}{0.00001}}
    %Real
    \BodeGraph[red,samples=1000]{0:8}{\POAmp{1}{0.00101}-\POAmp{0.09091}{0.00001}}
\end{scope}
% Phase plot
\begin{scope}[yshift=-4cm,xscale=7/5,yscale=3/180]
    \UniteDegre
    \OrdBode{30}
    \semilog{0}{8}{-90}{90}
    %Asymp
     \BodeGraph[green,samples=1000]{0:8}{\POArgAsymp{1}{0.00101}}
     \BodeGraph[blue,samples=1000]{0:8}{-\POArgAsymp{0.09091}{0.00001}}
    %Real
    \BodeGraph[red,samples=1000]{0:8}{\POArg{1}{0.00101}-\POArg{0.09091}{0.00001}}
\end{scope}
\end{tikzpicture}
\end{center}
 C'est un filtre passe-bas. Le bleu correspond au tracé asymptotique du numérateur, le vert correspond au tracé asymptotique du dénominateur et le rouge correspond au tracé de la fonction de transfert.
\item Comme nous avons un amplificateur opérationnel que l’on considère comme idéal, nous pouvons supposer que $Z_{out}=0\Omega$. Si cela ne vous apparaît pas vous pouvez toujours calculer l'impédance de sortie au moyen de la table des caractéristiques externes: 
\begin{equation*}
    Z_{out} = g_o-g_r\frac{g_fz_G}{1+g_iz_G}~~ \mbox{avec}~~ z_G = 100[\Omega]
\end{equation*}
\begin{equation*}
    \Leftrightarrow  Z_{out} = 0-0\cdot \frac{g_fz_G}{1+g_iz_G} = 0[\Omega]
\end{equation*}
\end{enumerate}
\end{solution}

\section{Question Dehez : circuit magnétique couplé}
\begin{center}
    \begin{circuitikz}
		\draw[american currents]
		(0,0) to [american voltage source,l=$V_s$] (0,3)
  		(0,0) -- (3,0)
  		(3,3) to [R,v^=$V_x$,l_=$5K\Omega$] (3,0)
  		(3,3) to [R, l_=$1k\Omega$] (0,3)
  		(3,0) -- (6,0);
		\draw[american currents]
  		(6,3) to [cI_=$0.04V_x$] (6,0)
  		(6,0) to [short, -o] (9,0)
  		(8,0) to [R,l=$10k\Omega$] (8,3)
  		(9,0) to [open,v_>=$V_o$] (9,3)
  		(9,3) to [short, o-] (8,3)
  		to [short, *-] (6,3);
	\end{circuitikz}
\end{center}
On demande de
\begin{enumerate}
    \item Représenter le dipôle équivalent de Thévenin de ce circuit
    \item Supposons qu'on connecte ce circuit à un transformateur idéal avec rapport $a:1$ et une résistance de $16\Omega$. Quelle valeur de $a$ maximise la puissance fournie à cette résistance?
\end{enumerate}

\begin{solution}
\begin{enumerate}
    \item Nous pouvons calculer $V_x$ sur base du diviseur de tension. 
    \begin{equation*}
        V_x = \frac{5000}{1000+5000} V_s = \frac{5}{6} V_s [V]
    \end{equation*}
    Sur base de la relation élémentaire $V=ZI$ nous pouvons écrire : 
    \begin{equation*}
        V_0 =- 10000*0.04V_x= -400 V_x = -\frac{1000}{3}V_s [V]
    \end{equation*}
    On remarque assez simplement que la résistance équivalente correspond à $10k\Omega$. Il est aussi possible d'arriver à cette conclusion en calculer le courant $I_N$ qui passe dans le résistance de $10k\Omega$ :
    \begin{equation*}
        I_N = 0.04V_x = 0.04*\frac{5}{6}V_s = \frac{1}{30}V_s \Rightarrow R_{eq}= - \frac{V_{Th}}{I_N}= 10[k\Omega]
    \end{equation*}
    Avec le signe négatif venant du fait que le courant $I_N$ est dans le même sens que $V_{Th}$. Le circuit équivalent peut être représenté comme ceci : 
    \begin{center}
        \begin{circuitikz}
        \draw
        (0,0) to [american voltage source,l=$V_{Th}:\frac{1000}{3}V_s$] (0,3) to [R,l=$R_{eq}:10k\Omega$](3,3)
        (0,0)--(3,0);
        \end{circuitikz}
    \end{center}
    \item Nous pouvons représenté le circuit comme ceci :
    
\begin{center}
    \begin{circuitikz}
    \draw
        (0,0) to [american voltage source,l=$V_{Th}:\frac{1000}{3}V_s$] (0,2.05) to [R,l=$R_{eq}:10k\Omega$](3,2.05)
        (0,-0.05)--(3,-0.05)
        (4,1) node [transformer](T){}
        (T.base) node{a:1}
        (5.05,2.05) --(6,2.05) to[R,l=$16\Omega$](6,-0.05)--(5.05,-0.05);
    \end{circuitikz}
\end{center}
En faisant passer la résistance du secondaire vers le primaire on obtient :
\begin{center}
        \begin{circuitikz}
        \draw
        (0,0) to [american voltage source,l=$V_{Th}:\frac{1000}{3}V_s$] (0,2) to [R,l=$R_{eq}:10k\Omega$](3,2) to [R, l=$16a^2 \Omega$](3,0)--(0,0);
        \end{circuitikz}
    \end{center}
Pour que la puissance soit maximale au niveau de la seconde résistance il faut que $R_{eq}= 16a^2$ et donc que $a = 25$ tours. Prouvons l'égalité sur base d'un cas général: 
\begin{center}
        \begin{circuitikz}
        \draw
        (0,0) to [american voltage source,l=$V$] (0,2) to [european resistor,l=$Z_1$](3,2) to [european resistor, l=$Z_2$](3,0)--(0,0);
        \end{circuitikz}
    \end{center}
    La puissance de $Z_2$ est donné par :
\begin{equation*}
    P_{Z_2}(Z_2) = Z_2 I^2 ~~\mbox{avec}~~I=\frac{V}{Z_1+Z_2}
\end{equation*}
\begin{equation*}
    \Leftrightarrow P_{Z_2}(Z_2) = V^2\frac{Z_2}{(Z_1+Z_2)^2}
\end{equation*}
Pour maximiser sa puissance nous allons voir pour quelles valeurs sa dérivée par rapport à $Z_2$ s'annule :
\begin{equation*}
    \frac{\partial P_{Z_2}}{\partial Z_2}(Z_2) = V^2\frac{(Z_1+Z_2)^2-2(Z_1+Z_2)Z_2}{(Z_1+Z_2)^2}= V^2\frac{Z_1^2-Z_2^2}{(Z_1+Z_2)^2} =0 
\end{equation*}
\begin{equation*}
    \Leftrightarrow Z_1^2=Z_2^2 \Rightarrow Z_1 = Z_2
\end{equation*}
On peut voir que cette condition amène un maximum en regardant le signe de la dérivée seconde : 
\begin{equation*}
    \frac{\partial^2 P_{Z_2}}{\partial Z_2}(Z_2)= 2V^2\frac{Z_2-2Z_1}{(Z_1+Z_2)^4}
\end{equation*}
Dont le signe est négatif lorsque $Z_2=Z_1$.
\end{enumerate}
\end{solution}

\section{Question Craeye : transitoire} 
Soit le circuit suivant dont l'interrupteur 2V s'ouvre et l'interrupteur 1V se ferme en $t=0$.
\begin{center}
    \begin{circuitikz} 
    	\draw
 		(0,0) -- (5,0)
 		(3.6,5)--(5,5) 
 		(-1.5,0) -- (0,0)
 		(-1.5,0) to[american voltage source,l=1V] (-1.5,2.5) to [closing switch,l=$t{=}0$] (-1.5,5) -- (0,5)
 		(0,0) to[american voltage source,l=2V] (0,2.5) to [opening switch,l=$t{=}0$] (0,5) to [R,l=$R_1$,*-] (1.8,5)
		(1.8,5) to [L,l=$L$] (3.6,5)
 		(5,5) to [R,l=$R_2$] (5,0)
		(6,3.6) to [open, v^=$V_o$] (6,1)
 		(3.6,5) to [C,l=$C$] (3.6,0);
 	\end{circuitikz}
\end{center}
On demande la tension $V_o$ en $t>0$ avec les données numériques suivantes : $R_1=R_2=1\Omega$, $L=1MH$, $C = 1mF$

\begin{solution}
En $t<0$, on peut réécrire le circuit comme :
\begin{center}
    \begin{circuitikz}
    \draw
    (0,0) to[american voltage source,l=2V] (0,2.5) -- (0,5) to [R,l=$R_1$,i=$I_L(0^-)$] (6,5)
 	 to [R,l=$R_2$] (6,0)
 		(4,5) to [european voltages,open,v^=$V_{C}(0^-)$] (4,0) (0,0)--(6,0)
 		(6.5,3.6) to [open, v^=$V_o$] (6.5,1);
    \end{circuitikz}
\end{center}
Sur base du diviseur de tension et la loi des mailles sur l’unique maille nous obtenons :
\begin{equation*}
    \left \{
\begin{array}{rcl}
V_{C}(0^-) &=& \frac{R_2}{R_1+R_2} *2[V] = 1[V]\\
I_L(0^-) &=& \frac{2}{R_1+R_2} = 1 [A]
\end{array}
\right.
\end{equation*}
En $t>0$, on peut réécrire le circuit comme :
\begin{center}
    \begin{circuitikz} 
    	\draw
 		(0,0) -- (5,0)
 		(3.6,5)--(5,5) 
 		(-1.5,0) -- (0,0)
 		(-1.5,0) to[american voltage source,l=1V]  (-1.5,5) -- (0,5)
 	  (0,5) to [R,l=$R_1$,v=$V_{R_1}(s)$,i=$I_1(s)$] (1.8,5)
		(1.8,5) to [L,l=$L$,v=$V_L(s)$] (3.6,5)
 		(5,5) to [R,l=$R_2$,i=$I_2(s)$] (5,0)
		(5.5,3.6) to [open, v^=$V_o(s)$] (5.5,1)
 		(3.6,5) to [C,l=$C$,v=$V_C(s)$] (3.6,0);
 	\end{circuitikz}
\end{center}
Sur base de la loi des mailles sur la maille de droite nous pouvons écrire : 
\begin{align*}
   & \Rightarrow &V_o(s) &=V_C(s)   ~~\mbox{avec}~~ V_C(s) = \frac{I_1(s)-I_2(s)}{sC}+\frac{V_C(0^-)}{s}\\
   & \Leftrightarrow &V_o(s) &= \frac{I_1(s)-I_2(s)}{sC}+\frac{V_C(0^-)}{s}  \\
    & \Leftrightarrow &V_o(s) &= \frac{I_1(s)}{sC}-\textcolor{red}{\frac{R_2}{R_2}}\text{\footnotemark}\frac{I_2(s)}{sC} + \frac{V_C(0^-)}{s}\\
    & \Leftrightarrow & \frac{I_1(s)}{sC} &= V_o(s)(1+\frac{1}{R_2Cs})-\frac{V_o(0^-)}{s}
\end{align*}
\footnotetext{$^1$Petit trick pour arriver à exprimer $I_1(s)$ en fonction de $V_o(s)$.}
\begin{equation*}
    \Rightarrow I_1(s) = sCV_o(s)(1+\frac{1}{R_2sC})-V_C(0^-)C
\end{equation*}
Sur base de la loi des mailles sur la maille totale nous pouvons écrire : 
\begin{align*}
   & \Rightarrow &\frac{1}{s} &= I_1(s)(R_1+Ls)-I_L(0^-)+ V_o(s)\\
   & \Leftrightarrow &\frac{1}{s} &= sCV_o(s)(1+\frac{1}{R_2Cs})(R_1+Ls)-V_C(0^-)C(R_1+Ls)-LI_L(0^-)+V_o(s)\\ 
    & \Leftrightarrow &1 &= sV_o(s)(1+sCR_1+s^2LC+\frac{R_1}{R_2}+\frac{sL}{R_2})-sV_C(0^-)CR_1-LI_L(0^-)s-s^2LCV_C(0^-)
\end{align*}
\begin{equation*}
    \Rightarrow V_o(s) = \frac{1+s(V_C(0^-)CR_1+LI_L(0^-))+s^2LCV_C(0^-)}{s(1+\frac{R_1}{R_2}+s(R_1C+\frac{L}{R_2})+s^2LC}
\end{equation*}
En substituant les valeurs nous obtenons :
\begin{equation*}
    V_o(s) = \frac{1+2*10^{-3}+10^{-6}}{s(2+2*10^{-3}+10^{-6}s^2} = \frac{s^2+s*2*10^{3}+10^6}{s(2*10^6+2*10^3+s^2)}
\end{equation*}
En effectuant la décomposition en fraction simple nous avons :
\begin{equation*}
    V_o(s) = \frac{A}{s} + \frac{B*10^3}{(s+10^3)^2+10^{6}}+\frac{C*(10^3+s)}{(s+10^3)^2+10^{6})} \Leftrightarrow 
\begin{dcases}
 A+C =1\\
 2*10^3 A+10^3B+10^3C = 2*10^3\\
 2*10^{6}A = 10^{6}\\
\end{dcases}
\Leftrightarrow
\begin{dcases}
 A= \frac{1}{2}\\
 B=\frac{1}{2}\\
 C=\frac{1}{2}\\
\end{dcases}
\end{equation*}
Nous obtenons donc dans le domaine de Laplace :
\begin{equation*}
    V_o(s) = \frac{1}{2}(\frac{1}{s} +\frac{10^3}{(s+10^3)^2+10^{6}}+\frac{(10^3+s)}{(s+10^3)^2+10^{6}}
\end{equation*}
Cela correspond dans le domaine temporel à : 
\begin{equation*}
    V_o(t) = \frac{1}{2} u(t)(1 +e^{-10^3t}(\cos(10^3t)-\sin(10^3t))
\end{equation*}
La solution équivalente avec seulement un cosinus est :
\begin{equation*}
    V_o(t) = \left[\frac{1}{2} + 0.708 e^{-10^3t} \cos(10^3 t + 0.785) \right]u(t)
\end{equation*}
Le graphe ressemble à :
\begin{center}
    \begin{tikzpicture}
        \begin{axis}[enlargelimits=true,grid=major,ylabel=$V_o(t)$,xlabel=$t$]
            \addplot [blue,domain=0:0.006,samples=200]{(1/2)+(1/2)*e^(-10^3*x)*cos(10^3*x)+(1/2)*e^(-10^3*x)*sin(10^3*x)};
        \end{axis}
    \end{tikzpicture}
\end{center}
\end{solution}
\end{document}