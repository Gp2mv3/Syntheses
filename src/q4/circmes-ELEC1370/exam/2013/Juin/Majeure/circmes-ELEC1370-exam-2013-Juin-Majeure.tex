\documentclass[fr]{../../../../../../eplexam}
\usepackage{../../../../../../eplunits}
\usepackage[oldvoltagedirection]{circuitikz}
\usepackage{bodegraph}
\usepackage{pgfplots}
\usepackage{SIunits}
\usepackage{amsmath}
\usepackage{enumitem}
\pgfplotsset{compat=newest}
\tikzset{meter/.style={draw,thick,circle,fill=white,minimum size =0.75cm,inner sep=0pt}}

\hypertitle{circmes-ELEC1370}{4}{ELEC}{1370}{2013}{Juin}{Majeure}
{Nicolas Verbeek\and Adrien Couplet\and Martin Van Essche\and Guillaume Gilson\and Guillaume Colinet \and Brieuc Balon} 
{Claude Oestges, Bruno Dehez and Christophe Craeye}
\section{Question Oestges : phaseurs}
Soit le circuit suivant avec $V_o=\SI{5.5\angle\ang{104.1}}{\volt}$,
\begin{center}
    \begin{circuitikz} 
    	\draw
  		(0,-2.5) to[american voltage source,l=$\SI{6\angle\ang{0}}{\volt}$] (0,2.5) 
  		(2.5,2.5) to[european resistor,l=$Z$] (0,2.5)
  		(2.5,2.5) -- (5,2.5)
  		(2.5,0) to[L,i=$I_L$,l=$\SI{2\imagj}{\ohm}$,-*] (2.5,2.5)
  		(2.5,0) to[R,v=$V_o$,l=$\SI{2}{\ohm}$] (2.5,-2.5)
 		(5,2.5) to[american current source,l=$\SI{2\angle\ang{0}}{\ampere}$] (5,-2.5)
		(0,-2.5) -- (5,-2.5);
 	\end{circuitikz}
\end{center}
On demande de calculer
\begin{enumerate}
    \item Le courant $I_L$ dans l'inductance
    \item L'impédance $Z$ et les caractéristiques des composants ($R$,$L$,$C$) avec une fréquence de $f=\SI{10}{\kilo\hertz}$
    \item La puissance active délivrée par la source de tension
    \item Le dipôle équivalent de Thévenin aux bornes de la résistance de $\SI{2}{\ohm}$.
    \item Le courant dans une résistance de $3\Omega$ mise en série avec ce dipôle équivalent.
\end{enumerate}
\begin{solution}
\begin{enumerate}
    \item Tout d'abord il faut remarquer que le sens du courant $I_L$ est dans le même sens que la tension $V_o$ dès lors :
    \begin{equation*}
        I_L = - \frac{V_o}{2} = \SI{2.75\angle\ang{-75.9}}{\ampere}
    \end{equation*}
    \item Pour la suite des calculs nous utiliserons les notations telles qu'illustrées : 
    \begin{center}
    \begin{circuitikz} 
    	\draw
  		(0,-2.5) to[american voltage source,l=$\SI{6\angle\ang{0}}{\volt}$, i= $I_s$] (0,2.5) 
  		 (0,2.5) to[european resistor,l=$Z$,v=$V_Z$] (2.5,2.5)
  		(2.5,2.5) -- (5,2.5)
  		(2.5,0) to[L,i=$I_L$,l=$\SI{2\imagj}{\ohm}$,v = $V_L$,-*] (2.5,2.5)
  		(2.5,0) to[R,v=$V_o$,l=$\SI{2}{\ohm}$] (2.5,-2.5)
 		(5,2.5) to[american current source,l=$\SI{2\angle\ang{0}}{\ampere}$] (5,-2.5)
		(0,-2.5) -- (5,-2.5)
		(2.5,2.5) node[label ={[font=\footnotesize]above:$A$}]{};
 	\end{circuitikz}
\end{center}
Sur base de la loi d'Ohm ($V=ZI$) nous pouvons écrire : 
\begin{equation*}
    V_L = 2\imagj  \cdot I_L = \SI{5.5\angle\ang{14.1}}{\volt}
\end{equation*}
Nous pouvons effectuer une loi des noeuds sur le noeud A :
\begin{equation*}
    I_s = 2\angle\ang{0} - I_L = \SI{2.98\angle\ang{63.49}}{\ampere}
\end{equation*}
Nous pouvons effectuer une loi des mailles sur la maille de gauche :
\begin{equation*}
    V_Z = 6\angle\ang{0} + V_L - V_o = \SI{13.288\angle\ang{-17.493}}{\volt}
\end{equation*}
Sur base de la relation élémentraire ($V = ZI$) nous pouvons écrire :
\begin{equation*}
    Z = \frac{V_Z}{I_S} = 4.458\angle\ang{-80.988}= \SI{0.698 - 4.404\imagj}{\ohm} \Rightarrow R=\SI{0.7}{\ohm}~\mbox{et}~ C = \frac{1}{4.404*2\pi*f}  = \SI{3.614}{\micro\farad}
\end{equation*}
\item La puissance\footnote{Nous travaillons avec des phaseurs, ils représentent la valeur efficace de la tension ou du courant. S'il n'est pas spécifié dans l'énoncé si la valeur de la tension $V_o$ représente la valeur de crête ou la valeur efficace de la tension, n'hésitez pas à le spécifier dans votre exercice. Nous considérons que $V_o$ correspond à la valeur efficace dans la suite des calculs. } active délivrée par la source est : 
\begin{equation*}
    P = \Re(S) = \Re(V\cdot I_S^\ast) = \Re(6\angle\ang{0} \cdot  2.98\angle\ang{63.49}) = \SI{7.98}{\watt}
\end{equation*}
\item Pour calculer le dipôle équivalent, nous devons calculer 2 des 3 données suivantes : $V_{Th}$, $I_N$, $Z_{eq}$. Nous calculerons $V_{Th}$ et $Z_{eq}$ par souci de simplicité. Pour calculer $V_{Th}$, il faut retirer la résistance de $\SI{2}{\ohm}$. Nous obtenons donc le circuit
\begin{center}
    \begin{circuitikz} 
    	\draw
  		(0,-2.5) to[american voltage source,l=$\SI{6\angle\ang{0}}{\volt}$, i= $I_s$] (0,2.5) 
  		 (0,2.5) to[european resistor,l=$Z$,v=$V_Z$,-*] (2.5,2.5)
  		(2.5,2.5) -- (5,2.5)
  		(2.5,2.5) to [european voltages,open,v^=$V_{Th}$]  (2.5,-2.5)  
 		(5,2.5) to[american current source,l=$\SI{2\angle\ang{0}}{\ampere}$] (5,-2.5)
		(0,-2.5) -- (5,-2.5)
		(2.5,2.5) node[label ={[font=\footnotesize]above:$A$}]{};
 	\end{circuitikz}
\end{center}
Sur base de la loi d'Ohm ($V=ZI$) nous pouvons écrire :
\begin{equation*}
    V_Z = Z \cdot 2\angle\ang{0} =  4.458\angle\ang{-80.988} \cdot 2\angle\ang{0}=\SI{8.916\angle\ang{-80.988}}{\volt}
\end{equation*}
Nous pouvons aussi écrire de façon élémentaire :
\begin{equation*}
    V_{Th} = 6 - V_Z = \SI{9.93\angle\ang{62.4}}{\volt}
\end{equation*}
Pour calculer $Z_{eq}$, il faut retirer la résistance de $\SI{2}{\ohm}$. Nous obtenons donc le circuit : 
\begin{center}
    \begin{circuitikz}
    \draw
       (0,2.5) to[european resistor,l=$Z$] (2.5,2.5)
       (0,-2.5) -- (0,2.5)
  	   (2.5,0) to[L,l=$\SI{2\imagj}{\ohm}$,*-] (2.5,2.5)
  	   (0,-2.5) to [short, -*] (2.5,-2.5); 
    \end{circuitikz}
\end{center}
De façon évidente : 
\begin{equation*}
    Z_{Th}= Z +2\imagj  = \SI{0.698-2.404\imagj}{\ohm}
\end{equation*}
\item Nous pouvons représenter le circuit comme tel : 
\begin{center}
     \begin{circuitikz}
     \draw
       (0,-2.5) to[american voltage source,l=$V_{Th}$, i= $I$] (0,2.5) to [european resistor,l=$Z_{eq}$](2.5,2.5)
       to [R, l=$\SI{3}{\ohm}$] (2.5 ,-2.5) --(0,-2.5);
     \end{circuitikz}
\end{center}
Sur base de la relation élementaire ($V=ZI$) nous obtenons :
\begin{equation*}
    I = \frac{V_{Th}}{Z_{eq}+3} = \SI{2.25\angle\ang{95.42}}{\ampere}
\end{equation*}
\end{enumerate}
\end{solution}
\newpage
\section{Question Dehez : triphasé}
Soit le circuit suivant
\begin{center}
    \begin{circuitikz}[scale=0.9]
        \coordinate (s1) at (0,0);
        \coordinate (s2) at ($ (s1) + ({3*cos(60)},{-3*sin(60)}) $);
        \coordinate (s3) at ($ (s1) + ({-3*cos(60)},{-3*sin(60)}) $);
        
        \draw (s1) to[V,l=$34\angle\ang{0}\,\si{\volt}$,i=$I_1$,*-*] (s2) to [V,l=$34\angle\ang{-120}\,\si{\volt}$,-*] (s3) to [V,l=$34\angle\ang{-240}\,\si{\volt}$] (s1);
        \draw (s1) to [R,l=\SI{2}{\ohm}] ++(5,0) coordinate (l11); 
        \draw (s3) to[short] ++(0,-2) coordinate (p1);
        
        \coordinate (l12) at ($ (l11) + ({3*cos(60)},{-3*sin(60)}) $);
        \coordinate (l13) at ($ (l11) + ({-3*cos(60)},{-3*sin(60)}) $);
        \coordinate (c1) at ($ (l11) + (0,{-sqrt(3)}) $);
        
        \draw (l11) to[L,l_=\SI{j}{\ohm},-*] (c1);
        \draw (c1) to[L,l=\SI{j}{\ohm}] (l12);
        \draw (c1) to[L,l_=\SI{j}{\ohm}] (l13);
        
        \draw (l13) to[R,l=\SI{2}{\ohm}] (s2);
        \draw (l12) to[short] ++(0,-2) to[R,l=\SI{2}{\ohm}] (p1);
        
        \coordinate (l21) at ($ (l11) + (3.5,0) $);
        \coordinate (l22) at ($ (l21) + ({3*cos(60)},{-3*sin(60)}) $);
        \coordinate (l23) at ($ (l21) + ({-3*cos(60)},{-3*sin(60)}) $);
        \coordinate (c2) at ($ (l21) + (0,{-sqrt(3)}) $);
        
        \draw (c2) to[L,l_=\SI{8j}{\ohm},*-] (l21);
        \draw (c2) to[L,l=\SI{8j}{\ohm}] (l22);
        \draw (c2) to[L,l_=\SI{8j}{\ohm}] (l23);
        
        \coordinate (d1) at ($ (l21) + (3.5,0) $);
        \coordinate (d2) at ($ (d1) + ({3*cos(60)},{-3*sin(60)}) $);
        \coordinate (d3) at ($ (d1) + ({-3*cos(60)},{-3*sin(60)}) $);
        \draw (d3) to[short] ++(0,-1) coordinate (p2);
        \draw (d2) to[short] ++(0,-2) coordinate (p3);
        \draw (l21) to[short] (d1);
        \draw (l23) to[short] ++(0,-1) to[short] (p2);
        \draw (l22) to[short] ++(0,-2) to[short] (p3);
        \draw (d1) to[C,l=\SI{-3j}{\ohm},i=$I_2$,*-*] (d2) to[C,l=\SI{-3j}{\ohm},-*] (d3) to[C,l=\SI{-3j}{\ohm}] (d1);
        
        \draw (l11) node[below left]{$\bullet$};
        \draw (l21) node[below right]{$\bullet$};
        \draw (l12) node[above]{$\bullet$};
        \draw (l22) node[below left]{$\bullet$};
        \draw (l13) node[below right]{$\bullet$};
        \draw (l23) node[above]{$\bullet$};
        \draw [<->,>=stealth] ($ (l11) + (-0.2,0.2) $)  to [bend left] node[pos=0.5,above] {\SI{6j}{\ohm}} ($ (l21) + (0.2,0.2) $);

    \end{circuitikz}
\end{center}
\begin{enumerate}
    \item Calculer le facteur de dispersion du couplage magnétique;
    \item Calculer l'amplitude et la phase de $I_1$ et $I_2$.
\end{enumerate}
\begin{solution}
\begin{enumerate}
    \item Le facteur de dispersion s'écrit comme :
    \begin{equation*}
        \sigma = 1 -\frac{M^2}{L_1L_2} = 1 - \frac{(6\imagj )^2}{\imagj \cdot 8\imagj } = -3.5 
    \end{equation*}
    On a ici un facteur de dispersion négatif (impossible physiquement mais erreur de Mr. Dehez. \textit{Errare humanum est}.)
    \item Nous pouvons réécrire le circuit en un circuit simple monophasé en modifiant la source triphasé en triangle en une source triphasée étoile et la charge capacitive triangle en une charge capacitive étoile. Ce faisant nous obtenons le circuit monophasé suivant. 
    \begin{center}
        \begin{circuitikz}
        \draw
        (0,0) to [american voltage source,l=$\SI{\frac{34}{\sqrt{3}}\angle\ang{-30}}{\volt}$, i= $I_{1}^{'}$] (0,5) to [R, l = $\SI{2}{\ohm}$] (5,5) to [L, l = $\SI{\imagj}{\ohm}$](5,0) --(0,0)
        (7,5) to [L, l =$\SI{8\imagj}{\ohm}$,i=$I_{2}^{'}$] (7,0)--(10,0) to[C, l=$\SI{-\imagj}{\ohm}$] (10,5) -- (7,5);
        \draw [fill=black] (4.8,3)node(a){} circle (2pt);
        \draw [fill=black] (7.3,3)node(b){} circle (2pt);
        \draw [<->,>=stealth] (a)  to [bend left] node[pos=0.5,fill=white] {$\SI{6\imagj}{\ohm}$} ++(2,0); 
        \end{circuitikz}
    \end{center}
En faisant la loi des mailles sur les 2 mailles du circuit nous obtenons :
\begin{equation*}
    \left \{
\begin{array}{rcl}
&\frac{34}{\sqrt{3}}\angle\ang{-30} &=(2+\imagj )I_{1}^{'} +6\imagj  I_{2}^{'} \\
&0 &= 6\imagj I_{1}^{'} + 7\imagj I_{2}^{'}
\end{array}
\right.
\Leftrightarrow 
\left \{
\begin{array}{rcl}
I_{1}^{'} &=& \SI{4.267\angle\ang{34.23}}{\ampere}\\
I_{2}^{'} &=& \SI{3.657\angle\ang{-145.77}}{\ampere}
\end{array}
\right.
\end{equation*}
Sachant que les courant $I_{1}^{'}$ et $I_{2}^{'}$ sont les courants résultant de la transformation triangle étoile, nous pouvons trouver les courants $I_1$ et $I_2$ comme tel : 
\begin{equation*}
\left \{
\begin{array}{rcl}
|I_{1} |&=& |\frac{I_{1}^{'}}{\sqrt{3}}|~~ \mbox{et} ~~\phi_{I_{1}} = \phi_{I_{1}^{'}} +\ang{30} \\
|I_{2} |&=& |\frac{I_{2}^{'}}{\sqrt{3}}|~~ \mbox{et} ~~\phi_{I_{2}} = \phi_{I_{2}^{'}} +\ang{30} 
\end{array}
\right.
\Leftrightarrow
\left \{
\begin{array}{rcl}
I_1 &=& \SI{2.463\angle\ang{64.23}}{\ampere}\\
I_2 &=& \SI{2.11\angle\ang{-115.77}}{\ampere}
\end{array}
\right.
\end{equation*}
\end{enumerate}
\end{solution}
\newpage
\section{Question Craeye : transitoire}
Soit le circuit commuté ci-dessous. Donnez l'expression temporelle de la tension $V_c$ pour $t>0$ (l'interrupteur passe de la borne $A$ à la borne $B$ en $t=0$). Les valeurs des éléments du circuits sont: $R_o = R_1 = R = \SI{1}{\kilo\ohm}$, $C=\SI{1}{\nano\farad}$ et $L_o = L = \SI{1}{\milli\henry} $.
\begin{center}
    \begin{circuitikz}
        \draw (0,0) to[american voltage source,l=$\SI{1}{\volt}$] ++(0,4) to[R,l=$R_1$] ++(0,4) to[short,-o] ++(1,0) coordinate (B);
        \draw (2,8) to[short] ++(2,0) to[R,l=$R$] ++(2,0) to[L,l=$L$] ++(0,-8) to[short] ++(-6,0);
        \draw (2,0) to[american voltage source,l=$\SI{2}{\volt}$] ++(0,2) to[L,l=$L_o$] ++(0,2) to[R,l=$R_o$] ++(0,2) to[short,-o] ++(0,1) coordinate (A);
        \draw (1.4,7.4) to[short,o-] (2,8);
        \draw (4,0) to[short] ++(0,3) to[C,l=$C$,v>=$V_c$] ++(0,2) to[short] ++(0,3);
        
        \draw [->,>=stealth] ($ (A) + (-0.2,0) $)  to [bend left] node[pos=0.5,below left] {$t=0$} ($ (B) + (0,-0.2) $);
        \draw (A) node[right]{$A$};
        \draw (B) node[above]{$B$};
    \end{circuitikz}
\end{center}
\begin{solution}
En $t<0$, on peut réécrire le circuit comme :
\begin{center}
    \begin{circuitikz}
    \draw 
    (0,0) to[american voltage source,l=$\SI{2}{\volt}$] ++(0,3) to[R, l = $R_o$] (0,6) -- (1,6) to [R, l=$R$,i= $I_L(0^-)$,*-](4,6) -- (4,0)--(0,0)
    (1,6) to [european voltages,open,v^=$V_{C}(0^-)$]  (1,0)  ;
    \end{circuitikz}
\end{center}
Sur base du diviseur de tension et la loi des mailles sur l'unique maille nous obtenons :
\begin{equation*}
    \left \{
\begin{array}{rcl}
V_{C}(0^-) &=& \frac{R}{R+R_o} \cdot 2[V] = 1[V]\\
I_L(0^-) &=& \frac{2}{R_0+R} = 10^{-3} [A]
\end{array}
\right.
\end{equation*}
En $t>0$, on peut réécrire le circuit comme :
\begin{center}
    \begin{circuitikz}
    \draw (0,0) to[american voltage source,l=$\SI{1}{\volt}$] ++(0,4) to[R,l=$R_1$,v =$V_{R_1(s)}$, i =$I_1(s)$] ++(0,4) --(2,8) to [C,l=$C$, i=$I_2(s)$,v=$V_C(s)$](2,0)
    (2,8) to [R, l=$R$,v=$V_R(s)$] (5,8) to [L,l=$L$, v=$V_L(s)$](5,0) --(0,0);
    \end{circuitikz}
\end{center}
Sur base de la loi des mailles sur la maille de droite nous pouvons écrire :
\begin{align*}
   & \Rightarrow &V_C(s) &= V_R(s)+V_L(s) ~~\mbox{avec}~~ V_L(s) = sL(I_1(s)-I_2(s))-LI_L(0^-)\\
   & \Leftrightarrow &V_C(s) &= R(I_1(s)-I_2(s))+sL(I_1(s)-I_2(s))-LI_L(0^-) \\
    & \Leftrightarrow &V_C(s) &= I_1(s)(R+sL)-I_L(0^-)-I_2(R+sL) ~~\mbox{avec}~~I_2(s) = sV_C(s)C -CV_C(0^-)\\
    & \Leftrightarrow & V_C(s) &= I_1(s)(R+sL)-I_L(0^-)-(sV_C(s)C -CV_C(0^-))(R+sL)\\
    &   \Leftrightarrow &I_1(s)(R+sL) &= V_C(s) (1+RSC+s^2CL)+LI_L(0^-)-CV_C(0^-)(R+sL)
\end{align*}
\begin{equation*}
   \Rightarrow  I_1(s)= \frac{V_C(s)(1+RCs+CLs^2)+LI_L(0^-)-RCV_C(0^-)-CLV_C(0^-)s}{R+sL}
\end{equation*}
Sur base de la loi des mailles sur la maille de gauche nous pouvons écrire :
\begin{align*}
    &\Rightarrow \frac{1}{s} = R_1I_1(s)+V_C(s)\\
   &\Leftrightarrow \frac{1}{s} = \frac{R_1}{R+sL}(V_C(s)(1+RCs+CLs^2)+LI_L(0^-)-RCV_C(0^-)-CLV_C(0^-)s) + V_C(s)\\
   & \Leftrightarrow  R+sL = sR_1V_C(s)(1+RCs+CLs^2)+sR_1((LI_L(0^-))-CV_C(0^-)R-CLV_C(0^-)s)+V_C(s)(sL+R)s\\
    & \Leftrightarrow V_{C}(s)(s(R+R)+s^{2}(R_{1}RC+L)+R_{1}CLs^{3})=R+sL-sR_{1}LI_{L}(0^-)+sRR_{1}CV_{C}(0^-)+s^{2}R_{1}CLV_{C}(0^-)
\end{align*}
\begin{equation*}
    \Rightarrow V_C(s) = \frac{R+s(L-R_1LI_L(0^-)+RR_1CV_C(0^-))+s^2R_1CLV_C(0^-)}{s(R_1+R+s(L+R_1RC)+s^2LCR_1)}
\end{equation*}
En substituant les valeurs nous obtenons :
\begin{equation*}
    V_C(s) = \frac{1000+10^{-3}s+10^{-9}s^2}{s(2000+2*10^{-3}s+10^{-9}s^2)} = \frac{s^2+10^6s+10^{12}}{s(s^2+2*10^6s+2*10^{12})}
\end{equation*}
En effectuant la décomposition en fraction simple nous avons :
\begin{equation*}
    V_C(s) = \frac{A}{s} + \frac{B*10^6}{(s+10^6)^2+10^{12}}+\frac{C*(10^6+s)}{(s+10^6)^2+10^{12})} \Leftrightarrow 
\begin{dcases}
 A+C =1\\
 2*10^6 A+10^6B+10^6C = 10^6\\
 2*10^{12}A = 10^{12}\\
\end{dcases}
\Leftrightarrow
\begin{dcases}
 A= \frac{1}{2}\\
 B=-\frac{1}{2}\\
 C=\frac{1}{2}\\
\end{dcases}
\end{equation*}
Nous obtenons donc dans le domaine de Laplace :
\begin{equation*}
    V_C(s) = \frac{1}{2}(\frac{1}{s} -\frac{10^6}{(s+10^6)^2+10^{12}}+\frac{(10^6+s)}{(s+10^6)^2+10^{12}}
\end{equation*}
Cela correspond dans le domaine temporel à : 
\begin{equation*}
    V_c(t) = \frac{1}{2} u(t) (1 +e^{-10^6t}(\cos(10^6t)-\sin(10^6t)))
\end{equation*}
La solution équivalente avec seulement un cosinus est 
\begin{equation*}
    V_c(t) = \frac{1}{2} + 0.707 e^{-10^6t}\cos(10^6t+0.785)
\end{equation*}
Le graphe ressemble à :  
\begin{center}
    \begin{tikzpicture}
        \begin{axis}[enlargelimits=true,grid=major,ylabel=$V_c(t)$,xlabel=$t(s)$]
            \addplot [blue,domain=0:0.00001,samples=200]{0.5 + 0.5*e^(-(10^6)*x)*cos(deg(10^6*x + 0.785))};
        \end{axis}
    \end{tikzpicture}
\end{center}
\end{solution}

\end{document}