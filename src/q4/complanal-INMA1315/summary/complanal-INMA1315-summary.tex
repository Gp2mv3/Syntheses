\documentclass[fr]{../../../eplsummary}

\newcommand{\sbt}{\,\begin{picture}(-1,1)(-1,-3)\circle*{2}\end{picture}\ }

\newcommand{\Le}{\mathcal{L}}
\renewcommand{\K}{\mathcal{K}}
\newcommand{\M}{\mathcal{M}}

\hypertitle{Compléments d'analyse}{4}{INMA}{1315}
{Benoît Legat}
{Michel Willem}

\begin{myrem}
\textbf{Attention}: le cours a changé de façon substantielle en 2019 et cette synthèse porte sur une ancienne version.
La plupart du contenu reste valide mais il est recommandé de baser son étude sur une autre synthèse plus récente,
et éventuellement d'utiliser celle-ci comme complément.
\end{myrem}

\begin{myrem}
  M. Willem insiste pour que
  chaque définition, proposition, théorème, ... doit commencer
  par un mot en français (ne pas commencer par ``$X$ un espace métrique'' mais
  par ``Soit $X$ un espace métrique'') et il faut utiliser le français à la
  place des symboles $\forall$ et $\exists$.
  Je n'en ferai donc pas exception dans cette synthèse.
\end{myrem}

\section{Distances et Normes}

\begin{mydef}[Produit scalaire]
  Un produit scalaire sur un espace vectoriel $X$ est une fonction
  \[ X \times X \to \R : (u,v) \to (u|v) \]
  telle que
  \begin{enumerate}
    \item[($\mathcal{S}_1$)] quels que soient $u \in X\setminus\{0\}$, $(u|u) > 0$;
    \item[($\mathcal{S}_2$)] quels que soient $u,v,w \in X$ et $\alpha,\beta \in \R$,
      \[ (\alpha u + \beta v|w) = \alpha(u|w) + \beta(v|w); \]
    \item[($\mathcal{S}_3$)] quel que soient $u,v \in X$, $(u|v) = (v|u)$.
  \end{enumerate}
\end{mydef}
\begin{mydef}[Norme]
  Une norme sur un espace vectoriel $X$ est une fonction
  \[ X \to \R : u \to \|u\| \]
  telle que
  \begin{enumerate}
    \item[($\mathcal{N}_1$)] quel que soit $u \in X\setminus\{0\}$, $\|u\| > 0$;
    \item[($\mathcal{N}_2$)] quels que soient $u \in X$ et $\alpha \in \R$, $\|\alpha u\| = |\alpha|\|u\|$;
    \item[($\mathcal{N}_3$)] (inégalité de {\sc Minkowski}) quels que soient $u,v \in X$,
      \[ \|u + v\| \leq \|u\| + \|v\|. \]
  \end{enumerate}
\end{mydef}
\begin{mydef}[Distance]
  Une distance sur un ensemble $X$ est une fonction
  \[ X \times X \to \R : (u,v) \to d(u,v) \]
  telle que
  \begin{enumerate}
    \item[($\mathcal{D}_1$)] quels que soient $u,v \in X$, $d(u,v) = 0$ si et seulement
      si $u = v$;
    \item[($\mathcal{D}_2$)] quels que soient $u,v \in X$, $d(u,v) = d(v,u)$;
    \item[($\mathcal{D}_3$)] quels que soient $u,v,w \in X$,
      \[ d(u,v) \leq d(u,w) + d(w,v). \]
  \end{enumerate}
\end{mydef}

\begin{myprop}[9.2, 11.5b]
  Le produit scalaire, la norme, la somme et la multiplication par un scalaire
  sont des fonctions continues sur un espace vectoriel.
  La distance est une fonction \emph{uniformément} continue.

  Cependant, ça ne veut \emph{pas} dire que si $u$ et $v$ sont continues,
  $x \mapsto d(u(x),v(x))$ est une fonction continue.
\end{myprop}

\begin{myprop}[11.5a]
  On peut induire une distance d'une norme et une norme d'un produit scalaire. Autrement dit,
  \begin{itemize}
    \item Un espace préhilbertien (resp. Hilbertien) est normé (resp. de Banach) avec $\|x\| = \sqrt{(x|x)}$ et
    \item un espace normé (resp. de Banach) est métrique (resp. complet) avec $d(x,y) = \|x-y\|$.
  \end{itemize}
  C'est à dire qu'on peut aller ``gratuitement'' vers le haut dans la \tabref{em}.
\end{myprop}

\begin{table}[!ht]
  \centering
  \begin{tabular}{lll}
    $d(\sbt,\sbt):X \times X\to\R$ & Métrique & Complet\\
    $\|\sbt\|:X\to\R$ & Normé & de Banach\\
    $(\sbt|\sbt):X \times X\to\R$ & Préhilbertien & Hilbertien\\
  \end{tabular}
  \caption{Noms des espaces métriques}
  \label{tab:em}
\end{table}

\begin{myrem}
  Pour définir une norme ou un produit scalaire, il faut qu'on soit dans un espace
  vectoriel car $u + v$ et $\alpha u$ doivent avoir du sens alors que qu'on peut
  définir une distance sur n'importe quel ensemble.
\end{myrem}

\begin{myprop}[Identités diverses 11.3]
  Soit $X$ un espace préhilbertien et $u,v \in X$. Alors
  \begin{description}
    \item[Parallélogramme] \[ \|u + v\|^2 + \|u - v\|^2 = 2\|u\|^2 + 2\|v\|^2. \]
    \item[Polarisation] \[ (u|v) = \frac{1}{4}\|u + v\|^2 + \frac{1}{4}\|u - v\|^2. \]
    \item[Pythagore] $(u|v) = 0$ si et seulement si \[ \|u+v\|^2 = \|u\|^2 + \|v\|^2. \]
  \end{description}
\end{myprop}

\begin{myineg}[Cauchy-Schwarz 11.4]
  Soit $X$ un espace préhilbertien et $u,v \in X$. Alors
  \[ |(u|v)| \leq \|u\| \|v\|. \]
\end{myineg}

\subsection{Suites}
\begin{mydef}[2.2]
  Soit $X$ un espace métrique,
  une suite $u_n$ est
  \begin{description}
    \item[convergente] si elle converge vers un $x \in X$,
      c'est à dire que
      \[ \lim_{n\to\infty} d(u_n,u) = 0; \]
    \item[de Cauchy] si
      \[ \lim_{j,k\to\infty} d(u_j,u_k) = 0; \]
    \item[bornée] si
      \[ \sup_n d(u_0,u_n) < \infty. \]
  \end{description}
\end{mydef}

\begin{myprop}[2.2]
  Toute suite convergente est de Cauchy et toute suite
  de Cauchy est bornée.
\end{myprop}

\subsubsection{Suite réelle croissante (pareil pour décroissante)}
\label{sec:Rcroi}
Lorsqu'une suite \emph{réelle} $(u_n)$ est croissante, alors
$\lim_{n\to\infty} u_n = \sup_{n} u_n$.
Comme le supremum existe toujours et appartient à $[-\infty,\infty]$,
la limite aussi, on a donc
\begin{myprop}[1.13]
  Si $(u_n)$ est croissante et bornée,
  alors $(u_n)$ converge.
\end{myprop}

\subsubsection{L'omniprésence des suites}
Beaucoup de concepts sont définis à l'aide de suite.
Dans une preuve où un exercice, on doit donc souvent construire des suites
pour nous permettre de retomber sur les définitions.

Le raisonnement est souvent semblable pour construire ces suites.
On utilise souvent la fonction $n \mapsto \frac{1}{n}$ et
la fonction $n \mapsto n$.

On peut par exemple passer d'un supremum à une suite convergent
vers ce supremum $c$ (s'il est fini) en arguant que par la définition
du supremum, il existe $c - \frac{1}{n} < x_n < c$, sinon
le supremum serait $\leq c - \frac{1}{n}$.
Si le supremum était infini, on aurait utilisé $n \mapsto n$.

C'est utilisé entre aute dans la démonstration du lemme~3.4.
Une variante est utilisé dans l'exercice~3.6 du livre.

\subsection{Compact, précompact et complet}
\begin{mydef}[2.5]
  Soit $X$ un espace métrique, $X$ est
  \begin{description}
    \item[complet] si toute suite de Cauchy converge;
    \item[précompact] si toute suite contient une sous-suite
      de Cauchy;
    \item[compact] si toute suite contient une sous-suite
      convergente.
  \end{description}
\end{mydef}

\begin{mytheo}[2.8]
  \label{theo:Rcomplet}
  $(\R,d:(x,y)\mapsto|x-y|)$ est complet.
\end{mytheo}
Le théorème~\ref{theo:Rcomplet} est très important.
Il n'est pas facile à démontrer,
lorsqu'on essaie de montrer qu'une autre espace est complet,
on l'utilise donc souvent.

\begin{myprop}[2.6, 2.7]
  Un espace métrique est compact si et seulement si
  il est précompact et complet.
  Ça se démontre en utilisant le fait que toute suite
  de Cauchy qui contient une suite convergente converge.
\end{myprop}

\begin{myrem}
  Un espace de dimension finie est compact ssi il est fermé et borné.
  Ce n'est pas le cas pour ceux de dimension infinie comme
  $(\mathcal{C}([0,1],\R), d_\infty)$ qui n'est pas compact
  alors qu'il est fermé et borné (voir exercice 3.3).
\end{myrem}

\subsection{Fermé, ouvert et dense}
\begin{mydef}[2.10]
  Soit $X$ un espace métrique et $S \subseteq X$.
  $u \in X$ est
  \begin{description}
    \item[adhérent] à $S$ si pour tout $r > 0$,
      \[ B(u,r) \cap S \neq \emptyset. \]
      L'ensemble des points adhérents est l'adhérence, notée
      $\bar{S}$;
    \item[intérieur] à $S$ s'il existe $r > 0$,
      \[ B(u,r) \subseteq S. \]
      L'ensemble des points intérieurs est l'intérieur, notée
      $\overset{\circ}{S}$.
  \end{description}
  $S$ est
  \begin{description}
    \item[fermé] si $\bar{S} = S$;
    \item[ouvert] si $\overset{\circ}{S} = S$;
    \item[dense] si $\bar{S} = X$.
  \end{description}
\end{mydef}

\begin{myprop}[2.11]
  Les énoncés suivant sont équivalents:
  \begin{itemize}
    \item $S$ est fermé;
    \item $X \setminus S$ est ouvert;
    \item toute suite convergente dans $X$ converge dans $S$.
  \end{itemize}
\end{myprop}

\begin{myprop}[2.12]
  Toute union d'ouvert est un ouvert et toute intersection
  \emph{finie} d'ouvert est un ouvert.
  Toute intersection de fermé est un fermé et toute union
  \emph{finie} de fermé est un fermé.
\end{myprop}

\begin{mytheo}[Baire 2.13]
  Dans un espace métrique \emph{complet},
  tout intersection d'\emph{ouverts} \emph{denses} est
  \emph{dense}.
\end{mytheo}

\subsection{Partie d'espace métrique complet}
Si on est une partie $S$ d'une espace métrique complet $X$.
On a les deux propriétés suivantes
\begin{myprop}[2.15]
  Soient $X$ un espace métrique \emph{complet} et $S \subseteq X$.
  $S$ est complet si et seulement si $S$ est fermé.
\end{myprop}

\begin{mytheo}[Critère de Fréchet 2.16]
  Soient $X$ un espace métrique \emph{complet} et $S \subseteq X$.
  Les énoncés suivant sont équivalents:
  \begin{itemize}
    \item $S$ est précompact;
    \item pour tout $\epsilon > 0$, il existe un recouvrement fini de $S$
      par de boules de rayon $\epsilon$.
  \end{itemize}
\end{mytheo}

\subsection{Continuité}
\begin{mydef}[Continuité et continuité uniforme 3.1]
  Soient $X$ et $Y$ des espaces métriques.
  \begin{description}
    \item $u:X \to Y$ est continue en $x \in X$ si pour tout $\epsilon > 0$,
      il existe $\delta > 0$ tel que
      \[ \sup \{d(u(x), u(y)) | y \in X, d(x,y) \leq \delta\} \leq \epsilon \]
      $f$ est continue si elle est continue en tout point de $X$.
    \item $u:X \to Y$ est uniformément continue si pour tout $\epsilon > 0$,
      il existe $\delta > 0$ tel que
      \[ \omega_u(\delta) = \sup \{d(f(x), f(y)) | x,y \in X, d(x,y) \leq \delta\} \leq \epsilon \]
      où $\omega_u$ est le module de continuité de $u$.
  \end{description}
\end{mydef}

\begin{myprop}
  Tout fonction uniformément continue est continue.
\end{myprop}

\begin{mylem}[3.2]
  Soient $X$ et $Y$ des espaces métriques, $u:X \to Y$
  et $x \in X$.
  Les deux énoncés suivant sont équivalents:
  \begin{itemize}
    \item $u$ est continue en $x$;
    \item pour toute suite $(x_n)$ convergant vers $x$ dans $X$,
      $(u(x_n))$ converge vers $u(x)$ dans $Y$.
  \end{itemize}
\end{mylem}

\begin{myprop}[3.3]
  Soient $X$ et $Y$ des espaces métriques, $K \subseteq X$ un compact
  et $u : X \to Y$ une application continue et constante
  sur $X \setminus K$.
  Alors $u$ est uniformément continue.
\end{myprop}

\section{Intégrale}
\begin{mydef}[Mesure 6.1]
  Une intégrale élémentaire sur l'ensemble $\Omega$ est définie par un
  espace vectoriel $\Le(\Omega,\mu)$ de fonctions définies sur $\Omega$
  à valeurs dans $\R$ par une fonctionnelle
  \[ \mu : \Le \to \R : u \to \int_\Omega u \dif \mu \]
  tels que
  \begin{enumerate}
    \item[($\mathcal{J}_1$)] quel que soit $u \in \Le$, $|u| \in \Le$;
    \item[($\mathcal{J}_2$)] quels que soient $u,v \in \Le$ et $\alpha,\beta \in \R$,
      \[ \int_\Omega \alpha u + \beta v \dif \mu = \alpha \int_\Omega u \dif \mu + \beta \int_\Omega v \dif mu; \]
    \item[($\mathcal{J}_3$)] quel que soit $u \in \Le$ tel que $u \geq 0$, $\int_\Omega u \dif \mu \geq 0$;
    \item[($\mathcal{J}_4$)] quel que soit la suite $(u_n) \subseteq \Le$ telle que $u_n \downarrow 0$,
      $\lim_{n\to\infty} \int_\Omega u_n \dif \mu \to 0$.
  \end{enumerate}
\end{mydef}

\begin{mydef}[Mesure positive 6.28]
  Une mesure positive sur $\Omega$ est une intégrale élémentaire $\mu:\Le \to \R$ sur $\Omega$
  telle que
  \begin{enumerate}
    \item[($\mathcal{J}_6$)] quel que soit $u \in \Le$, $\min(u,1) \in \Le$.
  \end{enumerate}
\end{mydef}

On définit ici une intégrale élémentaire qui pourrait être n'importe quelle fonctionnelle
qui respecte les 4 axiomes.
Le plus souvent, on s'intéresse à l'intégrale de Cauchy qui est l'intégrale classique
valant la somme de Riemann
\[ S_j = 2^{-jN} \sum_{k \in \Z^N} u(k/2^j) \]
On l'appelle ici l'intégrale \emph{contrète} et on la reconnait car on utilise
$\dif x$ comme d'habitude à la place de $\dif \mu$.

$\Le(\Omega,\dif x)$ est l'ensemble des fonctions $u$ telles que
que les sommes de Darboux
\begin{align*}
  A_j & = 2^{-jN} \sum_{k \in \Z^N} \min\{u(x) | x \in B[k/2^{j},2^{-j}]\}u(k2^{-j})\\
  B_j & = 2^{-jN} \sum_{k \in \Z^N} \max\{u(x) | x \in B[k/2^{j},2^{-j}]\}u(k2^{-j})
\end{align*}
qui satisfont bien entendu $A_j \leq S_j \leq B_j$, sont telles que
$\lim_{n\to\infty} A_j = \lim_{n\to\infty} B_j$.

Si $u \in \K(\Omega)$, alors $u \in \Le(\Omega,\dif x)$ car par la proposition~3.3,
ça signifie que $u$ est uniformément continue.
Cependant, $u$ ne doit pas nécessairement appartenir à $\K(\Omega)$ pour
appartenir à $\Le(\Omega,\dif x)$.

\begin{mydef}[Suite fondamentale 6.4]
  Une suite fondamentale est une suite $(u_n) \in \Le$ \emph{croissante} telle que
  $\int_\Omega u_n \dif \mu$ est borné.
\end{mydef}

Comme $u_n$ est croissante, son intégrale est une suite \emph{réelle} croissante,
et donc dire que l'intégrale est bornée revient à dire que (voir section~\ref{sec:Rcroi}),
\[ \lim_{n\to\infty} \int_\Omega u_n \dif \mu = \sup_n \int_\Omega u_n \dif \mu < \infty. \]

\begin{mydef}[6.8]
  Une fonction $u : \Omega \to ]-\infty,\infty]$ appartient à $\Le^+$ s'il existe
  une suite \emph{fondamentale} $(u_n) \subseteq \Le$ telle que $u_n$ converge
  \emph{simplement} vers $u$.
  L'intégrale par rapport à $\mu$ de $u$ est définie par
  \[ \int_\Omega u \dif \mu = \lim_{n\to\infty} \int_\Omega u_n \dif \mu. \]
\end{mydef}

\begin{mydef}[6.12]
  Une fonction réelle $u$ définie \emph{presque partout} sur $\Omega$
  appartient à $\Le^1$ s'il existe
  deux fonctions $f,g \subseteq \Le^+$ telles que $u = f - g$ \emph{presque partout}.
  L'intégrale par rapport à $\mu$ de $u$ est définie par
  \[ \int_\Omega u \dif \mu = \int_\Omega f \dif \mu - \int_\Omega g \dif \mu. \]
\end{mydef}

\begin{myprop}[Conditions suffisantes à l'intégrabilité]
  $u$ est intégrable, c'est à dire $u \in \Le^1$, si
  \begin{description}
    \item[Théorème de Levi (6.15)]
      il existe $(u_n) \subseteq \Le^1$ presque partout \emph{croissante} et dont $\int_\Omega u_n \dif \mu < \infty$.
    \item[Théorème de convergence dominée de Lebesgue (6.17)]
      il existe $(u_n) \subseteq \Le^1$ convergeant presque partout vers $u$
      et $f \in \Le^1$ tels que pour tout $n$, $|u_n| \leq f$ presque partout.
    \item[Théorème de comparaison (6.18)]
      il existe $(u_n) \subseteq \Le^1$ convergeant presque partout vers $u$
      et $f \in \Le^1$ tels que $|u| \leq f$ presque partout.
    \item[Mesurable comparée avec une intégrable (6.20d)] $u \in \M$ et qu'il existe $f \in \Le^1$ tel que $|u| \leq f$ presque partout.
    \item[Produit de mesurables (6.31)] il existe $v,w$ tels que $u = vw$.
    \item[Continuité (pour concrète uniquement)] En effet, comme la mesure contrète est une mesure positive
      et que $\{u > t\}$ est l'antécédent de $]t,+\infty[$ qui est ouvert par une application continue, il est ouvert.
      Le théorème~6.36 nous garanti alors ue $\{u > t\}$ est mesurable.
      Dès lors, comme la mesure concrète est positive, le théorème~6.30 montre que $u$ est mesurable.
  \end{description}
\end{myprop}


\begin{mydef}[Fonction mesurable 6.19]
  Une fonction réelle $u$ définie \emph{presque partout} sur $\Omega$
  appartient à $\M(\Omega,\mu)$ s'il existe une suite fondamentale $(u_n) \subseteq \Le(\Omega,\mu)$ telle
  que $u_n$ converge presque partout vers $u$.
  deux fonctions $f,g \subseteq \Le^+$ telles que $u = f - g$ \emph{presque partout}.
  L'intégrale par rapport à $\mu$ de $u$ est définie par
  \[ \int_\Omega u \dif \mu = \int_\Omega f \dif \mu - \int_\Omega g \dif \mu. \]
\end{mydef}

\begin{myprop}[Conditions suffisantes à la mesurabilité]
  $u$ est mesurable, c'est à dire $u \in \M$, si
  \begin{description}
    \item[Limite de mesurables (6.24)] $\exists (u_n) \subseteq \M$ convergent presque partout vers
      une limite finie $u$.
    \item[Produit de mesurables (6.31)] $\exists v,w$ tels que $u = vw$.
    \item[Continuité (pour concrète uniquement)] En effet, comme la mesure contrète est une mesure positive
      et que $\{u > t\}$ est l'antécédent de $]t,+\infty[$ qui est ouvert par une application continue, il est ouvert.
      Le théorème~6.36 nous garanti alors ue $\{u > t\}$ est mesurable.
      Dès lors, comme la mesure concrète est positive, le théorème~6.30 montre que $u$ est mesurable.
  \end{description}
\end{myprop}

\annexe
\section{Preuves manquantes à connaitre}
Voici la preuve des propositions de \cite{willem2008principes}
qu'on doit connaitre mais qui ne sont pas donnée.
\subsection{Proposition 2.6}
Soit $(u_n)$ une suite de Cauchy et soit
$(u_{n_j})$ une sous-suite convergente de $(u_n)$ qui converge
vers $u$.

Soit $\epsilon > 0$, on sait que
\begin{itemize}
  \item $\exists m_1$ tel que $\forall j, k \geq m_1$,
    \[ d(u_j, u_k) \leq \frac{\epsilon}{2}; \]
  \item $\exists m_2$ tel que $\forall n_j \geq m_2$,
    \[ d(u_{n_j}, u) \leq \frac{\epsilon}{2}. \]
\end{itemize}
Soit $m = \max(m_1, m_2)$, $\forall n \geq m$,
soit $j$ tel que $n_j \geq n$, on a
\begin{align*}
  d(u_n, u) & \leq d(u_n, u_{n_j}) + d(u_{n_j}, u)\\
            & \leq \frac{\epsilon}{2} + \frac{\epsilon}{2}.
\end{align*}
$(u_n)$ converge donc vers $u$.

\subsection{Proposition 2.7}
\begin{description}
  \item[si]
    Si $S$ est précompact et complet,
    soit une suite $(u_n) \subseteq S$.
    Comme $S$ est précompact, on sait qu'il existe
    une sous-suite $(u_{n_j})$ qui est de Cauchy.
    Comme $S$ est complet, cette sous-suite converge.
    $(u_n)$ a donc une sous-suite convergente.
    Comme $(u_n)$ est quelconque, $S$ est compact.
  \item[seulement si]
    Si $S$ est compact, il est également précompact car
    toute suite contient une sous-suite convergente,
    cette dernière est donc aussi de Cauchy.
    Soit une suite $(u_n) \subseteq S$ de Cauchy.
    Comme $S$ est compact, $(u_n)$ possède une sous-suite convergente.
    Par la proposition~2.6, $(u_n)$ converge dans $S$.
    Comme $(u_n)$ est quelconque, $S$ est complet.
\end{description}

\subsection{Proposition 2.15}
\begin{description}
  \item[si]
    Si $S$ est complet,
    soit une suite $(u_n) \subseteq S$ convergente dans $X$.
    Comme elle converge, elle est de Cauchy.
    Comme $S$ est complet, elle converge dans $S$.
    Par la proposition~2.11, ça signifie que $S$ est fermé.
  \item[seulement si]
    Si $S$ est fermé,
    soit une suite $(u_n) \subseteq S$ de Cauchy.
    Comme $X$ est complet, $(u_n)$ converge dans $X$ vers $u$.
    Par la proposition~2.11, on sait que $u$ appartient à $S$ car il est fermé.
    $(u_n)$ est donc convergente dans $S$ qui est par conséquent complet.
\end{description}

\subsection{Proposition 9.2}

\section{Utilisation des théorèmes}
Pour permettre d'avoir un exemple d'application des propositions et théorèmes de \cite{willem2008principes}
ou un aperçu de son utilité, la \tabref{use} reprend leur utilisation.
\begin{table}
  \centering
  \begin{tabular}{|l|l||l|l|}
    \hline
    Théorème & Proposition & Preuve & Exercice\\
    \hline
    1.13 &      & 9.5  & 1.5\\
    \hline
         & 2.3  & 10.7 &\\
    \hline
         & 2.6  & 2.7  &\\
    \hline
    2.8  &      &      & 1.3, 2.1\\
    \hline
         & 2.11 & 2.15 & 1.6, 3.1c, 3.1d, 3.2\\
    \hline
    2.13 &      & 10.6 & 3.1d, 3.2\\
    \hline
         & 2.15 &      & 1.3b, 3.2\\
    \hline
         & 3.3  & 5.2  &\\
    \hline
    4.2  &      & 5.2  &\\
    \hline
    6.6  &      & 6.7  &\\
    \hline
    6.7  &      & 6.8, 6.9c  &\\
    \hline
         & 6.10 & 6.11, 6.22 &\\
    \hline
         & 6.11 &      & 6.2\\
    \hline
         & 6.14 & 6.22 &\\
    \hline
    6.15 &      & 6.23, 6.26, 6.27 & 7.3\\ % Levi
    \hline
    6.16 &      & 6.17 &\\
    \hline
    6.17 &      & 6.18 & 7.1, 7.2\\
    \hline
    6.18 &      & 6.20 &\\
    \hline
         & 6.20 &      &\\
    \hline
         & 6.21 & 6.22 &\\
    \hline
         & 6.22 & 6.23 &\\
    \hline
         & 6.23 & 6.24 &\\
    \hline
    6.24 &      & 6.26, 6.30 &\\
    \hline
    6.26 &      &      &\\
    \hline
    6.27 &      &      &\\
    \hline
    6.29 &      & 6.30 &\\
    \hline
    6.30 &      & 6.31 &\\
    \hline
    6.31 &      &      &\\
    \hline
         & 9.2  & 10.7 &\\
    \hline
    10.6 &      & 10.7 &\\
    \hline
  \end{tabular}
  \caption{Utilisation des théorèmes et proposition dans une preuve ou un exercice.}
  \label{tab:use}
\end{table}

\biblio

\end{document}
