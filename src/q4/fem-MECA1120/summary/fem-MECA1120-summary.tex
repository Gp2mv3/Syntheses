\documentclass[fr,license=none]{../../../eplsummary}

%\usepackage{MnSymbol}
\makeatletter
\newsavebox{\@brx}
\newcommand{\llangle}[1][]{\savebox{\@brx}{\(\m@th{#1\langle}\)}%
\mathopen{\copy\@brx\kern-0.5\wd\@brx\usebox{\@brx}}}
\newcommand{\rrangle}[1][]{\savebox{\@brx}{\(\m@th{#1\rangle}\)}%
\mathclose{\copy\@brx\kern-0.5\wd\@brx\usebox{\@brx}}}
\makeatother

\usepackage{bm}

\usepackage{pifont}
\newcommand{\cmark}{\ding{51}}%
\newcommand{\xmark}{\ding{55}}%

\usepackage{../../../eplmath}

% widehat pour parent et hat quand = 0 sur Gamma_D
\newcommand{\hO}{\widehat{\Omega}}
\newcommand{\nh}{\widehat{n}}
\newcommand{\x}{\mathbf{x}}
\newcommand{\bxi}{\bm{\xi}}
\newcommand{\X}{\mathbf{X}}
\newcommand{\n}{\mathbf{n}}
\newcommand{\uh}{u^h}
\newcommand{\hu}{\hat{u}}
\newcommand{\huh}{\hat{u}^h}
\newcommand{\U}{\mathbf{\mathcal{U}}}
\newcommand{\Uh}{\mathbf{\mathcal{U}}^h}
\newcommand{\hU}{\hat{\mathbf{\mathcal{U}}}}
\newcommand{\hUh}{\hat{\mathbf{\mathcal{U}}}^h}

\DeclarePairedDelimiterX{\inti}[2]{{\langle}}{{\rangle}}{#1, #2}
\DeclarePairedDelimiterX{\intb}[2]{{\llangle}}{{\rrangle}}{#1, #2}
\DeclarePairedDelimiterX{\intd}[2]{{\llangle}}{{\rrangle_D}}{#1, #2}
\DeclarePairedDelimiterX{\intn}[2]{{\llangle}}{{\rrangle_N}}{#1, #2}

\newcommand{\Pe}{\mathrm{Pe}}
\newcommand{\Peh}{\Pe^h}

\hypertitle[']{Introduction aux méthodes d'éléments finis}{4}{MECA}{1120}
{Benoît Legat\and Léa Paulus\and Gilles Peiffer}
{Vincent Legat}

% TODO : - remove integration (?)
%        - triangles de Turner
%        - conditions ess. inhomogènes
%        - éléments discrets (grains)
%        - résolution de systèmes linéaires
%        - théorie de l'approximation
%        - advection-diffusion

\section{Maillage non structuré}
\subsection{Élément parent}
On divise le domaine compliqué $\Omega$ avec coordonnées $(x,y)$ en une multitude d'éléments plus simples,
e.g. triangles, quadrilatères.
On définit un élément parent $\hO$ dans lequel les coordonnées sont $(\xi,\eta)$, c'est l'élément de référence.
L'indice pour les éléments est typiquement $e$.
Le maillage est composé de $n$ \emph{sommets} et chaque élément de $\nh$ sommets.

$\Omega$ est l'union des éléments $\Omega_e$.
$\Omega_e$ peut s'exprimer comme un isomorphisme de l'élément parent $\hO$.
\begin{align*}
  x(\xi,\eta) & = \sum_{i=1}^{\nh} X_i \phi_i(\xi,\eta)\\
  y(\xi,\eta) & = \sum_{i=1}^{\nh} Y_i \phi_i(\xi,\eta).
\end{align*}

Même si intuitivement, ça parait logique que les $\phi_i$ soient linéaires,
ce n'est pas possible pour des éléments quadrilatères.
Pour ces derniers on a des $\phi_i$ quadratiques mais ils forment toujours une bijection si le quadrilatère $\Omega_e$ est convexe
(ce n'est pas un ``n\oe{}ud papillon'').

\subsection{Interpolation: valeurs nodales et fonctions de forme locales}
On définit des n\oe{}uds dans le maillage et on interpolera une fonction $u$ à l'aide des valeurs qu'elle prend à ces n\oe{}uds.
Typiquement, on définit la position des n\oe{}uds dans l'élément parent et les n\oe{}uds de chaque élément du maillage sont obtenus par isomorphisme de ceux de l'élément parent.
Supposons ici pour simplifier qu'on définisse un n\oe{}ud à chaque sommet.

On approche une fonction $u$ par les \emph{valeurs nodales globales} $U_j = u(\X_j)$ que prend la fonction $u$ à chaque n\oe{}ud
à l'aide de \emph{fonctions de forme globales} $\tau_j$
\begin{equation}
  \label{eq:approx}
  u(\x) \approx u^h(\x) = \sum_{j=1}^n U_j\tau_j(\x).
\end{equation}
On aimerait au moins interpoler exactement aux valeurs nodales, c'est-à-dire $u(\X_j) = u^h(\X_j)$.
On impose donc que $\tau_j(\X_j) = 1$ et $\tau_j(\X_{j'}) = 0$ pour $j' \neq j$.

Tout comme avec les splines cubiques ou les B-splines (qui sont les fonctions de base des splines cubiques),
les fonctions de forme $\tau_j$ n'auront pas d'expression valable pour tout $\x \in \Omega$.
Elle seront définies différemment dans chaque élément.
\begin{itemize}
  \item Dans un élément $\Omega_e$ dont aucun sommet n'est $j$,
    $\tau_j$ s'annule à tous les sommets.
    On y impose donc que $\tau_j$ soit nul.
  \item Dans un élément $\Omega_e$ dont un sommet est $j$,
    $\tau_j$ vaut 1 pour ce sommet et zéro pour les 2 autres.
    $\tau_j$ ne sera donc pas nul dans cet élément mais vaudra $\phi_i^e$ (voir plus loin)
    où $j$ est le sommet $i \in \{1,2,3\}$ de l'élément $\Omega_e$.
\end{itemize}
On donnera typiquement aux $\phi_i^e$ une expression polynomiale.
Le \emph{support} d'une fonction est l'ensemble des valeurs où elle est non nulle.
Le support d'une fonction $\tau_j$ est l'union des éléments $\Omega_e$ où $j$ est un des sommets.
Les $\tau_j$ sont donc des fonctions polynomiales par morceau à support compact.


$u^h(\x)$ pour $\x \in \Omega_e$ est une interpolation des valeurs $U_i$ des sommets de l'élément.
Tout comme avec l'interpolation avec les fonctions de Lagrange,
cette interpolation se fait à l'aide des \emph{fonctions de forme locales} $\phi_i^e$.
\begin{align*}
  u^h(\x) & = \sum_{i=1}^{\nh} U_i \phi_i^e(\x)\\
          & = \sum_{i=1}^{\nh} U_i \phi_i(\bxi(\x))
\end{align*}
pour $i \in \Omega_e$.
Notons qu'on a $\x(\bxi)$ mais pas $\bxi(\x)$.
Heureusement, nous n'aurons jamais besoin de calculer la relation inverse $\bxi(\x)$ \cite[p.~10,13]{legat2018fem}.

Si en faisant varier les $U_i$, on sait avoir tous les polynômes de degré $p$ ou moins,
c'est à dire que les $\phi_i$ forment une famille génératrice de l'ensemble des polynômes de degré $p$ ou moins,
on dit que que les fonctions de formes sont complètes à l'ordre $p$.
Pour cela, il faut déjà que tous les monômes (e.g. $1, \xi, \eta, \xi^2, \xi\eta, \eta^2, \dots$) de degré $p$ ou moins
se retrouvent dans au moins un $\phi_i$ (voir le triangle de Pascal pour les monômes en dimension 2).

Il est aussi adéquat de regarder si on sait former tous les polynômes d'ordre $p$ en $\x$.
C'est à dire, l'ensemble des polynômes en $\bxi$ qu'on sait obtenir en remplaçant la valeur de ce $\x$ par $\x(\bxi)$.
On remarque que pour un élément quadrilatère, l'élément lagrangien à 9 n\oe{}uds et l'élément de Serendip à 8 n\oe{}uds sont tous les deux complets à l'ordre 2
mais seulement le premier est complet à l'ordre 2 en $\x$ \cite[p.~15]{legat2018fem}.

\begin{myrem}
  On distingue les sommets qui sont des éléments du maillage aux n\oe{}uds qui sont des points
  où est définie une valeur nodale de l'approximation~\cite[p.~5,8]{legat2018fem}.

  De la même manière,
  il est important de ne pas confondre les fonctions utilisées
  pour la correspondance de $x,y$ avec $\xi,\eta$ et les fonctions de forme locales utilisées pour l'interpolation de $u$.

  Les premières dépendent uniquement du maillage et des sommets,
  elles servent juste à faire un isomorphisme entre les éléments et l'élément parent.
  Les deuxièmes dépendent des n\oe{}uds, elles servent à approximer $u$ en l'interpolant aux valeurs nodales,
  il y a un intérêt à augmenter le nombre de n\oe{}uds et donc le nombre de fonctions de forme locales et leur degré
  pour affiner l'interpolation ainsi que de vérifier que les fonction de forme soient complètes.

  Elles sont souvent identiques (lorsqu'on décide de mettre un n\oe{}ud à chaque sommet, mais ce n'est pas imposé).

  Elles doivent toutes les deux satisfaire la condition
  $\phi_i(\X_i) = 1$ et $\phi_i(\X_{i'}) = 0$ pour $i \neq i'$. mais pour l'une c'est à chaque sommet et pour l'autre à chaque n\oe{}ud.

  En passant, on remarque que
  la condition $\sum_{i=1}^3 \phi_i(\x_i) = 1$ $\forall \x_i \in \Omega_e$ s'interprète différemment dans les deux cas.
  Pour les fonction utilisées pour l'isomorphisme, il faut l'imposer car elle garantit que
  $\x$ est une combinaison convexe des $\X_i$.
  C'est important car $\Omega_e$ est l'enveloppe convexe des sommets $X_i$.
  Si on avait $\sum_{i=1}^3 \phi_i(\x_i) \neq 1$,
  certains $\x$ pourraient sortir du triangle.
  Pour les fonction de forme locales par contre, la condition $\sum_{i=1}^3 \phi_i(\x_i) = 1$
  assure juste qu'on interpole parfaitement les fonctions $u$ constantes.
\end{myrem}

\subsection{Intégration}

% (???) TODO : remove this
On peut estimer l'intégrale d'une fonction dans un élément $\Omega_e$ à l'aide de la somme des valeurs
que prend la fonction à différents points pondérés par des poids.
Supposons qu'on utilise $p$ points.
On doit choisir où se situent ces $p$ points ainsi que le poids.
En $N$ dimensions, ça fait $(N+1)p$ degrés de libertés.

Dans la méthode de Gauss-Legendre, on impose un degré de précision $d$.
C'est à dire qu'on veut intégrer exactement tous les polynômes de degré $d$ ou moins.
Comme l'intégrale et notre approximation sont linéaires en la fonction à intégrer,
on peut tout aussi bien imposer qu'on intègre parfaitement tous les monômes,
i.e. $1, \xi, \eta, \zeta, \xi^2, \xi\eta, \dots$
En effet, les monômes forment la base canonique de l'espace des polynômes.

On dit qu'un monôme est de degré $d$ si la somme des exposants des différentes variables est $d$.
Par exemple, $\xi^3\eta^2\zeta^3$ est de degré $3+2+3 = 8$.
Un polynôme de degré $d$ est une combinaison linéaire de monômes de degré $d$ ou moins avec au moins un monôme de degré $d$.
La dimension de l'ensemble des polynômes de degré $d$ est égal au nombre de monômes différents.
Il est d'ailleurs assez cocasse de remarquer que le nombre de monômes de degré $d$ ou moins à $N$ variables est \emph{égal}
au nombre de monômes de degré $d$ à $N+1$ variables (on obtient une bijection en ajoutant une variable avec comme exposant ce qu'il manque pour faire $d$).
En se souvenant du cours FSAB1101, on voit directement que le nombre de monômes de degré $d$ à $N+1$ variables est $B^*(N+1,d) = {N+d \choose d}$.
Pour avoir un degré de précision $d$ en dimension $N$, il faut donc intégrer parfaitement ${N+d \choose d} = {N+d \choose N}$ monômes.

Regardons ce que ça donne en dimension 2.
On a $3p$ degrés de liberté, et ${2+d \choose 2} = (d+1)(d+2)/2 = 1+2+3+\cdots+d+(d+1)$ contraintes.
Pour avoir un degré de précision $d$ avec Gauss-Legendre il faut donc choisir un nombre de points $p$ tel que (voir \tabref{gl2})
\[ \frac{(d+1)(d+2)}{2} \leq 3p. \]
\begin{table}
  \centering
  \begin{tabular}{cccc}
    $d$ & $(d+1)(d+2)/2$ & $p$ minimum & $3p \stackrel{?}{=} (d+1)(d+2)/2$\\
    1  & 1  & 1\\
    2  & 3  & 1 & \cmark\\
    3  & 6  & 2 & \cmark\\
    4  & 10 & 4\\
    5  & 15 & 5 & \cmark\\
    6  & 21 & 7 & \cmark\\
    7  & 28 & 10\\
    8  & 36 & 12 & \cmark\\
    9  & 45 & 15 & \cmark\\
    10 & 55 & 19\\
  \end{tabular}
  \caption{Nombre minimum de points pour Gauss-Legendre en 2 dimensions.}
  \label{tab:gl2}
\end{table}

On définit une règle d'intégration en donnant les poids et les coordonnées $\xi, \eta$ des $p$ points dans l'élément parent.
Pour intégrer dans un élément $\Omega_e$, on se ramène à l'élément parent à l'aide de
\[ \int_{\Omega_e} f(x,y) \dif x \dif y = \int_{\hO} f(x(\xi,\eta), y(\xi, \eta)) \abs{J_e(\bxi)} \dif \xi \dif \eta. \]
où $J_e(\bxi)$ est donné par
\[
  J_e(\bxi) = \det \begin{bmatrix}
  \fpart{x}{\xi}(\xi,\eta) & \fpart{x}{\eta}(\xi,\eta)\\
  \fpart{y}{\xi}(\xi,\eta) & \fpart{y}{\eta}(\xi,\eta)
\end{bmatrix}
\]

\begin{myrem}
     Il est important de noter qu'en toute généralité, ce jacobien n'est pas
     constant.
\end{myrem}

\begin{myrem}
     Le degré de précision d'une méthode porte sur $fJ_e$, pas uniquement sur
     $f$!
\end{myrem}


\section{Éléments finis pour des problèmes elliptiques}
\subsection{Différentes formulations}
Soit $\U_s, \U, \hU$ trois espaces de fonctions où
les fonctions de $\U_s$ sont deux fois différentiables,
les fonctions $v \in  \U$ sont telles que $v(\x) = t(\x)$ $\forall x \in \Gamma_D$ et
les fonctions $v \in \hU$ sont telles que $v(\x) = 0$ $\forall x \in \Gamma_D$.

On pourrait choisir que $\U = \{v \in \U_s \suchthat v(\x) = t(\x), \forall \x \in \Gamma_D\}$ et $\hU = \{v \in \U_s \suchthat v(\x) = 0, \forall \x \in \Gamma_D\}$ mais on ajoutera souvent d'autres restrictions~\cite[p.~22]{legat2018fem}.

\begin{myprob}[Formulation forte d'un problème elliptique]
  \label{prob:strong}
  Trouver $u(\x) \in \U_s$ tel que
  \begin{equation*}
       \left\{
  \begin{array}{rcl@{\qquad}l}
    \grad \cdot (a \grad u) + f & = & 0, & \forall \x \in \Omega,\\
    \n \cdot (a \grad u) & = & g,        & \forall \x \in \Gamma_N,\\
    u & = & t,                           & \forall \x \in \Gamma_D,
  \end{array}
  \right.
  \end{equation*}
  où $a$ est strictement positif~\cite[pp.~24--25]{legat2018fem}.
\end{myprob}

On introduit les 4 produits scalaires suivants%
\footnote{Le $\cdot$ de la première définition est un produit scalaire pour si jamais $f$ et $g$ sont des vecteurs (car on a pris un gradient par exemple comme avec $a$ défini juste après).}
\begin{align*}
  \inti{f}{g} & = \int_\Omega f \cdot g \dif \Omega\\
  \intb{f}{g} & = \int_{\partial\Omega} f g \dif s\\
  \intn{f}{g} & = \int_{\Gamma_N} f g \dif s\\
  \intd{f}{g} & = \int_{\Gamma_D} f g \dif s
\end{align*}

et les formes suivantes (ce sont des fonctions de fonction!)
\begin{align*}
  a(v,w) & \eqdef \inti{\grad v}{a \grad w}\\
  b(v) & \eqdef \inti{f}{v} + \intn{g}{v}\\
  J(v) & \eqdef \frac{1}{2} a(v,v) - b(v),
\end{align*}
pour $v,w \in \U$.

\begin{myprob}[Formulation faible d'un problème elliptique]
  \label{prob:weak}
  Trouver $u(\x) \in \U$ tel que
  \begin{align*}
    a(\hu, u) & = b(\hu), & \forall \hu \in \hU.\\
  \end{align*}
\end{myprob}

On parle de conditions essentielles et naturelles pour les conditions de
Dirichlet et de Neumann. Les conditions de Dirichlet sont dites
``essentielles'' car elles sont imposées a priori dans l'espace fonctionnel,
alors que les conditions de Neumann sont ``naturelles'' car elles peuvent être
incluses naturellement dans l'écriture du principe variationnel.

\begin{myprob}[Problème de minimisation d'un problème elliptique]
  \label{prob:min}
  Trouver $u(\x) \in \U$ tel que
  \begin{align*}
    J(u) & = \min_{v \in \U} J(v)
  \end{align*}
  c'est-à-dire trouver une fonction $u \in \U$ qui minimise $J$.
\end{myprob}

Le Problème~\ref{prob:weak} est une formulation faible du Problème~\ref{prob:strong}.
C'est-à-dire que toute solution du Problème~\ref{prob:strong} est également solution du Problème~\ref{prob:weak}.
Le Problème~\ref{prob:weak} et le Problème~\ref{prob:min} sont des formulations équivalentes si $a$ est symétrique, ce qui est le cas pour un problème elliptique; cependant, le Problème~\ref{prob:min} ne pourra pas être utilisé lorsqu'il y a un terme $\grad u$ dans l'EDP.

\subsection{Approximation}
En plus de travailler avec une formulation faible du problème de départ,
on va travailler avec un modèle approximé à l'aide de l'approximation du $u$ donnée en \eqref{eq:approx}.
Le Problème~\ref{prob:weak} s'appelle alors méthode de Galerkin et le Problème~\ref{prob:min} la méthode de Ritz. Cependant, comme ces formulations sont équivalentes, on peut parler aussi de \emph{formulation discrète}.

Dans nos deux formulations, on remplace $\U$ par $\Uh$ et $\hU$ par $\hUh$.

On a donc
\begin{myprob}[Formulation discrète]
     \label{prob:disc}
     Trouver $u^h(\x) \in \Uh$ tel que

     \begin{equation*}
          \inti{\grad \huh}{a \grad \uh} = \inti{\huh}{f} + \intn{\huh}{g}, \qquad \forall \huh \in \Uh,
     \end{equation*}

     ou de manière équivalente

     \begin{equation*}
          J\left(\uh \right) = \min_{v^h \in \Uh} \left(\frac{1}{2} a\left(v^h,v^h\right) - b\left(v^h\right)\right).
     \end{equation*}
\end{myprob}

Soit $I$ l'ensemble des n\oe{}uds qui ne sont pas à la frontière où on impose une condition de Dirichlet, i.e. $\X_i \notin \Gamma_D$.
On remarque avec \eqref{eq:approx} que $\Uh$ est engendré par la base $\tau_1, \ldots, \tau_n$
mais pour $v \in \hUh$, $V_i = 0$ pour $i \in \Gamma_D$.
Dans la base de $\hUh$, il n'y a donc que les $\tau_i$ pour $i \in I$.

La condition du Problème~\ref{prob:weak} devient
\begin{equation*}
\renewcommand{\arraystretch}{1.4}
\begin{array}{rcl@{\qquad}l}
  a(\huh, \uh) & = & b(\huh), & \forall \huh \in \hUh\\
  \inti{\grad \huh}{a \grad \uh} & = & \inti{f}{\huh} + \intn{g}{\huh}, & \forall \huh \in \hUh\\
  \inti{\grad \tau_i}{a \sum_{j=1}^n U_j \tau_j} & = & \inti{f}{\tau_i} + \intn{g}{\tau_i}, & \forall i \in I\\
  \sum_{j = 1}^n \inti{\grad \tau_i}{a \grad \tau_j} U_j & = & \inti{f}{\tau_i} + \intn{g}{\tau_i}, & \forall i \in I\\
  \sum_{j = 1}^n a(\grad \tau_i, \grad \tau_j) U_j & = & b(\tau_i), & \forall i \in I.
\end{array}
\end{equation*}
Pour passer de la 2\ieme{} à la 3\ieme{} ligne, on a utilisé la linéarité de l'équation en $\huh$ pour dire que la condition est satisfaite
pour tout $\huh$ si elle est satisfaite pour tout élément de la base de $\hUh$.
Si on ajoute les conditions $U_j = t(\X_j)$ pour les n\oe{}uds $j \notin I$, on obtient un système linéaire de $n$ équations et $n$ inconnues $U_j$.

Soit $U$ le vecteur des inconnues et $A$ et $b$ tels que le système linéaire est $AU = b$.

On voit que $A_{ij} = a(\grad\tau_i, \grad \tau_j)$ si $i \in I$ et $A_{ij} = \delta_{ij}$ sinon.
Notons que si $i$ et $j$ ne sont pas les n\oe{}uds d'un même élément, le support de $\tau_i$ et $\tau_j$ n'a aucune intersection et $a(\tau_i, \tau_j) = 0$.
Sinon on utilise
\[ a(\grad\tau_i, \grad\tau_j) = \sum_{\Omega_e} a(\grad \phi_{i'}^e, \grad \phi_{j'}^e)  \]
où on somme sur tous les éléments $\Omega_e$ qui ont le n\oe{}ud $i$ et le n\oe{}ud $j$.
On leur donne respectivement les indices locaux $i'$ et $j'$.

De la même manière, $b_i = b(\tau_i)$ si $i \in I$ et $b_i = t(\X_i)$ sinon.

En faisant le raisonnement pour le Problème~\ref{prob:min}, on obtient le même système à résoudre.

La matrice de raideur $A$ et le vecteur des forces nodales $B$ sont donc définis
par

\begin{align*}
     A_{ij} &= \inti{\grad \tau_i}{\grad \tau_j},\\
     B_i &= \inti{f}{\tau_i} + \intn{\tau_i}{g}.
\end{align*}

On peut aussi obtenir ce résultat avec une méthode de résidus pondérés~\cite[pp.~27--29]{legat2018fem}.

\subsubsection{Construction du système algébrique}

\paragraph{En 1D}

En connaissant les fonctions de forme globales $\tau_i$ et la fonction $f$,
nous devons calculer les composantes $A_{ij}$ de la matrice de raideur et $B_i$
du vecteur de forces nodales.

On sait qu'une fonction de forme $\tau_i$ est nulle en hors des éléments
auxquels appartient le n\oe{}ud $i$. Au lieu de calculer les $A_{ij}$ et $B_i$
sur le domaine $\Omega$, il suffit de sommer les matrices de raideur locales
$A_{ij}^e$ pour trouver $A_{ij}$ et les vecteurs de forces nodales locaux
$B_i^e$ pour trouver $B_i$.

\begin{myrem}
     On remarque le grand nombre de coefficients nuls de la matrice $A$. Dans
     la plupart des problèmes, cette matrice est donc creuse, ce qui est
     fondamental pour la résolution du système algébrique.
\end{myrem}

Ces $A_{ij}^e$ et $B_i^e$ sont donnés par

\begin{equation*}
\renewcommand{\arraystretch}{1.5}
\begin{array}{rcccl}
     A_{ij}^e & = & \int_{\Omega_e} \phi^e_{i,x}(x) \phi^e_{j,x}(x) \dif x & =
     & \frac{2}{X_{e+1} - X_e} \int_{-1}^1 \phi_{i,\xi}(\xi) \phi_{j,\xi}(\xi)
     \dif \xi,\\
     B_i^e & = & \int_{\Omega_e} f(x) \phi^e_{i}(x) \dif x & = & \frac{X_{e+1}
     - X_e}{2} \int_{-1}^1 \phi_{i}(\xi) f(x(\xi)) \dif \xi.\\
\end{array}
\end{equation*}

\paragraph{En 2D}

Lorsqu'on essaie de généraliser cette méthode aux éléments bidimensionnels, on
remarque a priori deux aspects gênants dans le calcul des coefficients de la
matrice $A_{ij}^e = \int_{\Omega_e} \grad \phi_i^e \cdot \grad \phi_j^e \dif
\Omega$:

\begin{itemize}
     \item il faut effectuer l'intégration sur un élément aux frontières
     quelconques;
     \item on ne dispose pas d'une expression du gradient des fonctions de
     forme par rapport aux coordonnées.
\end{itemize}

Pour contourner la première difficulté, il suffit de passer à l'élément parent
$\hO$. Il faut alors tenir compte du jacobien de la transformation cependant.

\begin{itemize}
\item \textbf{Matrice jacobienne}
\end{itemize}

Elle est donnée par

\begin{equation*}
     J_e(\bxi) = \det \left( \fpart{\x}{\bxi}(\bxi) \right) = \det
     \begin{bmatrix}
     \fpart{x}{\xi}(\xi,\eta) & \fpart{x}{\eta}(\xi,\eta)\\
     \fpart{y}{\xi}(\xi,\eta) & \fpart{y}{\eta}(\xi,\eta)
     \end{bmatrix}.
\end{equation*}

On trouve donc que

\begin{align*}
\dif \Omega &= \det \left( \fpart{\x}{\bxi} \dif \xi \dif \eta \right)\\
&= J_e \dif \hO.
\end{align*}

\begin{itemize}
\item \textbf{Calcul du gradient des fonctions de forme}
\end{itemize}

On observe que

\[
\fpart{\phi_i}{\x} = \fpart{\phi_i}{\bxi} \cdot \left( \sum_{i=1}^n
\mathbf{X}_i^e \fpart{\phi_i^x}{\bxi}(\bxi) \right)^{-1}.
\]

L'inverse du gradient de la transformation (c'est-à-dire de la somme
vectorielle ci-dessus) est donné par

\[
\fpart{\bxi}{\x} =
\begin{bmatrix}
\fpart{\xi}{x}(\xi,\eta) & \fpart{\xi}{y}(\xi,\eta)\\
\fpart{\eta}{x}(\xi,\eta) & \fpart{\eta}{y}(\xi,\eta)
\end{bmatrix}
= \frac{1}{J_e(\xi,\eta)}
\begin{bmatrix}
\fpart{y}{\eta}(\xi,\eta) & -\fpart{x}{\eta}(\xi,\eta)\\
-\fpart{y}{\xi}(\xi,\eta) & \fpart{x}{\xi}(\xi,\eta)
\end{bmatrix}.
\]

Finalement, on trouve alors ce qu'on recherchait :

\begin{align*}
\fpart{\phi_i}{\x} &= \fpart{\phi_i}{\bxi} \cdot \fpart{\bxi}{\x}\\
\begin{bmatrix}
\fpart{\phi_i}{x}(\xi,\eta)\\
\fpart{\phi_i}{y}(\xi,\eta)
\end{bmatrix}^T
&=
\begin{bmatrix}
\fpart{\phi_i}{\xi}(\xi,\eta)\\
\fpart{\phi_i}{\eta}(\xi,\eta)
\end{bmatrix}^T
\cdot
\begin{bmatrix}
\fpart{\xi}{x}(\xi,\eta) & \fpart{\xi}{y}(\xi,\eta)\\
\fpart{\eta}{x}(\xi,\eta) & \fpart{\eta}{y}(\xi,\eta)
\end{bmatrix}.
\end{align*}

\begin{myrem}
     On voit qu'il n'a pas été nécessaire de construire la transformation
     inverse $\bxi(\x)$.
\end{myrem}

Le calcul des coefficients de la matrice se réduit alors à intégrer sur le
triangle parent la fonction suivante:

\begin{align*}
     A_{ij}^e &= \int_{\Omega_e} \grad \phi_i^e \cdot \grad \phi_j^e \dif
     \Omega\\
     &= \int_{\hO} \left( \fpart{\phi_i}{x}(\xi, \eta) \fpart{\phi_j}{x}(\xi,
     \eta) + \fpart{\phi_i}{y}(\xi, \eta) \fpart{\phi_j}{y}(\xi, \eta) \right)
     J_e(\xi,\eta) \dif \hO.
\end{align*}

Pour ce type d'intégration, on a recours aux techniques de l'intégration
numériqe.

\subsubsection{Triangles de Turner}

% TODO

\subsubsection{Traitement de conditions essentielles inhomogènes}

% TODO

\section{Éléments discrets}

% TODO

\section{Techniques de résolution des systèmes linéaires}

% TODO

\section{Théorie de la meilleure approximation}

% TODO

\subsection{Analyse}
On remarque que $a$ est une forme bilinéaire et symétrique (ce n'aurait pas été le cas s'il y avait eu un terme en $\grad u$ dans l'EDP).
Pour que $a$ soit un produit scalaire, il faut également que $a(\grad v, \grad v) \implies v = 0$ pour $v \in \U$.

On remarque que $a(\grad v, \grad v) = 0 \implies \int_\Omega a(\grad v)^2 \dif \Omega = 0$.
Si $a$ est strictement positif et que les fonction de $\U$ sont continues, ça signifie que $\grad v = 0$.
Seulement, cela n'implique pas que $v = 0$. En effet, souvenons-nous qu'une primitive est définie à une constante près.
Par contre si on impose que pour qu'une fonction $u \in \U$, il faut que $u = 0$ pour tout $\x \in \partial\Omega$,
on a bien que $v$ doit être la fonction nulle.

$a$ est donc un produit scalaire si $\U = \{v \in H^1(\Omega) \mathbin{\mathrm{ et }} v(\x) = 0, \forall \x \in \partial \Omega\}$.
En fait, pour un tel $\U$, $a$ est également coercive.
Comme $a$ est continue et que $b$ est linéaire et continue,
le Théorème de Lax-Milgram permet de garantir que la formulation faible (Problème~\ref{prob:weak} ou Problème~\ref{prob:min}) a une solution unique.

\section{Éléments finis pour des problèmes d'advection-diffusion}

% TODO

\biblio

\end{document}
