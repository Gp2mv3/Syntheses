\documentclass[a4paper, 12pt]{article}
\usepackage[utf8]{inputenc}
\usepackage[french]{babel}
\usepackage{amsmath,amsfonts,amssymb}
\usepackage{tikz}
\usetikzlibrary{babel, scopes, arrows.meta, calc, decorations.pathmorphing, decorations.markings, shapes.geometric}
\usepackage{fullpage}
\usepackage{pgfplots}

%\Star[opt]{r1}{r2}{pos}
\newcommand\Star[4][]{%
\path[#1]  #4 +(0:#2) -- +(45:#3) -- +(90:#2) -- +(135:#3) -- +(180:#2) -- +(225:#3) -- +(270:#2) -- +(315:#3) -- cycle;}

%polstar[opt]{radius}{position}
\newcommand{\polstar}[3][]{
\Star[fill=black,#1]{1*#2}{0.28*#2}{#3}
\Star[#1, fill=white]{0.8*#2}{0.08*#2}{#3}
}
% Syntax: \centerarc[thick] draw options] (x,y) (r, a1, a2) !spaces are important&
\def \centerarc[#1] (#2) (#3,#4,#5)
{\draw[#1] (#2) + ({#3*cos(#4)},{#3*sin(#4)}) arc (#4:#5:#3);}

%\base[type_de_ligne]{texte après}
\newcommand{\base}[2][]{
\draw (0,0) node {\tiny{$\bullet$}};
\draw [thick, .-latex, #1] (0,0) -- (2,0) node[below]{$\vec{e_{x}}$#2};
\draw [thick, .-latex, #1] (0,0) -- (0,2) node[right]{$\vec{e_{y}}$#2};
}

%\bigarrow[options]{scale}{loc}
\newcommand{\bigarrow}[4][]{
  \draw [#1] (#3,#4+0.35*#2)-|(1.2*#2+#3,#4+#2)--(2.5*#2+#3,#4+0)--(1.2*#2+#3,#4-#2)|-(#3,#4-0.35*#2)--cycle;
}

\definecolor{dgreen}{RGB}{0,150,0}
\definecolor{lblue}{RGB}{100,100,255}


\newcommand{\eye}[5][]% size, x, y, rotation
{   \draw[rotate around={#5:(#3,#4)}, #1] (#3,#4) -- ++(-.5*55:#2) (#3,#4) -- ++(.5*55:#2);
    \draw[#1] (#3,#4) ++(#5+40:.75*#2) arc (#5+40:#5-40:.75*#2);
    % IRIS
    \draw[fill=blue] (#3,#4) ++(#5+55/3:.75*#2) arc (#5+180-55:#5+180+55:.28*#2);
    %PUPIL, a filled arc
    \draw[fill=black] (#3,#4) ++(#5+55/3:.75*#2) arc (#5+55/3:#5-55/3:.75*#2);
}


%\rgb{r}{g}{b} (de 0 à 255)
\newcommand{\rgb}[3]{rgb,255:red,#1; green,#2; blue,#3}



%\tbase[options]{xi}{additional text}
\newcommand{\tbase}[3][]{
\draw [thick, .-latex, #1] (0,0) -- ({2*(cosh(#2)}, {2*sinh(#2))}) node[below]{$\vec{e_{x}}$#3};
\draw [thick, .-latex, #1] (0,0) -- ({2*(sinh(#2)}, {2*cosh(#2))}) node[right]{$\vec{e_{y}}$#3};
}



\begin{document}



%base
\begin{tikzpicture}[scale=1.5]
  %\node [black] at (0,0) {\Large{.}};
  \base[]{}
  \centerarc[thick, -latex] (0,0) (0.7,0,30)
  \centerarc[thick, -latex] (0,0)  (0.7,90,120)
  \node [right] at (0.7,0.25) {$\theta$};
  \begin{scope}[rotate=30]
    \base[dashed]{'}
  \end{scope}

  %\draw [#1](a.#4) arc (#4:{#4+#5}:#3);
  %\draw [thick, .-latex] (0,0)  arc (90:200:1);
\end{tikzpicture}

\begin{tikzpicture}[scale=1.5]
  \base{}
  \tbase[dashed]{0.5}{'}
\end{tikzpicture}

\begin{tikzpicture}[scale=0.75]
  %eau de la rivière
  \draw [thick, fill=\rgb{200}{200}{255}, color=\rgb{200}{200}{255}] decorate [decoration={snake}] {(14,4) -- (14,0)} -- (0,4) decorate [decoration={snake}] {(0,4) -- (0,0)} -- (14,0);
  \path [fill=\rgb{200}{200}{255}] (0.2,0) -- (13.8,0) -- (13.8,4) -- (0.2,4);
  %grosses flèches
  \bigarrow[color=\rgb{150}{150}{255}, fill=\rgb{150}{150}{255}, rounded corners=1mm]{1}{1}{2}
  \draw [color=\rgb{50}{50}{210}] node at (2,2) {$\vec{v_0}$};

  \bigarrow[color=\rgb{150}{150}{255}, fill=\rgb{150}{150}{255}, rounded corners=1mm]{1}{10}{2}
  \draw [color=\rgb{50}{50}{210}] node at (11,2) {$\vec{v_0}$};

  %les berges noires
  \draw (0,0) -- (14,0);
  \draw (0,4) -- (14,4);

  %double flehce cotation evec le "L"
  \draw [|<->|] (-1,0) -- (-1,4);
  \draw node [left] at (-1,2) {$L$};

  %flèches des parcours
  \draw [latex-latex, ultra thick, black] (5,0) -- (5,4);
  \draw [latex-latex, ultra thick, color=\rgb{0}{0}{0}] (5,0) -- (9,0);

  %labels des parcours
  \draw node [rotate=90, left, anchor=south, black] at (5,2) {Parcours $A$};
  \draw node [above, color=\rgb{0}{0}{0}] at (7,0) {Parcours $B$};

  %triangle à droite
  \draw [thick] (16,0) |- (18,4) --cycle;
  \draw node [above] at (17,4) {$v_0$};
  \draw node [left] at (16,2) {$v_a$};
  \draw node [right] at (17,2) {$c$};
\end{tikzpicture}
\vspace{5cm}

\begin{tikzpicture}[scale=0.4]
  \draw [thick] (0,5) -- (0,0)  (1,0) -- (1,5);
  \draw [thick] (.5,5) ellipse (0.5 and 0.18);
  \draw [thick] (0,0) arc (180:360:0.5 and 0.18);
  \draw [white] node at (0.5,5.2) {\tiny{$\bullet$}};
  \draw [dotted] (.5,10)--(.5,5);
  \polstar[rounded corners=0.64mm]{0.8}{(.5,10)}


  \begin{scope}[shift={(8,0)}]
    \begin{scope}[rotate=-30]
      \draw [thick] (0,5) -- (0,0)  (1,0) -- (1,5);
      \draw [thick] (.5,5) ellipse (0.5 and 0.18);
      \draw [thick] (0,0) arc (180:360:0.5 and 0.18);
      \draw [white] node at (0.5,5.2) {\tiny{$\bullet$}};
      \draw [dotted] (.5,11.835680518387326)--(.5,5);
      %\polstar[rounded corners=0.64mm]{0.8}{(.5,10.823762841892224)}
    \end{scope}
    \draw [help lines, -latex] (0,9.2) -- (0,5);
    \draw [help lines] (0,5.1) -- (0,0);
    \polstar[rounded corners=0.64mm]{0.8}{(0,10)}
    \polstar[rounded corners=0.64mm, fill=\rgb{180}{180}{180}]{0.8}{(6.35085296,10)}
    \centerarc[help lines] (0,0) (2.5,60,90)
    \node [above] at (70:2.5) {$\alpha$};
    \draw [thick, -latex](0,0)++(-30:1)++(60:2.5)++(0.1,0)--++(2,0) node [above] {$\vec{v_0}$} --++(0.5,0);

    \node [above] at (0,11) {\tiny{Position réelle}};
    \node at (0,11) {\tiny{de l'étoile}};

    \node [above] at (6.35085296,11) {\tiny{Position perçue}};
    \node at (6.35085296,11) {\tiny{de l'étoile}};
  \end{scope}
\end{tikzpicture}



\begin{tikzpicture}
  \draw [thick] (3,0) -- (0,0);
  \node [below] at (1.5,0) {$v_0$};

  \draw [thick] (3,2) -- (0,0);
  \node [anchor=south,rotate=atan(2/3)] at (1.5,1) {$c'$};

  \draw [thick] (3,2) -- (3,0);
  \node [right] at (3,1) {$c$};

  \draw [help lines] (0,0) -- (0,1);
  \centerarc[help lines] (0,0) (0.6,atan(2/3),90)
  \node [above] at (50:0.6) {\small{$\alpha$}};
\end{tikzpicture}


\begin{tikzpicture}
  \draw [thick] (3,0) -- (0,0);
  \node [below] at (1.5,0) {$v_0 \cdot \Delta t'$};

  \draw [thick] (3,2) -- (0,0);
  \node [anchor=south,rotate=atan(2/3)] at (1.5,1) {$c \cdot \Delta t'$};

  \draw [thick] (3,2) -- (3,0);
  \node [right] at (3,1) {$c \cdot \Delta t$};

  \draw [help lines] (0,0) -- (0,1);
  \centerarc[help lines] (0,0) (0.6,atan(2/3),90)
  \node [above] at (50:0.6) {\small{$\alpha$}};
\end{tikzpicture}



\begin{tikzpicture}
  \draw[latex-latex] (0,4) node[left] {$\vec{y}$} -- (0,0) -- (4,0) node[below] {$\vec{x}$};
  \begin{scope}[shift={(7,0)}, scale=3]
    \draw [thick,latex-](0,0) -- (1,0);
    \draw [thick,latex-](1,0) -- (1,1);
    \draw [thick,latex-](1,1) -- (0,1);
    \draw [thick,latex-](0,1) -- (0,0);
    \node [below] at (0.5,0) {$L_x(-\xi_2)$};
    \node [above] at (0.5,1) {$L_x(\xi_2)$};
    \node [left]  at (0,0.5) {$L_y(\xi_1)$};
    \node [right] at (1,0.5) {$L_y(-\xi_1)$};
  \end{scope}
\end{tikzpicture}


\begin{tikzpicture}
  \draw[thick, latex-latex] (0,4) node[left] {$ct$} -- (0,0) -- (4,0) node[below] {$\vec{x}$};
  \draw[dashed, help lines] (0,0) -- (3.4,3.4);
  \draw [Circle-Circle, thick] plot [smooth, tension=0.7] coordinates {(0.4,1.1) (0.6,1.7) (0.45,2.3) (0.5,2.7) (0.8,3.3)};
  \node [below] at (0.4,1.1) {\small{$e_1$}};
  \node [right] at (0.8,3.3) {\small{$e_2$}};

  \draw [very thin, -latex] (0.45,2.3) -- (2.7,2.3) node [right] {Ligne univers};
  \draw [very thin, -latex] (1.2,1.2) -- (2.7,1.2) node [right] {Cône de lunière};
\end{tikzpicture}

\begin{tikzpicture}
  \begin{axis}[
      samples=121, %nbre de points dans courbes parametrees
      xmin=-3,
      xmax=3,
      ymin=-3,
      ymax=3,
      width=10cm, %taille de la figure
      height=10cm,
      disabledatascaling,
      grid=both, %afficher la grille
      %font=\footnotesize, %taille de la police par defaut
      %grid style={line width=.1pt, draw=red},
      %major grid style={line width=.2pt,draw=gray!50},
      %minor tick num=1,
      axis lines=middle,enlargelimits=0.05, %rendre le plot un rien plus grand
      %execute at begin axis={
      execute at end axis={ %cadre autour
        \draw[thick] (rel axis cs:0,0) -- (rel axis cs:1,0) -- (rel axis cs:1,1) -- (rel axis cs:0,1) --cycle;},
      xticklabels={}, %masquer les nombres en x
      yticklabels={},
      ylabel={$ct$},
      xlabel={$\vec{x}$},
      title={Espace de Minkowski $1$D},
      legend pos=south east,
    ]

    \addplot[ultra thick, blue] expression {x};
    \addplot[ultra thick, blue] expression {-x};
    \legend{Région invariante sous Lorentz}

    \addplot [help lines, domain=-1.9:1.9] ({cosh(x)}, {sinh(x)});
    \addplot [help lines, domain=-1.9:1.9] ({-cosh(x)}, {sinh(x)});
    \addplot [help lines, domain=-1.9:1.9] ({sinh(x)}, {cosh(x)});
    \addplot [help lines, domain=-1.9:1.9] ({sinh(x)}, {-cosh(x)});
  \end{axis}
\end{tikzpicture}



lol



\begin{tikzpicture}
  \begin{axis}[
      samples=121,
      xmin=-0.4,
      xmax=2.,
      ymin=-0.4,
      ymax=2.,
      width=10cm,
      height=10cm,
      grid=both,
      disabledatascaling,
      %font=\footnotesize,
      %grid style={line width=.1pt, draw=red},
      %major grid style={line width=.2pt,draw=gray!50},
      %minor tick num=1,
      axis lines=middle,enlargelimits=0.07,
      %execute at begin axis={
      execute at end axis={
        \draw[thick] (rel axis cs:0,0) -- (rel axis cs:1,0) -- (rel axis cs:1,1) -- (rel axis cs:0,1) --cycle;},
      xticklabels={},
      yticklabels={},
      xtick={-10,-9,...,10},
      ytick={-10,-9,...,10},
      ylabel={$ct$},
      xlabel={$\vec{x}$},
      title={Espace de Minkowski $1$D},
      legend pos=south east,
    ]

    \addplot[help lines] expression {x};
    \addplot[help lines] expression {-x};

    \addplot [help lines, domain=-1.9:1.9] ({cosh(x)}, {sinh(x)});
    \addplot [help lines, domain=-1.9:1.9] ({-cosh(x)}, {sinh(x)});
    \addplot [help lines, domain=-1.9:1.9] ({sinh(x)}, {cosh(x)});
    \addplot [help lines, domain=-1.9:1.9] ({sinh(x)}, {-cosh(x)});
    %\addlegendentry{$x^2$}

    \pgfmathsetmacro{\i}{sinh(0.6)}
    \pgfmathsetmacro{\j}{cosh(0.6)}

    \draw [dgreen, thick] (axis cs:0,0) -- (axis cs:2*\i,2*\j);
    \draw [dgreen, thick] (axis cs:0,0) -- (axis cs:2*\j,2*\i);

    \draw[ultra thick, -latex] (axis cs:0,0) -- (axis cs:0,1);
    \draw[ultra thick, -latex, red] (axis cs:0,0) -- (axis cs:\i,\j);

    \node [right] at (axis cs:1.76*\i, 1.76*\j) {$ct'$};
    \node [above] at (axis cs:1.76*\j, 1.76*\i) {$\vec{x}'$};

    %\pgfmathparse{10*0.05}
  \end{axis}
\end{tikzpicture}





\begin{tikzpicture}
  \begin{axis}[
      samples=121,
      xmin=-.25,
      xmax=2.,
      ymin=-0.2,
      ymax=2.5,
      width=9cm,
      height=10.8cm,
      disabledatascaling,
      %grid=both,
      %font=\footnotesize,
      %grid style={line width=.1pt, draw=red},
      %major grid style={line width=.2pt,draw=gray!50},
      %minor tick num=1,
      axis lines=none,enlargelimits=0.05,
      %execute at begin axis={
      %execute at end axis={        \draw[thick] (rel axis cs:0,0) -- (rel axis cs:1,0) -- (rel axis cs:1,1) -- (rel axis cs:0,1) --cycle;},
      xticklabels={},
      yticklabels={},
      xtick={-10,-9,...,10},
      ytick={-10,-9,...,10},
      ylabel={$ct$},
      xlabel={$\vec{x}$},
      %title={Espace de Minkowski $1$D},
      legend pos=south east,
    ]

    \addplot [color=blue, domain=-1.35:1.35] ({sinh(x)}, {cosh(x)});
    %\addlegendentry{$x^2$}

    \pgfmathsetmacro{\i}{sinh(0.7)}
    \pgfmathsetmacro{\j}{cosh(0.7)}
    \pgfmathsetmacro{\alxi}{atan(tanh(0.7))}

    \draw [dashed, help lines] (0.1,-0.1) -- (-2,2) (-0.1,-0.1) -- (2,2);
    \draw [red, thick, latex-latex] (0,2.3) -- (0,0) -- (2,0);
    \draw [dgreen, thick, -latex] (0,0) -- (1.7*\i,1.7*\j);
    \draw [dgreen, thick, -latex] (0,0) -- (1.7*\j,1.7*\i);


    \node [right] at (1.7*\i, 1.7*\j) {$ct'$};
    \node [above] at (1.7*\j, 1.7*\i) {$\vec{x}'$};
    \node [right] at (0,2.3) {$ct$};
    \node [above] at (2,0) {$\vec{x}$};

    \draw [dashed, thick] (0,1) node {$\bullet$} node [below left] {$A'$} -- (\i/\j,1) node {$\bullet$} node [right] {$K$};
    \draw (0,.6) node {$\bullet$} node [left] {$A_1'$} -- (0.6*\i/\j,.6) node {$\bullet$} node [above left] {$A_1$} -- (1.8,0.6) node {$\bullet$} node [right] {$e_1'$};
    \draw (0,1.7) node {$\bullet$} node [left] {$A_2'$} -- (1.7*\i/\j,1.7) node {$\bullet$} node [above left] {$A_2$} -- (1.8,1.7) node {$\bullet$} node [right] {$e_2'$};

    \draw [help lines] (0.33,0) arc (0:\alxi:0.33);
    \node [right] at (0.33,0.1) {$\alpha$};
    \draw (0,0) node [below] {$O$} node {$\bullet$};
    \draw (\i,\j) node [above left] {$A$} node {$\bullet$};

    %\pgfmathparse{10*0.05}
  \end{axis}
\end{tikzpicture}



\begin{tikzpicture}
  \begin{axis}[
      samples=121,
      xmin=-0.4,
      xmax=2.,
      ymin=-0.4,
      ymax=2.,
      width=10cm,
      height=10cm,
      grid=both,
      disabledatascaling,
      %font=\footnotesize,
      %grid style={line width=.1pt, draw=red},
      %major grid style={line width=.2pt,draw=gray!50},
      %minor tick num=1,
      axis lines=middle,enlargelimits=0.07,
      %execute at begin axis={
      execute at end axis={
        \draw[thick] (rel axis cs:0,0) -- (rel axis cs:1,0) -- (rel axis cs:1,1) -- (rel axis cs:0,1) --cycle;},
      xticklabels={},
      yticklabels={},
      xtick={-10,-9,...,10},
      ytick={-10,-9,...,10},
      ylabel={$ct$},
      xlabel={$\vec{x}$},
      title={Espace de Minkowski $1$D},
      legend pos=south east,
    ]

    \addplot[help lines] expression {x};
    \addplot[help lines] expression {-x};

    \addplot [help lines, domain=-1.9:1.9] ({cosh(x)}, {sinh(x)});
    \addplot [help lines, domain=-1.9:1.9] ({-cosh(x)}, {sinh(x)});
    \addplot [help lines, domain=-1.9:1.9] ({sinh(x)}, {cosh(x)});
    \addplot [help lines, domain=-1.9:1.9] ({sinh(x)}, {-cosh(x)});
    %\addlegendentry{$x^2$}

    \pgfmathsetmacro{\i}{sinh(0.6)}
    \pgfmathsetmacro{\j}{cosh(0.6)}
    \pgfmathsetmacro{\d}{sqrt(\i^2 + \j^2)}
    \pgfmathsetmacro{\alxi}{atan(tanh(0.6))}

    \draw [dgreen, thick] (0,0) -- (2*\i,2*\j);
    \draw [dgreen, thick] (0,0) -- (2*\j,2*\i);

    \draw[ultra thick, -latex] (0,0) -- (.5,0) -- (1,0);
    \draw[thin, |<->|] (0,-.1) -- (.5,-.1) node[fill=white, inner sep=0pt] {1} -- (1,-.1);

    \draw[ultra thick, -latex, red] (0,0) -- (0.5*\j,0.5*\i) -- (\j,\i);
    \draw[thin, rotate=\alxi, |<->|] (0,.1) -- (.5*\d,.1) node[circle, fill=white, inner sep=0pt] {1} -- (\d,.1);

    \node [right] at (1.76*\i, 1.76*\j) {$ct'$};
    \node [above] at (1.76*\j, 1.76*\i) {$\vec{x}'$};

    %\pgfmathparse{10*0.05}
  \end{axis}
\end{tikzpicture}


blahblahblah






\begin{tikzpicture}
  \begin{axis}[
      samples=121,
      xmin=-0.1,
      xmax=2.4,
      ymin=-0.3,
      ymax=2.4,
      width=10cm,
      height=10.8cm,
      disabledatascaling,
      %grid=both,
      %font=\footnotesize,
      %grid style={line width=.1pt, draw=red},
      %major grid style={line width=.2pt,draw=gray!50},
      %minor tick num=1,
      axis lines=none,enlargelimits=0.05,
      %execute at begin axis={
      %execute at end axis={        \draw[thick] (rel axis cs:0,0) -- (rel axis cs:1,0) -- (rel axis cs:1,1) -- (rel axis cs:0,1) --cycle;},
      xticklabels={},
      yticklabels={},
      xtick={-10,-9,...,10},
      ytick={-10,-9,...,10},
      ylabel={$ct$},
      xlabel={$\vec{x}$},
      %title={Espace de Minkowski $1$D},
      legend pos=south east,
    ]

    \addplot [color=blue, domain=-0.2:1.35] ({cosh(x)}, {sinh(x)});
    %\addlegendentry{$x^2$}

    \pgfmathsetmacro{\i}{sinh(0.7)}
    \pgfmathsetmacro{\j}{cosh(0.7)}
    \pgfmathsetmacro{\alxi}{atan(tanh(0.7))}

    \draw [dashed, help lines] (0.1,-0.1) -- (-2,2) (-0.1,-0.1) -- (2,2);
    \draw [red, thick, latex-latex] (0,2.3) -- (0,0) -- (2,0);
    \draw [dgreen, thick, -latex] (0,0) -- (1.7*\i,1.7*\j);
    \draw [dgreen, thick, -latex] (0,0) -- (1.7*\j,1.7*\i);


    \node [right] at (1.7*\i, 1.7*\j) {$ct'$};
    \node [above] at (1.7*\j, 1.7*\i) {$\vec{x}'$};
    \node [right] at (0,2.3) {$ct$};
    \node [above] at (2,0) {$\vec{x}$};

    \draw (.6,0) node {$\bullet$} node [below] {$A_1$} -- (.6,0.6*\i/\j) node {$\bullet$} node [below right] {$A_1'$} -- (.6,1.8) node {$\bullet$} node [above] {$e_1'$};

    \draw [dashed, thick] (1,0) node {$\bullet$} node [below left] {$A$} -- (1,\i/\j) node {$\bullet$} node [above] {$K'$};

    \draw (1.7,0) node {$\bullet$} node [below] {$A_2$} -- (1.7,1.7*\i/\j) node {$\bullet$} node [below right] {$A_2'$} -- (1.7,1.8) node {$\bullet$} node [above] {$e_2'$};

    \draw [help lines] (0,0.33) arc (90:90-\alxi:0.33);
    \node [above] at (0.1,0.33) {$\alpha$};
    \draw (0,0) node [below] {$O$} node {$\bullet$};
    \draw (\j,\i) node [below right] {$A'$} node {$\bullet$};

  \end{axis}
\end{tikzpicture}


\begin{tikzpicture}[scale=0.75]
  \begin{scope}[shift={(-7,0)}]
    \foreach \i in {0,...,4}{
      \draw[red] (-\i*0.1,0) circle[radius=2-\i*0.35];}

    \draw[-latex] (-.4,0) node[blue]  {$\bullet$} node[right] {\footnotesize{\textbf{E}}} --++(-0.8,0);
    \draw (2,0) node[blue]  {$\bullet$} node[right] {\footnotesize{\textbf{R}}};
    \node[text width=4cm,align=center] at (0,-3.2) {Émetteur s'éloignant du récepteur:\\ $f_r < f_e$};
  \end{scope}

  \begin{scope}
    \foreach \i in {0,...,4}{
      \draw[red] (0,0) circle[radius=2-\i*0.35];}

    \draw (0,0) node[blue]  {$\bullet$} node[left] {\footnotesize{\textbf{E}}};
    \draw (2,0) node[blue]  {$\bullet$} node[right] {\footnotesize{\textbf{R}}};
    \node[text width=4cm,align=center] at (0,-3.2) {Émetteur et récepteur fixes:\\ $f_r = f_e$};
  \end{scope}

  \begin{scope}[shift={(7,0)}]
    \foreach \i in {0,...,4}{
      \draw[red] (\i*0.1,0) circle[radius=2-\i*0.35];}

    \draw[-latex] (0.4,0) node[blue]  {$\bullet$} node[left] {\footnotesize{\textbf{E}}} --++(0.8,0);
    \draw (2,0) node[blue]  {$\bullet$} node[right] {\footnotesize{\textbf{R}}};
    \node[text width=4cm,align=center] at (0,-3.2) {Émetteur se rapprochant du récepteur:\\ $f_r > f_e$};
  \end{scope}
\end{tikzpicture}

\begin{tikzpicture}
  \begin{scope}[shift={(-3,0)}]
    \draw (0,0) node {\tiny{$\bullet$}};
    \draw [thick, latex-latex] (0,2) node[right] {$y$} -- (0,0) -- (2,0) node[below] {$x$};
    \draw[blue, -latex] decorate [decoration={snake}] {(0,0) -- (32:1)}  --++(32:0.7) node[right] {$\left(\dfrac{\omega}{c}, \vec{k}\right)$};
    \draw[help lines, -latex] (1,0) arc (0:32:1);
    \node[right] at (16:1) {$\theta$};
    \node[right] at (.9,2.1) {$\mathcal{R}$};
  \end{scope}

  \begin{scope}[shift={(3,0)}]
    \draw (0,0) node {\tiny{$\bullet$}};
    \draw [thick, latex-latex] (0,2) node[right] {$y'$} -- (0,0) -- (2,0) node[below] {$x'$};
    \draw[blue, -latex] decorate [decoration={snake}] {(28:-0) -- (28:1)}  --++(28:0.7) node[right] {$\left(\dfrac{\omega'}{c}, \vec{k'}\right)$};
    \draw[help lines, -latex] (1,0) arc (0:28:1);
    \node[right] at (14:1) {$\theta'$};
    \node[right] at (.9,2.1) {$\mathcal{R'}$};
  \end{scope}
\end{tikzpicture}


Bonjour coucou hellow lolilol



\begin{tikzpicture}
  \begin{axis}[
      samples=121, %nbre de points dans courbes parametrees
      xmin=-3,
      xmax=3,
      ymin=-3,
      ymax=3,
      width=10cm, %taille de la figure
      height=10cm,
      disabledatascaling,
      grid=both, %afficher la grille
      %font=\footnotesize, %taille de la police par defaut
      %grid style={line width=.1pt, draw=red},
      %major grid style={line width=.2pt,draw=gray!50},
      %minor tick num=1,
      axis lines=middle,enlargelimits=0.05, %rendre le plot un rien plus grand
      %execute at begin axis={
      execute at end axis={ %cadre autour
        \draw[thick] (rel axis cs:0,0) -- (rel axis cs:1,0) -- (rel axis cs:1,1) -- (rel axis cs:0,1) --cycle;},
      xticklabels={}, %masquer les nombres en x
      yticklabels={},
      ylabel={$\dfrac{E}{c}$},
      xlabel={$\vec{p}$},
      title={Espace de Minkowski $1$D},
      legend pos=south east,
    ]

    \addplot[blue] expression {x};
    \addplot[blue] expression {-x};

    \addplot [blue, domain=-1.9:1.9] ({cosh(x)}, {sinh(x)});
    \addplot [blue, domain=-1.9:1.9] ({-cosh(x)}, {sinh(x)});
    \addplot [blue, domain=-1.9:1.9] ({sinh(x)}, {cosh(x)});
    \addplot [blue, domain=-1.9:1.9] ({sinh(x)}, {-cosh(x)});
  \end{axis}
\end{tikzpicture}




\begin{tikzpicture}
  \begin{axis}[
      samples=121, %nbre de points dans courbes parametrees
      xmin=-1,
      xmax=3,
      ymin=-2,
      ymax=2,
      width=10cm, %taille de la figure
      height=10cm,
      disabledatascaling,
      %grid=both, %afficher la grille
      %font=\footnotesize, %taille de la police par defaut
      %grid style={line width=.1pt, draw=red},
      %major grid style={line width=.2pt,draw=gray!50},
      %minor tick num=1,
      axis lines=middle,enlargelimits=0.0, %rendre le plot un rien plus grand
      %execute at begin axis={
      execute at end axis={ %cadre autour
        \draw[thick] (rel axis cs:0,0) -- (rel axis cs:1,0) -- (rel axis cs:1,1) -- (rel axis cs:0,1) --cycle;},
      xticklabels={}, %masquer les nombres en x
      yticklabels={},
      xtick={-10,-9,...,10},
      ytick={-10,-9,...,10},
      ylabel={$ct$},
      xlabel={$\vec{x}$},
      title={Trajectoire d'un MRUA},
      legend pos=south west,
    ]

    \addplot [thick, blue, domain=-1.9:1.9] ({x^2+1}, {x});
    \addlegendentry{$x(t)$ selon Newton}
    \addplot [thick, red, domain=-1.9:1.9] ({cosh(x)}, {sinh(x)});
    \addlegendentry{$x(t)$ selon Einstein}

    \addplot[help lines] expression {x};
    \addplot[help lines] expression {-x};
  \end{axis}
\end{tikzpicture}




\begin{tikzpicture}[scale=0.9]
  \begin{scope}[shift={(-5,0)}]
    \draw (-1,-.5)--++(6,0)--++(0,3.5)--++(-6,0)--++(0,-3.5);
    \draw[dashed] (-1,-.5)++(2,0)--++(0,3.5);
    \node[circle, draw] at (0,2) {q};
    \eye[black]{.9}{0}{0}{90}
    \node[circle, draw] at (4,2) {q};
    \eye[black]{.9}{2}{0}{90}
    \draw[thick, -latex] (2,0)++(-55*.5+90:0.35)++(0.1,0)--++(.6,0) node[above]{\tiny{MRU}}--++(.6,0);
  \end{scope}

  \begin{scope}[shift={(5,0)}]
    \draw (-1,-.5)--++(6,0)--++(0,3.5)--++(-6,0)--++(0,-3.5);
    \draw[dashed] (-1,-.5)++(2,0)--++(0,3.5);
    \node[circle, draw] at (0,2) {q};
    \eye[black]{.9}{0}{0}{90}
    \eye[black]{.9}{2}{0}{90}
    \draw[thick, -latex] (4,2) node[circle, draw, fill=white] {q} to ++(-.8,0) node[above]{\tiny{MRU}}--++(-.6,0);
  \end{scope}

  \draw[rounded corners=1pt, fill=red] (1,0.7) rectangle ++(2,0.3);
  \draw[rounded corners=1pt, fill=red] (1,1.8) rectangle ++(2,-.3);
\end{tikzpicture}


\begin{tikzpicture}
  \draw[-latex] (-5,0) -- (5,0) node[above] {$x_m$};
  \draw[-latex] (0,-2) -- (0,4) node[right] {$x_0=ct$};

  \draw[-latex, gray] (3.2,1.5)--++(2,0);
  \draw[-latex, gray] (-3.2,1.5)--++(-2,0);
  \node[below right] at (0,1) {$x^0$};
  \node[above right] at (0,2) {$x^0{}'$};
  \begin{scope}[decoration={markings,mark=at position 0.65 with {\arrow[ultra thick]{latex}}}]
    \draw[  thick, black, postaction={decorate}] (-3,1)--(3,1);
    \draw[     thick, dgreen, postaction={decorate}] (3,1) arc (-90:90:0.2 and 0.5);
    \draw[  thick, black, postaction={decorate}] (3,2)--(-3,2);
    \draw[     thick, dgreen, postaction={decorate}] (-3,2) arc (90:270:0.2 and 0.5);
  \end{scope}
\end{tikzpicture}






\begin{tikzpicture}
  \draw[-latex, thick] (-3,0) -- (3,0) node[above] {$\vec{p}$};
  \draw[-latex, thick] (0,-.5) -- (0,3) node[right] {$\dfrac{E}{c}$};
  \draw[help lines] (-.5,-.5) -- (2.5,2.5) (.5,-.5) -- (-2.5,2.5);
  \draw[ultra thick, red, -latex] (0,0) -- ( 1,1) node[right] {$\gamma_1$};
  \draw[ultra thick, red, -latex] (0,0) -- (-1,1) node[left] {$\gamma_2$};
  \draw[ultra thick, dgreen, -latex] (0,0) -- (0,2)node[right] {$\pi_0$} ;
\end{tikzpicture}


\begin{tikzpicture}[scale=1.5]
  \coordinate (a) at (2,1.2);
  \draw[-latex, thick] (0,0) -- (3,0) node[right] {$\vec{e_1}$};
  \draw[-latex, thick] (0,0) -- (1.5,2.5) node[above] {$\vec{e_2}$};
  \draw[dashed, dgreen, thick] (a) -- (1.28,0) node[below] {$a^1$};
  \draw[dashed, dgreen, thick] (a) -- (.72,1.2) node[left] {$a^2$};
  \draw[dashed, red, thick] (a) -- (2,0) node[below] {$a_1$};
  \draw[dashed, red, thick] (a) -- (9/8.5,15/8.5) node[left] {$a_2$};
  \draw[lblue, ultra thick, -latex] (0,0) -- (a) node[right] {$\vec{a}$};
  \draw (0,0) node {\tiny{$\bullet$}};
\end{tikzpicture}


\begin{tikzpicture}[scale=1.5]
  \begin{scope}[shift={(-2,0)}]
    \draw[-latex, thick] (0,0) -- (1.5, 0) node[right] {$\vec{e_1}$};
    \draw[-latex, thick] (0,0) -- (0.3,1.1) node[above] {$\vec{e_2}$};
    \draw[lblue, ultra thick, -latex] (0,0) -- (2,1.2) node[right] {$\vec{a}$};
    \draw (0,0) node {\tiny{$\bullet$}};
  \end{scope}
  \begin{scope}[shift={( 2,0)}]
    \draw[-latex, thick] (0,0) -- (2/3, -1/5.5) node[right] {$\vec{e^1}$};
    \draw[-latex, thick] (0,0) -- (0,1/1.1) node[above] {$\vec{e^2}$};
    \draw[lblue, ultra thick, -latex] (0,0) -- (2,1.2) node[right] {$\vec{a}$};
    \draw (0,0) node {\tiny{$\bullet$}};
  \end{scope}
\end{tikzpicture}

\begin{tikzpicture}[scale=1.5]
  \draw[-latex, thick] (0,0) -- (1.5, 0)  node[right] {$\vec{e_1}$};
  \draw[-latex, thick] (0,0) -- (0.3,1.1) node[above] {$\vec{e_2}$};
  \draw[-latex, thick, red, rotate=20]  (0,0) -- (1.5, 0)  node[right] {$\vec{e_1'}$};
  \draw[-latex, thick, red, rotate=20]  (0,0) -- (0.3,1.1) node[above] {$\vec{e_2'}$};
  \draw[lblue, ultra thick, -latex] (0,0) -- (2,1.2) node[right] {$\vec{a}$};
  \draw (0,0) node {\tiny{$\bullet$}};
  \centerarc[help lines, -latex] (0,0) (0.7,0,20)
  \centerarc[help lines, -latex] (0,0) (0.7,atan(1.1/0.3),atan(1.1/0.3)+20)
\end{tikzpicture}



\begin{tikzpicture}
  \begin{scope}[shift={(-3,0)}]
    \draw (0,0) node {\tiny{$\bullet$}};
    \draw [thick, latex-latex] (0,2) node[right] {$ct'$} -- (0,0) -- (2,0) node[below] {$x'$};
    \node[above] at (1.2,2.1) {$\mathcal{R'}$};
  \end{scope}

  \begin{scope}[shift={(3,0)}]
    \draw (0,0) node {\tiny{$\bullet$}};
    \draw [thick, latex-latex] (0,2) node[right] {$ct$} -- (0,0) -- (2,0) node[below] {$x$};
    \node[above] at (1.2,2.1) {$\mathcal{R}$};
    \draw[-latex, ultra thick, red] (1.2,1.1)++(-0.7,0)--++(1.5,0) node[right] {$\vec{v_0}$};
  \end{scope}
\end{tikzpicture}


\begin{tikzpicture}
  \begin{axis}[
      samples=121,
      xmin=-3,
      xmax=3,
      ymin=-1,
      ymax=1,
      width=12cm,
      height=4.5cm,
      disabledatascaling,
      grid=both,
      %font=\footnotesize,
      %grid style={line width=.1pt, draw=red},
      %major grid style={line width=.2pt,draw=gray!50},
      %minor tick num=1,
      axis lines=middle,enlargelimits=0.07,
      %execute at begin axis={
      execute at end axis={        \draw[thick] (rel axis cs:0,0) -- (rel axis cs:1,0) -- (rel axis cs:1,1) -- (rel axis cs:0,1) --cycle;},
      %xticklabels={},
      %yticklabels={},
      xtick={-10,0,10},
      ytick={-10,-9,...,10},
      %ylabel={$\tanh(\xi)$},
      xlabel={$\xi$},
      title={$\tanh(\xi)$},
      %legend pos=south east,
    ]

    \addplot [thick, domain=-4:4] ({x}, {tanh(x)});
    %\node[anchor=south,rotate=90] at (0,-.2) {$\xi$};
  \end{axis}
\end{tikzpicture}

\end{document}
