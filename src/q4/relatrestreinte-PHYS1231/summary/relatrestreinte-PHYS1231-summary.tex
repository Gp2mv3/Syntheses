\documentclass[fr]{../../../eplsummary}

\hypertitle{Relativité restreinte}{4}{PHYS}{1231}
{FONCK Valentin \and LABOUREUR Pierre Author2\and Author3}
{Jean-Marc Gérard}
\newpage

\section{Approche géométrique}
Il sera ici question d'aborder certaines équations et concepts qui seront développés tout au long du cours.

\subsection{Pythagore pour mesurer des distances}

   Considérons un chemin quelconque \textbf{linéaire} dans un repère cartésien muni de deux axes: x et y. Le repère cartésien correspond donc à un espace vectoriel engendré par la base: $<\vec{e_{x}},\vec{e_{y}}>$. Le théorème de Pythagore permet de calculer la distance au carré $\Delta l^2$ entre le point de départ et le point d'arrivée de notre chemin. \footnote{A noter que $\Delta l^2$ équivaut bien à $(l_2 - l_1)^2$ et non $(l_2^2 - l_1^2)$. Cette remarque reste valable tout au long du document.}
   \begin{equation}
       \Delta l^2 = \Delta x^2 + \Delta y^2
       \label{pyth}
   \end{equation}
   Cette équation bien que triviale repose sur l'hypothèse d'un espace euclidien, considérer les cas où cette hypothèse n'est pas vérifiée sera la pierre angulaire de la suite du raisonnement. On peut ainsi aborder le cas où la trajectoire n'est pas rectiligne, ce qui va nous permettre d'introduire la notion de métrique.

\end{document}
