\documentclass[fr]{eplexam}
\usepackage[utf8]{inputenc}
\usepackage{listings}
\usepackage{float}

\hypertitle{Systèmes informatiques}{4}{SINF}{1252}{2017}{Juin}
{Quentin Dessain}
{Olivier Bonaventure}

\begin{center}
    \Large{Ce document reprend uniquement la partie théorique de l'examen du 10 juin 2017 de Systèmes Informatiques.}
\end{center}

\section{Exécution conditionnelle [4 points]}
Le shell \texttt{bash} permet l'exécution conditionnelle de commandes. Celle-ci est décrite comme suit dans la page de manuel de \texttt{bash}:

\vspace{0.3cm}
\texttt{An OR list has the form}

\vspace{0.3cm}
\texttt{command1} $\vert\vert$ \texttt{command2}
\vspace{0.3cm}

\texttt{command2 is executed if and only if command1 returns a non-zero exit status.The return status of AND and OR lists is the exit status of the last command executed in the list.}
\vspace{0.5cm}


Expliquez en détails \textbf{tous} les appels systèmes utilisés \textbf{par le shell} lors de l'éxecution de la ligne de commandes bash suivante:
\vspace{0.3cm}

/bin/prog \textgreater/tmp/t  $\vert\vert$ /bin/prog2


\section{Mémoire virtuelle [3 points]}


Considérons un processus qui s'exécute sur un ordinateur utilisant la mémoire virtuelle. Pour simplifier, supposons que ce processus utilise une page pour son segment de code, une page pour ses variables globales, une page pour son heap et une page pour son stack. Ce processus dans sa fonction \texttt{main} exécute les lignes suivantes:

\begin{lstlisting}

int main(int argc, char** argv){
int *ptr =(int *) malloc(sizeof(int));
*ptr = 1252;
}

\end{lstlisting}
\newpage
\begin{itemize}


\item[$\bullet$] Expliquez l'organisation du processus en mémoire et indiquez dans quelles zones se trouvent le code des fonctions \texttt{main} et \texttt{malloc}, la variable \texttt{*ptr} et la zone mémoire retournée par malloc.

\begin{table}[H]
\centering
\begin{tabular}{|l|l|l|l|l|l|l|l|l|l|l|l|l|l|l|l|l|l|l|l|}
\hline
 &  &  &  &  &  &  &  &  &  &  &  &  &  &  &  &  &  &  &  \\ \hline
 &  &  &  &  &  &  &  &  &  &  &  &  &  &  &  &  &  &  &  \\ \hline
 &  &  &  &  &  &  &  &  &  &  &  &  &  &  &  &  &  &  &  \\ \hline
 &  &  &  &  &  &  &  &  &  &  &  &  &  &  &  &  &  &  &  \\ \hline
 &  &  &  &  &  &  &  &  &  &  &  &  &  &  &  &  &  &  &  \\ \hline
 &  &  &  &  &  &  &  &  &  &  &  &  &  &  &  &  &  &  &  \\ \hline
 &  &  &  &  &  &  &  &  &  &  &  &  &  &  &  &  &  &  &  \\ \hline
 &  &  &  &  &  &  &  &  &  &  &  &  &  &  &  &  &  &  &  \\ \hline
 &  &  &  &  &  &  &  &  &  &  &  &  &  &  &  &  &  &  &  \\ \hline
 &  &  &  &  &  &  &  &  &  &  &  &  &  &  &  &  &  &  &  \\ \hline
 &  &  &  &  &  &  &  &  &  &  &  &  &  &  &  &  &  &  &  \\ \hline
 &  &  &  &  &  &  &  &  &  &  &  &  &  &  &  &  &  &  &  \\ \hline
 &  &  &  &  &  &  &  &  &  &  &  &  &  &  &  &  &  &  &  \\ \hline
 &  &  &  &  &  &  &  &  &  &  &  &  &  &  &  &  &  &  &  \\ \hline
 &  &  &  &  &  &  &  &  &  &  &  &  &  &  &  &  &  &  &  \\ \hline
 &  &  &  &  &  &  &  &  &  &  &  &  &  &  &  &  &  &  &  \\ \hline
 &  &  &  &  &  &  &  &  &  &  &  &  &  &  &  &  &  &  &  \\ \hline
 &  &  &  &  &  &  &  &  &  &  &  &  &  &  &  &  &  &  &  \\ \hline
 &  &  &  &  &  &  &  &  &  &  &  &  &  &  &  &  &  &  &  \\ \hline
 &  &  &  &  &  &  &  &  &  &  &  &  &  &  &  &  &  &  &  \\ \hline
 &  &  &  &  &  &  &  &  &  &  &  &  &  &  &  &  &  &  &  \\ \hline
 &  &  &  &  &  &  &  &  &  &  &  &  &  &  &  &  &  &  &  \\ \hline
 &  &  &  &  &  &  &  &  &  &  &  &  &  &  &  &  &  &  &  \\ \hline
 &  &  &  &  &  &  &  &  &  &  &  &  &  &  &  &  &  &  &  \\ \hline
 &  &  &  &  &  &  &  &  &  &  &  &  &  &  &  &  &  &  &  \\ \hline
 &  &  &  &  &  &  &  &  &  &  &  &  &  &  &  &  &  &  &  \\ \hline
 &  &  &  &  &  &  &  &  &  &  &  &  &  &  &  &  &  &  &  \\ \hline
 &  &  &  &  &  &  &  &  &  &  &  &  &  &  &  &  &  &  &  \\ \hline
\end{tabular}
\end{table}

\newpage
\item[$\bullet$] Dessinez la table des pages de ce processus avec les valeurs des différents drapeaux au démarrage du processus. Expliquez le rôle de ces drapeaux.


\begin{table}[H]
\centering
\begin{tabular}{|l|l|l|l|l|l|l|l|l|l|l|l|l|l|l|l|l|l|l|l|}
\hline
 &  &  &  &  &  &  &  &  &  &  &  &  &  &  &  &  &  &  &  \\ \hline
 &  &  &  &  &  &  &  &  &  &  &  &  &  &  &  &  &  &  &  \\ \hline
 &  &  &  &  &  &  &  &  &  &  &  &  &  &  &  &  &  &  &  \\ \hline
 &  &  &  &  &  &  &  &  &  &  &  &  &  &  &  &  &  &  &  \\ \hline
 &  &  &  &  &  &  &  &  &  &  &  &  &  &  &  &  &  &  &  \\ \hline
 &  &  &  &  &  &  &  &  &  &  &  &  &  &  &  &  &  &  &  \\ \hline
 &  &  &  &  &  &  &  &  &  &  &  &  &  &  &  &  &  &  &  \\ \hline
 &  &  &  &  &  &  &  &  &  &  &  &  &  &  &  &  &  &  &  \\ \hline
 &  &  &  &  &  &  &  &  &  &  &  &  &  &  &  &  &  &  &  \\ \hline
 &  &  &  &  &  &  &  &  &  &  &  &  &  &  &  &  &  &  &  \\ \hline
 &  &  &  &  &  &  &  &  &  &  &  &  &  &  &  &  &  &  &  \\ \hline
 &  &  &  &  &  &  &  &  &  &  &  &  &  &  &  &  &  &  &  \\ \hline
 &  &  &  &  &  &  &  &  &  &  &  &  &  &  &  &  &  &  &  \\ \hline
 &  &  &  &  &  &  &  &  &  &  &  &  &  &  &  &  &  &  &  \\ \hline
 &  &  &  &  &  &  &  &  &  &  &  &  &  &  &  &  &  &  &  \\ \hline
 &  &  &  &  &  &  &  &  &  &  &  &  &  &  &  &  &  &  &  \\ \hline
 &  &  &  &  &  &  &  &  &  &  &  &  &  &  &  &  &  &  &  \\ \hline
 &  &  &  &  &  &  &  &  &  &  &  &  &  &  &  &  &  &  &  \\ \hline
 &  &  &  &  &  &  &  &  &  &  &  &  &  &  &  &  &  &  &  \\ \hline
 &  &  &  &  &  &  &  &  &  &  &  &  &  &  &  &  &  &  &  \\ \hline
 &  &  &  &  &  &  &  &  &  &  &  &  &  &  &  &  &  &  &  \\ \hline
 &  &  &  &  &  &  &  &  &  &  &  &  &  &  &  &  &  &  &  \\ \hline
 &  &  &  &  &  &  &  &  &  &  &  &  &  &  &  &  &  &  &  \\ \hline
 &  &  &  &  &  &  &  &  &  &  &  &  &  &  &  &  &  &  &  \\ \hline
 &  &  &  &  &  &  &  &  &  &  &  &  &  &  &  &  &  &  &  \\ \hline
 &  &  &  &  &  &  &  &  &  &  &  &  &  &  &  &  &  &  &  \\ \hline
 &  &  &  &  &  &  &  &  &  &  &  &  &  &  &  &  &  &  &  \\ \hline
 &  &  &  &  &  &  &  &  &  &  &  &  &  &  &  &  &  &  &  \\ \hline
\end{tabular}
\end{table}

\end{itemize}
\newpage
\section{Algorithme de Peterson [3 points]}

Un site Internet propose l'implémentation suivante pour résoudre le problème de l'exclusion mutuelle en C. Cette implémentation est-elle correcte? Si oui, justifiez en détails l'absence de deadlock et de violation de section critique. Si non, expliquez via un exemple pourquoi elle ne fonctionne pas et proposez une correction en utilisant les mêmes variables.

\begin{lstlisting}

// Initialisation
int a=false;
int b=false;
int c=0;

// thread 0
a=true;
c=0;

while(b || (c==1) ) {}
section_critique();
a=false;

//thread 1
b=true;
c=1;
while(a || (c==0) ) {}
section_critique();
b=false;

\end{lstlisting}

\end{document}
