\section{Core Problems}
\subsection{Consensus}
Consensus is the process of agreeing on a number.
Problem is that all the nodes propose a value and some nodes might
crash \& stop responding.

\subsection{Atomic Broadcast}
If a node broadcast a message, all nodes must deliver
the message in the same order.

\subsection{Relation}
Atomic broadcast $\equiv$ consensus (proof slide 13)

It's possible to resolve consensus if we have atomic broadcast and vice-versa.
\begin{enumerate}
    \item broadcast $\to$ consensus: We take the first proposal as
    messages are received in the same order.
    \item consensus $\to$ broadcast: The subject of the consensus is the order to take.
\end{enumerate}

Paxos est ce qui est le plus utilisé pour les consensus\footnote{\url{http://research.microsoft.com/en-us/um/people/lamport/pubs/paxos-simple.pdf}}
 %J'ai vu un TODO mais je suis pas sur qu'il faille connaitre.

\section{Concurrency Aspects}

\begin{itemize}
    \item Asynchronous: There is no bound on the time for a message to
     arrive and to be computed, it resolve consensus iff 0 node crashes
    \item Partially synchronous: It start asynchronous and then become
        synchronous (it gets an upper bound, we know it will happen but we
        don't know when.)
	  Consensus up to $\frac{n}{2}$ crashes
    \item Synchronous: Bound known for delivering and computation of message. Consensus with n-1 crashes
\end{itemize}

\paragraph{Asynchronous vs Synchronous}

Bound is simulated with an expected bound to be in partially synchronous.

\section{Failure Aspects}
Each node use a failure detector that is implemented by
heartbeat and waiting.\\
Problem $\to$ bound exist but we don't know the exact value because
this bound can change with time (if RTT increase for example),
We need to adapt the bound.

Other kinds fault than crash can appears
\begin{itemize}
    \item \textbf{Byzantine faults}: Sending wrong information, omit
        messages,\ldots
        \begin{enumerate}
            \item[$\to$] Byzantine algorithm tolerate $1/3$ faulty node and
                non-byzantine only $1/2$
        \end{enumerate}
    \item \textbf{Self-stabilizing}: It's important to know that system
        can be in a \textit{legitimate} or an
        \textit{illegitimate} state.

        It's robust to failure and don't need initialization!

        \begin{enumerate}
            \item[Need]
                \begin{enumerate}
                    \item Convergence = from any illegitimate state,
                        system can eventually goes to a legitimate state
                    \item Closure = if in legitimate state, it remains
                        in a legitimate state.
                \end{enumerate}
        \end{enumerate}
\end{itemize}
For example in a token ring algorithm:
\begin{itemize}
	\item Illegitimate state: 0,2,3\ldots token.
	\item Legitimate state: only one token.
\end{itemize}
