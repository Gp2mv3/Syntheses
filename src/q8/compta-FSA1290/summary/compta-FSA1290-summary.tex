\documentclass[fr]{../../../eplsummary}

\hypertitle{compta-FSA1290}{8}{FSA}{1290}
{Rodolphe de Schaetzen}
{Professor}

\section*{Chapitre 1. Les principes comptables}

\section{Les principes de quantification ou de mesure}
\subsection{Le principe de monétarisation}
Toutes les transactions sont exprimées en unités monétaires. On exclut donc les éléments non monnayables. \\
L'immatériel n'est pas appréhendé par la comptabilité, or c'est une richesse non négligeable d'une entité. Par exemple la capacité d'une société à innover et créer.\\ \\
Ce principe suppose la stabilité de l'unité monétaire dans le temps. Or les monnaies ont des taux qui varient et la valeur nette des acquis peuvent varier dans le temps. Prenons pour exemple une société Européenne libellés en EUR qui à une activité aux US, les comptes doivent être gardés en EUR en prenant en compte le taux de change actuel. Ou prenons l'exemple d'un immeuble acquis il y a de nombreuses années. Sa valeur réelle actuelle risque d'être éloignée de sa valeur nette comptable. \\

\subsection{Le principe des coûts historiques}
En droit comptable belge, pour évaluer la valeur d'un élément, on retient son coût historique. Càd son coût d'acquisition où son coût de production.\\ \\
Un premier correctif est effectué à la fin de chaque exercice comptable (durée: +/- un an) pour une partie des biens qui seront amortis pour enregistrer la diminution de valeur qu'ils ont subis. \\ \\
Un deuxième correctif consiste en la possibilité de revaloriser certaines catégories d'actifs. (Sous certaines conditions) \\

\subsection{le principe de prudence}
Le principe de prudence consiste à prendre immédiatement en compte une diminution probable de la valeur d'un bien alors qu'on ne prend en compte l'augmentation de valeur que lorsqu'on réalise réellement l'augmentation (revente ou autre). \\ \\
Cela évite la surestimation de la valeur du patrimoine d'une entité, mais à l'inverse on risque de le sous-estimer sa valeur.\\ \\
Le principe de prudence a pour objectif premier d'inspirer la confiance dans les comptes publiés par les entreprises. \\

\section{Les principes d'observation}
\subsection{Le principe de l'entité}
''Chaque société ou organisation non-marchande est considée comme une entité distincte de ses propriétaires, membres ou partenaires économiques.'' \\ \\
\begin{itemize}
    \item Entreprise sous forme sociétaire: a généralement une personnalité juridique distincte des personnes physiques ou morales qui sont les propriétaires.
    \item Entreprise individuelle: le patrimoine se confond avec celui de son propriétaire, mais on la compta isole le patrimoine personnel et les résultats proféssionels.
    \item Organisation non-marchande: pareil que pour les sociétés avec une personne morale.
\end{itemize}

\subsection{Le principe de découpage du temps}
Un exercice comptable est d'une durée de plus ou moins un an. Ca ne doit pas forcément commencer en Janvier comme l'année civile. (ex. les clubs de sports qui se calquent à la saison sportive) \\
Cela a pour but de rattacher les faits comptables à un exercice comptable, mais ca ne correspond pas tout le temps aux opérations de l'entité. \\
Les effets des transactions sont comptabilisés au moment où ils se produisent et sont enregistrés bans la comptabilité et états financiers qui se rapportent aux exercices comptables.\\
\subsection{Le principe de continuité}
L'entreprise compte continuer son activité dans els années à venir. Il faut donc qu'il y ait suffisamment de fonds pour pouvoir continuer l'activité l'anéne suivante. \\


\section*{Chapitre 2. Représentation synthétique du patrimoine et de son évolution}
\section{Le bilan "Balance Sheet"}
\subsection{Notion}
Le bilan est une photographie instantanée du patrimoine monnayable de l'entité. Comme pour une photo il faut en prendre régulièrement pour éviter que les sujets de la photo ne changent avec le temps. \\
Il faut donc des états financiers récents.

\subsection{Le passif}
Le passif comprends l'ensemble des ressources qui sont mises à la disposition de l'entité à la date d'établissement du bilan. 
\begin{itemize}
    \item Les capitaux propres:
    \begin{itemize}
        \item Capital des actionnaires
        \item Bénéfices conservés
        \item Pertes
    \end{itemize}
    \item Les dettes auprès des:
    \begin{itemize}
        \item Banques
        \item Fournisseurs
    \end{itemize}
\end{itemize}

\subsection{l'actif}
Tout ce qui est utilisé ou tous les emplois de fonds qui ont été effectués.
\begin{itemize}
    \item Immobilisations: elles permettent d'exercer son activité et donc de faire du profit
    \begin{itemize}
        \item brevets
        \item batiments
        \item équipements
        \item parts de sociétés
        \item ...
    \end{itemize}
    \item Créances:
    \begin{itemize}
        \item clients livrés (du bien ou du service) qui doivent encore payer
        \item promesse de subvention de l'état à une organisation non-marchande
    \end{itemize}sur des clients (ou sur les pouvoirs publics)
    \item Marchandises et produits en stock destinés à être vendus
    \item Les avoirs monétaires (en caisse ou sur compte en banque)
\end{itemize}
$$actif = capitaux propres + dettes$$

\subsection{Le tableau de bilan}
Tableau sous la forme de deux colonnes: l'actif et le passif. \\
$$actif = passif$$

\section{Le compte de résultats}
\subsection{Notion}
Le compte de résultats représente la traduction comptable de l'évolution des activités de l'entité. Il synthétise au cours de l'année (sans prendre en compte la date d'encaissement ou de paiement):
\begin{itemize}
    \item les charges: les éléments qui influencent négativement la valeur de l'entité
    \item les produits: les éléments qui influencent positivelement la valeur de l'entité
\end{itemize}
$$résultat = produits - charges$$
La comptabilité d'engagement: basé sur produits et charges. \\ \\
La comptabilité de caisse: base sur les recettes et les dépenses. \\ \\
On ne crée pas de compre de résultat après chaque transaction. On les enregistre dans le livre-journal. 

\section*{Chapitre 3. La Méthode Comptable: le Principe de la Partie Double}

\section{La comptabilité à partie double appliquée aux éléments du bilan}
\subsection{Principe}
$$Actif = Passif$$

\begin{center}
\includegraphics[scale = 0.3]{actif_passif.png}
\end{center}

\section{La comptabilité à partie double étendue aux éléments du résultat -  A FINIR}
Les enregistrements quotidiens combinent bilan et résultat, il serait donc simplificateur de ramener la comptabilité à des mouvements sur le bilan. \\

\subsection{Comptes de gestion}
Les comptes de gestion couvrent des emplois et des ressources définitifs pour une période donnée (exercice comptable). \\
Il existe 2 types de comptes de gestion:
\begin{itemize}
    \item Les comptes de charges: enregistrent les charges engagées ou subies par l'entité
\end{itemize}

\section*{Chapitre 4. L'Organisation du Traitement de l'Information Comptable}
\section{Le livre-journal}
Le livre-journal enregistre les transactions opération après opération. On comptabilise donc toutes les opérations en indiquant le compte du grand livre débité et le compte du grand livre crédité. \\
Au bas de chaque page du grand livre il faut s'assurer de l'égalité des colonnes "débit" et "crédit".\\
Le livre-journal permet de retrouver la trace des opérations chronologiquement. \\
Le livre-journal peut être divisé en journaux "auxialiaires" (ex: journal des achats, journal des ventes, ... )
\section{Le grand-livre}
Le grand-livre regroupe l'ensemble des comptes de l'entité. Il s'agit d'un résumé des transactions par compte. Cela permet de classer et récapituler les opérations. \\
\section{La balance des comptes}
Les soldes (débiteur ou créditeur), à une date
déterminé (fin de l’exercice comptable), des
comptes du grand-livre.
\section{Le livre d'inventaire}
\begin{itemize}
    \item L'inventaire
    \item Les règles d’évaluation: conversion des éléments à inventorier en unités monétaires 
    \item Le bilan, le compte de résultats et les annexes
\end{itemize}
\section{Methodologie du traitement comptable d'une opération}

\includegraphics[scale = 0.25]{traitementcomptable.png}
\section*{Chapitre 6. Les Rubriques du Bilan}
\section{Les rubriques de bilan}
\subsection{Introduction aux rubriques de bilan}
Le bilan reflète en langage codée la situation patrimoniale de l'entité à un moment précis. Il recense :
\begin{itemize}
    \item du passif: les sources ou moyens de financement nécessaires aux investissements et à l'exploitation de l'entité.
    \begin{itemize}
        \item par ordre croissant de liquité
    \end{itemize}
    \item de l'actif: les utilisations ou emplois effectués par l'entité dans ses investissements et son exploitation. 
    \begin{itemize}
        \item par ordre décroissant d'exigibilité
    \end{itemize}
\end{itemize}
La rédaction du bilan respecte un schéma officiel belge. En rubriques en utilisant le chiffre romain. \\
Les actifs sont présentés en valeurs nettes, cad après réduction des amortissements et réductions de valeur cummulés. Il y a donc une perte d'information compensée par des indications supplémentaires dans l'annexe. Contradiction entre emplois et ressources: les amortissements sont des ressources qui permettent à l'entité de financer des emplois. 
\\
\subsection{Rubriques de Passif}
\subsubsection*{Trois grandes masses}
Le passif se subdivise en 3 grandes masses:
\begin{itemize}
    \item Les capitaux propres: doivent couvrir le risque de l'entité. (peut être négatif)
    \item Les provision (et impots différés): pas de moyens propres et pas de dettes (liés à une future décision incertaine)
    \item Les dettes: les montants que l'entité doit rembourser
\end{itemize}

\subsubsection*{Capital}
Le capital correspond au patrimoine affecté de manière intangible aux besoins d'une entreprise par ses actionnaires ou associés, au moment de sa constitution ou suite à une augmentation de capital. \\
Le capital minimum pour les SA: €61 500; et SPRL: €18 550 \\
Pour une entreprise individuelle, le capital ne s'agit pas du résultat de l'accumulation de bénéfices antérieurs. Le capital correspond aux moyens propres affectés durablement par la personne physique à son activité professionnelle. \\
\subsubsection*{Primes d'émission}
Imaginons A et B qui lancent une société et investissent 500.000€ chacun. Ils recoivent chacun 500 actions de 1000€. Apres quelques années, C vient augmenter le capital avec 1.500.000 et en prenant compte des résulats de la société (plus-values latentes (a venir) + bénéfices + ...) les actions valent maintenant 3.000€. C recoit donc 500 actions et détient donc $1/3$ de la société. La différencte entre la valeur nominale (1 000€) des actions et la valeur réelle (3.000€) constitue une prime d'émission des actions nouvelles. 
\subsubsection*{Moyens d'autofinancement}
Les moyens d'autofinancement correspond aux ressources propres de l'entité qui financent les actifs. \\
Leur utilisation revêt des formes diverses:
\begin{itemize}
    \item La réserve légale (130): montant cumulé des bénéfices réalisé par certaines société qui ne peuvent pas être distribué par le droit des sociétés.
    \begin{itemize}
        \item Décidé par l'assemblée générale
        \item Au moins 1/20 des bénéfices nets par an dans la réserve
        \item Prélèvement de 1/20 jusqu'à atteindre 1/10 du capital social
        \item Permet de renforcer le capital et d'accroitre la garantie aux tiers
    \end{itemize}
    \item Les réerves indisponibles (131): bénéfices réalisés et réservés pour des décisions spéciales (p.e.: rachat de propres actions (1310) )
    \item Les réserves immunisées (132): plus-values réalisées sur immobilisés. 
    \item Les réserves disponibles (133): a utiliser dans le futur par une décision à la majorité
    \item Le bénéfice reporté (14): bénéfice ni distribué, ni reporté
    \item La perte reportée (14): pour "nettoyer" le bilan de pertes reportées: réduction de capital ou prélèvement sur la réserve légale ou les primes d'émission. 
\end{itemize}

\subsubsection*{Plus-values de réévaluation (12)}
Permet de comptabiliser l'augmentation de valeur d'un bien (p.e.: terrain) \\
Sous certaines conditions strictes (augmentation de la valeur doit être durable), on peut comptabiliser des plus-values latentes (non-réalisées). \\
le droit comptable belge tend à décourager/canaliser la réévaluation des actifs (principe de prudence). \\

\subsubsection*{Subsides en capital (15)}
Subventions accordées par les pouvoirs publics pour le financement d'investissements en immobilisations.\\
A la fin de l'exercice comptable on exerce une imputation au crédit du compte de résultats (produits financiers).

\subsubsection*{Provisions pour risques et charges (160-165)}
Pour couvrir des pertes ou charges futures qui sont probables à la date de clôture mais indéterminées au niveau du montant. (p.e.: grosses réparations, amendes, retour de produits, ...)\\

\subsubsection*{Impots différés (168)}
Résultats de l'exercice comptable =/= la base imposable.

\subsubsection*{Subdivisions de dettes}
\begin{itemize}
    \item Les dettes à plus d'un an (17): La partie des dettes qu'on ne devra pas rembourser dans les douze prochains mois.
    \item Les dettes à plus d'un an échéant dans l'année (42): La partie des dettes à plus d'un an (17) qu'on doit rembourser dans les douze mois.
    \item Les dettes à un an ou plus (43-48): les dettes à rembourser dans les 12 prochaines mois.
\end{itemize}
\subsection{Rubriques d'actif}
\subsubsection*{Deux grandes masses}
\begin{itemize}
    \item Les actifs immobilisés (investissments): pas réalisés en numéraire (infrastructure - immobilisations)
    \item Les actifs circulants: valeurs disponibles ou réalisables (stocks, créances et liquidités)
\end{itemize}
Par exemple: une vache viandeuse est un actif circulant car il sera transformé en liquidité une fois vendue. La vache laitière, elle est un actif immobilisés car elles sont maintenues durablement et amorties comme investissement productif de lait.

\subsubsection*{Frais d'établissement (20)}
A ne pas confondre avec les frais de batiments. \\ \\
Ce sont les frais rattachés à la constitution, développement ou restructuration de l'entité. 

\subsubsection*{Immobilisations incorporelles (21)}
Il s'agit d'investissements en éléments immatériels, tels que des frais de recherche et dévelopmment (r\&d), des brevets, une clientèle, ... \\
\begin{itemize}
    \item Le goodwill: différence entre le coût d'acquisition d'une entreprise et la valeur réelle actuelle de cette entreprise (son actif)
\end{itemize}

\subsubsection*{immobilisations corporelles (22-27)}
Il s'agit d'investissements en éléments matériels tels que des terrains, des constructions, des installations, des machines, de l'outillage, du mobilier, du matériel roulant, ...\\ 
Les biens détenus en location-financement (leasing) doivent être portés sous les immobilisations corporelles.\\
Les actifs corporels subissent des amortissements: \\
\begin{center}
(Coût d’acquisition – valeur résiduelle) / période d’utilisation économique \\
\end{center}
(Pas pour des terrains) \\

\subsubsection*{Immobilisations financières (28)}
Il s'agit d'investissements en titres et en créances, tels que des actions ou prêts à des filiales. Plus généralement dans des entreprises avec lesquelles un lien durable et spécifique est établi. (en vue d'augmenter le nombre de commandes d'un client par exemple)
\subsubsection*{Subdivision des créances}
\begin{itemize}
    \item Créances à plus d'un an (29): plutôt des prêts
    \item Créances à un an ou plus (40-41): échéances dans les 12 mois. 
\end{itemize}
\subsubsection*{Stocks et commandes en cours d'exécution}
\begin{itemize}
    \item Approvisionnements (matières premières et fournitures)(coût d'acquisition) (30-31)
    \item Marchandises (coût d'acquisition) (34)
    \item Produits finis (coût de production/revient) (33)
    \item Commandes en cours d'exécution. Quelle est la valeur à la fin de l'exercice comptable? (32-37)
\end{itemize}
Il y a des règles d'évaluations spécifiques: LIFO, FIFO, CMP, ..
\subsubsection*{Placements de trésorerie (50-53)}
Il s'agit des placements bancaires sur compte et en valeurs mobilières (actions, obligations, ..) sans avoir de caractère d'immobilisations financières (donc sans lien durable et spécifique).

\subsubsection*{Valeurs disponibles (55-57)}
Il s'agit des comptes financiers à vue et caisse. 
Si le solde est négatif ils figurent au passif (433)
\section*{Chapitre 7. Les Rubriques du Compte de Résultats}
\section{Rubriques du compte de résultats}
\subsection{Intruduction}
Le compte des résultats assure la traduction en langage comptable (codé) de l'évolution des activités de l'entité au cours d'une période donnée. Il synthétise les charges et les produits.  \\ 
\\
Les rubriques du compte de résultats sont classées par ordre décroissant du lien qu'elle entretiennent avec l'acitivité habituelle de l'entité. 
\begin{itemize}
    \item Produits et charges d'exploitations (liés à l'activité habituelle)
    \item Produits et charges financiers (ex. intérêts)
    \item Produits et charges exceptionnels (ex. plus-values ou moins-values liées à la vente d'un actif immobilisé)
\end{itemize}
\subsection{Rubriques et sous-rubriques}
\begin{itemize}
    \item Le chiffre d'affaires (70): Lié à l'activité habituelle (tjs hors TVA) (451 TVA à payer) Sous déduction des réductions commerciales
    \item Approvisionnement et marchandises (60): Lié au coût de revient direct des fabrications. Sous déductions des réductions commerciales. (HTVA -> 411 TVA à récupérer)
    \item Services et biens divers (61): lié à l'exploitation de l'entité (coûts indirects)
    \item Rémunérations (62)
    \item Amortissements, réductions de valeur et provisions pour risques et charges (63) 
    \item Autres charges d'exploitation (64)
    \item Charges financières (65) (ex. intérêts)
    \item Charges exceptionnelles (66) (ex. réduction de valeur exceptionnelles)
    \item Impôts (67)
\end{itemize}
\subsection{Affectation des résultats}
A la fin de l'exercice comptable il faut:
\begin{itemize}
    \item Rédiger le compte de résultats (calculer le résultat)
    \item Décider sur l'affectation du résultat (ex. créer des réserves, divivdenes, reporter, ..)
    \item Rediger le bilan en tenant compte de l'affectation du résultat
\end{itemize}
\section*{Chapitre 8. Comptabilité de quelques opérations particulières}
\section{Comptabilité de quelques opérations particulières}
\subsection{Les variations de stocks}
Les stocks forment souvent une composante essentielle de l'actif d'une entité. \\
Cela ne veut pas dire qu'il ne faut pas inclure la variation des stocks dans le compte des résultats. \\
En réalité:
\begin{itemize}
    \item On achète des marchandises (prix d'achat) et on les revends (prix de vente)
    \item On fabrique des produits et on les vends (coût de production <=> prix de vente)
    \item A la fin de l'exercice comptable, pas toutes les unités achetées ou fabriquées sont déjà vendues.
\end{itemize}

\subsubsection*{La démarche comptable}
En cours d'exercice il faut comptabiliser les achats et les ventes. \\
En fin d'exercice il faut intégrer les variations des stocks dans le résultat de \textbf{l'activité}. Dans le compte de résultat de \textbf{l'exercice} on ne ne doit prendre en charge que le coût de production des unités \textbf{vendues}.\\
On a donc: \\
Si le nombre d’unités achetées/fabriquées > le nombre d’unités vendues -> le stock augmente car pas toutes les unités achetées/fabriques cet exercice comptable sont déjà vendues -> le coût de production (prix d’achat) des unités vendues < le coût de production (prix d’achat) des unités fabriquées/achetées -> ce dernier a été enregistré en cours d’exercice, donc on doit enregistrer une diminution des charges d’exploitation à la fin de l’exercice. \\
Si le nombre d’unités fabriquées/achetées < le nombre d’unités vendues -> le stock diminue car on a du prendre des unités fabriquées/achetées et stockées pendant l’exercice précédente -> le coût de production (prix d’achat) des unités vendues > le coût de production (prix d’achat) des unités fabriquées/achetées -> on doit prendre en charge cet exercice une partie du coût de production (prix d’achat) des unités fabriquées/achetées l’exercice précédente, donc on doit enregistrer une augmentation des charges d’exploitation à la fin de l’exercice. \\

\subsubsection*{Valeur des unités de stock}
Si les prix d'achat et le coût de production restent constant durant l'exercice: aucun problème. \\
Si les prix ou les coûts varient durant l'exercice. On a 3 manières de valuriser la consommation de stock:
\begin{itemize}
    \item FIFO (first in, first out): Les premiers stocks seront les premiers vendus.
    \item LIFO (last in, first out): Les derniers stocks seront les premiers vendus.
    \item PMP (prix moyen pondéré): On recalcule avant chaque sortie la valeur d'acquisition moyenne des éléments en stock.
\end{itemize}
Le choix de la méthode de valorisation a un impact sur le résultat opérationnel de l'exercice! Dès qu'on a choisi une méthode de valorisation, on ne peut plus la changer.\\ \\
Voir exemple livre p.93\\
\subsubsection*{Ecriture comptable}
\begin{itemize}
    \item Augmentation, diminution des stocks de marchandise, approvisionnement:
    \begin{itemize}
        \item Classe 3: 300, 310 ou 340
        \item Classe 6: 6094
    \end{itemize}
    \item Augmentation, diminution des stocks de produits finis
    \begin{itemize}
        \item Classe 3: 330
        \item Classe 7: 713
    \end{itemize}
    \item ?
\end{itemize}

\subsection{Les frais de personnel}
(section a retravailler)
\subsubsection*{Calcul des rémunérations et du coût salarial}
Sur le salaire brut, l'employeur retient d'abord les cotisations personnelles de sécurité sociale (13,07pourcent). Ensuite il retient le précompte professionnel. Le solde constitue la rémunération nette du travailleur. \\
\\
Par ailleurs l'employeur paie des cotisations patronales de sécurité sociale (jusqu'à environ 35pourcent du salaire brut), l'assurance accident du travail, d'éventuelles assurances extra-légales, le coût de l'administratif, ...
\\ \\
\subsubsection*{Personnel intérimaire, gérants, adminisatrateurs}
Ils sont tous considérés comme des prestataires des services (617/618). Il n'y a donc pas de côtisation patronale de sécurité sociale à payer ainsi que pas de cotisation ONSS personnelle et précompte professionnel à précompter. 
 
\subsubsection*{Pécules de vacances et gratifications de fin d'année}
La pécule de vacances est basée sur les prestations de l'année précédente, donc à la fin de l'exercice comptable, on doit la considérer comme un coût salarial (623) et une dette latente aux employées (456)\\
On part du même principe pour les gratifications/primes de fin d'année (623/457)

\subsubsection*{résumé}
\begin{center}
    \includegraphics[scale=0.25]{frais_personnel.png}
\end{center}

\subsection{Les amortissements}
Les actifs immobilisés représentent l’infrastructure (la capacité de production) de l’entité qui n’est pas « consommée » immédiatement. L’usage de l’ infrastructure permet de générer des avantages économiques (créer de la valeur) pendant plusieurs années. C’est l’usage de l’infrastructure qui cause une perte de sa valeur. \\
L'amortissement doit refléter l'utilisation réelle d'un actrif. Souvent les amortissement sont aussi inspirés par des considérations fiscales. (=charge déductible)\\
La non-utilisation d'un actif n'exclut pas son amortissement.\\
On amorti:
\begin{itemize}
    \item Les frais d'établissement
    \item Les immobilisations incorporelles
    \item Les immobilisations corporelles (sauf terrains)
\end{itemize}
Par contre, on n'ammorti \textbf{pas} les immobilisations financières. \\

\subsubsection*{Les méthodes d'amortissements}
\begin{itemize}
    \item L'amortissement linéaire: l'amortissement annuel reste le même pendant la durée de vie de l'actif. c'est la méthode la plus simple, mais elle ne reflète pas toujours la réalité économique. Utilisé pour des frais d'établissement, des immobilisations incorporelles, des batiments et des voitures.
    \item L'amortissement basé sur l'utilisation de l'actif: l'amortissement annuel correspont à l'utilisation réelle de la capacité pendant l'année. Eventuellement utilisé pour des installations, machines et outillage. (pas évident d'estimer en avance la capacité réelle)
    \item L'amortissement dégressif:
    \begin{itemize}
        \item Egal au double du taux linéaire correspondant à une durée de dépréciation normale du bien. (2x le taux linéaire)
        \item Base amortissable d'année en année: le solde non-encore amorti au terme de l'exercice précédent. Le taux doublé s'applique sur la valeur résiduelle.
        \item Dotation annuelle maximale = 40pourcent de la valeur d'acquisition.
        \item Retour à l'équivalent linéaire dès que la méthode linéaire aura livré une dotation plus importante.
        \item Utilisé pour des installations, machines, outillage qui perdent la plupart de leur valeur dans les premières années d'usage)
        \item Interdit pour des voitures ou immobilisations corporelles
    \end{itemize}
\end{itemize}
\subsubsection*{Amortissements et réductions de valeur}
Les amortissements et les réductions de valeur ont en commun qu'ils ont pour effet de convertir des débits figurant à l'actif en débits correspondant à des charges. Leurs écritures comptables sont assez voisines.\\ \\
Les amortissements sont annuels et pour des actifs dont l'utilisation est limitée dans le temps. \\ \\
Les réductions de valeurs sont occasionnels et pour des actifs dont l'utilisation n'est pas limitée dans le temps (ex. un terrain) et pour les autres éléments d'actifs (les stocks, les créances et les placements) \\ \\
\subsubsection*{Ecriture comptable générique pour une réduction de valeur}
Augmentation des charges (631-634)\\
Diminution de la valeur des actifs amortis (3..9/409/419/5..9) -> compte correcteur d'actif

\subsection{Les provisions}
Principe des provisions: la "décision" est prise, mais le montant n'est pas encore défini/certain. Donc on est sur de devoir payer, mais on ne sait pas exactement combien. \\
Un provision a pour but de surtout répartir le futur impact comptable d'une grande décision sur plusieurs années. Cela évite un résultat largement négatif dans un des exercices comptables suivants. \\
Une provision a également pour bur d'optimiser la situation fiscale car une provision est déductible.
Il y a les provisions pour prépensions (635/160), pour grosses réparations et entretien (162/636) et provisions pour litiges en cours (673/163). \\
\subsubsection*{Ecriture comptable générique pour les provisions}
Au moment de la constitution de la provision:
\begin{itemize}
    \item Augmentation des charges (635/636637)
    \item Augmentation du passif (160/162/123)
\end{itemize}
Au moment de la survenance effective de la charge:
\begin{itemize}
    \item Diminution du passif (160/162/123) -> utiliser la provision
    \item Diminution des charges (6351/6361/6371) -> neutraliser la charge effectivement subie
    \item
    \item Augmentation des charges effectives (61)
    \item Augmentation des dettes (440)
\end{itemize}
Une provision n'entraine aucune sortie d'argent. C'est une charge pas une dépense. \\
\end{document}
