\documentclass[en]{../../../../../../eplexam}
\usepackage{../../../../../../eplunits}
%\usepackage{amsmath}
%\usepackage{amssymb}

\hypertitle{sigproc}{8}{ELEC}{2900}{2022}{Juin}{All}
{Julien Giunta}
{Laurent Jacques, François Rottenberg and Luc Vandendorpe}

\section{}

We have a signal at $\SI{5}{\mega}$ samples/$\SI{\second}$ and we would like to know the same signal at a rate of $\SI{6}{\mega}$ samples/$\SI{\second}$

\paragraph{1.1} Provide a structure base on polyphase components which implements this sampling rate conversion. Justify.

\paragraph{1.2} We instead would like to obtain a solution based on the DFT. Explain which operations need to be performed by means of DFT/IDFT to obtain this sampling rate conversion. A plot is welcome.

\nosolution

\section{IIR filter design}

In the context of the design of discrete time IIR filters from an analog prototype, explain the main differences that exist between the \textit{impulse invariance} and \textit{bilinear transform} methods. You can stay synthetic but you must define all the concepts you provide.

\nosolution

\section{Kalman filter}

\paragraph{3.1} Given a (hidden) state vector $x_k \in \mathbb{R}^n$ and a measurement vector $y_k \in \mathbb{R}^m$ (with $k$ the discrete time step) related by a linear state-space representation, \textit{explain} the main assumptions supporting the estimation of the state vector $x_k$ and its covariance with a Kalman filter at each $k \geq 0$. Why are they useful ? 

\paragraph{3.2} What are the definition of the (LMMSE) estimates $\hat{x}_{k|k}$ and $\hat{P}_{k|k}$, that is the (a posteriori) estimates of the state vector $x_k$ and its covariance ?

$$\hat{x}_{k|k} = \framebox(80,20){} \;\;\; \hat{P}_{k|k} = \framebox(150,20){}$$

\nosolution

\section{Compressive sensing (CS)}

Let us consider the (noiseless) general linear inverse problem (followed by CS)
$$\mathbf{y} = \mathbf{\Phi x} \in \mathbb{R}^m$$
where $\mathbf{y}$ is a measurement vector encoding (indirect) information about a signal $\mathbf{x} \in \mathbb{R}^n$.

If $m < n$, what is the condition we must impose on the matrix $\mathbf{\Phi}$ to be sure that two distinct $k$-sparse signal $\mathbf{x}, \mathbf{x}'$ are sent to two distinct measurement vectors $\mathbf{y}$ and $\mathbf{y'}$ ? Explain why this condition is appropriate (define all the concepts used in your explanation).

\nosolution

\end{document}
