\documentclass[en]{../../../../../../eplexam}

\usepackage{../../../../../../eplcode}

\hypertitle{Signal processing}{8}{ELEC}{2900}{2019}{Juin}{All}
{Olivier Leblanc \and Julien Verecken}
{Luc Vandendorpe, Benoît Macq and Laurent Jacques}

Exam during 4h with 6 sheets of questions: 2 questions of LV, 3 questions of LJ and one bonus question about the Python sessions.

\section{Questions of Luc Vandendorpe}

\subsection{LV1}

We are given a signal $x_5[n]$ sampled at 5Ms/s, grouped into blocks of length $N$. We would like to produce the signal $x_6[n]$ at 6Ms/s and use the DFT operator in this aim.
\begin{enumerate}
    \item Detail the steps necessary to produce the signal $x_6[n]$.
    \item Discuss the impact of $N$.
\end{enumerate}
We are now interested in the complementary operation which is the conversion of a signal $y_6[n]$ at 6Ms/s into a signal $y_5[n]$ at 5Ms/s.
\begin{enumerate}
    \item Detail the steps necessary to produce the signal $y_5[n]$.
    \item Discuss the impact of $N$.
\end{enumerate}
\nosolution
\subsection{LV2}

\begin{enumerate}
    \item Give the temporal expressions of the polyphase components of type I and II of a signal $h[n]$, written $h_k[n]$ and $\Tilde{h}_k[n]$, respectively. Give the link which exists between them.
    \item Express the spectrum of the polyphase component in terms of the spectrum of the initial signal $H(\e^{j\Omega})$, for type I and II again. Check the validity of the link given in the first part.
\end{enumerate}

\nosolution

\section{Questions of Laurent Jacques}

\subsection{LJ1}

We are given the following difference equation that implements a digital filter which is designed starting from a prototype analog filter.
\[
(2-\sqrt{2})y[n] = x[n] - 2 x[n-1] + x[n-2] + (2+\sqrt{2})y[n-2]
\]
\begin{enumerate}
    \item Recover the analog prototype of this filter.
    \item The low-pass filter used to produce the analog high-pass prototype is a Butterworth filter of order 2, whose cutoff frequency is $\Omega_p = \frac{1}{2}$ has the following expression:
    \[
    H(s)=\frac{1}{(s-\e^{\frac{j3\pi}{4}})(s-\e^{\frac{j5\pi}{4}})}
    \]
    Give the expression of the corresponding high-pass filter\footnote{The spectral transformation of a filter from low-pass to high-pass is given by $F(\hat{s}) = \frac{\Omega_p\hat{\Omega}_p}{\hat{s}}$.}. Give the corresponding value of $\hat{\Omega}_p$ to implement the filter above.
    \item What is the cutoff frequency of the digital filter?
\end{enumerate}
\nosolution
\subsection{LJ2}

\begin{enumerate}
    \item Remind what the conditions are for a system function $H(z)=\dfrac{\sum_{k=0}^M a_k z^{-k}}{\sum_{k=0}^N b_k z^{-k}}$ to be minimum phase.
    
    \item Prove mathematically that any rational system function can be written as the product of a minimum phase filter and an all-pass filter.
    
    \item As a reminder, the phase lag of a filter is defined as:
    \begin{equation*}
        P_L[H](\omega) = - \angle\{H(\e^{j\Omega}) \}
    \end{equation*}
    where $\angle \{\e^{j\Omega}\}=\Omega$.
    
    Knowing that an all-pass filter has a phase lag which is always positive on the range $[0,\pi]$, prove that among the filters which can produce $|H(\e^{j\Omega})|$, the minimum phase filter is the one with the lowest phase lag on the range $[0,\pi]$.
\end{enumerate}
\nosolution
\subsection{LJ3}

We are interested in studying the features of the Haar wavelet transform.

\begin{enumerate}
    \item As a reminder, the \emph{scaling} function is of amplitude 1 between 0 and 1, and the ``mother'' \emph{wavelet} function is of amplitude 1 between 0 and $\frac{1}{2}$ and $-1$ between $\frac{1}{2}$ and 1. Deduce what are the impulse responses of the filters $h[n]$ and $g[n]$ used to apply the fast wavelet transform.

    \item Give the schematic which serves to compute the approximation and detail coefficients of the Fast Wavelet Transform for a 1-dimensional signal of $N=2^J$ coefficients. Suppose we can assume the \emph{wavelet crime} and explain this concept.
    
    \item What is the computational complexity of this transform? Explain briefly.
    
    \item Suppose we have a signal $x[n] \in \Rn$ such that  $x[n] = (x[0],x[1],\dots,x[N-1])$, we build the coefficients $(c[0],c[1],\dots,c[N-1])$, where $c[0]$ is the approximation coefficient at level 0, $c[1]$ is the detail coefficient at level 0, and $(c[2j],\dots,c[2j+1])$ are the detail coefficients at level $j \in (0,J-1)$.
    
    For the following signals, what are the indexes where the coefficients $c$ are non-zero?
    \begin{enumerate}
        \item $x[n] = (-1)^n$
        \item $x[n] = 1,\quad\forall n$
    \end{enumerate}
    Explain your answers.
\end{enumerate}

\nosolution

\section{Python}

This part aimed only to give a bonus going from 0 to 1.

\begin{enumerate}
    \item During the exercise session on image inpainting, what were the names of the algorithms used?
    \item The following code contains a few errors, list them:
    \begin{lstlisting}[language=Python]
    sum = 0
    for n in range(5)
        sum = sum + 2**n
    print(sum)
    \end{lstlisting}
    \item During the session on multi-rate processing, a sinus was decimated and then interpolated. However, after these operations, we observed a slight temporal shift of the signal. What was the cause of this?
    \item Give the properties that are obtained when using a FIR filter. When could an IIR filter be useful?
\end{enumerate}

\nosolution

\end{document}
