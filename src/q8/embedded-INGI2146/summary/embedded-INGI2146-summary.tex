\documentclass[en]{../../../eplsummary}

\hypertitle{embedded-INGI2146}{8}{INGI}{2146}
{Gorby Nicolas Kabasele Ndonda\and Author2\and Author3}
{Professor}

\section{Introduction}
\begin{description}
	\item[Mobile:] Portable devices with wireless communication,
	running stand-alone or client applications.
	\item[Embedded:] An embedded system is a computer system with
	a dedicated function within a larger mechanical or electrical system.
\end{description}
Embedded systems are everywhere in smartphones, cars,\ldots Typical
characteristics is the following:

\begin{itemize}
	\item Cheap
	\item Reduced power consumption
	\item Real-time
	\item Robust
\end{itemize}
To achieve these characteristic, there need to be a tight coupling between
the hardware, the OS and the application.

\paragraph{Real-Time Systems}
Real-Time Systems monitor and have an impact on the physical world via sensors
and actuators. Because of their nature, they have requirements/constraints on timing.
\begin{description}
	\item[Hard constraint] Example Electronic Engine
	\begin{itemize}
		\item Potentially severe consequences if reaction/result is produced after
		deadline.
		\item Results has no value (or even negative value) after deadline
	\end{itemize}
	\item[Firm constraint] Cruise Control
	\begin{itemize}
		\item Occasional miss tolerated, but degrades QoS
		\item No or little value after deadline
	\end{itemize}
	\item[Soft constraint]  User Interface
		\begin{itemize}
			\item Result still has value after deadline
		\end{itemize}
\end{description}

\paragraph{Real-Time Tasks}
\begin{description}
	\item[Periodic:] Must be executed with fixed interarrival time
	\item[Sporadic:] Have known minimum interarrival time
	\item[Aperiodic:] No known interarrival time
\end{description}

\section{Wireless Sensor Networks}
Wireless Sensor Networks are composed of node used for monitoring and they
communicates with each other and/or a remote server.
\subsection{Resource Constraint}
\begin{itemize}
	\item Power
	\begin{itemize}
		\item Batteries and Energy Harvesting
		\item Turn off idle devices
		\item Requires efficient OS and applications
		\item Reduce communication
		\item Duty cycle is the percentage of one period in which a signal or system
		active.
	\end{itemize}
	\item Bandwidth
	\item Memory: just a few kBytes
	\item CPU: a few MHz
	\item Data Transmission
	\begin{itemize}
		\item Multihop wireless Network to reach servers
		\item Self-Organizing Networks
	\end{itemize}
	\item Security: Not enough resources for sophisticated intrusion detection,
	encryption.
	\item Hardware: Must be robust enough to support harsh environment
\end{itemize}
IoT = network of physical objects equipped with electronics and network
connectivity that can be sensed and controlled remotely.

\section{Operating Systems}
\subsection{Hardware}
\begin{description}
	\item[Microcontroller Unit] It contains the
	\begin{itemize}
		\item CPU
		\item Memory (RAM and non-volatile)
		\item Interfaces to sensors and actuators
		\item Communication interfaces and interfaces to external memory
	\end{itemize}
	\item[JTAG/BSL] Allows to write code and data into flash memory or RAM
	of the MCU without using one of the communication interface
	\begin{itemize}
		\item Can be used to debbuging code on the MCU from a PC
		\item Can upload code/data to the memory without any OS or program
		running.
	\end{itemize}
	\item[Transceiver] Used to send data and receive data 
	\item[Interfaces and Sensors] Interfaces to sensor/Actuators
\end{description}

\subsection{Interrupts}
In order to make a software that react to the physical world two strategy:
\begin{description}
	\item[Polling] Read status of I/O every few milliseconds $\to$ waste of
	energy because CPU always busy
	\item[Interrupts] Signal to processor that current code execution should be interrupted
	because something important has happened
	\begin{itemize}
		\item CPU stops execution and jumps to a specific place in the code that is 
		responsible for handling the interrupt
		\item Exist in all CPUs
		\item Handle by OS and device drivers
		\item Interrupts are expensive becauce of context switching
	\end{itemize} 
\end{description}

\subsection{OS for embedded systems}
OS are not necessarly needed but they free the programmer from some concerns
\begin{itemize}
	\item Handle interrupts (events triggered when something happens)
	\item Manage memory
	\item Write your own process scheduler if you need multi-processing for
	background task
\end{itemize}
\paragraph{Note:}
Linux can be used for embedded systems but it requires some memory and is 
really suited for these kind of systems$\to$ preemptive scheduling leads to 
latency for user code as kernel code cannot be preempted.

\subsubsection{Real-Time OS}
OS specifically designed for real-time systems:
\begin{itemize}
	\item Much smaller than desktop/server OS
	\item Reduced set of functionality
	\item Fast context switch
	\item Guaranteed time-bounded response to interrupts
	\item Tries to avoid code that cannot be preempted

\end{itemize}

\end{document}
