
\section{Operating Systems}

OS are not necessary needed but they free the programmer from some concerns
\begin{itemize}
    \item Handle interrupts (events triggered when something happens)
    \item Manage memory
    \item Write your own process scheduler if you need multi-processing for
        background task
\end{itemize}

\paragraph{Linux}
Linux can be used for embedded systems but
\begin{itemize}
    \item requires enough memory 
    \item preemptive scheduling leads to latency for user code as
        kernel code cannot be preempted
\end{itemize}


\subsection{Hardware}

\begin{description}
    \item[Microcontroller Unit]: Contains 
        \begin{itemize}
            \item CPU
            \item Memory (RAM and non-volatile)
            \item Interfaces to sensors and actuators
            \item Communication interfaces and interfaces to external memory
        \end{itemize}

    \item[JTAG] (Joint test action group): Allows to write code and data
        into flash memory or RAM of the MCU without using one of the
        communication interface
    \item[BSL] (Bootstrap Loader): similar JTAG but from a serial interface

        \begin{itemize}
            \item Can be used to debbuging code on the MCU from a PC
            \item Can upload code/data to the memory without any OS or program
                running.
        \end{itemize}

    \item[$\Rightarrow$] JTAG/BSL are useful for completely empty
        device where we can upload code/data without any OS.

    \item[Transceiver] Used to send data and receive data 
    \item[Interfaces and Sensors] Interfaces to sensor/Actuators
\end{description}

\subsection{Interrupts}
In order to make a software that react to the physical world two strategy:

\begin{description}
    \item[Polling] Read status of I/O every few milliseconds $\to$ waste of
        energy because CPU always busy

    \item[Interrupts] Signal to processor that current code execution should be interrupted
        because something important has happened.
        \begin{itemize}
            \item CPU stops execution and jumps to a specific place in the code that is 
                responsible for handling the interrupt
            \item Exist in all CPUs
            \item Handle by OS and device drivers
            \item Interrupts are expensive because of \textbf{context switching}
        \end{itemize} 
\end{description}

\subsection{Real-Time OS}
OS specifically designed for real-time systems:
\begin{itemize}
    \item Much smaller than desktop/server OS
    \item Reduced set of functionality
    \item Fast context switch
    \item Guaranteed time-bounded response to interrupts
    \item Tries to avoid code that cannot be preempted
\end{itemize}

\subsection{OSs for WSN}
\begin{itemize}
    \item Thin abstraction layer to hardware(I/O, timers, \ldots)
    \item Limited multiprocessing
    \item Specific network protocol implementations
    \item Power saving
\end{itemize}

\subsubsection{TinyOS}
TinyOS applications consist of components that you can use in your own
application (as a \textit{set of libraries}). Each component is provided
with an interfaces defining the commands (function) and event available
in the component.

\begin{itemize}
    \item The device sleeps most of the time and they wake up when
        an event is triggered.
    \item Event-handling function execute and then device goes back to sleep.
\end{itemize}

\paragraph{Task}
Tasks are functions that can be preempted by interrupts but not by other tasks.
\begin{itemize}
    \item Task scheduler is a FIFO queue
    \item Not possible to post the same task while it is in the queue
    \item Device sleeps when task queue is empty and no event to process
\end{itemize}

\paragraph{Synchronous vs Asynchronous}
Hardware events can happen at any time and interrupt the current execution.
This can result in race conditions for variables shared between hardware and
task.
Compile differentiates between
\begin{itemize}
    \item Synchronous code: reachable only from tasks
    \item Asynchronous code: reachable from interrupts
\end{itemize}

Static analysis allows compiler to detect race conditions so programmer
are forced to explicitly use atomic section.

\paragraph{Optimization}
\begin{itemize}
    \item TinyOS programmed in NesC (subset of C)$\to$ No pointers or 
        dynamic allocation.
    \item Static analysis is used for optimization $\to$ can reduce apps by 60\%
\end{itemize}

\subsubsection{Contiki}
Differences with TinyOS
\begin{itemize}
    \item No special description language
    \item Cooperative process instead of task queue
    \item Core (kernal + libraries) is uploaded once
    \item Applications can be loaded at run-time
\end{itemize}
Application are composed of one or multiple processes. Proccesses are
cooperative, meaning that other processes can only run if current process
stops.
\begin{itemize}
    \item Processes communicate with each other by posting event
        \begin{itemize}
            \item Synchronous events: delivered immediately
            \item Asynchronous events: delivered later
        \end{itemize}
    \item Processes implemented as Proto-Threads
        \begin{itemize}
            \item Like threads but cheaper
            \item No separate stack per process $\to$ Local variables lose
                their value after a context switch
        \end{itemize}
    \item Process can be preempted by interrupts triggerd by hardware events.
\end{itemize}
\subsubsection{RIOT}
RIOT application programmed in C or C++ and used Multi-threading 
(not event-driven). $\to$ Use priority based scheduling.

\subsection{Networking Stacks}
uIP : Ipv4 stack in 4-5 KBytes of code.
\begin{itemize}
    \item One single packet buffer $\to$ easier to manage
    \item Incoming packet must be processed immediately $\to$ Radio interface queue
        incoming packet while processing.
    \item Retransmission is left to the application to recreate lost
        packet segments (not too bad if we considerer that packet
        contained sensor data, in this case new data are sent)
    \item Send segment one at the time, wait ACK before sending next segment

        $\Rightarrow$ saves memory, simplifies implementation but
        affects performance
    \item One single buffer for IP fragment reassembly
\end{itemize}


