\clearpage{}
\section{Describe elements of project schedules: work breakdown structure,
tasks, deliverables and milestones. Explain the critical path method.
Illustrate on PERT and Gantt charts. Discuss resource planning and
tracking.}

\subsection{Elements of project schedules}

\subsubsection{Work breakdown structure:}
A hierarchical decomposition of a project into tasks.

\begin{figure}[!ht]
    \centering
    \begin{tikzpicture}[node distance=0.5cm,on grid, auto]
[-,thick]
\footnotesize
\node[draw] {Build communications software} [edge from parent fork down]
  child {node[draw, left=0.5cm] {System planning (1.0)}
    child {node[draw] {Review specification (1.1)}
        child {node[draw] {Review budget (1.2)}
            child {node[draw] {Review schedule (1.3)}
                child {node[draw] {Develop plan (1.4)}}
            }
        }
    }
  }
  child {node[draw] {System design (2.0)}
    child {node[draw] {Top-level design (2.1)}
        child {node[draw] {Prototyping (2.2)}
            child {node[draw] {User interface (2.3)}
                child {node[draw] {Detailed design (2.4)}}
            }
        }
    }
  }
  child {node[draw, right=0.4cm] {Coding (3.0)}}
  child {node[draw, right=1cm] {Testing (4.0)}}
  child {node[draw, right=1.6cm] {Delivery (5.0)}};
\end{tikzpicture}

    \caption{Work breakdown structure}
\end{figure}

\subsubsection{Deliverable}
An item to be delivered to the customer (software, documents, technical demonstrations).

\subsubsection{Task (activity)}

A part of the project that takes place over a period of time.

\begin{itemize}
    \item Predecessors (precursors): events that must occur in order for a task to start
    \item Duration: length of time needed to complete a task
    \item Due date: date by which a task must be completed
    \item Endpoint: event marking the end of the task
\end{itemize}

\subsection{PERT Chart}

\begin{figure}[!ht]
    \centering
    \includegraphics[width=\linewidth]{pert_chart.png}
    \caption{PERT Chart}
\end{figure}

Way to represent the arrangement of the different tasks of the project and the relations
among them.

\subsection{Gantt Chart}

\begin{figure}[!ht]
    \centering
    \includegraphics[width=\linewidth]{gantt_chart.png}
    \caption{Gantt Chart}
\end{figure}

Another way to represent the arrangement of the tasks. The X-axis represents the planning
(time duration) of the tasks.

\subsection{Critical Path Method (CPM)}

The CPM is an algorithm for scheduling a set of project activities.
\newline

The critical path shows us the minimum amount of time it will take to complete the project,
given our estimates of each activity’s duration and predecessors. It also reveals those
activities that are most critical to completing the project on time.
\newline
The critical path is the path along which the activities have no slack time. \newline

Gantt and PERT chart are appropriate to apply the algorithm because they contain all the
required information. Although, in my opinion, PERT is a best representation to apply the
algorithm and Gantt is better to show the plan resulting from the algorithm.

\subsubsection{Algorithm:}

\begin{figure}[!ht]
\begin{minipage}{\linewidth}
    \begin{minipage}[t]{0.6\linewidth}
        \begin{lstlisting}[mathescape]
For each task $i$:
    duration $d_i$
    earliest start time $s_i$
    earliest end time $e_i$
Given all $d_i$, compute all $s_i$ and $e_i$
Start task: $s_{START} = 0, e_{START} = d_{START}$
For all tasks $i$ (in predecessor order):
    $s_i = \max \{e_j \mid j \textrm{ is a predecessor of } i\}$
    $e_i = s_i + d_i$
        \end{lstlisting}
    \end{minipage}
    \begin{minipage}{0.35\linewidth}
        \includegraphics[width=\linewidth]{cpm_algorithm.png}
        \caption{CPM Algorithm}
    \end{minipage}
\end{minipage}
    \centering
    \includegraphics[width=\linewidth]{cpm_results.png}
    \caption{CPM results}
\end{figure}

\subsection{Resource planning and tracking}

Consist to estimate the resource usage for each task:

\begin{itemize}
    \item Staff [pers]
    \item Supplies [units/day]
    \item Expenditure [\$/day]
\end{itemize}

Sum the resources required at each time. \newline
Compare to the available resources. \newline
Track actual vs.\ planned resource usage. \newline

\begin{figure}[!ht]
    \centering
    \includegraphics[width=0.8\linewidth]{resource_planning_and_tracking.png}
    \caption{Resource planning and tracking}
\end{figure}
