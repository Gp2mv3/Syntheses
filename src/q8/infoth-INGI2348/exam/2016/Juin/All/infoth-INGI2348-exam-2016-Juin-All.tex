\documentclass[en]{../../../../../../eplexam}

\usepackage{enumitem}
\usepackage{../../../../../../eplunits}

\hypertitle{Information theory and coding}{8}{INGI}{2348}{2016}{Juin}{All}
{John de Wasseige}
{Benoît Macq, Jérôme Louveaux and Olivier Pereira}

\section{BM-1}
\begin{solution}
  We have $\mu_x = \mu_{\epsilon} = 0$ and $D = 4q$.
  % \[
  %   \sigma_\epsilon^2 
  %   = \int_{-\frac{D}{2}}^{\frac{D}{2}} \epsilon^2 p(\epsilon) \dif \epsilon
  % \]
  which is decomposed in
  % 2* integrate (x+q)^2*(1/4)*dx from -D/2 to -D/4; q = D/4
  % 1* integrate x^2*(1/2)*dx from -D/4 to D/4; q = D/4
  \[
    \int_{-\frac{D}{2}}^{-\frac{D}{4}} (x + q)^2 \frac{1}{4} \dif x
    + \int_{-\frac{D}{4}}^{\frac{D}{2}} x^2 \frac{1}{2} \dif x
    + \int_{\frac{D}{4}}^{\frac{D}{4}} (x - q)^2 \frac{1}{4} \dif x
    = \frac{D^3}{768} + \frac{D^3}{192} + \frac{D^3}{768}
    = \frac{D^3}{128}
  \]
  Then
  \[
    \sigma_x^2 = \int_{-\frac{D}{2}}^{\frac{D}{2}} x^2 \frac{1}{D} \dif x
    = \frac{D^2}{12}
  \]
  The \emph{distorsion} is then
  \[
    \frac{\sigma_\epsilon^2}{\sigma_x^2} = \frac{3 D}{32}
  \]
  We now compute the $\tilde{p}$.
  \begin{align*}
    \tilde{p}(-1) &= \frac{D}{4} \\
    \tilde{p}(0) &= \frac{D}{2} \\
    \tilde{p}(1) &= \frac{D}{4}
  \end{align*}
  The \emph{rate} is given by
  \[
    R = - \sum_{k=-1}^{1} p_k \log_2(p_k)
  \]
\end{solution}

\section{BM-2}
\nosolution

\section{BM-3}
\begin{solution}
  \[
    I(X;Y) = H(X) - H(X|Y) = H(Y) - H(Y|X)
  \]
  See course for detailed formulas.
\end{solution}

\section{JL-1}
\begin{solution}
  \begin{enumerate}
    \item The capacity is given by
    \[
      C = \max I(X;Y) = H(Y) - H(Y|X)
    \]
    with a uniform distribution on $X$.
    This gives $C = 0.531$ bits/symbol and $2.124$ Mbits/s.
    \item Only 360p can be transmitted \emph{reliably}, for the others we are
    in the case where
    \[
      \frac{H(U)}{\tau_s} > \frac{C}{\tau_c}
    \]
    \item 
    \[
      R = \frac{\tau_c}{\tau_s} = \frac{10/9}{2.124} \approx 0.523
    \]
    \item No because to obtain almost error-free transmission,
    the length cannot be fixed.
    \item
    \[
      n \geq \frac{\log_2(10^6)}{0.531 - 0.523} \approx 2491.45
    \]
    and thus the required code length is $n = 2492$.
  \end{enumerate}
\end{solution}

\section{JL-2}
\begin{solution}
  \begin{enumerate}
    \item The length is $n=9$, the rank $k=7$, the redundancy $n-k=2$ and the rate $R=7/9$.
    \item Yes. For a code to be MDS, all sets of $n-k$ columns of $H$ have to be
    linearly independent, which is the case here.
    \item No. For it to be a RS code, we need $k < n \leq q$ but $n > q$.
    \item Trying out some possible codewords and looking at what the parity matrix
    implies, we find that the minimal distance is $d(\mathcal{C}) = 3$.
    However, one can also justify it in a more rigorous way by recalling that the minimum
    distance is the largest integer $s$ such that all sets of $s-1$ columns of $H$
    are independent. As 3 columns are not independent, $s=3$ and it confirms what we supposed.
    The correction capability is $t(\mathcal{C}) = 1$.
    \item It is not a perfect as the Hamming bound is not attained, indeed
    \[
      M \leq \frac{2^9}{1 + 9}
    \]
    and equality is not possible.
    \item Since the redundancy is fixed, one needs to make both $k$ and $n$ bigger
    for the rate $k/n$ to grow. However, having a larger $n$ would mean have a 
    parity matrix with more columns, which is not possible as columns would become
    linearly dependent (try adding any column to it), resulting in a non MDSC code.
  \end{enumerate}
\end{solution}

\end{document}
