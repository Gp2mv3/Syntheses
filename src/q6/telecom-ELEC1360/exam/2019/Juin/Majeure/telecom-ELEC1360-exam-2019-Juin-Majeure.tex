\documentclass[fr]{../../../../../../eplexam}

\hypertitle{Télécommunications}{6}{ELEC}{1360}{2019}{Juin}{Majeure}
{Quentin Dessain}
{Luc Vandendorpe}

\section{}
Soit $X(t)$ un signal aléatoire à bande étroite $B$, faiblement stationnaire. On note sa densité spectrale de puissance par $\gamma_{X}(\omega)$ et ses composantes de Rice, extraites autour de la porteuse à pulsation $\omega_{0}$, $X_{1}(t)$ et $X_{2}(t)$.
\begin{enumerate}
    \item Donnez et démontrez les expressions permettant de calculer $X_{1}(t)$ et $X_{2}(t)$ à partir de $X(t)$. 
    \item Démontrez que la densité spectrale des composantes de Rice est donnée par:
\begin{equation}
    \gamma_{X_{1}}(\omega) = \gamma_{X_{2}}(\omega) = \left\{\begin{array}{r c l}
        \gamma_{X}(\omega-\omega_{0}) + \gamma_{X}(\omega+\omega_{0}) & \text{pour} \quad \abs{\omega} \leq \pi B\\
        0 & \text{sinon}
        \end{array}\right.
\end{equation}
\end{enumerate}


\begin{solution}

\begin{enumerate}
    \item On sait qu'un signal aléatoire à bande étroite possède une enveloppe complexe $E_{X}(t) = X_{1}(t) - jX_{2}(t)$ où $X_{1}(t)$ et $X_{2}(t)$ sont les composantes de Rice du signal $X(t)$. De plus, le signal analytique d'un signal peut s'écrire en fonction du signal lui-même et de sa transformée d'Hilbert : $X_{a}(t) = X(t) + \imagj\widehat{X}(t)$ avec $\widehat{X}(t)$ la transformée d'Hilbert du signal $X(t)$. En se rappelant que l'enveloppe complexe est simplement le signal analytique décalé en bande de base, on peut alors retrouver assez facilement l'expression des composantes de Rice en fonction du signal $X(t)$ et sa transformée d'Hilbert. En effet, nous avons :
    \begin{align*}
        X_{1}(t) &= \Re [E_{X}(t)] \\
                 &= \Re [X_{a}(t)\e^{-\imagj\omega _{0}t}]\\
                 &= \Re [(X(t) + \imagj\widehat{X}(t))\e^{-\imagj\omega _{0}t}]\\
                 &= X(t)\cos(\omega _{0}(t)) + \widehat{X}(t)\sin(\omega _{0}(t))\\
        X_{2}(t) &= -\Im [E_{X}(t)] \\
                 &= -\Im [X_{a}(t)\e^{-\imagj\omega _{0}t}]\\
                 &= -\Im [(X(t) + \imagj\widehat{X}(t))\e^{-\imagj\omega _{0}t}]\\
                 &= X(t)\sin(\omega _{0}(t)) - \widehat{X}(t)\cos(\omega _{0}(t))\\
    \end{align*}
    \item Pour cette question, Il faut, dans un premier temps, calculer les fonctions de covariance des composantes de Rice du signal, en se rappelant que $\Gamma _{X} = \Gamma _{\widehat{X}}$ et $\Gamma _{X\widehat{X}} = -\Gamma _{\widehat{X}X}$ :
    \begin{align*}
        \Gamma _{X_{1}}(\tau ) &= \mathbb{E}[X_{1}(t)X^*_{1}(t')]\\
                               &= \mathbb{E}[(X(t)\cos(\omega _{0}(t)) + \widehat{X}(t)\sin(\omega _{0}(t)))(X^*(t')\cos(\omega _{0}(t')) + \widehat{X}^*(t')\sin(\omega _{0}(t')))]\\
                               &= \Gamma _{X}(\tau )\cos(\omega _{0}t)\cos(\omega _{0}t') + \Gamma _{X\widehat{X}}(\tau )\cos(\omega _{0}t)\sin(\omega _{0}t')\\
                               &+ \Gamma _{\widehat{X}X}(\tau )\sin(\omega _{0}t)\cos(\omega _{0}t') + \Gamma _{\widehat{X}}(\tau )\sin(\omega _{0}t)\sin(\omega _{0}t')\\
                               &= \frac{1}{2}\Gamma _{X}(\tau )[\cos(\omega _{0}\tau) + \cos(\omega _{0}(t+t'))] + \frac{1}{2}\Gamma _{X\widehat{X}}(\tau )[\sin(\omega _{0}(t+t')) - \sin(\omega _{0}\tau)]\\
                               &+ \frac{1}{2}\Gamma _{\widehat{X}X}(\tau )[\sin(\omega _{0}(t+t')) - \sin(\omega _{0}(-\tau))] + \frac{1}{2}\Gamma _{\widehat{X}}(\tau )[\cos(\omega _{0}\tau) - \cos(\omega _{0}(t+t'))]\\
                               &= \Gamma _{X}(\tau )\cos(\omega _{0}\tau) 
                               + \Gamma _{\widehat{X}X}(\tau ) \sin(\omega _{0}(\tau))
    \end{align*}
Avec un raisonnement similaire, nous obtenons que $\Gamma _{X_{1}}(\tau ) = \Gamma _{X_{2}}(\tau )$.\\
\\
La densité spectrale d'un signal étant la transformée de Fourier de la fonction de covariance de ce signal, il faut désormais dériver la transformée de Fourrier des fonctions de covariance des composantes de Rice $\Gamma _{X_{1}}(\tau ) = \Gamma _{X_{2}}(\tau )$.\\
Pour cette partie, il faut être particulièrement prudent sur l'étude de signe qui va être réalisée : 
\begin{align*}
    \gamma _{X_{1}} &= \gamma _{X_{2}} = \frac{1}{2\pi }\gamma _{X}\circledast\pi [\delta (\omega +\omega _{0}) + \delta (\omega -\omega _{0})] + \frac{1}{2\pi }(-\imagj sign(\omega ))\gamma _{X}(\omega )\circledast \imagj\pi [\delta (\omega +\omega _{0}) - \delta (\omega -\omega _{0})]\\
    \Leftrightarrow 2\gamma _{X_{1}} &= \gamma _{X}(\omega )\circledast[\delta (\omega +\omega _{0}) + \delta (\omega -\omega _{0})] + sign(\omega )\gamma _{X}(\omega )\circledast[\delta (\omega +\omega _{0}) - \delta (\omega -\omega _{0})]\\
    & = \gamma _{X}(\omega + \omega _{0}) + \gamma _{X}(\omega - \omega _{0}) + sign(\omega + \omega _{0})\gamma _{X}(\omega + \omega _{0}) - sign(\omega - \omega _{0})\gamma _{X}(\omega - \omega _{0})
\end{align*}
Or 
\begin{align*}
    sign(\omega +\omega _{0})& = \begin{cases}
 & +1 \text{ si } \omega > -\omega _{0} \\ 
 & -1 \text{ si } \omega < -\omega _{0}
\end{cases} \\
-sign(\omega -\omega _{0}) &= \begin{cases}
 & -1 \text{ si } \omega > \omega _{0} \\ 
 & +1 \text{ si } \omega < \omega _{0}
\end{cases}
\end{align*}
Donc
\begin{align*}
    \gamma_{X_{1}}(\omega) = \gamma_{X_{2}}(\omega) = \left\{\begin{array}{r c l}
        \gamma_{X}(\omega-\omega_{0}) + \gamma_{X}(\omega+\omega_{0}) & \text{pour} \quad \abs{\omega} \leq \pi B\\
        0 & \text{sinon}
        \end{array}\right.
\end{align*}
\end{enumerate}

\end{solution}

\section{}
Soit un signal d'information $m(t)$ réel que l'on souhaite moduler en BLU (SSB) autour d'une porteuse de pulsation $w_{0}$.
\begin{enumerate}
    \item Montrer comment on fabrique un signal SSB qui contient uniquement les fréquences positives du signal $m(t)$. Expliquez.
    \item Montrer comment on fabrique un signal SSB qui contient uniquement les fréquences négatives du signal $m(t)$. Expliquez.
    \item Soit un signal noté $\hat{m}(t)$ qui représente la transformée de Hilbert de $m(t)$. Qu'obtient-on si l'on applique à nouveau une transformée de Hilbert à $\hat{m}(t)$ ? Démontrez-le.
    \item On part d'un signal modulé en DSB $x(t) = m(t) \sin(w_{0}t)$. Quel signal du type $v(t) \cos(w_0 t)$ faut-il ajouter (additionner) à $x(t)$ pour obtenir que $x(t) + v(t) cos(w_{0}t)$ soit une modulation SSB. Autrement dit, comment choisir $v(t)$ pour obtenir un résultat qui soit SSB ? Justifiez.
\end{enumerate}

\begin{solution}
\begin{enumerate}
    \item Le signal analytique, qui ne contient donc que les fréquences positives du signal de départ (avec une amplitude multiplié par 2 pour être à puissance constante), est obtenu en ajoutant au signal de départ $\imagj$ multiplié par le signal en quadrature. Pour donc gommer les fréquences négatives de $m(t)$, il faut constituer le signal analytique $m_a(t)$ donné par $m_a(t) = m(t) + \imagj\hat{m}(t)$ où $\hat{m}(t)$ est le signal en quadrature de $m(t)$ ou sa transformée de Hilbert.
     L’on déplace ensuite ce lobe latéral à droite de la porteuse (dans le domaine spectral) et cela se fait en l’affectant d’une exponentielle complexe, soit \[s_+(t) = m_a(t) e^{\imagj\omega_c t}\]
     Il faut ensuite suivre des étapes semblables avec les fréquences négatives du signal modulant. L’on obtient les fréquences négatives du signal modulant en le passant dans un filtre de transmittance $U(-\omega)$. Le signal temporel correspondant, noté $m_a(t)^*$ est donné par:
     \begin{align*}
         m_a(t)^* &= m(t) \otimes u(-t)\\
         &= m(t) \otimes [\delta(-t) - \frac{\imagj}{\pi t}]\\
         &= m(t) - \imagj m(t) \otimes [\frac{1}{\pi t}]\\
         &= m(t) -\imagj \hat{m}(t)\\
     \end{align*}
     Ce signal-ci doit être déplacé autour de $-\omega_c$. Le signal SSB complet est donc donné par:

    \begin{align*}
        x_{Upper SSB}(t) &= m_a(t) \e^{\imagj\omega_c t} + {m}_a(t)^* \e^{-\imagj\omega_c t}\\
        &= ( m(t) + \imagj \hat{m}(t)) \e^{\imagj\omega_c t} + ( m(t) - \imagj \hat{m}(t)) \e^{-\imagj\omega_c t}\\
        &= [ m(t) + \imagj \hat{m}(t)] [\cos(\omega_c t) + \imagj \sin(\omega_c t) ] + [ m(t) - \imagj \hat{m}(t)] [\cos(\omega_c t) - \imagj \sin(\omega_c t) ]\\
        &= 2 m(t) \cos(\omega_c t) - 2 \hat{m}(t)  \sin(\omega_c t)\\
    \end{align*}
    \item
    Le signal analytique, qui ne contient donc que les fréquences négatives du signal de départ (avec une amplitude multiplié par 2 pour être à puissance constante), est obtenu en ajoutant au signal de départ $\imagj$ multiplié par le signal en quadrature. Pour donc gommer les fréquences négatives de $m(t)$, il faut constituer le signal analytique $m_a(t)$ donné par $m_a(t) = m(t) + \imagj\hat{m}(t)$ où $\hat{m}(t)$ est le signal en quadrature de $m(t)$ ou sa transformée de Hilbert.
     L’on déplace ensuite ce lobe latéral à gauche de la porteuse (dans le domaine spectral) et cela se fait en l’affectant d’une exponentielle complexe négative, soit \[s_+(t) = m_a(t) \e^{-\imagj\omega_c t}\]
     Il faut ensuite suivre des étapes semblables avec les fréquences négatives du signal modulant. L’on obtient les fréquences négatives du signal modulant en le passant dans un filtre de transmittance $U(-\omega)$. Le signal temporel correspondant, noté $m_a(t)^*$ est donné par:
     \begin{align*}
         m_a(t)^* &= m(t) \otimes u(-t)\\
         &= m(t) \otimes [\delta(-t) - \frac{\imagj}{\pi t}]\\
         &= m(t) - \imagj m(t) \otimes [\frac{1}{\pi t}]\\
         &= m(t) -\imagj \hat{m}(t)\\
     \end{align*}
     Ce signal-ci doit être déplacé autour de $\omega_c$. Le signal SSB complet est donc donné par:

    \begin{align*}
        x_{Upper SSB}(t) &= m_a(t) \e^{-\imagj\omega_c t} + {m}_a(t)^* \e^{\imagj\omega_c t}\\
        &= ( m(t) + \imagj \hat{m}(t)) \e^{-\imagj\omega_c t} + ( m(t) - \imagj \hat{m}(t)) \e^{\imagj\omega_c t}\\
        &= [ m(t) + \imagj \hat{m}(t)] [\cos(\omega_c t) - \imagj \sin(\omega_c t) ] + [ m(t) - \imagj \hat{m}(t)] [\cos(\omega_c t) + \imagj \sin(\omega_c t) ]\\
        &= 2 m(t) \cos(\omega_c t) + 2 \hat{m}(t)  \sin(\omega_c t)\\
    \end{align*}
    \item En passent $\widehat{\hat{m}(t)}$ en fréquentielle et en effectuant les simplification nécessaire, on trouve finalement une expression qui repassé en temporelle donne: $\widehat{\hat{m}(t)} = - m(t)$
    
    \item 
    Pour que $v(t) \cos(\omega_0 t) + m(t) \sin(\omega_0 t)$ soit un signal SSB, il faut que $m(t) = \hat{v}(t)$. Si on applique hilbert des deux côtés de l'équation on trouve: 
    \[\hat{m}(t) = \widehat{\hat{v}(t)}\]
    En reprenant le résultat de l'exercices 3 on a:
    \[\hat{m}(t) = - v(t) \]
    \[v(t)=-\hat{m}(t) \]
    Le signal SSB vaut donc:
    \[V_{SSB}= -m(t) \cos(\omega_0 t) + m(t) \sin(\omega_0 t)\]
\end{enumerate}
\end{solution}


\section{}
Un flux de données binaires est modulé et transmis (en bande de base) à l’aide des deux fonctions de base $s_0(t)$, qui correspond à l'envoi d'un "0", et $s_1(t)$ qui correspond à l'envoi d'un "1". Ces deux fonctions de base $s_0(t)$ et $s_1(t)$ sont à priori quelconques mais satisfont aux propriétés suivantes:

\begin{enumerate}[label=-]
    \item Elles ont des énergies $\epsilon_0$ et $\epsilon_1$ qui peuvent être différentes.
    \item Elles ont à priori une corrélation $C_{01}$ quelconque, où $C_{01} = \int^{+\infty}_{-\infty} s_{0}(\tau)s_{1}(\tau) \dif\tau $ 
\end{enumerate}



On étudie la situation où la transmission est corrompue par un BBGA $w(t)$ de densité spectrale bilatérale $\frac{N_0}{2}$. Le signal reçu $x(t)$ est décrit par
\begin{equation}
    X(t) = \left\{\begin{array}{r c l}
       x_0(t) = s_0(t) + \omega(t) & \text{pour l'envoi d'un "0"}\\
        x_1(t) = s_1(t) + \omega(t) & \text{pour l'envoi d'un "1"}
        \end{array}\right.
\end{equation}

On s'intéresse à la probabilité d'erreur.

\begin{enumerate}
    \item Définissez proprement la probabilité d’erreur. Les symboles sont équiprobables.
    \item Proposez une structure de réception qui soit optimale au sens de la minimisation de cette probabilité. Expliquez et justifiez.
    \item Donnez la variable de décision avec laquelle vous travaillez (mathématiquement, exprimée à partir de $x(t)$) et dites quelle est votre règle de décision.
    \item Calculez la probabilité d’erreur que vous obtenez pour la structure et la règle proposées au point précédent.
    \item Vérifiez ce que donne votre résultat si vous supposez $\epsilon_0 = \epsilon_1$ et si vous prenez une valeur connue de $C_{01}$ pour laquelle vous connaissez la probabilité d’erreur (par exemple signaux opposés ou orthogonaux).
\end{enumerate}

\nosolution


\end{document}
