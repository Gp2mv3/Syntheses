\documentclass[fr]{../../../../../../eplexam}

\hypertitle{Télécommunications}{6}{ELEC}{1360}{2019}{Juin}{Majeure}
{Quentin Dessain}
{Luc Vandendorpe}

\section{Soit $X(t)$ un signal aléatoire à bande étroite B, faiblement stationnaire. On note sa densité spectrale de puissance par $\gamma_{X}(w)$ et ses composantes de Rice, extraites autour de la porteuse à pulsation $w_{0}$, $X_{1}(t)$ et $X_{2}(t)$.}
\begin{enumerate}
    \item Donnez et démontrez les expressions permettant de calculer $X_{1}(t)$ et $X_{2}(t)$ à partir de $X(t)$. 
    \item Démontrez que la densité spectrale des composantes de Rice est donnée par:
\begin{equation}
    \gamma_{X_{1}}(w) = \gamma_{X_{2}}(w) = \left\{\begin{array}{r c l}
        \gamma_{X}(w-w_{0}) + \gamma_{X}(w+w_{0}) & \text{pour} \quad |w| \leq \pi B\\
        0 & \text{sinon}
        \end{array}\right.
\end{equation}
\end{enumerate}


\nosolution

\section{Soit un signal d'information $m(t)$ réel que l'on souhaite moduler en BLU (SSB) autour d'une porteuse de pulsation $w_{0}$.}

\begin{enumerate}
    \item Montrer comment l'on fabrique un signal SSB qui contient uniquement les fréquences positives du signal $m(t)$. Expliquez.
    \item Montrer comment l'on fabrique un signal SSB qui contient uniquement les fréquences négatives du signal $m(t)$. Expliquez.
    \item Soit un signal noté $\hat{m}(t)$ qui représente la transformée de Hilbert de $m(t)$. Qu'obtient-on si l'on applique à nouveau une transformée de Hilbert à $\hat{m}(t)$? Démontrez-le.
    \item On part d'un signal modulé en DSB $x(t) = m(t) sin(w_{0}t)$. Quel signal du type $v(t) cos(w_0 t)$ faut-il ajouter (additionner) à $x(t)$ pour obtenir que $x(t) + v(t) cos(w_{0}t)$ soit une modulation SSB. Autrement dit, comment choisir $v(t)$ pour obtenir un résultat qui soit SSB? Justifiez.
\end{enumerate}

\nosolution


\section{Un flux de données binaires est modulé et transmis (en bande de base) à l’aide des deux fonctions de base $s_0(t)$, qui correspond à l'envoi d'un "0", et $s_1(t)$ qui correspond à l'envoi d'un "1". Ces deux fonctions de base $s_0(t)$ et $s_1(t)$ sont à priori quelconques mais satisfont aux propriétés suivantes:}


\begin{enumerate}[label=-]
    \item Elles ont des énergies $\epsilon_0$ et $\epsilon_1$ qui peuvent être différentes.
    \item Elles ont à priori une corrélation $C_{01}$ quelconque, où $C_{01} = \int^{+\infty}_{-\infty} s_{0}(\tau)s_{1}(\tau) d\tau $ 
\end{enumerate}



On étudie la situation où la transmission est corrompue par un BBGA $w(t)$ de densité spectrale bilatérale $\frac{N_0}{2}$. Le signal reçu $x(t)$ est décrit par
\begin{equation}
    X(t) = \left\{\begin{array}{r c l}
       x_0(t) = s_0(t) + \omega(t) & \text{pour l'envoi d'un "0"}\\
        x_1(t) = s_1(t) + \omega(t) & \text{pour l'envoi d'un "1"}
        \end{array}\right.
\end{equation}

On s'intéresse à la probabilité d'erreur.

\begin{enumerate}
    \item Définissez proprement la probabilité d’erreur. Les symboles sont équiprobables.
    \item Proposez une structure de réception qui soit optimale au sens de la minimisation de cette probabilité. Expliquez et justifiez.
    \item Donnez la variable de décision avec laquelle vous travaillez (mathématiquement, exprimée à partir de x(t)) et dites quelle est votre règle de décision.
    \item Calculez la probabilité d’erreur que vous obtenez pour la structure et la règle proposées au point précédent.
    \item Vérifiez ce que donne votre résultat si vous supposez $\epsilon_0 = \epsilon_1$ et si vous prenez une valeur connue de $C_{01}$ pour laquelle vous connaissez la probabilité d’erreur (par exemple signaux opposés ou orthogonaux).
\end{enumerate}

\nosolution


\end{document}
