\documentclass[fr]{../../../../../../eplexam}
\usepackage{enumitem}

\hypertitle{Télécommunications}{6}{ELEC}{1360}{2019}{Juin}{Mineure}
{François De Keersmaeker}
{Luc Vandendorpe}

\section{Question 1}
Soit un bruit blanc gaussien $N(t)$ satisfaisant la propriété de spectre étroit :
\[ N(t) = N_I(t)\cos(\omega_ct) + N_Q(t)\sin(\omega_ct) \]
\begin{enumerate}
	\item Exprimer la covariance entre les deux composantes de Rice $\Gamma_{N_IN_Q}(\tau)$ en fonction de la covariance de $N$, $\Gamma_N(\tau)$.
	\item \'Evaluer cette covariance en $\tau=0$ et justifier.
\end{enumerate}

\nosolution

\section{Question 2}
On veut transmette des bits. Au lieu d'utiliser le QPSK, on se propose de tester la SSB. Soit les signaux $x_1(t)$ et $x_2(t)$ tels que
\[ x_1(t) = \sum_n a[n] u(t-nT) \]
\[ x_2(t) = \sum_n b[n] u(t-nT) \]
où les éléments de $a$ et $b$ valent 1 ou $-1$, et $u$ est le filtre.
\begin{enumerate}
	\item Soit $x(t) = x_1(t)cos(\omega_0t) + x_2(t)sin(\omega_0t)$. Donner le signal $x_t(t)$ (en fonction de $x_1(t)$ et $x_2(t)$) tel que $x_c(t) = x(t) + x_t(t)$ corresponde à la transmission (en fréquences positives) du lobe de droite de $x_1(t)$ et du lobe de gauche de $x_2(t)$.
	\item Donner la méthode de génération et l'expression du signal $x_t(t)$ en fonction des symboles $a[n]$ et $b[n]$.
\end{enumerate}

\nosolution

\section{Question 3}
Transmission de bits. Les 0 sont représentés par $s_0(t)$ et les 1 par $s_1(t)$. On a les informations suivantes :
\begin{itemize}
	\item Les deux signaux ont la même énergie : $\varepsilon_1 = \varepsilon_2 = \varepsilon$.
	\item Les deux signaux ont une covariance quelconque : $C_{01} = \int_{-\infty}^{+\infty} s_0(t)s_1(t)\,\mathrm{d}t$.
	\item Les 0 et les 1 sont équiprobables.
	\item Le signal reçu, noté $x(t)$, a subi un bruit blanc gaussien additif $w(t)$ sur le canal. Il vaut donc $s_0(t) + w(t)$ si un 0 a été envoyé, et $s_1(t) + w(t)$ si un 1 a été envoyé.
\end{itemize}

\begin{enumerate}
	\item Définir la probabilité d'erreur.
	\item Proposer une structure de réception qui minimise la probabilité d'erreur. Expliquer et justifier.
	\item Donner la variable de décision (en fonction de $x(t)$) et le seuil.
	\item Calculer la probabilité d'erreur pour la structure et la variable définies.
\end{enumerate}
On s'intéresse au cas $s_0(t) = A \sin(2\pi f_0 t$ et $s_1 = A \sin(2 \pi f_1 t)$, ou les deux sinus sont évalués sur une durée $T$. On a $1/f_0 \ll T$ et $1/f_1 \ll T$. Puisque $f_0$ et $f_1$ sont grandes, $f_0 + f_1$ l'est aussi, ce qui implique que certains termes pourront être négligés lors des calculs. Cependant, $f_0 - f_1$ peut être petit.
\begin{enumerate}[resume]
	\item Donner $C_{01}$ et $\rho$ (tels que définis dans le cours).
	\item Donner la (les) condition(s) sur $f_0$ et $f_1$ pour que la probabilité d'erreur soit optimale.
\end{enumerate}

\nosolution

\end{document}
