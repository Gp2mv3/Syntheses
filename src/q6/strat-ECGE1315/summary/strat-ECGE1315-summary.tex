\documentclass[fr]{../../../eplsummary}

\hypertitle{Stratégie d'entreprise}{6}{ECGE}{1315}
{Florian Thuin}
{Vincent Meurisse}

\section{Introduction à la stratégie}

\subsection{Définitions}

La \textbf{stratégie} est l'orientation à \textit{long terme} d'une
organisation. Elle est définie par des décisions \textbf{délibérées} et
\textbf{rationnelles}, \textbf{émergentes} et \textbf{incrémentales}.

Typiquement, un stratégie consiste à développer un avantage
concurrentiel durable, une coopération avec d'autres organisations  ou
une imitation de celles-ci.

\textbf{DAS} : Domaine d'activité stratégique. Dans une entreprise de
grande taille, on divise l'entreprise en parties pour lesquelles on
appliquera une stratégie différentiée (ex: Virgin Airlines et Virgin
Music n'ont pas la même stratégie).

\subsection{Horizons}

\paragraph{Management de l'analyse} L'horizon 1 consiste à étendre et à
défendre l'activité principale.

\paragraph{Management de l'exploration} L'horizon 2 consiste à
construire des activités émergentes.

\paragraph{Management de l'imagination} L'horizon 3 consiste à créer des
options viables de développement.

\subsection{Caractéristiques}

Une bonne stratégie a pour objectifs de satisfaire toutes les parties
prenantes en obtenant et développant un \textbf{avantage concurrentiel
durable}. Elle consiste à allouer des ressources qui engage
l'organisation dans le long terme créant ainsi un \textbf{périmètre d'activité}.

Elle doit :
\begin{enumerate}
    \item entraîner un \color{red} surcroit de valeur \color{black} pour les clients
    \item définir un \color{red} modèle économique difficilement
        imitable \color{black}
\end{enumerate}

\subsection{Le modèle VIP}

\begin{description}
    \item[V]aleur : définir le modèle de création de valeur pour les
        parties prenantes (\textit{value proposition}).
    \item[I]mitation : la value proposition doit être difficilement
        imitable pour assurer l'avantage concurrentiel \textbf{durable}.
    \item[P]érimètre : définir un périmètre (que faire ou non : marchés
        et activités)
\end{description}

\subsubsection{Exemple Ikea}

\begin{description}
    \item[Valeur] : bas prix et tout au même endroit (économie
        d'échelle), stockage facile, repartir directement avec ce qu'on
        achète, très large gamme de produits, services annexes
    \item[Imitation] : impossible de faire la même chose à grande
        échelle ou à prix inférieur, designers exclusifs, adaptations
        locales
    \item[Périmètre] : activité : limité au mobilier d'intérieur et à la
        présentation ; marché : international
\end{description}

\subsection{Les niveaux de stratégie}

\begin{enumerate}
    \item Stratégie d'entreprise (CORPORATE) : stratégie globale de
        l'entreprise
    \item Stratégie par domaine d'activité (BUSINESS) : une entreprise
        de grande taille est divisée en DAS qui auront chacun une
        stratégie différentiée basée sur la stratégie de l'entreprise.
    \item Décisions opérationnelles : déploiement  de la stratégie
        décidée par des décisions précises
\end{enumerate}

\subsection{Formuler une stratégie}

Une stratégie doit définir les buts fondamentaux d'une organisation,
autrement dit :

\begin{description}
    \item[La mission] : expression du but général de l'organisation
    \item[La vision] : état futur souhaité pour l'organisation
    \item[Les objectifs] : précis en horizon temporel et quantitatifs
    \item[Le périmètre d'activité] : notre métier et nos clients
    \item[L'avantage concurrentiel] : ce qui fait qu'on ajoute plus de
        valeur au produit qu'un concurrent (selon le client)
\end{description}

\subsubsection{Objectifs SMART}

Tout objectif défini par une entreprise doit être SMART, autrement dit :

\begin{description}
    \item[S]pécifique
    \item[M]esurable
    \item[A]tteignable
    \item[R]éaliste
    \item[T]emporel
\end{description}

\subsection{Etablir une stratégie}

Une stratégie se définit par trois composantes interdépendantes :

\begin{description}
    \item[Le diagnostic stratégique] : définit le contexte de l'organisation
    (culturel, concurrentiel, économique,\ldots)
    \item[Le déploiement stratégique] : définit les processus utilisés
    \item[Les choix stratégiques] : définit le contenu de la stratégie
\end{description}

\section{Le diagnostic stratégique}



\end{document}
