\documentclass[fr]{../../../eplsummary}

\hypertitle{Stratégie d'entreprise}{6}{ECGE}{1315}
{Florian Thuin}
{Vincent Meurisse}

\section{Introduction à la stratégie}

\subsection{Définitions}

La \textbf{stratégie} est l'orientation à \textit{long terme} d'une
organisation. Elle est définie par des décisions \textbf{délibérées} et
\textbf{rationnelles}, \textbf{émergentes} et
\textbf{incrémentales}.\newline

Typiquement, un stratégie consiste à développer un avantage
concurrentiel durable, une coopération avec d'autres organisations  ou
une imitation de celles-ci.\newline

\textbf{DAS} : Domaine d'activité stratégique. Dans une entreprise de
grande taille, on divise l'entreprise en parties pour lesquelles on
appliquera une stratégie différentiée (ex: Virgin Airlines et Virgin
Music n'ont pas la même stratégie).\newline

\textbf{Degré de concentration} : plus le nombre d'entreprises à exercer
une activité est faible, plus le degré de concentration est élevé.
Autrement dit, un degré de concentration élevé entraîne un pouvoir de
négociation fort des personnes qui exercent ces activités.\newline

\textbf{Intégration vers l'amont} : fusion ou rachat d'un
fournisseur\newline

\textbf{Intégration vers l'aval} : fusion ou rachat d'un client\newline

\textbf{Coût de transfert} : prix à payer pour passer d'un collaborateur
à un autre (par ex: changer de fournisseur). Un prix faible entraîne une
intensité concurrentielle plus forte, on peut augmenter le prix en liant
les clients par des contrats avec clause de résiliation.\newline

\textbf{Facteur clé de succès (FCS)} : éléments stratégiques qu'une
organisation doit maîtriser afin de surpasser la concurrence.
Généralement lié à la création de valeur pour le client, ils permettent
de contrecarrer les forces (modèle de Porter) de la concurrence.

\subsection{Horizons}

\paragraph{Management de l'analyse} L'horizon 1 consiste à étendre et à
défendre l'activité principale.

\paragraph{Management de l'exploration} L'horizon 2 consiste à
construire des activités émergentes.

\paragraph{Management de l'imagination} L'horizon 3 consiste à créer des
options viables de développement.

\subsection{Caractéristiques}

Une bonne stratégie a pour objectifs de satisfaire toutes les parties
prenantes en obtenant et développant un \textbf{avantage concurrentiel
durable}. Elle consiste à allouer des ressources qui engage
l'organisation dans le long terme créant ainsi un \textbf{périmètre d'activité}.

Elle doit :
\begin{enumerate}
    \item entraîner un \color{red} surcroit de valeur \color{black} pour les clients
    \item définir un \color{red} modèle économique difficilement
        imitable \color{black}
\end{enumerate}

\subsection{Le modèle VIP}

\begin{description}
    \item[V]aleur : définir le modèle de création de valeur pour les
        parties prenantes (\textit{value proposition}).
    \item[I]mitation : la value proposition doit être difficilement
        imitable pour assurer l'avantage concurrentiel \textbf{durable}.
    \item[P]érimètre : définir un périmètre (que faire ou non : marchés
        et activités)
\end{description}

\subsubsection{Exemple Ikea}

\begin{description}
    \item[Valeur] : bas prix et tout au même endroit (économie
        d'échelle), stockage facile, repartir directement avec ce qu'on
        achète, très large gamme de produits, services annexes
    \item[Imitation] : impossible de faire la même chose à grande
        échelle ou à prix inférieur, designers exclusifs, adaptations
        locales
    \item[Périmètre] : activité : limité au mobilier d'intérieur et à la
        présentation ; marché : international
\end{description}

\subsection{Les niveaux de stratégie}

\begin{enumerate}
    \item Stratégie d'entreprise (CORPORATE) : stratégie globale de
        l'entreprise
    \item Stratégie par domaine d'activité (BUSINESS) : une entreprise
        de grande taille est divisée en DAS qui auront chacun une
        stratégie différentiée basée sur la stratégie de l'entreprise.
    \item Décisions opérationnelles : déploiement  de la stratégie
        décidée par des décisions précises
\end{enumerate}

\subsection{Formuler une stratégie}

Une stratégie doit définir les buts fondamentaux d'une organisation,
autrement dit :

\begin{description}
    \item[La mission] : expression du but général de l'organisation
    \item[La vision] : état futur souhaité pour l'organisation
    \item[Les objectifs] : précis en horizon temporel et quantitatifs
    \item[Le périmètre d'activité] : notre métier et nos clients
    \item[L'avantage concurrentiel] : ce qui fait qu'on ajoute plus de
        valeur au produit qu'un concurrent (selon le client)
\end{description}

\subsubsection{Objectifs SMART}

Tout objectif défini par une entreprise doit être SMART, autrement dit :

\begin{description}
    \item[S]pécifique
    \item[M]esurable
    \item[A]tteignable
    \item[R]éaliste
    \item[T]emporel
\end{description}

\subsection{Etablir une stratégie}

Une stratégie se définit par trois composantes interdépendantes :

\begin{description}
    \item[Le diagnostic stratégique] : définit le contexte de l'organisation
    (culturel, concurrentiel, économique,\ldots)
    \item[Le déploiement stratégique] : définit les processus utilisés
    \item[Les choix stratégiques] : définit le contenu de la stratégie
\end{description}

\section{Le diagnostic stratégique}

Un diagnostic stratégique consiste à mettre en lumière le contexte dans
lequel l'entreprise évolue pour mieux comprendre comment développer une
stratégie viable à long terme.

\subsection{Analyse PESTEL}

Le modèle PESTEL met en avant les \textit{variables pivots} de
l'évolution du macroenvironnement (une seule analyse est nécessaire pour
toute l'entreprise).

\begin{description}
    \item[P]olitiques : rôle des pouvoirs publics
    \item[E]conomiques : facteurs macroéconomiques (taux d'intérêt, PIB)
    \item[S]ociologiques : évolutions culturelles/démographiques
    \item[T]echnologiques : impact des innovations
    \item[E]nvironnementales : préoccupations écologiques
    \item[L]égales : contraintes juridiques
\end{description}

On liste tout ce qui rend en compte (positivement et négativement) et
ensuite on définit des facteurs majeurs (variables pivots, \textit{key
drivers}) ce qui va nous permettre de définir 2 ou 4 scénarios possibles
pour le secteur d'activité.

\subsection{Analyse des 5+1 forces (Porter)}

Le modèle de Porter permet d'évaluer l'attractivité d'une industrie en
termes d'\textbf{intensité concurrentielle} (une analyse est nécessaire
par DAS). \newline

Plus l'intensité de ces forces est élevée, plus le secteur est
inattractif :

\begin{itemize}
    \item[1.] Les \color{green!80!black} entrants \color{black} potentiels
    \item[2.] Le pouvoir de négociation des
        \color{green!80!black}acheteurs\color{black}
    \item[3.] Le pouvoir de négociation des
        \color{green!80!black}fournisseurs\color{black}
    \item[4.] La menace des \color{green!80!black}produits de
        substitution\color{black}
    \item[5.] \textbf{L'intensité concurrentielle}
    \item[+1] Le rôle des \color{green!80!black}pouvoirs publics
\end{itemize}

\bigskip
Une entreprise doit prendre en compte les \textbf{barrières} à l'entrée et à la
sortie, autrement dit les coûts qu'a une entreprise pour entrer dans le
secteur ainsi que les coûts pour en sortir (barrières financières,
commerciales ou de ressources et compétences). Si une entreprise a des
coûts fixes important (secteur de l'aéronautique) ou des coûts de sortie
important (grande présence des syndicats ou de l'Etat), alors le secteur
est moins attractif. \newline

L'analyse conduit à définir si l'entreprise se situe dans un monopole,
un oligopole, une hyper-compétition ou une concurrence parfaite.
\newline

Les résultats de l'analyse peuvent être placés dans un \textit{hexagone
sectoriel} qui permettent de voir rapidement les résultats. \newline

Les principaux défauts de ce modèle est qu'il est \textbf{statique} et qu'il
ne prend pas en compte la possibilité de coopération. \newline

\subsection{Analyse du cycle de vie}

Un avantage concurrentiel est toujours temporaire et aucune strztégie
n'assure un succès définitif. En effet, chaque produit suit un cycle de
vie. \newline

Le modèle des 5+1 forces peut être mis en parallèle au cycle de vie du
produit :

\begin{tabular}{c|l|l}
Etapes de la vie & Description & Modèle 5+1 forces \\ \hline \hline
\textbf{Emergence} & Découverte du produit & Rivalité faible (différenciation) \\
\hline
\textbf{Croissance} & Début de l'achat du produit & Rivalité faible (faible
pouvoir des acheteurs) \\ \hline
\textbf{Sélection} & Le produit est connu & Rivalité croissante (concurrence) \\
\hline
\textbf{Maturité} & Tout le monde en achète & Rivalité intense (concurrence
forte) \\ \hline
\textbf{Déclin} & Tout le monde en a déjà & Rivalité extrême (concurrence sur les
prix) \\ \hline
\end{tabular}

\subsection{Les groupes stratégiques}

Lors de l'analyse, gardez à l'esprit que certains acteurs du marché ne
sont pas pour autant vos concurrents. En effet, il y a ce qu'on appelle
des \textbf{facteurs de concurrence} qui font que \textit{Ferrari} et
\textit{Renault},
bien que tous deux producteurs de voiture, ne sont pas en concurrence.
\newline

On place souvent les acteurs sur un même marché dans un espace à 2
dimensions en prenant des axes dépendant du secteur (p.e: selon les
investissements en recherche et développement, selon les zones
géographiques, selon les prix, selon le prestige,\ldots). En effet, il
existe différents \textbf{segments de marché} dont les caractéristiques
diffèrent selon qu'on soit en B to C ou en B to B. \newline

\subsubsection{L'approche Océan Bleu}

Si on découvre qu'il existe un segment de marché qui n'a pas encore été
exploité, c'est un \color{blue!70!black} Océan Bleu \color{black}. Au
contraire, si le segment est bien connu et très concurrentiel, c'est un
océan rouge. L'approche Océan Bleu est parfois utilisée par les managers
pour pousser à la recherche de segment non exploités (type stratégie
Wii). \newline

\begin{description}
    \item[Exclure] : on enlève tout ce qui n'apporte rien au client
    \item[Renforcer] : on développe tout ce qui a de la valeur pour le
        client
    \item[Atténuer] : on simplifie tous les élements qui n'augmentent
        pas la valeur perçue
    \item[Créer] : on ajoute des services ou des fonctions qui
        n'existaient pas et qui ajoutent de la valeur
\end{description}

\subsection{Analyse SWOT}

L'analyse SWOT est une analyse de forces (S), faiblesses (W),
opportunités (O) et menaces (T). On fait une analyse par DAS.

Les \textbf{forces} et \textbf{faiblesses} font partie de l'intérieur de
l'organisation (p.e: un scandale touche mon entreprise).

Les \textbf{opportunités} et \textbf{menaces} sont à l'extérieur de
l'organisation, elle touche l'industrie (secteur d'activité) de manière générale voire même
le macroenvironnement (p.e : une crise économique, une reprise
économique).


\section{La capacité stratégique}

La \color{red!70!black}capacité stratégique \color{black} d'une
organisation est l'ensemble des \textbf{ressources} et
\textbf{compétences} dont elle a besoin pour survivre et prospérer.
\newline

L'idée est de marier ce qu'une entreprise possède et ce qu'elle sait
bien faire. \newline

Les \color{red!70!black}capacités dynamiques \color{black} correspondent
aux capacités de renouvellement des compétences par rapport aux
changements d'environnement de l'entreprise. \newline

Les \color{red!70!black}capacités seuil \color{black} correspondent aux
capacités minimales nécessaires pour prendre part à un marché.

Aux capacités seuil, on ajoute les \textbf{capacités distinctives} qui
sont des éléments uniques (idées, brevets, talents) et ce sont ces
spécificités qui permettent de dégager un avantage concurrentiel qu'on
ne pourra pas imiter de manière durable. \newline

\subsection{Modèle VRIN}

Ce modèle permet d'évaluer les capacités stratégiques, plus elles sont
solides et conduisent à un avantage concurrentiel, plus on pourra
répondre vrai à toutes les questions. \newline

\begin{description}
    \item[V]aleur : les capacités génèrent-elles une valeur pour le
        client (supérieure aux coûts) ?
    \item[R]areté : est-ce que tout le monde les possède ?
    \item[I]nimitabilité : est-ce qu'ils pourraient nous imiter ? 
    \item[N]on-substituabilité : est-ce qu'ils pourraient faire un autre
        produit qui remplace le nôtre ? 
\end{description}

Typiquement, on rend un produit inimitable s'il provient de la culture
de l'entreprise, s'il provient de liens complexes (voire inexplicables)
entre plusieurs éléments qui ne sont pas reproductibles. \newline

Parfois, on appelle ce modèle VRIO, le O est pour le fait que les
capacités sont supportées par \textbf{l'organisation} mais c'est élément
évident et sous-jacent du modèle de VRIN qui fait oubier les substituts.
\newline

\end{document}
