\documentclass[fr,skiptoc]{../../../eplnotes}

\hypertitle{autolin}{6}{INMA}{1510}
{Nathan Jacques}
{Denis Dochain}

\section{Partie commandable/ observable}
Soit une système linéaire $\mathcal{S}$ donné par la représentation d'état
\begin{equation}
	\mathcal{S} :
    \left\{
    \begin{array}{r c l}
        \dot{x}(t) &=& A x(t) + B u(t) \\
        y(t) &=& C x(t)
    \end{array}
    \right.
    \label{eqn:ss}
\end{equation}
avec $x(t)\in \mathbb{R}^n$, $u(t)\in \mathbb{R}^m$, $A\in \mathbb{R}^{n\times n}$ et $B\in \mathbb{R}^{n\times m}$.  La matrice de commandabilité de $\mathcal{S}$ est

\[
    \mathcal{C} = \left[ B, AB, \dots, A^{n-1}B \right]
\]
Soit $\text{rang}(\mathcal{C}) = n_c < n$. Posons

\[
    T = \left[ \vec{t}_1 \dots \vec{t}_{n_c} | \vec{t}_{n_c+1} \dots \vec{t}_n \right]
\]
où $\vec{t}_1 \dots \vec{t}_{n_c}$ dont les $n_c$ colonnes libres de $\mathcal{C}$ et les $n-n_c$ colonnes restantes sont choisient arbitrairement linéairement indépendantes de $\vec{t}_1 \dots \vec{t}_{n_c}$. Il est clair que $T$ est donc de plein rang.
Soit une transformation d'état $x = Tz$. On a alors

\begin{align*}
    \mathcal{S}: \left\{
    \begin{array}{r c l}
        \dot{x} &=& A x + B u \\
        y &=& C x
    \end{array}
    \right.
    &\iff 
    \left\{
    \begin{array}{r c l}
        \dot{z} &=& T^{-1} \dot{x} = T^{-1}Ax +T^{-1} B u \\
        y &=& C Tz
    \end{array}
    \right. \\
    &\iff
    \left\{
    \begin{array}{r c l}
        \dot{z} &=& T^{-1}A T z +T^{-1} B u \\
        y &=& C Tz
    \end{array}
    \right.
\end{align*}
Notons 

\[
    U = T^{-1} = 
    \begin{bmatrix}
    U_a \\ U_b
    \end{bmatrix}
\]
avec $U_a$ de dimensions $n_c\times n$ et $U_b$ de dimensions $(n-n_c)\times n$.
La dynamique de $z$ peut donc se réécrire

\begin{equation}
    \dot{z} = UAT z + UB u
    \label{dynz}
\end{equation}
On a

\[
    UT =
    \begin{bmatrix}
        U_a \\ U_b
    \end{bmatrix}
    \cdot
    \begin{bmatrix}
        T_a & T_b
    \end{bmatrix}
    = 
    \begin{bmatrix}
        U_aT_a & U_aT_b \\
        U_bT_a & U_bT_b
    \end{bmatrix}
    =
    \begin{bmatrix}
        I_{n_c}& \\
        & I_{n-n_c}
    \end{bmatrix}
    = I_n
\]
De là, on tire que $U_bT_a = 0$, ce qui signifie que les colonnes de $T_a$ appartiennent au  noyeau de $U_b$. Comme les colonnes de $T_a$ génerent tout le columnspace de $\mathcal{C}$, on peut dire que les colonnes de $\mathcal{C}$ appartiennent au noyeau de $U_b$ \footnote{Pour ceux qui ont un mauvais souvenir du cours d'algèbre, cela revient à dire que multiplier $U_b$ à droite par une colonne de $\mathcal{C}$ (ou une combili des colonnes de $\mathcal{C}$) donnera toujours le vecteur nul }.
Comme $B$ est une colonne de $\mathcal{C}$ et $AT_a$ est composé de combili de colonnes de $\mathcal{C}$, il vient
\[U_bB = 0 \quad \text{ et } \quad U_bAT_a = 0\]
Les matrices de la dynamique de $z$ (\ref{dynz}) deviennent donc

\begin{equation}
    UAT = 
    \begin{bmatrix}
        U_aAT_a & U_aAT_b \\
        U_bAT_a & U_bAT_b
    \end{bmatrix}
    =
    \begin{bmatrix}
        A_c & A_{c\bar{c}}\\
        0 & A_{\bar{c}}
    \end{bmatrix}
    \quad 
    \text{ et }
    \quad
    UB =
    \begin{bmatrix}
        U_aB \\ U_bB
    \end{bmatrix}
    =
    \begin{bmatrix}
        B_c \\ 0
    \end{bmatrix}
    \label{dynmat}
\end{equation}
On a alors

\begin{equation}
    \dot{z} \overset{(\ref{dynz})}{=} UAT z + UB u 
    \overset{(\ref{dynmat})}{=}
    \begin{bmatrix}
        A_c & A_{c\bar{c}}\\
        0 & A_{\bar{c}}
    \end{bmatrix}
    z + 
    \begin{bmatrix}
        B_c \\ 0
    \end{bmatrix}
    u
    \label{decomp}
\end{equation}
Le vecteur d'état $z$ peut être décomposé en deux parties $z = [z_a|z_b]^T$ où $z_a$ sont les états commandables du système et $z_b$ les états non-commandables. Si on réécrit (\ref{decomp}) sous forme décomposée,

\begin{equation*}
\dot{z} = 
\left\{
\begin{array}{r c ll}
\dot{z}_a &=& A_c z_a + A_{c\bar{c}} z_b + B_c u & \text{(partie commandable)}\\
\dot{z}_b &=& A_{\bar{c}} z_b & \text{(partie non-commandable)}
\end{array}
\right.
\end{equation*}
On remarque qu'en effet, la dynamique des états $z_b$ ne dépend plus de l'entrée $u$. On ne peut donc pas les commander.
Analysons le rang de la matrice de commandabilité pour cette transforamtion d'état:

\begin{align*}
    \text{rang}(\mathcal{C}_z) &= \text{rang}
    \left(
    \begin{bmatrix}
        UB & (UAT)UB & \dots & (UAT)^{n-1}UB
    \end{bmatrix}
    \right)\\
    \text{comme $U = T^{-1}$ , }&= \text{rang}
    \left(
    \begin{bmatrix}
        UB & UAB & \dots & UA^{n-1}B
    \end{bmatrix}
    \right)\\
    &= \text{rang}
    \left(
    U
    \begin{bmatrix}
        B & AB & \dots & A^{n-1}B
    \end{bmatrix}
    \right)\\
    &= \text{rang}(U\mathcal{C})\\
    \text{comme $U$ est de plein rang, }&= \text{rang}(\mathcal{C}) = n_c
\end{align*}

Il y a toujours $n_c$ états commandables sur $n$ états au total\footnote{Comme on pouvait le prévoir, le changement de base ne permet pas de changer la propriété de commandabilité ou d'observabilité, ces propriétés étant intrinsèques au système lui-même}.
Cependant, la transformation d'état a permis de mettre en évidence la partie commandable et la partie non-commandable. La dynamique de la partie commandable est donnée par $A_c$ tandis que la partie non-commandable est gérée par la sous-matrice $A_{\bar{c}}$. En effet, il est facile (en utilisant le même raisonnement que ci-dessus, i.e. calculer le rang de la matrice de commandabilité) de montrer que $(A_c,B_c)$ est commandable.
Par ailleurs, cette décomposition en partie commandable/non-commandable est utile pour vérifier la stabilisabilité du système. En effet, la partie non-commandable étant maintenant bien identifiée, on peut vérifier la stabilisabilité en s'intéressant à la matrice $A_{\bar{c}}$.
Si $A_{\bar{c}}$ est stable\footnote{Analyse de stabilité habituelle: signe des valeurs propres, etc}, alors le système est stabilisable.

Tout ce développement se fait de manière parfaitement similaire pour l'analyse d'observabilité / détectabilité.

\end{document}
