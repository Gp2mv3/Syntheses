\documentclass[fr]{../../../../../../eplexam}
\usepackage{../../../../../../eplunits}
\usepackage{../../../../../../eplcode}

\usepackage{tikz,pgfplots,bodegraph}
\usetikzlibrary{arrows,calc}

\tikzstyle{block} = [draw, rectangle, minimum height=2em, minimum width=4em]
\tikzstyle{sum} = [draw, fill=blue!10, circle, node distance=1cm]
\tikzstyle{input} = [coordinate] \tikzstyle{output} = [coordinate]
\tikzstyle{pinstyle} = [pin edge={to-,thin,black}]

\hypertitle[']{Automatique Linéaire}{6}{INMA}{1510}{2018}{Juin}{Majeure}
{Martin Braquet \and Margo Hauwaert}
{Denis Dochain}

\section{}

Soit le système ouvert en boucle ouverte suivant: 
\[
    \begin{cases} \dot{x}=Ax+Bu \\ y = Cx  \end{cases}
\]

\[
    A = \begin{bmatrix}
        -1&2&1\\
        0&1&-1\\
        -2&4&2\\
        \end{bmatrix}
    \qquad
    B = \begin{bmatrix}
        0\\
        1\\
        0\\
        \end{bmatrix}
    \qquad
    C = \begin{bmatrix}
        0 & 0 & 1\\
        \end{bmatrix}
\]

Réécrivez la dynamique en séparant parties commandables et non commandables. 
Expliquez chaque étape de votre développement. 
Quelles conclusions tirez-vous? 
Justifiez votre réponse. 

\begin{solution}
 
Calculons la matrice de commandabilité : 
\[ \mathcal{C} = \begin{bmatrix}B & AB & A^2B \end{bmatrix} = \begin{bmatrix} 0 & 2 & 4 \\ 1 & 1 & -3 \\ 0 & 4 & 8 \end{bmatrix}, \]
le rang de cette matrice est de 2.

Ensuite, il faut choisir une matrice \( T \) inversible et composée de \( T_1 \) et \( T_2 \) où \( T_1 \) est composée de 2 colonnes indépendantes de la matrice de commandabilité et où \( T_2 \) est un vecteur assurant l'inversibilité de la matrice \( T \) (c'est-à-dire que son déterminant sera non nul, ou encore que la matrice est de rang plein).

Prenons, par exemple, la première et la troisième colonne comme vecteurs pour la matrice \( T_1 \) et 
\[ T_2 = \begin{bmatrix} 0 \\ 0 \\ 1 \end{bmatrix} .\] 
On obtient alors une matrice 
\[ T = \begin{bmatrix} 0 & 4 & 0 \\ 1 & -3 & 0 \\ 0 & 8 & 1 \end{bmatrix} .\]
Ensuite, calculons la matrice inverse de \( T \): 
\[ T^{-1} = \begin{bmatrix} \frac{3}{4} & 1 & 0 \\ \frac{1}{4} & 0 & 0 \\ -2 & 0 & 1 \end{bmatrix} .\]
Pour avoir les parties commandable et non commandable, il nous reste à calculer $ T^{-1} A T $. Cette matrice sera composée de plusieurs sous-matrices/sous-vecteurs :
\[ T^{-1} A T = \begin{bmatrix} A_{11} & A_{12} \\ A_{21} & A_{22} \end{bmatrix} ,\] 
où $ A_{11} $ sera de taille $ q \times q $, $ A_{12} $ de taille $ q \times (n-q) $, $ A_{21} $ de taille $ (n-q) \times q $ et $ A_{22} $ de taille $ (n-q) \times (n-q) $ où $ n = 3 $ est le nombre de lignes de $ T $ et $ q = 2 $ est le rang de la matrice de commandabilité.

De plus, $ A_{11} $ sera la matrice commandable et $ A_{22} $ sera la matrice non commandable.

Si on calcule cette matrice, on va obtenir : 
\[ T^{-1} A T = \begin{bmatrix} \frac{5}{2} & -\frac{25}{2} & -\frac{1}{4} \\ \frac{1}{2} & -\frac{1}{2} & \frac{1}{4} \\ 0 & 0 & 0 \end{bmatrix} .\]
On a donc une partie commandable qui vaut 
\[ A_{11} = \begin{bmatrix} \frac{5}{2} & -\frac{25}{2} \\ \frac{1}{2} & -\frac{1}{2} \end{bmatrix} \]
et une partie non commandable $ A_{22} = 0 $. 

\medskip
Si l'on s'intéresse à la stabilisabilité de cette dernière partie, nous pouvons voir que selon les critères de Hautus-Popov-Belevitch, elle n'est pas stabilisable. Cela signifie que le système ne l'est pas non plus. 
 
\end{solution}


\section{}

Soit un système dont la dynamique est décrite par le système suivant: 

\begin{equation*}
    G(s)=\frac{0.0054-0.01s}{(s+0.325)(s^2-0.015s+0.0065)}
\end{equation*}

Soient les trois compensateurs suivants ($K_{p}, \tau_{i}, \tau_{d}$ sont positifs)
\begin{gather*}
\begin{aligned}
P &: C_{1}(s)=K_{p}, \\
PI &: C_{2}(s)=K_{p}\frac{1+s\tau_{i}}{s\tau_{i}}, \\
PD &: C_{3}(s)=K_{p}(1+s\tau_{d}).
\end{aligned}
\end{gather*}

Que pouvez-vous dire sur le système en boucle ouverte, en particulier en termes de pôles et zéros? Déterminez pour chaque compensateur si le système peut être stabilisé en boucle fermée par un retour unitaire de sortie. 

\begin{solution}

Relations générales:
\[
 C(s)=\frac{N_C(s)}{D_C(s)} \qquad G(s)=\frac{N_G(s)}{D_G(s)}
\]

\[
 Tr(s)=\frac{C(s)G(s)}{1+C(s)G(s)}=\frac{N_C(s)N_D(s)}{N_C(s)N_D(s)+D_C(s)D_D(s)}=\frac{num(s)}{den(s)}
\]

En boucle ouverte, les pôles sont les racines du dénominateur de $G(s)$ et les zéros sont les racines du numérateur de $G(s)$: $p=\{-0.325,0.0075 \pm 0.080273\, \imag \}$ et $z=0.54$. Il s'agit donc d'un système instable à nom minimum de phase.

En boucle fermée par retour unitaire de sortie:
\begin{enumerate}
    \item P: $C(s) = k_p $
    
    \[ den(s)= s^3 + 0.31 s^2 + (0.001625-0.01k_p) s + (0.0021125+0.0054k_p)\]
    Voici le critère d'Hurwitz pour une équation du $3^e$ degré : $a_0>0,a_2>0$ et $a_1a_2>a_0$.
    
    \[
     \left\{
     \begin{array}{rcl}
        0.0021125+0.0054k_p & > & 0 \\
        0.31(0.001625-0.01k_p)& > & 0.0021125+0.0054k_p
     \end{array}
     \right.
    \]
    Le système est donc stable si $-0.3912<k_p<-0.1893$.
    
    \item PI: $C(s) = k_p\left(1+\dfrac{1}{\tau_is}\right)$
    \[
     den(s)=\tau s^4 +0.31\tau s^3 +(0.001625-0.01 k_p) \tau s^2+ (0.0054 k_p  \tau - 0.01 k_p+0.0021125 \tau) s + 0.0054 k_p
    \]
    Voici le critère d'Hurwitz pour une équation du $4^e$ degré : 
    \[
     \left\{
     \begin{array}{rcl}
        a_0 & > & 0 \\
        a_3 & > & 0 \\
        a_2a_3-a_1a_4 & > & 0 \\
        a_1a_2a_3-a_1^2a_4-a_0a_3^2 & > & 0 
     \end{array}
     \right.
     \qquad\Longrightarrow\qquad
     \left\{
     \begin{array}{rcl}
        k_p & > & 0 \\
        \tau & > & 0 \\
        a_2a_3-a_1a_4 & > & 0 \\
        a_1a_2a_3-a_1^2a_4-a_0a_3^2 & > & 0 
     \end{array}
     \right.
    \]

    Ces équations sont bien sûr impossibles à résoudre analytiquement, il est donc opportun de s'arrêter là en cherchant un couple de valeurs pour $k_p$ et $\tau$ qui soit solution de ce système. Il semblerait qu'il n'y en ait pas.
    
    \item PD: $C(s) = k_p(1+\tau_ds)$
    \[
     den(s)=s^3 + (0.31-0.01k_p\tau_d) s^2 + (0.001625+0.0054k_p\tau_d-0.01k_p) s + (0.0021125+0.0054k_p)
    \]
    C'est aussi une équation du $3^e$ degré:
    \[
     \left\{
     \begin{array}{rcl}
        0.0021125+0.0054k_p & > & 0 \\
        0.31-0.01k_p\tau_d & > & 0 \\
        (0.31-0.01k_p\tau_d)(0.001625+0.0054k_p\tau_d-0.01k_p)& > & 0.0021125+0.0054k_p
     \end{array}
     \right.
    \]
    On trouve le couple $k_p=1$ et $\tau_d=20$ qui vérifie le système, donc le système est stabilisable en boucle fermée.
    
\end{enumerate}

\end{solution}

\section{}

Soit le système dynamique linéaire $\dot{x}=Ax+Bu$.

Sous conditions de stabilité de la paire $(A,B)$ et détectabilité de la partie $(Q_x,A)$ et avec connaissance de l'équation de Riccati algébrique:

\begin{equation*}
    A^TS+SA-SBQ_u^{-1}B^TS+Q_x=0,
\end{equation*}
montrez que la commande 
\begin{equation*}
    u(t)=-Q_u^{-1}B^TSx(t)
\end{equation*}
minimise la fonction quadratique 
\begin{equation*}
    \int_{0}^{\infty} (x^TQ_xx+u^TQ_uu)\:\mathrm{d} t.
\end{equation*}

\begin{solution}
 
On cherche la commande optimale $u(t)$ qui minimise un critère d'énergie (pondérée par $Q_x$ et $Q_u$, des matrices diagonales choisies par le constructeur) de $u(t)$ et $x(t)$. Ce critère est appelé $J(u(t))$ et s'exprime comme
\[
 J(u)= \int_{0}^{\infty} (x^TQ_xx+u^TQ_uu)\:\mathrm{d} t.
\]
On cherche donc 
\[
 u_{opt}(t) = \min_{u(t)}J(u).
\]
Or, on sait que par l'équation de Riccati,
\[
 Q_x = -A^TS-SA+SBQ_u^{-1}B^TS
\]
avec $S$ la matrice symétrique solution de l'équation de Riccati.

On utilise également le théorème de complétion des carrés\footnote{On peut tirer profit du fait que $Q_u^{-1}$ est diagonal puisque $Q_u$ est diagonal, donc $(Q_u^{-1})^T=Q_u^{-1}$.},
\[
 \left(u+Q_u^{-1}\alpha\right)^TQ_u\left(u+Q_u^{-1}\alpha\right) = u^TQ_uu+u^T\alpha+\alpha^Tu+\alpha^TQ_u^{-1}\alpha
\]
pour lequel on remplace $\alpha$ par $B^TSx$ dans le développement ci-dessous.

Ainsi,
\begin{align*}
 J(u) & = \int_{0}^{\infty} (x^TQ_xx+u^TQ_uu)\:\mathrm{d} t \\
 & = \int_{0}^{\infty} \left(x^T\left(-A^TS-SA+SBQ_u^{-1}B^TS\right)x+u^TQ_uu\right)\:\mathrm{d} t \\
 & = \int_{0}^{\infty}\left(-x^TA^TSx-x^TSAx+x^TS^TBQ_u^{-1}B^TSx+u^TQ_uu\right)\:\mathrm{d} t \\
 & = \int_{0}^{\infty}\left(\left(u+Q_u^{-1}B^TSx\right)^TQ_u\left(u+Q_u^{-1}B^TSx\right)
 -\left(x^TA^TSx+x^TSAx+u^TB^TSx+x^TS^TBu\right) \right)\:\mathrm{d} t \\
 & = \int_{0}^{\infty}\left(\left(u+Q_u^{-1}B^TSx\right)^TQ_u\left(u+Q_u^{-1}B^TSx\right)
 -\left(((Ax)^T+u^TB^T)Sx+x^TS(Ax+Bu)\right) \right)\:\mathrm{d} t \\
 & = \int_{0}^{\infty}\left(\left(u+Q_u^{-1}B^TSx\right)^TQ_u\left(u+Q_u^{-1}B^TSx\right)
 -\left(\dot{x}^TSx+x^TS\dot{x}\right) \right)\:\mathrm{d} t \\
 & = \int_{0}^{\infty}\left(u+Q_u^{-1}B^TSx\right)^TQ_u\left(u+Q_u^{-1}B^TSx\right)\:\mathrm{d} t
 -\int_{0}^{\infty}\fdif{}{t}(x^TSx)\:\mathrm{d} t \\
 & = \int_{0}^{\infty}\left(u+Q_u^{-1}B^TSx\right)^TQ_u\left(u+Q_u^{-1}B^TSx\right)\:\mathrm{d} t+x_0^TSx_0
\end{align*}

Puisque le second terme est indépendant de $u(t)$, il faut minimiser le premier terme. Celui-ci étant positif ou nul, la fonctionnelle $J(u)$ est minimale pour 
\[
 u+Q_u^{-1}B^TSx = 0 \qquad\Longrightarrow\qquad u = -Q_u^{-1}B^TSx
\]
Lors d'un retour d'état, $u=-Kx$, on choisit ainsi 
\[
 K = Q_u^{-1}B^TS
\]
en se rappelant que $S$ est déterminé par l'équation de Riccati pour laquelle $Q_x$ et $Q_u$ sont choisis et connus.
 
\end{solution}


\section{}

Soit un système en boucle fermée avec \[G(s)=\frac{1}{3}\frac{3-s}{s^3+2.5s^2+s}.\] 

\begin{figure}[!ht]
	\centering
	\begin{tikzpicture}
	\node [input, name=input] at (0,0) {};
	\node [sum] (sum) at (1,0) {};
	\node [block] (C) at (3,0) {$k_0C(s)$};
	\node [block] (G) at (6,0) {$G(s)$};
	\node [output, name=output] at (8,0) {};
	
	\draw [->] (input) -- node[above left] {$r(t)$} node[above left,pos=1] {$+$} (sum);
	\draw [->] (sum) -- (C);
	\draw [->] (C) -- node[above, midway] {$u(t)$} (G);
	\draw [->] (G) -- node[above, pos=0.8] {$y(t)$} (output);
	
	\draw [->] (7,0) -- (7,-1) -| node[below right, pos=1] {$-$} (sum);
	\end{tikzpicture}
\end{figure}

Le diagramme de Bode a les caractéristiques suivantes:
\[
G_m=\SI{2.69}{dB} \mbox{ à } \SI{0.739}{rad/sec}
\]
\[
P_m=\ang{10.4} \mbox{ à } \SI{0.615}{ rad/sec}
\]

% Be sure to have bodegraph package and gnuplot installed on your OS !
\begin{figure}[h!]
    \centering
    \begin{tikzpicture}[xscale=2.5,gnuplot def/.append style={prefix={}}]
    \tikzset{semilog lines/.style={thin,blue},
             semilog lines 2/.style={semilog lines,red!50},
             semilog half lines/.style={semilog lines 2,dotted},
             semilog label x/.style={semilog lines,below,font=\tiny},
             semilog label y/.style={semilog lines,right,font=\tiny}
            }
    \begin{scope}[yscale=2/80]
    \semilog{-2}{2}{-100}{50}
    \BodeGraph[smooth,black]{-2:2}{\SOAmp{1}{1.25}{1}+\IntAmp{1}-\POAmp{1}{-0.33333}}
    \end{scope}
    \begin{scope}[yshift=-8cm,yscale=2/160]
    \OrdBode{30}
    \semilog{-2}{2}{0}{360}
    \BodeGraph[smooth,black]{-2:2}{\SOArg{1}{1.25}{1}-\POArg{1}{-0.33333}-\IntArg{1}-\IntArg{1}-\IntArg{1}}
    \end{scope}
    \end{tikzpicture}
    \caption{Diagramme de Bode de l'amplitude (dessus) et de la phase (dessous)}
\end{figure}

Déterminez un filtre à avance de phase $C(s)$ et la valeur du paramètre $k_0$ pour satisfaire aux spécifications suivantes: 
\[
    \phi_M = \ang{45} \qquad\mathrm{pour}\qquad  \omega=\SI{0.5}{ rad/sec}
\]

\begin{solution}

Le filtre à avance de phase $C(s)$ s'écrit : 
\[ C(s) = \frac{1+T_\alpha s}{1+\alpha T_\alpha s} .\]
Les paramètres $\alpha$ et $T_\alpha$ servent à déterminer la phase que l'on veut ajouter à une fréquence choisie.

On utilisera les deux relations suivantes : 
\[ \phi = \mathrm{atan}\left(\frac{1-\alpha}{2\sqrt{\alpha}}\right), \qquad \omega = \frac{1}{T_\alpha \sqrt{\alpha}}.\]
On remarque sur le graphe donné que la marge de phase en $\omega=0.5$ est de $\ang{21.5}$. On veut la faire passer à $\ang{45}$, donc $\phi = \ang{45}-\ang{21.5}=\ang{23.5}$. Cela permet de trouver $\alpha = 0.429725$.

Pour que cette augmentation de phase de $\ang{23.5}$ ait lieu en $\omega = 0.5$, il faut que $T_\alpha = 3.0509$. Le filtre pour l'augmentation de phase est bon, mais on nous demande aussi que 0.5 soit la fréquence de coupure.

On utilise pour ça une autre relation: $\left|k_0C(j\omega_c)G(j\omega_c)\right| = 1$. En écrivant les expressions avec $s = j\omega$:
\[ \left|\frac{k_0(3-j\omega)(1+T_\alpha j\omega)}{3j\omega(j\omega-0.5)(j\omega-2)(1+\alpha T_\alpha j\omega)}\right| = 1 \]
On peut réécrire que la norme d'un produit est égale au produit de toutes les normes :
\[ \frac{k_0|3-j\omega||1+T_\alpha j\omega|}{3|j\omega||j\omega-0.5||j\omega-2||1+\alpha T_\alpha j\omega|} = 1 \]
La norme d'un complexe $a+bj$ c'est $\sqrt{a^2+b^2}$ :
\[ \frac{k_0\sqrt{9+\omega^2}\sqrt{1+T_\alpha^2\omega^2}}{3\omega\sqrt{\omega^2+0.25}\sqrt{\omega^2+4}\sqrt{1+\alpha^2T_\alpha^2\omega^2}} = 1 \]
Donc on trouve $k_0 = 0.4713$.

\end{solution}


\section{}

Soit un réacteur chimique continu avec 2 réactions isothermes en cascade 

\centerline{$A\rightarrow B$ et $B\rightarrow C$.} 
La dynamique du procédé peut être caractérisée par les équations de bilan de masse suivantes: 

\[
\fdif{C_A}{t} = \frac{q}{V}C_{A,in}-\frac{q}{V}C_{A}-k_{01}C_A
\]
\[
\fdif{C_B}{t} = \frac{-q}{V}C_{B}+b\,k_{01}C_A-k_{02}C_B
\]
où $q, C_A, C_{A,in}, C_B, V, k_{01}, k_{02}$ et $b$ sont respectivement 
\begin{itemize}
  \item le débit volumique d'entrée $(m^3/h)$,
  \item la concentration du réactif A dans le réacteur et sans l'alimentation $(g/L)$,
  \item la concentration de B dans le réacteur $(g/L)$,
  \item le volume du réacteur $(m^3)$,
  \item les constantes cinétiques de la première et de la deuxième réaction $(h^{-1})$,
  \item le coefficient stoechiométrique associé à la première réaction
\end{itemize}  
L'entrée de commende est $C_{A,in}$.

\begin{enumerate}
  \item Vous avez accès à la mesure $C_B$ mais pas à celle de $C_A$, que vous souhaitez déterminer en temps réel en utilisant un observateur de Luenberger, dont vous souhaitez fixer les pôles. Donnez les équations de l'observateur dans ce cas. 
  \item Vous êtes dans la situation inverse où vous mesurez $C_A$ mais pas $C_B$ et vous utilisez la même approche qu'au point $1$ pour synthétiser un observateur de Luenberger. Donnez les équations de l'observateur dans ce cas. 
  \item Comparez les $2$ observateurs en termes de performances dynamiques. Justifiez votre réponse. 
  \item Donnez les valeurs des gains de l'observateur $K$ de Kalman dans le premier cas, sachant que l'équation de Riccati et le gain $K$ s'écrivent ainsi:
    \[
    0=-RC^TCR+RA^T+AR,
    \]
    \[
    K=RC^T,
    \]
    où $R=R^T$ et $A$ et $C$ sont des matrices d'état et de sortie du système dynamique linéaire. 
\end{enumerate}

\begin{solution}
 
On peut réécrire le système de manière compacte:
\[ x = \begin{pmatrix}C_A\\C_B\end{pmatrix} \hspace{2mm} u = C_{A,in}.\]
Donc,
\[ \begin{cases} \dot{x} = Ax + Bu \\ y = Cx \end{cases} ,\]
avec
\[ A = \begin{pmatrix} \frac{-q}{V}-k_1&0 \\ bk_1 & \frac{-q}{V}-k_2\end{pmatrix}, \qquad B = \begin{pmatrix} \frac{q}{V} \\0\end{pmatrix}.\]
$C$ va varier selon la question.
Pour commencer, on connait $C_B$ donc $C = \begin{pmatrix}0&1\end{pmatrix}$.

L'observateur de Luenberger s'écrit :
\[ \dot{\hat{x}} = (A-LC)\hat{x} + Bu + Ly, \qquad L = \begin{pmatrix}l_1 \\ l_2 \end{pmatrix}.\]
Les équations complètes sont :
\[ \begin{cases} \frac{d\hat{C}_A}{dt} = (\frac{-q}{V}-k_1)\hat{C}_A - l_1\hat{C}_B + \frac{q}{v}C_{A,in} + l_1C_B\\ \frac{d\hat{C}_B}{dt} = bk_1\hat{C}_A + (\frac{-q}{V}-k_2-l_2)\hat{C}_B + l_2C_B \end{cases}.\]
Si on connait $C_A$, alors $C = \begin{pmatrix}1&0\end{pmatrix}$, et le résultat est presque le même :
\[ \begin{cases} \frac{d\hat{C}_A}{dt} = (\frac{-q}{V}-k_1-l_1)\hat{C}_A + \frac{q}{v}C_{A,in} + l_1C_A\\ \frac{d\hat{C}_B}{dt} = (bk_1-l_2)\hat{C}_A + (\frac{-q}{V}-k_2)\hat{C}_B + l_2C_A \end{cases}. \]
 
\end{solution}


\section{}

Soit le système en boucle fermée suivant: 

\begin{figure}[!ht]
	\centering
	\begin{tikzpicture}[auto,>=latex']
		\node [input, name=input] at (0,0) {};
		\node [sum] (sum1) at (1,0) {};
		\node [block] (C1) at (3,0) {$C_1(s)$};
		\node [sum] (sum2) at (5,0) {};
		\node [block] (C2) at (7,0) {$C_2(s)$};
		\node [block] (G2) at (9,0) {$G_2(s)$};
		\node [block] (G1) at (13,0) {$G_1(s)$};
		\node [output, name=output] at (14.5,0) {};
		
		\draw [->] (input) -- node[pos=0.2] {$y^*(t)$} node[pos=0.9] {$+$} (sum1);
		\draw [->] (sum1) -- (C1);
		\draw [->] (C1) -- node[pos=0.4] {$y_1^*(t)$} node[pos=0.9] {$+$} (sum2);
		\draw [->] (sum2) -- (C2);
		\draw [->] (C2) -- (G2);
		\draw [->] (G2) -- node {$y_1(t)$} (G1);
		\draw [->] (G1) -- node[pos=0.8] {$y(t)$} (output);
		
		\draw [->] (11,0) -- (11,-1) -| node[below right, pos=1] {$-$} (sum2);
		\draw [->] (14,0) -- (14,-1.5) -| node[below right, pos=1] {$-$} (sum1);
	\end{tikzpicture}
\end{figure}

avec 
\begin{equation*}
     C_1(s)=K_{p1}\frac{1+s\tau_{i1}}{s\tau_{i1}} \qquad C_2(s)=K_{p2} \qquad G_1(s)=\frac{4}{4S+1} \qquad G_2(s)=\frac{5}{s+1}
\end{equation*}

\begin{enumerate}
    \item Comment s'appelle cette structure? Quel est son intérêt? Quels sont ses avantages et inconvénients? 
    \item Déterminez les valeurs de $K_{p1}, K_{p2}, \tau_{i1}$ de manière à avoir une dynamique en boucle fermé avec seulement deux pôles en $-4$ et $-5$.
    \item Cette structure de commande permet-elle d'annuler l'erreur statique? Justifiez vos réponses. 
\end{enumerate}

\nosolution

\end{document}
