\section{Induction}

\subsection{Notions}
Un \textbf{raisonnement inductif} est le fait d'inférer une règle générale à partir d'observations particulières par opposition à un raisonnement déductif. Dans le cadre de l'induction expérimentale, on part d'observations expérimentales et on en déduit un modèle général. C'est une méthode incertaine, ce n'est qu'une accumulation d'évidences.\\

L'\textbf{induction mathématique}, quant à elle, part de propriétés de cas particuliers pour en déduire une propriété générale. Cette méthode est certaine car prouvée.\\

Ensembles inductifs : ex. un naturel est 0 ou le successeur d'un naturel.\\

Définitions inductives : 0!=0, (n+1)!=(n+1)xn!

\subsubsection{Preuve calculatoire d'égalité}
On prouve $E_0=E_n$
\begin{align*}
	E_0 &= E_1 [w_1]\\
	&= E_2 [w_2]\\
	&= ...\\
	&= E_n [w_n]
\end{align*}
E sont des expressions et w sont des indices (explications, justifications). Chaque $w_i$ justifie pourquoi $E_i = E_{i+1}$
\subsection{Induction simple}
Soit P[n] une propriété dépendant de n.\\
\textbf{Si\\ $\bullet$ P[0] est vrai et \\ $\bullet$ si P[n] est vrai alors P[n+1] est vrai \\ Alors P[n] est vrai pour tout n}\\
(P[0] $\wedge$ ($\forall$n : P[n] $\Rightarrow$ P[n+1])) $\Rightarrow$ $\forall$n : P[n]\\

Induction simple $\equiv$ récurrence !
\subsubsection{Induction simple à partir de k}
Soit P[n] une propriété dépendant de n.\\
Si\\ $\bullet$ P[k] est vrai et \\ $\bullet$ si P[n] est vrai alors P[n+1] est vrai pour tout n $\geq$ k \\ Alors P[n] est vrai pour tout n$\geq$ k\\

(P[k] $\wedge$ ($\forall$n$\geq$ k : P[n] $\Rightarrow$ P[n+1])) $\Rightarrow$ $\forall$n $\geq$ k : P[n]\\

Induction simple $\equiv$ récurrence !
\subsubsection{Induction d'ordre 2}
Soit P[n] une propriété dépendant de n.\\
Si\\ $\bullet$ P[0] est vrai et P[1] est vrai \\ $\bullet$ si P[n] et P[n+1] sont vrais alors P[n+2] est vrai\\ Alors P[n] est vrai pour tout n\\

(P[k] $\wedge$P[1] $\wedge$ ($\forall$n: P[n] $\wedge$ P[n+1] $\Rightarrow$ P[n+2])) $\Rightarrow$ $\forall$n  : P[n]\\

\subsubsection{Induction d'odre k}
Soit P[n] une propriété dépendant de n.\\
Si\\ $\bullet$ P[0] est vrai et ... et P[k-1] est vrai \\ $\bullet$ si P[n] et ... et P[n+k-1] sont vrais alors P[n+k] est vrai\\ Alors P[n] est vrai pour tout n\\

(P[k] $\wedge$ ... $\wedge$ P[1] $\wedge$ ($\forall$n: P[n] $\wedge$ ... $\wedge$ P[n+1] $\Rightarrow$ P[n+2])) $\Rightarrow$ $\forall$n  : P[n]\\

\subsubsection{Preuve calculatoire d'implication}
On prouve $P_0\Leftarrow P_n$
\begin{align*}
P_0 &\Leftarrow P_1 [w_1]\\
&\Leftarrow P_2 [w_2]\\
&\Leftarrow ...\\
&\Leftarrow P_n [w_n]
\end{align*}
P sont des assertions (propositions, formules) et w sont des indices (explications, justifications). Chaque $w_i$ justifie pourquoi $E_i \Leftarrow E_{i+1}$\\
On peut avoir des $\Leftrightarrow$ (si P $\Leftrightarrow$ P' alors P $\Leftarrow$ P')
\subsubsection{Preuve calculatoire d'inégalité}
On prouve $E_0 \geq E_n$
\begin{align*}
E_0 &\geq E_1 [w_1]\\
&\geq E_2 [w_2]\\
&\geq ...\\
&\geq E_n [w_n]
\end{align*}
E sont des expressions et w sont des indices (explications, justifications). Chaque $w_i$ justifie pourquoi $E_i \geq E_{i+1}$\\
On peut avoir des = (si E = E' alors E $\geq$ E').\\
On peut avoir des $>$ (si E $>$ E' alors E $\geq$ E').
\vfill

\subsection{Induction complète}
Soit P[n] une propriété dépendant de n.\\
\textbf{Si:\\ si P[k] est vrai pour k $<$n alors P[n] est vrai \\  Alors P[n] est vrai pour tout n}\\

\textbf {($\forall$n :($\forall$ k $<$ n: P[k]) $\Rightarrow$ P[n])) $\Rightarrow$ $\forall$n : P[n]}\\

Il n'y a \textbf{pas de cas de base}, P[0] est couvert par le cas inductif. Mais la définition de P[n] peut avoir un cas particulier pour n=0 ou 1 ou autre...\\

$\bullet$\textbf{Induction complète et variants}:
La règle de l'itération de correction totale $\equiv$ une induction complète sur le variant.
\subsection{Induction bien-fondée}
Une \textbf{relation} $\prec$ est \textbf{bien-fondée} (sur un ensemble W) ssi il n'existe pas de chaîne décroissante infinie (dans l'ensemble W). Ssi tout sous-ensemble de W a au moins un élément minimal.\\
$ \nexists \; x_1,x_2,x_3,... \in W : x_1 \succ x_2 \succ x_3 \succ ... $ et
$\forall X \subseteq W : \exists x \; \in X : \forall y \in X : y \nprec x $\\

Une relation bien-fondée est:\\
$\bullet$ \textbf{Irréflexible :} $x \nprec x$\\
$\bullet$ \textbf{Asymétrique :} $x \prec y \Rightarrow y \nprec x$\\
$\bullet$ \textbf{Pas nécessairement transitive :} on peut avoir $x \prec y, y \prec z, x \nprec z$, il n'y a donc pas nécessairement d'ordre.\\
$\bullet$ \textbf{Pas nécessairement totale :} on peut avoir $x \neq y, x \nprec y, y \nprec x$\\

\subsubsection{Principe d'induction bien-fondée}
Pour une relation ($\prec$) bien-fondée dans W\\
Si\\
si P[y] est vrai pour tout y $\prec$ x dans W \\ alors P[x] est vrai\\
Alors P[x] est vrai pour tout x dans W.\\

\textbf{Induction complète} $\equiv$ \textbf{induction bien-fondée} sur l'ordre ($<$) des entiers naturels.

\subsubsection{Preuve calculatoire d'inégalité stricte}
On prouve $E_0 > E_n$
\begin{align*}
E_0 &\geq E_1 [w_1]\\
&\geq E_2 [w_2]\\
&> ...\\
&\geq ...\\
&\geq E_n [w_n]
\end{align*}
E sont des expressions et w sont des indices (explications, justifications). Chaque $w_i$ justifie pourquoi $E_i \geq E_{i+1}$\\
On peut avoir des = (si E = E' alors E $\geq$ E').\\
Il faut au moins un $>$ (si E $\geq$ E'' $>$ E''' $\geq$ E' alors E $>$ E').

\subsubsection{Structures}
Une structure est composée de  structures, de formules, de phrases, de programmes, de \textbf{termes}. Toutes les valeurs sont à \textbf{construction finie}.\\

Une \textbf{relation de sous-terme} strict se définit comme:\\
t $<$ t'\\
ssi t est un \textbf{sous-terme strict} de t'\\
ssi t \textbf{apparait dans} et est \textbf{différent de} t'\\
t $<$ t' $\Leftrightarrow$ t' = u[t] $\wedge$ t' $\neq$ t\\
La relation de sous-terme est \textbf{bien-fondée} !
\subsubsection{Principe d'induction structurale}
Induction bien-fondée sur les sous-termes $\equiv$ \textbf{induction structurale}\\ \vspace{0.5mm}\\
Si\\ si P[y] est vrai pour tout sous-terme y de x\\ alors P[x] est vrai\\ Alors P[x] est vrai pour tout terme x

\subsection{Résumé}
$\bullet$ induction simple : P[0], P[n] $\Rightarrow$ P[n+1] \\

$\bullet$ induction complète: ($\forall$ k$<$n : P[k]) $\Rightarrow$ P[n]\\

$\bullet$ induction bien-fondée: ($\forall$ k$\prec$n : P[k]) $\Rightarrow$ P[n]\\

$\bullet$ induction structurale: ($\forall$ k sous-terme de n : P[k]) $\Rightarrow$ P[n]\\
