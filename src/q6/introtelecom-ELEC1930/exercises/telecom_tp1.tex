\subsection{Rappel}

\[ x = |x|e^{j\phi} \Rightarrow Y = \sqrt{x}e^{j\phi/z}=\sqrt{|x|}(\cos(\frac{\phi}{2}+j * \sin(\frac{\phi}{2}))) \]
\[ |x| = \sqrt{X^2_R + X^2_I} \]
\[ \phi = tg^{-1} (\frac{X_I}{X_R}) \]

\begin{center}
\setlength{\unitlength}{1mm}
\begin{picture}(70,40)
	\put(10,5){\vector(1,0){60}}
	\put(10,5){\vector(0,1){35}}
	\put(0,30){$X_I$}
	\multiput(10,30)(2,0){20}{\line(1,0){1}}
	\multiput(50,5)(0,1.6){16}{\line(0,1){1}}
	\put(48,0){$X_R$}
	\put(10,5){\line(40,25){40}}
	\put(18,5){\line(-25,40){2.2}}
	\put(18,7){$\phi$}
	\put(25,20){$|x|$}
\end{picture}
\end{center}

\[ X_R = |x| \cos \phi \]
\[ X_I = |x| \sin \phi \]
Soit $d$ la distance
\[ |V_{out}|^2 = e^{-2\alpha d} V_{in} \]
\[ P_{out} = e^{-2\alpha d}P_{in} \]

\[
    V(x) = V_{1}e^{\gamma x} + V_{2}e^{\gamma x}
\]
\[
    e^{-\gamma x} = e^{-\alpha x}e^{-j\beta}
\]
\[
    \mid e^{-j\beta} \mid = 1
\]

$V_{1}$ est la différence de voltage à l'envoi

$V(x)$ est la différence de voltage à la réception

$\alpha$ est l'atténuation par mètre (en dB)


\subsection{Exercice 1}
\textit{Considérons un cable coaxial RG218/U utilisé pour les
applications vidéo avec des bandes UHF\@. Les paramètres de ligne
$R'$, $C'$ et $L'$ sont donnés:$R' = 0.368$, $C'=102.4 \;pF/m$, $L'=250 \;nH/m$. La conductance est supposé négligeable.}
\begin{enumerate}
	\item \textit{Déterminer l'impédance caractéristique du cable, l'atténuation et la vitesse de phase pour un signal de $100 MHz$}

	\begin{framed}
	Pour cette partie nous allons couper une partie du cable (la
	taille doit tendre vers 0). A la ligne (5) on se débarasse de la
	partie réelle qui est beaucoup plus faible que la partie réelle
	du calcul.

	$R'$ représente la résistance en série par unité de longueur de
	ligne ($\ohm$/mètres)

	$L'$ représente les inductances en série par unité de longueur
	de ligne
	\begin{eqnarray*}
		\omega &=&100MHz * 2\pi\\
		&=& 6.28*10^8\\
		Z_0 &=& \sqrt{\frac{R'+j\omega L'}{G' + j\omega C'}}\\
		&=& \sqrt{\frac{0.368 + j * 6.28 * 25}{j*6.28 * 0.01024}} \\
		&\approx& \sqrt{\frac{j * 6.28 * 25}{j*6.28 * 0.01024}}\\
		&\approx& \sqrt{\frac{25}{0.01}} \approx 50 \Omega
	\end{eqnarray*}
	\end{framed}

	\item \textit{En considérant que la puissance du signal est de $66\mu W$ à la sortie du transmetteur, déterminez la longueur maximum du cable quand une puissance minimale de $0.42\mu W$ est requise à la sortie de celui-ci.}
	\begin{framed}
	\begin{eqnarray*}
		\gamma &=& \sqrt{(R' + j\omega L')j \omega C'} \\
		&=& \sqrt{j\omega R' C' - \omega^2 L' C'} \\
		&=& \sqrt{-10.106 + j*2.3677 * 10^{-2}} \\
		|A| &=& \sqrt{10.106+2.3677^2*10^{-4}} \\
		\psi &=& tg^{-1} * \frac{2.3677*10^{-2}}{10.106}
	\end{eqnarray*}
	\end{framed}

\end{enumerate}

\subsection{Exercice 2}
\textit{Prouvez l'équation $L_{FS}|_{dB} \simeq 32.4 + 20 \log(F_{MHz}) + 20 \log(D_{Km})$ (la vitesse de la lumière est $299.792.458 m/s$)}
\begin{framed}
	\[ X \in \mathbb{R} \]
	\[ \lambda = \frac{C}{F} \]
	\[ X|_{dB} = 20 \log_{10}X \]
	A partir de l'équation:
	\[ L_{FS}|_{dB} \simeq 20 \log(\frac{4\pi}{\lambda}) + 20 \log(D) \]
	On va démontrer l'équation
	\begin{eqnarray}
		L_{FS}|_{dB} &\simeq& 20 \log(4\pi) - 20 \log(c(f) + 20) \log(D_{Km} 10^{+3}) \\
		 &\simeq& 20 \log(4\pi) - 20 \log(c) + 20 \log(F_{MHz}10^{+6}) 20 \log(D_{Km}) + 20 \log(10^{+3})\\
		 &\simeq& 20 \log(4\pi) - 20 \log(c) -120dB + 20 \log(F_{MHz}) 20 \log(D_{Km}) -60dB\\
		 &\simeq&
	\end{eqnarray}
	La ligne (3) se justifie car, $20 \log(10^{-3}) = -60dB$ et  $20 \log(10^{-6}) = -120dB$
\end{framed}
