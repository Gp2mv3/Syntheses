\documentclass[fr]{../../../../../../eplexam}

\hypertitle{Compléments d'Analyse}{6}{INMA}{1315}{2018}{Juin}
{Nathan JACQUES}
{Michel WILLEM}

\section{Théorie}

	\subsection{Question 1}
	
		Montrer que l'enveloppe supérieure d'une famille de fonctions s.c.i. est s.c.i.
		
	\subsection{Question 2}

		Définir la notion d'intégrale élémentaire et donner deux exemples.
	
	\subsection{Question 3}
	
		Définir la notion d'espace pré-hilbertien et donner un exemple.
	
\section{Exercices}

	\subsection{Question 1}
		Soit $X$ et $S\subset X$ une partie non vide de $X$. Pour tout $x\in X$, on a
		$$\varphi (x) = d(x,S) = \inf_{y\in S} d(x,y)$$
		
		Montrer que, $\forall x,y \in X$,
		$$ \vert \varphi(x)-\varphi(y)\vert \leq d(x,y)$$
	\subsection{Question 2}
		Soit $\Omega = B(0,1) \subset \mathbb{R}^4$. On définit sur $\Omega \backslash \{0\}$,
		$$ u(x) = \frac{\ln (1+\vert x\vert)}{\vert x\vert ^3}$$
		
		\begin{enumerate}
			\item Monter que $u\in \mathcal{M} \left( \Omega \right)$. \underline{Indication:} $\min (u,n) \to u$, presque partout
			\item Monter que $u\in \mathcal{L}^1 \left( \Omega \right)$. \underline{Indication:} $\ln (1+\vert x\vert) \leq \vert x\vert$ et $\alpha > -4 \implies \vert x\vert ^{\alpha } \in \mathcal{L}^1 \left( \Omega \right)$
			\item Calculer $\int_{\Omega} {u dx}$. \underline{Indication:} Utiliser le passage en coordonnées polaires.
		\end{enumerate}
	\subsection{Question 3}
		Soit $X$ un espace pré-hilbertien, montrer que, $\forall u,v \in X$ et pour tout $\epsilon >0$,
		$$ \vert (u\vert v)\vert \leq \frac{1}{2\epsilon} \Vert u\Vert ^2 + \frac{\epsilon}{2} \Vert v\Vert ^2$$
		En déduire l'inégalité de Cauchy-Schwarz.
\end{document}
