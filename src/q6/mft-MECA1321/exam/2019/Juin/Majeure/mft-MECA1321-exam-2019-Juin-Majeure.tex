\documentclass[fr]{../../../../../../eplexam}

\hypertitle{Mécanique des Fluides et Transferts}{6}{MECA}{1321}{2019}{Juin}{Majeure}
{Cédric Legrand}
{Vincent Legat et Grégoire Winckelmans}

\section{Partie GW}

On considère des écoulements établis et incompressibles en canal. La distance mesurée à partir de la paroi inférieure est notée $y$. On définit aussi $\eta = \frac{y}{h}$. Dans le cas avec écoulement laminaire, le profil de vitesse est noté $u(y)$ ; dans le cas avec écoulement turbulent, il est noté $\Bar{u}(y)$ (où $\Bar{\cdot}$ désigne la moyenne temporelle).

Le \textit{nombre de Reynolds global}, $Re_d$, est basé sur la distance $d=2h$ entre les parois du canal, et sur la \textit{vitesse moyenne} (= \textit{vitesse de débit} : $u_m$ ou $\Bar{u}_m$, selon le cas) : la définir.

\begin{enumerate}
    \item On considère d'abord le cas avec écoulement laminaire (= écoulement de Poiseuille), et donc avec :
    
        \[
            0 = -\fdif{p}{x} + \mu \ffpart{u}{y}
        \]

        \begin{enumerate}
            \item Obtenez l'expression du profil de la contrainte visqueuse $\tau(y)$ pour $0 \leq \eta \leq 1$ ; puis, de là, celle du profil $u(y)$.
            
            \item Obtenez l'expression de la vitesse de frottement $u_{\tau}$. Pour la suite, on utilisera la notation $u^{+}$ pour définir le rapport $\frac{u}{u_{\tau}}$.
            
            Exprimez alors le profil de vitesse sous la forme $u^+=f(y^+,\eta)$, avec $y^+$ défini comme un écoulement turbulent.
            
            Au vu du résultat obtenu : que proposez-vous ici comme ``bonne limite'' pour la ``zone proche de la paroi'' : $\eta \leq \dots$ ?
            
            \item Obtenez l'expression de $h^+$ en fonction de $Re_d$.
            
            \item On considère le cas $Re_d = 1.0 \cdot 10^5$ (oui, c'est beaucoup trop élevé pour que l'écoulement soit laminaire ! Mais ça servira juste pour comparer avec le cas d'un écoulement turbulent considéré ci-après) : obtenez la valeur de $h^+$ correspondante.
    
        \end{enumerate}
    
    \item On considère ensuite le cas d'un écoulement turbulent hydrauliquement lisse :
    
    On a alors obtenu l'expression générale du profil des contraintes : $\Bar{\tau}(y) + \Bar{\tau}^t(y) = \Bar{\tau}_\omega (1-\eta)$, avec $\Bar{\tau}_\omega$ la contrainte à la paroi, $\Bar{\tau}(y)$ la contrainte due à la viscosité $\mu$ et $\Bar{\tau}^t(y)$ la ``contrainte effective due à la turbulence'': la définir. Pour modéliser cette contrainte, on utilisera une ``viscosité effective due à la turbulence'' $\mu_t$ (avec aussi $\nu_t = \frac{\mu_t}{\rho}$): la définir aussi.
    
    On obtient alors que $\Bar{u}^+ = y^+$ pour la zone toute proche de la paroi à dominance laminaire (zone I) : comparez ceci avec le résultat obtenu ci-dessus pour le cas d'un écoulement laminaire. La fin de cette zone se situe en environ quelle valeur de $y^+$ ?
    
    Pour la suite, on considère uniquement la zone à dominance turbulente (zone III). Le début de cette zone se situe en environ quelle valeur de $y^+$ ?
    
    On utilise ici un modèle très simplifié pour $\nu_t$ : $\nu_t = \kappa y \; \Bar{u}_{\tau}(1-\eta)$ avec $\kappa = 0.385$ pour $\eta \leq \eta_r$ (zone III-a, proche de la paroi) ; et puis $\nu_t = \beta \Bar{u}_{\tau} h$ avec $\beta = 0.070$ pour $\eta_r \leq \eta \leq 1$ (zone III-b) :
    
    \begin{enumerate}
        \item Dessinez (proprement et précisément !) le profil de $\frac{\nu_t}{\Bar{u}_{\tau} h}$ en fonction de $\eta$ : zone III-a et zone III-b.
        
        Calculez aussi la valeur précise du point de transition $\eta_r$ (car vous en aurez besoin pour la suite).
        
        \item On obtient alors, par intégration, que $\Bar{u}^+ = \frac{1}{\kappa} \log y^+ + C$ dans la zone III-a (cela a été fait en séance de TP). La valeur calibrée de $C$ est $C = 4.2$.
        
        Obtenez ici, par intégration, l'expression pour le profil modèle de $\Bar{u}^+$ dans la zone III-b. (Aide : imposer la continuité en $\eta = \eta_r$).
        
        Le profil modèle obtenu peut finalement aussi s'écrire sous la forme $\Bar{u}^+ = (\frac{1}{\kappa} \log y^+ + C) + G(\eta)$ : obtenez l'expression pour la fonction complément modèle $G(\eta)$.
        
        Note : Faites tous vos développements mathématiques en utilisant des symboles ($\eta$, $\eta_r$, $\kappa$, etc.), et non leurs valeurs numériques ! Fournissez chaque résultat final encadré.
        
        Dessinez finalement (proprement et précisément !) la fonction complément modèle $G(\eta)$ ; bien sûr en utilisant cette fois les valeurs numériques.
        
        \item Pourquoi la dérivée du profil modèle obtenu, $\frac{d\Bar{u}}{dy}$, est-elle aussi continue en $\eta = \eta_r$ ? (Aide : réponse très courte ; aucun calcul requis !)
        
        \item On considère de nouveau le cas $Re_d = 1.0 10^5$ ; donc avec le même débit :
        
        On devrait ici utiliser la formule des pertes de charge en canal en intégrant le profil modèle obtenu ci-dessus. Pour gagner du temps, on utilisera plutôt la formule déjà obtenue en séance de TP en intégrant un modèle plus sophistiqué :
        
        \[
            \frac{1}{\sqrt{\lambda}} = -3.03 \log_{10} \left( \frac{2.03}{Re_d} \frac{1}{\sqrt{\lambda}} \right)
        \]
        
        Obtenez alors la valeur de $\lambda$ (Aide : utiliser $\frac{1}{\sqrt{\lambda}} = 10.0$ comme première valeur et itérer) ; puis, de là, celle de $h^+$.
        
        Au vu du résultat obtenu : par combien de fois les pertes de charge en écoulement turbulent sont-elles ici plus importantes que celles en écoulement laminaire ?
    \end{enumerate}
    
\end{enumerate}

\nosolution

\end{document}
