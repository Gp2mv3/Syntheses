\documentclass[fr]{../../../../../../eplexam}

\hypertitle{Management humain}{6}{ECGE}{1321}{2015}{Juin}
{Florian Thuin}
{Nathalie Delobbe}

\section*{Exemples de questions ouvertes sur 8 points}

\begin{enumerate}
    \item Sur base d'un texte annexé (histoire d'un mec qui s'est fait
        virer parce qu'il gérait mal un département d'une société
        composé d'ingénieurs, de techniciens et d'assistants qui se
        mettaient sur la gueule et qui communiquaient mal avec
        l'extérieur, abusait des heures supp' et le mec laissait tout
        faire et il était en retard sur ses documents administratifs et
        dépassait les budgets), faire une analyse de son style de
        leadership et utiliser Ardoino pour expliquer les problèmes,
        illustrer avec des éléments du texte.
\end{enumerate}

\section*{Exemples de questions ouvertes sur 6 points}

\begin{enumerate}
    \item Définir les étapes du modèle de vie d'un groupe de Tuckman,
        expliquer en quoi elle consiste et quels sont les enjeux et
        risques de ces étapes. Illustrez à l'aide d'exemples au choix.
\end{enumerate}

\section*{Exemples de questions sur les travaux pratiques}

\begin{enumerate}
    \item Sur base du modèle de Schein, définir chaque partie du modèle
        et l'appliquer pour analyser la culture d'Ikea.
\end{enumerate}

\end{document}
