\documentclass[fr]{../../../../../../eplexam}

\hypertitle{Management humain}{6}{ECGE}{1321}{2015}{Juin}
{Florian Thuin}
{Nathalie Delobbe}

\section*{Exemples de questions ouvertes sur 8 points}

\begin{enumerate}
    \item Expliquez en quoi le leadership peut être relevant à analyser
        sur chacun des critères du modèle d'Ardoino.
    \item Cas sur le DRH de Godiva : quels sont les rôles du DRH qu'il
        assume (parmi les 4 rôles d'Ulrich). Argumenter, donner des
        exemples qu'il assume. Tensions et complémentarités entre ces
        rôles.
    \item Décrire la \og{}pensée de groupe\fg{}.
    \item Expliquer le modèle de Decy-Ryan sur l'auto-détermination et
        la motivation autonome. Définir les trois concepts et expliquer.
        Mettre en relation et tirer les converges/divergences avec le
        modèle bi-factoriel de Herzberg. Ensuite dire si c'était adapté
        à tout type d'organisation au 21e siècle et dire les critiques.
\end{enumerate}

\section*{Exemples de questions ouvertes sur 6 points}

\begin{enumerate}
    \item Analysez le texte sur Michael (texte introuvable sur un homme
        qui reçoit une prime pour un emploi supérieur mais il se rend
        compte qu'en fait il a moins que d'autre. Faire une analyse du
        problème de Michael via la théorie de l'équité et aussi celle de
        la justice organisationnelle). Quelles sont les possibilités qui
        s'offrent à lui sur base du modèle VIE.
    \item Expliquer le modèle de Hackmann et Oldham. Analyser
        l'entreprise libérée du documentaire \og{}le bonheur au
        travail\fg{} sur base de ce modèle. Illustrer à l'aide
        d'exemples du documentaire pour critique le modèle.
    \item Analysez l'entreprise libérée selon les 4 rôles d'Ulrich.
    \item Texte sur le leadership dans une société où tout
        partait en couille dans un service et fallait analyser sur trois
        niveaux d'Ardoino : individuel, relationnel/groupal et
        organisationnel.
\end{enumerate}

\section*{Exemples de questions sur les travaux pratiques}

\begin{enumerate}
    \item Faire le modèle de Hofdstede pour Ikea. Citez deux autres
        modèles d'analyse de la culture et expliquez les brièvement.
    \item Comparer le fonctionnement de Moults et Imagen Byas sur base
        d'un modèle vu au cours.
    \item Appliquez le modèle d'Ardoino pour votre travail de groupe.
    \item Développer brièvement le modèle de Quinn, dire où se situe
        IKEA avec deux arguments et citer deux autres théories/modèles
        sur la culture organisationnelle et expliquer brièvement (j'ai
        mis Schein et Hofstede)
\end{enumerate}

\end{document}
