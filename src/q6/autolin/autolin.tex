\documentclass[11pt,a4paper]{article}

% French
\usepackage[utf8x]{inputenc}
\usepackage[frenchb]{babel}
\usepackage[T1]{fontenc}
\usepackage{lmodern}
\usepackage{ifthen}

% Color
% cfr http://en.wikibooks.org/wiki/LaTeX/Colors
\usepackage{color}
\usepackage[usenames,dvipsnames,svgnames,table]{xcolor}
\definecolor{dkgreen}{rgb}{0.25,0.7,0.35}
\definecolor{dkred}{rgb}{0.7,0,0}

% Floats and referencing
\newcommand{\sectionref}[1]{section~\ref{sec:#1}}
\newcommand{\annexeref}[1]{annexe~\ref{ann:#1}}
\newcommand{\figuref}[1]{figure~\ref{fig:#1}}
\newcommand{\tabref}[1]{table~\ref{tab:#1}}

% Listing
\usepackage{listings}
\lstset{
  numbers=left,
  numberstyle=\tiny\color{gray},
  basicstyle=\rm\small\ttfamily,
  keywordstyle=\bfseries\color{dkred},
  frame=single,
  commentstyle=\color{gray}=small,
  stringstyle=\color{dkgreen},
  %backgroundcolor=\color{gray!10},
  %tabsize=2,
  rulecolor=\color{black!30},
  %title=\lstname,
  breaklines=true,
  framextopmargin=2pt,
  framexbottommargin=2pt,
  extendedchars=true,
  inputencoding=utf8x
}
\newcommand{\oz}{\textsc{Oz}}
\newcommand{\java}{\textsc{Java}}
\newcommand{\clang}{\textsc{C}}
\newcommand{\keyword}{mot clef}

% Math symbols
\usepackage{amsmath}
\usepackage{amssymb}
\usepackage{amsthm}

% Sets
\newcommand{\R}{\mathbb{R}}
\newcommand{\Rn}{\R^n}
\newcommand{\Rnn}{\R^{n \times n}}
\newcommand{\C}{\mathbb{C}}
\newcommand{\K}{\mathbb{K}}
\newcommand{\Kn}{\K^n}
\newcommand{\Knn}{\K^{n \times n}}

% Chemistry
\newcommand{\std}{\ensuremath{^{\circ}}}
\newcommand\ph{\ensuremath{\mathrm{pH}}}

% Theorem and definitions
\theoremstyle{definition}
\newtheorem{mydef}{Définition}
\newtheorem{mynota}[mydef]{Notation}
\newtheorem{myprop}[mydef]{Propriétés}
\newtheorem{myrem}[mydef]{Remarque}
\newtheorem{myform}[mydef]{Formules}
\newtheorem{mycorr}[mydef]{Corrolaire}
\newtheorem{mytheo}[mydef]{Théorème}
\newtheorem{mylem}[mydef]{Lemme}
\newtheorem{myexem}[mydef]{Exemple}
\newtheorem{myineg}[mydef]{Inégalité}

% Unit vectors
\usepackage{esint}
\usepackage{esvect}
\newcommand{\kmath}{k}
\newcommand{\xunit}{\hat{\imath}}
\newcommand{\yunit}{\hat{\jmath}}
\newcommand{\zunit}{\hat{\kmath}}

% rot & div & grad & lap
\DeclareMathOperator{\newdiv}{div}
\newcommand{\divn}[1]{\nabla \cdot #1}
\newcommand{\rotn}[1]{\nabla \times #1}
\newcommand{\grad}[1]{\nabla #1}
\newcommand{\gradn}[1]{\nabla #1}
\newcommand{\lap}[1]{\nabla^2 #1}


% Elec
\newcommand{\B}{\vec B}
\newcommand{\E}{\vec E}
\newcommand{\EMF}{\mathcal{E}}
\newcommand{\perm}{\varepsilon} % permittivity

\newcommand{\bigoh}{\mathcal{O}}
\newcommand\eqdef{\triangleq}

\DeclareMathOperator{\newdiff}{d} % use \dif instead
\newcommand{\dif}{\newdiff\!}
\newcommand{\fpart}[2]{\frac{\partial #1}{\partial #2}}
\newcommand{\ffpart}[2]{\frac{\partial^2 #1}{\partial #2^2}}
\newcommand{\fdpart}[3]{\frac{\partial^2 #1}{\partial #2\partial #3}}
\newcommand{\fdif}[2]{\frac{\dif #1}{\dif #2}}
\newcommand{\ffdif}[2]{\frac{\dif^2 #1}{\dif #2^2}}
\newcommand{\constant}{\ensuremath{\mathrm{cst}}}

% Numbers and units
\usepackage[squaren, Gray]{SIunits}
\usepackage{sistyle}
\usepackage[autolanguage]{numprint}
%\usepackage{numprint}
\newcommand\si[2]{\numprint[#2]{#1}}
\newcommand\np[1]{\numprint{#1}}

\newcommand\strong[1]{\textbf{#1}}
\newcommand{\annexe}{\part{Annexes}\appendix}

% Bibliography
\newcommand{\biblio}{\bibliographystyle{plain}\bibliography{biblio}}

\usepackage{fullpage}
% le `[e ]' rend le premier argument (#1) optionnel
% avec comme valeur par défaut `e `
\newcommand{\hypertitle}[7][e ]{
\usepackage{hyperref}
{\renewcommand{\and}{\unskip, }
\hypersetup{pdfauthor={#6},
            pdftitle={Synth\`ese d#1#2 Q#3 - L#4#5},
            pdfsubject={#2}}
}

\title{Synth\`ese d#1#2 Q#3 - L#4#5}
\author{#6}

\begin{document}

\ifthenelse{\isundefined{\skiptitlepage}}{
\begin{titlepage}
\maketitle

 \paragraph{Informations importantes}
   Ce document est grandement inspiré de l'excellent cours
   donné par #7 à l'EPL (École Polytechnique de Louvain),
   faculté de l'UCL (Université Catholique de Louvain).
   Il est écrit par les auteurs susnommés avec l'aide de tous
   les autres étudiants et votre aide est la bienvenue.
   Il y a toujours moyen de l'améliorer surtout que si le cours
   change, la synthèse doit être changée en conséquence.
   On peut retrouver le code source à l'adresse suivante
   \begin{center}
     \url{https://github.com/Gp2mv3/Syntheses}.
   \end{center}
   On y trouve aussi le contenu du \texttt{README} qui contient de plus
   amples informations, vous êtes invité à le lire.

   Il y est indiqué que les questions, signalements d'erreurs,
   suggestions d'améliorations ou quelque discussion que ce soit
   relative au projet
   % signalisations ou signalements ?
   sont à spécifier de préférence à l'adresse suivante
   \begin{center}
     \url{https://github.com/Gp2mv3/Syntheses/issues}.
   \end{center}
   Ça permet à tout le monde de les voir, les commenter et agir
   en conséquence.
   Vous êtes d'ailleurs invité à participer aux discussions.

   Vous trouverez aussi des informations sur le wiki
   \begin{center}
     \url{https://github.com/Gp2mv3/Syntheses/wiki}.
   \end{center}
   comme le status des synthèses pour chaque cours
   \begin{center}
     \url{https://github.com/Gp2mv3/Syntheses/wiki/Status}.
   \end{center}
   vous pouvez d'ailleurs remarquer qu'il en manque encore beaucoup,
   votre aide est la bienvenue.

   Pour contribuer au bug tracker et au wiki, il vous suffira de
   créer un compte sur Github.
   Pour interagir avec le code des synthèses,
   il vous faudra installer \LaTeX.
   Pour interagir directement avec le code sur Github,
   vous devez utiliser \texttt{git}.
   Si cela pose problème,
   nous sommes évidemment ouvert à des contributeurs envoyant leur
   changement par mail ou n'importe quel autre moyen.
\end{titlepage}
}{}

\ifthenelse{\isundefined{\skiptableofcontents}}{
\tableofcontents
}{}
}


\usepackage{graphicx}
\usepackage{caption}
\usepackage{subcaption}

\newcommand{\sbt}{\,\begin{picture}(-1,1)(-1,-3)\circle*{2.5}\end{picture}\ }
\newcommand{\umin}{u_\mathrm{min}}
\newcommand{\umax}{u_\mathrm{max}}

\usepackage{tikz}
\usepackage{pgfplots}
\usetikzlibrary{arrows,calc}

\hypertitle{Automatique Linéaire}{6}{INMA}{1510}
{Benoît Legat}
{Denis Dochain}

% http://me.berkeley.edu/~maxchen/_static/block_diagram_in_latex.pdf
\tikzstyle{block} = [draw, rectangle, minimum height=2em, minimum width=4em]
%fill=blue!20
\tikzstyle{sum} = [draw, fill=blue!20, circle, node distance=1cm]
\tikzstyle{input} = [coordinate] \tikzstyle{output} = [coordinate]
\tikzstyle{pinstyle} = [pin edge={to-,thin,black}]

% Pour éviter d'avoir un zéro (s+a) dans Tr(s) car il dégrade la dynamique,
% prendre un précompensateur a/(s+a)
% Par exemple, on ne peut pas l'identifier avec w_n, xhi et 10wn.
% voir 6.1, 6.2, 8.1

% Si état observable/commandable, utiliser retour d'état !

\section{Règle de Mason}
Le principe est expliqué assez clairement dans \cite{chau2002mason}
avec des exemples pour illustrer.
En résumé, on considère le schéma bloc comme un graphe dirigé et
pour calculer la fonction de transfert entre $a$ et $b$.
On prend
\begin{itemize}
  \item tous les chaines directes $F_k$ (c'est des chemins pour prendre la vocable des graphes, ce qui signifie qu'on ne fait pas de cycle),
  \item toutes les boucles $\{L_i\}$ (les cycles dans le vocable des graphes),
  \item toutes les boucles qui ne touchent \emph{pas} $F_k$: $\{L_{k;i}\}$.
\end{itemize}
Et on a la fonction de transfert par la formule suivante
\[ H(s) = \frac{\sum_k F_k\Delta(\{L_{k;i}\})}{\Delta(\{L_i\})} \]
où
\[ \Delta(\{L_j\}) = 1 - \sum_{j_1} L_{j_1} + \sum_{j_1,j_2} L_{j_1}L_{j_2} - \sum_{j_1,j_2,j_3} L_{j_1}L_{j_2}L_{j_3} + \cdots \]
avec $j_1, j_2, \ldots, j_n$ tels que les boucles $L_{j_1}, L_{j_2}, \ldots, L_{j_n}$ ne se touchent \emph{pas}.
Si toutes les boucles, se touchent, ça devient donc simplement
\( 1 - \sum_{j_1} L_{j_1}. \)
On remarque que $\Delta(\emptyset) = 1$.
La plupart du temps, les chaines directes touchent toutes les boucles et le numérateur devient donc $\sum_k F_k$ car $\{F_{k;i}\} = \emptyset$.

\begin{myexem}
  \label{ex:masonex}
  Pour l'exemple de la \figuref{masonex}, on a
  \begin{itemize}
    \item deux chaines directes qui valent respectivement $A(s)C(s)$ et $A(s)D(s)$,
    \item deux boucles qui valent respectivement $-A(s)B(s)$ et $D(s)E(s)$,
    \item la première chaine directe ne touche que par la première boucle et la deuxième passe les touchent toutes.
  \end{itemize}
  On a donc, comme les deux boucles ne se touchent pas,
  \[ H(s) = \frac{A(s)C(s)(1 - D(s)E(s)) + A(s)D(s)}{1 + A(s)B(s) - D(s)E(s) - A(s)B(s)D(s)E(s)}. \]
  \begin{figure}[!ht]
    \centering
    \begin{tikzpicture}[auto, node distance=2cm,>=latex']
      \node [input, name=input] {};
      \node [sum, right of=input] (sum) {};
      \node [block, right of=sum, node distance=2cm] (A) {$A(s)$};
      \node [block, below of=A] (B) {$B(s)$};
      \node [block, right of=A, node distance=6cm] (C) {$C(s)$};
      \node [block, below of=C, node distance=2cm] (D) {$D(s)$};
      \node [block, below of=D, node distance=2cm] (E) {$E(s)$};
      \node [sum, right of=C, node distance=2cm] (sum2) {};
      \node [sum, left of=D, node distance=1.5cm] (sum3) {};
      \node [output, right of=sum2] (output) {};
      \draw [->] (input) -- node {$a(t)$} (sum);
      \draw [->] (A) -- node[name=Ar,pos=0.3] {} node[name=Cl,pos=0.7] {} (C);
      \draw [->] (Ar) |- (B);
      \draw [->] (B) -| node[pos=0.45] (Bl) {} node[pos=0.95] {$-$} (sum);
      \draw [->] (sum) -- (A);
      \draw [->] (Cl) |- (sum3);
      \draw [->] (sum3) -- (D);
      \draw [->] (C) -- (sum2);
      \draw [->] (D) -| node[pos=0.3] (Dr) {} (sum2);
      \draw [->] (Dr) |- (E);
      \draw [->] (E) -| (sum3);
      \draw [->] (sum2) -- node {$b(t)$} (output);
    \end{tikzpicture}
    \caption{Schéma-block de l'exemple~\ref{ex:masonex}.}
    \label{fig:masonex}
  \end{figure}
\end{myexem}

\section{Emballement}

\begin{figure}[!ht]
  \centering
  \begin{tikzpicture}[auto, node distance=2cm,>=latex']
    \node [input, name=input] {};
    \node [sum, right of=input] (sum) {};
    \node [block, right of=sum, node distance=2cm] (C) {$k_p + \frac{k_i}{s}$};
    \node [block, right of=C, node distance=4cm] (S) {$\max(\umin,\min(\umax,\sbt))$};
    \node [block, right of=S, node distance=4cm] (G) {$G(s)$};
    \node [sum, right of=G, node distance=2cm] (sum2) {};
    \node [block, above of=sum2, node distance=1.5cm] (H) {$H(s)$};
    \node [input, above of=H, name=pert, node distance=1cm] {};
    \node [output, right of=sum2] (output) {};
    \node [below of=S] (feedback) {};
    \draw [->] (input) -- node {$r(t)$} (sum);
    \draw [->] (pert) -- node {$v(t)$} (H);
    \draw [->] (H) -- (sum2);
    \draw [->] (C) -- node {$u_c(t)$} (S);
    \draw [->] (S) -- node {$u(t)$} (G);
    \draw [->] (sum) -- node {$e(t)$} (C);
    \draw [->] (G) -- (sum2);
    \draw [->] (sum2) -- node[name=y] {$y(t)$} (output);
    \draw (y) |- ($(S)+(0,-1.5cm)$);
    \draw [->] ($(S)+(0,-1.5cm)$) -| node[pos=0.95] {$-$} (sum);
  \end{tikzpicture}
  \caption{Schéma-block d'un système avec saturation.}
  \label{fig:windup-block}
\end{figure}

Toute notre analyse se base sur le fait que nous avons des systèmse LTI (Linear Time Invariant).
Pourtant, en pratique, il y a pas mal de cause de non-linéaire.
Une assez courante est la \emph{saturation} de la commande.
Ça a lieu lorsque le système ne sait qu'appliquer une commande entre $\umin$ et $\umax$.
On modélise alors notre système avec une composante non-linéaire appelée saturation.
Son entrée est la commande voulue $u_c$ et sa sortie la commande vraiment appliquée $u$.
On a la relation $u = \min(\umax, \max(\umin, u_c))$, c'est à dire que si $u_c \geq \umax$,
on prend $u = \umax$ et si $u_c \leq \umin$, on prend $u = \umin$, sinon on a $u = u_c$.
Le système est représenté par la \figuref{windup-block}.

Comme la composante de saturation est non-linéaire on ne peut plus représenter cette composante
par une convolution en temporelle ou un produit en fréquentiel et on ne peut donc faire d'analyse
fréquentielle.

\begin{figure}
  \centering
  \includegraphics[width=0.6\linewidth]{schmidwindup.png}
  \caption{Exemple de l'application d'un système anti-emballement pris de~\cite{schmid2005windup}.
  La courbe rouge montre le comportement du système sans saturation.
  On rajoute une saturation à \num{1.5} pour la courbe bleue.
  La courbe verte a la saturation et le système emballement.}
  \label{fig:schmidwindup}
\end{figure}

Dans la \figuref{schmidwindup}, on voit que sans la saturation,
on arrive plus vite à la consigne.
Avec la saturation, $u$ n'est pas aussi grand qu'on aimerait et on va donc
moins vite.
Cette lenteur fait que l'intégrale a plus le temps de gonfler.
En effet, on voit que l'aire entre la consigne 1 et la courbe rouge dans
le graphe de $y$ est plus petite que celle entre la consigne et la courbe bleue.
Lorsqu'on atteint la consigne, l'intégrale est donc plus grosse pour le système
avec saturation.
Il va donc prendre beaucoup plus de temps à réaliser qu'il faut redescendre
et l'intégrale risque d'à nouveau trop gonfler.

\begin{figure}
  \centering
  \begin{subfigure}{0.49\linewidth}
    \includegraphics[width=\linewidth]{windup.png}
    \caption{Exemple d'emballement.}
    \label{fig:windup}
  \end{subfigure}
  \begin{subfigure}{0.49\linewidth}
    \includegraphics[width=\linewidth]{anti-windup.png}
    \caption{Application d'un système anti-emballement à l'exemple de la \figuref{windup}.}
    \label{fig:anti-windup}
  \end{subfigure}
  \caption{Exemple d'emballement avec application d'un système anti-emballement.}
  \label{fig:windup-cm}
\end{figure}

Il arrive qu'il prenne tellement de temps à s'en rendre compte, c'est à dire
que l'intégrale prend tellement de temps à dégonfler que l'erreur est très
grande lorsque l'intégrale est enfin nulle se qui fait qu'on atteint l'autre borne
pour de la commande très rapidement.
On est alors reparti pour un tour.
C'est ce qu'on observe à la \figuref{windup} en $t = 19$,
dans le graphe de $u$, on passe tellement vite d'une borne à l'autre qu'on a l'impression
que c'est un point de discontinuité.
On voit en effet, en regardant le graphe pour $y$ à $t = 19$ qu'on avait atteint une erreur
assez importante.

\subsection{Système anti-emballement}

\begin{figure}[!ht]
  \centering
  \begin{tikzpicture}[auto, node distance=2cm,>=latex']
    \node [input, name=input] {};
    \node [sum, right of=input] (sum) {};
    \node [sum, right of=sum] (sum4) {};
    \node [block, right of=sum4, node distance=2cm] (C) {$k_p + \frac{k_i}{s}$};
    \node [block, right of=C, node distance=4cm] (S) {$\max(\umin,\min(\umax,\sbt))$};
    \node [block, right of=S, node distance=4cm] (G) {$G(s)$};
    \node [sum, right of=G, node distance=2cm] (sum2) {};
    \node [block, above of=sum2, node distance=1.5cm] (H) {$H(s)$};
    \node [input, above of=H, name=pert, node distance=1cm] {};
    \node [output, right of=sum2] (output) {};
    \node [below of=S] (feedback) {};
    \draw [->] (input) -- node {$r(t)$} (sum);
    \draw [->] (pert) -- node {$v(t)$} (H);
    \draw [->] (H) -- (sum2);
    \draw [->] (C) -- node[below,name=uc] {$u_c(t)$} (S);
    \draw [->] (S) -- node[below,name=u] {$u(t)$} (G);
    \draw [->] (sum) -- node {$e(t)$} (sum4);
    \draw [->] (sum4) -- (C);
    \draw [->] (G) -- (sum2);
    \draw [->] (sum2) -- node[name=y] {$y(t)$} (output);
    \draw (y) |- ($(S)+(0,-1.5cm)$);
    \draw [->] ($(S)+(0,-1.5cm)$) -| node[pos=0.95] {$-$} (sum);

    \node [block, above of=C, node distance=1.5cm] (a) {$\alpha$};
    \node [sum, above of=uc, node distance=1.85cm] (sum3) {};
    \draw [->] (sum3) -- (a);
    \draw [->] (uc) -- node[pos=0.9] {$-$} (sum3);
    \draw [->] (u) |- (sum3);
    \draw [->] (a) -| (sum4);
  \end{tikzpicture}
  \caption{Schéma-block d'un système avec saturation et système anti-emballement.}
  \label{fig:anti-windup-block}
\end{figure}

On voit que les problèmes d'emballement sont liés à la composantes intégrale.
Il nous faut alors, pour remédier à l'emballement, ajouter un système qui détectera quand
$u_c \geq u$, c'est à dire que $u_c \geq \umax$ ou $u_c \leq u$, c'est à dire $u_c \leq \umin$,
et qui videra l'intégrale qui est trop positive ou trop négative.
Un système anti-emballement classique est représenté par la \figuref{anti-windup-block}.

\begin{itemize}
  \item Si $u = u_c$, ce système n'a aucun effet car $u - u_c = 0$.
  \item Si $u \leq u_c$, ça peut signifier 2 choses
    \begin{description}
      \item[Une trop grosse action proportionnelle]
        L'erreur est très grande par rapport à $\umax$, ou plus précisément que $k_p e(t) \gg \umax$,
        on va donc être trop lent pour arriver à $e(t) = 0$ et on va donc intégrer plus que voulu.
        $u - u_c$ est négatif, avec $e + \alpha (u-u_c)$ comme entrée à $C(s)$,
        on aura toujours $\umax \leq u_c$\footnote{même si $\alpha$ est grand, on ne peut pas avoir $u_c < \umax$ car alors $u = u_c$ et on et donc il n'y a plus que $e$ à l'entrée de $C(s)$ et donc $\umax \leq u_c$.}
        mais on intégrera $e + \alpha (u-u_c) < e$, même si on intégrera plus longtemps que sans la saturation, on intégrera moins,
        l'intégrale sera alors moins gonflée lorsqu'on aura $e(t) = 0$.
      \item[Une trop grosse action intégrale]
        L'erreur n'est plus spécialement grande, elle est même peut-être déjà devenue négative mais
        on continue à vouloir $u_c \geq \umax$ à cause de l'intégrale qui a beaucoup gonflé.
        Pour la même raison que précédemment, on aura toujours $u_c \geq \umax$,
        l'effet va être donc uniquement dans l'intégrale, on va la vider plus rapidement qu'elle se viderait normalement.
    \end{description}
  \item Si $u \geq u_c$, la discussion est pareille que pour $u \leq u_c$.
\end{itemize}
En résumé, l'action du système va être sur l'intégrale, elle va l'empêcher de trop gonfler et puis va la vider plus rapidement.
Elle ne va cependant agir que lorsque la commande sature et ne va donc pas poser de problème dans les cas où il n'y a pas de risque
d'emballement.
Elle ne va faire \emph{pas} descendre $u_c$ en dessous de $\umax$ ou au dessus de $\umin$ quand on sature,
elle ne va donc \emph{pas} augmenter le temps de réponse.

Illustrons cela avec pour commencer la \figuref{schmidwindup} où on voit que le seul effet du système anti-emballement est lorsqu'on dépasse
$\umax = 1.5$.
On voit en pointillé $u_c$ qui est plus petit que pour a courbe bleue mais qui ne descend pas en dessous de $\umax$ tant que $y(t) \leq r(t)$.
Il est par contre beaucoup plus rapide à passer en dessous de $\umax$ lorsuqe $y(t) \geq r(t)$ ce qui montre bien qu'il a empêché à l'intégrale de trop gonfler.

Pour la \figuref{anti-windup}, on voit à nouveau que le système anti-emballement ne fait pas passer $u_c$ en dessous de $\umax$.
Au tout début, on voit que l'intégrale intégre quelque chose de très négatif, ça doit être dû au fait qu'au début, on demande un
très grand $u_c$ et donc $\alpha (u_c-u)$ est très négatif.
Par la suite, comme $\alpha(u_c-u)$ est négatif, l'action proportionnelle est moins forte et l'action intégrale est négative donc
$u_c - u$ est moins important, de telle sorte que $e(t) \geq -\alpha (u_c - u)$, on intègre donc du positif et l'intégrale remonte.
Elle monte d'ailleurs beaucoup plus lentement que pour la \figuref{windup} grâce au fait qu'on intègre aussi $\alpha (u_c - u)$ qui
est négatif.
Au final, lors $y(t) \geq r(t)$, l'intégrale est vide et $u$ ne sature plus, on repasse donc dans un système LTI et le système
anti-emballement n'a plus d'effet.

% TODO astrom p. 307pdf 295vrai

\section{Nyquist}
Nyquist est un outil qui nous permet de trouver le nombre de zéros instables de
$1 + C(s)G(s)$ si on connait le nombre de pôle instables de $C(s)G(s)$ et
donc de déterminer si notre fonction de transfert avec $1+C(s)G(s)$ au dénominateur
est stable, c'est à dire qu'il n'y a aucun zéro instable.

Soit $C(s)G(s) = \frac{N(s)}{D(s)}$, on a $1 + C(s)G(s) = \frac{D(s)+G(s)}{D(s)}$.
Les pôles de $C(s)G(s)$ sont donc les mêmes que ceux de $1 + C(s)G(s)$.
Le théorème de Cauchy nous dit que si on prend un parcours fermé $\Gamma$,
\[ N = P - Z \]
où $N$ est le nombre de fois que $1 + CG$ tourne autours de l'origine dans le sens
trigonométrique et $P$ et $Z$ sont le nombre de pôles et de zéros qui sont dans
le contour (on ne peut pas l'appliquer s'il y en a sur le contour).

\begin{figure}
  \centering
  \begin{subfigure}{0.49\linewidth}
    \includegraphics[width=\linewidth,height=\linewidth]{nyquistpath.jpg}
    \caption{Courbe fermée regroupant les racines instables.}
    \label{fig:nyquistpath}
  \end{subfigure}
  \begin{subfigure}{0.49\linewidth}
    \includegraphics[trim=22cm 0cm 0cm 0cm,clip,width=\linewidth,height=\linewidth]{nyquistplot.jpg}
    \caption{Image de la courbe par $C(s)G(s)=\frac{90}{(s+3)(s+6)}$.}
    \label{fig:nyquistplot}
  \end{subfigure}
  \caption{Exemple de diagramme de Nyquist~\cite{cheever2013nyquist}.}
  \label{fig:nyquist}
\end{figure}

Si on prend le coutour de la \figuref{nyquistpath},
on regroupe donc toutes les racines instables.
Si on connait le nombre de pôles instables de $CG$, on a $P$ et ce qu'on cherche,
$Z$ est simplement $P - N$.
Si $CG$ a des racines ou pôles sur l'axe imaginaire, on ne peut pas appliquer le
critère car ils sont sur le contour, c'est problématique car on a souvent un
pôle nul avec l'action intégrale.
La solution est de passer par un petit demi-cercle de rayon $\epsilon$ autour
de 0.
Pour ce cours, on ne s'en soucie pas trop.
\cite{cheever2013nyquist} apporte de plus amples informations à ce sujet.

Au lieu de plotter $1 + CG$ et de regarder combien de fois on tourne autour
de l'origine,
on va plotter $CG$ et regarder combien de fois on tourne autour de $-1 + 0j$
ce qui revient au même.
On obtient alors un graphe semblable à la \figuref{nyquistplot}.

Pour trouver $N$, il suffit de prendre une demi-droite partant de $-1$ et
allant à l'infini dans n'importe quel sens, à chaque fois qu'on rencontre
la courbe, si elle va vers la gauche, on fait $+1$, sinon, on fait $-1$.
Au final, on aura $N$ (c'est bien illustré par \cite{cheever2013nyquist}).

\subsection{Allure classique}
Au final, on a souvent $P = 0$ et $CG$ a souvent soit des pôles ou racines
conjuguées, soit des pôles ou racines réelles négatives, soit un pôle
nul dû à l'action intégrale.
On nos pôles et racines sont toujours avec leur conjuguées car on a des coefficients
réels dans nos polynômes.

Dans le parcours de la \figuref{nyquistpath}, on traverse la droite
$s = j\omega$ de $\omega = -\infty$ à $\omega = \infty$ en passant par $\omega = 0$
(ou plutôt en faisant un demi-cercle de rayon $\epsilon$ autour).
En $\omega = 0$, la phase de $CG$ est soit 0 comme pour la \figuref{nyquistplot},
  soit $-\ang{90}$ si on a une action
intégrale.
En effet la phase est la somme des phase des différents facteurs.
\begin{itemize}
  \item
    Pour les racines et pôles réels positifs, on a $s + r_i$ avec $r_i > 0$ donc
    pour $s = 0$, on a une phase nulle.
  \item
    Pour les racines et pôles conjuguées, ils ont des phases opposées donc leur somme
    est nulle.
  \item
    L'action intégrale a la phase de $\frac{1}{\epsilon j} = -\frac{1}{\epsilon} j \to -\infty j$
    qui a un module infini et une phase de $-\ang{90}$.
    Le module infini explique pourquoi on part souvent vers $-\infty j$ quand on a une action
    intégrale.
\end{itemize}

Si $CG$ est strictement propre (le degré du dénominateur est plus grand que celui du dénominateur),
lorsque $|\omega| \to \infty$, le module de $CG$ tend vers 0, on passe donc par l'origine comme
le montre la \figuref{nyquistplot}.

Comme on a supposé que $CG$ n'a pas de pôles instables, $P = 0$ et donc $Z = -N$.
Il faut donc que $N = 0$ pour avoir un système stable.

Comme nos polynômes sont à coefficients réels,
$C(-j\omega)G(-j\omega) = C(\bar{j\omega})G(\bar{j\omega}) = \bar{C(j\omega)G(j\omega)}$.
On fait donc le même trajet pour $\omega$ allant de $-\infty$ à $0$ qu'on a fait
pour $\omega$ allant de $0$ à $\infty$ mais dans l'autre sens et de l'autre côté de l'axe réel.
C'est exactement ce qu'on voit à la figure \figuref{nyquistplot}.

S'il n'y a pas d'action intégrale, la partie de la courbe de $\omega = 0$ à $\omega = \infty$
part donc de $C(0)G(0)$ qui est un réel positif
et abouti en $0$ si $CG$ est strictement propre, comme on a vu, on fait après exactement pareil
de l'autre côté de l'axe réel.
Avec action intégrale, on part de $-\infty j$ et on abouti en $0$ puis on fait pareil
mais vers $\infty j$ puis pendant qu'on fait notre petit demi-cerce de rayon $\epsilon$ autour
de l'origine, on passe de $\infty j$ à $-\infty j$ par la droite (parce que).

On peut donc juste regarder ce qu'on fait de $\omega = 0$ à $\omega = \infty$.
Si on passe à gauche du point $-1$,
alors on fera un tour dans le sens horlogique et donc
$N = -1$ ce qui fait $Z = 1$, on est donc instable.
Si on y passe à droite, alors $N = 0$ et donc $Z = 0$ et on est stable!

\section{Robustesse}

\[ T_r(s) = \frac{\omega_n^2}{s^2 + 2 \zeta \omega_n s + \omega_n^2}. \]
On a
\[ \sigma_M = \inf \left|1 + C(j\omega)G(j\omega)\right| =
\inf 1 + \frac{1 - 2\left(\frac{\omega}{\omega_n}\right)^2}{\left(\frac{\omega}{\omega_n}\right)^4 + 4\zeta^2\left(\frac{\omega}{\omega_n}\right)^2} \]
$\sigma_M$ est atteint pour
\[ \left(\frac{\omega}{\omega_n}\right)^2 = \frac{1 + \sqrt{1 + 8\zeta^2}}{2}. \]

De plus,
\begin{align*}
  \frac{\omega_c}{\omega_n} & = \sqrt{\sqrt{4\zeta^4 + 1} - 2\zeta^2}.
\end{align*}

% Bode phase et log module s'additionnent log(k0) = 0

\biblio

\end{document}
