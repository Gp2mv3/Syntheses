\subsection*{Rappel}
\begin{framed}
\begin{itemize}
\item[A.]$A \leq_a B$ ssi B recursif $\Rightarrow$ A recursif
\item[B.]$A \leq_f B$ ssi $\exists f$ total et calculable tel que $a \in A \Leftrightarrow f(a) \in B$
\end{itemize}
\end{framed}

% Exercice 1
\subsection{}
\begin{itemize}
	\item[(a)] $T(n) = 1+2+...+n+1 = \frac{(n+1)n}{2}+1=O(n^2)$
	\item[(b)] $T(n) = 2^{log_2(n)} = O(n)$
\end{itemize}

% Exercice 2
\subsection{}
\begin{itemize}
	\item[(a)] $\forall A: A\leq_a \overline{A}$
	Soit $\overline{A}$ recursif
	$$\exists P_{\overline{A}} : \phi_{P_{\overline{A}}}(a) = 
	\left\{
	\begin{array}{ll}
	1 & if\ a \in \overline{A} \\
	0 & otherwise
	\end{array}
	\right.
	$$
	$$P_A(a) \equiv return 1 - P_{\overline{A}}(a);$$
	Nous avons donc $\overline{A}$ recursif qui entraine la récursivité de $A$. Des lors
 $A\leq_a \overline{A}$
	\item[(b)] $\forall A,B: A \leq_f B \land B \ r.e. \Rightarrow A \ r.e.$
	Nous allons utiliser les propriétés suuivantes:
	\begin{itemize}
	\item $A \leq_f B$ ssi $\exists f$ total et calculable tel que $a \in A \Leftrightarrow f(a) \in B$
	\item $B \ r.e. \Rightarrow \exists P_B : \phi_{P_B}(b) = 
	\left\{
	\begin{array}{ll}
	1 & if \ b \in B \\
	0 & otherwise
	\end{array}
	\right.$
	\end{itemize}
	Soit $f$ une fonction totale calculable $\Rightarrow \exists P_f : \phi_{P_f} = f$.
	$$P_A(a)=\left[\begin{array}{l}b = P_f(a);\\ return\ P_B(b);\end{array}\right.$$
	Nous avons donc $A$ récursif énumérable.
	\item[(c)] $A \leq_f \overline{A}$
	
	Soit $A = \left\{q : P_q(0) = \perp\right\}$. Nous avons donc $A$ qui est non recursif énumérable. Nous pouvons alors définir $\overline{A} = \left\{q : P_q(0) \neq \perp\right\}$, c'est à dire que $\overline{A}$ est une récursif énumérable. Cela conduit à une contradiction, si l'on reprend la regle \textsc{B.}, nous devrions avoir une fonction $f$ tel que $a \in A \Leftrightarrow f(a) \in \overline{A}$.

	\item[(d)] $A \leq_a B \land B \ r.e. \Rightarrow A \ r.e.$

	Soit $A = \overline{k}$ et $B = k$, avec $k$ le programme déterminant les programmes qui s'arretent, on a:
	$$\overline{k} \leq_a k \land k \ r.e.$$
	Si $\overline{k} \ r.e.$ alors comme $k \ r.e.$, on aurait $k$ récursif et ça ca n'est pas possible.
\end{itemize}

% Exercice 3
\subsection{}
\begin{itemize}
	\item[(a)] Deux conditions nécessaires
		\begin{itemize}
			\item[i.] $Halt \ r.e.$
			\item[ii.] $\forall B \in \ r.e. : B \leq_a Halt$
		\end{itemize}
	\item[(b)] $P_{Halt} \equiv \left[ \begin{array}{l}P_n(x);\\return\ 1; \end{array} \right.$

Soit $B \in \ r.e.$, $\exists P_B : \phi_{P_B}(b) = 
\left\{
\begin{array}{ll}
	1 & if \ b \in B \\ 
	0 & otherwise
\end{array}
\right.$

Supposons $Halt$ recursif
$$P_{recB}(b) \equiv 
\left[
\begin{array}{l}
if\ P_{Halt}(B,b)\\
\ \ \ return \ P_B(b);\\
else\ return \ 0;
\end{array}
\right.
$$
\end{itemize}

% Exercice 4
\subsection{}

Nous allons montrer $H_{CYCLE} \leq_f H_{PATH}$ et $H_{PATH} \leq H_{CYCLE}$
\begin{enumerate}
	\item $H_{PATH} \leq H_{CYCLE}$ ssi $\exists f$ totale et calculable tel que $G \in H_{PATH} \Leftrightarrow f(G) \in H_{CYCLE}$ Nous allons montrer grace à un exemple que:
	\begin{itemize}
		\item $G \in H_{PATH} \Rightarrow f(G) \in H_{CYCLE}$
		\item $G \notin H_{PATH} \Rightarrow f(G) \notin H_{CYCLE}$
	\end{itemize}
	Il suffira de créer une fonction $f$ qui va ajouter un noeud relié à tout les noeuds.
	\item
\end{enumerate}

% Exercice 5
\subsection{}

\begin{itemize}
	\item[(a)]
	\item[(b)]
\end{itemize}