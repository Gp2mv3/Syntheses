\subsection{} % Exercise 1.
Prove the three following implications:
\begin{enumerate}
	\item \(SD \land U \implies SA\);
	\item \(CA \implies CD\);
	\item \(CA \land SD \land U \implies SA \land CD \land S\).
\end{enumerate}

\begin{solution}
	Assume we have a language \(D\),
	and that \(\mathbf{J}\) is the set of \java{} programs.
	\begin{enumerate}
		\item Let \(f\) be the interpreter for \(D\).
		By \(U\), we know that \(f\) is then \(D\)-computable.
		Applying \(SD\), \(f\) is also computable.
		This is what \(SA\) states,
		hence it completes the proof.
		\item We know that \ has the \(SA\) property.
		By \(CA\), this means that
		\(\forall p^J \in \mathbf{J}, \exists d \in D :
		\phi_d = \phi_{p^J}\).
		This means that every \java-computable function function
		is \(D\)-computable.

		We also know that \java{} has the \(CD\) property,
		meaning that every computable function is \java-computable.
		Hence, combining these results,
		we find that every computable function is \(D\)-computable,
		which is precisely the \(CD\) property for \(D\).
		\item We can simplify this expression first:
		we know that \(SD \land U \implies SA\),
		and that \(CA \implies CD\).
		We only have to prove
		\(CA \land SD \land U \land SA \land CD \implies S\).
		The outline of the proof is as follows:
		say we have a two-argument function in \(D\).
		We can find an equivalent two-argument function
		in \(\mathbf{J}\),
		which we can specialize for one of its arguments,
		yielding an equivalent one-argument function.
		This function has a counterpart in \(D\),
		which also has one argument.
		Since this function is equivalent to the initial function,
		this means that it is possible to specialize functions in \(D\).

		Since \(D\) has the \(SA\) property,
		and \(\mathbf{J}\) has the \(CA\) property,
		\[
		\forall d(x, y) \in D, \exists f \textnormal{ total and computable such that } f(d) = p^J \in \mathbf{J}:
		\phi_d(x, y) = \phi_{p^J}(x, y)\,.
		\]

		Next, since \(\mathbf{J}\) has the \(S\) property,
		\[
		\exists s^J \textnormal{total and computable such that } s^J(x) = p_x^J \in \mathbf{J}:
		\phi_{p^J}(x, y) = \phi_{p_x^J}(y)\,.
		\]

		Now, we know that \(D\) has the \(CA\) property,
		and that \(\mathbf{J}\) has the \(SA\) property.
		This allows us to write
		\begin{align*}
		\exists f' \textnormal{ total and computable such that } f(p_x^J) = d_x \in D &:
		\phi_{p_x^J}(y) = \phi_{d_x}(y)\\
		\implies \forall d(x, y) \in D, \exists d_x(y) \in D &: \phi_{d_x} = \phi_{p_x^J} = \phi_{p^J}(x, y) = \phi_{d}(x, y)\\
		\implies \forall d(x, y) \in D, \exists d_x(y) \in D &: \phi_{d_x}(y) = \phi_d(x, y)\,.
		\end{align*}

		Finally, we know that \(\exists f'' = f' \circ s^J \circ f\),
		which is total and computable by virtue of being
		a composition of total computable functions.
		This means we can write
		\[
		\forall d \in D, \exists f'' \textnormal{ total and computable such that } f''(x) = d_x: \phi_d(x, y) = \phi_{f''(x)}(y) = \phi_{d_x}(y)\,.
		\]
		This is the \(S\) property for \(D\),
		which ends the proof.
	\end{enumerate}
\end{solution}

\subsection{} % Exercise 2.
What is the difference between \(U\) and \(SA\)?
Does one imply the other?
If not, what property do you have to add to \(SA\) to imply \(U\)?

\begin{solution}
	The difference lies in which model the interpreter for \(D\)
	is computable in.
	For the \(U\) property, the interpreter is \(D\)-computable,
	whereas for the \(SA\) property, it is computable.

	Neither implies the other,
	but by adding \(CD\) to \(SA\),
	we guarantee that every computable function is \(D\)-computable.
	Applying this to the interpreter, which we know is computable by \(SA\),
	we find that the interpreter for \(D\) is \(D\)-computable,
	which is precisely the \(U\) property.
\end{solution}

\subsection{} % Exercise 3.
Among the properties \(SD, CD, SA, CA, U\) and \(S\),
which ones hold for the following formalisms?

\begin{enumerate}
	\item The primitive recursive functions.
	\item The Oracle Turing Machines, where the oracle set is
	\(K = \{n \mid P_n(n) \textnormal{ terminates}\}\).
	\item \java.
\end{enumerate}

\begin{solution}
\begin{enumerate}
	\item Primitive recursive functions have the following properties:
	\begin{itemize}
		\item \(SD\): OK.
		Primitive recursive functions are a subset
		of recursive functions, which are all computable
		since they form a valid model of computability.
		\item \(CD\): KO.
		Consider \(\bot\), which is a computable function,
		yet is not primitive recursive.
		\item \(SA\): OK.
		The interpreter for primitive
		recursive functions is computable,
		since the interpreter for recursive functions is computable
		by virtue of them being a valid model of computability,
		and the primitive recursive functions are a subset
		of recursive functions.
		\item \(CA\): KO.
		There is \java{} function which computes
		the interpreter for primitive recursive functions,
		which is not computable
		by primitive recursive functions themselves.
		\item \(U\): KO.
		By the Hoare-Allison theorem,
		any non trivial language which only computes total functions
		cannot express its own interpreter.
		\item \(S\): OK.
		The proof of the \(\smn\) theorem,
		of which the \(S\) property is a special case,
		also works for primitive recursive functions.
	\end{itemize}
	\item Oracle Turing Machines with oracle \(K\)
	have the following properties:
	\begin{itemize}
		\item \(SD\): KO.
		This oracle set is not recursive,
		since it allows us to decide \(K\).
		By a reduction argument,
		this means we can compute the halting function
		for standard Turing machines with this Oracle Turing Machine,
		which is not computable by a standard Turing machine.
		\item \(CD\): OK.
		If one does not use the oracle,
		the oracle Turing machine reduces to a regular Turing machine.
		\item \(SA\): KO.
		If it were computable,
		then we could decide the halting problem.
		\item \(CA\): OK.
		For every standard Turing machine,
		there exists an Oracle Turing Machine
		which computes the same function
		(e.g. by not using the oracle).
		\item \(U\): OK.
		\item \(S\): OK.
		The proof of the \(\smn\) theorem also works
		for Oracle Turing Machines.
	\end{itemize}
	\item The \java{} language has the following properties:
	\begin{itemize}
		\item \(SD\): OK.
		The \java{} language is a valid model of computability.
		\item \(CD\): OK.
		The \java{} language is a valid model of computability.
		\item \(SA\): OK.
		The interpreter exists,
		otherwise we would not be able to run \java{} programs.
		\item \(CA\): OK.
		The \java{} language is a valid model of computability.
		\item \(U\): OK.
		The interpreter can be written as a \java{} program.
		\item \(S\): OK.
		Since we have \(CA, SD\) and \(U\),
		\(S\) is implied by the proof of Exercise~9.1.3.
		This proof builds on the fact that \java{}
		is a valid model of computability.
	\end{itemize}
\end{enumerate}
\end{solution}

\subsection{} %  Exercise 4.
True or false?
\begin{enumerate}
	\item There exists languages that have the \(SD\) property,
	but not the \(SA\) property.
	\item If problem \(A\) can be reduced to problem \(B\)
	and \(A\) is computable, then \(B\) is computable.
	\item A formalism satisfying the \(SA, SD\) and \(U\) properties
	is a good computability formalism.
\end{enumerate}

\begin{solution}
\begin{enumerate}
	\item True.
	One can construct a set which satisfies \(SD\)
	but not \(SA\) as follows:
	\[
	D \equiv \mathbf{J} \cup \{P^i(x) \equiv \textnormal{\mintinline{julia}{return halt(i, 0)}}\}\,.
	\]
	We have that for all \(P^i\), \(\phi_{P^i}\) is a constant function,
	hence it is computable.

	This means that all functions of this language are computable,
	and hence \(D \in SD\),
	however its interpreter would be able to solve the halting problem,
	and hence is not computable,
	meaning the language does not have the \(SA\) property.
	\item False.
	The reduction only works if \(B\) is computable,
	which then implies that \(A\) is also computable.
	To see why it doesn't work as stated,
	simply ``reduce'' to \(B\) by introducing the halting problem in it.
	By introducing this uncomputable problem, \(B\) is no longer computable,
	whereas \(A\) might be.
	\item False.
	Consider a trivial language
	which can only write functions which return \(1\).
	This language satisfies all three properties,
	yet does not form a valid model for computability.
	In order to be a good model for computability,
	one also requires the \(CA\) property.
\end{enumerate}
\end{solution}

\subsection{} % Exercise 5.
Prove \(CD \land S \implies CA\).

\nosolution

\subsection{} % Exercise 6.
Show that there exist languages that have the \(CD\) property,
but not the \(CA\) property.

\nosolution
