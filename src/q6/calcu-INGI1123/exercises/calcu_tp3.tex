\subsection{} % Exercise 1
Let \(H = \{(n, k): P_n(k) \textnormal{ terminates}\}\).
Using the reduction method, prove that the following sets are undecidable
because \(H\) is undecidable.
\begin{enumerate}
	\item \(S_1 = \{n : P_n(0) \textnormal{ terminates}\}\).
	\item \(S_2 = \{n : \phi_n(k) = k \textnormal{ for all } k\}\).
	\item \(S_3 = \{(n, m) : \phi_n = \phi_m\}\).
	\item \(S_4 = \{n : \phi_n \textnormal{ is a non total function}\}\).
	\item \(S_5 = \{(n, m) : \forall k\,, \phi_n(k) \ne \phi_m(k)\}\).
\end{enumerate}

\begin{solution}
\begin{enumerate}
\item
\begin{proof}
We start by assuming that \(S_1\) is decidable.
This means we have the following characteristic function for \(S_1\):
\[
f_{S_1}(n) =
\begin{cases}
1\,, & \textnormal{if } P_n(0) \textnormal{ halts,}\\
0\,, & \textnormal{otherwise.}
\end{cases}
\]
We now write a program, which we assume to have Gödel number \(d\):
\begin{minted}{julia}
function P_d(q)
	P_n(k)
end
\end{minted}
Here, \(n\) and \(k\) are fixed outside of the program, to any value.
We now observe that for any input, including zero,
this program executes \mintinline{julia}{P_n(k)}.
Using the characteristic function of \(S_1\),
one can then construct the following characteristic function for \(H\),
using the fact that if \(P_d(0)\) halts, then so does \(P_n(k)\):
\(\halt(n, k) = f_H(n, k) = f_{S_1}(d)\).
This function would be computable,
yet we know there is no such function for \(H\).
This leads us to conclude that the initial assumption, that is,
\(S_1\) is decidable, is false.
\end{proof}
\item
\begin{proof}
We start by assuming that \(S_2\) is decidable.
This means we have the following characteristic function for \(S_2\):
\[
f_{S_2}(n) =
\begin{cases}
1\,, & \textnormal{if } \phi_n(k) = k \textnormal{ for all } k\,,\\
0\,, & \textnormal{otherwise.}
\end{cases}
\]
We now write a program, which we assume to have Gödel number \(d\):
\begin{minted}{julia}
function P_d(q)
	P_n(k)
	return q
end
\end{minted}
Here, \(n\) and \(k\) are fixed outside of the program, to any value.
We now observe that for any input,
this program executes \mintinline{julia}{P_n(k)},
then returns the input as output.
Using the characteristic function of \(S_2\),
one can then construct the following characteristic function for \(H\),
using the fact that if \(P_d(k)\) returns \(k\), then \(P_n(k)\) halted:
\(\halt(n, k) = f_H(n, k) = f_{S_2}(d)\).
This function would be computable,
yet we know there is no such function for \(H\).
This leads us to conclude that the initial assumption, that is,
\(S_2\) is decidable, is false.
\end{proof}
\item
\begin{proof}
We start by assuming that \(S_3\) is decidable.
This means we have the following characteristic function for \(S_3\):
\[
f_{S_3}(n, m) =
\begin{cases}
1\,, & \textnormal{if } \phi_n = \phi_m\,,\\
0\,, & \textnormal{otherwise.}
\end{cases}
\]
We now write two programs,
which we assume to have Gödel numbers \(d_1\) and \(d_2\):
\begin{minted}{julia}
function P_d_1(q)
	P_n(k)
	return 1
end

function P_d_2(q)
	return 1
end
\end{minted}
Here, \(n\) and \(k\) are fixed outside of the program, to any value.
We now observe that for any input,
the second program simply returns \(1\),
while the other program executes \mintinline{julia}{P_n(k)},
then returns \(1\).
Using the characteristic function of \(S_3\),
one can then construct the following characteristic function for \(H\),
using the fact that if \(P_{d_1}(k)\) returns \(1\), then \(P_n(k)\) halted:
\(\halt(n, k) = f_H(n, k) = f_{S_3}(d_1, d_2)\).
This function would be computable,
yet we know there is no such function for \(H\).
This leads us to conclude that the initial assumption, that is,
\(S_3\) is decidable, is false.
\end{proof}
\item
\begin{proof}
We start by assuming that \(S_4\) is decidable.
This means we have the following characteristic function for \(S_4\):
\[
f_{S_4}(n) =
\begin{cases}
1\,, & \textnormal{if } \phi_n \textnormal{ is a non total function,}\\
0\,, & \textnormal{otherwise.}
\end{cases}
\]
We now write a program which we assume to have Gödel number \(d\):
\begin{minted}{julia}
function P_d(q)
	P_n(k)
end
\end{minted}
Here, \(n\) and \(k\) are fixed outside of the program, to any value.
We now observe that for any input,
this program executes \mintinline{julia}{P_n(k)}.
Using the characteristic function of \(S_4\),
one can then construct the following characteristic function for \(H\),
using the fact that if \(P_d(k)\) halts
(and hence \(\phi_d\) is not a non total function), then \(P_n(k)\) also halted:
\(\halt(n, k) = f_H(n, k) = 1 - f_{S_4}(d)\).
This function would be computable,
yet we know there is no such function for \(H\).
This leads us to conclude that the initial assumption, that is,
\(S_4\) is decidable, is false.
\end{proof}
\item
\begin{proof}
We start by assuming that \(S_5\) is decidable.
This means we have the following characteristic function for \(S_5\):
\[
f_{S_5}(n, m) =
\begin{cases}
1\,, & \textnormal{if } \phi_n \ne \phi_m\,,\\
0\,, & \textnormal{otherwise.}
\end{cases}
\]
We now write two programs,
which we assume to have Gödel numbers \(d_1\) and \(d_2\):
\begin{minted}{julia}
function P_d_1(q)
	P_n(k)
	return 1
end

function P_d_2(q)
	return 1
end
\end{minted}
Here, \(n\) and \(k\) are fixed outside of the program, to any value.
We now observe that for any input,
the second program simply returns \(1\),
while the other program executes \mintinline{julia}{P_n(k)},
then returns \(1\).
Using the characteristic function of \(S_5\),
one can then construct the following characteristic function for \(H\),
using the fact that if \(P_{d_1}(k)\) returns \(1\), then \(P_n(k)\) halted:
\(\halt(n, k) = f_H(n, k) = 1 - f_{S_5}(d_1, d_2)\).
This function would be computable,
yet we know there is no such function for \(H\).
This leads us to conclude that the initial assumption, that is,
\(S_5\) is decidable, is false.
\end{proof}
\end{enumerate}
\end{solution}

\subsection{} % Exercise 2
Look at the Hoare--Allison diagonalization proof in the lecture slides.
Why does the Hoare--Allison theorem not apply to a formalism
that allows one to compute non total functions?

\begin{solution}
	In the diagonalisation proof,
	the diagonal element \(\interpret(d, d)\)
	is changed to \(\interpret(d, d) + 1\).
	For non total functions, \(\interpret(d, d)\) could be \(\bot\).
	This means that \(\interpret(d, d) + 1\) could be \(\bot + 1 = \bot\).
	For total functions, this is not an issue,
	as \(\bot\) never occurs.
	It would make no sense for the interpreter function
	to be undefined at a certain point.
	however for non total functions, this is possible.
\end{solution}

\subsection{} % Exercise 3
Let \(L\) be a (non trivial) programming language
in which the function \(\halt(n, x) = 1\) if \(P_n\) stops on \(x\),
\(0\) otherwise, is computable.
Using the diagonalization, prove that the function \(\interpret(n, x)\)
is not computable in \(L\).

\begin{solution}
Let \(L_0, \ldots, L_k, \ldots\) be the programs of \(L\).
Assume \(\interpret(n, x)\) is computable in \(L\).
One can then put it in a table as follows:
\[
\begin{array}{c|cccc}
	& 0 & \cdots & k & \cdots \\
	\hline
	L_0 & \interpret(0, 0) & \cdots & \interpret(0, k) & \cdots\\
	\vdots & \vdots & \ddots & \cdots & \cdots\\
	L_k & \interpret(k, 0) & \vdots & \interpret(k, k) & \cdots\\
	\vdots & \vdots & \vdots & \vdots & \ddots
\end{array}
\]
We take the diagonal, name it \(L_d\), and modify it:
\(L_{d'}(n) = \interpret(n, n) + 1\).
As mentioned in Exercise~3.2, this is possible
because \(\halt\) being computable guarantees that all functions are total.
This new program also has to be in the table,
yet by construction we have a contradiction for \(L_{d'}(d')\).
This means that \(\interpret(n, x)\) is not computable in \(L\).
\end{solution}

\subsection{} % Exercise 4
True or false?
Let \(L\) be a non-trivial recursive subset of the \java{} language
computing only total functions.
\begin{enumerate}
	\item The \(\halt\) function of \(L\) is computable in \(L\).
	\item The \(\halt\) function of \(L\) is computable in \java.
	\item The interpreter of \(L\) is computable in \(L\).
	\item The interpreter of \(L\) is computable in \java.
	\item There exists a total computable function \(f\)
	not programmable in \(L\).
	\item There exists some language in which both the \(\halt\) function
	and the interpreter are computable.
\end{enumerate}

\begin{solution}
\begin{enumerate}
	\item True.
	By definition, \(L\) only computes total functions,
	hence the \(\halt\) function of \(L\) is simply the unit function,
	that always returns \(1\).
	\item True.
	Since it is computable in \(L\), and \(L\) is a subset of \java,
	it is also computable in \java.
	\item False.
	Since the \(\halt\) function of \(L\) is computable in \(L\),
	by the Hoare--Allison theorem, the interpreter of \(L\)
	cannot be computable in \(L\).
	\item True.
	Since \java{} can compute all total functions,
	it can compute the interpreter of \(L\).
	\item True.
	As mentioned above, its interpreter is not programmable in \(L\).
	\item True.
	A trivial language could have both a computable halting function
	and a computable interpreter.
	For non trivial functions however, this is not possible.
\end{enumerate}
\end{solution}
