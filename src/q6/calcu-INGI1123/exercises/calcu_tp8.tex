\subsection*{Rappel}
Soit $D$ un formalisme de la calculabilité, $D$ a les propriétés (page 5.8 du cours):
\begin{itemize}
\item $SD$: Toute fonction D-calculable est calculable
\item $CD$: Toute fonction calculable est D-calculable
\item $SA$: L'interpreteur D est calculable
\item $CA$: Soit $P$ un formalisme $SA$, il existe un compilateur tel que, étant donné $p\in P$, cela produit $d \in D$ tel quel $\varphi_d = \varphi_p$
\item $U$: L'interpreteur de D est D-calculable
\item $S$: Il existe un transformateur de programme, de façon plus formel, $\forall d \in D \exists S$ calculable tel que $D(x,y) = (S(x))(y)$
\end{itemize}

% Exercice 1
\subsection{}
\begin{itemize}
\item[(a)] $CA \Rightarrow CD$

Soit une fonction $f$ calculable 
\begin{eqnarray*}
&\Rightarrow& \exists P\ formalism\ tel\ que\ \exists p \in P\ calculant\ f\\
&\Rightarrow& \exists d\in D\ tel\ que\ \varphi_d=\varphi_p
\end{eqnarray*}

\item[(b)] $SD \land U \Rightarrow SA$

\begin{eqnarray*}
U &\Rightarrow& L'interpreteur\ de\ D\ est\ D-calculable\\
SD \land U &\Rightarrow& L'interpreteur\ de\ D\ est\ calculable\\
&\Rightarrow& SA
\end{eqnarray*}

\item[(c)] $CA \land SD \land U \Rightarrow SA \land CD \land S$

Nous allons, dans un premier temps, montrer que $CA \land SD \Rightarrow S$. Soit $d(x,y) \in D$ par $SD$ on sait que $d$ est calculable. 

Nous allons utiliser Java, qui est un language pour lequel nous connaissons les propriétés pour prouver ce que l'on veut.

\begin{itemize}
	\item Java est $CD \Rightarrow \exists P_{Java}$ calculant $d$
	\item Java est $S \Rightarrow \exists S^{Java}_{P_{Java}}$ tel que $P_{Java}(x,y) = (S_{P_{Java}}(x))(y)$
	\item De cela nous avons donc $\Rightarrow \exists$ un compilateur qui produit $S_{P_{Java}} \Rightarrow S_d \rightarrow D$ est $S$
\end{itemize}
\end{itemize}

% Exercice 2
\subsection{}

Est ce que $U \Leftrightarrow SA$? Non, et voici les propriétés à ajouter pour déduire $U$ et $SA$ relativement à l'un l'autre:
\begin{itemize}
	\item $SA \land CD \Rightarrow U$
	\item $U \land SD \Rightarrow SA$
\end{itemize}

% Exercice 3
\subsection{}
\nosolution

% Exercice 4
\subsection{}

\begin{center}
\begin{tabular}{c|c|c|c}
& (a) & (b) & (c) \\
\hline
SD & Oui & Non & Oui \\
CD & Non & Oui & Oui \\
SA & Oui & Non & Oui \\
CA & Non & Oui & Oui \\
U & Non & Oui & Oui \\
S & Oui & Oui & Oui \\
\end{tabular}
\end{center}

% Exercice 5
\subsection{}
\begin{itemize}
	\item[(a)] Vrai, $Halt(n,x) \preceq B$, il existe une réduction de $Halt$ vers $B$, $B$ est au moins aussi compliqué à résoudre.
	\item[(b)] Faux
	\item[(c)] Vrai
	\item[(d)] Vrai, ex: $D = \{Java\} \cup \{Instruction\ P_i(x)\equiv[halt(i,0)]\}$
\end{itemize}
