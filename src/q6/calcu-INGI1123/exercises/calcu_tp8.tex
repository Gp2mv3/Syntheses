\subsection{} % Exercise 1.
Write the following function as a primitive recursive function:
\(f(n) = \sum_{i=0}^n i^2\).

\begin{solution}
	We can define this function using the rules
	of primitive recursive functions as follows:
	\begin{align*}
		f(0) &= g() = 0\,,\\
		f(n+1) &= h(n, f(n))\\
		&= \mathrm{add}(h_1(n, f(n)), h_2(n, f(n)))\,,\\
		h_1(n, f(n)) &= p_2^2(n, f(n)) = f(n)\,,\\
		h_2(n, f(n)) &= \mathrm{mult}(h_{21}(n, f(n)), h_{22}(n, f(n)))\,,\\
		h_{21}(n, f(n)) &= s(p_1^2(n, f(n))) = s(n)\,,\\
		h_{22}(n, f(n)) &= s(p_1^2(n, f(n))) = s(n)\,,\\
		\mathrm{add}(m, 0) &= m\,,\\
		\mathrm{add}(m, n+1) &= s(\mathrm{add}(m, n))\,,\\
		\mathrm{mult}(m, 0) &= 0\,,\\
		\mathrm{mult}(m, n+1) &= \mathrm{add}(m, \mathrm{mult}(m, n))\,.
	\end{align*}
	The definition for \(f(0)\) uses a constant function,
	whereas the general definition uses primitive recursion with \(h\).
	We define this function itself using primitive recursion,
	using projection for \(h_1\) and primitive recursion for \(h_2\).
	Both \(h_{2i}\) functions are defined by projection,
	and using the successor function.
	We define helper functions for addition and multiplication.
\end{solution}

\subsection{} % Exercise 2.
In the following expressions, which ones are lambda expressions?
\begin{multicols}{2}
\begin{enumerate}
	\item \(x\);
	\item \(\lambda x y\);
	\item \((\lambda xx)\);
	\item \(\lambda \lambda xxy\);
	\item \(((\lambda xxy)y)\);
	\item \(\lambda x (\lambda x y)\);
	\item \(((\lambda x y ((x \lambda z x) \lambda x z))\).
\end{enumerate}
\end{multicols}

\begin{solution}
	Lambda expressions can be either
	\begin{itemize}
		\item variables;
		\item \(\lambda x B\), where \(B\) is a lambda expression;
		\item \((F A)\), where \(F\) and \(A\) are lambda expressions.
	\end{itemize}
	Using these rules we determine that
	\begin{enumerate}
		\item \(x\) is a variable, hence it is a lambda expression;
		\item \(\lambda x y)\) is of the form \(\lambda x B\),
		where \(B = y\) is a variable and thus a lambda expression.
		This makes \(\lambda x y\) a valid lambda expression.
		\item By a similar argument as before, \(\lambda x x\)
		is a lambda expression.
		However, \((\lambda x x)\) is of the form \((A)\),
		hence it is not a lambda expression.
		\item In valid lambda expressions,
		\(\lambda\) is always followed by a variable.
		\(\lambda \lambda xxy\) is not a valid lambda expression.
		\item \((\lambda x x y)\) is a valid lambda expression,
		since it is of the form \((F A)\), with
		\(F = \lambda xx\) valid since it is
		of the form \(\lambda x B\),
		and \(A = y\) valid since it is a variable.
		Finally, \(((\lambda xxy)y)\) is also valid,
		since it is of the form \((F' A')\),
		where \(F' = (\lambda x x y)\) and \(A' = y\).
		\item This expression is not valid,
		because \((\lambda x y)\) is of the form \((A)\),
		where \(A = \lambda x y\) is a valid lambda expression.
		\item This expression also contains \((\lambda x y)\),
		and is thus not valid for the same reason as above.
	\end{enumerate}
\end{solution}

\subsection{} % Exercise 3.
Lambda calculus can be used for boolean operations.
\(\mathrm{true}\) is represented by \(\lambda a \lambda b a\),
and false is represented by \(\lambda a \lambda b b\).
The \(\mathrm{And}(X, Y)\) operation is represented by
\(((XY)\lambda a \lambda b b)\).
The \(\mathrm{Not}(X)\) operation is represented by
\(\lambda a \lambda b ((X b) a)\).
\begin{enumerate}
	\item Write the lambda expressions
	for \(\mathrm{And}(\mathrm{true}, \mathrm{true})\),
	\(\mathrm{And}(\mathrm{true}, \mathrm{false})\),
	\(\mathrm{Not}(\mathrm{true})\),
	and reduce them.
	\item Based on your experience, try to express the lambda function
	corresponding to \(\mathrm{Or}\).
\end{enumerate}

\begin{solution}
\begin{enumerate}
	\item We start by replacing the values in the function definition:
	\begin{align*}
		\mathrm{And}(\mathrm{true}, \mathrm{true}) &=
		((\lambda a \lambda b a \lambda a \lambda b a) \lambda a \lambda b b)\,.
		\intertext{Now, we apply the reduction rule,
		until reaching an irreducible expression.
		The reduction rule takes expressions of the form
		\((\lambda x D A)\) and replaces them by \(D[x/A]\),
		which is \(D\) with all bound occurrences of \(x\)
		replaced by \(A\).
		One needs to rename variables
		when either \(D\) contains bound occurrences of \(x\)
		or \(A\) contains unbound variables which are bound in \(D\).}
		&= (\lambda c \lambda a \lambda b a \lambda a \lambda b b)\,.
		\intertext{In this case, we had to rename \(b\),
		since \(b\) is already bound in \(D\).
		We now observe that \(c\) has no bound occurrences,
		meaning we just take \(\lambda c D A\) reduces to \(D\).}
		&= \lambda a \lambda b a\,.
	\end{align*}
	This expression is represents true, as expected.

	Next, we do the same
	for \(\mathrm{And}(\mathrm{true}, \mathrm{false})\):
	\begin{align*}
		\mathrm{And}(\mathrm{true}, \mathrm{false}) &= ((\lambda a \lambda b a \lambda a \lambda b b)\lambda a \lambda b b)\\
		&= (\lambda c \lambda a \lambda b b \lambda a \lambda b b)\\
		&= \lambda a \lambda b b\,.
	\end{align*}
	Again, this gives the expected result: false.

	Finally, we do this for \(\mathrm{Not}(\mathrm{true})\):
	\begin{align*}
		\mathrm{Not}(\mathrm{true}) &= \lambda a\lambda b((\lambda a \lambda b a b) a)\\
		&= \lambda a \lambda b (\lambda c b a)\\
		&= \lambda a \lambda b b\,.
	\end{align*}
	This expression represents false, as expected.
	\item Let us start by giving some general observations;
	let \(X, Y\) be boolean values:
	\begin{itemize}
		\item expressions of the form \(((\lambda a \lambda b b X) Y)\)
		always return \(Y\);
		\item expressions of the form \(((\lambda a \lambda b a X) Y)\)
		always return \(X\).
	\end{itemize}
	The true and false functions are thus projection functions.

	Next, we reason about what \(\mathrm{Or}(X, Y)\) does:
	\begin{itemize}
		\item if \(X\) is true, return true;
		\item if \(X\) is false, return \(Y\).
	\end{itemize}

	Combinig both observations, we can write
	\[
	\mathrm{Or}(X, Y) = ((X \lambda a \lambda b a) Y)\,.
	\]
	To see why this works, consider what happens when \(X\) is true:
	the expression is then of the second form above,
	meaning that it returns \(\lambda a \lambda b a\), which is true.
	If \(X\) is false, the expression is of the first form above,
	meaning it returns \(Y\).
	This is what we wanted after reasoning
	about what the disjunction should do.
\end{enumerate}
\end{solution}

\subsection{} % Exercise 4.
True of false?
\begin{enumerate}
	\item The halt function for primitive recursive functions
	is computable.
	\item The interpret function for primitive recursive functions
	can be represented by a primitive recursive function.
	\item Reducing a lambda expression always leads
	to a reduced lambda expression.
	\item If two reduction sequences of a lambda expression
	lead to reduced forms, they are equivalent.
	\item The reduction algorithm of lambda expressions
	can be represented as a lambda expression.
\end{enumerate}

\begin{solution}
	\begin{enumerate}
		\item True.
		Primitive recursive functions are all total,
		hence their halt function is simply the constant unit function,
		which is trivially computable.
		\item False.
		Since primitive recursive functions
		only compute total functions,
		they cannot compute all total functions,
		with interpret being one of them.
		\item False.
		Consider the expression
		\((\lambda x (xx) \lambda x (xx))\).
		Upon reduction, this expression yields itself.
		\item True.
		This result is known as the Church--Rosser theorem.
		\item True.
		Since the lambda calculus is a complete model of computability,
		it is possible to express the reduction algorithm in it.
	\end{enumerate}
\end{solution}
