\documentclass[fr]{../../../../../../eplexam}

\hypertitle{Calculabilité}{6}{INGI}{1123}{2018}{Juin}{Mineure}
{Béatrice Desclée \and François Duchêne}
{Yves Deville}

\section{Démontrer qu'un programme n'est pas récursif}
Soit $A = \{i \mid P_i \quad \textnormal{avec deux outputs distincts et au moins deux inputs distincts} \}$
\begin{enumerate}
	\item Énoncer théorème de Rice
	\item Démontrer pourquoi la non-récursivité de \(A\) est une conséquence du théorème de Rice.
\end{enumerate}

\nosolution

\section{Questions vrai/faux avec justification}
\begin{enumerate}
	\item L'ensemble des fonctions de $\N \rightarrow \{0,1\}$ est non énumérable
	\item Un sous-ensemble infini d'un ensemble récursivement énumérable est récursivement énumérable 
	\item \ldots
	\item \ldots
	\item Machine de Turing non déterministe plus puissante que déterministe
	\item Si $SAT <_p A$ et si $A \in NP$, alors $A$ est-il NP-complet ?
\end{enumerate}

\nosolution

\section{Définir la réduction algorithmique}

\nosolution

\section{Définir la réduction polynomiale}

\nosolution

\section{Démontrer $A \leq_p B \implies A \leq_a B$}

\nosolution

\end{document}
