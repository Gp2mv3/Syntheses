\documentclass[fr]{../../../../../../eplexam}

\hypertitle{}{6}{INGI}{1123}{2018}{Juin}{Majeure}
{Béatrice Desclée \and François Duchêne}
{Yves Deville}

\section{Énoncer et démontrer le théorème de Rice}
\nosolution
\section{Questions vrai/faux avec justification}
\begin{enumerate}
	\item ensemble infini de chaînes de caractères finies est énumérable
	\item sous-ensemble infini d'un ensemble récursif est récursif
	\item toute fonction à domaine fini est calculable
	\item \(A\) algorithmiquement réductible à son complément
	\item un des S-m-n
	\item une machine de Turing avec oracle peut calculer halt(n, x)
	\item NP inclus dans EXPTIME
	\item \ldots
\end{enumerate}
\nosolution
\section{Définir CA, CD, la relation entre les deux et démontrer cette relation}
\nosolution
\section{Définir :}
\begin{enumerate}
	\item La réduction polynomiale
	\item La classe NP
	\item La classe NP-Complet
\end{enumerate}
\nosolution
\end{document}
