\documentclass[fr]{../../../../../../eplexam}

\hypertitle{Calculabilité}{6}{INGI}{1123}{2018}{Juin}{Majeure}
{Béatrice Desclée \and François Duchêne}
{Yves Deville}

\section{Énoncer et démontrer le théorème de Rice}

\nosolution

\section{Questions vrai/faux avec justification}
\begin{enumerate}
	\item L'ensemble infini de chaînes de caractères finies est énumérable
	\item Un sous-ensemble infini d'un ensemble récursif est récursif
	\item Toute fonction à domaine fini est calculable
	\item \(A\) algorithmiquement réductible à son complément
	\item Un des S-m-n
	\item Une machine de Turing avec oracle peut calculer $\textrm{halt}(n, x)$\footnote{Question fondamentale, qui n'a pas été conservé dans cette retranscription d'examen: l'ensemble pour lequel on a un oracle est-il énumérable?}
	\item NP inclus dans EXPTIME
	\item \ldots
\end{enumerate}

\nosolution

\section{Définir CA, CD, la relation entre les deux et démontrer cette relation}

\nosolution

\section{Définir :}
\begin{enumerate}
	\item La réduction polynomiale
	\item La classe NP
	\item La classe NP-Complet
\end{enumerate}

\nosolution

\end{document}
