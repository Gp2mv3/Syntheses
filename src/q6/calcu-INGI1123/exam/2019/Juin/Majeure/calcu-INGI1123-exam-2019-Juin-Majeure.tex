\documentclass[fr]{../../../../../../eplexam}

\hypertitle{Calculabilité}{6}{INGI}{1123}{2019}{Juin}{Majeure}
{Antoine Grollinger \and Donatien Schmitz \and François De Keersmaeker \and François Duchêne}
{Yves Deville}

\section{Machine de Turing}
\textit{4 pts}
Soit la machine de Turing suivante avec $\Sigma = \{0,1\}$
\begin{tabular}{|l|l||l|l|r|}
	\hline 
	État & Symbole & État & Symbole & Action \\ 
	\hline 
	S0 & 0 & S1 & 0 & D \\ 
	\hline 
	S0 & 1 & S1 & 1 & D \\ 
	\hline 
	S0 & B & Stop & B & G \\ 
	\hline 
	S1 & 0 & S0 & 1 & D \\ 
	\hline 
	S1 & 1 & S0 & 0 & D \\ 
	\hline 
	S1 & B & Stop & B & G \\ 
	\hline 
\end{tabular} 
\begin{description}
	\item[Machine déterministe ?] Vrai/Faux
	\item[Machine a 2 etats ?] Vrai/Faux
	\item[Sortie pour entrée 110011 ?]
\end{description}

\nosolution

\section{Questions vrai/faux avec justification}
\textit{7 pts}
\begin{enumerate}
	\item L’ensemble des fonctions de $\N$ vers $\{0,1\}$ est non énumérable
	\item Il n’existe pas de langage tel que \emph{halt} est calculable dans ce langage
	\item Un sous ensemble infini d’un ensemble récursivement énumérable est récursivement énumérable 
	\item Le théorème du Point Fixe est une conséquence du théorème de Rice
	\item Une grammaire hors contexte ne permet que de construire des langages récursifs
	\item Un formalisme \(D\) qui a un interpréteur calculable possède la propriété U
	\item Si $A \in NP$ et $A \le_{p} SAT$ alors $A$ est NP-Complet.
\end{enumerate}

\nosolution

\section{Énoncer précisément le théorème de Rice, donner sa signification et ses conséquences}
\textit{4 pts}

\nosolution

\section[Réduction polynomiale et SAT appartenant à P]{Définir la réduction polynomiale et prouver que si $SAT \in P$ alors $P=NP$}
\textit{4 pts}

\nosolution

\end{document}
