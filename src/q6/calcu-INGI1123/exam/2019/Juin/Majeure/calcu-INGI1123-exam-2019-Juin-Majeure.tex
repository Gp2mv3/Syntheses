\documentclass[fr]{../../../../../../eplexam}

\hypertitle{Calculabilité}{6}{INGI}{1123}{2019}{Juin}{Majeure}
{Antoine Grollinger \and Donatien Schmitz \and François De Keersmaeker \and François Duchêne \and Lucas Maris}
{Yves Deville}

\section{Machine de Turing}
\textit{4 pts}
Soit la machine de Turing suivante avec $\Sigma = \{0,1\}$

\begin{tabular}{|l|l||l|l|r|}
	\hline 
	État & Symbole & État & Symbole & Action \\ 
	\hline 
	S0 & 0 & S1 & 0 & D \\ 
	\hline 
	S0 & 1 & S1 & 1 & D \\ 
	\hline 
	S0 & B & Stop & B & G \\ 
	\hline 
	S1 & 0 & S0 & 1 & D \\ 
	\hline 
	S1 & 1 & S0 & 0 & D \\ 
	\hline 
	S1 & B & Stop & B & G \\ 
	\hline 
\end{tabular} 


\begin{description}
	\item[Machine déterministe ?] Vrai/Faux
	\item[Machine a 2 etats ?] Vrai/Faux
	\item[Sortie pour entrée 110011 ?]
\end{description}

\begin{solution}
    \begin{enumerate}
    	\item Vrai. La MT est déterministe car il n'y jamais qu'une seule ligne pour chaque paire \texttt{<état, symbole>}.
    	\item Faux. L'ensemble des états de la machine est $S=\{S0, S1, Stop\}$, elle a donc 3 états.
    	\item 100110. La MT alterne entre laisser le bit inchangé, et l'inverser (elle garde le 1er, inverse le 2e, garde le 3e, inverse le 4e, garde le 5e, inverse le 6e, puis s'arrête).
    \end{enumerate}
\end{solution}

\section{Questions vrai/faux avec justification}
\textit{7 pts}
\begin{enumerate}
	\item L’ensemble des fonctions de $\N$ vers $\{0,1\}$ est non énumérable
	\item Il n’existe pas de langage tel que \emph{halt} est calculable dans ce langage
	\item Un sous ensemble infini d’un ensemble récursivement énumérable est récursivement énumérable 
	\item Le théorème du Point Fixe est une conséquence du théorème de Rice
	\item Une grammaire hors contexte ne permet que de construire des langages récursifs
	\item Un formalisme \(D\) qui a un interpréteur calculable possède la propriété U
	\item Si $A \in NP$ et $A \le_{p} SAT$ alors $A$ est NP-Complet.
\end{enumerate}

\begin{solution}
    \begin{enumerate}
    	\item \textbf{Vrai}. On peut le prouver par diagonalisation, c'est-à-dire qu'on va supposer que cet ensemble est énumérable pour aboutir à une contradiction. On considère un tableau dont chacune des lignes correspond à une de ces fonctions $f_n$, et chaque colonne à chaque input $x$ possible pour chacune de ces fonctions. Vu que les fonctions vont $\N$ vers $\{0,1\}$, ce tableau ne contient que des 0 et des 1. Si on considère la diagonale de ce tableau, on peut définir une fonction \emph{diag\_mod(n)} = 1 si $f_n(x)=0$, et \emph{diag\_mod(n)} = 0 si $f_n(x)=1$. Cette fonction est une fonction de $\N$ vers $\{0,1\}$, et par notre hypothèse d'énumérabilité de l'ensemble des fonctions de $\N$ vers $\{0,1\}$, elle se trouve donc forcément à une certaine ligne d du tableau. Au croisement de cette ligne d avec la diagonale du tableau, on a une case contenant à la fois \emph{diag\_mod(d)} et $f_d(d)$, or ils sont différents par définition. L'ensemble des fonctions de $\N$ vers $\{0,1\}$ ne peut donc pas être énumérable.
    	\item \textbf{Faux}. Dans BLOOP/MiniJava, \emph{halt(n,x)} est calculable (et vaut toujours 1).
    	\item \textbf{Faux}. $\N$ est récursivement énumérable, et $\overline{K}$ en est un sous-ensemble infini, or si $\overline{K}$ était récursivement énumérable, K serait récursif, ce qui n'est pas vrai.
    	\item \textbf{Faux}. C'est l'inverse.
    	\item \nosubsolution
    	\item \textbf{Faux}. La propriété U désigne le fait d'avoir un interpréteur qui soit D-calculable.
    	\item \textbf{Faux}. Il faut $A \in NP$ et $SAT \le_{p} A$ pour que $A$ soit NP-Complet.
    \end{enumerate}
\end{solution}

\section{Énoncer précisément le théorème de Rice, donner sa signification et ses conséquences.}
\textit{4 pts}

\begin{solution}
L'énoncé du théorème de Rice est le suivant :
$$ \textrm{Si A est récursif} \land A \ne \emptyset \land A \ne \N \textrm{, alors } \exists i \in A, \exists j \in \overline{A} \textrm{ tq. } \phi_i=\phi_j$$
Sa contraposée dit :
$$ \textrm{Si } \forall i \in A, \forall j \in \overline{A} \textrm{ : } \phi_i \ne \phi_j \textrm{, alors A est non récursif} \lor A = \emptyset \lor A = \N$$
Ce théorème nous dit que si une propriété de programmes (vérifiée par certains programmes mais pas tous) est décidable, alors il existe deux programmes équivalents (calculant la même fonction), dont un vérifie la propriété et l’autre pas.\\
\\
Il n'est donc jamais possible d'analyser la fonction calculée par un autre programme. Pour toute paire de programmes ne calculant pas la même fonction telle que l’un a une certaine propriété, et l’autre pas, A est non récursif (on ne peut pas le délimiter).
\end{solution}

\section[Réduction polynomiale et SAT appartenant à P]{Définir la réduction polynomiale et prouver que si $SAT \in P$ alors $P=NP$.}
\textit{4 pts}

\begin{solution}
Un ensemble A est polynomialement réductible à un ensemble B ($A \le_p B$) s’il existe une fonction totale calculable f de complexité temporelle polynomiale telle que $a \in A \Rightarrow f(a) \in B$. \\
\\
On sait que $P \subseteq NP$, mais pour démontrer $NP \subseteq P$, il faudrait prendre un élément E représentatif de NP (le plus difficile) et démontrer que $E \in P$. Pour que cet élément E soit représentatif, il doit être NP-complet. Dans notre cas, SAT est effectivement NP-complet, donc si on parvient à prouver qu'il appartient à P, on prouve que tous les problèmes appartenant à NP appartiennent aussi à P, et donc que P=NP.
\end{solution}

\end{document}
