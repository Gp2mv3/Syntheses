\documentclass[fr]{../../../../../../eplexam}

\hypertitle{calcu}{6}{INGI}{1123}{2020}{Septembre}{All}
{Stanislas Gorremans}
{Yves Deville}

\subsection*{Note}
Vu la situation exceptionnelle de cette année (COVID-19), les sections suivantes ne font pas partie de la matière d’examen :
Attention que ces numéros de slides correspondent aux numéros de section des slides.

\begin{itemize}
\item 3.12 : autres problèmes non calculables
\item 3.14 : Nombres calculables
\item 4.4 :  Automates à pile
\item 4.5 : Grammaires
\item 4.8 : Lambda calcul
\item 4.9 : Machine de Post
\item 5.5 : Au-delà de la calculabilité
\item 6.2 :  Calcul de complexité
\item 6.3 : Exemple
\item 6.4 : Notation grand )
\item 6.5 : Exemple Tout de Hanoi
\item 7.6 : NP-complétude (2)
\item 7.8 : Démonstration théorème de Cook
\end{itemize}

\part{Questions à choix multiple (sur 6 points)}
Pour chacune des questions, cocher les affirmations qui sont vraies.

\section{Question 1 (sur 1 point)}
Ensemble énumérable, récursif et récursivement énumérable.

\renewcommand\labelitemi{$\square$}
\begin{itemize}
    \item Il existe des ensembles récursifs qui ne sont pas récursivement énumérables
    \item Tout ensemble de paires d’entiers est récursif
    \item L’ensemble des sous-ensembles récursivement énumérables de N (entiers naturels) est énumérable
    \item Un sous-ensemble d’un ensemble récursif est récursif
    \item Il existe des ensembles récursifs qui ne sont pas énumérables
\end{itemize}

\nosolution


\section{Question 2 (sur 1 point)}
Calculabilité

\begin{itemize}
    \item Si le domaine d’une fonction est fini, alors cette fonction est calculable
    \item Si le domaine d’une fonction est infini, alors cette fonction est non calculable
    \item Une fonction constante est toujours calculable
    \item Un programme Java représente une infinité de fonctions
    \item Il existe une fonction partielle calculable telle qu’aucune fonction totale calculable n’est une extension de cette fonction partielle
\end{itemize}

\nosolution

\section{Question 3 (sur 1 point)}
Calculabilité \\
Soit la fonction  $f(i) = 1$ si $\phi_i(i)\ne \perp$, 0 sinon


\begin{itemize}
    \item La fonction f est calculable
    \item Le domaine de f est récursif
    \item Le domaine de f est NP-complet
    \item L’image de f est un ensemble récursif
    \item $\phi_i$ dénote la fonction numéro i (selon l’énumération choisie pour les fonctions)
\end{itemize}

\nosolution

\section{Question 4 (sur 1 point)}
Théorèmes fondamentaux

\begin{itemize}
    \item L’ensemble des programmes Java calculant une fonction f telle que $f(10)=10$ est un ensemble récursivement énumérable.
    \item Il existe un langage de programmation (non trivial) dans lequel on peut programmer le fonction halt ainsi que l’interpréteur de ce langage
    \item Le théorème du point fixe permet de démontrer que la fonction halt est non calculable
    \item Si f est un transformateur de programmes (f fonction totale calculable), alors il existe deux programmes P1 et P2 tels que f(P1)=P2 ainsi que P1 et P2 calculent la même fonction
    \item Il existe un langage de programmation qui calculent uniquement toutes les fonctions totales calculables en Java
\end{itemize}
\nosolution

\section{Question 5 (sur 1 point)}
Analyse
\begin{itemize}
    \item Si une fonction est calculable par une Machine de Turing, alors cette fonction est calculable par un programme Java
    \item  Soit A un ensemble récursivement énumérable mais non récursif. Une Machine de Turing avec A comme oracle est plus puissante qu’une machine de Turing sans oracle.
    \item Soit D un nouveau modèle de calculabilité. Si toute fonction calculable est calculable dans D et si toute fonction calculable dans D est effectivement calculable, alors D est un modèle complet de la calculabilité
    \item Un formalisme D de calculabilité possède la propriété U (description universelle) lorsque l’interpréteur de D est calculable
    \item Une Machine de Turing dont le ruban serait fini à gauche est un modèle complet de la calculabilité.
\end{itemize}
\nosolution

\section{Question 6 (sur 1 point)}
Classes de complexité
\begin{itemize}
    \item Un problème intrinsèquement complexe est dans EXPTIME
    \item NP est un sous-ensemble de EXPTIME
    \item Si A est dans NDTIME(f) alors A est aussi dans DTIME(f)
    \item Si A est dans NDTIME($n^3$) alors A est dans NP
    \item S’ il existe un algorithme Java de complexité temporelle O($n^3$) décidant l’ensemble A, alors il existe une machine de Turing de complexité temporelle O($n^3$) décidant l’ensemble A
\end{itemize}

\part{Questions ouvertes (sur 8 points)}
Répondre en format texte aux différentes questions.  Si vous souhaitez utiliser des notations mathématiques, vous êtes libres d'utiliser les conventions de notations que vous voulez pour autant que cela soit clair.

\section{Question 1 (sur 1 point)}
Donner un exemple de fonction totale non calculable (autre que le problème de l’arrêt ou la détection d’un virus informatique)
\nosolution

\section{Question 2 (sur 3 points)}
Enoncer les propriétés CD et CA des formalismes de calculabilité.  Enoncer et démontrer quelle relation existe entre ces deux propriétés.  
\nosolution

\section{Question 3 (sur 1 point)}
Définir la réduction polynomiale
\nosolution

\section{Question 4 (sur 1 point)}
Qu’est-ce qu’un problème NP-complet ?  
\nosolution


\section{Question 5 (sur 2 points)}
Montrer que si $SAT \in P$, alors $P=NP$.  
\nosolution

\end{document}
