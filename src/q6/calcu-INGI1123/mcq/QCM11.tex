\begin{mcqs}
  \mcq{Si la complexité temporelle d'un algorithme est $O(n^{2})$, alors elle est aussi $O(n^{3})$}{1}
  {}
  \mcq{Si la complexité spatiale d'un algorithme est $O(n^{3})$, alors elle ne peut pas être $O(n^{2})$}{0}
  {$O(n^{3})$ est une borne supérieure mais ça peut être moins!}
  \mcq{Un algorithme de complexité $O(n^{2})$ est toujours plus rapide qu'un algorithme de complexité $O(n^{3})$}{0}
  {}
  \mcq{Un problème qui peut être résolu par un algorithme (de complexité) polynomial est pratiquement faisable.}{1}
  {Pratiquement faisable = Polynomial}
  \mcq{Un problème qui peut être résolu par un algorithme (de complexité) exponentiel est pratiquement infaisable}{0}
  {Il peut très bien être réalisable à la fois par un algorithme exponentiel ET un algorithme polynomial.}
  \mcq{Si un problème est intrinsèquement complexe en MT alors il est aussi intrinsèquement complexe pour le langage Java.}{1}
  {Le choix du modèle de calculabilité impacte la complexité temporelle. Néanmoins, la différence de complexité est tout au plus polynomiale entre tous les modèles de calcul déterministes. Or, la composition de fonctions polynomiales reste polynomiale!}
  \mcq{Un ensemble est a-réductible (algorithmiquement réductible) à son complément.}{1}
  {Dire que A est a-réductible à B signifie que l'algorithme pouvant décider B nous permet de décider A. Ici, décider le complément permet bien évidemment de décider l'ensemble initial, il suffit d'inverser la réponse.}
  \mcq{Tout ensemble récursivement énumérable est a-réductible à HALT.}{0}
  {Halt est le ``plus difficile''}
  \mcq{Un ensemble est $f$-réductible (fonctionnellement réductible) à son complément.}{0}
  {La plupart du temps, il n'est pas possible de trouver une fonction qui transforme une instance d'un ensemble en une instance de son complément. Par exemple ensemble PAIR est $f$-réductible à IMPAIR. Il suffit de rajouter 1. Mais ce n'est pas nécessairement toujours possible...}
  \mcq{Si $A$ peut être décidé par un algorithme polynomial et si $B$ est $f$-réductible à $A$, alors $B$ peut être décidé par un algorithme polynomial.}{0}
  {Parce que ce que la $f$-réductibilité veut dire : on peut décider $B$ en donnant f(x) (on veut décider si x est dans B) à A. Mais f(x) peut être exponentiel.}
\end{mcqs}
