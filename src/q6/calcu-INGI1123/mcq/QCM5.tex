\begin{mcqs}
   % Si $A$ est un sous-ensemble (strict et non vide) récursif de programmes Java, alors toute fonction calculable peut être calculée par un programme Java qui n'est pas dans $A$. & FAUX \\
    %\textit{Justification:} Il existe une fonction et pas \og{} toute \fg{} & \\
  \mcq{L'ensemble des programmes Java calculant une fonction $f$ telle que $f(10)=10$ est un ensemble récursif.}{0}
  {Si ça concerne le comportement de la fonction, Rice nous dit que c'est impossible}
  \mcq{L'ensemble des programmes Java calculant une fonction $f$ telle que $f(10) = 10$ est un ensemble récursivement énumérable.}{1}
  {C'est vrai mais ce n'est pas un théorème, on lance le programme avec input 10.
  Si ça termine et que le résultat est 10, on renvoit 1. Ça n'a aucun lien avec Rice.}
  \mcq{Toute propriété relative aux programmes est non calculable.}{0}
  {Certaines propriétés concernant le programme (comme sa longueur ou sa syntaxe) sont calculables.}
  \mcq{Si $A$ est un sous-ensemble (strict et non-vide) récursif de programmes Java, alors toute fonction calculée par un programme de $A$ est aussi calculée par un programme du complément de $A$.}{0}
  {Tous les programmes de + de 10 lignes n'ont pas un équivalent de moins de 10 lignes. Il y a une différence entre $\forall$ et $\exists$, il en existe un mais ce n'est pas pour tous.}
  \mcq{La propriété S-m-n affirme que tout numéro de programme calculable peut être transformé en un numéro équivalent, mais avec moins de paramètres.}{0}
  {Il est pas équivalent, il marche que pour $m$ paramètre avec des valeurs précises.}
  \mcq{Les propriétés S-m-n et S sont équivalentes}{0}
  {Non, S-m-n est plus fort.}
  \mcq{Tous les langages de programmation satisfont la propriété S-m-n}{1}
  {Il suffit de spécialiser des paramètres,
  on sait faire ça dans tous les languages.}
\end{mcqs}
