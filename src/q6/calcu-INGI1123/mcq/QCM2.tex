\begin{mcqs}
  \mcq{L'ensemble des rationnels est énumérable}{1}
  {On peut les lister par diagonale}
  \mcq{Un sous-ensemble infini d'un ensemble énumérable est énumérable}{1}
  {}
  \mcq{Tout ensemble infini de chaînes finies de caractères est énumérable}{1}
  {}
  \mcq{Tout ensemble infini de chaînes infinies de caractères est énumérable}{0}
  {Elements entre $[0,1]$ n'est pas énumérable}
  \mcq{L'ensemble des fonctions de $\mathbb{N}$ vers $\{0,1\}$ est non énumérable}{0}
  {Pour chaque fonction de $\{0,1\}$ vers $\mathbb{N}$ correspond à une fonction. Comme les S.E de $\{0,1\}$ sont non-énumérables, c'est non-énumérable}
  \mcq{L'ensemble des fonctions de $\{0,1\}$ vers $\mathbb{N}$ est énumérable}{1}
  {}
  \mcq{Tout langage (alphabet fini) est énumérable}{0}
  {Comme les chaines peuven-être infinies. Par contre en informatique les chaines sont finies donc VRAI dans ce cas.}
  \mcq{Toute fonction bijective est injective}{1}
  {Injectif: Ensemble d'arrivée n'est pas la cible de deux éléments de l'ensemble de départ; Bijectif: tous élément est cible de 1 et 1 seul.}
  \mcq{Une fonction dont la table est infinie ne peut être décrite de manière finie}{0}
  {$f:\mathbb{R}\rightarrow\mathbb{R} : f(x)=x^{2}$}
  \mcq{Toute fonction totale est surjective}{0}
  {}
  \mcq{Toute extension d'une fonction surjective est surjective}{1}
  {}
  \mcq{Tout ensemble non-énumérable peut être mis en bijection avec l'ensemble des réels}{0}
  {}
  \mcq{L'ensemble des fonctions de $\mathbb{N}$ dans $\mathbb{N}$ est calculable}{0}
  {}
  \mcq{L'ensemble des programmes Java est calculable}{1}
  {}
  \mcq{L'énumérabilité des programmes Java et la non énumérabilité des fonctions de $\mathbb{N}$ vers $\mathbb{N}$ est une preuve de l'existence de fonctions non calculables}{1}
  {On ne sait pas faire assez de programmes pour calculer toutes les fonctions.}
  \mcq{L'ensemble des fonctions calculables est énumérable}{1}
  {Pour chaque fonction calculable, il existe un programme qui la calcule.
  Un programme ne calcule pas 2 fonctions différentes mais 2 programmes différents peuvent calculer une même fonction.
  Il n'y a donc pas plus de fonctions calculables que de programmes.
  Comme le nombre de programme est énumérable,
  il y a donc un nombre énumérable de fonction calculables.}
  \mcq{L'ensemble des fonctions non-calculables est énumérable}{0}
  {L'union de deux ensembles énumérables est énumérables.
  Seulement, l'union de l'ensemble des fonctions calculables et non-calculables
  est l'ensemble des fonctions de $\mathbb{N}$ dans $\mathbb{N}$ qui est non-énumérable.
  Comme l'ensemble des fonctions calculables est énumérable,
  l'ensemble des fonctions non-calculables ne peut pas être énumérable.}
\end{mcqs}
