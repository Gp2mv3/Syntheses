\documentclass[fr]{../../../../../../eplexam}

\hypertitle{Processus Stochastiques
}{6}{INMA}{1731}{2018}{Juin}{Mineure}
{Victor Anckaert}
{Luc Vandendorpe}

\section{}

On a des observations $y$, on essaye d'estimer $I$.
$$y(n)=\sum_{l=0}^{N-1} h(l) I(n-l)+w(n)$$
$$\hat{I}(n)= \sum_{l=-K}^K c(l) y(n-l) + \sum_{l=0}^{L-1} [\delta(l) - d(l)] I(n-l)$$
$w$ et $I$ non-corrélés, $w$ bruit blanc, $d(l)$ monique
\begin{enumerate}
    \item Donnez les équations que les coefficients $c(l)$ et $d(l)$ doivent respecter pour que le filtre soit optimal au sens de Wiener.
    \item Donnez l'équation  que les coefficients $c(z)$ et $d(z)$ doivent respecter pour que le filtre soit optimal pour un estimateur LMMSE.
    \item Obtenez $C(z)$ en fonction de $D(z)$.
    \item Ecrivez $E(z) = I(z) - \widehat I(z)$ uniquement en fonction de $D(z)$.
\end{enumerate}

\nosolution

\section{}

Soit un processus aléatoire AR d'ordre N, décrit par:
$$X(n) = - \sum_{k=1}^N a(k) X(n-k) + W(n)$$
$W(n)$ est un bruit blanc gaussien de moyenne nulle et de variance $\sigma_W^2$ indépendant de $X(n-j)$ pour $j>0$. On étudie le prédicteur d'horizon 1, optimal selon le critère de Wiener, donné par:
$$\hat{X}(n) = - \sum_{k=1}^N c(k) X(n-k)$$
$c(k)$ sont les coefficients prédiction.

\begin{itemize}
    \item Fournissez les équations qui doivent être respectées par $c(k)$.
    \item Est-ce que la solution correspondant aux $c(k)$ optimaux vous semble logique? Expliquez.
    \item A quel prédicteur qui correspond à un autre critère est-ce que cette solution correspond? Commentez.
\end{itemize}

\nosolution

\section{}

\begin{itemize}
    \item Donnez une introduction précise et concise (quelques ligne max) du concept de LMMSE.
    \item Donnez sa formule.
    \item Soit $X = [X_1\; X_2\; X_3]^T$, un vecteur de valeurs aléatoires $\mathbb{R}^3$ de moyenne nulle et de matrice de covariance:
    $$C_{X,X}= \left[\begin{matrix} a & d & e\\ d & b & f\\ e & f & c\end{matrix}\right]$$
    L'estimateur est $X_1$ et vous avez le choix entre mesurer $X_2$ ou $X_3$ pour estimer $X_1$. Proposez une stratégie pour faire ce choix. La stratégie doit être exprimée par une condition algébrique sur $a,b,c,d,e,f$ et doit être justifiée en détail.
\end{itemize}

\nosolution

\end{document}
