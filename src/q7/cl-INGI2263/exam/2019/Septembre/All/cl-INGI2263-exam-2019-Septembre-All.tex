\documentclass[en]{../../../../../../eplexam}

\hypertitle{}{7}{INGI}{2263}{2019}{Septembre}{All}
{Nicolas Vanvyve}
{Pierre Dupont and Cedrik Fairon}

\section{We are interested inthe BLEU metric used to assess machine translation systems.}
\subsection{An information retrival system can be summarized as a system receiving a query, for instance in the form of keyword or natural language sentence, and producing an answer, for instance a (ranked) list of relevant documents. A typical evaluation of such system is based on the notions of precision and recall. At first glance, one could try to evaluate a machine translation system in the same way by considering that each query is a sentence in source language and the expected answer is its translation in the target language. This approach is however inappropriate to evaluate a machine translation system. Explain why}

\nosolution

\subsection{The BLEU metric is a common alternative to evaluate mahine translation system. It relies on the notion of modified N-gram precision. Define this notion by a mathematical formula and explain in a few word the ingredients of this formula. In which sense is this presision metric modified? Why is this modification needed?}

\nosolution

\subsection{Give the compllete formula of the BLEU evaluation metric. Explain in a few words the role of each quantity ussed in this formula and why they are relevant as part of a quality metric for machine translation. In others words, argue how and why the BLEU metric has been defined in this specific way.}

\nosolution

\section{We are interested in the Brill POS tagger}
\subsection{Describe precisly what needs to be a prioro defined in the Brill tagger as oppased to what in automatically learned from a corpus}

\nosolution

\subsection{Discuss the benefits and limitations of the Btill POS tagger}

\nosolution

\section{The following senstences illustrate different types of restrictions on the paradigmatic and syntagmatic axis.}
\begin{enumerate}
	\item Action speak louder than qords.
	\item She'll be burning her bridges of she goes to work for their competitor.
	\item The term blue collar was first used in referance to trade jobs in 1924
\end{enumerate}

\subsection{For each sentence, explain the paradigmatic and syntagmatic restrictions that you can identify.}

\nosolution

\subsection{Explain why it is useful to take these restriction into account for building NLP applications.}

\nosolution

\subsection{In the first sentence, indentify all the smallest meaningful uits and name their type (being as presise as possible).}

\nosolution

\section{An algorithm receives string A as input and produces string B as output.}
\begin{itemize}
	\item[A] "Python is very useful for processing textual data."
	\item[B] "Python is veri use for process textual data."
\end{itemize}

\subsection{Which algorithm is used here? Name the underlying process and provide a concise definition.}

\nosolution

\subsection{How did the algorithm transform the following elements?}
\begin{enumerate}
	\item very $\rightarrow$ veri
	\item useful $\rightarrow$ use
	\item processing $\rightarrow$ process
\end{enumerate}

\nosolution

\subsection{What other task can be used as alternative to this algoritmic approach? What would this alternative yield as output B? Conpare the two outcomes and briefly explain the commmonalities and/or différences and advantages and/or disadvantages to these two strategies.}

\nosolution

\end{document}
