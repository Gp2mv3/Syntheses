

\subsection{Scheduling CP}

\begin{center}
    \texttt{cumulative($[s_1,..., s_n], [d_1, ..., d_n], [e_1, ...,
    e_n], [c_1, ..., c_n], C$)}
\end{center}

\begin{itemize}
    \item $\forall_i: s_i + d_i = e_i$
    \item $\forall_t: \sum_{s_i \leq t < e_i} c_i \leq C$
\end{itemize}

\subsubsection{Sweep line algorithm}
It's use to compute the cumulated
profile and check that it does not exceed the capacity.


\begin{enumerate}
    \item \begin{tabular}{m{5cm}m{10cm}}
            \begin{itemize}
                \item One time point for start $(start, +capa)$
                \item One time point for end $(start, -capa)$ 
            \end{itemize}&
                    \includegraphics[width=7cm]{sweepLine}
                \end{tabular}


            \item \begin{tabular}{m{5cm}m{8cm}}
                    \begin{itemize}
                        \item Sort time points
                    \end{itemize}
                    &

                    (0,+7), (0,+10), (10,-7), (10,4), (13,-10), (13,9), (15,4), (18,-9), (22,7), (29,-4), (31,-4), (38,-7)
                \end{tabular}

            \item \begin{tabular}{m{5cm}m{10cm}}
                    \begin{itemize}
                        \item Sweep-line(t) = height of the profile at time $t$.
                    \end{itemize}
                    &
                    \includegraphics[width=7cm]{sweepLine2}
                \end{tabular}

                \begin{lstlisting}[mathescape]
input = eventQueue$[(time,c)]$
t = input.head.time   // Current time of the sweep line
h = 0                 //  current capacity of the sweep line 

while (input.nonEmpty) {
    while (input.nonEmpty && input.head.time == t){
    (time,c) = input.dequeue
    h = h + c
    }
    add (t,h) to the profile
    t = input.top.time
}
            \end{lstlisting}
    \end{enumerate}


\subsubsection{Optimistic Resource Profile}

\begin{tabular}{m{11cm}m{6cm}}
The optimistic resource profile is built based on the
\textbf{mandatory parts} which is the time where the activity will
use the resource whatever it final position.
&
\includegraphics[width=4cm]{mandatory}
\end{tabular}

\begin{itemize}
    \item \textbf{Update minimum start time} at the earliest
        time such that it is not in conflict with the optimistic
        resource profile.

        \paragraph{Time complexity}: $O(n)$ per task since recource
        profile has $O(n)$ intervals. So, $O(n^2)$ overall.
\end{itemize}


\subsection{Large Neighborhood Search}

\begin{tabular}{m{12cm}m{3cm}}
    The diversification is the most weakness of CP for hard COPs.
    The idea of LNS is stuck for too long $\to$ jump in the search space
    by using restart!
    &
    \includegraphics[width=2cm]{LNS}
\end{tabular}

\begin{tabular}{m{2cm}m{11cm}}
    Randomized restart & 
    \begin{itemize}
        \item Use some random decisions in your heuristic (value or variable)
        \item Use a limit on the search (number of backtracks or
            time)
        \item If no feasible solution is found within this limit,
            restart.
    \end{itemize}
\end{tabular}

\subsubsection{LNS work}

\begin{tabular}{m{10cm}m{6cm}}
    \begin{enumerate}

        \item Find a first initial solution, $S*$
        \item Randomly relax $S*$ and re-optimize with search limit

            Relax = fix some variables to their values in $S*$ and CP
            search the other

            \begin{itemize}
                \item A portion of the variables is selected (=\textbf{fragment})
                    and are relaxed to the \textbf{initial domain}
                \item The other variables are frozen to their value in the current solution
                \item A limited CP search \textbf{improving solutions}
            \end{itemize}

        \item Replace $S*$ by the best solution found
    \end{enumerate}

    &
    \includegraphics[width=6cm]{lns.png}
\end{tabular}

\paragraph{Advantages}
\begin{itemize}
    \item Good diversification if fragment well chosen $\Rightarrow$ no
        meta-heuristic needed (tabu,...) because neighborhood is large
        enough
    \item No need to design complex feasible neighborhoods because CP
        is in charge of feasibility

    \item Intensification with CP search and efficient 
        exploration of neighborhoods with CP
    \item Scalability of LS
\end{itemize}


\subsubsection{LNS parameters}
Parameters of LNS :
\begin{itemize}
    \item \textbf{Fragment size} and \textbf{time/backtrack limit}:
       strongly linked parameters because
       \begin{enumerate}
           \item fragment size determines neighborhood size
           \item limit determine the maximal effort to explore
       neighborhood.
       \end{enumerate}

       \paragraph{Note:} a good LNS should never be stopped only by the limit\ldots

       \paragraph{Adaptive versions}:
       \begin{enumerate}
           \item Fix a time/backtrack limit
           \item if neighborhood fully explored $\to$ increase fragment size
           \item else: decrease fragment size
       \end{enumerate}

    \item \textbf{fragment selection} (can be combined).

        Should contain important variables and related variables.

        \begin{itemize}
            \item \begin{tabular}{m{3cm}m{10cm}}
                    Random selection &
                \begin{itemize}
                    \item Surprisingly good
                    \item Generic
                    \item Excellent diversification
                    \item Intensification from CP search
                \end{itemize}
            \end{tabular}

        \item \begin{tabular}{m{3cm}m{10cm}}
                    Specific selection &
                \begin{itemize}
                    \item Only for one problem
                    \item Use knowkedge of the problem to select variables
                    \item Usually randomized to some extend
                \end{itemize}
            \end{tabular}
        \end{itemize}
\end{itemize}

\subsubsection{QAQ}
%TODO LNS relax

\subsubsection{Vehicle routing}
%TODO explain

\subsubsection{Cutting stock}
%TODO column generation








