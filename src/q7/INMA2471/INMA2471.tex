\documentclass[11pt,a4paper]{article}

% French
\usepackage[utf8x]{inputenc}
\usepackage[frenchb]{babel}
\usepackage[T1]{fontenc}
\usepackage{lmodern}
\usepackage{ifthen}

% Color
% cfr http://en.wikibooks.org/wiki/LaTeX/Colors
\usepackage{color}
\usepackage[usenames,dvipsnames,svgnames,table]{xcolor}
\definecolor{dkgreen}{rgb}{0.25,0.7,0.35}
\definecolor{dkred}{rgb}{0.7,0,0}

% Floats and referencing
\newcommand{\sectionref}[1]{section~\ref{sec:#1}}
\newcommand{\annexeref}[1]{annexe~\ref{ann:#1}}
\newcommand{\figuref}[1]{figure~\ref{fig:#1}}
\newcommand{\tabref}[1]{table~\ref{tab:#1}}

% Listing
\usepackage{listings}
\lstset{
  numbers=left,
  numberstyle=\tiny\color{gray},
  basicstyle=\rm\small\ttfamily,
  keywordstyle=\bfseries\color{dkred},
  frame=single,
  commentstyle=\color{gray}=small,
  stringstyle=\color{dkgreen},
  %backgroundcolor=\color{gray!10},
  %tabsize=2,
  rulecolor=\color{black!30},
  %title=\lstname,
  breaklines=true,
  framextopmargin=2pt,
  framexbottommargin=2pt,
  extendedchars=true,
  inputencoding=utf8x
}
\newcommand{\oz}{\textsc{Oz}}
\newcommand{\java}{\textsc{Java}}
\newcommand{\clang}{\textsc{C}}
\newcommand{\keyword}{mot clef}

% Math symbols
\usepackage{amsmath}
\usepackage{amssymb}
\usepackage{amsthm}

% Sets
\newcommand{\R}{\mathbb{R}}
\newcommand{\Rn}{\R^n}
\newcommand{\Rnn}{\R^{n \times n}}
\newcommand{\C}{\mathbb{C}}
\newcommand{\K}{\mathbb{K}}
\newcommand{\Kn}{\K^n}
\newcommand{\Knn}{\K^{n \times n}}

% Chemistry
\newcommand{\std}{\ensuremath{^{\circ}}}
\newcommand\ph{\ensuremath{\mathrm{pH}}}

% Theorem and definitions
\theoremstyle{definition}
\newtheorem{mydef}{Définition}
\newtheorem{mynota}[mydef]{Notation}
\newtheorem{myprop}[mydef]{Propriétés}
\newtheorem{myrem}[mydef]{Remarque}
\newtheorem{myform}[mydef]{Formules}
\newtheorem{mycorr}[mydef]{Corrolaire}
\newtheorem{mytheo}[mydef]{Théorème}
\newtheorem{mylem}[mydef]{Lemme}
\newtheorem{myexem}[mydef]{Exemple}
\newtheorem{myineg}[mydef]{Inégalité}

% Unit vectors
\usepackage{esint}
\usepackage{esvect}
\newcommand{\kmath}{k}
\newcommand{\xunit}{\hat{\imath}}
\newcommand{\yunit}{\hat{\jmath}}
\newcommand{\zunit}{\hat{\kmath}}

% rot & div & grad & lap
\DeclareMathOperator{\newdiv}{div}
\newcommand{\divn}[1]{\nabla \cdot #1}
\newcommand{\rotn}[1]{\nabla \times #1}
\newcommand{\grad}[1]{\nabla #1}
\newcommand{\gradn}[1]{\nabla #1}
\newcommand{\lap}[1]{\nabla^2 #1}


% Elec
\newcommand{\B}{\vec B}
\newcommand{\E}{\vec E}
\newcommand{\EMF}{\mathcal{E}}
\newcommand{\perm}{\varepsilon} % permittivity

\newcommand{\bigoh}{\mathcal{O}}
\newcommand\eqdef{\triangleq}

\DeclareMathOperator{\newdiff}{d} % use \dif instead
\newcommand{\dif}{\newdiff\!}
\newcommand{\fpart}[2]{\frac{\partial #1}{\partial #2}}
\newcommand{\ffpart}[2]{\frac{\partial^2 #1}{\partial #2^2}}
\newcommand{\fdpart}[3]{\frac{\partial^2 #1}{\partial #2\partial #3}}
\newcommand{\fdif}[2]{\frac{\dif #1}{\dif #2}}
\newcommand{\ffdif}[2]{\frac{\dif^2 #1}{\dif #2^2}}
\newcommand{\constant}{\ensuremath{\mathrm{cst}}}

% Numbers and units
\usepackage[squaren, Gray]{SIunits}
\usepackage{sistyle}
\usepackage[autolanguage]{numprint}
%\usepackage{numprint}
\newcommand\si[2]{\numprint[#2]{#1}}
\newcommand\np[1]{\numprint{#1}}

\newcommand\strong[1]{\textbf{#1}}
\newcommand{\annexe}{\part{Annexes}\appendix}

% Bibliography
\newcommand{\biblio}{\bibliographystyle{plain}\bibliography{biblio}}

\usepackage{fullpage}
% le `[e ]' rend le premier argument (#1) optionnel
% avec comme valeur par défaut `e `
\newcommand{\hypertitle}[7][e ]{
\usepackage{hyperref}
{\renewcommand{\and}{\unskip, }
\hypersetup{pdfauthor={#6},
            pdftitle={Synth\`ese d#1#2 Q#3 - L#4#5},
            pdfsubject={#2}}
}

\title{Synth\`ese d#1#2 Q#3 - L#4#5}
\author{#6}

\begin{document}

\ifthenelse{\isundefined{\skiptitlepage}}{
\begin{titlepage}
\maketitle

 \paragraph{Informations importantes}
   Ce document est grandement inspiré de l'excellent cours
   donné par #7 à l'EPL (École Polytechnique de Louvain),
   faculté de l'UCL (Université Catholique de Louvain).
   Il est écrit par les auteurs susnommés avec l'aide de tous
   les autres étudiants et votre aide est la bienvenue.
   Il y a toujours moyen de l'améliorer surtout que si le cours
   change, la synthèse doit être changée en conséquence.
   On peut retrouver le code source à l'adresse suivante
   \begin{center}
     \url{https://github.com/Gp2mv3/Syntheses}.
   \end{center}
   On y trouve aussi le contenu du \texttt{README} qui contient de plus
   amples informations, vous êtes invité à le lire.

   Il y est indiqué que les questions, signalements d'erreurs,
   suggestions d'améliorations ou quelque discussion que ce soit
   relative au projet
   % signalisations ou signalements ?
   sont à spécifier de préférence à l'adresse suivante
   \begin{center}
     \url{https://github.com/Gp2mv3/Syntheses/issues}.
   \end{center}
   Ça permet à tout le monde de les voir, les commenter et agir
   en conséquence.
   Vous êtes d'ailleurs invité à participer aux discussions.

   Vous trouverez aussi des informations sur le wiki
   \begin{center}
     \url{https://github.com/Gp2mv3/Syntheses/wiki}.
   \end{center}
   comme le status des synthèses pour chaque cours
   \begin{center}
     \url{https://github.com/Gp2mv3/Syntheses/wiki/Status}.
   \end{center}
   vous pouvez d'ailleurs remarquer qu'il en manque encore beaucoup,
   votre aide est la bienvenue.

   Pour contribuer au bug tracker et au wiki, il vous suffira de
   créer un compte sur Github.
   Pour interagir avec le code des synthèses,
   il vous faudra installer \LaTeX.
   Pour interagir directement avec le code sur Github,
   vous devez utiliser \texttt{git}.
   Si cela pose problème,
   nous sommes évidemment ouvert à des contributeurs envoyant leur
   changement par mail ou n'importe quel autre moyen.
\end{titlepage}
}{}

\ifthenelse{\isundefined{\skiptableofcontents}}{
\tableofcontents
}{}
}


\usepackage{pifont}
\usepackage{diagbox}
\usepackage[normalem]{ulem}
\newcommand{\cmark}{\ding{51}}%
\newcommand{\xmark}{\ding{55}}%

\DeclareMathOperator{\epi}{epi}
\DeclareMathOperator{\conv}{conv}
\DeclareMathOperator{\dom}{conv}

\hypertitle{en}{Optimization models and methods}{7}{INMA}{2471}
{Benoît Legat}
{François Glineur}

\section{Reminder}
\subsection{Convex set}
\begin{mydef}[Convex set]
  A set $X$ is convex if for all $x,y \in X$ and $\lambda \in [0,1]$,
  $\lambda x + (1-\lambda) y \in X$.
\end{mydef}

\begin{myprop}
  Let $X,Y$ be convex set.
  The intersection $X \cap Y$ and the cartesian product $X \times Y$ are convex
  but the union $X \cup Y$ is \emph{not} necessarily convex.
\end{myprop}

\subsection{Convex function}
\begin{mydef}[Epigraph]
  The epigraph of a function is defined as
  \[ \epi(f) = \{(x,t) \in \Rn \times \R | f(x) \leq t\}. \]
\end{mydef}

\begin{mydef}[Convex function]
  A function $f$ is convex iff $\epi(f)$ is convex.
\end{mydef}

\begin{myprop}
  Let $f$ and $g$ be convex functions.
  $\max(f,g)$ is convex but $\min(f,g)$ is \emph{not} necessarily convex.
  \begin{proof}
    \begin{align*}
      \epi(\max(f,g))
      & = \{(x,t) \in \Rn \times \R : \max(f(x),g(x)) \leq t\}\\
      & = \{(x,t) \in \Rn \times \R : f(x) \leq t \land g(x) \leq t\}\\
      & = \{(x,t) \in \Rn \times \R : f(x) \leq t\} \cap \{(x,t) \in \Rn \times \R : g(x) \leq t\}\\
      & = \epi(f) \cap \epi(g)\\
      \epi(\min(f,g))
      & = \{(x,t) \in \Rn \times \R : \min(f(x),g(x)) \leq t\}\\
      & = \{(x,t) \in \Rn \times \R : f(x) \leq t \lor g(x) \leq t\}\\
      & = \{(x,t) \in \Rn \times \R : f(x) \leq t\} \cup \{(x,t) \in \Rn \times \R : g(x) \leq t\}\\
      & = \epi(f) \cup \epi(g).
    \end{align*}
  \end{proof}
\end{myprop}

\begin{myprop}
  If $f$ is convex and $g$ is linear, $f \circ g$ is convex.
\end{myprop}

\section{Model Building}
\subsection{Building linear models}
\subsubsection{Objective function}
\begin{myprop}[Linear objective]
  \label{prop:linobj}
  Every optimization problem can be modeled with a linear objective.
  \begin{proof}
    Let
    \begin{align*}
      \min_{x \in \Rn} f(x)\\
      x & \in X
    \end{align*}
    be a modelisation of the problem.
    It can be reformulated as
    \begin{align*}
      \min_{x \in \Rn} t\\
      f(x) & \leq t\\
      x & \in X.
    \end{align*}
  \end{proof}
\end{myprop}

\begin{myrem}[Convex]
  If the original model was convex, $f$
  is convex and therefore, $f(x) \leq t \equiv \epi(f)$ defines % TODO define epi(f)
  a convex set.
  The intersection of convex sets is convex so the
  reformulation is convex too!
\end{myrem}

\begin{myrem}[Piecewise linear]
  If $f$ was piecewise linear, the reformulation is linear.
\end{myrem}

\subsubsection{Monotone transformation}
We can apply strictly monotone transformation to equations and/or the objective
(changing the inequality and/or the $\min$ in $\max$ if it is strictly decreasing).
The reformulated model is equivalent but can become convex or even linear !

\begin{myexem}[Geometric optimization]
  The problem
  \begin{align*}
    \min_{r \geq 0} r_1 r_2 r_3 r_4 r_5\\
    r_1r_2r_3 & \geq 1.05\\
    r_2r_3r_4 & \geq 1.10\\
    r_3r_4r_5 & \geq 1.15
  \end{align*}
  can be reformulated (since $\log$ is strictly increasing) as
  \begin{align*}
    \min_{r \geq 0} \log(r_1) + \log(r_2) + \log(r_3) + \log(r_4) + \log(r_5)\\
    \log(r_1) + \log(r_2 + \log(r_3) & \geq \log(1.05)\\
    \log(r_2) + \log(r_3 + \log(r_4) & \geq \log(1.10)\\
    \log(r_3) + \log(r_4 + \log(r_5) & \geq \log(1.15)
  \end{align*}
  which is linear in $x_i \eqdef \log(r_i)$.
\end{myexem}

\subsubsection{D.E.A Data Envelopment Analysis}
Let's analyse the non-convex problem
\begin{align*}
  \min_x \frac{c^T x + d}{f^T x + g}\\
  Ax & \leq b
\end{align*}
under the asumption $f^Tx+g > 0$ for all $x$ such that $Ax \leq b$.

Charnes and Cooper in \cite{charnes1985foundations} describes
how to reformulate this as a convex problem.
We introduce $t > 0$ and $y \in \in \mathbb{R}^n$
such that $x = y/t$.
The problem becomes
\begin{align*}
  \min_{y \in \mathbb{R}^n,t > 0} \frac{c^T y + dt}{f^T y + gt}\\
  Ay & \leq bt
\end{align*}
which is \emph{homogeneous} since for $\lambda > 0$,
$(y,t)$ and $(\lambda y, \lambda t)$ have the same objective
and $(y,t)$ is admissible iff $(\lambda y, \lambda t)$ is admissible).

We can therefore suppose that $f^T y + g t = 1$
(if not, we take $\lambda = \frac{1}{f^Ty + gt}$ and
$(y,t) \leftarrow (\lambda y, \lambda t)$), we have
\begin{align*}
  \min_x c^T y + dt\\
  Ay & \leq bt.
\end{align*}

\subsection{Building convex model}

\subsubsection{Convexification}
Let's suppose that we have a general (see property~\ref{prop:linobj}) problem
\begin{align*}
  \min_x c^T x\\
  x & \in X
\end{align*}

Its admissible domain is not necessarily convex but we can make it so while
keeping the \emph{same} optimal value.

\begin{mydef}[Convex hull]
  The convex hull $\conv(X)$ of a set $X$ is the smallest convex set containing $X$.
  It can be defined as the set of of all convex combinations of the points of $X$
  \[ \left\{\sum_{i=1}^k \lambda_ix_i | x_1, \ldots, x_k \in X \lambda \sum \lambda_i = 1 \land \lambda_i \geq 0\right\}. \]
\end{mydef}
From the simple observation
\begin{align*}
  c^T\left(\sum_{i=1}^k \lambda_ix_i\right)
  & = \sum_{i=1}^k \lambda_i c^Tx_i\\
  & \geq \min_i c^Tx_i
\end{align*}
we see that the convexified problem has the same optimal solution but its optimal
set is also a superset of the optimal set of the original problem.

Sadly this is very hard to do in general even if it is easy in some case.

\subsubsection{Strict convexity}
\begin{mydef}[Strict convexity]
  $f$ is strictly iff
  \[ f(\lambda x + (1-\lambda)y) < \lambda f(x) + (1-\lambda)f(y) \]
  for all $0 < \lambda < 1$ and for all $x,y$.
\end{mydef}

\begin{mytheo}
  The problem
  \begin{align*}
    \min_x f(x)\\
    x & \in X
  \end{align*}
  with $f$ stricly convex and $X$ convex has at most 1 optimal solution.
\end{mytheo}

If $\lap f(x)$ is positive definite for all $x$ then $f$ is strictly convex
but it is \emph{not} necessary.
For example $x^4$ is strictly convex but $12x^2$ is 0 for $x=0$ and
is therefore only positive semi-definite.

An usual trick to make a convex problem strictly convex is convexification
\begin{mytheo}[Regularization]
  If $f$ is convex then $f + \mu\|x\|^2$ is strictly convex for all $\mu > 0$.
\end{mytheo}

\subsubsection{Convex maximization}
Let's first notice that since a linear function is both convexe and concave,
a linear problem is both a convex minimizatoin \emph{and} a convex maximization.
It has therefore the properties of both.

\begin{mydef}[Extreme point]
  Let $X$ be a convex set.
  The point $x$ is said to be extreme if it does is not inside any segment formed
  by two distinct points of $X$.
\end{mydef}

\begin{mytheo}
  If the problem $\max_{x \in X} f(x)$ with $f$ and $X$ convex has an optimal solution,
  one of the optimal solution is an extreme point of $X$.
\end{mytheo}
This is a property that we already knew for linear optimization.
Now we see that it is actually a property of a bigger class of problems.

\subsubsection{Quasiconvexity}
\begin{mydef}[Quasiconvex]
  A function $f$ is quasiconvex if for all $\alpha$,
  its sub-level set
  \[ L_\alpha(f) = \{x \in \dom f | f(x) \leq \alpha\} \]
  is convex for all $\alpha$.
\end{mydef}

If $f$ is convex, $L_\alpha(f)$ is also convex for all $\alpha$
but the converse is not true (e.g. $\sqrt{|x|}$).

If $f$ is convex and $g$ is concave, $\frac{f}{g}$ is quasiconvex on $\{x | g(x) > 0\}$.
We see that $\frac{f}{g} \leq \alpha$ is equivalent to
$f - \alpha g \leq 0$ which is a convex set since $f - \alpha g$ is convex.
Here we see that it only works for $\alpha \geq 0$, since if $\alpha < 0$, $-\alpha g$ is concave.

\paragraph{Solving procedure}
If the objective $f$ is quasiconvex and the admissible set $X$ is convex then
we can find the optimal solution using bisection.
If the convex set
\begin{align*}
  f(x) & \leq \alpha\\
  x & \in X
\end{align*}
is empty, $\alpha$ is a lower bound.
Otherwise it is an upper bound.

Checking whether it is non-empty can be done by solving a convex problem
such as
\begin{align*}
  \min 0\\
  f(x) & \leq \alpha\\
  x & \in X
\end{align*}

% TODO relation restriction

\section{Alternative et duality}
\subsection{Theorems of the alternative}
\begin{mytheo}[Farkas' lemma]
  Only one of the two following systems has a solution
  \begin{align*}
    Ax & \leq b & A^T\lambda & = 0\\
       & & \lambda & \geq 0\\
       & & b^T\lambda & < 0.
  \end{align*}
  Only one of the two following systems has a solution
  \begin{align*}
    Ax & = b & A^T\lambda & \geq 0\\
     x & \geq 0 & b^T\lambda & < 0.
  \end{align*}
  Only one of the two following systems has a solution
  \begin{align*}
    Ax & = b & A^T\lambda & = 0\\
       & & b^T\lambda & \neq 0.
  \end{align*}
\end{mytheo}

\subsection{Duality and certificates}
Let's consider the problem primal (P)
\begin{align*}
  \max b^Ty\\
  Ay & \leq c
\end{align*}

If we want to prove that the optimal objective is $\theta$,
we have to find a proof that
\[ Ay \leq c \Rightarrow b^Ty \leq \theta. \]
Let's analyse the LHS
\begin{align*}
  a_1 y & \leq c_1\\
  \vdots \quad  & \quad\quad \vdots\\
  a_m y & \leq c_m
\end{align*}
For $x \geq 0$, this system implies the following system
\begin{align*}
  x_1 a_1 y & \leq x_1 c_1\\
  \vdots \quad  & \quad\quad \vdots\\
  x_m a_m y & \leq x_m c_m
\end{align*}
and also
\begin{align*}
  \sum_{i=1}^m x_m a_m y & \leq \sum_{i=1}^m x_m c_m\\
  x^T A y & \leq x^T c.
\end{align*}

We see that $\lambda$ is a certificate for $\theta$ if
\begin{align*}
  A^T x & = b\\
  c^T x & = \theta\\
  x & \geq 0.
\end{align*}

Of course, finding a certificate just gives us an upper bound.
Finding the certificate for the lowest upper bound is the optimization problem
\begin{align*}
  \min c^T x\\
  A^T x & = c\\
  x & \geq 0
\end{align*}
which is called the \emph{dual} (D) of (P).

Of course if we find an admissible couple $(y,x)$ for which $b^T y = \theta$ and
$c^T x = \theta$ we are sure that $\theta$ is the optimal value.
In this case $x$ is called the optimality certificate of $y$.

\subsubsection{Weak and strong duality}
\begin{mytheo}[Weak duality]
  For all admissible solution $y$ of the primal and all admissible solution $x$ of the dual,
  \[ b^Ty \leq c^Tx. \]
\end{mytheo}

\begin{mytheo}[Strong duality]
  Let $y^*$ be an optimal solution for (P).
  Then there exists an (optimal\footnote{The fact that this solution is optimal is a consequence of the weak duality}) solution $x^*$ to the problem (D) such that
  \[ c^Tx^* = b^Ty^*. \]
\end{mytheo}

\begin{mycorr}
  The different possibilites are summarized below
  \begin{center}
    \begin{tabular}{l|c|c|c}
      \diagbox{P}{D} & impossible & finite & unbounded\\
      \hline
      impossible & \checkmark & \xmark & \cmark\\
      \hline
      finite & \xmark & \cmark & \xmark\\
      \hline
      unbounded & \cmark & \xmark & \xmark\\
    \end{tabular}
  \end{center}
  \begin{itemize}
    \item If (P) is unbounded, (D) is impossible.
    \item If (D) is unbounded, (P) is impossible.
    \item If (P) is finite, (D) is finite.
    \item If (D) is finite, (P) is finite.
    \item \sout{If (P) is impossible, (D) is unbounded.}
    \item \sout{If (D) is impossible, (P) is unbounded.}
  \end{itemize}
  $\checkmark$ means that it is possible but is unstable
  in the sense that for a problem for which the dual and the primal is
  impossible, modifying a little bit $A$, $b$ or $c$ makes it fall in another category.

  In practice, we are never in this category except if the problem has
  been crafted on purpose.
\end{mycorr}

% TODO Analyse de sensibilité

\subsection{Robust optimization}
Let consider a linear problem
\begin{align*}
  \max b^T y\\
  a_i^Ty & \leq c_i & \forall i, a_i \in \mathcal{A_i} = \{a | C_ia \leq d_i\}.
\end{align*}

Each constraint is equivalent to
\[ \max_{a_i \in \mathcal{A}_i} y^Ta_i \leq c_i. \]
The LHS is a linear optimization problem in $a_i$ (not $y$ !).
By strong duality, we know that it is equivalent to
\begin{align*}
  \min d_i^Tx & \leq c_i\\
  C_i^T x & = y\\
  x & \geq 0
\end{align*}
or simply asking that there exists an $x$ such that
\begin{align*}
  d_i^Tx & \leq c_i\\
  C_i^T x & = y\\
  x & \geq 0
\end{align*}
but asking for the existence of such $x$ is done
by adding these 3 constraints in the model.

\section{Conic optimization}
For a cone $K$,
we define the parial order $\succeq_K$ as
\[ y \preceq x \iff x \succeq_K y \iff x - y \succeq_K \iff x - y \in K. \]
It is a \emph{partial} order since there can be $x,y$ such that
neither $x \succeq y$ nor $x \preceq y$.

\begin{mydef}[Convex cone]
  A convex cone is a set $K$ such that
  \begin{itemize}
    \item if $x \in K$ and $\lambda \in \mathbb{R}_+$, $\lambda x \in \preceq_K$
      (i.e. if $x \preceq_K y$, $\lambda x \preceq_K \lambda y$);
    \item $x, y \in K$, $x + y \in K$
      (i.e. if $x_1 \preceq_K y_1$ and $x_2 \preceq_K y_2$, $x_1 + x_2 \preceq y_1 + y_2$).
  \end{itemize}
\end{mydef}
It is interesting to compare it with the definition of a vector space.
The difference is that here we impose that $\lambda \geq 0$.
If we allowed $\lambda < 0$, it would mean that if $x \preceq y$, $-x \preceq -y$ and that wouldn't work.

\begin{mydef}[Regular cone]
  A regular cone is a convex cone $K$ such that
  \begin{itemize}
    \item $K$ is closed;
    \item $K$ has a non-empty interior;
    \item $K \cap (-K) = \{0\}$ (i.e. $K$ is \emph{pointed}).
  \end{itemize}
\end{mydef}
The fact that $K$ is pointed means that $x \preceq_K 0 \iff 0 \preceq_K -x$.
For example, the cone $\{(x,y) \in \mathbb{R}^2 | x \geq 0\}$ is not pointed.

\begin{mydef}[Lorentz cone]
  The Lorentz cone $\mathbb{L}^n$ is defined as
  \[ \mathbb{L}^n \eqdef \left\{(x_0, x_1, \ldots, x_n) \in \mathbb{R}^{n+1} | x_0 \geq \|(x_1, \ldots, x_n)\|_2 = \sqrt{x_1^2 + \cdots + x_n^2}\right\}. \]
\end{mydef}

\begin{myexem}[Ellipsoid]
  An ellipsoid is defined as $\{x|(x-c)^TE(x-c) \leq 1\}$ wher $c$ is the center
  and $E \succ 0$.
  Using Cholesky factorization, we have $E = LL^T$ and the condition becomes
  \begin{align*}
    \sqrt{(x-c)^TE(x-c)} & \leq 1\\
    \sqrt{(x-c)^TLL^T(x-c)} & \leq 1\\
    \|L^T(x-c)\|_2 & \leq 1\\
    \begin{pmatrix}
      1\\
      L^T(x-c)
    \end{pmatrix}
    & \in \mathbb{L}^n.
  \end{align*}
\end{myexem}

\biblio
%http://www2.isye.gatech.edu/~nemirovs/Lect_IPM.pdf

\end{document}
