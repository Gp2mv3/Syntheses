\documentclass[en]{../../../../../../eplexam}

\hypertitle{Fluid mechanics and transfer II}{7}{MECA}{2322}{2018}{Janvier}{All}
{Master 1 (2017-2018)}
{Grégoire Winckelmans and Matthieu Duponcheel}

\section{Écoulements compressibles}
On considère la tuyère d'une fusée. Le gaz de travail est supposé parfait et est caractérisé par un $\gamma=1.25$ et $R=598 \frac{J}{kg K}$. La température de stagnation est $4100K$. La poussée de la fusé est $T=4.075kN$ et le débit est $Q=10.32kg/s$. La pression extérieure est $P_e=1.6 .10^4 Pa$.

L'écoulement présente un choc droit dans la partie divergente de la tuyère, pour lequel le rapport $A_{sh}/A_e=0.64$

On demande :
\begin{enumerate}
\item Les conditions à la sortie de la tuyère : $p_e,T_e,\rho_e,u_e,c_e,M_e$
\item Les conditions après le choc : $M_{sh2},p_{sh2}$
\item Les conditions avant le choc : $M_{sh1},p_{sh1}$
\item La pression $p_0$ nécessaire dans le tank pour un tel écoulement
\item Le rapport $A_{t}/A_{sh}$
\item La valeur des surfaces $A_t,A_{sh}$ et $A_e$
\item Comment va évoluer la position du choc si la fusée prend de l'altitude? Décrire uniquement (pas de calculs)
\item Quel serait l'aire $A_e$ pour que l'écoulement soit optimal ? C'est-à-dire que l'écoulement soit supersonique dans tout le divergent jusqu'à la sortie où $M_e=1$ et $p_e=p_a$.\\
Dans ce cas-là quelle serait la poussée T ?
\item Dans ce cas-ci, quel serait l'effet d'un gain/perte d'altitude sur le choc ? Discuter uniquement (pas de calculs).
\end{enumerate}
Pour rappel : $T=Q u_e + (p_a-p_e) A_e$

\section{Écoulements incompressibles}
Soit l'écoulement autour d'un profil aérodynamique avec épaisseur mais sans cambrure. Pour rappel l'écoulement autour d'un cylindre est donné par la formule suivante :
$$F_1(Z)=U_{\infty} (Z e^{-i\alpha}+\frac{a^2}{Z} e^{i \alpha})+\frac{\Gamma}{2\pi i} log(\frac{Z}{a})$$
De plus on considère l'écoulement autour d'un cylindre de rayon a placé en $Z_s=d$, du à un point vortex :
$$F_2(Z)=\frac{\Gamma_2}{2 \pi i} (log(\frac{Z-d}{a})-log(\frac{Z-a^2/d}{a}))$$
Pour rappel, la transformée du cylindre en profil aérodynamique :
$$z(Z)=Z+\frac{(a-\epsilon)^2}{Z-\epsilon}$$

\subsection{Partie A ($F_1$)}

\begin{enumerate}
\item Expliquer comment obtenir l'écoulement sur le profil aérodynamique dans le plan z sur base de l'écoulement autour du cylindre dans le plan Z. On obtient la condition K-J : $\Gamma=-4\pi U_{\infty} a sin(\alpha)$. Physiquement que représente cette condition? Mathématiquement comment l'obtient t-on?
\item Dessiner l’écoulement autour de la sphère dans le plan Z et autour du profil aérodynamique dans le plan z.
\end{enumerate}
\subsection{Partie B}
On travail maintenant avec l'écoulement $F_2(Z)$.
\begin{enumerate}
\item Donner l'expression de U-iV
\item Donner la vitesse $U_{\theta}$ sur le cercle et montrer que $U_r(\theta)=0$ et que donc le cercle est bien une ligne de courant (Streamline)
\item Dessiner l'écoulement
\item En comparant $U_{\theta}^2$ en $\theta=0$ et $\theta=\pi$ dire si la force s'exerce vers la gauche ou la droite.
\end{enumerate}
\subsection{Partie C}
On considère l'écoulement due aux 2 conditions, $F(Z)=F_1(Z)+F_2(Z)$
\begin{enumerate}
\item Donner l'expression de F(Z) et déduiser la condition K-J
\item La vorticité totale est $\Gamma_t = \Gamma + (-\Gamma_2)$, obtenez son expression. Conseil : monter que la vorticité est bien la même que sans le point vortex si celui-ce est placé en l'infini 
\end{enumerate}

\section{Échange de chaleur - Théorie}
On étudie le fonctionnement d'un cycle fermé à $N_2$ pour lesquels les propriétés sont supposées constantes. Les gaz de sortie de la turbine sont utilisés pour réchauffer les gaz d'entrées de cette dernière à l'aide d'un échangeur (les gaz d'échanges sont donc les mêmes mais à $T°$ et/ou pressions différentes). Les propriétés du gaz sont considérées constantes tout au long du cycle. L'échangeur (contre-courant) est donc placé entre la pompe HP et la turbine de détente. On demande :
\begin{enumerate}
\item Quelle est la particularité de cet échangeur?
\item Donner le profil de température du fluide froid en fonction de A(x)
\item Tracer l'évolution des profils de température en fonction de A(x)
\item Démontrer et donner la formule de Hausenberg (LMTD)
\item Donner et démontrer epsilon-NTU
\end{enumerate}

\section{Échange de chaleur - Exercice}
On étudie le réservoir d'oxygène liquide d'une fusée. L'oxygène présente dans le réservoir est à l'état de liquide saturée en équilibre avec sa vapeur à 90.2K (table fournie). Le réservoir est modélisé par un cylindre de hauteur $H=16.46 m$ et un diamètre de $D=10.058m$ en alu de 6mm d'épaisseur dont la conductivité thermique est $237$ $W/Km^2$.

Le tank est soumis à un vent de 25 km/h ce qui donne un coefficient convectif de 12 $W/Km^2$. La température extérieure est de 20$^\circ C$.

On considère $C_{sf} = 0.015$ et $n=1.7$.

Valeurs $O_2$ : 

$ \mu_l = 194.7e^{-6} \\
   h_v = 76.69e^3 \\
   h_l = -133.4e^3 \\
   \sigma = 13.17e^{-3} \\
   \rho_l = 1141 \\
   \rho_v = 4.467 \\
   c_{p,l} = 1699 \\
   Pr_l = 2.19\\
$


\subsection{Partie A : échange de chaleur}
 L’oxygène du tank va commencer à bouillir dans le tank. Prenez Csf =0.0103 et n=1.7 avec la formule de Rohsenow. On néglige les transferts pour radiation sur le réservoir de la fusée, et on ne s’intéresse qu'aux échanges de chaleur uniquement sur la surface latérale du cylindre du réservoir.
\begin{enumerate}
\item 	Donner la température de la paroi intérieure du tank (conseil : équation non-linéaire à résoudre par itération)
\item 	Donner le flux de chaleur traversant la parois du tank
\item Donner la température de la paroi extérieure du tank 
\item 	Le tank est percé de 2 trous pour éviter les surpressions (un pour l'échappement de l'oxygène gazeux, l'autre pour l'entrée de l'oxygène liquide), donner le débit d’oxygène.
\end{enumerate}
\subsection{Partie B : échange par radiations}

Le tank est placé dans un milieu radiatif composé du sol à $15^\circ C$ et du ciel à $25\circ C$. Pour simplifier le problème on considère que l’angle entre la verticale et l’horizon est toujours de 90°, en d’autres mots, chaque point de la surface du tank voit la partie supérieure comme étant le ciel et l'inférieur le sol.\\
Le sol et le ciel sont considérés comme des corps noirs, et la peinture blanche du tank fourni une émissivité de $\epsilon =0.1$. Les surfaces sont des corps gris et diffus.
\begin{enumerate}
\item Tracer le network résistif associé au problème.
\item 	Donner les facteurs de vue entre le tank et le ciel $F_{ts}$ et entre le tank et le sol $F_{tg}$, uniquement sur base des propriétés des facteurs de vue (ne pas faire de calcul de facteur de vue sur base des formules).
\item 	Donner le flux de radiation reçu par le tank et comparer avec le flux du à la convection (si jamais vous n'avez pas trouver la T extérieure de parois prendre $T_{w,e}=190$ K)
\end{enumerate}
\end{document}
