\section{What is the main difference between traditional software systems and context-aware
systems?}

\emph{\enquote{A software system is context-aware
if it can extract, interpret and use context information
and adapt its functionality to the current context of use.}} \quad

So the main difference between a traditional software and a context-aware system is that context-aware systems ``operate on data that is not explicitly given to them, data that they observe or gather for themselves'' and ``have an explicit representation of context and contextual variations as first class citizens''.
%can detect changes in their surrounding environment and then adapt their behavior.

\section{Explain, in your own words, what problems context-oriented programming tries to solve.}

Nowadays, software needs to act differently based on the context in which they are executed.
A typical example is a smartphone receiving a call. It will vibrate in a quiet environment and use a loud ringtone in a noisy environment.
We need software to \textit{adapt} their behavior based on contexts.

Context-Oriented Programming aims to bring a new paradigm to ease building adaptable applications.

%Applications should become more aware of their execution context, and should adapt dynamically to such context to provide services that match their clients’ needs to the best extent possible.

\section{Explain, in your own words, what context-oriented programming is.}

Context-Oriented Programming (COP) is a new approach in software programming to bring context-dependent adaptations into applications.

Using COP, developers can explicitly represent contexts and decide when a context will become active or inactive. They will also define \textit{adaptations} that will be used instead of the default behavior when a context is activated.


\section{Two different techniques exist for implementing dynamic adaptation of software behavior
to context: method dispatch and method pre-dispatch. Briefly explain and compare these two techniques.}
\begin{itemize}
\item \textbf{Method pre-dispatch : }
Whenever a \textbf{context is (de)activated} : For every class c and selector s the context adapts, find all active (methods defined for contexts that are currently active) methods $$M(c, s)={ m_1 , m_2 , m_3 , ..., m_n  }$$
reorder them according to specificity :
$$m_1  < m_2  < m_3  < ... < m_n$$
and \textbf{deploy the first one} $m_1$.
So $m_1$ is the most specific implementation for the current context, resend invokes the remaining methods in order and $m_n$ is (usually) the default implementation

\item \textbf{Method dispatch : } Whenever \textbf{a message is sent} (to receiver r, with selector s and arguments a), find all active methods that match the message
 $$M(r,s,a)={ m_1 , m_2 , m_3 , ..., m_n  }$$
 reorder them according to specificity
$$m_1  < m_2  < m_3  < ... < m_n$$
and \textbf{invoke the first one} $m_1$
So $m_1$ is the most specific implementation for the current context, resend invokes the remaining methods in order and $m_n$ is (usually) the default implementation
\end{itemize}
\begin{table}[!ht]
\centering
\begin{tabular}{|cl|l|}
\hline
\multicolumn{3}{|c|}{Comparison} \\ \hline
& \textbf{Method Pre-Dispatch}     &  \textbf{Method Dispatch}     \\
& $M(c, s)={ m_1 , m_2 , m_3 , ..., m_n  } $     & $M(r,s,a)={ m_1 , m_2 , m_3 , ..., m_n  }$     \\
\textit{trigger} & context activation     & Message sending       \\
 \textit{action} & Method deployment     & Method invocation    \\
\textit{mechanism} & Structural reflection & BEhavioural reflection\\ \hline
& \textit{more commonly supported} & \textit{more powerful} \\ \hline
\end{tabular}
\caption{Comparison between method pre-dispatch and method dispatch}
\label{comparisonDispatch}
\end{table}


\section{One particular technique for implementing dynamic adaptation of software behaviour to
context is that of method pre-dispatch. Explain that technique in detail and illustrate it with a
concrete example.}

\textbf{Method pre-dispatch : }
Whenever a \textbf{context is (de)activated} : For every class c and selector s the context adapts, find all active (methods defined for contexts that are currently active) methods $$M(c, s)={ m_1 , m_2 , m_3 , ..., m_n  }$$
reorder them according to specificity :
$$m_1  < m_2  < m_3  < ... < m_n$$
and \textbf{deploy the first one} $m_1$.
So $m_1$ is the most specific implementation for the current context, resend invokes the remaining methods in order and $m_n$ is (usually) the default implementation



\subsection{Example in Pseudo-Ruby}
\begin{lstlisting}[language=Ruby]
class CallAdvertiser
    def advertise_call
        ringtone # default
    end
end

module QuietAdvertiser
    def advertise_call
        vibrate # quiet context
    end
end

module OnCallAdvertiser
    def advertise_call
        just_bip # on_call context
    end
end

module AdvertiseWithScreening
    def advertise_call
        show_caller_info + proceed # with_screening context
    end
end
\end{lstlisting}

\begin{enumerate}
\item By default, the \textit{default} implementation is used because it is the only one activated \textit{advertise\_call} method: \textit{ringtone}.
\begin{table}[!ht]
\centering
\begin{tabular}{|c|}
\hline
$$default$$
\\ \hline
\end{tabular}
\caption{\textit{advertise\_call} methods order}
\end{table}

\item Imagine that the user enters a school and the \textit{quiet} context becomes active.
Immediately, with method pre-dispatch, all methods \textit{advertise\_call} that are active are reordered. The implementation of \textit{quiet} context comes first because it is the last activated context. This implementation is deployed and used to advertise future calls.
\begin{table}[!ht]
\centering
\begin{tabular}{|c|}
\hline
$$quiet < default$$
\\ \hline
\end{tabular}
\caption{\textit{advertise\_call} methods order}
\end{table}

\item The user decide to activate a new feature that display the firstName, lastName and picture of the caller when receiving a call. The context \textit{with\_screening} is activated and all the activated \textit{advertise\_call} methods are reordered.
The implementation of \textit{with\_screening} comes first, then the one of \textit{quiet} and finally the \textit{default} one.
When a call is received, the caller info are shown and because of the call to \textit{proceed}, the next method in the chain is called, this is the \textit{quiet} one.
\begin{table}[!ht]
\centering
\begin{tabular}{|c|}
\hline
$$with\_screening < quiet < default$$
\\ \hline
\end{tabular}
\caption{\textit{advertise\_call} methods order}
\end{table}

\item The user leave the school, the \textit{quiet} context is deactivated and methods are reordered again. The first one is \textit{with\_screening} and the last one is \textit{default}. When a call is received, the caller info are shown and the phone plays the ringtone.
\begin{table}[!ht]
\centering
\begin{tabular}{|c|}
\hline
$$with\_screening < default$$
\\ \hline
\end{tabular}
\caption{\textit{advertise\_call} methods order}
\end{table}

\end{enumerate}
