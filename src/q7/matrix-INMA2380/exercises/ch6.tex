\section{Positive matrices}
\exo{1}
\begin{solution}
  The following inequalities are equivalent
  \begin{align*}
    \rho x & \leq Sx\\
    \rho x_i & \leq (Sx)_i & \forall i\\
    \rho & \leq \frac{(Sx)_i}{x_i} & \forall i\\
    \rho & \leq \min_i \frac{(Sx)_i}{x_i}.
  \end{align*}
\end{solution}

\exo{3}
\begin{solution}
  In can be seen as a sort of consequence of the power method
  applied on the column of $S$
  ($S^\infty = S^\infty
  \begin{pmatrix}
    a_{:1} & \cdots & a_{:n}
  \end{pmatrix}$).

  Let's prove it simply using the Jordan form of $S$.
  Without loss of generality, the eigenvalues of S are $1 = \rho(S) > |\lambda_2| \geq \cdots \geq |\lambda_k|$).

  \begin{equation}
    T^{-1}ST =
    \begin{pmatrix}
      1 & 0\\
      0 &
      \diag_{i = 2, \ldots, k}
      \{J_i\}
    \end{pmatrix} \triangleq D \label{D}
  \end{equation}

  where
  \[ J_i =
    \begin{pmatrix}
      \lambda_i & 1 &  & 0\\
                & \lambda_i & \ddots &\\
                & & \ddots & 1\\
      0 & & & \lambda_i
    \end{pmatrix}
  \]
  we have
  \begin{align*}
    S^k & =
    T
    \begin{pmatrix}
      1 & 0\\
      0 &
      \diag_{i = 2, \ldots, k}
      \{J_i^k\}
    \end{pmatrix}
    T^{-1}.
  \end{align*}
  We can check now that
  \begin{align*}
    J_i^k & =
    \begin{pmatrix}
      \lambda_i^k & {k \choose 1}\lambda_i^{k-1} & {k \choose 2}\lambda_i^{k-2} & \cdots\\
                  & \lambda_i^k & \ddots & \\
                  & & \ddots & {k \choose 1} \lambda_i^{k-1}\\
      0 & & & \lambda_i^k
    \end{pmatrix}\\
    J_i^k & =
    \lambda_i^k
    \begin{pmatrix}
      1 & {k \choose 1}\lambda_i^{-1} & {k \choose 2}\lambda_i^{-2}k & \cdots\\
          & 1 & \ddots & \\
          & & \ddots & {k \choose 1}\lambda_i^{-1}\\
      0 & & & 1
    \end{pmatrix}
  \end{align*}
  and ${k \choose j} \lambda_i^{-j}$ is a polynomial in $k$ while $\lambda_i^k$ is a decreasing exponential ($|\lambda_i| < 1$) in $k$ so
  $J_i^k$ converges to $0$ and $S^k$ to
  \begin{align}
    T
    \begin{pmatrix}
      1 & 0\\
      0 & 0
    \end{pmatrix}
    T^{-1}
    & =
    t_{:1}(t^{-1})_{1:}
  \end{align}

  However, since $T^{-1}T = I$, $(t^{-1})_{1:}t_{:1} = 1$ and $(t^{-1})_{1:}t_{:j} = 0 = (t^{-1})_{j:}t_{:1}$ for $j \neq 1$.
  We can see that
  \begin{align*}
    St_{:1}
    & = TDT^{-1}t_{:1}\\
    & = TDe_1\\
    & = Te_1\\
    & = t_{:1}\\
    (t^{-1})_{1:}S
    & = (t^{-1})_{1:}TDT^{-1}\\
    & = e_1^TDT^{-1}\\
    & = e_1^TT^{-1}\\
    & = (t^{-1})_{1:}.
  \end{align*}
  so $t_{:i}$ is a eigenvector of 1.
  Since the geometric multiplicity is 1, $t_{:i}$ is a multiple of $\mathbf{x}$,
  the normalised ($\1^T t_{:i} = 1$) Perron eigenvector.
  Also $(t^{-1})_{1:}$ is a multiple to the \emph{left} eigenvector $\1^T$.
  So, there are $\alpha,\beta$ such that
  \begin{align*}
    t_{:i} & = \alpha \mathbf{x}\\
    (t^{-1})_{1:} & = \beta \1^T
  \end{align*}
  but since $(t^{-1})_{1:}t_{:1} = 1$, $\alpha\beta = 1$ so
  \[ \lim_{n \to \infty} S^n = \mathbf{x}\1^T. \]
\end{solution}
