\section{Polynomial matrices}

\exo{2}
Show that the invertible polynomial matrices are exactly the unimodular matrices.

\begin{solution}
We start by showing that unimodular matrices are invertible.
This is trivially true, by considering (6.1) in the lecture notes.
We then need to show that invertible matrices must be unimodular.
Consider two invertible polynomial matrices \(E(\lambda), F(\lambda)\).
Since \(E(\lambda)F(\lambda) = I_n\), then \(\det(E(\lambda)) = 1/\det(F(\lambda))\), which means that both determinants are nonzero constants.
\(E(\lambda)\) and \(F(\lambda)\) are thus unimodular matrices.
\end{solution}

\exo{0}
Show that the elementary transformations of type 1 and 2 defined in (6.2) applied on the rows and on the columns of a polynomial matrix define a multiplicative group.

\nosolution


\exo{1}
The normal rank of \(P(\lambda)\) is equal to the rank of \(P(\lambda_0)\) for almost every (in the Lebesgue sense) \(\lambda_0 \in \C\) (or \(\R\)).
When they are not equal, the rank of \(P(\lambda_0)\) is smaller than the normal rank.

\begin{solution}
As \(M(\lambda), N(\lambda)\) are unimodular (and hence invertible), they don't influence the rank of \(P(\lambda_0)\).
We thus have \(\mathop{\mathrm{rank}}(P(\lambda_0)) = \mathop{\mathrm{rank}}(\diag\{e_i(\lambda_0)\}))\).
The rank of the latter is only reduced when \(\lambda_0\) is one of the roots of \(e_i(\lambda)\), which has a finite amount of roots (and the reduction thus occurs ``almost never'').
\end{solution}

\exo{2}
Verify that the Smith form of the matrix
\[
\lambda I_4 - \begin{bmatrix}
1 & & &\\
& 1 & &\\
&& 2 & 1\\
&&&2\\
\end{bmatrix}
\]
is \(\diag\{1, 1, \lambda-1, (\lambda-1)(\lambda-2)^2\}\).

\begin{solution}
This matrix looks like
\[
\begin{bmatrix}
\lambda - 1 & & &\\
& \lambda- 1 & &\\
&& \lambda- 2 & -1\\
&&&\lambda- 2\\
\end{bmatrix}
\]
By applying Algorithm~6.1 from the lecture notes, we obtain
\[
\begin{bmatrix}
1 & & &\\
& 1 & &\\
&& \lambda- 1 &\\
&&&(\lambda-1)(\lambda- 2)^2\\
\end{bmatrix}.
\]
\end{solution}
