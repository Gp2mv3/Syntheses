\section{Old questions}

\exo{3}
\begin{solution}
	See syllabus.
	
	If $P$ is orthogonal, we have
	\begin{align*}
	PP & = P\\
	PPP^* & = PP^*\\
	P & = PP^*
	\end{align*}
	and $P^*P = I$.
\end{solution}

\exo{2}
\begin{solution}
	See syllabus.
	
	\begin{align*}
	\sum_{i=0}^\infty \frac{(A+B)^i}{i!}
	& = \sum_{i=0}^\infty \frac{\sum_{k=0}^i \frac{i!}{k!(i-k)!} A^{i-k}B^k}{i!}\\
	& = \sum_{i=0}^\infty \sum_{k=0}^i \frac{A^{i-k}}{(i-k)!} \frac{B^k}{k!}\\
	& =
	\left(\sum_{i=0}^\infty \frac{A^i}{i!}\right)
	\left(\sum_{i=0}^\infty \frac{B^i}{i!}\right).
	\end{align*}
	because each term $A^aB^b$ is present with the term $\frac{1}{a!b!}$.
\end{solution}

\exo{1}
\begin{solution}
	We have
	
	\begin{align*}
	\begin{bmatrix}
	\lambda_0 & 1      &        & \\
	& \ddots & \ddots & \\
	&        & \ddots & 1\\
	&        &        & \lambda_0\\
	\end{bmatrix}
	& =
	\lambda_0 I + N
	\end{align*}
	for
	\[
	N =
	\begin{bmatrix}
	& 1 &        & \\
	&   & \ddots & \\
	&   &        & 1\\
	&   &        & \\
	\end{bmatrix}.
	\]
	The rest is a simple consequence of the exercise~3.7.
	
	It is important to note for the next page that $N^n = 0$.
	This is indeed a consequence of the fact that $N = J - \lambda_0I$
	where $J$ is a Jordan block of $\lambda_0$ of size $n$.
	$J$ is therefore a matrix with only one eigenvalue $\lambda_0$
	of algebraic multiplicity $n$ but geometric multiplicity $1$.
	Hence the whole set $\mathbb{C}^n$ is an invariant subspace of $N$
	which means that $(J - \lambda_0I)^n = 0$.
	
	We can see for example for $n = 4$ that
	
	\begin{align*}
	N & =
	\begin{bmatrix}
	0 & 1 & 0 & 0\\
	0 & 0 & 1 & 0\\
	0 & 0 & 0 & 1\\
	0 & 0 & 0 & 0
	\end{bmatrix}\\
	N^2 & =
	\begin{bmatrix}
	0 & 0 & 1 & 0\\
	0 & 0 & 0 & 1\\
	0 & 0 & 0 & 0\\
	0 & 0 & 0 & 0
	\end{bmatrix}\\
	N^3 & =
	\begin{bmatrix}
	0 & 0 & 0 & 1\\
	0 & 0 & 0 & 0\\
	0 & 0 & 0 & 0\\
	0 & 0 & 0 & 0
	\end{bmatrix}\\
	N^4 & =
	\begin{bmatrix}
	0 & 0 & 0 & 0\\
	0 & 0 & 0 & 0\\
	0 & 0 & 0 & 0\\
	0 & 0 & 0 & 0
	\end{bmatrix}
	\end{align*}
	We can see that $e_1$ is an eigenvector of $\lambda_0$ but
	$e_2$, $e_3$, $e_4$ are not.
	\begin{itemize}
		\item $Je_2 = \lambda_0 e_2 + e_1$
		so $(J - \lambda_0 I)e_2 = e_1$.
		\item $(J - \lambda_0 I)e_3 = e_2$ and $(J - \lambda_0 I)^2e_3 = e_1$.
		\item $(J - \lambda_0 I)e_4 = e_3$, $(J - \lambda_0 I)^2e_3 = e_2$
		and $(J - \lambda_0 I)^3e_4 = e_1$.
	\end{itemize}
	
	Note that $J^4 \mathbb{C}^n = \mathbb{C}^n$ if $\lambda_0 \neq 0$.
	It is not to be mistaken from $(J - \lambda_0 I)^4 \mathbb{C}^n = \{0\}$.
\end{solution}

\exo{2}
\begin{solution}
	For 4.7, let
	\[ \lambda = \argmin_{\lambda^* = j\omega} \sigma_{\mathrm{min}}(A - \lambda^*I). \]
	Let $C = A - \lambda I$.
	If $C = U^* \Sigma V$, we can take $\Delta = -u_n \sigma_n v_n^*$ which gives
	\[ C + \Delta = A - \lambda I + \Delta = (A + \Delta) - \lambda I \]
	of rank $n-1$.
	
	For 4.8, it is the same except that we take
	\[ \lambda = \argmin_{\lambda^* = \exp(j\omega)} \sigma_{\mathrm{min}}(A - \lambda^*I). \]
\end{solution}
