\section{}

\subsection{Exercise 0 (Reduction and/or attacks)}
\copypaste{1}{7}

\subsection{Exercise 1 (Variable-length MAC)}
Considering a known hash function $h^s:\{0,1\}^{2l}\rightarrow\{0,1\}^{l}$,
let's note by $H^s$ the corresponding Merkle-Damg{\aa}rd transform hash function, 
\emph{i.e.} 

\begin{center}
\begin{tikzpicture}[scale=0.5]
\tikzstyle{every node}=[text centered, inner sep = 2pt]

    \trapeze{$h^s$}{\position}
    \draw [->] \position +(-3,0) node [left] {$IV$} -- +(-1,0);
    \draw [<-] \position + (-1,1) -| ++(-2,3) node [above] {\mylabel};
    \draw \position + (1,0) -- +(2,0) node {};
    \renewcommand{\position}{(4,0)}
    \renewcommand{\mylabel}{$x_2$}
    \trapeze{$h^s$}{\position}
    \draw [->] \position +(-2,0) node [below] {} -- +(-1,0);
    \draw [<-] \position + (-1,1) -| ++(-2,3) node [above] {\mylabel};
    \draw [->]\position + (1,0) -- +(2,0) node [right] {$\ldots$};
    \renewcommand{\position}{(9.5,0)}
    \renewcommand{\mylabel}{$x_n$}
    \trapeze{$h^s$}{\position}
    \draw [->] \position +(-2.2,0) node [below] {} -- +(-1,0);
    \draw [<-] \position + (-1,1) -| ++(-2,3) node [above] {\mylabel};
    \draw \position + (1,0) -- +(2,0) node {};
    \renewcommand{\position}{(13.5,0)}
    \renewcommand{\mylabel}{$\left|x\right|$}
    \trapeze{$h^s$}{\position}
    \draw [->] \position +(-2,0) node [below] {} -- +(-1,0);
    \draw [<-] \position + (-1,1) -| ++(-2,2.95) node [above] {\mylabel};
    \draw [->] \position + (1,0) -- +(2,0) node [right] {$H^s(x)$};
\end{tikzpicture}
\end{center}
%
when $x=x_1||\cdots||x_n$ for some integer $n$ and when the $x_i$'s are 
$l$-bit strings.

Show why, with a private key $k$ of length $l$, the MAC scheme 
$$t:=H^s(k||m),$$
is \emph{not} existentially unforgeable under an adaptive chosen-message attack.


\medskip
\begin{solution}
  If we have the tag of $p$, which is (let's consider that $k$ and $p$ are $l$ bits long for simplicity)
  \[ t_p = H^s(k\|p) = h^s(h^s(h^s(IV \| k) \| p) \| 2l) \]
  we can find the tag of $p\|2l\|w$ (where $w$ is $l$ bits long for simplicity)
  without knowing $k$ since we know $h^s$.
  It is
  \begin{align*}
    H^s(k\|p\|2l\|w)
    & = h^s(h^s(h^s(h^s(h^s(IV \| k) \| p) \| 2l) \| w) \| 4l)\\
    & = h^s(h^s(H^s(k \| p) \| w) \| 4l)\\
    & = h^s(h^s(t_p \| w) \| 4l)
  \end{align*}
  Since $p\|2l\|w \neq p$, this gives us an existential forgery.
\end{solution}

\subsection{Exercise 2 (Fixed-length MAC)}
Consider the fixed-length MAC $\Pi:=\langle\Gen,\Mac,\Vrfy\rangle$
defined as follows:
\medskip
\begin{itemize}
  \item $\Gen$: choose random $k\leftarrow \{0,1\}^n$  \smallskip
  \item $\Mac$: on input $m,k \in\{0,1\}^n$, output $t:=F_k(m)$  \smallskip
  \item $\Vrfy$: on input $k, m, t \in \{0,1\}^n$ output 1 iff $t=F_k(m)$
\end{itemize}
\medskip
%
Prove that, if $F$ is a PRF, $\Pi$ is existentially unforgeable under
an adaptive chosen-message attack. Hint:
\medskip
%
\begin{enumerate}
  \item Consider the scheme $\Pi'$ defined as $\Pi$ except that a truly
        random function is used instead of a pseudo-random one. Show that
        $\Pi'$ is existentially unforgeable under an adaptive chosen-message
        attack.  \smallskip
  \item Consider a PPT adversary who can produce an adaptive forgery on
        $\Pi$ with non negligible probability $\epsilon(n)$. Using this
        adversary, show that $F$ cannot be a PRF.
\end{enumerate}

% homework 2 of Dan Boneh, Winter 2011, Problem 2 
\begin{solution}
  \begin{itemize}
    \item
      Let $\tilde{\Pi} = \langle \tilde{\Gen}, \tilde{\Mac}, \tilde{\Vrfy} \rangle$, defined as:
      \begin{itemize}
        \item $\tilde{\Gen}$: chooses a random $f$.
        \item $\tilde{\Mac}$: on input $m$, outputs $f(m)$.
        \item $\tilde{\Vrfy}$: on input $(m,t)$, outputs $1$ iff $f(m) = t$.
      \end{itemize}

      Let's analyse the maximum value of $\Pr[\MacForge_{\A, \tilde{\Pi}}(n) = 1]$ for an adversary $\mathcal{A}$.
      If after $q$ different queries (it gains no info doing the same query twice),
      $m_1, \ldots, m_q$, $\A$ outputs $(m, t)$, what are its chances of success ?
      Let $f:\{0,1\}^n \to \{0,1\}^n$.
      There are $(2^n)^{2^n}$ different $f$ and we pick a random one uniformly.
      However, there are only $(2^n)^{2^n-q}$ experiments such that $\A$ have received $(m_i,t_i)$ for $i = 1, \ldots, q$ because
      there are $(2^n)^{2^n-q}$ $f$ such that $f(m_i) = t_i$ for $i = 1, \ldots, q$.
      We could be in any of them.
      Among them, only $(2^n)^{2^n-(q+1)}$ are such that $f(m) = t$.
      Since $f$ is selected uniformly, we have

      \begin{align*}
        \Pr[\MacForge_{\A, \tilde{\Pi}}(n) = 1]
        & = \Pr[f(m) = t | f(m_i) = t_i, \forall i = 1, \ldots, q]\\
        & = \frac{\Pr[f(m) = t, f(m_i) = t_i, \forall i = 1, \ldots, q]}{\Pr[f(m_i) = t_i, \forall i = 1, \ldots, q]}\\
        & = \frac{\frac{(2^n)^{2^n-(q+1)}}{(2^n)^{2^n}}}{\frac{(2^n)^{2^n-q}}{(2^n)^{2^n}}}\\
        & = \frac{(2^n)^{2^n-(q+1)}}{(2^n)^{2^n-q}}\\
        & = \frac{1}{2^n}.
      \end{align*}
      A shortcut would have been to argued that since, $f(m)$ is independent of the $f(m_i)$ so

      \begin{align*}
        \Pr[\MacForge_{\A, \tilde{\Pi}}(n) = 1]
        & = \Pr[f(m) = t | f(m_i) = t_i, \forall i = 1, \ldots, q]\\
        & = \Pr[f(m) = t]\\
        & = \frac{(2^n)^{2^n-1}}{(2^n)^{2^n}}\\
        & = \frac{1}{2^n}.
      \end{align*}

      It is quite surprising that instead of a upper bound
      on $\Pr[\MacForge_{\A, \tilde{\Pi}}(n) = 1]$
      depending on $\A$ (and reached for $\A$ super smart),
      it is actually independent of $\A$.
    \item
      Let's now suppose that we have an adversary $\A$
      that win with non-negligible probability against a PRF $F$
      and show that under this assumption we can build
      a distinguisher $\D$ for $F$.

      $\D$ will simply take a function $g$ as input
      and run $\A$ using $g$ to create the tags.
      He has $g$ so he can see if $\A$ wins or lose.
      If $\A$ wins, $\D$ outputs $1$, otherwise, it outputs $0$.

      We know that if $g$ is a pseudo random function,
      $\Pr[\MacForge_{\A, \Pi_g} = 1] = \frac{1}{2^n}$
      and if $g$ is a PRF
      $\Pr[\MacForge_{\A, \Pi_g} = 1] = \eta(n)$
      where $\eta$ is non-negligible.
      We have therefore
      \[
        |\Pr[\D^{F_k}(1^n) = 1] - \Pr[\D^{f}(1^n) = 1]|
        = \left|\eta(n) - \frac{1}{2^n}\right|
      \]
      which is non-negligible.

      % Not sure it works
%    \item
%      Another simpler solution is possible. Using the first Hint, we can say that if $F_k$ is a PRF, it has a maximum of $2^n$ possible outputs
%      where a truly random function has exactly $2^n$ outputs. So if we suppose $\Pi$ secure with a PRF, then $\tilde{\Pi}$ is also secure because
%      $\epsilon_{\tilde{\Pi}}  = \frac{1}{2^n} \geq \epsilon_{\Pi}$. We then can play the PRF game to prove the security of $\Pi$ with the second hint.
  \end{itemize}

\end{solution}

\subsection{Exercise 3 (Hash functions from\ldots hash functions)}
Let $H_2:\{0,1\}^{2l}\rightarrow\{0,1\}^{l}$ and $H_3:\{0,1\}^{3l}\rightarrow\{0,1\}^{l}$ be
collision resistant hash functions. For $2l$-bit strings $x_i$'s, consider the following two constructions.
\medskip
\begin{itemize}
	\item $H_4:\{0,1\}^{4l}\rightarrow\{0,1\}^{l}$ ; 
	      $x=x_1||x_2\rightarrow H_2\left(H_2(x_1)||H_2(x_1\oplus x_2)\right)$
	      \smallskip
	\item $H_6:\{0,1\}^{6l}\rightarrow\{0,1\}^{l}$ ; 
	      $x=x_1||x_2||x_3\rightarrow H_3\left(H_2(x_1\oplus x_2)||H_2(x_2\oplus x_3)||H_2(x_3\oplus x_1)\right)$
\end{itemize}
\medskip
Determine whether these hash functions are still collision resistant or not.

\begin{solution}
  \begin{itemize}
    \item
      Let's show that from a collision of $H_4$, we generate a collision for $H_2$
      which prove that $H_4$ is collision resistant since $H_2$ is so.
      Let's suppose that we have $x_1\|x_2 \neq y_1\|y_2$ are such that $H_4(x_1\|x_2) = H_4(y_1\|y_2)$.
      \begin{itemize}
        \item
          If $H_2(x_1) \| H_2(x_1 \xor x_2) \neq H_2(y_1) \| H_2(y_1 \xor y_2)$,
          we have a collision for $H_2$ since their image by $H_2$ is identical.
        \item
          If $H_2(x_1) \| H_2(x_1 \xor x_2) = H_2(y_1) \| H_2(y_1 \xor y_2)$,
          we have $H_2(x_1) = H_2(y_1)$ \emph{and} $H_2(x_1 \xor x_2) = H_2(y_1 \xor y_2)$.
          \begin{itemize}
            \item If $x_1 \neq y_1$, we have a collision for $x_2$ since $H_2(x_1) = H_2(y_1)$.
            \item If $x_1 = y_1$, then $x_2 \neq y_2$ since $x_1\|x_2 \neq y_1\|y_2$.
              Therefore $x_1 \xor x_2 \neq y_1 \xor y_2$ and we have collision on $H_2$.
          \end{itemize}
      \end{itemize}
    \item
      $H_6$ is not collision resistant since $H_6(x_1\|x_2\|x_3) = H_6((x_1 \xor w)\|(x_2 \xor w)\|(x_3 \xor w))$
      for all $w$ (collision if $w\neq 0^{2l}$). 
      Indeed, since $\xor$ is associative and commutative,
      \begin{align*}
        & = H_6((x_1 \xor w)\|(x_2 \xor w)\|(x_3 \xor w))\\
        & = H_3(H_2((x_1 \xor w) \xor (x_2 \xor w))\|H_2((x_2 \xor w) \xor (x_3 \xor w))\|H_2((x_3 \xor w) \xor (x_1 \xor w)))\\
        & = H_3(H_2(x_1 \xor (w \xor w) \xor x_2)\|H_2(x_2 \xor (w \xor w) \xor x_3)\|H_2(x_3 \xor (w \xor w) \xor x_1)))\\
        & = H_3(H_2(x_1 \xor x_2)\|H_2(x_2 \xor x_3)\|H_2(x_3 \xor x_1)))\\
        & = H_6(x_1\|x_2\|x_3).
      \end{align*}
  \end{itemize}
\end{solution}


\subsection{Exercise 4 (Hash-MAC)}
Suppose $H_0$ and $H_1$ are compression functions but only one is beleived to be collision resistant.
Besides, suppose $\textsc{Mac}_0$ and $\textsc{Mac}_1$ are message authentication codes but only one
of the both schemes is known to be unforgeable. Is it possible to build a secure ''hash-MAC'' from these
inputs? Justify your answer.
%Homework 2 of Dan Boneh, Winter 2011, Problem 5 (DVD security)
\begin{solution}
  We build $H(m) = H_0(m)\|H_1(m)$.
  If we have $m_1 \neq m_2$ such that $H(m_1) = H(m_2)$ then $H_0(m_1) = H_0(m_2)$ and
  $H_1(m_1) = H_1(m_2)$ so the collision resistant hash function has a collision whichever it is.
  However, $H$ is no more a compression function and we cannot use Merkle-Damg\aa{}rd.

  The input of $H$ therefore cannot have arbitrary length but its output is twice the length of the output of $H_0$ and $H_1$ so it is twice the size of a tag.

  The output of $\Mac_0$ and $\Mac_1$ are the size of a tag so we can use the tag
  $H(\Mac_0(k,m)\|\Mac_1(k,m))$ for our Hash-MAC scheme.
  If we are able to output an existential forgery $(m, t)$, since $H$ is collision resistant, that means that we have found $\Mac_0(k,m)\|\Mac_1(k,m)$ and therefore we have found an existential forgery for both $\Mac_0$ \emph{and} $\Mac_1$ which is absurd since one of them is believed to be unforgeable.

  Our Hash-MAC scheme is therefore unforgeable.
\end{solution}

\subsection{Exercise 5 (MAC length extension)}
Let $\Pi':=\langle\Gen',\Mac',\Vrfy'\rangle$ be a secure fixed-length MAC. We define a variable-length MAC $\Pi:=\langle\Gen,\Mac,\Vrfy\rangle$ as follows:

\begin{itemize}
%
\item $\Gen$: choose random $k\leftarrow \{0,1\}^n$
\item $\Mac$: on input $k \in\{0,1\}^n$ and $m \in \{0,1\}^*$ of length $l<2^\frac n4$
\begin{itemize}
\item Parse $m$ into blocks $m_1,\ldots,m_d$ of length $\frac n4$ each (pad with 0's if necessary)
\item Choose random $r\leftarrow \{0,1\}^\frac n4$
\item Compute $t_i:=\Mac_k(r||l||i||m_i)$ for $1\leq i\leq d$, with $|r|=|l|=|i|=\frac n4$
\item Output $t:=\langle r,t_1,\ldots,t_d\rangle$
\end{itemize}
\item $\Vrfy$: on input $k, m, t=\langle r,t_1,\ldots,t_{d'}\rangle$,
\begin{itemize}
\item Parse $m$ into blocks $m_1,\ldots,m_d$ of length $\frac n4$ each
\item Output 1 iff $d=d'$ and, $\forall 1\leq i\leq d$, $\Vrfy'_k(r||l||i||m_i,t_i)=1$
\end{itemize}
%
\end{itemize}

The goal of this exercise is to prove by reduction that $\Pi$ is existentially unforgeable. Let $\A$ (resp. $\A'$) be an adversary attacking the unforgeability of $\Pi$ (resp. $\Pi'$) and let $\epsilon = \textsc{MacForge}_{\A,\Pi}(n)$ (resp. $\epsilon' = \textsc{MacForge}_{\A',\Pi'}(n)$) denotes its advantage.

\begin{enumerate}
  \item Make a quick draw sketching the proof.
  \item Describe formally how $\A'$ should react to a query $\textsc{Mac}_k(m)$.
  \item Define what is a mac forgery for the scheme $\Pi$. Does it necessarily implie a forgery on the scheme $\Pi'$ (justify your answer).
  \item Express $\epsilon$ in function of $\epsilon'$ and conclude.
\end{enumerate}

\begin{solution}
  (Interesting demonstration on the security of this scheme assuming F is a PRF : See reference book p117-118).
  \begin{enumerate}
    \item TODO
    \item $\A'$ should pick a random $r\leftarrow \{0,1\}^{\frac{n}{4}}$, parse m into d blocks $m_1,...,m_d$ and send to its oracle $d$ queries $r\|l\|i\|m_i$ for $i = 1, \ldots, d$.
      After the $d$ queries made to its oracle $O'$, $\A'$ computes $\langle r, t_1, \ldots, t_d \rangle$ where $t_i$ is the answer of the oracle to its $i$th query. He then sends it as an answer to $\A$s query.
    \item For $\Pi$, a forgery is a pair $(m, \langle r, t_1, \ldots, t_d \rangle)$ where $m$ has not been queried before
      and $t_i = \Mac'_k(r\|l\|i\|m_i)$ for $i = 1, \ldots, d$ where $l$ is the length of $m$ and $m = m_1\| \cdots \|m_d$.
      \begin{itemize}
        \item
          If none of its previous query has the same $r$ and $l$, $r\|l\|1\|m_1$ cannot be a query made by $\A'$ and
          $(r\|l\|1\|m_1, t_1)$ is an existential forgery for $\Pi'$ that $\A'$ can output.
        \item
          If 2 previous queries have both $r$ and $l$, then we do not necessarily have an existential forgery.
          However, since $r$ is picked at random ($\A'$ could cheat and make sure that the same $r$ is not picked twice)
          the probability (birthday paradox) of this to happen (if all $m$ have the same $l$ which is the worst case) is approximately
          $\frac{q(n)^2}{2 \cdot 2^n}$ where $q(n)$ is the number of queries made by $\A$.
        \item
          If one unique previous query $m^j$ has this $r$ and $l$, since $m$ is not one of the previous query, there must be $i$
          such that $m_i \neq m_i^j$.
          We know therefore that $r\|l\|j\|m_j$ has never been queried by $\A'$ so $(r\|l\|j\|m_j, t_j)$ is an existential forgery for $\Pi$.
      \end{itemize}
    \item In conclusion, we have
      \begin{align*}
        \Pr[\MacForge_{\A,\Pi}(n) = 1]
        & \leq \Pr[\MacForge_{\A',\Pi'}(n) = 1] + \frac{q(n)}{2^{n+1}}\\
        \epsilon(n)
        & \leq \epsilon'(n) + \frac{q(n)}{2^{n+1}}
      \end{align*}
      so since $\epsilon'(n)$ and $\frac{q(n)}{2^{n+1}}$ are negligible,
      $\epsilon(n)$ is negligible.
  \end{enumerate}
\end{solution}

\subsection{Exercise 6 (Block-cipher based hash function)}
Considering a block cipher
$E:\mathcal{K}\times\mathcal{M}\rightarrow\mathcal{C}$; $(k,m)\rightarrow E(k,m)=Enc_k(m)$
with $\mathcal{K}=\mathcal{M}=\mathcal{C}=\left\{0,1\right\}^l$, one may try to construct
a collision resistant compression function from $\left\{0,1\right\}^{2l}$ to $\left\{0,1\right\}^{l}$.
Show that the following methods do not work :
$$ f_1(x,y)=E(y,x)\oplus y \quad\mbox{ and }\quad f_2(x,y)=E(x,x)\oplus y $$
That is, show an efficient algorithm for constructing collisions for $f_1$ and $f_2$.
Recall that the block cipher E and the corresponding decryption algorithm D are both
known to you (and they are bijective functions).
\begin{solution}
  We will give 2 collisions for $f_1$ and $f_2$.

  We know that $E(k, \cdot)$ is surjective because it must be injective and $\M = \C$.
  Therefore the decryption exists for all $c \in \C$ ! We will use it for the second collision of $f_1$.

  For $f_1$, we have the 2 following collisions

  \begin{align*}
    f_1(D(E(y,x),y), E(y,x))
    & = E(E(y,x), D(E(y,x), y)) \xor E(y,x)\\
    & = y \xor E(y,x)\\
    & = E(y,x) \xor y\\
    & = f_1(x, y)\\
    f_1(D(0, E(y,x) \xor y), 0)
    & = E(0, D(0, E(y, x) \xor y)) \xor 0\\
    & = E(y, x) \xor y\\
    & = f_1(x, y).
  \end{align*}
  and for $f_2$ we have

  \begin{align*}
    f_2(x, E(x,x)) & = E(x,x) \xor E(x,x)\\
                   & = 0 & \forall x \in \{0,1\}^l\\
    f_2(y, E(x,x)) & = E(y,y) \xor E(x,x)\\
                   & = E(x,x) \xor E(y,y)\\
                   & = f_2(x, E(y,y)).
  \end{align*}
  \\
  \textbf{Other approach:}\\
  For $f_1$ : Let's define $k_1$ and $k_2$ such that $k_2$ is equal to $k_1$ except for the last bit which is flipped. Let's now take an arbitrary $x_1$ for which we ask the encryption $T_{x_1} = E(k_1,x_1)$. Now let's flip the last bit of $T_{x_1}$ and call the result $T_{x_2}$. We can now ask for the decryption of $T_{x_2}$ given $k_2$ as input key, which we know exists since D is bijective and $\C = \{0,l\}^{l}$. We then obtain $x_2$. We can now observe that:
  \begin{align*}
    f_1(x_1, k_1) & = E(k_1,x_1) \xor k_1\\
                   & = T_{x_1}  \xor k_1\\
    f_1(x_2, k_2) & = E(k_2,x_2) \xor k_2\\
                   & = T_{x_2} \xor k_2\\
                   & = (T_{x_1} \xor 0^{l-1}||1 )\:\xor\: (k_1 \xor 0^{l-1}||1)\\
                   & = T_{x_1}  \xor k_1\\
  \end{align*}
  For $f_2$ : We can ask for $T_x \:= \:E(x,x)$ and $T_y\: =\: E(y,y)$ for two arbitrary (but different) x and y.  We can see that:
  \begin{align*}
    f_2(y, T_x) & = E(y,y) \xor T_x\\
                   & = T_y  \xor T_x\\
    f_2(x, T_y) & = E(x,x) \xor T_y\\
                   & = T_x \xor T_y\\
  \end{align*}
  Which are both equal with different input, this is a collision
  
\end{solution}

\subsection{Exercise 7 (Blue-ray security)}
The movie industry wants to protect digital content distributed on DVD's. We study
one possible approach. Suppose there are at most a total of $n$ DVD players in the world (e.g.
$n=2^{32}$). We view these n players as the leaves of a binary tree of height $\log_2 n$. 
Each node $\nu_j$ in this binary tree contains an AES key $K_j$ such that 
$\mathrm{Enc}_{K_j}:\{0,1\}^l\rightarrow\{0,1\}^l$ is assumed to be a \emph{secure} encryption. 
These keys are kept secret from consumers and are fixed for all time. 
At manufacturing time each DVD player is assigned a serial number $i\in\left[0,n-1\right]$.
Consider the set $S_i$ of $\log_2(n+1)$ nodes along the path from the root to leaf number
$i$ in the binary tree. The manufacturer of the DVD player embeds in player number $i$ the
$\log_2(n+1)$ keys associated with the nodes in $S_i$. In this way each DVD player ships with 
$\log_2(n+1)$ keys embedded in it (these keys are supposedly inaccessible to consumers).

\begin{enumerate}
	\item Since all DVD players have the key \emph{root} (noted $K_{root}$), 
	      find a way to protect the content $M\in\{0,1\}^l$ of a DVD such that all players can decrypt 
	      the movie (and then read it).
	\item Now suppose that a hacker has been able to extract the key $K_{root}$ embedded in his 
	      DVD player and has published it on the Internet. 
	      Show how the movie industry can encrypt the contents of a new DVD $M\in\{0,1\}^l$ such that only 
	      the owners of a DVD player can read it.
	      Note that the movie indutry does not want to produce several encryptions of the
	      same content $M$ \emph{i.e.} there will be a single manner to protect the DVD.
	\item Suppose the $\log_2n$ keys embedded in DVD player number $r$ are exposed by hackers 
	      and published on the Internet. Show that when the movie industry is about to 
	      distribute a new DVD movie they can encrypt the contents of the DVD using a 
	      ciphertext of size $l\!\cdot\!(1+\log_2n)$ so that all DVD players can decrypt the movie except
	      for player number $r$. In effect, the movie industry disables player number $r$.
	      \\
	      \emph{Hint: the DVD will contain $\log_2n$ ciphertexts where each ciphertext is the 
	      encryption of a unique K under certain $\log_2n$ keys from the binary tree.}
\end{enumerate}
\begin{solution}
  \begin{enumerate}
    \item Encrypt all the DVDs with $K_\text{root}$, i.e. build the ciphertext : $c_{root}=Enc_{K_{root}}(M)$
    \item At the beginning of the DVD, encrypt a random key $K$ twice (using $K_2$ and then $K_3$): $Enc_{K_2}(K)$ and $Enc_{K_3}(K)$.

      We then encrypt all the content $M$ of the DVD using $K$. (We suppose $K_1$ is $K_{\text{root}}$)
      $K_2$ and $K_3$ are the keys associated to the 2 childs of the root. \\
      \textbf{Other possibility}
      \\
      Simply encrypt the contents of the new DVD M as : $Enc_{K2}(M)||Enc_{K3}(M)$. This way, each DVD player is still able to decrypt the content, but nobody that only has $K_{root}$.
    \item
      For every node $i$ on the path from the root to $r$, we must add an encryption of $K$ with the key of its child that is not in the path.
      For example, if $r = 10$ and $n = 16$, we must include the bold keys,
      so we will include the encryption of $K_7$ in a DVD which is quite something.
      \\
      \textbf{Expressed differently}
      \\
      If we want to disable the DVD player 11 for example, we can create a new ciphertext as : $c_{disable} = Enc_{K_2}(M)||Enc_{K_7}(M)||Enc_{K_{12}}(M)||Enc_{K_{player10}}(M)$ 
      
      \begin{center}
        \Tree [.{$K_{\text{root}} = K_1$}
          [.{$\mathbf{K_2}$}
            [.{$K_4$}
              [.{$K_8$}
                [.{0} ]
                [.{1} ]
              ]
              [.{$K_9$}
                [.{2} ]
                [.{3} ]
              ]
            ]
            [.{$K_5$}
              [.{$K_{10}$}
                [.{4} ]
                [.{5} ]
              ]
              [.{$K_{11}$}
                [.{6} ]
                [.{7} ]
              ]
            ]
          ]
          [.{$K_3$}
            [.{$K_6$}
              [.{$\mathbf{K_{12}}$}
                [.{8} ]
                [.{9} ]
              ]
              [.{$K_{13}$}
                [.{\textbf{10}} ]
                [.{11} ]
              ]
            ]
            [.{$\mathbf{K_7}$}
              [.{$K_{14}$}
                [.{12} ]
                [.{13} ]
              ]
              [.{$K_{15}$}
                [.{14} ]
                [.{15} ]
              ]
            ]
          ]
        ]
      \end{center}
  \end{enumerate}
\end{solution}

\subsection{Exercise 9 (Mode of operation)}
Show formally that ECB-mode encryption does not have indistinguishable encryptions in the presence of an eavesdropper.
\begin{solution}
  Let say we have to split the message into $m$ messages of $n$ bits.
  Choosing $m_0 = M_0 \| M_0 \| \cdots \| M_0$ and $m_1 = M_1 \| M_2 \| \cdots \| M_m$ with $M_i \neq M_j$ for $i \neq j$,
  if $b = 0$, $\A$ will get $c = C_0 \| C_0 \| \cdots \| C_0$ for some $C_0 \in \C$. But if b=1, $\A$ will get $c = C_1 \| C_2 \| \cdots \| C_m$ for some $C_i \in \C$.

  An adversary $\A$ can output $b = 0$ iff all the $C_i$s are equals.
  We have
  $$ \Pr[\PrivK^{\text{eav}}_{\A,\text{ECB}}(nm)] = \frac{1}{2} + \frac{1}{2} = 1 $$
  since the $C_i$ cannot be equal if $b = 1$ since we use a PRP.
  If two different $M_i$ were encrypted as same $C_i$, decryption wouldn't be possible.
\end{solution}