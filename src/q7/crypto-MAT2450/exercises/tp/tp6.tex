\section{}
\subsection{Exercise 0}
Same as exercise \ref{subsec:4b.4}
\copypaste{5}{4}

\subsection{Exercise 1}
Same as exercise \ref{subsec:4b.5}
\copypaste{5}{5}

\subsection{Exercise 2 (Jan 2011 evaluation)}
The Digital Signature Standard (DSS, also often called DSA) is one of
the most commonly used signature algorithms. Its three algorithms
$\Gen$, $\mathsf{Sign}$ and $\Vrfy$ work as follows.
\begin{itemize}
	\item $\Gen$: on input $1^n$, select prime integers $p$ and $q$ such
	that $|q|=n$, $q | (p-1)$ and $q^2 \not | \, (p-1)$, together with an
	integer $g$ that generates the subgroup of $\mathbb{Z}_p^*$ of prime
	order $q$. Also choose a hash function $H : \{0,1\}^* \rightarrow
	\mathbb{Z}_q$. Then, select $x \leftarrow \mathbb{Z}_q$ uniformly at
	random, and compute $y:= g^x \mod p$. The public key is $\langle H,
	p, q, g, y\rangle$, and the private key is $\langle x\rangle$.
	\item $\mathsf{Sign}$: in order to sign the message $m \in \{0,1\}^*$, choose
	$k \leftarrow \mathbb{Z}_q^*$ uniformly at random and set $r:= [g^k
	\mod p] \mod q$. Then, compute $s:= (H(m) + xr) \cdot k^{-1} \mod
	q$, and output the signature $(r,s)$.
	\item $\Vrfy$: compute $u_1 := H(m)\cdot s^{-1} \mod q$ and $u_2 := r
	\cdot s^{-1} \mod q$, and output 1 if and only if $r = [g^{u_1} y^{u_2} \mod p] \mod q$. 
\end{itemize}

\begin{enumerate}
	\item Show the correctness of the DSS algorithm.
	\item 
	As randomness is an expensive resource, it is proposed to select
	the random value $k$ once and for all, and to sign all messages
	using that value of $k$. Is this variant of DSS secure? \\
	\emph{(Hint: see what you can deduce from the signature of two
		different messages.)}
\end{enumerate}
\begin{solution}
  \begin{enumerate}
    \item
      $(r, s) = ([g^k \pmod{p}] \pmod{q}, [H(m) + xr]k^{-1} \pmod{q})$, then:
      $u_1 = H(m)s^{-1} \pmod{q}$,  $u_2 = rs^{-1} \pmod{q}$, $r = [g^{u_1} g^{u_2} \pmod{p}] \pmod{q}$, $y = g^x$

      $$\Rightarrow g^{u_1 + xu_2} = g^{s^{-1}(H(m) + rx)} = [g^k \pmod{p}] \pmod{q} = r$$
    \item
      $s = (H(m) + xr)k^{-1} \pmod{q}$, $s' = (H(m') + xr)k^{-1} \pmod{q}$ ($s \neq s'$ otherwise we have a collision).
      $s - s' = (H(m) - H(m'))k^{-1} \pmod{q}$, $k = \frac{H(m) - H(m')}{s - s'}$.
      $s = (H(m) + xr)k^{-1}$ so $\frac{sk - H(m)}{r} = x$ where x is the secret.
  \end{enumerate}
\end{solution}

\subsection{Exercise 3 (RSA permutation with modulus 221)}
Suppose we decide to use an RSA permutation with modulus $221$, we consider RSA encryption scheme, and RSA signature.
\begin{enumerate}
	\item What is the smallest non trivial public exponent $e$ than can be
	chosen?
	\item Can we choose $e=11$? What is the corresponding private exponent $d$? Give the public and private key of the corresponding RSA encryption scheme.
	\item Compute $c := 219^e \pmod{221}$.
	\item Verify that $c^d = 219 \pmod{221}$ as expected.\\
	
	\item How Alice (owning the private key) could sign a message $m$? Sign the message $m=3$ (hint: $22^7= 61 \pmod{221}$).
	\item Is $160$ a valid signature for $m=218$?
	% \emph{Hint: use the square and multiply algorithm.}
\end{enumerate}
\begin{solution}
We suggest you to use a calculator to do this exercise, the assistants said we could (pour faire taire les rageux).
  \begin{enumerate}
      \item \textbf{What is the smallest non trivial public exponent $e$ that can be chosen ?} \newline \newline
      First we have to find $\phi$(221). As 221 is not a prime, we must find $p$ and $q$ such that $221 = pq$. After a few try, we find $p = 13$ and $q = 17$. Wet get then $\phi(221) = (p-1)(q-1) = 192$. \newline 
      The searched smallest $e$ should respect the condition $gcd(192, e) = 1$. Again, after a few try, we find $e = 5$.
      \item \textbf{Could we choose $e = 11$ ?} Yeah, since 11 is a prime, then $gcd(192, 11) = 1$. \newline \newline
      \textbf{What is the corresponding private exponent d ?} $$ ed = 1 \text{ mod 192} \\ 11d = 1 \text{ mod 192} \\ d = \frac{192k + 1}{11}$$
      If we take k=2, we find $d = 35$.\newline \newline
      \textbf{Give the public and private key of the corresponding RSA encryption scheme} \newline
      pk = $(221, 11)$ 
      sk = $(221, 35)$
      \item \textbf{Compute c := $219^e$}(boring calculations) 
      \begin{itemize}
          \item $219^{1} = -2 $
          \item $219^{2} = 4  $
          \item $219^{4} = 16 $
          \item $219^{5} = -32$
          \item $219^{10} = 140$
          \item $219^{11} = 162$
      \end{itemize}
      \item \textbf{Verify that $c^d = 219$} 
      \begin{itemize}
          \item $162^{1} = -59 $
          \item $162^{2} = -55 $
          \item $162^{4} = -69 $
          \item $162^{5} = 93 $
          \item $162^{6} = 38 $
          \item $162^{7} = -32 $
          \item $162^{35} = -2 = 219 $
      \end{itemize}
      \item \textbf{How Alice could sign a messsage m ?} According the scheme described in the slides, $Sign_{(N, d)}(m) := [m^d mod N]$ \newline \newline
      \textbf{Sign the message m = 3} 
      \begin{itemize}
          \item $3^{1} = 3 $
          \item $3^{2} = 9 $
          \item $3^{4} = 81 $
          \item $3^{5} = 22 $
          \item $3^{35} = 61 $ (using the hint given)
      \end{itemize}
      \item \textbf{Is 160 a valid signature for m = 218}
      Let's calculate (yipie) 
      \begin{itemize}
          \item $218^{1} = -3 $
          \item $218^{2} = 9 $
          \item $218^{4} = 81 $
          \item $218^{5} = -22 $
          \item $218^{35} = -61 = 160$
      \end{itemize}
      Yes it is a valid signature
  \end{enumerate}
\end{solution}