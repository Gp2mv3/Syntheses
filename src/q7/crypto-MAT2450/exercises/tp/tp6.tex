
\section{}

\subsection{Exercise 1 (ElGamal Public Key Encryption and CCA Security)}

\begin{enumerate}
	\item Write the security definition of CCA security for a public key encryption scheme.
	\item Let $(c_1,c_2)$ and $(c_1',c_2')$ be two ElGamal ciphertexts, of plaintext $m$ and $m'$ respectively. Can $(c_1 c_1',c_2 c_2')$ be a ciphertext relatively to this scheme?
	\item From $(c_1,c_2)$ a ciphertext of $m$, can you build another ciphertext valid for $m$ (remember that the public key is $(\mathbb{G},g,q,h=g^x)$)? If yes, what is its decryption?
	\item Show that ElGamal Public Key Encryption is not CCA secure.
\end{enumerate}


\begin{solution}
	\begin{enumerate}
		\item
		Given  $\Pi \define \langle \Gen, \Enc, \Dec\rangle$, and adversary $\A$, define the following experiment $\PubK^\mathsf{cca}_{\A, \Pi}$:
		\begin{enumerate}
			\item $\Gen$ probabilistically selects $(pk, sk) \pick \Gen(1^n)$. $pk$ / $sk$ are the public/private key.
			\item $\A$ is given oracle access to $\Dec_{sk}(\cdot)$.
			\item $\A$ outputs $m_0, m_1$ of same length.
			\item Choose $b \pick \bset$, and send $c \define \Enc_{pk}(m_b)$ to $\A$.
			\item $\A$ is again given access to $\Dec_{sk}(\cdot)$ but he cannot ask $\Dec_{sk}(c)$.
			\item $\A$ outputs $b'$.
			\item Define $\PubK^\mathsf{cca}_{\A, \Pi}(n) \define 1$ iff $b=b'$
		\end{enumerate}
		$\Pi \define \langle \Gen, \Enc, \Dec \rangle$ is CCA-secure for a public key encryption scheme if $\forall$ PPT $\A, \exists \negl$:
		\[ \Pr[\PubK^\mathsf{cca}_{\A, \Pi}(n)=1] \leq \frac{1}{2} + \negl(n) \]

		\item $(c_1c_1', c_2c_2')$ is a valid ciphertext for the message $m \cdot m'$.
		Indeed, $\Enc_{pk}(m\cdot m')=(g^{y''}, m\cdot m' \cdot h^{y''})$. If we set $y''=y+y'$, then $=(g^y \cdot g^{y'}, m \cdot h^y \cdot m' \cdot h^{y'})=(c_1 c_1', c_2 c_2')$.
		This is what we call an homomorphic encryption scheme: some operations on the ciphertexts have analogous operations on the encrypted plaintexts.

		\item Yes. We have that $m = m \cdot 1$.
		To encrypt message $1$, simply do $\Enc_{pk}(1)=(c_1', c_2')=(g^{y'}, 1\cdot h^{y'})$ for some $y'$ chosen randomly (do-able since this is a public encryption scheme), so that $c_1'\neq 1$ (simply don't take $0$ as exponent).
		Then, by using the property proven before, we have that $(c_1^*, c_2^*)=(c_1c_1', c_2c_2')$ is a valid ciphertext for message $m$ too.

		\item Now, we are able to see why ElGamal does not hold CCA-security with the 2 properties found.

		If the adversary $A$, after received $c \define (c_1, c_2)$, compute $\Enc_{pk}(1) \define (c_1', c_2')$ and make the following oracle access $\Dec_{sk}(c_1c_1',c_2c_2') \define m_x$, he will now be able to (really) easily determine b from $m_x$. Then we can say that:
		\[ \Pr[\PubK^\mathsf{cca}_{\A, \text{ElGamal}}(n)] = 1 \]
		Which means ElGamal does not have CCA-security with public key encryption.

		\textbf{Other solution:}
		We build an attacker A as follows :
		\begin{enumerate}
			\item A outputs $m_0, m_1 \in M$.
			\item A receives $(c_1,c_2)=\Enc_{pk}(m_b)=(g^y,m_b \cdot g^{xy})$
			\item A queries the decryption of the transformed ciphertext $(c_1 \cdot g^{y'},c_2 \cdot h^{y'})$ for some arbitrary chosen y' from $\Z_q$, and receives directly $m_b$ as a response from his oracle.
		\end{enumerate}
	\end{enumerate}
\end{solution}



\subsection{Exercise 2 (A Variation of ElGamal in \texorpdfstring{$PKE$}{PKE})}

Let consider ElGamal public encryption scheme with Encryption algorithm modified in the following way, where $\M=\bset$:
\begin{itemize}
	\item If $b=0$ then choose a uniform $y\in \Z_{q}$ set $c_1=g^y$, $c_2=h^y$, and the ciphertext is $(c_1,c_2)$.
	\item If $b=1$ then choose independent uniform $y,z \in \Z_{q}$ set $c_1=g^y$, $c_2=g^z$, and the ciphertext is $(c_1,c_2)$.
\end{itemize}

\begin{enumerate}
	\item How is it possible to decrypt correctly such ciphertexts with the private key?
	\item Show that this scheme is CPA secure if $\DDH$ holds in $\mathbb{G}$.
\end{enumerate}


\begin{solution}
	\begin{enumerate}
		\item We set:
		\[ \Dec_{sk}(c_1, c_2)=\begin{cases} 0 \text{ if } c_2 = c_1^x \\ 1 \text{ if } c_2 \neq c_1^x \end{cases} \]
		If $b=0$, then $(c_1, c_2)$ is simply the ElGamal encryption of the message $m=1$.
		In that case, $c_1^x=c_2$, ElGamal decryption always succeeds, and so $\Pr[\Dec_{sk}(c_1, c_2)=0 | (c_1, c_2) \define \Enc_{pk}(0)] = 1$.

		If $b=1$, then $(c_1, c_2)$ is the encryption of a message $\frac{c_2}{c_1^x}$.
		We have that
		\[ \Pr[\Dec_{sk}(c_1, c_2) = 0 | (c_1, c_2) = \Enc_{pk}(1)] = \Pr[c_2 = c_1^x | c_2 = g^z, c_1 = g^y] = \Pr[g^z=g^{xy}] = \frac{1}{|\mathbb{G}|=q} \]
		and in contrast, $\Pr[\Dec_{sk}(c_1, c_2)=1 | (c_1, c_2) \define \Enc_{pk}(1)]=1-\frac{1}{q}$.
		Overall, probability of correct decryption is
		\[ \Pr[\Dec_{sk}(\Enc_{pk}(b))=b] = \frac{1}{2}\cdot 1 + \frac{1}{2}\cdot (1-\frac{1}{q}) = 1-\frac{1}{2q} \ge 1-\frac{1}{2^{n+1}}. \]
		So, there is a negligible probability of failure to decrypt correctly.
		This is allowed by the definition of a public-key encryption scheme, so decryption is OK.

		\item We assume that $\DDH$ is difficult in $\mathbb{G}$.
		We can prove that the scheme is CPA-secure by reduction.
		We assume that we have a PPT adversary $\A$ against $\Pi$ with advantage $\negl$, and we build a distinguisher $\D$ as follows:
		\begin{enumerate}[label=(\arabic*)]
			\item Run $(\mathbb{G},g,q) \pick \mathcal{G}(1^n)$, such that $|q|=n$. \\
			Choose uniformly at random $(x,y,z) \pick \Z_q^3$. \\
			Choose at random $b\pick \bset$. \\
			Set $h_0 \define g^z$, $h_1 \define g^{xy}$. \\
			Send $(\mathbb{G}, q, g, (g^x, g^y, h_b))$ to $\D$.

			\item $\D$ defines $pk = (\mathbb{G},g,q,g^x)$ and sends it to $\A$

			\item $\A$ sends its challenge $m_0$, $m_1$ to $\D$. We assume w.l.o.g that $m_0=0$ and $m_1=1$.

			\item $\D$ sends $c=(g^y,h_b)$ to $\A$

			\item $\A$ outputs its guess $b'$. Then $\D$ outputs $b''=1-b'$.

			\item Define $\DDH_{\A, \mathcal{G}}(n)=1$ iff $b=b''$, and $\PubKcpa(n)=1$ iff $b'$ is equal to the selected message.
			%\item \textcolor{red}{Olivier : Pas d'accord à partir de là. C'est la probabilité de l'output b' qui varie en fonction de $h_b$ autrement dit b", je mets ma proposition en dessous.} $\mathcal{A}$ outputs $b'$
			%\begin{itemize}
			%	\item if $b'=0\Rightarrow h_b=g^{xy}$
			%	\item if $b'=1\Rightarrow h_b=g^z$
			%\end{itemize}
			%$D$ outputs $b''=\Bar{b'}$
			%\item Define : $$\text{PubK}^\text{cpa}_{\mathcal{A},\Pi}(n) = 1 \Leftrightarrow b'\neq b$$ $$\text{DDH}_{\mathcal{A}',\mathcal{G}}(n) = 1 \Leftrightarrow b''=b$$
		\end{enumerate}
		We can observe that if $b=0$, then message $m_1=1$ was sent to $\A$ as we sent $h_0=g^z$, and if $b=1$, then message $m_0=1$ was sent to $\A$, as we sent $h_1=g^{xy}$.

		If we run the experiment, we have:
		\begin{align*}
			\Pr[\PubKcpa(n) = 1] &= \frac{1}{2}+\negl(n)\\
			&= \Pr[b'\neq b]\\
			&= \Pr[b''=b]\\
			&= \Pr[\DDH_{\A',\mathcal{G}}(n) = 1]
		\end{align*}
		If $\negl(n)$ is non negligible, $\A'$ is a PPT adversary that can break $\DDH$ with non negligible probability, which contradicts the assumption. Therefore $\A$ is CPA secure if $\DDH$ is hard in $\mathbb{G}$.

		\textbf{Note: there are a few problems in the computation of the probabilities; any fix / comment is welcome!}
	\end{enumerate}

	\textbf{Another solution}:

	\begin{enumerate}
		\item D outputs 1 $\Leftrightarrow b'=1$.
		\item We observe:

		If $b=0 \Rightarrow \Pr[D \text{ outputs } 1]=\frac{1}{2}$ (as $g^z$ looks random in $\mathbb{G}$)

		If $b=1 \Rightarrow \Pr[D\text{ outputs }1]=\frac{1}{2}+\eta(n)$.
		\item We thus have D distinguishing between $g^z$ and $g^{xy}$ with advantage $\eta(n)$. As we assume that $\DDH$ holds in $\mathbb{G}$, $\eta(n)$ must be negligible and this scheme is CPA-secure.
	\end{enumerate}
\end{solution}



\subsection{Exercise 3 (Decisional Diffie-Hellman, \texorpdfstring{$\Z^*_p$}{Zp}, and \texorpdfstring{$QR_p$}{QRp})}
%\subsection{Exercise 3 (Decisional Diffie-Hellman, $\Z^*_p$, and $QR_p$)}

\copypaste{7}{4}



\subsection{Exercise 4 (A variation of ElGamal in \texorpdfstring{$QR_p$}{QRp})}

Let $p=2q+1$ with $q$ prime, let $\mathbb{G}=QR_p$ the group of squares modulo $p$, and $g$ be a generator of $\mathbb{G}$.
We define ElGamal encryption scheme in this group: The private key is $(\mathbb{G},g,q,x)$, the public key is $(\mathbb{G},g,q,h=g^x)$
where $x \in \Z_{q}^*$ is chosen uniformly. To encrypt a message $m$ $\in \Z_{q}$, choose a uniform $r \in \Z_{q}$, compute $c_1=g^r \mod p$ and $c_2=h^r+m \mod p$ and let the ciphertext be $(c_1,c_2)$.

\begin{enumerate}
	\item What is the order of $g$?
	\item Is this scheme CPA-secure?
\end{enumerate}


\begin{solution}
	\begin{enumerate}
		\item As $g$ is a generator of $\mathbb{G}$, then $\ord(g)=|\mathbb{G}|$.
		According to the previous exercise and given the fact that $\mathbb{G} \define QR_p$, then $|\mathbb{G}| = \frac{p - 1}{2} = q =  \ord(g)$.

		\item The scheme is CPA-Secure and we will prove it by assuming that $QR_p$ holds in $\DDH$.

		Suppose that we have a PPT adversary $\A$ which can solve ElGamal in $QR_p$ with an advantage $\negl$. It works like this:
		\begin{enumerate}
			\item We send $pk$ to $\A$ where pk = $(\mathbb{G}, g, q, h = g^x)$.
			\item $\A$ outputs $m_0$, $m_1$.
			\item The ElGamal challenger-oracle picks a random $b \pick \bset$ and sends back $\Enc_{pk}(m_b) = (c_1, c_2)$  to $\A$.
			\item $\A$ then outputs $b''$. He wins if $b=b''$.
		\end{enumerate}
		We will use $\A$ to build a distinguisher $\D$ able to solve $\DDH_{\D, \mathbb{G}}(n)$:
		\begin{enumerate}
			\item Run $\mathbb{G}(n)$ to obtain $(\mathbb{G}, q, g)$.
			\item We choose $(x, y, z) \pick \mathbb{Z}^3_q$.
			\item We set $h_1 = g^{xy}$, $h_0 = g^{z}$ and $b \pick \bset$.
			We forward then $((\mathbb{G}, q, g), (g^x, g^y, h_{b}))$ to $\D$.
			\item $\D$ will then build $pk  = (\mathbb{G}, q, g, h_x \define g^x)$ and will forward $pk$ to $\A$.
			\item $\A$ will then output its challenge queries $m_0$, $m_1$.
			$\D$ will pick $b' \pick \bset$, and produce $c = (c_1, c_2) = (g^y, h_{b} + m_b \pmod p$.
			We use $r \define y$ and use $h_b$ as our value for $h_x^y=g^{xy}$.
			$\D$ sends $c$ to $\A$.
			\item $\A$ will output $b''$.
			\item $\D$ will output $ b''' \define 1 \Leftrightarrow b'' = b'$. We define $\DDH_{\D, \mathbb{G}}(n)=1$ iff $b'''=b$.
		\end{enumerate}
		We analyse the chance of success to determine $\DDH_{\D, \mathbb{G}}(n)$:
		\begin{itemize}
			\item if $b = 0$, then $\Pr[\DDH_{\D, \mathbb{G}}(n)=1] = \frac{1}{2}$: $\A$ is faced with random values, so he is not in the correct conditions.
			\item if $b = 1$, then $\Pr[\DDH_{\D, \mathbb{G}}(n)=1] = \frac{1}{2} + \negl(n)$: $\A$ operates correctly and its advantage is available.
		\end{itemize}
		Overall, we have
		\[ \Pr[\DDH_{\D, \mathbb{G}}(n)=1] = \frac12 + \frac{\negl(n)}{2} \]
		As we assume that $\DDH$ is hard in $QR_p$, then it means that this probability should have a negligible advantage, and so $\negl(n)$ should be negligible: thus, the scheme that was attacked by $\A$ is CPA-secure.
	\end{enumerate}
\end{solution}



\subsection{Exercise 5 (DDH PRG)}

Let $\mathbb{G}$ be a cyclic group of prime order $q$ generated by $g \in \mathbb{G}$.
Consider the following PRG defined over $(\Z^2-q, \mathbb{G}^3)$:
$G(\alpha, \beta) \define (g^\alpha, g^\beta, g^{\alpha\beta})$.
Define what it means for a PRG over $(\Z^2-q, \mathbb{G}^3)$ to be secure and show that $G$ is a secure PRG assuming $\DDH$ holds in $\mathbb{G}$.


\begin{solution}
	Taking the definition, $G\colon \Q_q^2 \mapsto \mathbb{G}^3$ is a secure PRG over $\mathbb{G}$ if $\forall n$, $\forall$ PPT distinguisher $\D$, $\exists$ negl. $\negl$ such that:
	\[ | \Pr[D(x, y, z)=1] - \Pr[D(G(\alpha, \beta))=1] | \le \negl(n) \]
	where $(x, y, z) \pick \mathbb{G}^3$ and $(\alpha, \beta) \pick \Z_q^2$, all randomly and uniformly and independent picked.

	The other property of expansion is trivially verified by the definition itself (as $|\Z_q|=|\mathbb{G}|=q$).

	We will proof the proposition by reduction.

	Assume that we have a PPT adversary $\D$ that can distinguish between the PRG $G$ as defined above and a random triplet of values with advantage $\negl$.
	That is,
	\[ | \Pr[\D(x, y, z)=1] - \Pr[\D(G(\alpha, \beta))=1] | \le \negl(n) \]
	Without loss of generality, we will assume that $\D$ outputs $1$ when he identifies that he is in front of the PRG.
	If that's not the case, simply reverse its output, and take $1-$ the probabilities above as the new probabilities of output, and the difference will be the same. Then, we can assume that
	\[ \Pr[\D(x, y, z)=1] = \Pr[\D \text{ outputs } 1 | \text{ 3 random values}] = k \]
	and
	\[ \Pr[\D(G(\alpha, \beta))=1] = \Pr[\D \text{ outputs } 1 | \text{in front of PRG}] = k + \negl(n) \]
	This is just a rewriting of the above condition. $k$ is some value between $0$ and $1$, and we don't care what it is: we're just interested in the difference between the probabilities.

	Then, let's build our PPT adversary $\A$ against $\DDH$ in $\mathbb{G}$. He will play the experiment $\DDH_{\A, \mathcal{G}}(n)$:
	\begin{enumerate}
		\item Run $\mathcal{G}(1^n)$ to generate $(\mathbb{G}, q, g)$ with $g$ generator of $\mathbb{G}$ or order $q$ and $|q|=n$.

		Choose $(x, y, z) \pick \Z_q^3$, $b \pick \bset$. Define $h_0=g^z$, $h_1=g^{xy}$.

		Send $(\mathbb{G}, q, g, h_x \define g^x, h_y \define g^y, h_b)$ to $\A$.

		\item Send to $\D(\mathbb{G}, q, g)$ the following ``challenge'': $w \define (h_x, h_y, h_b)$.

		If $b=0$, this is $(g^x, g^y, g^z)$, a triplet of pure random numbers.

		If $b=1$, this is $(g^x, g^y, g^{xy})$, the output of $G(x, y)$.

		\item $\D$ outputs $b'$. Then, $\A$ outputs $b''=b'$. Define $\DDH_{\A, \mathcal{G}}(n)=1$ iff $b''=b$.
	\end{enumerate}
	Then, we just have to compute:
	\begin{align*}
	\Pr[\DDH_{\A, \mathcal{G}}(n)=1] &= \Pr[b''=b] \\
	&= \Pr[b'=b] \\
	&= \Pr[b'=1 | b=1] \cdot \Pr[b=1] + \Pr[b'=0 | b=0] \cdot \Pr[b=0] \\
	&= (k + \negl(n)) \cdot \frac12 + (1-k) \cdot \frac12 \\
	&= \frac12 + \frac12 \negl(n) \le \frac12 + \negl'(n).
	\end{align*}
	As we assume that $\DDH$ is hard in $\mathbb{G}$, we assume that $\negl'$ is negligible, and so $\negl$ must be negligible too.
	This implies that $\D$ cannot distinguish between the PRG and random numbers efficiently, and thus that our PRG is secure.
\end{solution}



% TODO move
\subsection{Exercise X (A variant of ElGamal Encryption.)}
% This practical session was entirely corrected by the teaching assistants so, there is no need to correct the solutions.

Let us consider the following variant of ElGamal encryption.  Let
\begin{itemize}
	\item $\Gen$ output a pair
	$\langle pk, sk \rangle \define \langle(\Gr, q, g, h), (\Gr, q, g, x)
	\rangle$ as in traditional ElGamal encryption, except that $x$ is selected in $\Z_q - \{0\}$;
	\item $\Enc_{pk}(m) \define \langle m\cdot g^y, h^y\rangle$ with $y \leftarrow \Z_q$ and $m \in \Gr$.
\end{itemize}

\begin{enumerate}
	\item Define the corresponding decryption operation.
	\item Why did we exclude ``0'' from the set in which $x$ is selected?
	\item Prove that this variant of ElGamal is CPA-secure if the DDH
	problem is hard with respect to the group key generation algorithm $\Gr$.
\end{enumerate}


\begin{solution}
	\begin{enumerate}
		\item	$\D_{sk}(c_1, c_2) = \frac{c_1}{c_2^{1/x}}$, where $1/x$ is the inverse of $x \bmod q$.
		\item We exclude ``0'' as  $0 \not \in \Z_q^*$, so it has no inverse mod $q$.
		\item The reduction $A'$ proceeds as follows, using an attacker $A$ of
		the modified ElGamal scheme:
		\begin{enumerate}
			\item $A'$ starts an instance of $A$ and gets
			$\mathbb{G}, q, g, g^x, g^y, g^z$ from the DDH challenger (where $z$
			is or is not $xy$)
			\item $A'$ checks if $g^x = 1$ (that is, if $x=0$). If it is the
			case, it claims that it received a DDH tuple if $g^y = g^z$ (by
			returning 1) and a random tuple otherwise (by returning 0) and
			stops. The claim will be always correct, but this event will only
			happen with probability $1/q$.
			\item $A'$ forwards $(\mathbb{G},q, g, g^x)$ as ElGamal public key
			to $A'$.
			\item When $A'$ outputs two messages $m_0, m_1$, $A'$ flips a coin
			$b$ and returns $(m_b g^y, g^z)$.
			\item When $A$ outputs its guess $b'$, $A'$ claims that it received
			a DDH tuple if and only if $b = b'$, and returns ``1'' in this case.
		\end{enumerate}
		If $A$ runs in PPT, then so does $A'$ (its extra operations are
		clearly PPT).
		Let $X$ be the event that $A'$ outputs 1 when receiving a DDH tuple,
		and $Y$ be the event that $A'$ outputs 1 when receiving a random
		tuple.
		%Since DDH is hard, we know that $|\Pr[X] - \Pr[Y]|$ is a negligible quantity.
		Let us assume that $A$ wins the CPA
		game with probability $\frac 1 2 + \negl(n)$.

		We can see that
		$\Pr[X] = \Pr[X \wedge (x=0)] + \Pr[X \wedge (x\neq 0)] = \frac 1 q +
		\frac{q-1}{q}(\frac 1 2 + \negl(n))$.
		Indeed, when $x \neq 0$, $A$ exactly sees inputs that are
		distributed exactly as he expects them. Furthermore,
		$\Pr[Y] = \Pr[Y \wedge (x=0)] + \Pr[Y \wedge (x\neq 0)] = 0 +
		\frac{q-1}{q} \frac 1 2$.
		Indeed, when $x \neq 0$, the inputs of $A$ are independent of $b$, and
		therefore $A'$ wins with probability exactly $\frac 1 2$.

		As a result,
		$|\Pr[X] - \Pr[Y]| = \frac 1 q + \frac{q-1}{q}\negl(n) $. If $\negl(n)$ is non negligible (\textit{i.e.} this ElGamal variant is non CPA secure), then $|\Pr[X] - \Pr[Y]|$ is non negligible and DDH does not hold in $\Gr$. Reciprocally, if $\DDH$ holds in $\Gr$, this ElGamal variant is CPA secure.
	\end{enumerate}

\end{solution}
