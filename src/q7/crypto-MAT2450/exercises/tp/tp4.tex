\section{}
\subsection{Exercise 1 (Authenticated encryption, August 2019 exam)}

\begin{enumerate}
	\item Let $F\colon \K \times \bset^n \mapsto \bset^n$ be a PRF. Consider the following Authenticated Encryption scheme $\Pi=(\Gen, \Enc, \Dec)$ as follow:
	\begin{itemize}
		\item $\Gen$: pick a key uniformly at random in $\K$;
		\item $\Enc$: on input $m$ and $k$:
		\begin{itemize}
			\item Parse $m$ in $l$ blocks $m_1,\dots,m_l$ with $|m_1|=\dots=|m_l|=n$
			\item Pick $r$ uniformly at random in $\bset^{\frac{n}{2}}$
			\item For $i=1,\dots,l$:

				\hspace{1cm} $c_i=F_k(r||i) \oplus m_i$
			\item $c_{l+1}=F_k(r||l+1) \oplus \left(\oplus_{i=1}^l m_i\right)$
			\item Return $(r, c)$ with $c=(c_1,\dots,c_l,c_{l+1})$.
		\end{itemize}
		\item $\Dec$ consequently.
	\end{itemize}
	[For simplicity we suppose that for every message all blocks are \emph{full}, that is, when parsed the last block has length $n$ (i.e., $|m_l|=n$).

	When we write r||i we mean that the number $i$ is written in binary notation putting as many zeros on the left as necessary.]

	\begin{enumerate}
		\item Is $\Pi$ unforgeable? Prove or confute\footnote{*refute}.
		\item Is $\Pi$ CCA-secure? Prove or confute with an attack. [Hint: the previous answer may be useful\dots]
	\end{enumerate}

	\item Let $\Pi'=(\Gen', \Enc', \Dec')$ be an authenticated encryption scheme with binary messages of length $n$. For two binary vectors of length $n$ we denote $\oplus$ the coordinate-wise XOR. $1_n$ denotes the all-$1$ vector of length $n$. Consider the following schemes:
	\begin{itemize}
		\item $\Pi^1 \define (\Gen^1, \Enc^1, \Dec^1)$:
		\begin{itemize}
			\item $\Gen^1 \define \Gen'$,
			\item $\Enc^1_k(m) \define (c_1, c_2) = (\Enc'_k(m), \Enc'_k(m\oplus 1_n))$,
			\item $\Dec^1_k((c_1, c_2)) \define \Dec'_k(c_1)$ if $\Dec'_k(c_1)\oplus \Dec'_k(c_2)=1_n$, $\bot$ otherwise.
		\end{itemize}
		\item $\Pi^2 \define (\Gen^2, \Enc^2, \Dec^2)$:
		\begin{itemize}
			\item $\Gen^2 \define \Gen'$,
			\item $\Enc^2_k(m) \define \Enc'_k(m \oplus 1_n)$,
			\item $\Dec^2_k(c) \define 1_n$ if $\Dec'_k(c)=\bot$, $\Dec'_k(c)\oplus 1_n$ otherwise.
		\end{itemize}
	\end{itemize}
	\begin{enumerate}
		\item Is $\Pi^1$ an authenticated encryption scheme? If not, explain which property you can break and how.
		\item Is $\Pi^2$ an authenticated encryption scheme? If not, explain which property you can break and how.
	\end{enumerate}
\end{enumerate}

% TODO
\begin{solution}
	Note: in the following adversary descriptions, we skip the description of the unused phases of the security games.
	\begin{enumerate}
		\item Notice that there is a constraint on the size of $l+1$: $|l+1| \le \frac{n}{2} \implies l < 2^{n/2}-1$ and thus $|m|=n\cdot l < n\cdot (2^{n/2}-1)$.
		This constraint doesn't play a role in the proofs however.

		\begin{enumerate}
			\item Obviously, for two messages of the same length $m$ and $m'$, we have that \[\forall 1 \le i \le l\colon c'_i = m'_i \oplus F_k(r||i) = (m_i\oplus m'_i) \oplus m_i \oplus F_k(r||i) = (m_i\oplus m'_i) \oplus c_i.\]
			And,
			\begin{align*}
				c'_{l+1} &= F_k(r||l+1) \oplus (\oplus_{i=1}^l m'_i) = F_k(r||l+1) \oplus (\oplus_{i=1}^l m_i) \oplus \left((\oplus_{i=1}^l m_i) \oplus (\oplus_{i=1}^l m'_i)\right) \\
				&= c_{l+1} \oplus \oplus_{i=1}^l (m_i\oplus m'_i).
			\end{align*}
			So, we can build a forgery as follows:
			\begin{enumerate}
				\item Ask to the oracle for the encryption of a message, say, $m=0_n$ ($n$ times the bit $0$) so that it consists of only one block, and receive the answer $(r, c)$.
				\item Output $(r, c^*)$ with $c^* = c \oplus (0_n \oplus 1_n) = c \oplus 1_n = c \oplus m^*$, which is the encryption of message $m^*=1_n$.
			\end{enumerate}
			By construction, it is a forgery with probability 1. So, the scheme is not unforgeable.

			It is also possible to build a forgery by using a message such that $m_{l}=\oplus_{i=1}^{l-1} m_i$;
			then, $c_{l+1}=F_k(r||l+1) \oplus 0^n$, while $c_l=F_k(r||(l-1)+1)\oplus \oplus_{i=1}^{l-1} m_i$,
			and so we can send $(r, c^*)$ with $c^*=(c_1, c_2, \dots, c_{l})$: we drop the last $c_{l+1}$.

			\item In a similar manner, we can build an adversary winning against the CCA game (adversary's viewpoint):
			\begin{enumerate}
				\item The challenger-oracle picks $k \define \Gen(1^n) \pick \bset^n$ uniformly at random.
				\item Output the messages\footnote{We use the notation $m^0$ instead of $m_0$ to differenciate between a message and a message block.}
				$m^0=0_n$ and $m^1=1_n$, and get the challenge ciphertext $(r, c)=\Enc_k(m_b)=(r, c_1, c_2)=(r, F_k(r||1)\oplus m_b, F_k(r||2)\oplus m_b)$.
				\item Ask to the oracle the decryption of $(r, c^*)$ where $c^*=(c_1\oplus 0_{n-1}||1, c_2\oplus 0_{n-1}||1)$. We get its answer as $m^*$. This is of course not the same ciphertext as $c$.
				\item If $m^*=0_{n-1}||1$, output $0$, else ($m^*=1_{n-1}||0$), output $1$.
			\end{enumerate}
			By construction, in building $c^*$ we have constructed the encryption of $m_b\oplus 0_{n-1}||1$, so we flipped the last bit of the encrypted message, allowing us to decrypt it.
			So the probability $\Pr[\PrivKcca(n)=1]=1$, the adversary is PPT, and we have broken CCA security.

			Could we also break CPA security? No (90\% sure), and we can proof it by reduction (distinguisher of $g$ between PRF $F_k$ and random function $f$, based on adversary $A_\Pi$ against $\Pi$).
			Remark that, in the case that the ``PRF'' is a true random function, then each of the $g(r||i)$ are independent random values, and thus the $c_i$ are also independent random value, and so no relation can be found between them in a single message (the ciphertext is just purely random), and the only attack is to hope for a reused $r$.

			Note that the fact that $c_{l+1}$ uses $F_k(r||l+1)$ and not $F_k(r||l)$ is important, because otherwise there would be a reuse of argument, and we can build an attack that even breaks eavesdropper security.
		\end{enumerate}
		\item \begin{enumerate}
			\item $\Pi^1$ is not CCA-secure, and so is not an authenticated encryption scheme; our PPT adversary $\A$:
			\begin{enumerate}
				\item Key generation, as always.
				\item Output $m_0=0_n$ and $m_1=1_n$.

				Receive the challenge $c=(c_1, c_2)=(\Enc'_k(m_b), \Enc'_k(m_b\oplus 1_n))$.
				\item Ask for the decryption of $c'=(c_2, c_1)\neq c$ and receive $m'$. Observe that
				\begin{align*}
					c' &= (\Enc'_k(m_b\oplus 1_n), \Enc'_k(m_b)) = (\Enc'_k(m_b\oplus 1_n), \Enc'_k((m_b\oplus 1_n)\oplus 1_n)) \\
					&= \Enc^1_k(m_b\oplus 1_n) = \Enc^1_k(m_{1-b}).
				\end{align*}
				\item Output $0$ if $m'=m_1$, otherwise $1$ ($m'=m_0$).
			\end{enumerate}
			By construction, $\Pr[\PrivKcca[\A, \Pi^1](n)=1]=1$, which is a non-negligible advantage.

			The scheme is also forgeable, in the same way: simply ask for encryption of $m_0$, receive $c$, then build $c'$ as above: this is a valid ciphertext.

			\item $\Pi^2$ is forgeable, and so it not an authenticated encryption scheme; our PPT adversary $\A$:
			\begin{enumerate}
				\item Key generation, as usual.
				\item Output a random ciphertext $c$.
			\end{enumerate}
			There are two cases:
			\begin{itemize}
				\item either $c$ is the encryption of some message $m$ by $\Enc'_k(\cdot)$, and so $\Dec^2_k(c)=m\oplus 1_n$, and thus it is a valid ciphertext;
				\item or $c$ is not a valid ciphertext for $\Pi'$, in which case $\Dec'_k(c)=\bot$, and thus $\Dec_k(c)=1_n$, which means that $c$ is also a valid ciphertext for $\Pi^2$.
			\end{itemize}
			Thus, in both cases, this random $c$ is a valid ciphertext, and so $\Pr[\EncForge_{\A, \Pi^2}(n)=1]=1$: the scheme is forgeable.

			The fact that, when $\Dec'_k(c)=\bot$, we return $1_n$ instead of a more correct $\bot$, allows us to break the unforgeability.

			Is the scheme CCA-secure?
			Other than this issue, $\Enc^2$ is the same as $\Enc$, and the only exploitable change in behaviour between $\Pi'$ and $\Pi^2$ is the fact that, on $\Pi'$ invalid encryptions, $\Pi^2$ returns $1_n$.
			The only way an adversary could use this particularity would be if he asks for the decryption of an invalid ciphertext (generated at random by him, or by flipping a bit in a valid ciphertext), and he would always get the same $1_n$ answer, not very helpful.
			So the scheme is CCA-secure, and we can further prove it by doing a proof by reduction.
			% TODO do the proof by reduction; 95% sure it works
		\end{enumerate}
	\end{enumerate}
\end{solution}



% OK
% FIXME has been moved from TP5
% Note : this is from TP5 2018-2019, but TP4 2019-2020.
\subsection{Exercise 2 (Authenticated Encryption and sPRP)}

Consider the following scheme $\Pi=(\Gen,\Enc,\Dec)$ based on the strong pseudorandom permutation $\F \colon \K \times \lbrace 0,1 \rbrace^n$, defined as follow:
\begin{itemize}
	\item $\M=\bset^{\frac{n}{2}}$ (the message space)
	\item $\Gen$ picks a random key $k \in \K$
	\item $\Enc_k(m)$ picks a random value $r \in \bset^{\frac{n}{2}}$, and computes $c \define \F_k(m \| r)$
	\item $\Dec_k(c)$ computes $(m\|r)=\F^{-1}_k(c)$ and outputs $m$ (the first half).
\end{itemize}
Answers the following questions:
\begin{itemize}
	\item is $\Pi$ unforgeable?
	\item is $\Pi$ CCA-secure? (\emph{To do at home})
	\item is $\Pi$ an authenticated encryption scheme? (\emph{To do at home})
\end{itemize}

\paragraph{Definition 1} \label{def: sprp} (\emph{Strong PseudoRandom Permutation})

A function $\F \colon \K \times \M \mapsto \M$ is a $(q,t,\negl)$-\emph{ strong pseudorandom permutation} ($\sprp$) if for any $(q,t)$-bounded adversary, the advantage:
\[ \mathsf{Adv}^{\sprp}_{\adv}\define
\left| \Pr\left[ \adv^{\F_k(\cdot),\F_k^{-1}(\cdot)}\Rightarrow 1 \right] -
\Pr\left[ \adv^{\f(\cdot,\cdot),\f^{-1}(\cdot,\cdot)}\Rightarrow 1 \right] \right|
\leq \negl \]
with $k$  and $\f$ picked uniformly at random from their domains, respectively $\K$ and the set of permutations $\M \mapsto \M$.


\begin{solution}
	\begin{enumerate}
		\item It is not unforgeable because since $F_k$ is a PRP, it is bijective (i.e., every image has a pre-image). Then
		\[ \forall c \in \bset, \exists m, r \colon F_k^{-1}(c) = (m || r) \]
		and $\Pr[\EncForge_{\A, \Pi}(n)=1] = 1$.

		\item It is CCA-secure and we will prove it by reduction, assuming that the PRP $F$ is strong.
		Thus, assume we have a PPT adversary $\A$ against the scheme $\Pi$ with advantage $\negl_\A(n)$. Then, we can build a PPT distinguisher $\D$ between a sPRP and a random function, which plays the sPRP game as follows:
		\begin{enumerate}
			\item The challenger-oracle picks $k \define \Gen(1^n) \in \K$, and $b \pick \bset$. If $b=0$, then the challenger uses a true random function $f$ for $g$, if $b=1$, the challenger uses the sPRP $F_k(\cdot)$ for $g$.
			\item First query phase:

			When $\A$ asks for the encryption of a message $m$, $\D$ picks $r \pick \bset^{\frac{n}{2}}$ uniformly at random and queries its oracle on input $m || r$ obtaining, $c = g(m || r)$. $\D$ then forwards $c$ to $\A$.

			When $\A$ asks for the decryption of ciphertext $c$, $\D$ queries its oracle on input $c$ obtaining $m||r=g^{-1}(c)$, and $\D$ answers $m$ to $\A$.

			\item When $\A$ does the challenge query and outputs $m^*_0$, $m^*_1$ with $|m^*_0| = |m^*_1|=\frac{n}{2}$ and $m^*_0 \neq m^*_1$,
			$\D$ picks $r^* \pick \bset^{\frac{n}{2}}$ uniformly at random, and $b' \pick \bset$ uniformly at random too.
			$\D$ queries then its oracle on input $m^*_{b'} || r^*$ obtaining $c^*=g(m^*_{b'}||r^*)$. $\D$ forwards $c^*$ to $\A$.

			\item Second query phase, like the first one, except that $\A$ cannot ask for $\Dec_k(c^*)$.

			\item At the end of the game, $\A$ outputs its guess $b''$. $\D$ outputs $1$ iff $b''=b'$, $0$ otherwise: did the adversary $\A$ guess our correct pick of $b'$?.
		\end{enumerate}
		Let's compute the probability of success for $\D$:
		\begin{itemize}
			\item If $b=0$, then $g=f$, a random function, and so, each ciphertext generated by $g$ and each decrypted message generated by $g^{-1}$ is a random number, independent of each other. In this condition, the best thing the adversary $\A$ can do is wait for a collision on $m||r$ and thus $\Pr[\D \text{ outputs } 1 | b=0]=\Pr[b''=b']=\Pr[\PrivKcca(n)=1]=\frac12 + \frac{q(n)}{2^{n/2}}$.
			\item If $b=1$, then $g=F_k$, and $\A$ is in the right conditions (its interface is respected) to have the advantage $\negl_\A$: $\Pr[\D \text{ outputs } 1 | b=1]=\Pr[b''=b']=\Pr[\PrivKcca(n)=1]=\frac12 + \negl_\A$.
		\end{itemize}
		Thus, the difference between the probabilities is
		\[ \negl_\D(n) = \abs{ \Pr[\D^{F_k(\cdot), F^{-1}_k(\cdot)}(1^n)=1] - \Pr[\D^{f(\cdot), f^{-1}(\cdot)}(1^n)=1] } = \abs{ \negl_\A(n) - \frac{q(n)}{2^{n/2}} } \]
		As we assume that $\F$ is a strong PRP, that the whole construction above is PPT, then $\negl_\D(n)$ must be negligible, and thus the right-hand side must be negligible, and thus $\negl_\A$ must be negligible. The scheme is thus CCA-secure.

		\item Since the scheme $\Pi$ is CCA-Secure but not unforgeable, it is not an authenticated encryption
	\end{enumerate}
\end{solution}



\subsection{Exercise 3 (Hash functions from\ldots hash functions)}

Let $H_2\colon\bset^{2l}\mapsto\bset^{l}$ and $H_3\colon\bset^{3l}\mapsto\bset^{l}$ be
collision resistant hash functions. For $2l$-bit strings $x_i$'s, consider the following two constructions.
\begin{itemize}
	\item $H_4\colon\bset^{4l}\mapsto\bset^{l}$;
	$x=x_1||x_2\rightarrow H_2\left(H_2(x_1)||H_2(x_1\oplus x_2)\right)$
	\smallskip
	\item $H_6\colon\bset^{6l}\mapsto\bset^{l}$;
	$x=x_1||x_2||x_3\rightarrow H_3\left(H_2(x_1\oplus x_2)||H_2(x_2\oplus x_3)||H_2(x_3\oplus x_1)\right)$
\end{itemize}
Determine whether these hash functions are still collision resistant or not.


\begin{solution}
	\begin{itemize}
		\item
		Let's show that from a collision of $H_4$, we generate a collision for $H_2$
		which prove that $H_4$ is collision resistant since $H_2$ is so.
		Let's suppose that we have $x_1\|x_2 \neq y_1\|y_2$ are such that $H_4(x_1\|x_2) = H_4(y_1\|y_2)$.
		\begin{itemize}
			\item
			If $H_2(x_1) \| H_2(x_1 \xor x_2) \neq H_2(y_1) \| H_2(y_1 \xor y_2)$,
			we have a collision for $H_2$ since their image by $H_2$ is identical.
			\item
			If $H_2(x_1) \| H_2(x_1 \xor x_2) = H_2(y_1) \| H_2(y_1 \xor y_2)$,
			we have $H_2(x_1) = H_2(y_1)$ \emph{and} $H_2(x_1 \xor x_2) = H_2(y_1 \xor y_2)$.
			\begin{itemize}
				\item If $x_1 \neq y_1$, we have a collision for $x_2$ since $H_2(x_1) = H_2(y_1)$.
				\item If $x_1 = y_1$, then $x_2 \neq y_2$ since $x_1\|x_2 \neq y_1\|y_2$.
				Therefore $x_1 \xor x_2 \neq y_1 \xor y_2$ and we have collision on $H_2$.
			\end{itemize}
		\end{itemize}
		\item
		$H_6$ is not collision resistant since $H_6(x_1\|x_2\|x_3) = H_6((x_1 \xor w)\|(x_2 \xor w)\|(x_3 \xor w))$
		for all $w$ (collision if $w\neq 0^{2l}$).
		Indeed, since $\xor$ is associative and commutative,
		\begin{align*}
		& = H_6((x_1 \xor w)\|(x_2 \xor w)\|(x_3 \xor w))\\
		& = H_3(H_2((x_1 \xor w) \xor (x_2 \xor w))\|H_2((x_2 \xor w) \xor (x_3 \xor w))\|H_2((x_3 \xor w) \xor (x_1 \xor w)))\\
		& = H_3(H_2(x_1 \xor (w \xor w) \xor x_2)\|H_2(x_2 \xor (w \xor w) \xor x_3)\|H_2(x_3 \xor (w \xor w) \xor x_1)))\\
		& = H_3(H_2(x_1 \xor x_2)\|H_2(x_2 \xor x_3)\|H_2(x_3 \xor x_1)))\\
		& = H_6(x_1\|x_2\|x_3).
		\end{align*}
	\end{itemize}
\end{solution}



\subsection{Exercise 4 (Block-cipher based hash function)}

Considering a block cipher
$E\colon\K\times\M\mapsto\C$; $(k,m)\rightarrow E(k,m)=\Enc_k(m)$
with $\K=\M=\C=\bset^l$, one may try to construct
a collision resistant compression function from $\bset^{2l}$ to $\bset^{l}$.
Show that the following methods do not work :
\[ f_1(x,y)=E(y,x)\oplus y \quad\text{ and }\quad f_2(x,y)=E(x,x)\oplus y \]
That is, show an efficient algorithm for constructing collisions for $f_1$ and $f_2$.
Recall that the block cipher $E$ and the corresponding decryption algorithm $D$ are both
known to you (and they are bijective functions).


\begin{solution}
	We will give 2 collisions for $f_1$ and $f_2$.

	We know that $E(k, \cdot)$ is surjective because it must be injective and $\M = \C$.
	Therefore the decryption exists for all $c \in \C$! We will use it for the second collision of $f_1$.

	For $f_1$, we have the 2 following collisions
	\begin{align*}
	f_1(D(E(y,x),y), E(y,x))
	& = E(E(y,x), D(E(y,x), y)) \xor E(y,x)\\
	& = y \xor E(y,x)\\
	& = E(y,x) \xor y\\
	& = f_1(x, y)\\
	f_1(D(0, E(y,x) \xor y), 0)
	& = E(0, D(0, E(y, x) \xor y)) \xor 0\\
	& = E(y, x) \xor y\\
	& = f_1(x, y).
	\end{align*}
	and for $f_2$ we have
	\begin{align*}
	f_2(x, E(x,x)) & = E(x,x) \xor E(x,x)\\
	& = 0 & \forall x \in \{0,1\}^l\\
	f_2(y, E(x,x)) & = E(y,y) \xor E(x,x)\\
	& = E(x,x) \xor E(y,y)\\
	& = f_2(x, E(y,y)).
	\end{align*}

	\textbf{Other approach:}

	For $f_1$: Let's define $k_1$ and $k_2$ such that $k_2$ is equal to $k_1$ except for the last bit which is flipped. Let's now take an arbitrary $x_1$ for which we ask the encryption $T_{x_1} = E(k_1,x_1)$. Now let's flip the last bit of $T_{x_1}$ and call the result $T_{x_2}$. We can now ask for the decryption of $T_{x_2}$ given $k_2$ as input key, which we know exists since D is bijective and $\C = \bset^{l}$. We then obtain $x_2$. We can now observe that:
	\begin{align*}
	f_1(x_1, k_1) & = E(k_1,x_1) \xor k_1\\
	& = T_{x_1}  \xor k_1\\
	f_1(x_2, k_2) & = E(k_2,x_2) \xor k_2\\
	& = T_{x_2} \xor k_2\\
	& = (T_{x_1} \xor 0^{l-1}||1 )\:\xor\: (k_1 \xor 0^{l-1}||1)\\
	& = T_{x_1}  \xor k_1
	\end{align*}
	For $f_2$: We can ask for $T_x = \:E(x,x)$ and $T_y = E(y,y)$ for two arbitrary (but different) $x$ and $y$.  We can see that:
	\begin{align*}
	f_2(y, T_x) & = E(y,y) \xor T_x\\
	& = T_y  \xor T_x\\
	f_2(x, T_y) & = E(x,x) \xor T_y\\
	& = T_x \xor T_y
	\end{align*}
	Which are both equal with different input, this is a collision.
\end{solution}



% OK
\subsection{Exercise 5 (Authenticated encryption, or not)}

\copypaste{10}{1}



\subsection{Exercise 6 (Variable-length MAC)}

Considering a known hash function $h^s\colon\bset^{2l}\mapsto\bset^{l}$,
let's note by $H^s$ the corresponding Merkle-Damg{\aa}rd transform hash function,
\emph{i.e.}

\begin{center}
	\begin{tikzpicture}[scale=0.5]
	\tikzstyle{every node}=[text centered, inner sep = 2pt]

	\trapeze{$h^s$}{\position}
	\draw [->] \position +(-3,0) node [left] {$IV$} -- +(-1,0);
	\draw [<-] \position + (-1,1) -| ++(-2,3) node [above] {\mylabel};
	\draw \position + (1,0) -- +(2,0) node {};
	\renewcommand{\position}{(4,0)}
	\renewcommand{\mylabel}{$x_2$}
	\trapeze{$h^s$}{\position}
	\draw [->] \position +(-2,0) node [below] {} -- +(-1,0);
	\draw [<-] \position + (-1,1) -| ++(-2,3) node [above] {\mylabel};
	\draw [->]\position + (1,0) -- +(2,0) node [right] {$\ldots$};
	\renewcommand{\position}{(9.5,0)}
	\renewcommand{\mylabel}{$x_n$}
	\trapeze{$h^s$}{\position}
	\draw [->] \position +(-2.2,0) node [below] {} -- +(-1,0);
	\draw [<-] \position + (-1,1) -| ++(-2,3) node [above] {\mylabel};
	\draw \position + (1,0) -- +(2,0) node {};
	\renewcommand{\position}{(13.5,0)}
	\renewcommand{\mylabel}{$\left|x\right|$}
	\trapeze{$h^s$}{\position}
	\draw [->] \position +(-2,0) node [below] {} -- +(-1,0);
	\draw [<-] \position + (-1,1) -| ++(-2,2.95) node [above] {\mylabel};
	\draw [->] \position + (1,0) -- +(2,0) node [right] {$H^s(x)$};
	\end{tikzpicture}
\end{center}
%
when $x=x_1||\cdots||x_n$ for some integer $n$ and when the $x_i$'s are
$l$-bit strings.

Show why, with a private key $k$ of length $l$, the MAC scheme
\[t \define H^s(k||m),\]
is \emph{not} existentially unforgeable under an adaptive chosen-message attack.


\begin{solution}
	If we have the tag of $p$, which is (let's consider that $k$ and $p$ are $l$ bits long for simplicity)
	\[ t_p = H^s(k\|p) = h^s(h^s(h^s(IV \| k) \| p) \| 2l) \]
	we can find the tag of $p\|2l\|w$ (where $w$ is $l$ bits long for simplicity)
	without knowing $k$ since we know $h^s$ (it is public knowledge, only the secret $k$ is secret and requires an oracle).
	It is
	\begin{align*}
		H^s(k\|p\|2l\|w)
		& = h^s(h^s(h^s(h^s(h^s(IV \| k) \| p) \| 2l) \| w) \| 4l)\\
		& = h^s(h^s(H^s(k \| p) \| w) \| 4l)\\
		& = h^s(h^s(t_p \| w) \| 4l)
	\end{align*}
	Since $p\|2l\|w \neq p$, this gives us an existential forgery.
\end{solution}



% OK
\subsection{Exercise 7 (Hash-MAC)}

Suppose $H_0$ and $H_1$ are compression functions but only one is believed to be collision resistant.
Besides, suppose $\textsc{Mac}_0$ and $\textsc{Mac}_1$ are message authentication codes but only one
of the both schemes is known to be unforgeable. Is it possible to build a secure ''hash-MAC'' from these
inputs? Justify your answer.

%Homework 2 of Dan Boneh, Winter 2011, Problem 5 (DVD security)
\begin{solution}
	We build $H(m) = H_0(m)\|H_1(m)$.
	If we have $m_1 \neq m_2$ such that $H(m_1) = H(m_2)$ then $H_0(m_1) = H_0(m_2)$ and
	$H_1(m_1) = H_1(m_2)$ so the collision resistant hash function has a collision whichever it is.
	However, $H$ is no more a compression function and we cannot use Merkle-Damg\aa{}rd.

	The fact that we don't know the compressive factor also prevents us from building (simply) Merkle-Damg\aa{}rd or sponge constructions based on each of the functions.

	The input of $H$ therefore cannot have arbitrary length but its output is twice the length of the output of $H_0$ and $H_1$ so it is twice the size of a tag.

	The output of $\Mac_0$ and $\Mac_1$ are the size of a tag so we can use the tag
	$H(\Mac_0(k,m)\|\Mac_1(k,m))$ for our Hash-MAC scheme.
	If we are able to output an existential forgery $(m, t)$, since $H$ is collision resistant, that means that we have found $\Mac_0(k,m)\|\Mac_1(k,m)$ and therefore we have found an existential forgery for both $\Mac_0$ \emph{and} $\Mac_1$ which is absurd since one of them is believed to be unforgeable.

	Our Hash-MAC scheme is therefore unforgeable.
\end{solution}



\subsection{Exercise 8 (Blue-ray security)}

\copypaste{3}{7}
