\section{}
\subsection{Exercise 1 (Blue-ray security)}
\copypaste{2}{7}

\subsection{Exercise 2 (Authenticated encryption)}
Let $\Pi = \langle \Gen, \Enc, \Dec\rangle$ be an authenticated encryption
scheme such that $\Enc$ encrypts messages of $n$ bits.
%
Do the following systems provide authenticated encryption?  For those
that do, briefly explain why.  For those that do not, present an
attack that breaks one of the security properties of an authenticated
encryption scheme.

\begin{enumerate}
	\item $\Pi' = \langle \Gen, \Enc', \Dec'\rangle$ with
	$\Enc'_k(m) = (\Enc_k(m), \Enc_k(m \oplus (0^{n-1}\|1)))$ and
	$\Dec'_k(c_1, c_2) = \Dec_k(c_1)$ if
	$\Dec_k(c_1) \oplus \Dec_k(c_2) = 0^{n-1}\|1$ and $\bot$ otherwise.

	\item $\Pi' = \langle \Gen, \Enc', \Dec'\rangle$ with
	$\Enc'_k(m) = (\Enc_k(m), \Mac_k(m))$ and $\Dec'_k(c_1, c_2) = \Dec_k(c_1)$
	
	if $\Vrfy_k(\Dec_k(c_1), c_2)=1$ and $\bot$ otherwise. Here, $\Mac$
	and $\Vrfy$ are deterministic algorithms that are part of a secure
	MAC scheme that is compatible with $\Gen$.

\end{enumerate}

\begin{solution}
$\Pi := \langle \text{Gen, Enc, Dec} \rangle$ is an authenticated encryption scheme (AE) if it is CCA-secure and unforgeable.
\begin{enumerate}
    \item The system $\Pi'$ is not AE because it is \textbf{forgeable} and we can show it with this example. If the adversary A asks for the message $m$ ($m'$ corresponds to the message m with the last bit changed) to the oracle access, he will receive the cipher text $(c_1, c_2)$, where $c_1 = Enc_k(m)$ and $c_2 = Enc_k(m\xor 0^{n-1}||1) = Enc_k(m')$ . 
    
    If $\A$ outputs the pair (m',$(c_2, c_1)$), this is a forgery. 
    
    $Dec_k'(c_2,c_1) = Dec_k(c_2) = Dec_k(Enc_k(m')) = m' \neq \perp $ because $Dec_k(c_2) \oplus Dec_k(c_1) = m' \oplus m = m \oplus 0^{n-1}||1 \oplus m = 0^{n-1}||1 $. And $m'$ has not been requested before.
    
    Then we have EncForge$_{A, \Pi'}$(n) = 1  and Pr[EncForge$_{A, \Pi'}$(n)] = 1. $\Pi'$ is then forgeable and it is not an AE. 
    \newline
    With the same technique, an adversary can break the CCA-security of this scheme by querying two different messages $m_1$ and $m_2$, obtaining their encryption, sending \newline
    ($m'_1,m'_2$) = ($m_1 \oplus 0^{n-1}||1, m_2 \oplus 0^{n-1}||1$) for the challenge, and compare the encryption of $m'_b$ with the two previously received ciphertexts.
    \item The sytem $\Pi'$ is not AE because it is not \textbf{CCA-secure} and we can show it because $Mac_k(m)$ does not assure any security (only authentication). So if We use as Mac : 
    $$ Mac_k(m) = m||Mac'_k(m) $$
    It is a good mac but it is trivial to show that is does not hold $CCA-Secure$. $\Pi'$ is then not an AE.
\end{enumerate}
\end{solution}

\subsection{Exercise 3 (Euclidean algorithm for gcd)}
Let $a,b \in \mathbb{Z}$ , $b \neq 0$, consider the following algorithm, presented in Algorithm~\ref{algo:gcd}. ($r=a \% b$ means that $a=qb+r$ where $q$ is the quotient and $r$ is the remainder).

Prove that $x$, the value returned by Algorithm~\ref{algo:gcd}, is $\mathsf{gcd} (a,b)$.

Hint:
\begin{itemize}
\item Prove that $x$ divides  $ \mathsf{gcd} (a,b)$
\item Prove that $ \mathsf{gcd} (a,b)$ divides $x$
\end{itemize}

\begin{algorithm}
	\KwIn{$a$, $b$}
	\KwOut{$\mathsf{gcd}(a,b)$}
	
	\While{ $b\neq 0$}
	{
		$r \leftarrow a\%b$\;
		
		$a \leftarrow b$\;	
		
		$b \leftarrow r$\;	
	}
	\Return($a$)
	
\caption{The Euclidean $\mathsf{gcd}$ algorithm.}\label{algo:gcd}
\end{algorithm}
\begin{solution}
According to the algorithm, we will have as successive value for the different remainder : 
$$ (r_2 = r_0 \% r_1, r_3 = r_1 \% r_2, \ r_4 = r_2 \% r_3, \ ... \ , r_n =  r_{n-2} \% r_{n-1})$$
Where $r_0 = a$, $r_1 = b$ and $r_n$ is the last non null remainder. Then we have the property that :
$$ gcd(r_{i}, r_{i+1}) = gcd(r_{i+1}, r_{i+2}) \ \forall i : \ 0 \leq i \leq n - 2 $$
Otherwise if it was not the case, $\exists i < n $ such that $r_i = 0$. But as $r_n$ is the last non null remainder, we prove by contradiction this property.

As $gcd(r_{n-1}, r_n) = r_n $ because $r_n | r_{n-1}$ (since $r_{n+1} = r_{n-1} \% r_n = 0$), we can conclude that 
$$ gcd(r_0, r_1) = gcd(a, b) = gcd(r_{n-2}, r_{n-1}) = r_n$$
We have proved the value returned by the algorithm is the $gcd(a,b)$
\end{solution}

\subsection{Exercise 4}
Consider the group $\mathbb{Z}^{\ast}_{17}$.
\begin{enumerate}
\item Compute $5^{-1}$.
\item Compute $3^2$, $3^3$ and $3^4$.
\item Does $3$ generate the group?
\item Find $\log_{7}(11)$.
\end{enumerate}
\begin{solution}
Here because p is not too big, it is possible to evaluate "quickly" and "intuitively" the solutions. If it is too hard, there is an algorithm in the slides. 
\begin{enumerate}
    \item Because 35 mod 17 = 1, and $5 \cdot 7 = 35$. \newline Then $5^{-1} = 7$.  ($5 \cdot 7 = 1 \text{ (mod 17)}$)
    \item \begin{itemize}
        \item $3^2 = 9 \text{ (mod 17)}$
        \item $3^3 = 3^2 \cdot 3 = 27 \text{ (mod 17)} = 10 \text{ (mod 17)}$
        \item $3^3 = 3^3 \cdot 3 = 30 \text{ (mod 17)} = 13 \text{ (mod 17)}$
    \end{itemize}
    \item According to \textit{Fermat's little theorem}, if ord(g) = i then if $i|m = |G|$, where G is the commutative group. To see if 3 generate the group, we have to check if $3^i \neq 1$ where $i$ are the divisor of $(p-1) = 16$ (except 16 of course !).
    \begin{itemize}
        \item $3^1$ = 3 mod 17 (trivial)
        \item $3^2$ = 9 mod 17 (evaluated previously)
        \item $3^4$ = 13 mod 17 (evaluated previously)
        \item $3^8$ = $(3^4)^2$ = $(13)^2 \text{ (mod 17)}$ = $(-4)^2 \text{ (mod 17)}$ = 16 mod 17
    \end{itemize}
    We can see that 3 is a generator of the group. 
    
    \textbf{P.S.} : The trick here is to remember the property of the modulo operation here in a $Z^*_p$:
    $$ x = -(p - x) \text{ (mod p)} $$  
    It can make a lot of computing easier (it can become a real pain in the ass).
    \item Here we have to find x such that : 
    $$ 7^x = 11 \text{ (mod 17)} $$
    After (boring) computations, we have here :
    \begin{itemize}
        \item $7^1$ = 7 mod 17
        \item $7^2$ = 15 mod 17 = -2 mod 17
        \item $7^3$ = -14 mod 17 = 3 mod 17
        \item $7^4$ = 4 mod 17
        \item $7^5$ = 11 mod 17 (Bingo)
    \end{itemize}
    Then $log_7(11)$ = 5
\end{enumerate}
\end{solution}

\subsection{Exercise 5 (Group order)}
In this exercise we consider the group $\mathbb{Z}_{59}^*$.

\begin{enumerate}
	\item What is the order of $58$?
	\item What are the possible orders of an element of this group?
	\item Find an element of order more than $20$.
	
\end{enumerate}
\begin{solution}
The order of $g \in \mathbb{Z}^*_{59}$ is the smallest $i$ where $g^i = 1$ 
\begin{enumerate}
    \item ord(58) = 2 because :
    \begin{itemize}
        \item  $58^1$ = 58 mod 59 = -1 mod 59
        \item  $58^2$ = -58 mod 59 = 1 mod 59
    \end{itemize}
    \item According to \textit{Fermat's little theorem}, the possible orders of a group $\mathbb{Z}^*_{p}$ are the divisor of p-1. Then, the possible orders are : 
    \begin{itemize}
        \item 1
        \item 2
        \item 29
        \item 58
    \end{itemize}
    \item The best strategy here is to find a number g where 
    $$ ord(g) > 2 $$
    (To assure you this is correct, just look at the possible ordrers).  \newline
    2 is a correct candidate. 
\end{enumerate}
\end{solution}

\subsection{Exercise 6 (Decisional Diffie-Hellman and \texorpdfstring{$\mathbb{Z}_p^\ast$}{Zp*})}
\label{subsec:4.6}
The goal of this exercise is to show that in some groups DDH and CDH assumptions are conjectured not equivalent, as DDH is easy whereas CDH is conjectured to be hard.

\begin{enumerate}
	\item For all element $a$ of $\mathbb{Z}_{11}^*$, compute $a^2 \mod 11$.
	
	For a prime number $p$, we denote $QR_p$ the set $\{x \in \mathbb{Z}_{p}^* \; | \; \exists a\in \mathbb{Z}_{p}^*, a^2=x\}$, such $x$ are called quadratic residues modulo $p$. Show that if $p$ is odd then $|QR_p|=\frac{p-1}{2}$.
	
		\item For all element $a$ of $\mathbb{Z}_{11}^*$, compute $a^5 \mod 11$. Show that for any odd prime $p$, $x \in QR_p \Leftrightarrow x^{\frac{p-1}{2}}= 1 \mod p$, and that $x \not \in QR_p \Leftrightarrow x^{\frac{p-1}{2}}= -1 \mod p$.
	
	\item Show that $2$ is a generator of $\mathbb{Z}_{11}^*$. For the following pairs $(a,b)$, compute $g^a, g^b$ and $g^{ab}$ in $\mathbb{Z}_{11}^*$ where $g=2$:
	\begin{itemize}
		\item $(2,8)$,
		\item $(1,4)$,
		\item $(3,5)$.
	\end{itemize}
    Show that for $p$ an odd prime, $g^{ab} \not \in QR_p \Leftrightarrow g^a \not \in QR_p \text{ and } g^b \not \in QR_p$.
	
	\item Show that DDH does not hold in $\mathbb{Z}_{p}^*$ with $p$ an odd prime.
\end{enumerate}
\begin{solution}
For this exercise we will work with $\Z_{11}^* = \{1,2,3,4,5,6,7,8,9,10\}$
\begin{enumerate}
    \item 
		For all element $a$ of $\Z_{11}^*$, I've calculated $a^2$ mod $11$.
		$$1^2 = 1 \quad 2^2 = 4 \quad 3^2 = 9 \quad 4^2 = 5 \quad 5^2 = 3 \quad 6^2 = 3 \quad 7^2 = 5 \quad 8^2 = 9 \quad 9^2 = 4 \quad 10^2 = 1$$
		We see that with $p$ odd, we have $\left|QR_p\right| = \frac{p-1}{2}$. We can show it with this development:
		
	\item
		For all element $a$ of $\Z_{11}^*$, I've calculated $a^5$ mod $11$.
		$$1^5 = 3^5 = 4^5 = 5^5 = 9^5 = 1 \qquad 2^5 = 6^5 = 7^5 = 8^5 = 10^5 = 10$$
		We can see that for $p$ prime, we have $x \in QR_p \Leftrightarrow x^{\frac{p-1}{2}} = 1 \mod p$ and  $x \notin QR_p \Leftrightarrow x^{\frac{p-1}{2}} = p-1 \mod p$.
		\begin{itemize}
			\item $x \in QR_p \Leftrightarrow x^{\frac{p-1}{2}} = 1 \mod p$:\\
			We know that $$x \in QR_p \Leftrightarrow \exists a \st x = a^2 \mod p$$
			So we have now $$x = a^2 \mod p \Leftrightarrow x^{\frac{p-1}{2}} = 1 \mod p$$
			If we replace $x$ by $a$ we obtain $a^{2^{(\frac{p-1}{2})}} = 1 \mod p$.\\ 
			But also more simply $a^{p-1} = 1 \mod p$ which is true by the group theory.
			\item $x \notin QR_p \Leftrightarrow x^{\frac{p-1}{2}} = p-1 \mod p$:\\
			We know that $$x \notin QR_p \Leftrightarrow \exists a \st x = a^{1+2n} \mod p$$
			So we have now $$x = a^{1+2n} \mod p \Leftrightarrow x^{\frac{p-1}{2}} = -1 \mod p$$
			We replace $x$ by $a$ and we get $$a^{\frac{p-1}{2}} a^{n(p-1)} \mod p = -1 \mod p$$ 
			We know that $ a^{n(p-1)} \mod p = 1$, thus we simplify the equation like $$a^{\frac{p-1}{2}} \mod p = -1 \mod p$$ 
			We know that $g = a^{\frac{p-1}{2}} \mod p \ne 1 \mod p$ but $g^2 = a^{p-1} = 1 \mod p$. The only solution of these two equations is $g = -1 \mod p$ which is equivalent to $$x^{\frac{p-1}{2}} = p-1 \mod p$$
		\end{itemize}
	
	\item
		The number $2$ is a generator of $\Z_{11}^*$, because ord($2$) $= 10$. In fact, we have $2^1 = 2$, $2^2 = 4$, $2^5 = 10$ and $2^10 = 1$. (Fermat's little theorem)
		We have $g = 2$ so:
		\begin{itemize}
			\item $(2,8):\quad g^2 = 4$, $g^8 = 3$ and $g^{16} = -2$ 
			\item $(1,4):\quad g^1 = 2$, $g^4 = 5$ and $g^{4} = 5$
			\item $(3,5):\quad g^3 = -3$, $g^5 = -1$ and $g^{15} = -1$ TODO
		\end{itemize}
		We have to show that $g^{ab} \notin QR_p \Leftrightarrow g^a \notin QR_p \text{ and } g^b \notin QR_p$.\\
		We know by the definition of the $QR_p$ set that
		$$g^n \notin QR_p \Leftrightarrow \exists m \st n = 2m+1$$
		We can thus extract from $g^{ab} \notin QR_p$ that $\exists m \st ab = 2m+1$.
	    
	    % not necessary
		%We can do a proof by contradiction:\newline 
		%If $a = 2v$ then $ab = 2bv$ and can not be equal to $2m + 1$.\\
		%If $b = 2v$ then $ab = 2av$ and can not be equal to $2m + 1$.\\
		
		So we are assured that $a$ and $b$ are not pairs, so we have the relation $g^a \notin QR_p \text{ and } g^b \notin QR_p$ if and only if $ab = 2m+1$ which is equivalent to $g^{ab} \notin QR_p$. That was what we had to proof.
		
	\item
		We have to show that DDH does not hold in $\Z_p^*$ with p an odd prime number.
		
		We define an attacker that can see $p$, $g$, $g^a$, $g^b$ and receive $h_b = g^{ab}$ or $g^z$.\\ 
		The attitude of the attacker will be this one:
		\begin{itemize}
			\item It receives $g^a \notin QR_p$ and $g^b \notin QR_p$:\\
				It will answer in function of $h_b$:
				\begin{itemize}
					\item $h_b \notin QR_p$:\\
						It answers $h_b = g^{ab}$
					\item $h_b \in QR_p$:\\
						It answers $h_b = g^z$
				\end{itemize}
			\item It receives $g^a \in QR_p$ or $g^b \in QR_p$:\\
				It answers randomly.
		\end{itemize}
		We can identify four cases with their chances of success and appearance (we already know that $\left|QR_p\right|$ is of size $\frac{p-1}{2}$):
		\begin{enumerate}[a)]
			\item $g^a \in QR_p$ or $g^b \in QR_p$ appears $3/4$ of the time with success = $1/2$.
			\item $g^a \notin QR_p$ and $g^b \notin QR_p$ with $h_b = g^{ab}$  appears $1/8$ of the time with success = $1$.
			\item $g^a \notin QR_p$ and $g^b \notin QR_p$ with $h_b = g^z$ and $g^z \in QR_p$  appears $1/16$ of the time with success = $1$.
			\item $g^a \notin QR_p$ and $g^b \notin QR_p$ with $h_b = g^z$ and $g^z \notin QR_p$  appears $1/16$ of the time with success = $0$.
		\end{enumerate}
		We can now recalculate the expected value of success of our attacker:
		$$\begin{array}{rcl}
			\mathbb{E}(success) &=& \frac{3}{4}\cdot \frac{1}{2} +  \frac{1}{8}\cdot 1 + \frac{1}{16}\cdot 1 + \frac{1}{16}\cdot 0\\
								&=& \frac{1}{2} + \frac{1}{16}
		\end{array}$$
		This attacker has one sixteenth of probability more than one half which is not a negligible function (in fact it is a constant function). It is not DDH secure.
\end{enumerate}
\end{solution}