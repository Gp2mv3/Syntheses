
\section{}

% OK
\subsection{Exercise 1 (Zero knowledge Petersen)}

We work in a group $\mathbb{G}$ of prime order $q$ with generator $g$.
The Schnorr protocol, used to prove the knowledge of discrete
logarithm, is (honest-verifier) zero-knowledge. However, the value
$y=g^x \pmod{p}$ (for a safe prime $p=2q+1$) leaks some information
about the discrete logarithm $x$ (since for a given generator $g$ of
order $q$ there is exactly one such $x$ in $\Z_q$). On the
other hand, the Pedersen commitment is perfectly hiding and thus does
not reveal information about the committed value. The following
protocol attempts to merge the both properties i.e., to prove the
knowledge of a commited value under the Pedersen commitment scheme in
a zero-knowledge manner.

\emph{The protocol.} The public inputs of the proof are the prime $p$,
the Pedersen public key $(g, h)$, a security parameter $k$ and a
(hypothetic) commitment $c\in QR(p)$. The prover's private intputs are
$x$ and $r$ in $\mathbb{Z}_q$ s.t. $c=g^xh^r$ (mod $p$). The protocol executes as follows.
\begin{itemize}
	\item The prover randomly chooses $y,s \in_R \Z_q$ and sends $d=g^yh^s  \pmod{p}$ to the verifier.
	\item The verifier randomly chooses $e\in_R \bset^k$ and sends it to the prover.
	\item The prover computes $z=y-ex$ and  $t=s-er$ modulo $q$ and sends it to the verifier.
	\item The verifier accepts the proof iff $d = c^e g^zh^t
	\pmod{p}$.
\end{itemize}
If the verifier accepts the proof, we say that the conversation $\langle d,e,(z,t) \rangle$ is valid.
%
\begin{enumerate}
	\item Prove the correctness property of this construction.

	\item Assume that an adversary is able to produce two valid responses for two distinct challenges, under the same commit.
	How can you use this faculty to extract an opening of $c$?
	Discuss the soundness property of the protocol.

	\item Assume you are able to ``rewind'' an adversarial prover
	who tries to build a valid conversation. How can you use
	this faculty to extract an opening of $c$. Which property
	did you break ? Briefly discuss the soundness property of
	the protocol.

	\item Show how a valid conversation $\langle
	d,e,(z,t) \rangle$ can be simulated from $c$, without the use of any
	private inputs. (Assume that the valid conversation involves
	honest parties.)

	\item Generalize the process to prove the knowledge of an opening to
	a multi-Pedersen commitment as in exercise 3.
\end{enumerate}


\begin{solution}
	\begin{enumerate}
		\item The construction is correct if $Pr[d \neq c^eg^zh^t] \leq \negl(n)$.
		Let's evaluate this probability:
		\[ \Pr[d \neq c^eg^zh^t] = \Pr[g^y h^s \neq (g^x h^r)^e g^{y-ex}h^{s-er}]  = Pr[g^yh^s \neq g^y h^r] = 0 \]
		Then our construction is correct.

		\item When using the same commitment $c=g^xh^r$, if we get two different conversations $(e, z, t)$ and $e', z', t')$, we can recover the private secret $(x, r)$ by doing the following:
		\[ \begin{cases} z = y - ex \\ z' = y - e'x \end{cases} \Rightarrow x = \frac{z-z'}{e'-e} \]
		\[ \begin{cases} t = s - er \\ t' = s - e'r \end{cases} \Rightarrow r = \frac{t-t'}{e'-e} \]
		This means that the adversary $P^*$ should know $x$ and $r$, otherwise he could not build two valid responses for the same commit.

		Does that break the zero-knowledge property?
		In a sense, no, as in a practical scheme, the commit would be different each time due to a random $r$.

		\item (This answer has not been verified, and looks wrong, but as the subquestion has not been asked this year, I can't verify it.)

		When we get the conversation $(d, e, (z,t))$, we can ``rewind'' the conversation to submit another e' and get new z' and t'. Therefore we can obtain the private key (x,r) by doing those calculations:
		\[ \begin{cases} z = y - ex \\ z' = y - e'x \end{cases} \Rightarrow x = \frac{z-z'}{e'-e} \]
		\[ \begin{cases} t = s - er \\ t' = s - e'r \end{cases} \Rightarrow r = \frac{t-t'}{e'-e} \]
		We broke the zero-knowledge property since the verifier can extract the private key using such power.

		According to the assistants, since it is not zero-knowledge, there is no point of discussing the soundness property.

		\item It is easy to show, with honest parties, how we can simulate from $c$ a new valid conversation:
		\begin{enumerate}
			\item We pick $e$, $z$ and $t$ randomly.
			\item We evaluate $d \define c^e z^z h^t$.
		\end{enumerate}

		\item To generalize the process, we have:
		\begin{itemize}
			\item pk = $g^x_1$, ..., $g^x_n$, h
			\item sk = $x_1$, ..., $x_n$, r
			\item c = $g^x_1 \cdot ... \cdot g^x_n \cdot h^r$
		\end{itemize}
	\end{enumerate}
\end{solution}



\subsection{Exercise 2 (Schnorr ZKP with faulty PRG)}

Let us study what happens when the prover of a Schnorr ZKP uses a faulty random generator. this generator is used to choose the secret $\alpha$ used to generate the commitment $c=g^\alpha$, where $g$ is a generator of the group. Assuming that the prover made two proofs, one with secret $\alpha_0$, and the other one with secret $\alpha_1=a \alpha_0 + b$, how can you reciver the secret witness, knowing the public values, the transcripts of the proofs, $a$ and $b$?


\begin{solution}
	We then have
	\[ \begin{cases} g^r = g^{\alpha_0} \\ e \\ f = \alpha_0 + ex \pmod q \end{cases} \qquad \text{and} \qquad \begin{cases} g^{r'} = g^{\alpha_1}=g^{a\alpha_0+b} \\ e' \\ f' = a \alpha_0 + b + e'x \pmod q \end{cases} \]
	Then, we can immediately infer
	\[af-f' = aex - b - e'x\]
	\[x=\frac{a\cdot f-f'+b}{a\cdot e-e'}.\]
	And also, $\alpha_0=f-e\cdot x$.
\end{solution}



\subsection{Exercise 3 (Commitment scheme and batching)}

\copypaste{9}{0}

