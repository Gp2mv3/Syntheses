\PassOptionsToPackage{shortlabels}{enumitem}
\documentclass[en,license=none]{../../../eplexercises}

\usepackage{qtree}
\usepackage[]{algorithm2e}
\usepackage{algorithmic}
\usetikzlibrary{patterns}
\usetikzlibrary{shapes}

% Encryption
\renewcommand{\K}{\mathcal{K}}
\renewcommand{\C}{\mathcal{C}}
\newcommand{\M}{\mathcal{M}}
\newcommand{\D}{\mathcal{D}}
\renewcommand{\O}{\mathcal{O}}
\newcommand{\G}{\mathcal{G}}
\DeclareMathOperator{\Gen}{\mathsf{Gen}}
\DeclareMathOperator{\Enc}{\mathsf{Enc}}
\DeclareMathOperator{\Dec}{\mathsf{Dec}}
\DeclareMathOperator{\Mac}{\mathsf{Mac}}
\DeclareMathOperator{\Vrfy}{\mathsf{Vrfy}}
\DeclareMathOperator{\Com}{\mathsf{Com}}
\DeclareMathOperator{\Open}{\mathsf{Open}}
\DeclareMathOperator{\Sign}{\mathsf{Sign}}
\DeclareMathOperator{\Sig}{\mathsf{Sig}}
\DeclareMathOperator{\PRG}{\mathsf{PRG}}
\DeclareMathOperator{\F}{\mathsf{F}}
\DeclareMathOperator{\f}{\mathsf{f}}
\DeclareMathOperator{\sprp}{\mathsf{sprp}}
\DeclareMathOperator{\adv}{\mathsf{adv}}
\DeclareMathOperator{\Gr}{\mathsf{Gr}}
\DeclareMathOperator{\Ima}{\mathsf{Im}}


% Experiments
\newcommand{\PrivKeav}{\mathsf{PrivK}_{\A, \Pi}^{\mathsf{eav}}}
\newcommand{\PrivKmult}{\mathsf{PrivK}_{\A,\Pi}^\mathsf{mult}}
\newcommand{\PrivKcpa}{\mathsf{PrivK}_{\A,\Pi}^\mathsf{cpa}}
\newcommand{\PrivKcca}{\mathsf{PrivK}_{\A,\Pi}^\mathsf{cca}}

% Functions
\newcommand{\negl}{\ensuremath{\epsilon}}

% Adversary
\newcommand{\A}{\mathcal{A}}
\newcommand{\Adv}{\ensuremath{\mathcal{A}}}

% Arithmetic
\newcommand{\modn}{\, (\text{mod } n)}


\newcommand{\xor}{\oplus}
\newcommand{\st}{\text{ s.t. }}
\renewcommand{\Z}{\mathbb{Z}}
\DeclareMathOperator{\PrivK}{PrivK}
\DeclareMathOperator{\MacForge}{MacForge}
\DeclareMathOperator{\Sigforge}{Sig-forge}
\DeclareMathOperator{\Invert}{Invert}
\DeclareMathOperator{\DDH}{DDH}
\DeclareMathOperator{\DLog}{DLog}
\newcommand{\PrivKmultcpa}{\PrivK^{\text{multcpa}}}
\newcommand{\Sigforgeone}{\Sigforge^{\text{1-time}}}

% New commands
\newcommand{\flec}{\draw[->,>=latex]}
\newcommand{\flect}{\draw[->,>=latex,red]}
\newcommand{\flecc}{\draw[->,>=latex,blue]}
\newcommand{\mess}{node[midway,above]}
\newcommand{\messd}{node[midway,below]}
\newcommand{\trapeze}[2]{\draw #2 node {#1};
	\draw[line width=1pt] #2 ++(-1,2) -- ++(2,-1) {[rounded corners] -- ++(0,-2) -- ++(-2,-0)} -- ++(0,3);}
\newcommand{\position}{(0,0)}
\newcommand{\mylabel}{$x_1$}
\newcommand{\carref}[2] { \draw[line width = 1pt,fill=#2] #1 ++ (-1,-1) rectangle #1 ++ (1,1); }
\newcommand{\carrep}[3] { \draw[line width = 1pt,pattern color=#2,pattern= #3] #1 ++ (-1,-1) rectangle #1 ++ (1,1); }


\titleformat
{\subsection}[hang]{\bfseries\Large}{\thesubsection: }{0.5ex}{}

\hypertitle{Cryptography}{7}{MAT}{2450}
{Benoît Legat \and Luis Tascon Gutierrez \and Guillaume Gheysen \and Olivier Leblanc}
{François Koeune and Olivier Pereira}
\newpage
Special thanks to the assistants Francesco Berti (francesco.berti@uclouvain.be) and Pierrick M\'eaux (pierrick.meaux@uclouvain.be) who gave us the \LaTeX\ code of the statements of the exercises.

Note: I (Luis Tascon Gutierrez) had to merge the \LaTeX\ code of the solution we have written and the code the assistants sent to me and it means that there might still be some typo errors due to commands that were not the same between the two documents.

\section{}
\subsection{Exercise 1 (Perfect secrecy.)}
We define the following encryption scheme for messages, keys and
ciphertexts in $\mathbb{Z}_n$, where $\mathbb{Z}_n$ is essentially 
the integers in the interval $[0,n[$ 
(in fact $(\mathbb{Z}_n,+)$ forms a group):
\smallskip
\begin{itemize}
  \item $\Gen$ outputs a key $k \in \K$ selected uniformly at random.
  \item $\Enc_k(m) := k+m \mod n$
  \item $\Dec_k(c) := c-k \mod n$
\end{itemize}
\smallskip
Suppose messages are drawn from $\M$ according to the binomial
distribution. More precisely $M\sim \mathrm{Bi}(n-1,p)$ for some probability $p$ 
which means that $\forall m\in \M: \Pr[M=m]=\binom{n-1}{m}p^{m}(1-p)^{n-1-m}$.
\smallskip
\begin{enumerate}
  \item Show that the encryption scheme above is perfectly secret.
  \item Evaluate $\Pr[C=c]$ for every $c \in \C$.
  \item Evaluate $\Pr[K=k|C=c]$ for every $k\in \K$ and $c\in \C$. 
\end{enumerate}

\begin{solution}
  \begin{enumerate}
    \item
      We have secret privacy if : $Pr[C = c | M = m_0] = Pr[C = c | M = m_1] $ for every $m_0, m_1 \in \M$ and $c \in \C$.
      
      Let $c \in \C$ and $m\in \M$.
      We have :
      \begin{align*}
        \Pr[C = c | M = m]
        & = \Pr[M + K = c \pmod{n} | M = m]\\
        & = \Pr[m + K = c \pmod{n}]\\
        & = \Pr[K = c - m \pmod{n}]\\
        & = \frac{1}{n} \text{ (because $K$ is selected \textbf{uniformly at random} in } \K \text{ where } \abs{\K} = n )\\
        & = \Pr[C = c | M = m'] \text{ for every } m' \in \M
      \end{align*}
      Therefore, we have :
      \[
        \Pr[C = c | M = m_1] = \Pr[C = c | M = m_2]
      \]
      for every $c \in \C$ and $m_1,m_2 \in \M$. \\
      Which means we have perfect secrecy.
    \item
        Using the the result obtained at last exercice and the equivalent definitions about private secrecy, we can obtain :
      \begin{align*}
          \Pr[C = c]  & = \Pr[C = c | M = m] \text{ for every } m \in \M \\
          & = \frac{1}{n}
      \end{align*}
      Other way to solve it (thanks to Benoît Legat) : 
      \begin{align*}
        \Pr[C = c]
        & = \sum_{m \in \M} \Pr[\Enc_K(M) = c | M = m] \Pr[M = m]\\
        & = \frac{1}{n} \sum_{m \in \M} \Pr[M = m]\\
        & = \frac{1}{n}.
      \end{align*}
    \item
      \begin{align*}
        \Pr[K = k | C = c]
        & = \Pr[C - M \equiv k \pmod{n} | C = c]\\
        & = \Pr[c - M \equiv k \pmod{n}]\\
        & = \Pr[M \equiv c - k \pmod{n}]\\
        & = {n-1 \choose c-k} p^{c-k} (1-p)^{n-1-c+k}.
      \end{align*}
  \end{enumerate}
\end{solution}


\subsection{Exercise 2 (Negligible functions.)}
\begin{enumerate}
\item Let $f$ be a negligible function in $n$. Show that $g: n \mapsto
  1000\cdot f(n)$ is negligible too.
\item Show that the function $n \mapsto n^{-\log(n)}$ is negligible in $n$.
\end{enumerate}
\begin{solution}
  \begin{enumerate}
    \item Let $p$ be a polynomial,
      let's take $q = 1000p$,
      since it is also a polynomial, we know
      that there exists $N$ such that for all $n \geq N$,
      \[ f(n) \leq \frac{1}{q(n)}. \]
      But that implies that
      \[ 1000 \cdot f(n) \leq \frac{1}{p(n)} \]
      so
      \[ g(n) \leq \frac{1}{p(n)}. \]
    \item Let $p(n) = a_0 + a_1 n + \cdots + a_dn^d$ be an
      arbitrary polynomial of arbitrary degree $d$.
      Let $N_1$ such that $N_1 > r$ for each root $r$ of $p$.
      We know that for $n \geq N_1$, the sign of $p$ is the sign of $a_d$.
      Of course, if $a_d < 0$, our job is impossible but we do not consider these cases.
      Since $p(n) > 0$, our equation is equivalent to
      \[ n^{\log(n)} \geq p(n) \]
      For $n \geq \max(N_1,1)$ we also have
      $p(n) \leq n^d \sum_{i=0}^d|a_i|$.
      Taking the logarithm on both side (we can do it since the logarithm is strictly increasing),
      we have	%Why ? Can you explain this operation ?
      \[ \log^2(n) - d \log(n) - \log\sum_{i=0}^d|a_i| \geq 0 \]
      which is a second order polynomial in $\log(n)$.
      Let $r_1,r_2$ be its roots.
      We can take $N = \max(N_1,1,2^{r_1},2^{r_2})$.
      
      \textbf{There is an other way to show this}. We know that, f is negligible iff for all positive polynomial p, there exist an N such that for all n$\geq$ N : $ f(n) \leq \frac{1}{p(n)}$.
      
      In our case we have $f(n) = n^{-log(n)}$ and we represent any polynomial as $n^c$. Then : 
          $$n^{-log(n)} \leq n^{-c}$$
          $$log(n^{-log(n)}) \leq log(n^{-c})$$
          $$log(n) \geq c $$
      If we take N = exp(c), then our relation will be respected. As there exist an N where n$\leq$ N in wich the relation is respected, then the function is negligible. 
  \end{enumerate}
\end{solution}


\subsection{Exercise 3 (Efficiency.)}
Explain why the function that maps $n$ on a sequence of ``$1$'' of length
$\lfloor \sqrt{n}\rfloor$ cannot be evaluated by any efficient algorithm.

An example of such algorithm is given in Algorithm~\ref{alg1}.
\begin{algorithm}                        
\begin{algorithmic}
    \REQUIRE $n \geq 0$
    \ENSURE A sequence of $\sqrt{n}$ ``$1$''
    \FOR{$i=0$ to $\lfloor\sqrt{n}\rfloor$}
        \STATE Print `1'
    \ENDFOR
\end{algorithmic}    
\caption{example of algorithm}
\label{alg1}      
\end{algorithm}

Hint: see $n$ as a power of $2$.  
\begin{solution}
  An algorithm A is efficient if there exist a PPT p such that : 
  $$ A(x) \leq p(|x|) $$
  As we can see from the exercise : 
  $$A(n) \ = \ \sqrt{n} $$
  $$ |n| \ = \ log_2(n) \ \textbf{because n is encoded as a binary number} $$
  But for all PPT p, 
  \[ \sqrt{n}  >  p(\log_2(n)) \]
  So the algorithm is not efficient. 
  
  \textbf{P.S.} : It would have been efficient if we write the input as $1^n$.
  
  \textbf{Other more intuitive approach : }
  The input $n$ can be expressed under binary form as: $$n = 2^{|n|}$$ 
  Let's say that $k = |n|$. We know that the algorithm has to do at least $\sqrt{n}$ steps.
  $$\sqrt{n} = \sqrt{2^k} = 2^{\frac{k}{2}}$$
  Which is not polynomial.
\end{solution}


\subsection{Exercise 4 (Security model.)}
Let $\negl$ denote a negligible function.
Remember that $\Pi:=\langle \Gen, \Enc, \Dec \rangle$ has \emph{indistinguishable
multiple encryption in the presence of eavesdroppers} if $\forall$
PPT $\A$, $\exists$ $\negl$ :
  $$\Pr[\PrivKmult(n)=1]\leq\frac12+\negl(n) \,,$$
where $\PrivKmult(n)$ is defined as follows.
%
\smallskip
\begin{enumerate}
\item   $\A$ outputs $M_0=(m_0^1,\ldots,m_0^t),
M_1=(m_1^1,\ldots,m_1^t)$
\item Choose $k \leftarrow \G(1^n)$ and $b \leftarrow \{0,1\}$, and send
  $(\Enc_k(m_b^1),\ldots,\Enc_k(m_b^t))$ to $\A$
\item $\A$ outputs $b'$
\item Define $\PrivKmult(n):=1$ iff $b=b'$
\end{enumerate}
%
\smallskip
Also remember that $\Pi:=\langle \Gen, \Enc, \Dec \rangle$ has \emph{indistinguishable
encryption under a chosen-plaintext attack} if $\forall$ PPT $\A$,
$\exists$ $\negl$ :
  $$\Pr[\PrivKcpa(n)=1]\leq\frac12+\negl(n) \,,$$
where $\PrivKcpa(n)$ is defined as follows.
\smallskip
\begin{enumerate}
  \item Choose $k\leftarrow \Gen(1^n)$
  \item \textbf{$\A$ is given oracle access to $\Enc_k(\cdot)$}
  \item $\A$ outputs $m_0, m_1 \in \M$
  \item Choose $b\leftarrow\{0,1\}$ and send $\Enc_k(m_b)$ to $\A$
  \item \textbf{$\A$ is again given oracle access to $\Enc_k(\cdot)$}
  \item $\A$ outputs $b'$
  \item Define $\PrivKcpa(n):=1$ iff $b=b'$
\end{enumerate}
\smallskip

Define the concept of indistinguishable \emph{multiple} encryption under a chosen-plaintext attack.

\begin{solution}
%Sending it once (in a vector) or with a loop is exactly the same, so I think only one definition is sufficient...
  Two definition can be proposed.
  The first one is the one given in the reference \cite[p.~84]{katz2007introduction}.

  Both are equally good since it can be proven they are equivalent to the definition of indistinguishably of a \emph{single} encryption
  under CPA.
  Proving that if $\Pi$ has indistinguishable \emph{multiple} encryption under CPA then it also has indistinguishable \emph{single} encryption
  is trivial.
  The other way is quite tricky.
  However in public key cryptosystems, CPA is the same than EAV since $\A$ has the public key and can therefore oracle access to $\Enc$.
  There is therefore the same property in asymmetric crypto for EAV than for symmetric crypto with CPA.
  This is stated by the \cite[theorem~10.10]{katz2007introduction} which is proven.
  The proof is very similar to the proof we have to make to show the equivalence so if you are in doubt, just check it out.

  \begin{enumerate}

    \item
      $\Pi := \langle\Gen, \Enc, \Dec\rangle$ has indistinguishable \emph{multiple} encryption under a chosen-plaintext attack
      if $\forall$ PPT $\A$, $\exists \epsilon$:
      \[ \Pr[\PrivKmultcpa_{\A,\Pi}(n) = 1] \leq \frac{1}{2} + \epsilon(n), \]
      where $\PrivKmultcpa_{\A,\Pi}(n)$ is defined as follows.
      \begin{enumerate}
        \item Choose $k \leftarrow \Gen(1^n)$
        \item $\A$ is given oracle access to $\Enc_k(\cdot)$
        \item $\A$ outputs $M_0 = (m_0^1, \ldots, m_0^t)$, $M_1 = (m_1^1, \ldots, m_1^t)$
        \item Choose $b \leftarrow \{0,1\}$, and send $(\Enc_k(m_b^1), \ldots, \Enc_k(m_b^t))$ to $\A$
        \item $\A$ is again given oracle access to $\Enc_k(\cdot)$
        \item $\A$ outputs $b'$
        \item Define $\PrivKmultcpa_{\A,\Pi}(n) := 1$ iff $b = b'$
      \end{enumerate}
	

    \item
      $\Pi := \langle\Gen, \Enc, \Dec\rangle$ has indistinguishable \emph{multiple} encryption under a chosen-plaintext attack
      if $\forall$ PPT $\A$, $\exists \epsilon$:
      \[ \Pr[\PrivKmultcpa_{\A,\Pi}(n) = 1] \leq \frac{1}{2} + \epsilon(n), \]
      where $\PrivKmultcpa_{\A,\Pi}(n)$ is defined as follows.
      
      \begin{enumerate}
        \item Choose $k \leftarrow \Gen(1^n)$
        \item $\A$ is given oracle access to $\Enc_k(\cdot)$
        \item Choose $b \leftarrow \{0,1\}$
        \item For $k' \in \{1, \ldots, t\}$
          \begin{enumerate}
            \item $\A$ outputs $(m_0^{k'}, m_1^{k'})$
            \item Send $\Enc_k(m_b^{k'})$ to $\A$
            \item $\A$ is again given oracle access to $\Enc_k(\cdot)$
          \end{enumerate}
        \item $\A$ outputs $b'$
        \item Define $\PrivKmultcpa_{\A,\Pi}(n) := 1$ iff $b = b'$
      \end{enumerate} 
  \end{enumerate}
\end{solution}


\subsection{Exercise 5 (Pseudorandomness.)}
Let $F: \{0,1\}^* \times \{0,1\}^* \rightarrow \{0,1\}^*$ be a
(length-preserving) pseudorandom function, that is, if $k$ is selected
uniformly at random in $\{0,1\}^n$, then $F_k(\cdot)$ is
computationnaly indistinguishable from a function $f$ selected randomly in the set of
functions from $\{0,1\}^n$ to $\{0,1\}^n$. More formally, $\forall$ PPT $D$, $\exists$ negl. $\negl$:
$$\left|\Pr[D^{F_k(\cdot)}(1^n)=1]-\Pr[D^{f(\cdot)}(1^n)=1]\right|\leq\negl(n)$$

Show that F cannot seem random in front of an adversary who has an unbounded computational power, 
in the sense that she can distinguish it from a random function.
\begin{solution}
  There are $|\{0,1\}^n|^{|\{0,1\}|^n} = {2^n}^{2^n}$ function from $\{0,1\}^n$ to $\{0,1\}^n$.
  However, since there are only $2^n$ different $k$, $F_k$ can only be $2^n$ different functions.
  If the distinguisher $D^g$ is unbounded, he can just check the output of $g$ for every possible input and for all $k \in \{0,1\}^n$, he can check if it has the same output of $g$.
  If it has the same output of $F_k$ for at least one $k$, then $D^g(1^n) = 1$, else $D^g(1^n) = 0$.
  More formally
  \[
    D^g(1^n) \overset{\Delta}{=} 
    \left\{ \begin{array}{rl} 
        1 & \mbox{if }\exists k \in \{0,1\}^n, \forall m \in \{0,1\}^n, F_k(m) = g(m)\\
		0 & \mbox{otherwise.}\\
    \end{array} \right.
  \]
  We can see that
  \[ \Pr[D^{F_k}(1^n) = 1] = 1 \]
  for all $k \in \{0,1\}^n$.
  Since there could be $k_1,k_2$ such that $F_{k_1}(m) = F_{k_2}(m)$ for all $m \in \{0,1\}^n$,
  \[ |\{f : \{0,1\}^n \to \{0,1\}^n | \exists k \in \{0,1\}^n, \forall m \in \{0,1\}^n f(m) = F_k(m) \}| \leq 2^n. \]
  Therefore
  \[ \Pr[D^{f}(1^n) = 1] \leq \frac{2^n}{{2^n}^{2^n}} = {2^n}^{(1-2^n)}. \]
\end{solution}


\subsection{Exercise 6 (Reduction.)}
Let $\Pi=\langle \Gen,\Enc,\Dec\rangle$ be an encryption scheme having
indistinguishable encryption under a chosen plaintext attack. Suppose we
define a new scheme $\Pi':=\langle \Gen',\Enc',\Dec'\rangle$ as follows.
\smallskip
\begin{itemize}
  \item $\Gen':=\Gen$
  \item $\Enc_k'(m):=\Enc_k(m)||1$ (i.e. a `1' bit is appended to the ciphertext)
  \item $\Dec_k'(c):=\Dec_k(c_1)$, where $c_1$ is obtained by discarding the last bit of $c$.
\end{itemize}
\smallskip
Is $\Pi'$ also a CPA secure encryption scheme? Provide either an (efficient) attack/adversary
or a (polynomial) reduction, depending on your claim.

\begin{solution}
  $\Pi$ is a secure encryption scheme under CPA. $\Pi$ is public, only the key is hidden from $\A$. Adding a 1 at the end will just give no information to $\A$.

  %To prove it rigorously, we can prove that ``if $\Pi'$ is insecure then $\Pi$ is insecure'' since it is the contraposition of ``if $\Pi$ is secure then $\Pi'$ is secure''. % Perso je trouve la formulation rend confus
  This proof methodology is called ``reduction''.
  
    %TODO define more clearly the interface with the adversary and with the oracle
  Let $\C$ be the challenger trying to break $\Pi$ and an efficient adversary $\A$ that can break $\Pi'$ with a non-negligible probability. $\O$ is the oracle that gives the challenge to break the scheme $\Pi$.
  \begin{enumerate}
    \item $\O$ is given $1^n$ as input as $\C$ that will transmit it to $\A$.
    \item First query phase:
      \begin{itemize}
        \item $\A$ outputs $m_i$ as message to $\C$.
        \item $\C$ outputs $m_i$ as message to $\O$.
        \item $\O$ outputs $c_i = Enc_k(m_i)$ as message to $\C$.
        \item $\C$ sends back $c_i||1$ to $\A$.
      \end{itemize}
    \item Challenge phase:
      \begin{itemize}
        \item $\A$ outputs $m_0^\ast, m_1^\ast$ to $\C$.
        \item $\C$ outputs $m_0^\ast, m_1^\ast$ as message to $\O$.
        \item $\O$ choose randomly $b \leftarrow \{0,1\}$.
        \item $\O$ outputs $c^\ast = Enc_k(m_b^\ast)$ to $\C$.
        \item $\C$ sends back $c^\ast||1$ to $\A$.
      \end{itemize}
    \item Second query phase: same as the first one.
    \item $\A$ outputs $b'$ to $\C$.
    \item $\C$ outputs $b'$.
  \end{enumerate}
  We have:
  $$Pr[b'=b] = Pr[\A \text{ wins over } \Pi']$$
  If $\A$ has a non-negligible probability to win against the $\Pi'$ scheme then $\C$ has also a non negligible probability to win against the $\Pi$ scheme. We can conclude that $\Pi'$ is also a secure scheme.
\end{solution}

\subsection{Exercise 7 (Reduction and/or attacks.)}
Let $\Pi_1=\langle \Gen^1,\Enc^1,\Dec^1\rangle$ and $\Pi^2=\langle \Gen^2,\Enc^2,\Dec^2\rangle$ be an encryption scheme with $\Enc^1:\mathcal{K}\times \mathcal{M}^1 \longmapsto \mathcal{C}^1$ and $\Enc^2:\mathcal{K}\times \mathcal{M}^2 \longmapsto \mathcal{C}^2$ 
\begin{enumerate}
\item[a] If $\mathcal{C}^1 = \mathcal{M}^2$, let $\Pi=\langle \Gen,\Enc,\Dec\rangle$ with
\begin{itemize}
  \item $\Gen:=(\Gen_1,\Gen_2)$ (that is, we obtain two different keys $(k_1,k_2)$
  \item $\Enc_{(k_1,k_2)}(m):=\Enc_{k_2}^2(\Enc^1_{k_1}(m))$ 
  \item $\Dec_{(k_1,k_2)}(c):=\Dec^1_{k_1}(\Dec^2_{k_2}(c))$ 
\end{itemize}
\smallskip
\item If $\Pi^1$ is CPA secure, is it $\Pi$ CPA secure?
\item If $\Pi^2$ is CPA secure, is it $\Pi$ CPA secure? 
\item If $\Pi$ is CPA secure, is it $\Pi^1$ CPA secure?
\item If $\Pi$ is CPA secure, is it $\Pi^2$ CPA secure?
\item[b] If $\mathcal{M}^1 = \mathcal{M}^2$ and $\mathcal{C}^1 = \mathcal{C}^2$. let $\Pi'=\langle \Gen',\Enc',\Dec'\rangle$ with
\begin{itemize}
  \item $\Gen':=(\Gen^1,\Gen^2)$ (that is, we obtain two different keys $(k_1,k_2)$
  \item $\Enc'_{(k_1,k_2)}(m):=(c_1,c_2)$ with $c_1=\Enc^1_{k_1}(m),~c_2=\Enc^2_{k_2}(m))$ 
  \item $\Dec'_{(k_1,k_2)}(c):=\Dec_{k_1}(c_1)$ with $c=c_1\|c_2$ ($c_1$ is the first half of $c$)
\end{itemize}
\smallskip
\item If $\Pi^1$ is CPA secure, is it $\Pi'$ CPA secure?
\item If $\Pi^2$ is CPA secure, is it $\Pi'$ CPA secure? 
\item If $\Pi'$ is CPA secure, is it $\Pi^1$ CPA secure?
\item If $\Pi'$ is CPA secure, is it $\Pi^2$ CPA secure?
\end{enumerate}
\begin{solution}
\begin{enumerate}
	\item Let's assume $\Pi$ is not CPA secure: There exist an adversary A.\\
	We build an adversary $A^1$ for $\Pi_1$
	$$\begin{aligned}
		Pr[b''=b;b''\leftarrow a^1] &= Pr[b'=b;b'\leftarrow a]\\
		Pr[b''=b] &= Pr[b'=b] \le 1/2+\varepsilon
	\end{aligned}$$
	
	\item \textbf{If $\Pi^2$ is CPA secure, is it $\Pi$ CPA secure?}\\
	We ($D$) define an oracle ($O(\Pi^2)$) that can securely encode a message with $\Pi^2$ and instantiate an Attacker ($A$). As we have to challenge the $\Pi$ scheme knowing the $\Pi^2$ is CPA secure we will proceed as follow.
	\begin{description}
		\item[First learning phase:] We begin by encrypting the messages from the attacker with $\Pi^1$ to send them to the oracle. The oracle responds by encrypting the message received with $\Pi^2$ and we just pass this response to the attacker.
		\item[Challenge phase:] The attacker choose two messages and we transmit the two messages with the first encryption. The oracle will choose witch message to encrypt and will respond with one of the two messages encrypted that we will send back to the attacker.
		\item[Second learning phase:] Same as the first one.
	\end{description}
	\begin{center}
		\begin{tikzpicture}[scale=0.8]
			%structure
			\draw[rounded corners=10pt,thick] (0,0) rectangle (5,10);
			\draw[rounded corners=10pt,thick] (0.5,0.5) rectangle (2,8.5);
			\draw[rounded corners=10pt,thick] (13,0) rectangle (17,10);
			\node[above right] at (0,10) {$D$};
			\node[above right] at (0.5,8.5) {$A$};
			\node[above left] at (17,10) {$O(\Pi^2)$};
			\node[below left] at (5,10) {$k_1 \leftarrow gen^1(1^n)$};
			\node[below left] at (17,10) {$k_2 \leftarrow gen^2(1^n)$};
			
			%train phase
			\flect (2,8) -- (4,8) \mess {$m_i$};
			\flect (4,7) -- (2,7) \mess {$c_i$};
			\flect (5,8) -- (13,8) \mess {$m_i':=Enc_{k_1}^1(m_i)$};
			\flect (13,7) -- (5,7) \mess {$c_i:=Enc_{k_2}^2(m_i')$};
			
			%challenge phase
			\flecc (2,5.5) -- (4,5.5) \mess {$m^{\ast}_0,m^{\ast}_1$};
			\flecc (4,4.5) -- (2,4.5) \mess {$c^\ast$};
			\flecc (5,5.5) -- (13,5.5) \mess {$m
			_0^{\ast\prime}:=Enc_{k_1}^{1}(m_0^\ast),m_1^{\ast\prime}:=Enc_{k_1}^{1}(m_1^{\ast})$};
			\flecc (13,4.5) -- (5,4.5) \mess {$c^\ast:=Enc_{k_2}^{2}(m_b^{\ast\prime})$};
			\node[below right] at (13,5.5) {$b \leftarrow \{0,1\}$};
			
			%train phase
			\flect (2,3) -- (4,3) \mess {$m_i$};
			\flect (4,2) -- (2,2) \mess {$c_i$};
			\flect (5,3) -- (13,3) \mess {$m_i':=Enc_{k_1}^1(m_i)$};
			\flect (13,2) -- (5,2) \mess {$c_i:=Enc_{k_2}^2(m_i')$};
			
			% output
			\flec (2,1) -- (3,1) node[pos=1,right] {$b'$};
			\flec (5,1) -- (6,1) node[pos=1,right] {$b''=b'$};
		\end{tikzpicture}
	\end{center}
	As we can see in every case, the distinguisher will have the same probability to find the message encrypted by the oracle than the attacker to break the scheme. As the attacker can only have a probability of $1/2 + \varepsilon$ to succeed the distinguisher will have the same probability. So, the scheme $\Pi$ is secure.
	
	\item As seen in the previous development, if $\Pi^2$ is CPA secure, $\Pi$ is CPA secure. There is no restriction on $\Pi^1$ in that case. Therefore $\Pi^1$ could be such that $Enc^1_{k_1}(m):=m$ which is obviously not CPA secure. So the proposition is false.
	
	\item Idem
	
	\item \textbf{If $\Pi^1$ is CPA secure, is it $\Pi'$ CPA secure?}\\
	The $\Pi'$ scheme is CPA secure if and only if $\Pi^2$ is also CPA secure.
	
	For example, if $Enc_{k_2}^2(m) = m$ then the scheme $\Pi'$ is not CPA secure.
	
	TO DEVELOP. (solution of the teaching assistant?)
	
	\item TODO
	
	\item TODO
	
	\item TODO
\end{enumerate}

\end{solution}


\newpage
\section{TP 2}
%\addcontentsline{toc}{section}{TP 2}


% \section*{Rappel}

% \begin{center}
% \textbf{Liste des équivalences logiques}
% \end{center}

% \textsf{Lois commutatives}
% \begin{enumerate}
% 	\item $p \vee q \Lleftarrow\!\!\!\!\Rrightarrow q \vee p$ \textit{(commutativité de $\vee$)}
% 	\item $p \wedge q \Lleftarrow\!\!\!\!\Rrightarrow q \wedge p$ \textit{(commutativité de $\wedge$)}
% 	\item $p \Leftrightarrow q \Lleftarrow\!\!\!\!\Rrightarrow q \Leftrightarrow q$ \textit{(commutativité de $\Leftrightarrow$)}
% \end{enumerate}

% \textsf{Lois associatives}
% \begin{enumerate}
% 	\item $(p \vee q) \vee r \Lleftarrow\!\!\!\!\Rrightarrow p \vee (q \vee r)$ \textit{(associativité de $\vee$)}
% 	\item $(p \wedge q) \wedge r \Lleftarrow\!\!\!\!\Rrightarrow p \wedge (q \wedge r)$ \textit{(associativité de $\wedge$)}
% \end{enumerate}

% \textsf{Lois distributives}
% \begin{enumerate}
% 	\item $p \wedge (q \vee r) \Lleftarrow\!\!\!\!\Rrightarrow (p \wedge q) \vee (p \wedge r)$ \textit{(distributivité de $\wedge$ sur $\vee$)}
% 	\item $p \vee (q \wedge r) \Lleftarrow\!\!\!\!\Rrightarrow (p \vee q) \wedge (p \vee r)$ \textit{(distributivité de $\vee$ sur $\wedge$)}
% \end{enumerate}

% \textsf{Lois de De Morgan}
% \begin{enumerate}
% 	\item $\neg(p \wedge q) \Lleftarrow\!\!\!\!\Rrightarrow \neg p \vee \neg q$ \textit{(loi 1 de De Morgan)}
% 	\item $\neg(p \vee q) \Lleftarrow\!\!\!\!\Rrightarrow \neg p \wedge \neg q$ \textit{(loi 2 de De Morgan)}
% \end{enumerate}

% \textsf{Loi de la négation}
% \begin{enumerate}
% 	\item $\neg \neg p \Lleftarrow\!\!\!\!\Rrightarrow p$
% \end{enumerate}

% \textsf{Loi du tiers exclu}
% \begin{enumerate}
% 	\item $p \vee \neg p \Lleftarrow\!\!\!\!\Rrightarrow \textbf{true} $
% \end{enumerate}

% \textsf{Loi de la contradiction}
% \begin{enumerate}
% 	\item $p \wedge \neg p \Lleftarrow\!\!\!\!\Rrightarrow \textbf{false}$
% \end{enumerate}

% \textsf{Loi de l'implication}
% \begin{enumerate}
% 	\item $p \Rightarrow q \Lleftarrow\!\!\!\!\Rrightarrow \neg p \vee q$
% \end{enumerate}

% \textsf{Loi du contraposée}
% \begin{enumerate}
% 	\item $p \Rightarrow q \Lleftarrow\!\!\!\!\Rrightarrow \neg q \Rightarrow \neg p$
% \end{enumerate}

% \textsf{Loi de l'équivalence}
% \begin{enumerate}
% 	\item $p \Leftrightarrow q \Lleftarrow\!\!\!\!\Rrightarrow (p \Rightarrow q) \wedge (q \Rightarrow p)$
% \end{enumerate}

% \textsf{Lois de l'idempotence}
% \begin{enumerate}
% 	\item $p \Lleftarrow\!\!\!\!\Rrightarrow p \vee p$ \textit{(idempotence de $\vee$)}
% 	\item $p \Lleftarrow\!\!\!\!\Rrightarrow p \wedge p$ \textit{(idempotence de $\wedge$)}
% \end{enumerate}

% \textsf{Lois de simplification}
% \begin{enumerate}
% 	\item $p \wedge \textbf{true} \Lleftarrow\!\!\!\!\Rrightarrow p$
% 	\item $p \vee \textbf{true} \Lleftarrow\!\!\!\!\Rrightarrow \textbf{true}$
% 	\item $p \wedge \textbf{false} \Lleftarrow\!\!\!\!\Rrightarrow \textbf{false}$
% 	\item $p \vee \textbf{false} \Lleftarrow\!\!\!\!\Rrightarrow p$
% 	\item $p \vee (p \wedge q) \Lleftarrow\!\!\!\!\Rrightarrow p$
% 	\item $p \wedge (p \vee q) \Lleftarrow\!\!\!\!\Rrightarrow p$
% \end{enumerate}

% \begin{center}
% \textbf{Liste des règles d'inférence}
% \end{center}

% \begin{tabular}{c c c c}

% \textsf{Conjonction} & \textsf{Simplification} & \textsf{Addition} & \textsf{Syllogisme disjoint}   \\

% \begin{tabular}{l}
% $p$ \\
% $q$ \\
% \hline
% $p \wedge q$
% \end{tabular}

% &

% \begin{tabular}{l}
% $p \wedge q$ \\
% \hline
% $p$
% \end{tabular}

% &

% \begin{tabular}{l}
% $p$ \\
% \hline
% $p \vee q$
% \end{tabular}

% &

% \begin{tabular}{l}
% $p \vee q$ \\
% $\neg p$ \\
% \hline
% $q$
% \end{tabular} \\

% \textsf{Modus ponens} & \textsf{Modus tollens} & \textsf{Contradiction} & \textsf{Double négation} \\

% \begin{tabular}{l}
% $p \Rightarrow q$ \\
% $p$ \\
% \hline
% $q$
% \end{tabular}

% &

% \begin{tabular}{l}
% $p \Rightarrow q$ \\
% $\neg q$ \\
% \hline
% $\neg p$
% \end{tabular}

% &

% \begin{tabular}{l}
% $p$ \\
% $\neg p$ \\
% \hline
% $q$
% \end{tabular}

% &
% \begin{tabular}{l}
% $\neg \neg p$ \\
% \hline
% $p$
% \end{tabular} \\

% \textsf{Transitivité} & \textsf{Lois de l'équivalence} & \textsf{Théorème de la déduction} & \textsf{Réduction à l'absurde} \\

% \begin{tabular}{l}
% $p \Leftrightarrow q$ \\
% $q \Leftrightarrow r$ \\
% \hline
% $p \Leftrightarrow r$
% \end{tabular}

% &

% \begin{tabular}{l}
% $p \Leftrightarrow q$ \\
% \hline
% $p \Rightarrow q$ \\
% $q \Rightarrow p$
% \end{tabular}

% &

% \begin{tabular}{l}
% $p, \ldots, r, \boxed{s} \vdash t$ \\
% \hline
% $p, \ldots, r \vdash s \Rightarrow t$
% \end{tabular}

% &

% \begin{tabular}{l}
% $p, \ldots, q, \boxed{r} \vdash s$ \\
% $p, \ldots, q, \boxed{r} \vdash \neg s$ \\
% \hline
% $p, \ldots, q \vdash \neg r$
% \end{tabular}

% \end{tabular}

% \newpage
% \section*{Exercices}

\subsection*{Exercice 1}
Démontrez les équivalences logiques suivantes.

\begin{enumerate}
	\item $p \wedge (q \wedge r)  \Lleftarrow\!\!\!\!\Rrightarrow (p \wedge q) \wedge r$
	\item $p \Rightarrow (q \Rightarrow r) \Lleftarrow\!\!\!\!\Rrightarrow (p \Rightarrow q) \Rightarrow (p \Rightarrow r)$
	\item $p \wedge (p \Rightarrow q) \Rightarrow q \Lleftarrow\!\!\!\!\Rrightarrow \textbf{true}$
	\item $(p \vee q) \wedge (\neg p \vee q) \Lleftarrow\!\!\!\!\Rrightarrow q$
	\item $(p \vee q) \vee (\neg p \wedge \neg q) \Lleftarrow\!\!\!\!\Rrightarrow \textbf{true}$
% 	\item $(p \vee q) \wedge (\neg p \wedge \neg q) \Lleftarrow\!\!\!\!\Rrightarrow \textbf{false}$
	\item $p \vee (q \wedge r) \Lleftarrow\!\!\!\!\Rrightarrow \neg (\neg (p \vee q) \vee \neg (p \vee r))$
	\item $(p \vee q) \wedge \neg (p \wedge q) \Lleftarrow\!\!\!\!\Rrightarrow (p \wedge \neg q) \vee (\neg p \wedge q)$
	\item $p \wedge q \Lleftarrow\!\!\!\!\Rrightarrow (p \vee q) \wedge (p \Leftrightarrow q)$
\end{enumerate}

    \subsubsection*{Solution}
    Notons d'abord que toutes les preuves suivantes peuvent aussi être réalisées grâce aux table de vérités.
\begin{enumerate}
	\item
    \begin{flalign*}
    p \land (q \land r) &\Lleftarrow\!\!\!\!\Rrightarrow \lnot \lnot (p \land ( q \land r )) \tag*{Double négation}\\
    &\Lleftarrow\!\!\!\!\Rrightarrow \lnot ( \lnot (q \land r) \lor \lnot p ) \tag*{De Morgan}\\
    & \Lleftarrow\!\!\!\!\Rrightarrow \lnot ((\lnot r \lor \lnot q) \lor \lnot p) \tag*{De Morgan} \\
    & \Lleftarrow\!\!\!\!\Rrightarrow \lnot(( \lnot p \lor \lnot q) \lor \lnot r) \tag*{Associativité}\\
    & \Lleftarrow\!\!\!\!\Rrightarrow \lnot (\lnot p \lor \lnot q) \land \lnot \lnot r \tag*{De Morgan} \\
    & \Lleftarrow\!\!\!\!\Rrightarrow (p \land q) \land r \tag*{De Morgan et double négation}
    \end{flalign*}

	\item
	\begin{flalign*}
    p \Rightarrow (q \Rightarrow r)& \Lleftarrow\!\!\!\!\Rrightarrow p \Rightarrow (\lnot q \lor r) \tag*{Implication} \\
    & \Lleftarrow\!\!\!\!\Rrightarrow \lnot p \lor (\lnot q \lor r) \tag*{Implication} \\
    & \Lleftarrow\!\!\!\!\Rrightarrow (\lnot p \lor \lnot q) \lor r \tag*{Associativité}\\
    & \Lleftarrow\!\!\!\!\Rrightarrow (\text{true} \land (\lnot p \lor \lnot q)) \lor r \tag*{Simplification inverse} \\
    & \Lleftarrow\!\!\!\!\Rrightarrow ((\lnot p \lor p) \land  (\lnot p \lor \lnot q)) \lor r \tag*{Loi du tiers exclus}\\
    & \Lleftarrow\!\!\!\!\Rrightarrow (\lnot p \lor (p \land \lnot q)) \lor r \tag*{Distributivité}\\
    & \Lleftarrow\!\!\!\!\Rrightarrow ((p \lor \lnot q) \lor \lnot p) \lor r \tag*{Associativité} \\
    & \Lleftarrow\!\!\!\!\Rrightarrow (p \land \lnot q) \lor (\lnot p \lor r) \tag*{Associativité} \\
    & \Lleftarrow\!\!\!\!\Rrightarrow \lnot \lnot (p \land \lnot q) \lor (\lnot p \lor r) \tag*{Double négation} \\
    & \Lleftarrow\!\!\!\!\Rrightarrow \lnot (\lnot p \lor q ) \lor ( \lnot p \lor \lnot r) \tag*{De Morgan}\\
    & \Lleftarrow\!\!\!\!\Rrightarrow (p \Rightarrow q) \Rightarrow (p \Rightarrow r) \tag*{Implication}
    \end{flalign*}

	\item
	\begin{flalign*}
    p \land (p \Rightarrow q ) \Rightarrow q & \Lleftarrow\!\!\!\!\Rrightarrow p \land (\lnot p \lor q ) \Rightarrow q \tag*{Implication} \\
    & \Lleftarrow\!\!\!\!\Rrightarrow \lnot (p \land ( \lnot p \lor q )) \lor q \tag*{Implication}\\
    & \Lleftarrow\!\!\!\!\Rrightarrow \lnot p \lor \lnot ( \lnot p \lor q ) \lor q \tag*{De Morgan}\\
    & \Lleftarrow\!\!\!\!\Rrightarrow \lnot p \lor (p \land \lnot q) \lor q \tag*{De Morgan}\\
    & \Lleftarrow\!\!\!\!\Rrightarrow (( \lnot p \lor p ) \land ( \lnot p \lor \lnot q )) \lor q \tag*{Distributivité}\\
    & \Lleftarrow\!\!\!\!\Rrightarrow (\text{true} \land ( \lnot p \lor \lnot q )) \lor q \tag*{Loi du tiers exclu}\\
    & \Lleftarrow\!\!\!\!\Rrightarrow ( \lnot p \lor \lnot q ) \lor q \tag*{Simplification}\\
    & \Lleftarrow\!\!\!\!\Rrightarrow \lnot p \lor (\lnot q \lor q ) \tag*{Associativité}\\
    & \Lleftarrow\!\!\!\!\Rrightarrow \lnot p \lor \text{true} \tag*{Loi du tiers exclu}\\
    & \Lleftarrow\!\!\!\!\Rrightarrow \text{true} \tag*{Simplification}\\
    \end{flalign*}

	\item
    \begin{flalign*}
    (p \lor q) \land (\lnot p \lor q) & \Lleftarrow\!\!\!\!\Rrightarrow (q \lor p) \land (q \lor \lnot p) \tag*{Loi commutative}\\
    & \Lleftarrow\!\!\!\!\Rrightarrow q \lor (p \land \lnot p) \tag*{Distributivité}\\
    & \Lleftarrow\!\!\!\!\Rrightarrow q \lor \text{false} \tag*{Simplification}\\
    & \Lleftarrow\!\!\!\!\Rrightarrow q \tag*{Simplification}
    \end{flalign*}

	\item
    \begin{flalign*}
    (p \vee q) \vee (\neg p \wedge \neg q) & \Lleftarrow\!\!\!\!\Rrightarrow (p \vee q) \vee \neg ( p \vee q) \tag*{De Morgan} \\
    & \Lleftarrow\!\!\!\!\Rrightarrow \text{true} \tag*{Loi du tiers exclu}
    \end{flalign*}

	\item
    \begin{flalign*}
    \lnot ( \lnot ( p \lor q ) \lor \lnot (p \lor r ) ) & \Lleftarrow\!\!\!\!\Rrightarrow (p \lor q ) \land ( p \lor r) \tag*{De Morgan}\\
    & \Lleftarrow\!\!\!\!\Rrightarrow p \lor ( q \land r ) \tag*{Distributivité}
    \end{flalign*}

	\item
    \begin{flalign*}
    (p \lor q) \land \lnot ( p \land q) & \Lleftarrow\!\!\!\!\Rrightarrow \lnot (\lnot (p \lor q ) \lor (p \land q)) \tag*{Double négation et De Morgan}\\
    & \Lleftarrow\!\!\!\!\Rrightarrow \lnot (( \lnot p \land \lnot q) \lor ( p \land q)) \tag*{De Morgan}\\
    & \Lleftarrow\!\!\!\!\Rrightarrow \lnot ((\lnot p \lor p) \land (\lnot p \lor q) \land (\lnot q \lor p) \land (\lnot q \lor q)) \tag*{Distributivité}\\
    & \Lleftarrow\!\!\!\!\Rrightarrow \lnot (\text{true} \land (\lnot p \lor q) \land (\lnot q \lor p) \land \text{true}) \tag*{Simplification}\\
    & \Lleftarrow\!\!\!\!\Rrightarrow \lnot ((\lnot p \lor q) \land (\lnot q \lor p)) \tag*{Simplification}\\
    & \Lleftarrow\!\!\!\!\Rrightarrow \lnot (\lnot p \lor q) \lor \lnot(\lnot q \lor p) \tag*{De Morgan}\\
    & \Lleftarrow\!\!\!\!\Rrightarrow (p \land \lnot q) \lor ( q \land \lnot p) \tag*{De Morgan}
    \end{flalign*}

	\item
    \begin{flalign*}
    (p \lor q) \land (p \Leftrightarrow q) & \Lleftarrow\!\!\!\!\Rrightarrow (p \lor q) \land (\lnot p \lor q) \land (p \lor \lnot q) \tag*{Double implication + loi de l'équivalence}\\
    & \Lleftarrow\!\!\!\!\Rrightarrow (p \land \lnot p \land p) \lor (p \land \lnot p \land \lnot q) \lor (p \land q \land p) \lor (p \land q \land \lnot q)\\
    & \lor (q \land \lnot p \land p)\lor (q \land \lnot p \land \lnot q) \lor (q \land q \land p) \lor (q \land q \land \lnot q) \tag*{Distributivité}\\
    & \Lleftarrow\!\!\!\!\Rrightarrow \text{false} \lor \text{false} \lor (p \land q \land p) \lor \text{false} \lor \text{false} \lor \text{false} \lor (q \land q \land p) \lor \text{false} \tag*{Simplification}\\
    & \Lleftarrow\!\!\!\!\Rrightarrow (p \land q) \lor (q \land p) \tag*{Simplification}\\
    & \Lleftarrow\!\!\!\!\Rrightarrow (p \land q) \tag*{Simplification}
    \end{flalign*}

\end{enumerate}


\subsection*{Exercice 2}
Démontrez, à l'aide d'une table de vérité, la validité des arguments suivants:

\begin{enumerate}
	\item \enter

	\begin{flushleft}
	\begin{tabular}{l}
		$p \vee q$ \\
		$\neg p$ \\
	\hline
	$q$
	\end{tabular}
\end{flushleft}


	\item \enter

	\begin{flushleft}
	\begin{tabular}{l}
		$p$ \\
		\hline
	$p \vee q$
	\end{tabular}

\end{flushleft}

	\item \enter

	\begin{flushleft}
	\begin{tabular}{l}
		$p \Rightarrow q$ \\
		$q \Rightarrow r$ \\
	\hline
	$p \Rightarrow r$
	\end{tabular}

\end{flushleft}

\end{enumerate}

    \subsubsection*{Solution}

    \begin{enumerate}
    	\item \hspace{1em}

    \begin{center}
	\begin{tabular}{cc|ccc}
		$p$ & $q$ & $\lnot p$ & $p \lor q$ & $(\lnot p) \land (p \lor q)$ \\
		\hline
		T&T&F&T&F\\
		T&F&F&T&F\\
		F&\color{red}T&T&T&\color{red}T\\
		F&F&T&F&F\\
	\end{tabular}
    \end{center}

    On remarque que quand $(\lnot p) \land (p \lor q)$ est vrai, $q$ est vrai.

	\item  \hspace{1em}
    \begin{center}
    	\begin{tabular}{cc|c}
    		$p$ & $q$ & $p \lor q$ \\
    		\hline
    		\color{red}T&T&\color{red}T\\
    		\color{red}T&F&\color{red}T\\
    		F&T&T\\
    		F&F&F\\
    	\end{tabular}
    \end{center}

    On remarque que quand $p$ est vrai, $p \lor q$ est vrai.

	\item  \hspace{1em}
    \begin{center}
    	\begin{tabular}{ccc|cccc}
    		$p$ & $q$ & $r$ & $(p \Rightarrow q)$ & $\land$ & $(q \Rightarrow r)$ & $(p \Rightarrow r)$ \\
    		\hline
    		T&T&T&T&\color{red}T&T&\color{red}T\\
    		T&T&F&T&F&F&F\\
    		T&F&T&F&F&T&T\\
    		T&F&F&F&F&T&F\\
    		F&T&T&T&\color{red}T&T&\color{red}T\\
    		F&T&F&T&F&F&T\\
    		F&F&T&T&\color{red}T&T&\color{red}T\\
    		F&F&F&T&\color{red}T&T&\color{red}T\\
    	\end{tabular}
    \end{center}

    On remarque que quand $(p \Rightarrow q) \land (q \Rightarrow r)$ est vrai, $(p \Rightarrow r)$ est vrai.


\end{enumerate}
\subsection*{Exercice 3}
Démontrez que les arguments suivants ne sont pas valides.

\begin{enumerate}
	\item \enter

	\begin{flushleft}
	\begin{tabular}{l}
		$p \vee q$ \\
		$\neg p$ \\
	\hline
	$\neg q$
	\end{tabular}

\end{flushleft}

	\item \enter

	\begin{flushleft}
	\begin{tabular}{l}
		$p \Leftrightarrow q$ \\
		$p \Rightarrow r$ \\
	$r$ \\
	\hline
	$p$
	\end{tabular}

\end{flushleft}


% 	\item \enter
%
% 	\begin{flushleft}
% 	\begin{tabular}{l}
% 		$p \vee q$ \\
% 		$q$ \\
% 	\hline
% 	$p$
% 	\end{tabular}
%
% \end{flushleft}

	\item \enter

	\begin{flushleft}
	\begin{tabular}{l}
		$p \Rightarrow q$ \\
		$q \Rightarrow p$ \\
	\hline
	$p \wedge q$
	\end{tabular}

\end{flushleft}

% 	\item \enter
%
% 	\begin{flushleft}
% 	\begin{tabular}{l}
% 		$p \Rightarrow q$ \\
% 		$q$ \\
% 	\hline
% 	$p$
% 	\end{tabular}
% \end{flushleft}

\end{enumerate}

    \subsubsection*{Solution}
    Il y a deux façons de résoudre cet exercice.
    Nous faisons avec le premier un exemple de ces deux méthodes.
\begin{enumerate}
	\item
    Tout d'abord, l'algorithme de preuve :

    \begin{center}
    \begin{tabular}{|l|l|}
    \hline
    1. $p \lor q$ & Prémisse \\
    2. $\lnot p$ & Prémisse \\
    \hspace{0.5cm} 3. $\lnot q$ & Hypothèse \\
    \hspace{0.5cm} 4. $p$ & Syllogisme disjoint (1, 3) \\
    5. $q$ & Réduction à l'absurde \\
    \hline
    \end{tabular}
    \end{center}

    Ensuite une table de vérité :

    \begin{center}
    	\begin{tabular}{cc|ccc|c}
    		$P$ & $Q$ & $(P \lor Q) $ & $\land$ & $\neg P$ & $\neg Q$ \\
    		\hline
    		T & T & T & F & F & F\\
    		T & F & T & F & F & T\\
    		F & T & T & \color{red}T & T & \color{red}F\\
    		F & F & F & F & T & T\\
    	\end{tabular}
    \end{center}

    On constate que lorsque $(P \lor Q) \land \neg P$ est vrai, $\neg Q$ est faux.

	\item
    Il suffit de trouver une interprétation où c'est faux. Ici on peut prendre :

    \begin{flalign*}
    VAL_{I}(p) &= \text{false}\\
    VAL_{I}(q) &= \text{false}\\
    VAL_{I}(r) &= \text{true}\\
    \end{flalign*}

    Les prémisses sont vraies, pas la conclusion.

	\item
    Il suffit de trouver une interprétation où c'est faux. Ici on peut prendre :

    \begin{flalign*}
    VAL_{I}(p) &= \text{false}\\
    VAL_{I}(q) &= \text{false}\\
    \end{flalign*}

    Les prémisses sont vraies, pas la conclusion.
\end{enumerate}

\subsection*{Exercice 4}
Pour chaque ensemble de prémisses, démontrez la conclusion qui suit. Faites attention à bien identifier les
lois logiques et les règles d'inférence utilisées.
\begin{enumerate}

% \item Premisses: $A \vee B \vee C, \ \neg A, \ \neg B$ \\
% 			Conclusion: $C$
% \item Premisses: $(p \wedge q) \vee r$ \\
%       Conclusion: $\neg q \Rightarrow r$
\item Premisses: $p \Rightarrow q$, \ $q \Rightarrow r$ \\
      Conclusion: $p \Rightarrow r$
\item Premisses: $p \Rightarrow q$, \ $r \Rightarrow t$, \ $q \vee t \Rightarrow u$, \ $\neg u$ \\
      Conclusion: $\neg p \wedge \neg r$
\item Premisses: $\neg p \Rightarrow (q \Rightarrow r)$, \ $t \vee \neg r \vee u$, \ $p \Rightarrow t$, \ $\neg t$ \\
      Conclusion: $q \Rightarrow u$
% \item Premisses: $A \Rightarrow B, \ C \Rightarrow D, \ (B \vee D) \Rightarrow E, \ \neg E$ \\
% 			Conclusion: $\neg A \wedge \neg C$


\item Premisses: $p \Rightarrow \neg q, \ q \vee r \vee s, \ \neg r \vee s \Rightarrow p, \ \neg r$  \\
			Conclusion: $s$


\item Premisses: $\neg p \Rightarrow (q \Rightarrow r), \ s \vee \neg r \vee t, \ p \Rightarrow s, \ \neg s$ \\
			Conclusion: $q \Rightarrow t$


%   \item Premisses: $p \wedge q$ \\
%        ronclusion: $p \vee q$
%  \item Premisses: $(p \wedge q) \vee r$ \\
%        ronclusion: $r \vee q$
 \item Premisses: $\neg ( \neg p \wedge q), \ \neg (\neg q \vee r)$ \\
       Conclusion: $p$
%  \item Premisses: $p \vee q$ \\
%        ronclusion: $p \vee \neg \neg q$
 \item Premisses: $p \vee q, \ \neg q \vee r$ \\
       Conclusion: $p \vee r$
 \item Premisses: $(p \wedge q) \vee (r \wedge s), \ (q \wedge r) \vee (s \wedge t)$ \\
       Conclusion: $r \vee (p \wedge t)$

\end{enumerate}

    \subsubsection*{Solution}
    \begin{enumerate}

    %%% 4.1 %%%
	\item  \hspace{1em}
    \begin{center}
    \begin{tabular}{|l|l|}
    \hline
    1. $p \Rightarrow q$ & Prémisse \\
    2. $q \Rightarrow r$ & Prémisse \\
    \hspace{0.5cm} 3. $p$ & Hypothèse \\
    \hspace{0.5cm} 4. $q$ & Modus ponens (1, 3) \\
    \hspace{0.5cm} 5. $r$ & Modus ponens (2, 4) \\
    6. $p \Rightarrow r$ & Théorème de déduction (3, 5) \\
    \hline
    \end{tabular}
    \end{center}

    %%% 4.2 %%%
	\item  \hspace{1em}
    \begin{center}
    \begin{tabular}{|l|l|}
    \hline
    1. $p \Rightarrow q$ & Prémisse \\
    2. $r \Rightarrow t$ & Prémisse \\
    3. $q \lor t \Rightarrow u $ & Prémisse \\
    4. $\lnot u$ & Prémisse \\
    5. $\lnot (q \lor t)$ & Modus tollens (3, 4) \\
    6. $\lnot q \land \lnot t$ & De Morgan (5) \\
    7. $\lnot t$ & Simplification (6) \\
    8. $\lnot r$ & Modus tollens (2, 7) \\
    9. $\lnot q$ & Simplification (6) \\
    10. $\lnot p$ & Modus tollens (1, 9) \\
    11. $\lnot r \land \lnot p$ & Conjonction (8, 10) \\
    \hline
    \end{tabular}
    \end{center}

    %%% 4.3 %%%
	\item  \hspace{1em}
    \begin{center}
    \begin{tabular}{|l|l|}
    \hline
    1. $\lnot p \Rightarrow (q \Rightarrow r)$ & Prémisse \\

    2. $t \lor \lnot r \lor u$ & Prémisse \\
    3. $p \Rightarrow t$ & Prémisse \\
    4. $\lnot t$ & Prémisse \\
    5. $\lnot p$ & Modus tollens (3, 4) \\
    6. $q \Rightarrow r$ & Modus ponens (1, 5) \\
    7. $\lnot r \lor u$ & Syllogisme disjoint (2, 4) \\
    \hspace{0.5cm} 8. $q$ & Hypothèse \\
    \hspace{0.5cm} 9. $r$ & Modus ponens (6, 8) \\
    \hspace{0.5cm} 10. $u$ & Syllogisme disjoint (7,9) \\
    11. $q \Rightarrow u$ & Déduction (8, 10) \\
    \hline
    \end{tabular}
    \end{center}

    %%% 4.4 %%%
	\item  \hspace{1em}
    \begin{center}
    \begin{tabular}{|l|l|}
    \hline
    1. $p \Rightarrow \lnot q$ & Prémisse \\
    2. $q \lor r \lor s$ & Prémisse \\
    3. $\lnot r \lor s \Rightarrow p$ & Prémisse \\
    4. $\lnot r$ & Prémisse \\
    5. $q \lor s$ & Syllogisme disjoint (2, 4) \\
    6. $\lnot r \lor s$ & Addition(4) \\
    7. $p$ & Modus ponens (3, 6) \\
    8. $\lnot q$ & Modus ponens (1, 7) \\
    9. $s$ & Syllogisme disjoint (5, 8) \\
    \hline
    \end{tabular}
    \end{center}

    %%% 4.5 %%%
	\item  \hspace{1em}
    \begin{center}
    \begin{tabular}{|l|l|}
    \hline
    1. $\lnot p \Rightarrow (q \Rightarrow r)$ & Prémisse \\
    2. $s \lor \lnot r \lor t$ & Prémisse \\
    3. $p \Rightarrow s$ & Prémisse \\
    4. $\lnot s$ & Prémisse \\
    5. $\lnot p$ & Modus tollens (3, 4) \\
    6. $q \Rightarrow r$ & Modus ponens (1, 5) \\
    7. $\lnot r \lor t$ & Syllogisme disjoint (2, 4) \\
    \hspace{0.5cm} 8. $q$ & Hypothèse \\
    \hspace{0.5cm} 9. $r$ & Modus ponens (6, 8) \\
    \hspace{0.5cm} 10. $t$ & Syllogisme disjoint (7,9) \\
    11. $q \Rightarrow t$ & Déduction (8, 10) \\
    \hline
    \end{tabular}
    \end{center}

    %%% 4.6 %%%
	\item  \hspace{1em}
    \begin{center}
    \begin{tabular}{|l|l|}
    \hline
    1. $\lnot ( \lnot p \land q)$ & Prémisse \\
    2. $\lnot ( \lnot q \lor r)$ & Prémisse \\
    3. $p \lor \lnot q$ & De Morgan (1) \\
    4. $q \land \lnot r$ & De Morgan (2) \\
    5. $q$ & Simplification (4) \\
    6. $p$ & Syllogisme disjoint (3, 5) \\
    \hline
    \end{tabular}
    \end{center}

    %%% 4.7 %%%
	\item  \hspace{1em}
    \begin{center}
    \begin{tabular}{|l|l|}
    \hline
    1. $p \lor q$ & Prémisse \\
    2. $\lnot q \lor r$ & Prémisse \\
    3. $q \Rightarrow r$ & Loi de l'implication (2) \\
    \hspace{0.5cm} 4. $\lnot(p \lor r)$ & Hypothèse \\
    \hspace{0.5cm} 5. $\lnot p \land \lnot r$ & De Morgan (4) \\
    \hspace{0.5cm} 6. $\lnot p$ & Simplification (5) \\
    \hspace{0.5cm} 7. $\lnot r$ & Simplification (5) \\
    \hspace{0.5cm} 8. $q$ & Syllogisme disjoint (1, 6) \\
    \hspace{0.5cm} 9. $\lnot q$ & Syllogisme disjoint (2, 7) \\
    10. $p \lor r$ & Réduction à l'absurde (4)\\
    \hline
    \end{tabular}
    \end{center}

    %%% 4.8 %%%
	\item  \hspace{1em}
    \begin{center}
    \begin{tabular}{|l|l|}
    \hline
    1. $(p \land q) \lor (r \land s)$ & Prémisse \\
    2. $(q \land r) \lor (s \land t)$ & Prémisse \\
    3. $(p \lor r) \land (p \lor s) \land (q \lor s) \land (q \lor r)$ & Distributivité (1) \\
    4. $(q \lor s) \land (q \lor t) \land (r \lor s) \land (r \lor t)$ & Distributivité (2) \\
    5. $p \lor r$ & Simplification (3) \\
    6. $r \lor t$ & Simplification (4) \\
    7. $(p \lor r) \land (r \lor t)$ & Conjonction (5, 6) \\
    8. $r \lor (p \land t)$ & Distributivité (7) \\
    \hline
    \end{tabular}
    \end{center}
\end{enumerate}

\subsection*{Exercice 5}
Pour chaque ensemble de prémisses, démontrez la conclusion qui suit. Faites attention à bien identifier les
lois logiques et les règles d'inférence utilisées.
\begin{enumerate}
\item Premisses: \\
      Conclusion: $p \vee \neg (p \wedge q)$
\item Premisses: \\
      Conclusion: $(p \wedge q) \vee \neg p \vee \neg q$
\item Premisses: \\
      Conclusion: $\neg p \vee \neg (\neg q \wedge (\neg p \vee q))$
\end{enumerate}

    \subsubsection*{Solution}
    \begin{enumerate}

	\item  \hspace{1em}
    \begin{center}
    \begin{tabular}{|l|l|}
    \hline
    \hspace{0.5cm} 1. $\neg p$ & Hypothèse \\
    \hspace{0.5cm} 2. $\neg p \lor \neg q$ & Addition (1) \\
    \hspace{0.5cm} 3. $\neg \neg(\neg p \lor \neg q)$ & Négation (2) \\
    \hspace{0.5cm} 4. $\neg(p \land q)$ & De Morgan (3)\\
    5. $\neg p \Rightarrow \neg(p \land q)$ & Déduction (1, 4) \\
    6. $\neg \neg p \lor \neg(p \land q)$ & Implication (5)\\
    7. $p \lor \neg(p \land q)$ & Double négation (6)\\
    \hline
    \end{tabular}
    \end{center}

	\item  \hspace{1em}
    \begin{center}
    \begin{tabular}{|l|l|}
    \hline
    \hspace{0.5cm} 1. $\neg (p \land q)$ & Hypothèse \\
    \hspace{0.5cm} 2. $\neg p \lor \neg q$ & De Morgan (1) \\
    3. $\neg (p \land q) \Rightarrow (\neg p \lor \neg q)$ & Déduction (1, 2) \\
    4. $(\neg \neg(p \land q)) \lor (\neg p \lor \neg q)$ & Implication (3) \\
    5. $(p \land q) \lor (\neg p \lor \neg q)$ & Double négation (4) \\
    \hline
    \end{tabular}
    \end{center}

	\item  \hspace{1em}
    \begin{center}
    \begin{tabular}{|l|l|}
    \hline
    \hspace{0.5cm} 1. $p$ & Hypothèse \\
    \hspace{0.5cm} 2. $p \lor q$ & Addition (1) \\
    \hspace{0.5cm} 3. $(\neg p \lor q) \Leftrightarrow q$ & ex. 2.1 \\
    \hspace{0.5cm} 4. $((\neg p \lor q) \land \neg q) \Leftrightarrow (q \land \neg q)$ & Mystification \\
    \hspace{0.5cm} 5. $\neg ((\neg p \lor q) \land \neg q)$ & Contradiction \\
    6. $p \Rightarrow \neg ((\neg p \lor q) \land \neg q)$ & Déduction (1, 5) \\
    7. $\neg p \lor \neg ((\neg p \lor q) \land \neg q)$ & Implication (6) \\
    \hline
    \end{tabular}
    \end{center}

    \end{enumerate}

\newpage
\section{}
\subsection{Exercise 1 (Blue-ray security)}
\copypaste{2}{7}

\subsection{Exercise 2 (Authenticated encryption)}
Let $\Pi = \langle \Gen, \Enc, \Dec\rangle$ be an authenticated encryption
scheme such that $\Enc$ encrypts messages of $n$ bits.
%
Do the following systems provide authenticated encryption?  For those
that do, briefly explain why.  For those that do not, present an
attack that breaks one of the security properties of an authenticated
encryption scheme.

\begin{enumerate}
	\item $\Pi' = \langle \Gen, \Enc', \Dec'\rangle$ with
	$\Enc'_k(m) = (\Enc_k(m), \Enc_k(m \oplus (0^{n-1}\|1)))$ and
	$\Dec'_k(c_1, c_2) = \Dec_k(c_1)$ if
	$\Dec_k(c_1) \oplus \Dec_k(c_2) = 0^{n-1}\|1$ and $\bot$ otherwise.

	\item $\Pi' = \langle \Gen, \Enc', \Dec'\rangle$ with
	$\Enc'_k(m) = (\Enc_k(m), \Mac_k(m))$ and $\Dec'_k(c_1, c_2) = \Dec_k(c_1)$
	
	if $\Vrfy_k(\Dec_k(c_1), c_2)=1$ and $\bot$ otherwise. Here, $\Mac$
	and $\Vrfy$ are deterministic algorithms that are part of a secure
	MAC scheme that is compatible with $\Gen$.

\end{enumerate}

\begin{solution}
$\Pi := \langle \text{Gen, Enc, Dec} \rangle$ is an authenticated encryption scheme (AE) if it is CCA-secure and unforgeable.
\begin{enumerate}
    \item The system $\Pi'$ is not AE because it is \textbf{forgeable} and we can show it with this example. If the adversary A asks for the message $m$ ($m'$ corresponds to the message m with the last bit changed) to the oracle access, he will receive the cipher text $(c_1, c_2)$, where $c_1 = Enc_k(m)$ and $c_2 = Enc_k(m\xor 0^{n-1}||1) = Enc_k(m')$ . 
    
    If $\A$ outputs the pair (m',$(c_2, c_1)$), this is a forgery. 
    
    $Dec_k'(c_2,c_1) = Dec_k(c_2) = Dec_k(Enc_k(m')) = m' \neq \perp $ because $Dec_k(c_2) \oplus Dec_k(c_1) = m' \oplus m = m \oplus 0^{n-1}||1 \oplus m = 0^{n-1}||1 $. And $m'$ has not been requested before.
    
    Then we have EncForge$_{A, \Pi'}$(n) = 1  and Pr[EncForge$_{A, \Pi'}$(n)] = 1. $\Pi'$ is then forgeable and it is not an AE. 
    \newline
    With the same technique, an adversary can break the CCA-security of this scheme by querying two different messages $m_1$ and $m_2$, obtaining their encryption, sending \newline
    ($m'_1,m'_2$) = ($m_1 \oplus 0^{n-1}||1, m_2 \oplus 0^{n-1}||1$) for the challenge, and compare the encryption of $m'_b$ with the two previously received ciphertexts.
    \item The sytem $\Pi'$ is not AE because it is not \textbf{CCA-secure} and we can show it because $Mac_k(m)$ does not assure any security (only authentication). So if We use as Mac : 
    $$ Mac_k(m) = m||Mac'_k(m) $$
    It is a good mac but it is trivial to show that is does not hold $CCA-Secure$. $\Pi'$ is then not an AE.
\end{enumerate}
\end{solution}

\subsection{Exercise 3 (Euclidean algorithm for gcd)}
Let $a,b \in \mathbb{Z}$ , $b \neq 0$, consider the following algorithm, presented in Algorithm~\ref{algo:gcd}. ($r=a \% b$ means that $a=qb+r$ where $q$ is the quotient and $r$ is the remainder).

Prove that $x$, the value returned by Algorithm~\ref{algo:gcd}, is $\mathsf{gcd} (a,b)$.

Hint:
\begin{itemize}
\item Prove that $x$ divides  $ \mathsf{gcd} (a,b)$
\item Prove that $ \mathsf{gcd} (a,b)$ divides $x$
\end{itemize}

\begin{algorithm}
	\KwIn{$a$, $b$}
	\KwOut{$\mathsf{gcd}(a,b)$}
	
	\While{ $b\neq 0$}
	{
		$r \leftarrow a\%b$\;
		
		$a \leftarrow b$\;	
		
		$b \leftarrow r$\;	
	}
	\Return($a$)
	
\caption{The Euclidean $\mathsf{gcd}$ algorithm.}\label{algo:gcd}
\end{algorithm}
\begin{solution}
According to the algorithm, we will have as successive value for the different remainder : 
$$ (r_2 = r_0 \% r_1, r_3 = r_1 \% r_2, \ r_4 = r_2 \% r_3, \ ... \ , r_n =  r_{n-2} \% r_{n-1})$$
Where $r_0 = a$, $r_1 = b$ and $r_n$ is the last non null remainder. Then we have the property that :
$$ gcd(r_{i}, r_{i+1}) = gcd(r_{i+1}, r_{i+2}) \ \forall i : \ 0 \leq i \leq n - 2 $$
Otherwise if it was not the case, $\exists i < n $ such that $r_i = 0$. But as $r_n$ is the last non null remainder, we prove by contradiction this property.

As $gcd(r_{n-1}, r_n) = r_n $ because $r_n | r_{n-1}$ (since $r_{n+1} = r_{n-1} \% r_n = 0$), we can conclude that 
$$ gcd(r_0, r_1) = gcd(a, b) = gcd(r_{n-2}, r_{n-1}) = r_n$$
We have proved the value returned by the algorithm is the $gcd(a,b)$
\end{solution}

\subsection{Exercise 4}
Consider the group $\mathbb{Z}^{\ast}_{17}$.
\begin{enumerate}
\item Compute $5^{-1}$.
\item Compute $3^2$, $3^3$ and $3^4$.
\item Does $3$ generate the group?
\item Find $\log_{7}(11)$.
\end{enumerate}
\begin{solution}
Here because p is not too big, it is possible to evaluate "quickly" and "intuitively" the solutions. If it is too hard, there is an algorithm in the slides. 
\begin{enumerate}
    \item Because 35 mod 17 = 1, and $5 \cdot 7 = 35$. \newline Then $5^{-1} = 7$.  ($5 \cdot 7 = 1 \text{ (mod 17)}$)
    \item \begin{itemize}
        \item $3^2 = 9 \text{ (mod 17)}$
        \item $3^3 = 3^2 \cdot 3 = 27 \text{ (mod 17)} = 10 \text{ (mod 17)}$
        \item $3^3 = 3^3 \cdot 3 = 30 \text{ (mod 17)} = 13 \text{ (mod 17)}$
    \end{itemize}
    \item According to \textit{Fermat's little theorem}, if ord(g) = i then if $i|m = |G|$, where G is the commutative group. To see if 3 generate the group, we have to check if $3^i \neq 1$ where $i$ are the divisor of $(p-1) = 16$ (except 16 of course !).
    \begin{itemize}
        \item $3^1$ = 3 mod 17 (trivial)
        \item $3^2$ = 9 mod 17 (evaluated previously)
        \item $3^4$ = 13 mod 17 (evaluated previously)
        \item $3^8$ = $(3^4)^2$ = $(13)^2 \text{ (mod 17)}$ = $(-4)^2 \text{ (mod 17)}$ = 16 mod 17
    \end{itemize}
    We can see that 3 is a generator of the group. 
    
    \textbf{P.S.} : The trick here is to remember the property of the modulo operation here in a $Z^*_p$:
    $$ x = -(p - x) \text{ (mod p)} $$  
    It can make a lot of computing easier (it can become a real pain in the ass).
    \item Here we have to find x such that : 
    $$ 7^x = 11 \text{ (mod 17)} $$
    After (boring) computations, we have here :
    \begin{itemize}
        \item $7^1$ = 7 mod 17
        \item $7^2$ = 15 mod 17 = -2 mod 17
        \item $7^3$ = -14 mod 17 = 3 mod 17
        \item $7^4$ = 4 mod 17
        \item $7^5$ = 11 mod 17 (Bingo)
    \end{itemize}
    Then $log_7(11)$ = 5
\end{enumerate}
\end{solution}

\subsection{Exercise 5 (Group order)}
In this exercise we consider the group $\mathbb{Z}_{59}^*$.

\begin{enumerate}
	\item What is the order of $58$?
	\item What are the possible orders of an element of this group?
	\item Find an element of order more than $20$.
	
\end{enumerate}
\begin{solution}
The order of $g \in \mathbb{Z}^*_{59}$ is the smallest $i$ where $g^i = 1$ 
\begin{enumerate}
    \item ord(58) = 2 because :
    \begin{itemize}
        \item  $58^1$ = 58 mod 59 = -1 mod 59
        \item  $58^2$ = -58 mod 59 = 1 mod 59
    \end{itemize}
    \item According to \textit{Fermat's little theorem}, the possible orders of a group $\mathbb{Z}^*_{p}$ are the divisor of p-1. Then, the possible orders are : 
    \begin{itemize}
        \item 1
        \item 2
        \item 29
        \item 58
    \end{itemize}
    \item The best strategy here is to find a number g where 
    $$ ord(g) > 2 $$
    (To assure you this is correct, just look at the possible ordrers).  \newline
    2 is a correct candidate. 
\end{enumerate}
\end{solution}

\subsection{Exercise 6 (Decisional Diffie-Hellman and \texorpdfstring{$\mathbb{Z}_p^\ast$}{Zp*})}
\label{subsec:4.6}
The goal of this exercise is to show that in some groups DDH and CDH assumptions are conjectured not equivalent, as DDH is easy whereas CDH is conjectured to be hard.

\begin{enumerate}
	\item For all element $a$ of $\mathbb{Z}_{11}^*$, compute $a^2 \mod 11$.
	
	For a prime number $p$, we denote $QR_p$ the set $\{x \in \mathbb{Z}_{p}^* \; | \; \exists a\in \mathbb{Z}_{p}^*, a^2=x\}$, such $x$ are called quadratic residues modulo $p$. Show that if $p$ is odd then $|QR_p|=\frac{p-1}{2}$.
	
		\item For all element $a$ of $\mathbb{Z}_{11}^*$, compute $a^5 \mod 11$. Show that for any odd prime $p$, $x \in QR_p \Leftrightarrow x^{\frac{p-1}{2}}= 1 \mod p$, and that $x \not \in QR_p \Leftrightarrow x^{\frac{p-1}{2}}= -1 \mod p$.
	
	\item Show that $2$ is a generator of $\mathbb{Z}_{11}^*$. For the following pairs $(a,b)$, compute $g^a, g^b$ and $g^{ab}$ in $\mathbb{Z}_{11}^*$ where $g=2$:
	\begin{itemize}
		\item $(2,8)$,
		\item $(1,4)$,
		\item $(3,5)$.
	\end{itemize}
    Show that for $p$ an odd prime, $g^{ab} \not \in QR_p \Leftrightarrow g^a \not \in QR_p \text{ and } g^b \not \in QR_p$.
	
	\item Show that DDH does not hold in $\mathbb{Z}_{p}^*$ with $p$ an odd prime.
\end{enumerate}
\begin{solution}
For this exercise we will work with $\Z_{11}^* = \{1,2,3,4,5,6,7,8,9,10\}$
\begin{enumerate}
    \item 
		For all element $a$ of $\Z_{11}^*$, I've calculated $a^2$ mod $11$.
		$$1^2 = 1 \quad 2^2 = 4 \quad 3^2 = 9 \quad 4^2 = 5 \quad 5^2 = 3 \quad 6^2 = 3 \quad 7^2 = 5 \quad 8^2 = 9 \quad 9^2 = 4 \quad 10^2 = 1$$
		We see that with $p$ odd, we have $\left|QR_p\right| = \frac{p-1}{2}$. We can show it with this development:
		
	\item
		For all element $a$ of $\Z_{11}^*$, I've calculated $a^5$ mod $11$.
		$$1^5 = 3^5 = 4^5 = 5^5 = 9^5 = 1 \qquad 2^5 = 6^5 = 7^5 = 8^5 = 10^5 = 10$$
		We can see that for $p$ prime, we have $x \in QR_p \Leftrightarrow x^{\frac{p-1}{2}} = 1 \mod p$ and  $x \notin QR_p \Leftrightarrow x^{\frac{p-1}{2}} = p-1 \mod p$.
		\begin{itemize}
			\item $x \in QR_p \Leftrightarrow x^{\frac{p-1}{2}} = 1 \mod p$:\\
			We know that $$x \in QR_p \Leftrightarrow \exists a \st x = a^2 \mod p$$
			So we have now $$x = a^2 \mod p \Leftrightarrow x^{\frac{p-1}{2}} = 1 \mod p$$
			If we replace $x$ by $a$ we obtain $a^{2^{(\frac{p-1}{2})}} = 1 \mod p$.\\ 
			But also more simply $a^{p-1} = 1 \mod p$ which is true by the group theory.
			\item $x \notin QR_p \Leftrightarrow x^{\frac{p-1}{2}} = p-1 \mod p$:\\
			We know that $$x \notin QR_p \Leftrightarrow \exists a \st x = a^{1+2n} \mod p$$
			So we have now $$x = a^{1+2n} \mod p \Leftrightarrow x^{\frac{p-1}{2}} = -1 \mod p$$
			We replace $x$ by $a$ and we get $$a^{\frac{p-1}{2}} a^{n(p-1)} \mod p = -1 \mod p$$ 
			We know that $ a^{n(p-1)} \mod p = 1$, thus we simplify the equation like $$a^{\frac{p-1}{2}} \mod p = -1 \mod p$$ 
			We know that $g = a^{\frac{p-1}{2}} \mod p \ne 1 \mod p$ but $g^2 = a^{p-1} = 1 \mod p$. The only solution of these two equations is $g = -1 \mod p$ which is equivalent to $$x^{\frac{p-1}{2}} = p-1 \mod p$$
		\end{itemize}
	
	\item
		The number $2$ is a generator of $\Z_{11}^*$, because ord($2$) $= 10$. In fact, we have $2^1 = 2$, $2^2 = 4$, $2^5 = 10$ and $2^10 = 1$. (Fermat's little theorem)
		We have $g = 2$ so:
		\begin{itemize}
			\item $(2,8):\quad g^2 = 4$, $g^8 = 3$ and $g^{16} = -2$ 
			\item $(1,4):\quad g^1 = 2$, $g^4 = 5$ and $g^{4} = 5$
			\item $(3,5):\quad g^3 = -3$, $g^5 = -1$ and $g^{15} = -1$ TODO
		\end{itemize}
		We have to show that $g^{ab} \notin QR_p \Leftrightarrow g^a \notin QR_p \text{ and } g^b \notin QR_p$.\\
		We know by the definition of the $QR_p$ set that
		$$g^n \notin QR_p \Leftrightarrow \exists m \st n = 2m+1$$
		We can thus extract from $g^{ab} \notin QR_p$ that $\exists m \st ab = 2m+1$.
	    
	    % not necessary
		%We can do a proof by contradiction:\newline 
		%If $a = 2v$ then $ab = 2bv$ and can not be equal to $2m + 1$.\\
		%If $b = 2v$ then $ab = 2av$ and can not be equal to $2m + 1$.\\
		
		So we are assured that $a$ and $b$ are not pairs, so we have the relation $g^a \notin QR_p \text{ and } g^b \notin QR_p$ if and only if $ab = 2m+1$ which is equivalent to $g^{ab} \notin QR_p$. That was what we had to proof.
		
	\item
		We have to show that DDH does not hold in $\Z_p^*$ with p an odd prime number.
		
		We define an attacker that can see $p$, $g$, $g^a$, $g^b$ and receive $h_b = g^{ab}$ or $g^z$.\\ 
		The attitude of the attacker will be this one:
		\begin{itemize}
			\item It receives $g^a \notin QR_p$ and $g^b \notin QR_p$:\\
				It will answer in function of $h_b$:
				\begin{itemize}
					\item $h_b \notin QR_p$:\\
						It answers $h_b = g^{ab}$
					\item $h_b \in QR_p$:\\
						It answers $h_b = g^z$
				\end{itemize}
			\item It receives $g^a \in QR_p$ or $g^b \in QR_p$:\\
				It answers randomly.
		\end{itemize}
		We can identify four cases with their chances of success and appearance (we already know that $\left|QR_p\right|$ is of size $\frac{p-1}{2}$):
		\begin{enumerate}[a)]
			\item $g^a \in QR_p$ or $g^b \in QR_p$ appears $3/4$ of the time with success = $1/2$.
			\item $g^a \notin QR_p$ and $g^b \notin QR_p$ with $h_b = g^{ab}$  appears $1/8$ of the time with success = $1$.
			\item $g^a \notin QR_p$ and $g^b \notin QR_p$ with $h_b = g^z$ and $g^z \in QR_p$  appears $1/16$ of the time with success = $1$.
			\item $g^a \notin QR_p$ and $g^b \notin QR_p$ with $h_b = g^z$ and $g^z \notin QR_p$  appears $1/16$ of the time with success = $0$.
		\end{enumerate}
		We can now recalculate the expected value of success of our attacker:
		$$\begin{array}{rcl}
			\mathbb{E}(success) &=& \frac{3}{4}\cdot \frac{1}{2} +  \frac{1}{8}\cdot 1 + \frac{1}{16}\cdot 1 + \frac{1}{16}\cdot 0\\
								&=& \frac{1}{2} + \frac{1}{16}
		\end{array}$$
		This attacker has one sixteenth of probability more than one half which is not a negligible function (in fact it is a constant function). It is not DDH secure.
\end{enumerate}
\end{solution}
\newpage
\section{TP 4}
%\addcontentsline{toc}{section}{TP 4}


% \newpage

% \section*{Rappel}

% \textbf{Forme normale conjonctive (FNC):} \\

% $$
% \bigwedge_{i = 1}^n \bigvee_{j = 1}^{m_i} L_{ij}
% $$

% avec $L_{ij}$ litéral, cet-à-dire, une proposition primaire (ex. $A$) ou sa négation (ex. $\neg A$). \\

% \textbf{Example:}

% $$
% (\neg A \vee B) \wedge (B \vee C) \wedge (\neg B \vee C \vee \neg D)
% $$

% \vspace{0.5cm}

% \textbf{Algorithme de normalisation:} \\

% \begin{enumerate}
%  \item Eliminer les $\Rightarrow$ et $\Leftrightarrow$ en les remplaçant par des formules équivalentes a l'aide des lois de l'implication et
%  de l'équivalence. 
%  $$
%  p \Rightarrow q \loeq \neg p \vee q \quad \text{et} \quad p \Leftrightarrow q \loeq (p \Rightarrow q) \wedge (q \Rightarrow p)
%  $$
 
%  \item Déplacer les négations vers l'intérieur et utilisant les lois de De Morgan.
 
%  \item Déplacer les disjonctions ($\vee$) vers l'intérieur en utilisant les lois distributives.
 
%  \item Simplifier en éliminant les formes ($P \vee \neg P$) dans chaque disjonction.
 
% \end{enumerate}

% \vspace{0.5cm}

% \textbf{Règle de résolution:} \\

% \begin{tabular}{l}
% $p_1 \vee p_2 \vee \ldots \vee p_{i - 1} \vee c \vee p_{i + 1} \vee \ldots \vee p_n$ \\
% $q_1 \vee q_2 \vee \ldots \vee q_{j - 1} \vee \neg c \vee q_{j + 1} \vee \ldots \vee q_m$ \\
% \hline
% $p_1 \vee \ldots \vee p_{i - 1} \vee p_{i + 1} \vee \ldots \vee p_n \vee q_1 \vee \ldots \vee q_{j - 1} \vee q_{j + 1} \vee \ldots \vee q_m$ 
% \end{tabular}

% \vspace{0.5cm}

% \textbf{Example:}

% \begin{tabular}{l}
% $p \vee \neg q $ \\
% $\neg r \vee q$ \\
% \hline
% $p \vee \neg r$ 
% \end{tabular}

% \newpage

% \textbf{Algorithme de résolution:} \\

% \begin{enumerate}
%  \item Mettre les prémisses et la négation en FNC dans $S$
%  \item Tant que \textbf{false} $\notin S$ et qu'il existent $p, q$ résolvables, non résolues
%  \begin{enumerate}
%   \item Apliquer la résolution sur $p$ et $q$
%   \item Ajouter le résultat dans $S$
%  \end{enumerate}
%   \item Si \textbf{false} $\in S$ alors c'est prouvé
%   \item Sinon il n'y a pas de preuve
% \end{enumerate}

% \vspace{0.5cm}

% \textbf{Propriétés de l'algorithme:} \\

% Soit $P = \{p_1, \ldots p_n \}$ l'ensemble de prémisses et $c$ une proposition. \\

% \textit{Soundness (adéquat):} Si $P \vdash c$ alors $P \models c$ 

% \textit{Completeness (complet):} Si $P \models c$ alors $P \vdash c$

% \textit{Decidable (décidable):} L'algorithme se termine après un nombre fini d'étapes.

% \newpage

% \section*{Exercices}

\subsection*{Exercice 1}
Prouvez que la régle de résolution est valide.

\paragraph*{NB:} Règle de résolution : 
 \begin{tabular}{l}
 $p_1 \vee p_2 \vee \ldots \vee p_{i - 1} \vee c \vee p_{i + 1} \vee \ldots \vee p_n$ \\
 $q_1 \vee q_2 \vee \ldots \vee q_{j - 1} \vee \neg c \vee q_{j + 1} \vee \ldots \vee q_m$ \\
 \hline
 $p_1 \vee \ldots \vee p_{i - 1} \vee p_{i + 1} \vee \ldots \vee p_n \vee q_1 \vee \ldots \vee q_{j - 1} \vee q_{j + 1} \vee \ldots \vee q_m$ 
 \end{tabular}

 \vspace{0.5cm}

    \subsubsection*{Solution}

La règle de résolution peut s'écrire comme suit :

\begin{center}
\begin{tabular}{c}
$p \lor q$ \\
$r \lor \neg q$\\
\hline
$p \lor r$\\
\end{tabular}\\
\end{center}

Elle peut être démontrée de la manière suivante :

\begin{tabular}{|l|l|}
\hline
1. $p \lor q$ & prémisse \\
2. $r \lor \neg q$ & prémisse \\
\indent 3. $\neg (p \lor r)$ & supposition \\
\indent 4. $\neg p \land \neg r$ & De Morgan 2 \\ 
\indent 5. $\neg p$ & simplification \\
\indent 6. $q$ & syllogisme disjoint (1, 5) \\
\indent 7. $\neg r$ & simplification (4)\\
\indent 8. $\neg q$ & syllogisme disjoint (2, 7) \\
9. $\neg (\neg (p \lor r))$ & contradiction (3) \\
10. $p \lor r$ & négation \\
\hline
\end{tabular}

\subsection*{Exercice 2}
Est-ce que cette application de la règle de résolution est correcte? Justifiez.
$$
res(P \vee Q, \neg P \vee \neg Q) = \textbf{false}
$$

    \subsubsection*{Solution}
    
res($P \lor Q$, $\neg P \lor \neg Q$) = \textbf{false} n'est pas une application valable de la règle de résolution.\\
En effet, si on prend $R$ tel que $R \Lleftarrow\!\!\!\!\Rrightarrow \neg P$, on peut appliquer :
\begin{center}
\begin{tabular}{c}
$P \lor Q$ \\
$R \lor \neg Q$\\
\hline
$P \lor R$\\
\end{tabular}\\
\end{center}
On obtient donc $P \lor R$.
Or $(P \lor R) \Lleftarrow\!\!\!\!\Rrightarrow (P \lor \neg P)$.
Puisque $(P \lor \neg P)$ est toujours vrai, on en conclut que res($P \lor Q$, $\neg P \lor \neg Q$) = \textbf{true}.

\subsection*{Exercice 3}
Pour chaque ensemble de prémisses montrez avec l'algorithme de résolution que la conclusion est une conséquence logique des prémisses.
\begin{enumerate}
 \item 
 Prémisses: $P \Rightarrow Q, Q \Rightarrow R$ \\
 Conclusion: $P \Rightarrow R$
 \item
 Prémisses: $P \vee Q, P \Rightarrow R, Q \Rightarrow R$ \\
 Conclusion: $R$
 \item
 Prémisses: $(P \Leftrightarrow Q) \Leftrightarrow R, P$ \\
 Conclusion: $Q \Leftrightarrow R$
 \item
 Prémisses: $P \Rightarrow Q, R \Rightarrow T, Q \vee T \Rightarrow U, \neg U$ \\
 Conclusion: $\neg P \wedge \neg R$
 \item
 Prémisses: $\neg P \Rightarrow (Q \Rightarrow R), T \vee \neg R \vee U, P \Rightarrow T, \neg T$ \\
 Conclusion: $Q \Rightarrow U$
\end{enumerate}

    \subsubsection*{Solution}
  
    \subsection*{Rappel : Algorithme de résolution}
    
        \begin{enumerate}
            \item On met les prémisses et la $\neg$ conclusion en FNC
            \item $While(false \notin S$ AND p q resolvable)
            \begin{itemize}
                \item Résolution sur p et q
                \item Ajouter résultat dans S
            \end{itemize}
            \item Si $false \in S \rightarrow Proof$
            \item Sinon $\rightarrow No proof $
        \end{enumerate}
        
    \subsection*{1.}
    
    \begin{table}[H]
        \centering
        %\caption{My caption}
        \label{my-label}
        \begin{tabular}{lll}
        $ p \Rightarrow q$ & \textbf{Conversion} & $ \neg p \lor q $  \\
        $ q \Rightarrow r$ &                     & $ \neg q \lor r $  \\ \cline{1-1} \cline{3-3} 
        $ p \Rightarrow r$ &                     & $ p \land \neg r $
        \end{tabular}
    \end{table}
    
    
    \begin{flalign*}
        &S = \left \{ \neg p \lor q, \neg q \lor r,    
        \neg r \right \} &\\
        & res \left (\neg p \lor q, \neg q \lor r \right ) = \neg p \lor r &\\
        &S = \left \{ \neg p \lor q, \neg q \lor r, p \land \neg r , \neg p \lor r \right \}&\\
        & res \left (\neg p \lor r, \neg r \right ) = \neg p &\\
        &S = \left \{ \neg p \lor q, \neg q \lor r, p \land \neg r , \neg p \lor r, \neg p \right \}&\\
        & res \left (\neg p , p \right ) = \textbf{false} &\\
    \end{flalign*}
    
    $p \Rightarrow r$ est une conséquence logique des prémisses.\\
        
        
    \subsection*{2.}
    
    On convertit les prémisses, et on les ajoute dans S (ainsi que $\lnot \text{conclusion}$).\\
    
    \begin{flalign*}
        &S = \left\{ P \lor Q, \lnot P \lor R, \lnot Q \lor R, \lnot R, P\right\} &\\
        & res \left (P \lor Q,  \lnot P \lor R \right ) = Q \lor R &\\
        &S = \left\{ Q \lor R, \lnot Q \lor R, \lnot R \right\} &\\
        & res \left (Q \lor R, \lnot Q \lor R, \right ) = R \lor R = R &\\
        &S = \left\{ R, \lnot R \right\} &\\
        & res \left (R, \lnot R \right ) = \textbf{false} &\\
    \end{flalign*}
    
    $R$ est une conséquence logique des prémisses.\\
        


\subsection*{Exercice 4}
Demontrez que $p_1 \wedge \ldots \wedge p_n \models c$ si et seulement si $p_1 \wedge \ldots \wedge p_n \wedge \neg c \models \textbf{false}$.

    \subsubsection*{Solution}

\noindent \fbox{$\implies$} Par hypothèse, tous les modèles de $p_1 \wedge \ldots \wedge p_n$ sont des modèles de $c$. Autrement dit, $c$ est vraie à chaque fois que  $p_1,p_2,...,p_n$ sont vrais en même temps. Par conséquent, la formule $p_1 \wedge \ldots \wedge p_n \wedge \neg c$ est fausse pour tous les modèles de $p_1 \wedge \ldots \wedge p_n$. Autrement dit, $p_1 \wedge \ldots \wedge p_n \wedge \neg c$ n'a aucun modèle ou $p_1 \wedge \ldots \wedge p_n \wedge \neg c \models \textbf{false}$. \\

\noindent \fbox{$\impliedby$} Soit $M$ un modèle quelconque de  $p_1 \wedge \ldots \wedge p_n$. Puisqu'il n'y a aucun modèle pour  $p_1 \wedge \ldots \wedge p_n \wedge \neg c$, $c$ doit être vraie dans $M$ pour satisfaire le membre de droite. Autrement dit, tous les modèles de $p_1 \wedge \ldots \wedge p_n$ sont des modèles de $c$ ou $p_1 \wedge \ldots \wedge p_n \models c$.
\newpage

\section{}
\subsection{Exercise 0 (A variation of ElGamal in \texorpdfstring{$PKE$}{PKE})}
\copypaste{4}{4}

\subsection{Exercise 1 (Commitment scheme)}
\copypaste{4}{7}

\subsection{Exercise 2}
By design secure public-key encryption schemes are perfectly binding commitment schemes (which are also computationally hiding, why?). 
Then, if perfect hiding property is not a concern, do commitment schemes really consist of a new usefull cryptographic building block? 
This exercise aims to build a perfectly hiding commitment scheme which supports a \emph{batching} property that encryption schemes
cannot achieve. \medskip \\
Let $p$ be a prime and let $g \in QR(p)$ be an element of prime order $q>2^l$. 
We let $G$ denote the group generated by $g$ and we let $I$ denote the set of integers $\{1,\dots, q\}$. 
Fix $n$ random values $g_1,\dots, g_n \in G$ and define the commitment function $\Com : I^n \rightarrow G$ by 
$$\Com(x_1,\dots, x_n\,; r) = g^r g_1^{x_1}g_2^{x_2}\cdots g_n^{x_n}$$
\begin{enumerate}
  \item Describe formally the commitment scheme. Discuss its efficiency and its correctness.
	\item Show that the scheme is computationally binding assuming that DLog is intractable in $G$. 
	      That is, show that an adversary computing two openings of a commitment $c$ for random 
				$g,g_1,\dots, g_n \in G$ can be used to compute discrete-log in $G$.\\
				\emph{Hint:} given a pair $g, h \in G$ your goal is to find an $\alpha \in \mathbb{Z}_q$ such that $g^\alpha = h \mod p$. 
				             Choose $g_1,\dots, g_n \in G$ so that two valid openings will reveal $\alpha$.
	\item Show that the scheme results in a perfectly hiding commitment on several messages. Compare the size of the construction
	      with respect to an encryption (viewed as a commitment) of all these messages.
\end{enumerate}


\begin{solution}
  \begin{enumerate}
    \item
      We define $\langle \Gen, \Com, \Open \rangle$ as:
      \begin{itemize}
        \item $\Gen(1^n, 1^l)$ sets $pk$ as $(p,q,g)$ where $q > 2^l$ ang $g$ has order $q$ modulo $p$ (since $\phi(p)$ is even, that means that $g \in QR(p)$).
        \item $\Com_{pk}(x_1, \ldots, x_n)$ provides $(c,d)$ where:
          \begin{itemize}
            \item $c := g^r g_1^{x_1} g_2^{x_2} \cdots g_n^{x_n}$ (for a random $r \in \mathrm{Z}_q$)
            \item $d := r,x_1,...,x_n$
          \end{itemize}
        \item $\Open_{pk}(c,d)$ outputs $x_1, \ldots, x_n$ if it can recompute $c$ from $d$ and $pk$,
          or $\perp$ otherwise.
      \end{itemize}
      We can see that there are different possible $x_1, \ldots, x_n$ that are valid.
      If we fix $x_2, \ldots, x_n$,
      there is an $x_1$ such that $g_1^{x_1} = c / (g^r g_2^{x_2} \cdots g_n^{x_n})$
      so there is $q^n$ possible opening.
      However, it is not easy to find for a PPT algorithm.

      I should maybe have defined $d := (r, x_1, \ldots, x_n)$ because here it is weird because $\Open$ can have different outputs.
    \item %\textcolor{red}{Olivier : pas trop d'accord avec ce qu'ils disent.}
    
      For a random $\alpha$, we need to find it from $g^\alpha$.

      Pick a random $i^*$ and set $g_{i^*} = g^\alpha$ and for $i \neq i^*$, random $\alpha_i$ and $g_i = g^{\alpha_i}$.
      From $g^rg_1^{x_1} \cdots g_n^{x_n} = g^{r'}g_1^{x_1'} \cdots g_n^{x_n'}$ we get
      \[ r + \alpha_1 x_1 + \cdots + \alpha_n x_n \equiv r' + \alpha_1 x_1' + \cdots + \alpha_n x_n' \pmod{q} \]
      so we get
      \[ (x_{i^*} - x_{i^*}') \alpha \equiv r' - r + \sum_{i \neq i^*}^n (x_i' - x_i) \alpha_i \pmod{q} \]
      We know that at least one $x_i \neq x_i'$ so we have at least one chance out of $q$ that $x_{i^*} \not\equiv x_{i^*}' \pmod{q}$.
      If this is the case, we can find the inverse of $(x_{i^*} - x_{i^*}')$ and solve the $\DLog$ problem.
    \item
      As r is randomly selected in $\mathrm{Z}_q$ and g generates G, $g^rg_1^{x_1}...g_n^{x_n}$ could be any element of G, whatever the values of ($g_1,...,g_n$) and ($x_1,...,x_n$) so the commitment looks random and is thus perfectly hiding.

      The size is $n$ times smaller.
  \end{enumerate}
\end{solution}

\subsection{Exercise 3 (Zero knowledge Petersen)}
The Schnorr protocol, used to prove the knowledge of discrete
logarithm, is (honest-verifier) zero-knowledge. However, the value
$y=g^x \pmod{p}$ (for a safe prime $p=2q+1$) leaks some information
about the discrete logarithm $x$ (since for a given generator $g$ of
order $q$ there is exactly one such $x$ in $\mathbb{Z}_q$). On the
other hand, the Pedersen commitment is perfectly hiding and thus does
not reveal information about the committed value. The following
protocol attempts to merge the both properties i.e., to prove the
knowledge of a commited value under the Pedersen commitment scheme in
a zero-knowledge manner.
\\
\indent \emph{The protocol.} The public inputs of the proof are the prime $p$,
the Pedersen public key $(g, h)$, a security parameter $k$ and a
(hypothetic) commitment $c\in QR(p)$. The prover's private intputs are
$x$ and $r$ in $\mathbb{Z}_q$ s.t. $c=g^xh^r$ (mod $p$). The protocol executes as follows. 
\begin{itemize}
	\item The prover randomly chooses $y,s \in_R \mathbb{Z}_q$ and sends $d=g^yh^s  \pmod{p}$ to the verifier.
	\item The verifier randomly chooses $e\in_R \{0,1\}^k$ and sends it to the prover.
	\item The prover computes $z=y-ex$ and  $t=s-er$ modulo $q$ and sends it to the verifier.
	\item The verifier accepts the proof iff $d = c^e g^zh^t
          \pmod{p}$.
\end{itemize}
If the verifier accepts the proof, we say that the conversation $\langle d,e,(z,t) \rangle$ is valid.
%
\begin{enumerate}
	\item Prove the correctness property of this construction.
	\item Assume you are able to ``rewind'' an adversarial prover
          who tries to build a valid conversation. How can you use
          this faculty to extract an opening of $c$. Which property
          did you break ? Briefly discuss the soundness property of
          the protocol.

          	\item Show how a valid conversation $\langle
          d,e,(z,t) \rangle$ can be simulated from $c$, without the use of any
          private inputs. (Assume that the valid conversation involves
          honest parties.)
	\item Generalize the process to prove the knowledge of an opening to 
	        a multi-Pedersen commitment as in exercise 2.
\end{enumerate}

\begin{solution}
  \begin{enumerate}
      \item The conversation is correct if Pr[$d \neq c^eg^zh^t$] $\leq \epsilon(n)$. Let's evaluate this probability : 
      $$Pr[d \neq c^eg^zh^t] = Pr[g^yh^s \neq g^{xe+z}h^{re+t}]  = Pr[g^yh^s \neq g^{xe+z}h^{re+t}] = 0$$
      Then our construction is correct.
      \item When we get the conversation (d, e, (z,t)), we can "rewind" the conversation to submit another e' and get new z' and t'. Therefore we can obtain the private key (x,r) by doing those calculations :
      $$\begin{cases} z = y - ex \\ z' = y - e'x \end{cases} \Rightarrow x = \frac{z-z'}{e'-e}
      $$
      $$\begin{cases} t = s - er \\ t' = s - e'r \end{cases} \Rightarrow r = \frac{t-t'}{e'-e}
      $$
      We broke the zero-knowledge property since the verifier can extract the private key using such power. \newline
      According to the assistants, since it is not zero-knowledge, there is no point of discussing the soundness property.
      \item It is easy to show, with honest parties, how we can simulate from $c$ a new valid conversation : 
      \begin{enumerate}
          \item We pick e 
          \item We pick z and t
          \item We evaluate d as : $d = c^ez^zh^t$
      \end{enumerate}
      \item To generalize the process, we have : 
      \begin{itemize}
          \item pk = $g^x_1$, ..., $g^x_n$, h
          \item sk = $x_1$, ..., $x_n$, r
          \item c = $g^x_1 \cdot ... \cdot g^x_n \cdot h^r$
      \end{itemize}
  \end{enumerate}
\end{solution}

\subsection{Exercise 4}
\label{subsec:4b.4}
Let $f$ be a one-way permutation on $\{0,\,1\}^\lambda$. 
Consider the following signature scheme for messages in the set 
$\{1,\ldots,\,n\}$, where $n\in\mathsf{poly}(\lambda)$: \vspace{3mm}

\begin{enumerate}
	\item[\textbullet] To generate keys, choose $x\leftarrow \{0,\,1\}^\lambda$ at random
	      and set $y:=f^n(x)$. The public key is $y$ and the private key is $x$.
	\item[\textbullet] To sign message $i\in\{1,\ldots,\,n\}$, output $f^{n-i}(x)$
	      (where $f^0(x)\stackrel{\mbox{\small def}}{=}x$).
	\item[\textbullet] To verify signature $\sigma$ on message $i$ with respect to
	      public key $y$, check whether $y\stackrel{?}{=}f^i(\sigma)$. 
\end{enumerate}

\begin{enumerate}
	\item Show that the above is not a one-time signature scheme. Given
	      a signature on a message $i$, for what messages $j$ can an 
	      adversary output a forgery?
	\item Prove that no \texttt{PPT} adversary given a signature on $i$
	      can output a forgery on any message $j>i$ except with negligible
	      probability.
	\item Suggest how to modify the scheme so as to obtain a one-time 
	      signature scheme. \\
	      \emph{Hint: include two values $y,\,y'$ in the public key.}
\end{enumerate}
\begin{solution}
  \begin{enumerate}
    \item
      A has $(i, \sigma(i))$ with $\sigma (i) = f^{n-i} (x)$. We know (because $f$ is a permutation function) that:
      $$f(\sigma(i)) = f^{n-i+1}(x) = f^{n-(i-1)}(x) = \sigma(i-1)$$
      Then it's possible to compute a valid forgery for every $j < i$. The scheme is then not a one time-signature.
    \item
      Need schema drawn at TP !! It's a lot simpler with it...

      $\Pr[Success (\A_\sigma)] = \epsilon(\lambda)$, $\Pr[Abort] = \frac{n-k}{n}$, $\Pr[Success] = \frac{n-k}{n-m-1}$ then:
      $$ \Pr[Success(\A_{owf})] = \epsilon(\lambda) \frac{n-k}{n} \frac{n-k}{n-m-1}$$

      If $\epsilon(\lambda)$ is not negligible, then the probability of success is not negligible.
    \item
      We have $s_k = (x, x')$, $p_k = (f^n(x), f^n(x'))$.
      Then $m \rightarrow \sigma = (f^{n-m}(x), f^m(x'))$.

  \end{enumerate}
\end{solution}


\subsection{Exercise 5 (Jan 2011 evaluation)}
\label{subsec:4b.5}
Consider the following one-time signature scheme $\Pi:=\langle \Gen,
\mathsf{Sign}, \Vrfy\rangle$, parameterized by a PPT function $f: \{0,1\}^*
\rightarrow \{0,1\}^*$.
\begin{itemize}
\item $\Gen$: on input $1^n$, select $(x_0,x_1) \leftarrow
  \{0,1\}^n\times \{0,1\}^n$ uniformly at random, compute $(y_0,y_1) :=
  (f(x_0),f(x_1))$ and output the pair $(pk,sk):=
  ((y_0,y_1),(x_0,x_1))$.
\item $\mathsf{Sign}$: the signature $\sigma$ of the bit $m$ is $x_m$.
\item $\Vrfy$: on input $(m,\sigma)$, output 1 iff $y_m = f(\sigma)$.
\end{itemize}

Show that if $\Pi$ is existentially unforgeable under a single-message
attack, then $f$ is a one-way function.
\begin{solution}
  Two solution have been proposed. (Actually they are the same but with a different explanation)
  \begin{itemize}
    \item
      Let's show that if $f$ is not one way, then $\Pi$ is not existentially unforgeable.
      Let $\A$ be the inverter of $f$, we will build $\A'$ that builds an existential forgery with non-negligible probability.

      \begin{itemize}
        \item $\A'$ receives $pk = (y_0, y_1)$
        \item $\A'$ ask the signature of 0 and gets $\sigma$, he does not really care about it
        \item $\A'$ gives $y_0$ (or $y_1$) to $\A$ which outputs $x_0$ (or $x_1$) with non-negligible probability.
        \item $\A'$ outputs $(0,x_0)$ (or $(1, x_1)$)
      \end{itemize}
      Since $y_1$ is the image of a random $x_1$, we are exactly in the inverting experiment so $f(x_1') = y_1$
      with probability $\Pr[\Invert_{\A,f}(n) = 1]$.

      We know that
      \[
        \Pr[\Sigforgeone_{\A',\Pi}(n) = 1] = \Pr[\Invert_{\A,f}(n) = 1]
      \]
    \item
      Let assume that $f$ is not one way function $y = f(x)$.
      Then, we can recover $x$ with a non negligible probability $\epsilon_x (n)$.

      So, $(y_0, y_1) \Rightarrow (x_0, x_1)$ with probabilities $(\epsilon_{x_0} (n), \epsilon_{x_1} (n))$.
      We cannot compute a pre-image by asking the oracle. So I output $(0, x_0)$ and $(1, x_1)$ as a forgery.

      The probability $\Pr[\Sigforge_{\A,\Pi}(n)=1]= \Pr[f^{-1}(\cdot)_{Inv,f(\cdot)}(n)=1] \leq \epsilon(n)$. As $\Pi$ is supposed to be existentially unforgeable under a single-message attack, then $\epsilon(n)$ is negligible and this implies that f is a one-way function.
  \end{itemize}
\end{solution}
\newpage
\section{}
\subsection{Exercise 0}
Same as exercise \ref{subsec:4b.4}
\copypaste{5}{4}

\subsection{Exercise 1}
Same as exercise \ref{subsec:4b.5}
\copypaste{5}{5}

\subsection{Exercise 2 (Jan 2011 evaluation)}
The Digital Signature Standard (DSS, also often called DSA) is one of
the most commonly used signature algorithms. Its three algorithms
$\Gen$, $\mathsf{Sign}$ and $\Vrfy$ work as follows.
\begin{itemize}
	\item $\Gen$: on input $1^n$, select prime integers $p$ and $q$ such
	that $|q|=n$, $q | (p-1)$ and $q^2 \not | \, (p-1)$, together with an
	integer $g$ that generates the subgroup of $\mathbb{Z}_p^*$ of prime
	order $q$. Also choose a hash function $H : \{0,1\}^* \rightarrow
	\mathbb{Z}_q$. Then, select $x \leftarrow \mathbb{Z}_q$ uniformly at
	random, and compute $y:= g^x \mod p$. The public key is $\langle H,
	p, q, g, y\rangle$, and the private key is $\langle x\rangle$.
	\item $\mathsf{Sign}$: in order to sign the message $m \in \{0,1\}^*$, choose
	$k \leftarrow \mathbb{Z}_q^*$ uniformly at random and set $r:= [g^k
	\mod p] \mod q$. Then, compute $s:= (H(m) + xr) \cdot k^{-1} \mod
	q$, and output the signature $(r,s)$.
	\item $\Vrfy$: compute $u_1 := H(m)\cdot s^{-1} \mod q$ and $u_2 := r
	\cdot s^{-1} \mod q$, and output 1 if and only if $r = [g^{u_1} y^{u_2} \mod p] \mod q$. 
\end{itemize}

\begin{enumerate}
	\item Show the correctness of the DSS algorithm.
	\item 
	As randomness is an expensive resource, it is proposed to select
	the random value $k$ once and for all, and to sign all messages
	using that value of $k$. Is this variant of DSS secure? \\
	\emph{(Hint: see what you can deduce from the signature of two
		different messages.)}
\end{enumerate}
\begin{solution}
  \begin{enumerate}
    \item
      $(r, s) = ([g^k \pmod{p}] \pmod{q}, [H(m) + xr]k^{-1} \pmod{q})$, then:
      $u_1 = H(m)s^{-1} \pmod{q}$,  $u_2 = rs^{-1} \pmod{q}$, $r = [g^{u_1} g^{u_2} \pmod{p}] \pmod{q}$, $y = g^x$

      $$\Rightarrow g^{u_1 + xu_2} = g^{s^{-1}(H(m) + rx)} = [g^k \pmod{p}] \pmod{q} = r$$
    \item
      $s = (H(m) + xr)k^{-1} \pmod{q}$, $s' = (H(m') + xr)k^{-1} \pmod{q}$ ($s \neq s'$ otherwise we have a collision).
      $s - s' = (H(m) - H(m'))k^{-1} \pmod{q}$, $k = \frac{H(m) - H(m')}{s - s'}$.
      $s = (H(m) + xr)k^{-1}$ so $\frac{sk - H(m)}{r} = x$ where x is the secret.
  \end{enumerate}
\end{solution}

\subsection{Exercise 3 (RSA permutation with modulus 221)}
Suppose we decide to use an RSA permutation with modulus $221$, we consider RSA encryption scheme, and RSA signature.
\begin{enumerate}
	\item What is the smallest non trivial public exponent $e$ than can be
	chosen?
	\item Can we choose $e=11$? What is the corresponding private exponent $d$? Give the public and private key of the corresponding RSA encryption scheme.
	\item Compute $c := 219^e \pmod{221}$.
	\item Verify that $c^d = 219 \pmod{221}$ as expected.\\
	
	\item How Alice (owning the private key) could sign a message $m$? Sign the message $m=3$ (hint: $22^7= 61 \pmod{221}$).
	\item Is $160$ a valid signature for $m=218$?
	% \emph{Hint: use the square and multiply algorithm.}
\end{enumerate}
\begin{solution}
We suggest you to use a calculator to do this exercise, the assistants said we could (pour faire taire les rageux).
  \begin{enumerate}
      \item \textbf{What is the smallest non trivial public exponent $e$ that can be chosen ?} \newline \newline
      First we have to find $\phi$(221). As 221 is not a prime, we must find $p$ and $q$ such that $221 = pq$. After a few try, we find $p = 13$ and $q = 17$. Wet get then $\phi(221) = (p-1)(q-1) = 192$. \newline 
      The searched smallest $e$ should respect the condition $gcd(192, e) = 1$. Again, after a few try, we find $e = 5$.
      \item \textbf{Could we choose $e = 11$ ?} Yeah, since 11 is a prime, then $gcd(192, 11) = 1$. \newline \newline
      \textbf{What is the corresponding private exponent d ?} $$ ed = 1 \text{ mod 192} \\ 11d = 1 \text{ mod 192} \\ d = \frac{192k + 1}{11}$$
      If we take k=2, we find $d = 35$.\newline \newline
      \textbf{Give the public and private key of the corresponding RSA encryption scheme} \newline
      pk = $(221, 11)$ 
      sk = $(221, 35)$
      \item \textbf{Compute c := $219^e$}(boring calculations) 
      \begin{itemize}
          \item $219^{1} = -2 $
          \item $219^{2} = 4  $
          \item $219^{4} = 16 $
          \item $219^{5} = -32$
          \item $219^{10} = 140$
          \item $219^{11} = 162$
      \end{itemize}
      \item \textbf{Verify that $c^d = 219$} 
      \begin{itemize}
          \item $162^{1} = -59 $
          \item $162^{2} = -55 $
          \item $162^{4} = -69 $
          \item $162^{5} = 93 $
          \item $162^{6} = 38 $
          \item $162^{7} = -32 $
          \item $162^{35} = -2 = 219 $
      \end{itemize}
      \item \textbf{How Alice could sign a messsage m ?} According the scheme described in the slides, $Sign_{(N, d)}(m) := [m^d mod N]$ \newline \newline
      \textbf{Sign the message m = 3} 
      \begin{itemize}
          \item $3^{1} = 3 $
          \item $3^{2} = 9 $
          \item $3^{4} = 81 $
          \item $3^{5} = 22 $
          \item $3^{35} = 61 $ (using the hint given)
      \end{itemize}
      \item \textbf{Is 160 a valid signature for m = 218}
      Let's calculate (yipie) 
      \begin{itemize}
          \item $218^{1} = -3 $
          \item $218^{2} = 9 $
          \item $218^{4} = 81 $
          \item $218^{5} = -22 $
          \item $218^{35} = -61 = 160$
      \end{itemize}
      Yes it is a valid signature
  \end{enumerate}
\end{solution}
\newpage

\section{}


\subsection{Exercise 1 (Commitment scheme)}
\label{subsec:commit-scheme}

Define the bit-commitment scheme $\langle \G, \Com, \Open \rangle$ with the following PPT algorithms:
\begin{itemize}
	\item $\Gen(1^n)$ sets $pk$ as $(\PRG,R)$, where
	\begin{itemize}
		\item $\mathsf{G}$ is a random generator $\lbrace 0,1 \rbrace^n \longmapsto \lbrace 0,1\rbrace^{3n}$
		\item $R$ is a random $3n$-bit string
	\end{itemize}
	\item $\Com_{pk}(b)$ with $b\in\{0,1\}$ provides $(c,d)$ where:
	\begin{itemize}
		\item $Y$ is an $n$-bit string
		\item  if $b=0$ $c=\mathsf{G}(Y)$
		\item if $b=1$, $c=\mathsf{G}(Y) \oplus R$
		\item $d=(b,Y)$
	\end{itemize}
	\item $\Open_{pk}(c,d)$ outputs $b$ if it can recompute $c$ from $d$ and $pk$, or $\bot$ otherwise
\end{itemize}

\begin{enumerate}
	\item Is this scheme perfectly hiding?
	\item Is this scheme computationaly binding?
	\item If the committer choose $R$ is the scheme secure?
\end{enumerate}


\begin{solution}
	\begin{enumerate}
		\item For a scheme to be perfectly hiding, we need that $\forall \A$:
		\[ \Pr[\ComHide_{\A,\Pi}=1]=\frac12 \Leftrightarrow \Pr[c|b=0] = \Pr[c|b=1] \]
		If $b=0$, then $c$ has as much randomness as $\mathsf{G}$, which has as much randomness as $Y$, so $n$ bits of randomnes (=there are $2^n$ possible values for $c$).

		If $b=1$, then $c$ has as much randomness as $\mathsf{G}$ and $R$, so basically $3n$ bits of randomness (=there are $2^{3n}$ possible values for $c$).

		If we have unbounded computational power, then we could enumerate all possible outputs $\mathsf{G}(Y)$, and see if $c$ is in this set of values.
		If $b=0$, we are sure they are in;
		if $c=1$, there are $2^{2n}$ possible $R$ such that $\mathsf{G}(y) \oplus R$ cannot be distinguished from $\mathsf{G}(Y')$ (simply, $R=\mathsf{G}(Y)\oplus \mathsf{G}(Y')$), so there is a probability $\frac{2^{2n}}{2^{3n}}=\frac{1}{2^{n}}$ that $c \in \{\mathsf{G}(Y)\}$.
		So the probability of success for an unbounded adversary is
		\[ \frac{1}{2} \cdot 1 + \frac{1}{2} (1-\frac{1}{2^{n}}) = 1-\frac{1}{2^{n+1}} \]
		So, an unbounded adversary has near-certainty of breaking the hiding property.

		To break the hiding property, an adversary would need to enumerate all possible outputs of $\mathsf{G}$ (requires $2^n$ steps) if $G(Y) \oplus R \notin \{x | \forall u : x = G(u)\}$.

		But this kind of event has a negligible chance of probability ($\Pr = \frac{|\mathsf{G}(Y)|}{|\mathsf{G}(Y) \oplus R|} = \frac{2^n}{2^{3n}} = \frac{1}{2^{2n}} = \negl(n)$) so an adversary with an unbounded power of calculation can easily break the property of perfectly hiding.

		It is however simple to prove that the scheme is computationally hiding.

		\item For the scheme to be computationally binding, it should be intractable to find $(c, d_0, d_1)$ such that $\Open_{pk}(c, d_0)=0$ and $\Open_{pk}(c, d_1)=1$.
		If we replace, we find that it is equivalent to find $Y_0, Y_1$ such that
		\[ c=\mathsf{G}(Y_0) = \mathsf{G}(Y_1) \oplus R \]
		For this to be possible at all, we need to have $R=\mathsf{G}(Y_0) \oplus \mathsf{G}(Y_1)$ for some $Y_0, Y_1$.
		But, we have that $|\{R\}|=2^{3n}$ while $|\{ \mathsf{G}(Y_0) \oplus \mathsf{G}(Y_1) \}| \le 2^{2n}$, so the probability of $R$ being correct for this to happen is at most $\frac{2^{2n}}{2^{3n}}=\frac{1}{2^n}$, which is negligible.
		And so, \[ \Pr[\ComBind_{\A, \Pi}(n)=1] \le \frac{1}{2^n} \quad \forall \A. \]
		So even if we have an adversary capable of finding $Y_0$ and $Y_1$ with near-certainty, the fact that $R$ can just be badly chosen for him causes its probability of success to be \emph{in all cases} negligible.
		So, the scheme is computationally binding.

		\item Not secure, because if the committer chooses $R = G(Y)$, then the opposite player can easily deduce the value of b. If $c = 0$, $b = 1$, else $b = 0$.
	\end{enumerate}
\end{solution}



\subsection{Exercise 2 (Commitment with DL)}

Let $(\mathbb{G}, \cdot)$ be a group in which the discrete logarithm is difficult, with $|\mathbb{G}|=q$.
Let $g$ be a generator of the group and $h$ be a random element of the group ($(g, h)$ may be seen as the key of the hash function).
Define the following hash funtion $\mathsf{H}\colon \Z^*_q \times \Z^*_q \mapsto \mathsf{G}$:
\[ \mathsf{H}_{g,h} (\alpha, \beta) \define g^\alpha h^\beta \]
Prove that if the DL is difficult, then, the hash function is collision resistant.
For simplicity we assume that $q$ is prime.


\begin{solution}
	I don't know, maybe a reduction might be useful? It's been such a long time!

	For the reduction, let's assume that we have an adversary $\A$ that can break the collision resistance by finding a collision with advantage $\negl(2n)$.
	We have $2n$ instead of $n$ because the seed of the hash function has $2n$ bits instead of $n$.
	Then, we can build an adversary $\D$ that can solve the DL problem:
	\begin{enumerate}
		\item Run $\mathcal{G}(1^n)$ to obtain $(\mathbb{G}, q, g)$ where $g$ generates $\mathbb{G}$ of order $q$ with $|q|=n$.
		\item Choose $h \pick \mathbb{G}$.
		\item Send $(\mathbb{G}, q, g, h)$ to $\D$.
		\item $\D$ uses $\A$: he sends $(g, h)$ as the seed.
		Then, with probability $\negl(2n)$, $\A$ answers with $(\alpha, \beta)$ and $(\alpha', \beta')$ such that $\alpha\neq\alpha' \vee \beta\neq\beta'$ and
		\[ g^\alpha h^\beta = g^{\alpha'} g^{\beta'} \]
		From this, if we want to find $x$ such that $g^x=h$, then we just replace:
		\begin{align*}
		g^\alpha (g^x)^\beta &= g^{\alpha'} (g^x)^{\beta'} \\
		\alpha + x \beta &= \alpha' + x \beta' \\
		x &= \frac{\alpha'-\alpha}{\beta-\beta'}
		\end{align*}
	\end{enumerate}
	We have
	\[ \Pr[\DLog_{\D, \mathcal{G}}(n)=1] = \Pr[g^x=h] = \Pr[\HashColl_{\A, \mathsf{H}}(n)=1] = \negl(2n) \le \negl'(n)\]
	And, as we know that the DL problem is hard is $\mathbb{G}$, then we know that these probabilities should be negligible, and so that finding a collision is also hard.
\end{solution}



% OK
\subsection{Exercise 3 (Commitment scheme and batching)}

\copypaste{9}{0}



\subsection{Exercise 4 (Decisional Diffie-Hellman and \texorpdfstring{$\mathbb{Z}_p^\ast$}{Zp*})}

The goals of this exercise are to define $QR_p$, prove some of its properties, and to show that in some groups DDH and CDH assumptions are conjectured not equivalent, as DDH is easy whereas CDH is conjectured to be hard.

\begin{enumerate}
	\item For all element $a$ of $\mathbb{Z}_{11}^*$, compute $a^2 \mod 11$.

	For a prime number $p$, we denote $QR_p$ the set $\{x \in \mathbb{Z}_{p}^* \; | \; \exists a\in \mathbb{Z}_{p}^*, a^2=x\}$, such $x$ are called quadratic residues modulo $p$. Show that if $p$ is odd then $|QR_p|=\frac{p-1}{2}$.

	\item Show that, if $p$ is odd, $QR_p$ is a cyclic group (therefore, $QR_p$ is a subgroup of $\Z^*_p$).

	\item For all element $a$ of $\mathbb{Z}_{11}^*$, compute $a^5 \mod 11$. Show that for any odd prime $p$, $x \in QR_p \Leftrightarrow x^{\frac{p-1}{2}}= 1 \mod p$, and that $x \not \in QR_p \Leftrightarrow x^{\frac{p-1}{2}}= -1 \mod p$.

	\item Show that $2$ is a generator of $\mathbb{Z}_{11}^*$. For the following pairs $(a,b)$, compute $g^a, g^b$ and $g^{ab}$ in $\mathbb{Z}_{11}^*$ where $g=2$:
	\begin{itemize}
		\item $(2,8)$,
		\item $(1,4)$,
		\item $(3,5)$.
	\end{itemize}
	Show that for $p$ an odd prime, $g^{ab} \not \in QR_p \Leftrightarrow g^a \not \in QR_p \text{ and } g^b \not \in QR_p$.

	\item Show that DDH does not hold in $\mathbb{Z}_{p}^*$ with $p$ an odd prime.
\end{enumerate}


% TODO rewrite this to use the official proof
\begin{solution}
	An official solution was given in the exercise session.

	For this exercise we will work with $\Z_{11}^* = \{1,2,3,4,5,6,7,8,9,10\}$
	\begin{enumerate}
		\item
		For all element $a$ of $\Z_{11}^*$, I've calculated $a^2$ mod $11$.
		\[1^2 = 1 \quad 2^2 = 4 \quad 3^2 = 9 \quad 4^2 = 5 \quad 5^2 = 3 \quad 6^2 = 3 \quad 7^2 = 5 \quad 8^2 = 9 \quad 9^2 = 4 \quad 10^2 = 1\]
		We see that with $p$ odd, we have $\left|QR_p\right| = \frac{p-1}{2}$. We can show it with this development:

		\item
		\nosubsolution

		\item
		For all element $a$ of $\Z_{11}^*$, I've calculated $a^5$ mod $11$.
		\[1^5 = 3^5 = 4^5 = 5^5 = 9^5 = 1 \qquad 2^5 = 6^5 = 7^5 = 8^5 = 10^5 = 10\]
		We can see that for $p$ prime, we have $x \in QR_p \Leftrightarrow x^{\frac{p-1}{2}} = 1 \mod p$ and  $x \notin QR_p \Leftrightarrow x^{\frac{p-1}{2}} = p-1 \mod p$.
		\begin{itemize}
			\item $x \in QR_p \Leftrightarrow x^{\frac{p-1}{2}} = 1 \mod p$:\\
			We know that \[x \in QR_p \Leftrightarrow \exists a \st x = a^2 \mod p\]
			So we have now \[x = a^2 \mod p \Leftrightarrow x^{\frac{p-1}{2}} = 1 \mod p\]
			If we replace $x$ by $a$ we obtain $a^{2^{(\frac{p-1}{2})}} = 1 \mod p$.\\
			But also more simply $a^{p-1} = 1 \mod p$ which is true by the group theory.
			\item $x \notin QR_p \Leftrightarrow x^{\frac{p-1}{2}} = p-1 \mod p$:\\
			We know that \[x \notin QR_p \Leftrightarrow \exists a \st x = a^{1+2n} \mod p\]
			So we have now \[x = a^{1+2n} \mod p \Leftrightarrow x^{\frac{p-1}{2}} = -1 \mod p\]
			We replace $x$ by $a$ and we get \[a^{\frac{p-1}{2}} a^{n(p-1)} \mod p = -1 \mod p\]
			We know that $ a^{n(p-1)} \mod p = 1$, thus we simplify the equation like \[a^{\frac{p-1}{2}} \mod p = -1 \mod p\]
			We know that $g = a^{\frac{p-1}{2}} \mod p \ne 1 \mod p$ but $g^2 = a^{p-1} = 1 \mod p$. The only solution of these two equations is $g = -1 \mod p$ which is equivalent to
			\[x^{\frac{p-1}{2}} = p-1 \mod p\]
		\end{itemize}

		\item
		The number $2$ is a generator of $\Z_{11}^*$, because ord($2$) $= 10$. In fact, we have $2^1 = 2$, $2^2 = 4$, $2^5 = 10$ and $2^10 = 1$. (Fermat's little theorem)
		We have $g = 2$ so:
		\begin{itemize}
			\item $(2,8):\quad g^2 = 4$, $g^8 = 3$ and $g^{16} = -2$
			\item $(1,4):\quad g^1 = 2$, $g^4 = 5$ and $g^{4} = 5$
			\item $(3,5):\quad g^3 = -3$, $g^5 = -1$ and $g^{15} = -1$ TODO
		\end{itemize}
		We have to show that $g^{ab} \notin QR_p \Leftrightarrow g^a \notin QR_p \text{ and } g^b \notin QR_p$.\\
		We know by the definition of the $QR_p$ set that
		\[g^n \notin QR_p \Leftrightarrow \exists m \st n = 2m+1\]
		We can thus extract from $g^{ab} \notin QR_p$ that $\exists m \st ab = 2m+1$.

		% not necessary
		%We can do a proof by contradiction:\newline
		%If $a = 2v$ then $ab = 2bv$ and can not be equal to $2m + 1$.\\
		%If $b = 2v$ then $ab = 2av$ and can not be equal to $2m + 1$.\\

		So we are assured that $a$ and $b$ are not pairs, so we have the relation $g^a \notin QR_p \text{ and } g^b \notin QR_p$ if and only if $ab = 2m+1$ which is equivalent to $g^{ab} \notin QR_p$. That was what we had to proof.

		\item
		We have to show that DDH does not hold in $\Z_p^*$ with p an odd prime number.

		We define an attacker that can see $p$, $g$, $g^a$, $g^b$ and receive $h_b = g^{ab}$ or $g^z$.
		The behaviour of the attacker will be this one:
		\begin{itemize}
			\item It receives $g^a \notin QR_p$ and $g^b \notin QR_p$:\\
			It will answer in function of $h_b$:
			\begin{itemize}
				\item $h_b \notin QR_p$:\\
				It answers $h_b = g^{ab}$
				\item $h_b \in QR_p$:\\
				It answers $h_b = g^z$
			\end{itemize}
			\item It receives $g^a \in QR_p$ or $g^b \in QR_p$:\\
			It answers randomly.
		\end{itemize}
		We can identify four cases with their chances of success and appearance (we already know that $\left|QR_p\right|$ is of size $\frac{p-1}{2}$):
		\begin{enumerate}[a)]
			\item $g^a \in QR_p$ or $g^b \in QR_p$ appears $3/4$ of the time with success = $1/2$.
			\item $g^a \notin QR_p$ and $g^b \notin QR_p$ with $h_b = g^{ab}$  appears $1/8$ of the time with success = $1$.
			\item $g^a \notin QR_p$ and $g^b \notin QR_p$ with $h_b = g^z$ and $g^z \in QR_p$  appears $1/16$ of the time with success = $1$.
			\item $g^a \notin QR_p$ and $g^b \notin QR_p$ with $h_b = g^z$ and $g^z \notin QR_p$  appears $1/16$ of the time with success = $0$.
		\end{enumerate}
		We can now recalculate the expected value of success of our attacker:
		\begin{align*}
		\mathbb{E}(success) &= \frac{3}{4}\cdot \frac{1}{2} +  \frac{1}{8}\cdot 1 + \frac{1}{16}\cdot 1 + \frac{1}{16}\cdot 0\\
		&= \frac{1}{2} + \frac{1}{16}
		\end{align*}
		This attacker has one sixteenth of probability more than one half which is not a negligible function (in fact it is a constant function). It is not DDH secure.
	\end{enumerate}
\end{solution}


\subsection{Exercise 5}

\copypaste{8}{1}


\newpage
\bibliographystyle{plain}
\bibliography{biblio}

\end{document}
