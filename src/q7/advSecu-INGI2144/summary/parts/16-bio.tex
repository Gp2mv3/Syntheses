
\subsection{Machine Readable Zone}

\subsubsection{Construction}
\begin{itemize}
    \item An Machine Readable Zone is composed of two lines of 44 characters
    \item Numbers and punctuations not authorized in the name field
    \item Hyphens are replaced by a filler character ('<')
    \item Apostrophes and commas are omitted
    \item First and last names are separated by 2 filler characters
    \item White characters are replaced by filler characters
\end{itemize}

\subsubsection{Check Digits Calculation}
Check digits are computed for each protecfields. They are calculated modulo 10
with continous repetitive weihting of 731
\begin{itemize}
    \item Letters are mapped to their corresponding numerical value:
    A=10, B=11, \ldots, Z=35, '<'=0.
    \item From left to right, each numerical value is multiplied by the weight
    appearing in the same sequential position.
    \item The product of each multiplication is added modulo 10
\end{itemize}

\paragraph{Example} Calculate the check digit of the document number "EH123456<"

\begin{tabular}{m{10cm}m{6cm}}
    \begin{tabular}{|c|c|c|c|c|c|c|c|c|c|}
        \hline
        Number & E & H & 1 & 2 & 3 & 4 & 5 & 6 & < \\
        & 14 & 17 & 1 & 2 & 3 & 4 & 5 & 6 & 0 \\
        \hline
        Weight & 7 & 3 & 1 & 7 & 3 & 1 & 7 & 3 & 1\\
        \hline
        $N\times W \mod{10}$ & 98 & 51 & 1 & 14 & 9 & 4 & 35 & 18 & 0\\
        \hline
    \end{tabular}
    &
    \begin{eqnarray*}
        98+51+1+14+9+4+35+18 &=& 230 \\
        230 \mod 10 &=& 0 
        \end{eqnarray*}
        $\Rightarrow$ The remainder of the division is the check digit $0$.
\end{tabular}

\subsection{DOC 9303}

\begin{center}
    \includegraphics[width=11cm]{img/9303.png}
    \end{center}

\subsubsection{Technical facts about the passport}
\begin{itemize}
    \item Tag is passive (no internal battery)
    \item Tag has a microprocessor (public-key crypto)
    \item Official distance is 10cm
    \item EEPROM capacity: 32KB (minimum)
\end{itemize}

\subsubsection{Passport Content}
\begin{tabular}{|m{2cm}|m{7cm}|m{6cm}|}
    \hline
    Publicly & Publicly possibly after authentication & Never supplied
    by the tag \\
    \hline
    \begin{itemize}
        \item UID 
    \end{itemize}
    &
    \begin{itemize}
        \item Some data groups (DG)
        \item List of data groups on the considered passport (COM)
        \item Cryptographic material signature and hashes (SOD)
    \end{itemize}
    &
    \begin{itemize}
        \item Two symmetric key $K_{ENC},K_{MAC}$ (can be retrieved from MRZ)
        \item One private key $K_{pr}$ (protected memory)
    \end{itemize}
    \\
    \hline
\end{tabular}


\subsection{Protection Mechanisms}
\begin{center}
    \begin{tikzpicture}[node distance= 0.3cm]
        \node [draw=blue, rectangle, text width=5cm] (A) {Modifying data
            of a given passport. Forging a fake passport};

        \node [draw=blue, rectangle, text width=5cm, below=of A] (B) {Cloning a given passport};
        \node [draw=red, rectangle, text width=5cm, below=of B] (C) {Skimming a passport};
        \node [draw=red, rectangle, text width=5cm, below=of C] (D) {Eavesdropping the communication};

        \node [right=2cm of A] (AA) {Passive Authentication
        (Signature)};
        \node [right=2cm of B] (BB) {Active Authentication
(Challenge Response)};
        \node [right=2cm of C] (CC) {Basic Access Control
(Reader Authentication)};
        \node [right=2cm of D] (DD) {Secure Messaging
(Encryption)};

        \draw[->, >=latex, blue] (AA) to node[] {} (A);
        \draw[->, >=latex, blue] (BB) to node[] {} (B);
        \draw[->, >=latex, red] (CC) to node[] {} (C);
        \draw[->, >=latex, red] (DD) to node[] {} (D);


    \end{tikzpicture}
\end{center}

\subsubsection{Passive Authentication (Signature)}
Passive authentication is a \textbf{mandatory security mechanism}: it proves
that the file EF.SOD and LDS are authentic and not modified.

\begin{tabular}{m{8cm}m{7cm}}
    \centering
\begin{tikzpicture}[node distance=0.2cm]
    \node[text width=1.5cm,text centered, rectangle, draw] (1) {$EF.COM$};
    \node[text width=1.5cm,text centered, rectangle, draw, below=of 1] (2) {$DG_1$};
    \node[text width=1.5cm,text centered, rectangle, draw, below=of 2] (3) {$DG_2$};
    \node[text width=1.5cm,text centered,  below=of 3] (4) {$\vdots$};
    \node[text width=1.5cm,text centered, rectangle, draw, below=of 4] (5) {$DG_n$};

    \node[rectangle, draw, dotted, right=0.5cm of 2] (H1) {$hash$};
    \node[rectangle, draw, dotted, right=0.5cm of 3] (H2) {$hash$};
    \node[below=0.5cm of H2] (H3) {$\vdots$};
    \node[rectangle, draw, dotted, right=0.5cm of 5] (H4) {$hash$};

    \node[rectangle, draw,text width=2.5cm, text centered, dotted, right=1cm of H3] (S) {Signature};
    \node[rectangle, draw,text width=2.5cm, text centered, dotted, below =of S] (D) {DS certificate};

    \node[above right=-0.2cm and 1.3cm of H1] (EF) {$EF.SOD$};

    \node [draw, double, rectangle, fit={(H1) (H2) (H3) (H4) (D) (EF) }] (FF) {};

    \node [draw, red, dashed, rectangle, fit={(2) (H1)}] (FF) {};
    \node [draw, red, dashed, rectangle, fit={(3) (H2)}] (FF) {};
    \node [draw, red, dashed, rectangle, fit={(5) (H4)}] (FF) {};
\end{tikzpicture}
& 
\begin{itemize}
    \item \textbf{EF.SOD} contains the hash value of each present
        \textbf{DG}, and signature calculated by the issuing State over values.
    \item The signature can be checked using the Document Signer
        (\textbf{DS}) X.509 certificate. (Available from \textbf{EF.SOD})
    \end{itemize}
\end{tabular}

\begin{itemize}
    \item The DS certificate can be checked using the Country Signing CA (CSCA)
    X.509 cerficate.
    \begin{itemize}
        \item The ICAO PKD does not publish the CSCA certificates
        \item CSCA certificates and revocation lists should be exchanged
        according to bilateral agreements
    \end{itemize}
\end{itemize}

\paragraph{Recommandation} According to DOC 9303:
\begin{itemize}
    \item The passive authentication should use \textsc{RSA},
        \textsc{DSA} or \textsc{ECDSA} for the
    signature schemes
    \item SHA-1, SHA-224/256/384/512 for the hash algorithm
    \item CSCA keys should be renewed every 3-5~years and the DS
    keys every 3~months
\end{itemize}

\subsubsection{Active Authentication (CR)}

The active authentication is an optional security mechanism:
it prove that the \textbf{EF.SOD} belongs to the authentic ePassport, i.e
it is not a cloned one.

\begin{center}
\begin{tabular}{rcl}
    \bf Reader & & \bf ePassport\\
               & \fr{$M_2$} & \\
               & \fl{$Sign(M_2, M_1)$} & \\
    \end{tabular}
\end{center}

\begin{itemize}
    \item Two-pass CR protocol ISO 9796i\text{-}2 Digital Signature Scheme 1
    \item ePassport's public key is stored in DG15
\end{itemize}

The active authentication should rely on RSA, DSA or ECDSA with minimum sizes for
the security parameters should be respectively 1024 bits, 1024 and 160
bits, and 160 bits.

\paragraph{Example with RSA/SHA~1}

\begin{center}
\begin{tabular}{rcl}
    \bf Reader & & \bf ePassport\\
    $M_2 \in_R \{0, 1\}^{64}$   & \fr{$M_2$} & $M_1 \in_R \{0, 1\}^{848}$ \\
                                & & $F = 6A||M_1||SHA1(M_1, M_2)||BC$\\
    $F^* = RSA_{PK_{DG15}}(S)$ &\fl{$S$} & $S = RSA_{SK_{DG15}}(F)$ \\
    $6A||M_1^*||H^*||T^* = F^*$ & & \\
    $SHA1(M_1^*||M_2) ?=? H^*$ &&\\
    \end{tabular}
\end{center}

\subsubsection{Access Control and Secure Messaging}
%TODO schema
\begin{figure}[ht!]
    \centering
    \includegraphics[scale=0.5]{img/access-control}
\end{figure}
\begin{itemize}
    \item Three-pass CR protocol according to ISO 11770\text{-}2 Key Establishment
    Mechanism 6, 2-key 3DES as block cipher
    \item Nonces should be 8-byte long
    \item Encryption done using 3DES in CBC mode with zero-IV according to
    ISO 11568\text{-}2
    \item A cryptographic checksum is calculated over: ISO 9797\text{-}1 MAC
    Algorithm 3 (i.e Retail-MAC), based on DES, zero-IV, ISO 9797\text{-}1 Padding
    Method 2
    \item Encryption and MAC keys derived from the MRZ using SHA-1
\end{itemize}

\begin{tabular}{m{3cm}m{11cm}}
\textbf{Key Derivation} &
\begin{enumerate}
    \item Set $ K_{seed} = trunc_{16}(SHA-1(MRZ\_info))\quad or\ (K_r \oplus K_p)$
    \item Set $ D = K_{seed}||00000001 $
    \item Compute $ H = SHA-1(D) $
    \item First 16 bytes of H are set to the 2-key 3DES $ K_{ENC} $
    \item Set $ D = K_{seed}||00000002 $
    \item Compute $ H = SHA-1(D) $
    \item First 16 bytes of H are set to the 2 DES keys $K_{MAC} $
    \item Adjust the parity bits of the DES keys
\end{enumerate}
\end{tabular}

%TODO weaknesses
\subsection{Weaknesses}
BAC keys are derived from the MRZ, especially date of birth, date of expiry,
passport number. But passport numbers are usually not random. DOB and DOE are
neither random.

\begin{itemize}
    \item Relay Attacks: Passport is based on ISO 14443, so it can require to
    increase the timeouts.
    \item Evidence of Presence: Abuse the active authentication which can be
    doable without passing the BAC
    \item Chain og Trust: If the root certificate cannot be verified, making a
    fake passport is quite easy
\end{itemize}
