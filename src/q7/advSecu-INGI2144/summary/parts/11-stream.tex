
\subsection{Basics}
\subsubsection{Well-known}
\begin{center}
    \begin{tabular}{|c|c|c|}
        \hline
        Primitive & Legacy & Future\\
        \hline
        Rabbit & \textcolor{green!50!black}{v} & \textcolor{green!50!black}{v}\\
        SNOW3G & \textcolor{green!50!black}{v} & \textcolor{green!50!black}{v}\\
        \hline
        Trivium & \textcolor{green!50!black}{v} &
        \textcolor{red!50!black}{x}\\
        SNOW2.0 & \textcolor{green!50!black}{v} & \textcolor{red!50!black}{x}\\
        \hline
        RC4 & \textcolor{red!50!black}{x} & \textcolor{red!50!black}{x}\\
        A5/1 & \textcolor{red!50!black}{x} & \textcolor{red!50!black}{x}\\
        A5/2 & \textcolor{red!50!black}{x} & \textcolor{red!50!black}{x}\\
        E0 & \textcolor{red!50!black}{x} & \textcolor{red!50!black}{x}\\
        \hline
    \end{tabular}
\end{center}

\subsubsection{One-Time Pad} 
$$C_i = M_i \oplus K_i\quad for\ i=1,2,3,\ldots$$

\begin{tabular}{m{3cm}m{8cm}}
    Key must be:&
\begin{itemize}
    \item generated randomly
    \item at least as long as the message
    \item used once
\end{itemize}
\end{tabular}


\subsubsection{Stream ciphers}
\begin{tabular}{m{2cm}m{14cm}}
    Drawbacks: &
    \begin{itemize}
        \item Key $\neq$ Keystream (keystream is generated
            pseudo-randomly)
        \item Key is shorter than the message
        \item Key must be used \textbf{once}: An IV is appended to the key to
            counter this problem.
    \end{itemize}
\end{tabular}

\begin{center}
    \begin{tikzpicture}[node distance=0.6cm]
    \node [draw, rectangle] (S) {Stream cipher};
    \node [above = of S]  (k) {Secret key};
    \node [right=of S]  (x) {$\oplus$};
    \node [draw, rectangle,above = of x]  (p) {Plaintext};
    \node [draw, rectangle,below=of x]  (c) {Ciphertext};
    \path[->] (k) edge (S)
    (S) edge (x)
    (p) edge (x)
    (x) edge (c);
    \end{tikzpicture}
    \end{center}

As with block ciphers, stream ciphers do not ensure integrity (Associativity
and commutativity)

\paragraph{Modes}
\begin{itemize}
    \item \textbf{Synchronized} mode
    \begin{itemize}
        \item Use a single long stream during the session
        \item A and B initialize the stream with the shared key
        \item A uses the first $|m_A|$ bits of the stream to encrypt message
        $m_A$
        \item B uses the next $|m_B|$ bits of the stream to encrypt message
        $m_B$
        \item A new key must be used for each session.
    \end{itemize}

\item \textbf{Unsynchronized} mode
    \begin{itemize}
        \item The keystream is reinitialized for each new packet
        \item Initialization depends on two inputs k and IV\@.
        \item k secret but constant key
        \item IV non-secret but changing initial vector
    \end{itemize}
\end{itemize}

\subsection{Linear Feedback Shift Registers}
LFSR of length $L$ consists of $L$ stages numbered 0, 1, \ldots, L-1. 
Each stage is capable of storing one bit, having one input, one output and a
clock which controls the movements of data.

\begin{center}
    \begin{tikzpicture}
        \node [shape = rectangle](b3) {$b_{L-1}$};
        \node [shape = rectangle, right=of b3] (b2b) {$\cdots$};
        \node [shape = rectangle, right=of b2b] (b2) {$b_2$};
        \node [shape = rectangle, right=of b2] (b1) {$b_1$};
        \node [shape = rectangle, above=0.7cm of b2] (xor) {$\oplus$};
        \node [shape = rectangle, right=of b1] (b0) {$b_0$};
        \node [right=of b0] (c) {\begin{tabular}{c}Add to
        pseudo-\\random sequence\end{tabular}};


        \draw [->] (b3) -- (b2b);
        \draw [->] (b2b) -- (b2);
        \draw [->] (b2) -- (b1);
        \draw [->] (b1) -- (b0);
        \draw [->] (b1) -- (xor);
        \draw [->] (b2) -- (xor);
        \draw [dotted, ->] (b2b) -- (xor);

        \draw[->] (b0) |- (xor);
        \draw [->] (xor) -| (b3);
        \draw[->] (b0) -- (c);
    \end{tikzpicture}
\end{center}

\subsubsection{At each unit of time}
During each unit of time the following operations are performed:
\begin{enumerate}
    \item Content of stage 0 is outputed and forms part of the output
        sequence.
    \item $\forall_{1 \leq i \leq L-1}$: content of stage $i$ is moved to stage $i-1$ 
    \item Content of stage L-1 is the feedback bit $s_j$ which is
        calculated by adding together modulo 2 the previous contents of a fixed
        subsets of stages 0,1,\ldots,L-1.
\end{enumerate}

\subsubsection{4-bits LFSR example}

\begin{tabular}{m{8cm}m{8cm}}
            \begin{tikzpicture}
                \node [shape = rectangle](b3) {$b_3$};
                \node [shape = rectangle, right=of b3] (b2) {$b_2$};
                \node [shape = rectangle, right=of b2] (b1) {$b_1$};
                \node [shape = rectangle, above=0.5cm of b1] (xor) {$\oplus$};
                \node [shape = rectangle, right=of b1] (b0) {$b_0$};
                \node [right=0.5cm of b0] (c) {\scriptsize\begin{tabular}{c}Add to
                pseudo-\\random sequence\end{tabular}};


                \draw [->] (b3) -- (b2);
                \draw [->] (b2) -- (b1);
                \draw [->] (b1) -- (b0);
                \draw [->] (b1) -- (xor);

                \draw[->] (b0) |- (xor);
                \draw [->] (xor) -| (b3);
                \draw[->] (b0) -- (c);
            \end{tikzpicture}
            &
            If key $1001$ is used as seed, the generated sequence is
                    $$\textcolor{red}{1001}10101111000\textcolor{red}{1001}\cdots$$
        \end{tabular}

\subsubsection{Advantage}
\begin{itemize}
    \item LFSR should never be used by itself as keystreamn generator but they have the advantage of having low implementation costs.
    \item To destroy the linearity:
        \begin{itemize}
            \item A nonlinear combining function on the outputs of several LFSRs
            \begin{center}
                \begin{tikzpicture}[node distance=0.5cm]
            		\node [draw,shape=circle] at (0,0) (func){f};
            		\node [draw,shape=rectangle, left = 1cm of func] (lfsr2) {LFSR2};
                    \node [draw,shape=rectangle,above = of lfsr2] (lfsr1) {LFSR1};
            		\node [right = of func] (keystream) {keystream};
            		\node [below = 0cm of lfsr2] (dot) {\vdots};
            		\node [draw,shape=rectangle, below = 0cm of dot] (lfsrn) {LFSRn};
            		
            		\draw [->] (lfsr1) -- (func);
            		\draw [->] (lfsr2) -- (func);
            		\draw [->] (lfsrn) -- (func);
            		\draw [->] (func) -- (keystream);
            	\end{tikzpicture}
            \end{center}
            \item A nonlinear filtering function on the contents of a single LFSR
            \begin{center}
                \begin{tikzpicture}[node distance=0.6cm]
                \node [shape = rectangle](b3) {$b_3$};
                \node [shape = rectangle, right=of b3] (b2) {$b_2$};
                \node [shape = rectangle, right=of b2] (b1) {$b_1$};
                \node [shape = rectangle, above=0.5cm of b1] (xor) {$\oplus$};
                \node [shape = rectangle, right=of b1] (b0) {$b_0$};
                
                \node [below = 0.5cm of b0] (a) {};
                \node [below = 0.5cm of b1] (b){};
                \node [below = 0.5cm of b2] (c){};
                \node [below = 0.5cm of b3] (d){};
                
                \node [draw,circle, below = of b1] (func) {f};
                \node [right = 1cm of func] (keystream) {keystream};


                \draw [->] (b3) -- (b2);
                \draw [->] (b2) -- (b1);
                \draw [->] (b1) -- (b0);
                \draw [->] (b1) -- (xor);

                \draw[->] (b0) |- (xor);
                \draw [->] (xor) -| (b3);

                
                \draw [->] (b0) -- (func);
                \draw [->] (b1) -- (func);
                \draw [->] (b2) -- (func);
                \draw [->] (b3) -- (func);
                \draw [->] (func) -- (keystream);
            \end{tikzpicture}
            \end{center}
            \item The output of one (or more) LFSRs to control the clock of one
                (or more) other LFSRs.
		\begin{center}
            \begin{tikzpicture}
				\node [draw,shape=circle] (func) {f};
				\node [draw,shape=rectangle, above left =0.3cm and 1cm of func] (lfsrc) {LFSR clocking};
				\node [draw,shape = rectangle, below left = 0.3cm and 1cm of func] (lfsr) {LFSR};
				\node [below right =0.3cm and 3cm of func] (tmp) {output $a_i$};
				\node [above right = 0.3cm and 3cm of func] (tmp1) {discard $a_i$};
				
				\draw [->] (lfsrc) -| (func) node [near start,above] {$s_i$};
				\draw [->] (lfsr) -| (func) node [near start,above] {$a_i$};
				\draw [->] (func) -- (tmp) node [midway,below] {$f(s_i) = 0$};
				\draw [->] (func) -- (tmp1) node [midway,above] {$f(s_i)= 1$};
			\end{tikzpicture}
		\end{center}
        \end{itemize}
\end{itemize}

\subsection{RC4}
Use a secret key of length from 1 to 256 bytes and is efficient in
software.

\begin{itemize}
    \item Array $K$ contains $L$ bytes of key 
    \item $K_i$ denote $K_{(i mod L)}$
    \item Array $S$ always has a permutation of $0, 1, \cdots, N-1$
\end{itemize}
\begin{center}
\begin{tabular}{m{4cm}m{4cm}m{4cm}}
    Initialization & Scrambling & Keystream generation \\
    \begin{lstlisting}[mathescape]
for i in (0 to N-1):
    $S_i$ = i
    \end{lstlisting}
    &
    \begin{lstlisting}[mathescape]
j =0
for i in (0 to N-1):
    j = (j + $S_i$ + $K_i$ ) mod N
    Swap($S_i, S_j$ )
    \end{lstlisting}
    &
    \begin{lstlisting}[mathescape]
i = (i + 1) mod N
j = (j + $S_i$ ) mod N
Swap($S_i , S_j$ )
Output $S_{(S_i +S_j ) mod N}$
\end{lstlisting}
\end{tabular}
\end{center}


\subsection{Conclusion}
Using a stream cipher is tricky \textbf{but} the security of stream ciphers is not
well-established, therefore it should be used carefully.


