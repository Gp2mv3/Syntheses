\documentclass[en]{../../../../../../eplexam}

\hypertitle{Communication systems}{7}{ELEC}{2795}{2022}{Septembre}{All}
{Julien Giunta}
{Jérôme Louveaux, Claude Oestges, Luc Vandendorpe and Charles Wiame}

\section{Question 1}

% Insérer ci-dessous la solution à la question
We consider the downlink of a cellular transmission at 900 MHz, using a bandwidth of 30 kHz per user. At the base station, the transmit power is 20 W and the antenna gain is 6 dB. For the considered environment, the path-loss is given in [dB] by :

$$[L]_{dB} = 129 + 10 n \log_{10} d + [S]_{dB}$$

where $d$ is the distance in [km], $n = 3.5$ is the path-loss exponent and $[S]_{dB}$ is the lognormal shadowing : $[S]_{dB}$ is therefore a zero-mean Gaussian variable with a standard deviation of 8 dB. The mobile terminal is characterized by an antenna of 1 dB gain and a noise figure (measured at 290 K, wich is also taken as the ambient temperature) of 9 dB. Small-scale fading is neglected in the entire exercise.

\paragraph{1.}
What should be the cell radius $R$ to garantee that the averaged received power is at least -100 dBm over the entire cell ?

\paragraph{2.}
What is therefore the average SNR at the cell edge (i.e. at a distance $d=R$) ?

\paragraph{3.}
What is the probability that the received power falls below -120 dBm at the cell edge ?

\paragraph{4.}
At which distance D can the network operator re-use the same frequency to maintain a Signal-to-Interference ratio (SIR) of 18 dB, considering ideal hexagonal cells ? What is in this case the required frequency  reuse factor $K$ ? As a reminder, for hexagonal cells, the re-use distance $D$ is given by $R\sqrt{3K}$ ($R$ being the cell radius).


\nosolution

\section{Question 2}
\subsection{Short-answer questions}
An answer and a brief justification (max. 3-4 lines) is expected for each of the items below.

\paragraph{1.}
Briefly explain the role of the cyclic prefix in the OFDM modulation.

\paragraph{2.}
We consider a wireless system with N transmit antennas and one receive antenna. All transmit antennas send the same symbol sequence to the receiver and are affected by independent Rayleigh fading channels. The transmit power budget is given by $P_t$ and is uniformly distributed between the transmit antennas : each of them therefore transmits with power $\frac{P_t}{N}$. The received signal is also affected by an additive white Gaussian noise. Provide the diversity and array gains of the system. How would the system performance evolve as $N \rightarrow \infty$ ?

\paragraph{3.}
We consider a wireless transmission between a transmitter and a receiver. The channel is frequency selective. In addition, the received signal is affected by an additive white Gaussian noise of high noise power. The engineer in charge of the equalization can either implement a zero forcing equalizer, or a MMSE equalizer. Which of the two types should most likely be selected ?

\subsection{Wiener filtering}

We are interested in estimating a signal $d_n$ from noisy observations $x_n$ :

$$x_n = d_n + v_n$$

where $v_n$ is a zero-mean (not necessarily white) noise independent from $d_n$. We consider a smoother which hence produces the estimations as

$$\hat{d_n} = \sum_{k=-p}^p w_k x_{n-k}$$

For certain applications, it is important to have an equaliser whose coefficients fulfill the property $w_n = w_{-n}$ (linear phase property).

\paragraph{1.} Starting from the MMSE criterion, provide the Wiener-Hopf equations that must be fulfilled by the $w_n$ when one requires this property $w_n = w_{-n}$. Obtain the final matrix form of these Wiener-Hopf equations.

\paragraph{2.} Once you have obtained these equations in a matrix form, compute the optimal linear phase filter when the $d_n$'s are independent zero-mean symbols with a variance $\sigma_d^2$ and $v_n$ is a zero-mean white noise with variance $\sigma_v^2$.

\nosolution

\section{Question 3}

An OFDM system is transmitting some information around a carrier frequency of 2.4 GHz, using an FFT size of 512 with a subcarrier spacing $\Delta f = 10$ kHz. The bandwidth allocated to the system is 4 MHz (2.4 GHz $\pm$ 2 MHz). In the environment, the delay spread of the channel is around $1 \mu$s.

\paragraph{1.} What is the sampling rate of the system ?

\paragraph{2.} What do you suggest to ensure that the bandwidth restrictions are satisfied ?

\paragraph{3.} Is it a narrowband or a wideband system ?

\paragraph{4.} Suggest a value of the length of the cyclic prefix.

\paragraph{5.} What is the total duration of an OFDM symbol (including cyclic prefix) ?

\paragraph{6.} If all the used subcarriers are able to carry 4-QAM, what is the total bit rate of the system ?

\paragraph{7.} In practice, due to the frequency selective fading, some subcarriers will have a lower SNR and may be subject to high BER. What do you suggest to solve this issue if you are forced to use 4-QAM on all subcarriers ?

\paragraph{8.} What do you suggest to solve the same issue if you are allowed to modify the constellations independently on each subcarriers ?
\nosolution

\end{document}
