\documentclass[fr]{../../../../../../eplexam}
\usepackage{../../../../../../eplcode}
\usepackage{syntax}

\hypertitle{Logique et structures discrètes}{5}{INGI}{1101}{2018}{Janvier}{Majeure}
{Gilles Charlier}
{Peter Van Roy}

\lstset{emph={%  
   {:-}%
     },emphstyle={\color{red}\bfseries\underbar}%
}%

\section{Logique propositionnelle (5 points)}
\paragraph{Remarque}
Vous obtiendrez la cote maximale entre la première question de l'examen
et la cote de l'interro mi-quadri.
\begin{enumerate}
	\item
	Définissez le schéma de la preuve par contradiction
	et justifiez-le en raisonnant sur les interprétations.
	\item L'algorithme de preuve est complet et adéquat, qu'est-ce que cela veut dire?
	\item Définissez la sémantique propositionnelle:
	\begin{enumerate}
		\item D'abord, donnez une grammaire de la syntaxe de la logique propositionnelle
		\item Définissez le concept d'interprétations d'une formule en logique
		\item Définissez le concept de modèles d'un ensemble de formules
	\end{enumerate}
\end{enumerate}

\begin{solution}

\begin{enumerate}
    \item On suppose les prémisses $p,\ldots,q$ vraies (càd on ne peut trouver de contradiction). On ajoute $r$ aux prémisses. Si il est possible de prouver $s$ et $\neg s$, cela signifie qu'il y a une erreur dans les prémisses, on suppose que l'ajout $r$ est fautif.
    \begin{center}
            \begin{tabular}{c}
            $p,...,q,r \vdash s$ \\
            $p,...,q,r \vdash \lnot s$\\
            \hline
            $p,...,q \vdash \lnot r$
            \end{tabular}
    \end{center}
    On justifie qu'il n'y a pas de contradiction dans $p,\ldots,q$ car on part du principe qu'il existe un modèle de $p,\ldots,q$.
    
    \item 
    \begin{itemize}
    \item \textbf{Complet} : si $p \models q$, alors $p \vdash q$.
    
    C’est-à-dire, si $q$ est vrai quand $p$ est vrai, alors l’algorithme peut trouver une preuve.
    \item \textbf{Adéquat / consistent} : si $p \vdash q$ alors $p \models q$.
    
    C’est-à-dire, si l’algorithme peut prouver la requête $q$ à partie du programme $p$, alors $q$ est effectivement vrai quand $p$ est vrai.
    \end{itemize}
    
    \item 
    \begin{enumerate}
     \item 
        \begin{grammar}
        <identificateur> ::= A | B | C | D | \ldots
        
        <proposition> ::= true 
        \alt false 
        \alt <identificateur> 
        \alt (<proposition>) 
        \alt $\neg$<proposition> 
        \alt <proposition> $\land$ <proposition> 
        \alt <proposition> $\lor$ <proposition> 
        \alt <proposition> $\Rightarrow$ <proposition> 
        \alt <proposition> $\Leftrightarrow$ <proposition>
        \end{grammar}
     \item L'\textbf{interprétation} est une fonction: $val_{I}(Ep)\rightarrow \left \{true, false \right \}$.
     \item Un \textbf{modèle} est une interprétation telle que    $VAL_{I}(p) = True$.
    \end{enumerate}
\end{enumerate}
 
\end{solution}


\section{Prolog (7 ou 8 points)}
Considérez le programme suivant
\begin{lstlisting}[language=Prolog]
sumList([],0).
sumList([H|T],S):- sumList(T,S1), plus(H,S1,S).
\end{lstlisting}
et la requête suivante
\begin{lstlisting}[language=Prolog]
?- sumList([5,6],L).
\end{lstlisting}

\begin{enumerate}
	\item Donnez la formule normale conjonctive (FNC).
	\item Donnez la résolvante initiale.
	\item Pourquoi la première clause n'est pas résoluble?
	\item Donnez la deuxième étape de résolution. N'oubliez pas
	de remplacer les noms de variables pour éviter d'éventuelles erreurs.
	\item
	Effectuez les dernières étapes de résolution.
	Vous pouvez supposer que le prédicat \lstinline$plus(x,y,z)$ est résoluble
	si deux des arguments sont des entiers pour satisfaire l'équation $x+y=z$.
\end{enumerate}

\begin{solution}
 
\begin{enumerate}
 \item sumlist(nil, 0)
 
    $\land$
    
    sumlist(cons(H, T), S) $\lor \neg$ sumlist(T, S1) $\lor \neg$ plus(H, S1, S)
 \item r = \{ sumlist(cons(5, cons(6, nil)), L\}
 \item La première clause ne peut pas être combinée/résolue avec la résolvante initiale car nil et [5|6] sont des constantes, on ne peut donc pas substituer l'une par l'autre.
 
 La substitution avec la deuxième clause sera: $\sigma$ = \{(H, 5), (T, cons(6, nil)), (S, L)\}.
 
 La nouvelle réolvante sera: r = \{sumlist(cons(6, nil), S1) $\land$ plus(5, S1, L)\}.
 \item Pour l'étape suivante il faut renommer des variables : H $\rightarrow$ H', S $\rightarrow$ S', T $\rightarrow$ T', S1 $\rightarrow$ S1'.
 
 sumlist(nil, 0) $\land$ sumlist(cons(H', T'), S') $\lor \neg$ sumlist(T', S1') $\lor \neg$ plus(H', S1', S')
 
 La nouvelle substitution est donc: $\sigma' = \sigma \cup$ \{(H', 6), (T', nil), (S', S1)\}.
 
 La nouvelle résolvante: r = \{sumlist(nil, S1') $\land$ plus(6, S1', S1) $\land$ plus(5, S1, L)\}.
 \item $\sigma '' = \sigma ' \cup$ \{(S1', 0)\}\\
 r = \{plus(6, 0, S1) $\land$ plus(5, S1, L)\}\\
 $\sigma ''' = \sigma '' \cup$ \{(S1, 6)\}\\
 r = \{plus(5, 6, L)\}\\
 $\sigma '''' = \sigma ''' \cup$ \{(L, 11)\}\\
 r = \{\}\\
 r est vide, l'exécution est terminée, on peut substituer L par 11.
\end{enumerate}

\end{solution}


\section{Graphes (7 ou 8 points)}
\begin{enumerate}
	\item
	Définissez le concept d'équilibre structurel faible
	comme une propriété locale du graphe.
	\item
	Donnez l'énoncé du théorème d'équilibre structurel faible
	qui fait un lien entre la propriété locale et la propriété globale du graphe.
	\item Donnez la preuve du théorème, faites attention au raisonnement récursif.
\end{enumerate}

\begin{solution}
 
\begin{enumerate}
 \item L’équilibre structurel faible, à l’instar de l’équilibre structurel fort, se base sur la notion de stabilité des triangles du graphe. Mais contrairement à l’équilibre fort, on ajoute deux cas instables.
 
 \item Si un graphe complet annoté est faiblement équilibré, alors on peut diviser ses noeuds en groupes:
    \begin{enumerate}
        \item Dans un groupe: amis
        \item Dans un autre groupe: ennemis
    \end{enumerate}
 
 \item 
 \begin{enumerate}
        \item Noeud $A$ + tous ses amis appartiennent au groupe $X$ \\ $\rightarrow$ amis mutuels car $(+ + +)$.
        \item $A$ + ses amis sont ennemis avec tous les autres
        \item Enlever $A$ + ses amis : nouveau graphe \\ $\rightarrow$ raisonnement récursif.
 \end{enumerate}
\end{enumerate}
 
\end{solution}

\end{document}
