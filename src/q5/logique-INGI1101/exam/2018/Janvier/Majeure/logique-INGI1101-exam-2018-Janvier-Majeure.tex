\documentclass[fr]{../../../../../../eplexam}
\usepackage{../../../../../../eplcode}
\hypertitle{Logique et structures discrètes}{5}{INGI}{1101}{2018}{Janvier}{Majeure}
{Gilles Charlier}
{Peter Van Roy}


\lstset{emph={%  
   {:-}%
     },emphstyle={\color{red}\bfseries\underbar}%
}%

\section{Logique propositionnelle (5 points)}
\paragraph{Remarque}
Vous obtiendrez la cote maximale entre la première question de l'examen
et la cote de l'interro mi-quadri.
\begin{enumerate}
	\item
	Définissez le schéma de la preuve par contradiction
	et justifiez-le en raisonnant sur les interprétations.
	\item L'algorithme de preuve est complet et adéquat, qu'est-ce que cela veut dire?
	\item Définissez la sémantique propositionnelle:
	\begin{enumerate}
		\item D'abord, donnez une grammaire de la syntaxe de la logique propositionnelle
		\item Définissez le concept d'interprétations d'une formule en logique
		\item Définissez le concept de modèles d'un ensemble de formules
	\end{enumerate}
\end{enumerate}

\nosolution

\section{Prolog (7 ou 8 points)}
Considérez le programme suivant
\begin{lstlisting}[language=Prolog]
sumList([],0).
sumList([H|T],S):- sumList(T,S1), plus(H,S1,S).
\end{lstlisting}
et la requête suivante
\begin{lstlisting}[language=Prolog]
?- sumList([5,6],L).
\end{lstlisting}

\begin{enumerate}
	\item Donnez la formule normale conjonctive (FNC).
	\item Donnez la résolvante initiale.
	\item Pourquoi la première clause n'est pas résoluble?
	\item Donnez la deuxième étape de résolution. N'oubliez pas
	de remplacer les noms de variables pour éviter d'éventuelles erreurs.
	\item
	Effectuez les dernières étapes de résolution.
	Vous pouvez supposer que le prédicat \lstinline$plus(x,y,z)$ est résoluble
	si deux des arguments sont des entiers pour satisfaire l'équation $x+y=z$.
\end{enumerate}

\nosolution

\section{Graphes (7 ou 8 points)}
\begin{enumerate}
	\item
	Définissez le concept d'équilibre structurel faible
	comme une propriété locale du graphe.
	\item
	Donnez l'énoncé du théorème d'équilibre structurel faible
	qui fait un lien entre la propriété locale et la propriété globale du graphe.
	\item Donnez la preuve du théorème, faites attention au raisonnement récursif.
\end{enumerate}

\nosolution

\end{document}
