\documentclass[fr]{../../../../../../eplexam}
\usepackage{../../../../../../eplcode}

\hypertitle{Logique et structures discrètes}{5}{INGI}{1101}{2017}{Janvier}{Majeure}
{Nicolas Vanvyve\and Jean-Martin Vlaeminck\and Divers contributeurs\thanks{
Christophe Crochet (graphes, méthode scientifique),
Luca Derumier (induction, déduction, abduction),
Maxime de Streel (diverses remises en page).
}}
{Peter Van Roy}

% Crochet: partie graphes, avec les pas sûr, la méthode scientifique
% Derumier: réponses à induction, déduction, abduction
% de Streel: création du document, mises en page

\paragraph{Remarque}
Les solutions ont été proposées par de nombreux étudiants
lors de l'année académique 2017--2018, dans un document partagé (ShareLaTeX).
L'exactitude des réponses n'est dès lors pas acquise, mais elles peuvent servir
de point de repère quand à ce qui était demandé pendant l'examen.

\section{Base de la logique}
\paragraph{Note:}
\textit{Pour la question 1, vous aurez le maximum entre la note de l'interro et celle de la question.}

\begin{enumerate}
	\item Définissez les trois formes de raisonnement que l'on connaisse:
	déduction, abduction, induction. Donnez un exemple de chaque forme.
	\item Expliquez le méthode scientifique avec les trois formes de raisonnement.
	\item Définissez la sémantique en logique des prédicats.
	Vous pouvez faire cela en trois étapes:
	\begin{enumerate}
		\item d'abord, donnez une grammaire de la syntaxe de la logique des prédicats;
		\item ensuite, définissez le concept d'interprétation d'une formule logique;
		\item enfin, définissez le concept de modèle d'un ensemble de formules.
	\end{enumerate}
\end{enumerate}

\begin{solution}
\begin{enumerate}
\item
\begin{description}
	\item[Déduction]
	Il s’agit de faire des calculs et des raisonnements logiques par rapport à une théorie.
	Avec ces raisonnements, on déduit le résultat qu’une expérience donnerait selon la théorie.
	Par exemple, en utilisant les équations de Maxwell on peut déduire
	la trajectoire d’un objet avec une charge électrique dans un champ électromagnétique.
	\item[Induction]
	L’induction est le fait de trouver une règle générale à partir des expériences répétées.
	On choisit en général une règle moyenne qui deviendra la règle générale.
	Il faut souligner que les résultats expérimentaux ne sont pas totalement fiables ou complets.
	Dès lors, la règle trouvée n’est pas nécessairement exacte.
	Par exemple, si par induction nous avons trouvé la règle, \og les oiseaux volent\fg{},
	cela est vrai tant que l’on n’a pas vu un pingouin.
	Autre exemple, nous pouvons supposer que demain le soleil va se lever
	comme depuis des milliers d’années, même si rien ne l’assure.
	\item[Abduction]
	On compare la règle générale trouvée lors de l’induction avec la théorie.
	S’il y a une incohérence entre la règle générale et la théorie
	qui ne rentre pas dans la marge d’erreur expérimentale,
	on suppose qu’il y a une erreur dans la théorie.
	Il faut alors corriger la théorie existante ou en inventer/deviner une nouvelle.
	Ce type de raisonnement s’appelle l’abduction:
	trouver une explication (= la théorie corrigée) pour une règle ou un fait.
	On applique l’abduction couramment dans la vie de tous les jours;
	par exemple, lorsqu’un élève entre trempé dans la classe, nous supposons
	qu’il pleut dehors. La pluie est une explication possible pour l’état de l’élève.
\end{description}
\item La méthode scientifique consiste, afin de formaliser un système
dans le monde réel, de faire une abstraction vers un modèle théorique (théorie).
\item \nosubsolution
\end{enumerate}
\end{solution}

\section{Prolog}
Considérez le programme Prolog suivant qui fait la définition logique
du dernier élément d'une liste et qui définit en même temps un programme
pour calculer le dernier élément d'une liste:
\begin{lstlisting}
last([L|],L).
last([H|T],L) :-last(T,L).
\end{lstlisting}
avec la requête:
\begin{lstlisting}
?- last([1,2],L).
\end{lstlisting}

Répondez au questions suivantes:
\begin{enumerate}
	\item Quelle est la forme normale conjonctive de ce programme?
	\item Quelle est la résolvante initiale qui correspond a la requête?
	\item
	En faisant une première résolution avec le programme, expliquez pourquoi
	la première clause n'est pas résoluble avec la résolvante initiale.
	Ensuite, donnez la substitution et la nouvelle résolvante obtenue
	en faisant une résolution avec la deuxième clause.
	\item
	Faites une deuxième résolution avec le programme pour éviter une erreur.
	Quelle est la solution trouvée par le programme?
	\item
	[Bonus] Est ce que cette requête a d'autre solutions?
	Si oui, expliquez pourquoi. Si non expliquez pourquoi pas.
\end{enumerate}

\nosolution{}

\section{Structures discrètes}
Pour cette question, vous allez investiguer les réseaux sociaux dans leurs contextes.
\begin{enumerate}
	\item
	Définissez avec précision le concept de réseau d'affiliation social.
	Une propriété importante des personnes dans un tel réseau est la similitude:
	les personnes avec des liens entres eux tendent à se ressembler.
	\item
	Définissez les mécanisme de sélection et  d'influence sociale, qui peuvent
	augmenter la similitude entre les n\oe{}uds d'un tel réseau d'affiliation social.
	\item
	Définissez les trois différentes formes de fermeture
	qui peuvent faire évoluer un réseaux d'affiliation sociale.
	\item
	En utilisant l'exemple de Wikipedia, expliquez comment
	on peut comparer les effets de la sélection et de l'influence sociale.
	Définissez d'abord Wikipedia comme un réseau d'affiliation social
	et comment la sélection et l'influence sociale se montrent.
	Ensuite résumez les résultats d'une expérience qui compare les effets des deux mécanismes.
\end{enumerate}

\begin{solution}
\begin{enumerate}
\item
Un réseau d'affiliation social (qui est toujours bipartite) permet d'inclure
les facteurs de similitudes sociales dans un graphe,
il est donc dynamique et peut donc changer au cours du temps.
Ainsi on distingue 2 types de n\oe{}uds: les noeuds \og Focus \fg{} représentant
un centre d'intérêt, une activité et les n\oe{}uds \og Personne \fg{} représentant un individu physique.
Ces types de n\oe{}uds vont définir à leur tour deux types de liens;
les liens \og Personne-Personne \fg{} et les liens \og Personne-Focus \fg{}.
\item
(L'auteur de cette réponse a indiqué son incertitude.)
La notion sociologique de similitude entraîne de nouveaux mécanismes nécessaires
pour comprendre les raisons de formations de liens entre deux individus.
Ainsi on distingue la \og sélection \fg{} qui implique que
les similitudes avant la rencontre des gens se font par choix
(souvent par rapport aux caractéristiques communes),
et l'\og influence sociale \fg{} qui implique que des liens peuvent se créer
par l'influence des amis, personnes avec qui on est en contact.
\item
Fermeture triadique qui implique d'une personne ayant de forts liens
avec deux autres peut engendrer au moins un lien faible entre ceux ci.

Fermeture focale qui implique que deux personnes ayant des similitudes
dans leurs centres d'intérêts peuvent devenir amis.

Fermeture d'adhésion qui implique qu'un personne ami avec une autre
peut influencer ce dernier à rejoindre son activité.
\item
Voir page 100 et suivantes du livre \textit{Networks, Crowds and Markets}.
\end{enumerate}
\end{solution}

\section{}
Quelle est la loi qui gouverne la popularité des pages Web?
Pour quantifier cette loi, définissez d'abord une mesure pour la popularité d'une page Web.
Ensuite, donnez le modèle d'attachement préférentiel qui explique cette loi.
Enfin, expliquez le phénomène de la longue traine et comment cette loi donne lieu à ce phénomène.

\nosolution{}

\end{document}
