\documentclass[fr]{../../../../../../eplexam}

\hypertitle{Logique et structures discrètes}{5}{INGI}{1101}{2019}{Janvier}{Majeure}
{Arthur van Stratum}
{Peter Van Roy}

\section{Base de la logique}
Cette question porte sur l’algorithme de preuve pour la logique des propositions.
\begin{enumerate}
	\item Donnez le pseudocode de l’algorithme de preuve pour la logique des propositions.
		Expliquez bien les entrées de cet algorithme.
	\item Donnez la règle de résolution et expliquez comment elle marche.
	\item Quelles sont les trois propriétés théoriques de cet algorithme?
		Définissez tous les termes que vous utilisez pour expliquer ces propriétés.
	\item Supposez les deux axiomes $Z \implies (Q \land R)$ et $(R \implies Z)$.
		Utilisez l’algorithme de preuve pour prouver $(R \implies Q)$.
		Attention à montrer exactement l’éxécution de l’algoritme, avec les entrées correctes et une sortie correcte.
\end{enumerate}

\nosolution

\section{Logique des prédicats}
Cette question concerne les preuves manuelles en logique des prédicats.
\begin{enumerate}
	\item Définissez avec précision les deux règles « Élimination de $\exists$ » et « Élimination de $\forall$ ».
	\item Définissez avec précision les deux règles « Introduction de $\exists$ » et « Introduction de $\forall$ ».
	\item Donnez une preuve manuelle pour
		\[
			\forall x(P_{(x)} \implies Q_{(x)}) \implies (\exists x P_{(x)} \implies \exists x Q_{(x)})
		\]
		attention aux justifications pour toutes les règles!
\end{enumerate}

\nosolution

\section{Graphes}
Pour cette question, vous allez investiguer l’équilibre structurel faible.
\begin{enumerate}
	\item Définissez avec précision le concept d’équilibre stucturel faible comme une propriété locale d’un graphe.
	\item Donnez l’énoncé du théorème d’équilibre structurel faible qui fait un lien entre la propriété locale et une proriété globale du graphe.
	\item Donnez la preuve du théorème.
		Faites attention au raisonnement récursif.
\end{enumerate}

\nosolution

\end{document}
