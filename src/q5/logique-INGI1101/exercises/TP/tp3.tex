\section{TP 3}

\subsection*{Exercice 1}

Démontrez avec un preuve formelle que $P \Rightarrow (Q \Rightarrow P)$ est une tautologie.

\subsubsection*{Solution}

\begin{tabular}{|l|l|}
	\hline
	\indent 1. $P$ & supposition \\
    \indent \indent 2. $Q$& supposition\\
    \indent \indent 3. $P$ & (1)\\
    \indent 4. $P\Rightarrow Q$& thm déduction (2-3)\\
    \indent 5. $Q\Rightarrow (P\Rightarrow Q)$& thm déduction (1-4)\\
    \hline
\end{tabular}

\subsection*{Exercice 2}
Une formule propositionelle est en \textit{forme normale disjonctive} si elle est composée de disjonctions de conjonctions de literaux, cet-`a-dire, elle est de la forme

$$\bigvee_{i=1}^{n} \bigwedge_{j=1}^{m} L_{ij}$$

où $L_{ij}$ sont des literaux.
\begin{enumerate}
	\item Quel est l’avantage d’une formule en forme normale disjonctive ?
	\item Comment peut-on faire pour automatiser la transformation d’une table de vérité en une formule qui possède la même table de vérité ?
	\item Est-ce que toute formule propositionelle peut s’écrire en forme normale disjonctive ?
\end{enumerate}


\subsubsection*{Solution}

\begin{enumerate}
	\item On peut facilement trouver un modèle, il suffit de satisfaire une des conjonctions.
	\item Pour chaque ligne qui rends la formule vraie on construit une conjonction avec un literal positive pour chaque T et un negatif pour chaque F. Après on prends la disjonction de tous.

	Exemple :


	\begin{tabular}{cc|ccc}
		$p$ & $q$ & &\\
		\cline{1-3}
		T&T&T&$\rightarrow$& $(p \land q)$\\
		T&F&F&&\\
		F&T&T&$\rightarrow$ & $(\neg p \land q)$\\
		F&F&F&&\\
	\end{tabular}

	$$ (p \land q) \lor (\neg p \land q)$$
	\item Oui par le point 2.
\end{enumerate}

\subsection*{Exercice 3}
Mettez les formules suivantes en forme normale conjonctive.
\begin{enumerate}
	\item $ p \oplus q$
	\item $ (p \Rightarrow q) \Rightarrow r  $
	\item $ p \Rightarrow (q \Rightarrow r)  $
	\item $ (p \land q \land \neg s)\lor (\neg p \land q \land s) $
	\item $ (p\Leftrightarrow q) \Leftrightarrow r $
	\item $ (a_1 \land b_1 ) \lor (a_2 \land b_2 ) \lor ... \lor (a_n \land b_n )  $

\end{enumerate}

\subsubsection*{Solution}
\begin{enumerate}
	\item \hspace{1em}
	\begin{center}
		\begin{tabular}{cc|c|cccc}
		$p$ & $q$ & $p \oplus q$ & $((p\lor q)$ &$\land$&$\neg$&$(p \land q))$\\
		\hline
		T&T&F&T&F&F&T\\
		T&F&T&T&T&T&F\\
		F&T&T&T&T&T&F\\
		F&F&F&T&F&T&F\\
		\end{tabular}
	\end{center}
$$p \oplus q  \Lleftarrow\!\!\!\!\Rrightarrow (p\lor q)\land\neg (p \land q) \Lleftarrow\!\!\!\!\Rrightarrow (p\lor q)\land (\neg p\lor \neg q)$$

	\item \hspace{1em}

$$ (p \Rightarrow q) \Rightarrow r \Lleftarrow\!\!\!\!\Rrightarrow \neg(p \Rightarrow q) \lor r \Lleftarrow\!\!\!\!\Rrightarrow \neg (\neg p \lor q ) \lor r \Lleftarrow\!\!\!\!\Rrightarrow (p \land \neg q) \lor r \Lleftarrow\!\!\!\!\Rrightarrow (p \lor r) \land (\neg q \lor r)$$
	\item \hspace{1em}

$$ p \Rightarrow (q \Rightarrow r) \Lleftarrow\!\!\!\!\Rrightarrow \neg p \lor (q \Rightarrow r) \Lleftarrow\!\!\!\!\Rrightarrow \neg p \lor (\neg q \lor r) \Lleftarrow\!\!\!\!\Rrightarrow \neg p (\neg q \lor r) $$
	\item \hspace{1em}

\begin{flushright}
$(a + b + c) * (e + f + g) = ae + af + ag + be + bf + bg + ce + cf + cg$\\
$(p \land q \land \neg s)\lor (\neg p \land q \land s) \Lleftarrow\!\!\!\!\Rrightarrow$\\
$(p \lor \neg p) \land (p \lor q) \land (p \lor s) \land$\\
$(q \lor \neg p) \land (q \lor q) \land (q \lor s) \land$\\
$\neg s \lor \neg p) \land (\neg s \lor q) \land (\neg s \lor s) \Lleftarrow\!\!\!\!\Rrightarrow$\\
$(p\lor q) \land ( p\lor s) \land (q\lor \neg p)\land q \land (q\lor s)\land (\neg s\lor \neg p ) \land (\neg s \lor q)$\\

\end{flushright}
	\item \hspace{1em}

On a que $$ (p \Leftrightarrow q) \Leftrightarrow r \Lleftarrow\!\!\!\!\Rrightarrow ((p\Leftrightarrow q) \Rightarrow r ) \and ( r \Rightarrow (p \Leftrightarrow q)) $$

On va faire les deux parties séparément :
\begin{flushright}
	$(p \Leftrightarrow q) \Leftrightarrow r \Lleftarrow\!\!\!\!\Rrightarrow \neg(p \Leftrightarrow q) \lor  r \Lleftarrow\!\!\!\!\Rrightarrow (p \oplus q) \lor r \Lleftarrow\!\!\!\!\Rrightarrow$\\
	$ [ (p \lor q) \land (\neg p \lor \neg q) ] \lor r \Lleftarrow\!\!\!\!\Rrightarrow (p \lor q \lor r) \land (\neg p \lor \neg q \lor r) $\\
\end{flushright}
et
\begin{flushright}
	$ r \Rightarrow (p \Leftrightarrow q) \Lleftarrow\!\!\!\!\Rrightarrow \neg r \lor (p \Leftrightarrow q) \Lleftarrow\!\!\!\!\Rrightarrow  \neg r  \lor ( (p \land q) \lor (\neg p \land \neg q)) \Lleftarrow\!\!\!\!\Rrightarrow $\\
	$ \neg r \lor (p \land q) \lor (\neg p \land \neg q) \Lleftarrow\!\!\!\!\Rrightarrow $\\
	$ (\neg r \lor p \neg p) \land (\neg r \lor p \lor \neg q) \land (\neg r \lor q \lor \neg p) \land (\neg r \lor q \lor /neg q) \Lleftarrow\!\!\!\!\Rrightarrow $\\
	$ \textbf{true} \land (\neg r \lor p \lor \neg q) \land (\neg r \lor q \lor \neg p) \land  \textbf{true} \Lleftarrow\!\!\!\!\Rrightarrow $\\
	$ (\neg r \lor p \lor \neg q) \land (\neg r \lor q \lor \neg p) $\\
\end{flushright}
Donc
\begin{flushright}
	$(p \Leftrightarrow q ) \Leftrightarrow r \Lleftarrow\!\!\!\!\Rrightarrow (p \lor q \lor r) \land (\neg p \lor \neg q \lor r) \land (\neg r \lor p \lor \neg q) \land (\neg r \lor q \lor \neg p)$
\end{flushright}

	\item \hspace{1em}
\begin{align*}
	(a_1 \land b_1) \lor (a_2 \land b_2) \lor ... \lor (a_n \land b_n) & \Lleftarrow\!\!\!\!\Rrightarrow \\
	(a_1 \lor a_2 \lor ... \lor a_{n-1} \lor a_n) \land &\\
	(a_1 \lor a_2 \lor ... \lor a_{n-1} \lor b_n) \land &\\
	... \land &\\
	(b_1 \lor b_2 \lor ... \lor b_{n-1} \lor b_n) \land &\\
\end{align*}

Contient $2^n$ clauses, chacune contient soit $a_i$ soit $b_i$ à la position $i$.

\end{enumerate}

\subsection*{Exercice 4}
Montrez que la règle d'inférence suivante est valide :

\subsubsection*{Solution}

\begin{tabular}{|l|l|}
\hline
1. $p$ & prémisse \\
2. $ p \Rightarrow q $ & prémisse \\
3. $ p \Rightarrow q \land q \Rightarrow p$ & loi de l'équivalence (2)\\
4. $ p \Rightarrow q$ & simplifcation (3)\\
5. $ q$ & modus ponens (1,4)\\
\hline
\end{tabular}\\

\subsection*{Exercice 5}
Montrez que la règle d'inférence suivante est valide :


\subsubsection*{Solution}

\begin{tabular}{|l|l|}
\hline
1. $ \neg p $ & prémisse \\
2. $ p \Leftrightarrow q $ & prémisse \\
3. $ p \Rightarrow q \land q \Rightarrow p $ & loi de l'équivalence (2)  \\
4. $ q \Rightarrow p $ & simplification (3) \\
5. $ \neg q $ & modus tollens (1,4)\\
\hline
\end{tabular}\\

\newpage
\subsection*{Exercice 6}
Montrez avec une preuve formelle que la règle d'inférence suivante est valide :
\subsubsection*{Solution}
\begin{tabular}{|l|l|}
\hline
1.  $ p \lor q$ & prémisse \\
2.  $ p \Rightarrow r $ & prémisse \\
3.  $ q \Rightarrow r$ & prémisse \\
\indent 4.  $ \neg(p\lor q \Rightarrow r)$ & supposition \\
\indent 5.  $ \neg (\neg (p \lor q) \lor r)$ & loi de l'implication (4) \\
\indent 6.  $ \neg ((\neg p \land \neg q ) \lor r )$ &  De Morgan (5)\\
\indent 7.  $ (\neg \neg \lor \neg\neg q) \land \neg r$ & De Morgan (6) \\
\indent 8.  $ (p \lor q ) \land \neg r$ & 2x double négation (7) \\
\indent 9.  $ \neg r$ &  simplification (8)\\
\indent 10. $ \neg p$ &  modus tollens (9, 2)\\
\indent 11. $ \neg q$ &  modus tollens (9, 3)\\
\indent 12. $ \neg p \land \neg q$ &  conjonction (10, 11)\\
\indent 13. $ \neg (p \lor q)$ &  conjonction (10, 11)\\
\indent 14. $ p \lor q$ &  simplification (8))\\
15. $ \neg \neg (p \lor q \Rightarrow r )$ &  réduction a l'absurde (4-14)\\
16. $ p \lor q \Rightarrow r $ &  double négation (15)\\
17. $ r $ &  modus ponens (1, 15)\\
\hline
\end{tabular}\\
\newpage
\subsection*{Exercice 7}
Montrez avec une preuve formelle que $(p\Leftrightarrow q)\Leftrightarrow r,p \vdash q \Leftrightarrow r$
\subsubsection*{Solution}
\begin{tabular}{|l|l|}
\hline
1.  $(p\Leftrightarrow q)\Leftrightarrow r$ & prémisse \\
2.  $ p $ & prémisse \\
\indent 3.  $ \neg (q \Rightarrow r)$ & supposition \\
\indent 4.  $ \neg (\neg q \lor r)$ & loi de l'implication (3) \\
\indent 5.  $ q \land \neg r$ & De Morgan + double négation (4) \\
\indent 6.  $ \neg r$ & simplification (5) \\
\indent 7.  $ \neg (p \Leftrightarrow q)$ & exercice 5 (1, 6) \\
\indent 8.  $ \neg (p \Rightarrow q) \land \neg (q \Rightarrow p)$ & De Morgan + loi de l'équivalence (7) \\
\indent \indent 9.  $ \neg (p \Rightarrow q) $ &supposition  \\
\indent \indent 10.  $ p \land \neg q$ & loi de l'implication + De Morgan + double négation (9) \\
\indent \indent 11.  $ \neg q$ & simplification (10) \\
\indent \indent 12.  $ q$ & simplification (5) \\
\indent 13.  $ p \Rightarrow q$ & réduction a l'absurde + double négation (9-12) \\
\indent 14.  $ \neg (q \Rightarrow p$ & syllogisme disjoint (8, 13) \\
\indent 15.  $ q \land \neg p$ & loi de l'implication + De Morgan + double négation (14) \\
\indent 16.  $ \neg p$ & simplification (15) \\
17.  $ q \Rightarrow r$ & réduction a l'absurde + double négation (3-16) \\
\indent 18.  $ \neg(r \Rightarrow q)$ & supposition \\
\indent 19.  $ r \land \neg q$ & loi de l'implication + De Morgan + double négation (18) \\
\indent 20.  $ r$ & simplification (19) \\
\indent 21.  $ p  \Leftrightarrow q$ & exercice 4 (20, 1) \\
\indent 22.  $ q $ & exercice 4 (21, 2) \\
\indent 23.  $ \neg q$ & simplification (19) \\
24.  $ r \Rightarrow q$ & réduction a l'absurde + double négation (18-22) \\
25.  $ (q \Rightarrow r) \land (r \Rightarrow q)$ & conjonction (17, 24) \\
26.  $ q \Leftrightarrow r $ & loi de l'équivalence (25) \\
\hline
\end{tabular}\\
