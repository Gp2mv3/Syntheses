\section{TP 3}

\paragraph{Note :} Ceci est une retranscription de la correction mise par erreur sur Moodle par l'assistant.

\subsection*{Exercice 1}

Démontrez avec un preuve formelle que $P \Rightarrow (Q \Rightarrow P)$ est une tautologie.

\subsubsection*{Solution}

\begin{tabular}{|l|l|}
	\hline
	\indent 1. $P$ & supposition \\
    \indent \indent 2. $Q$& supposition\\
    \indent \indent 3. $P$ & (1)\\
    \indent 4. $P\Rightarrow Q$& thm déduction (2-3)\\
    \indent 5. $Q\Rightarrow (P\Rightarrow Q)$& thm déduction (1-4)\\
    \hline
\end{tabular}


\subsection*{Exercice 2}
Une formule propositionelle est en \textit{forme normale disjonctive} si elle est composée de disjonctions de conjonctions de literaux, cet-`a-dire, elle est de la forme

$$\bigvee_{i=1}^{n} \bigwedge_{j=1}^{m} L_{ij}$$

où $L_{ij}$ sont des literaux.
\begin{enumerate}
	\item Quel est l’avantage d’une formule en forme normale disjonctive ?
	\item Comment peut-on faire pour automatiser la transformation d’une table de vérité en une formule qui possède la même table de vérité ?
	\item Est-ce que toute formule propositionelle peut s’écrire en forme normale disjonctive ?
\end{enumerate}


\subsubsection*{Solution}

\begin{enumerate}
	\item On peut facilement trouver un modèle, il suffit de satisfaire une des conjonctions.
	\item Pour chaque ligne qui rends la formule vraie on construit une conjonction avec un literal positive pour chaque T et un negatif pour chaque F. Après on prends la disjonction de tous. 
	
	Exemple :
	

	\begin{tabular}{cc|ccc}
		$p$ & $q$ & &\\
		\cline{1-3}
		T&T&T&$\rightarrow$& $(p \land q)$\\
		T&F&F&&\\
		F&T&T&$\rightarrow$ & $(\neg p \land q)$\\
		F&F&F&&\\
	\end{tabular}
	 
	$$ (p \land q) \lor (\neg p \land q)$$
	\item Oui par le point 2.
\end{enumerate}

\subsection*{Exercice 3}
Mettez les formules suivantes en forme normale conjonctive.
\begin{enumerate}
	\item $ p \oplus q$
	\item $ (p \Rightarrow q) \Rightarrow r  $
	\item $ p \Rightarrow (q \Rightarrow r)  $
	\item $ (p \land q \land \neg s)\lor (\neg p \land q \land s) $
	\item $ (p\Leftrightarrow q) \Leftrightarrow r $ 
	\item $ (a_1 \land b_1 ) \lor (a_2 \land b_2 ) \lor ... \lor (a_n \land b_n )  $
	
\end{enumerate}

\subsubsection*{Solution}
\begin{enumerate}
	\item \hspace{1em}
	\begin{center}
		\begin{tabular}{cc|c|cccc}
		$p$ & $q$ & $p \oplus q$ & $((p\lor q)$ &$\land$&$\neg$&$(p \land q))$\\
		\hline
		T&T&F&T&F&F&T\\
		T&F&T&T&T&T&F\\
		F&T&T&T&T&T&F\\
		F&F&F&T&F&T&F\\
		\end{tabular}
	\end{center}
$$p \oplus q  \Lleftarrow\!\!\!\!\Rrightarrow (p\lor q)\land\neg (p \land q) \Lleftarrow\!\!\!\!\Rrightarrow (p\lor q)\land (\neg p\lor \neg q)$$
	
	\item \hspace{1em}
	
$$ (p \Rightarrow q) \Rightarrow r \Lleftarrow\!\!\!\!\Rrightarrow \neg(p \Rightarrow q) \lor r \Lleftarrow\!\!\!\!\Rrightarrow \neg (\neg p \lor q ) \lor r \Lleftarrow\!\!\!\!\Rrightarrow (p \land \neg q) \lor r \Lleftarrow\!\!\!\!\Rrightarrow (p \lor r) \land (\neg q \lor r)$$ 
	\item \hspace{1em}
	
$$ p \Rightarrow (q \Rightarrow r) \Lleftarrow\!\!\!\!\Rrightarrow \neg p \lor (q \Rightarrow r) \Lleftarrow\!\!\!\!\Rrightarrow \neg p \lor (\neg q \lor r) \Lleftarrow\!\!\!\!\Rrightarrow \neg p (\neg q \lor r) $$ 
	\item \hspace{1em}
	
\begin{flushright}
$(a + b + c) * (e + f + g) = ae + af + ag + be + bf + bg + ce + cf + cg$\\
$(p \land q \land \neg s)\lor (\neg p \land q \land s) \Lleftarrow\!\!\!\!\Rrightarrow$\\
$(p \lor \neg p) \land (p \lor q) \land (p \lor s) \land$\\
$(q \lor \neg p) \land (q \lor q) \land (q \lor s) \land$\\
$\neg s \lor \neg p) \land (\neg s \lor q) \land (\neg s \lor s) \Lleftarrow\!\!\!\!\Rrightarrow$\\
$(p\lor q) \land ( p\lor s) \land (q\lor \neg p)\land q \land (q\lor s)\land (\neg s\lor \neg p ) \land (\neg s \lor q)$\\

\end{flushright}
	\item \hspace{1em}
	

	\item \hspace{1em}
	

\end{enumerate}

\subsection*{Exercice 4}
Montrez que la règle d'inférence suivante est valide :

\subsubsection*{Solution}

\subsection*{Exercice 5}
Montrez que la règle d'inférence suivante est valide :


\subsubsection*{Solution}

\subsection*{Exercice 6}
Montrez avec une preuve formelle que la règle d'inférence suivante est valide :

\subsubsection*{Solution}

\subsection*{Exercice 7}
Montrez avec une preuve formelle que $(p\Leftrightarrow q)\Leftrightarrow r,p \vdash q \Leftrightarrow r$

\subsubsection*{Solution}

