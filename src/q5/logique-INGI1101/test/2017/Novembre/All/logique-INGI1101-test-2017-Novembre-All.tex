\documentclass[fr]{../../../../../../epltest}

\hypertitle{Logique et Stuctures Discrètes}{5}{INGI}{1101}{2017}{Novembre}{All}
{Arthur van Stratum}
{Peter Van Roy}

\section{Schéma de preuve en logique des propositions}
Définissez les deux schémas de preuve en logique des propositions~:
\begin{enumerate}
	\item Preuve conditionelle
	\item Preuve par contradiction
\end{enumerate}
Expliquez chaque schéma en détail.

\nosolution

\section{Preuve par résolution en logique des propositions}
Supposez les prémisses suivantes:
\begin{align*}
	&P \\
	&P \Rightarrow Q \\
	&(P \Rightarrow Q) \Rightarrow (Q \Rightarrow R)
\end{align*}
Prouvez le théorème R en faisant l'exécution à la main de l'algorithme de preuve pour la logique des propositions. Il fut d'abord convertir les prémisses en Forme Normale Conjonctive. Ensuite il faut utiliser la Règle de Résolution jusqu'a trouver une preuve.

\nosolution

\end{document}
