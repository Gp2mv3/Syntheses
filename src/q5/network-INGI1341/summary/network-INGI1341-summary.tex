\documentclass[en,license=none]{../../../eplsummary}

\usepackage{multirow}
\usepackage{multicol}
\usepackage{graphicx}
\usepackage{listings}
\usepackage{../../../eplunits}
% Footnote in tabular
\usepackage{footnote}
\makesavenoteenv{tabular}
\makesavenoteenv{table}

\hypertitle{Computer Networks: Information transfer}{5}{INGI}{1341}
{Nicolas Houtain\and Benoît Legat\and Gorby Nicolas Ndonda Kabasele\and Gilles
Peiffer \and Florian Thuin}
{Olivier Bonaventure}

\graphicspath{{images/}}
%\lsinputpath{algorithm/}
\lstset{inputpath=algorithm}

\paragraph{French}
There is some part in french because this is a merge between 2 summaries.
It needs to be translated.

\section{Terminology}

\begin{description}
    \item[ADSL]: \textit{Asymmetric Digital Subscriber Line}. the working
    principle here is to have high download speeds but low upload speeds on the
    consumer's end
    \item[ARP]: \textit{Address Resolution Protocol}.
    \item[BPDU]: \textit{Bridge Protocol Data Unit}.
    \item[CRC]: \textit{Cyclical Redundancy Check}.
    \item[DIFS]: \textit{DCF Interframe Space}. If a station detects the
    medium has been continuously idle for DIFS duration, it is then permitted
    to transmit a frame. DIFS duration can be calculated by the following
    method: DIFS = SIFS + (2 * Slot time).
    \item[DNS]: \textit{Domain Name System}. Naming system for associating
    various information (such as an IP address) with a domain name.
    \item[EIFS]: \textit{Extended Interframe Space}. Similar to DIFS but is
    only activated if the last frame contains an error. EIFS = Transmission
    time of Ack frame at lowest phy mandatory rate + SIFS + DIFS.
    \item[HTTP]: \textit{HyperText Transfer Protocol}.
    \item[IEEE]: \textit{Institute of Electrical and Electronic Engineers}.
    \item[LAN]: \textit{Local Area Network}.
    \item[MSL]: \textit{Maximum Segment Life}.
    \item[MTU]: \textit{Maximum Transmission Unit} Maximum size of a protocol
    data unit that can be communicated in a single network layer transaction.
    \item[Path MTU]: Maximum transmission unit size on the network between two
    Internet Protocol hosts.
    \item[PDU]: \textit{Protocol data unit}. Information that is transmitted
    as a single unit among peer entities of a computer network.
    \item[RIFS]: \textit{Reduced Interframe Space}.
    \item[SDU]: \textit{Service Data Unit}.
    \item[SIFS]: \textit{Short Interframe Spacing}. Amount of time in
    microseconds required for a wireless interface to process a received frame
    and to respond with a response frame.
    \item[TLV]: \textit{Type-length-value}. Encoding scheme used for optional
    information element in a certain protocol. The first byte is a type, the
    second byte is the size of the value field and the rest is the value field.
\end{description}

\section{Layers}

The different layers are represented by \tabref{layers}.
A router only has 3 layers: Network, Datalink and Physical. The TCP/IP model
combines the Datalink and Physical layer into a Link layer.
\begin{table}[!ht]
  \centering
  \begin{tabular}{|c|c|c|p{4cm}|c|c|}
    \hline
    \multicolumn{3}{|c|}{Layers} & Protocols & PDU\footnote{Protocol Data Unit}
    & Id\footnote{I know it is a bit of an overgeneralization\dots}\\
    \cline{1-3}
    CNP3 & TCP/IP & OSI & & & \\
    \hline
    \multirow{3}{*}{Application} & \multirow{3}{*}{Application}             &
    Application  & BGP, DHCP, DNS, FTP, HTTP, NFS, NTP, RIP, SMTP, SNMP, Telnet
    & ADU & \\
    \cline{3-6}
                                 &                                          &
                                 Presentation & MIME, XDR, SSL & & \\
    \cline{3-6}
                                 &                                          &
                                 Session      & RTP, TLS & SDU & \\
    \hline
    Transport                    & Transport                                &
    Transport    & UDP, MPTCP, TCP, SCTP & segment & Port\\
    \hline
    Network                      & Internet                                 &
    Network      & ICMP, IPsec, IPv4, IPv6, IPX, MPLS\footnote{MultiProtocol
    Label Switching operates at a layer that is generally considered to lie
    between traditional definitions of layer 2 (data link layer) and layer 3
    (network layer), and thus is often referred to as a ``layer 2.5'' protocol}
    & packet & IP\\
    \hline
    Datalink                     & \multirow{2}{*}{Link}                    &
    Datalink     & IEEE 802.3 (Ethernet), IEEE 802.4, IEEE 802.5, ATM, ARP,
    Frame Relay, HDLC, IS-IS, LAPB, LLC, MAC, OSPF, PPP, SLIP & frame & MAC\\
    \cline{1-1}
    \cline{3-6}
    Physical                     &                                          &
    Physical     & DSL, IEEE 802.3 (Ethernet), IEEE 802.11 (WiFi), ISDN, Modems
    & bit & \\
    \hline
  \end{tabular}
  \caption{This table contains all the protocols of
  \cite{bonaventure2011computer}. See \cite{wiki:osimodel} for more protocols.}
  \label{tab:layers}
\end{table}

\begin{table}[!ht]
\centering
	\begin{tabular}{|c|l|l|}
	\hline
	Layer&Feature&Protocol \\
	\hline
    \multirow{2}{*}{Application}&&BGP, DHCP, DNS,\\
                                &&HTTP, FTP\\
	\hline
	\multirow{2}{*}{Transport}&Connectionless unreliable (CRC)&UDP\\
                              &connection-oriented reliable (graceful)&TCP,
                              STCP\\
	\hline
	\multirow{2}{*}{Network}&Connectionless unreliable(datagram)&IPv6, IPv4,
     OSPF, RIP\\
                            &connection-oriented(virtual circuit)&\\
	\hline
	\multirow{2}{*}{Datalink}&Unreliable (WiFi), uses window-based
     protocol&WiFi\\
                             &Reliable(Wires)&Ethernet, SLIP, PPP\\
	\hline
	Physical&Unreliable (add/delete/change bit)&\\
	\hline
	\end{tabular}
\end{table}

\begin{myexem}
  An HTTPS request will use the following protocols in each layer of the OSI
  model:
  \begin{description}
    \item[Application] HTTP,
    \item[Presentation] SSL,
    \item[Session] TLS,
    \item[Transport] TCP,
    \item[Network] IPv6,
    \item[Datalink] IEEE 802.3 (Ethernet),
    \item[Physical] IEEE 802.3 or IEEE 802.11 (Wireless Local Area Network).
  \end{description}
\end{myexem}

\subsection{Physical Layer}
The Physical Layer service is provided by
\begin{itemize}
  \item \textbf{Electrical cable} twisted pairs or coaxial cables;
  \item \textbf{Optical fiber} multimode or monomode;
  \item \textbf{Wireless} laser for point-to-point and radio-based for spread
  signal (e.g. WiFi).
\end{itemize}
Its PDU is the bit and the following terms are used
\begin{itemize}
    \item \textbf{Bit rate}: Expressed in bits/sec;
    \item \textbf{Bandwith}: Range of frequency usable.
\end{itemize}

\subsubsection{Characteristic}

It is \textit{not perfect} (unreliable),
and to us it is like a black box with these characteristics:
\begin{itemize}
    \item \textcolor{red}{changes} the value of a bit (\textit{becasue of
    electromagnetic interference});
    \item delivers \textcolor{red}{more or fewer} bits than requested
    (\textit{because of an imprecise clock frequency}).
\end{itemize}

\paragraph{Manchester encoding}
It's an encoding that consists in dividing time in fixed length periods. To
send a 1, the voltage must be high in the first half of a period and then
become low. For zero, it's the opposite. It uses the InvH (high voltage
during last period) and InvB symbols as special markers.

\subsection{Datalink Layer}
The PDU of Datalink Layer service is a \textbf{frame} (\textit{sequence of bits
with a particular syntax or structure}) because we want to share data blocks.
A frame can be separated into 3 parts:
\begin{itemize}
    \item \textbf{Header} It contains a flag that tells whether it's an
    \textcolor{red}{ACK or DATA}, a \textcolor{red}{sequence number} and
    sometimes the length of the payload.
  \item \textbf{Payload} It contains the information that needs to be
  transmitted.
  \item \textbf{Error-detecting code} It allows the receiver to detect
  transmission errors.
    It is either
    \begin{itemize}
        \item a \textsc{hamming code} which is simply a parity bit, it can only
        detect an odd number of errors;
        \item a \textsc{checksum} such as the Internet checksum chosen by the
        TCP/IP community and the Fletcher checksum chosen by the OSI community;
        \item or a \textsc{Cyclic Redundancy Check} (CRC).
        It was slow to implement in software before 1995 and the publication of
        \cite{feldmeier1995fast}. It is now preferred since it has better error
        detection \cite{stone1998performance}.
        An $n$-bit CRC will detect errors bursts not longer than $n$ bits and
        will detect a fraction ($1-2^{-n}$) of all longer error bursts.
        \item a \textsc{hash function} like MD5 or SHA: however, these are
        built to be collision resistant against an active adversary, not random
        modifications. They are also a lot slower than CRC or checksums so they
        are only used in cryptography.
    \end{itemize}
  \item \textbf{Error-correcting code} It allows the receiver to correct
  transmission errors. No widely used datalink protocol uses this.
\end{itemize}

\subsubsection{Framing}

The separation of frames is done using \emph{bit stuffing}
or \emph{character stuffing}.

\begin{description}
    \item[Bit stuffing]: \textbf{01111110} is a frame \textcolor{red}{boundary
    marker}, so it can't be used inside the transmitted frames. \textit{Adds a
    0 after 5 consecutive 1's to ensure that this marker is not in the frame}.
    \begin{enumerate}
      \item Easy to implement in hardware.
      \item Increases the number of bits transmitted.
    \end{enumerate}
\item[Character stuffing]: In software it's easier to work with characters. We
add a DLE in front of each DLE contained in the data of the frame. (The
character used as a marker is not printable.) Beginning of frame: \textsc{DLE
STX}, end of frame: \textsc{DLE ETX}.
\end{description}

\paragraph{Note}: Bit stuffing is implemented in hardware and character
stuffing is usually implemented in software.

\subsubsection{Recovering from failures}
We can have errors due to different events:
\begin{enumerate}
  \item The frame has been \textbf{lost} or has been \textbf{corrupted}
    by a transmission error.
  \item When we are to slow to treat incoming frames (buffer \textbf{overflow}).
\end{enumerate}

\paragraph{ }
Thanks to the ACK flag in the header,
we have 2 types of frames: \textbf{data frames} and \textbf{acknowledgment
frames}.
Using the \textbf{Error-detecting code} (ex:parity bit), we can try to recover
from failures of the physical layers and provide a reliable service.

\paragraph{Note:}
Since the \textbf{Physical Layer does not reorder} the bits, the frames will
not be reordered either.
Providing a reliable service is therefore easier than for the Transport Layer
which has to cope with the reordering of packets in the Network Layer (it has to
discard packets to old packets and have maximal throughput because of that).

\paragraph{Pipelining} This technique allows a sender to transmit several
consecutive frames without being forced to wait for an ack after each frame.

\subsubsection{Reliable datalink layer}
There are 3 ways of achieving a reliable Datalink Layer.
\begin{enumerate}
     \item \textbf{ABP}:
     The Alternating bit protocol is a particular case of Go-Back-N for $n = 2$
     (only one bit for the sequence number).
     \item \textbf{Go-Back-N}: is simple, the receiver discards all of sequence
     frames and the ACK always contains the last in-sequence frame received. It
     uses three variables, \textit{lastack, next, maxseq}

     The sender simply has one timer and when it expires, it retransmits
     \emph{all} its unacked frames.

     \begin{center}
          \textit{Good performance if few frames are lost but otherwise the
          performance drops quickly because out-of-sequence frames aren't
          accepted and all unacked frames are retransmitted when a loss is
          detected.}
     \end{center}

  \item \textbf{Selective Repeat}
      The difference with the Go-Back-N is that the receiver \textbf{stores
      received out-of-sequence} frames, even if cumulative acknowledgment is
      still used. (\textit{ACK still contains the last in-sequence frame
      received even if an out-of-sequence frame is stored in buffer}).

     The sender now has a timer  for each frame  of the sending window. An ACK
     covers every frame in the buffer up to the acknowledged sequence number.

    \begin{itemize}
        \item[$\to$] The ACK sometimes also contains the list of
        out-of-sequence received frames (\emph{selective acknowledgment}) to
        avoid useless retransmission.
    \end{itemize}
\end{enumerate}

\paragraph{Buffer size}
If the sequence number has $n$ bits, we use $2^n$ \textbf{different sequence}
numbers.

A \textit{Sliding window} is used to manage the buffer of sequence numbers.

Because the Physical Layer will not reorder the frames,
the maximum window size for Go-Back-N is $\bf 2^n-1$ and $\bf 2^{n-1}$ for
Selective Repeat. (Think when all the acks are lost.)

\paragraph{  }  Not  all  Datalink Services  provide  a  \emph{reliable}
service. As a rule of thumb, datalink services above very unreliable
wireless physical services (e.g. WiFi) do provide a reliable service and
datalink services above almost reliable wired physical services (e.g. cable or
fiber) do not include additional retransmission mechanisms and are also
\emph{almost reliable}.

\paragraph{Piggybacking}
When DATA is sent in both directions, an ACK frame and a DATA frame sent by one
side are sometimes merged into one because an ACK frame does not need a lot of
bits to do its job. (\textit{Used to reduce the overhead caused by
acknowledgments}.)

\subsection{Building a network}
The network layer is used to send packets between two hosts that cannot
directly be relied by a cable. Host and router \textbf{send packets}.

There are two possible organisations for the network layer.

\begin{itemize}
    \item \textbf{Data plane:} Protocol and algorithm used to forward data.
    \item \textbf{Control plane:} Protocol and algorithm used to make
    forwarding table.
\end{itemize}

\subsubsection{The datagram organisation}

\textit{Inspired by the postal service}, each host is identified by a
\textbf{network layer address}.

For each \textsc{packet}, sender must define their address, the receiver's
address and data.

\paragraph{Forwarding}
In datagram organisation, router uses \textbf{hop-by-hop} forwarding.
(\textit{Each router forwards the packet with its forwarding table}.)

In a network, \textbf{black-holes} (\textit{when routers discard packets
because there is no entry in their forwarding table for this destination}) and
\textbf{cycles} ({packet consumes bandwidth}) must be avoided!

\paragraph{Computing forwarding tables}

\begin{description}
    \item[Port-address table]: If we have a \textbf{tree-shaped network}
    (\textit{drawback}), we only need to inspect the received packets to create
    our forwarding table, without risking making a loop.
     \begin{enumerate}
          \item If the destination address is in the forwarding table, the
          packet is forwarded on the right interface.
          \item Else, the packet is sent on all interfaces except the interface
          from which the packet was received. It's called \textbf{broadcasting}
          (\textit{drawback}).
     \end{enumerate}

     $\to$ The problem with a tree-shaped network is that if a link fails, the
     network is \textit{split} into two network since there is no redundancy.

     \begin{center}
     \textit{If the network is not a tree, we can also use a port-address table
     but we need to use a distributed algorithm to ensure that we have a tree
     (e.g. Spanning Tree Protocol for Ethernet (Datalink Layer)).}
     \end{center}

    \item[Source routing]: There is no destination address but only the path
    to reach the destination host. Two types of packets:
     \begin{enumerate}
          \item Data packet: to exchange data.
          \item Control packet: to discover the path between endhosts.
          When a router receives a control packet it forwards this one via
          all interfaces.

          Avoids possible loops because the control packet contains a list of
          \textbf{intermediate nodes}.
     \end{enumerate}

     Complexity is placed on the endhost and network node is simplest.

\end{description}

\paragraph{Flat or hierachical address}
Flat is like a telephone number (small match to forward packet) and hierachical
is like a postal address (smaller forwarding table but address changes when
attached to another node\ldots\ Issues with mobile host).

\paragraph{Dealing with heterogeneous datalink}

\begin{enumerate}
     \item \textbf{Retransmitting}: Discard packet and send a control packet to
     the source to indicate that it cannot forward packets longer than
     \SI{500}{\byte}. Source retransmitts the information in smaller packets.
     \begin{center}
       \textit{Router can be really simple and no additional operations to
       perform, but may be inefficient because of retransmitting.}
     \end{center}
     \item \textbf{Fragmenting}: The router can fragment packets. There are
     two ways to achieve this:
     \begin{enumerate}
          \item The next router reassembles the fragment
          \begin{center}
               \textit{Takes so much CPU time and memory to fragment and
               reassemble again.}
          \end{center}
          \item The endhost reassembles the fragment
          \begin{center}
               \textit{Compromise between the two others.}
          \end{center}
     \end{enumerate}
\end{enumerate}

\begin{figure}[ht]
    \centering
    \includegraphics[width=12cm]{heterogeneous.png}
    \caption{Example of heterogeneous datalink layer.}
\end{figure}

\subsubsection{Virtual circuit organisation}
Each host has an address. Packet forwarding is not done by looking in the
packet destination but by checking the label (an \texttt{int}). Enables control
over the path used.

\paragraph{Label switching} Each node has a label forwarding table. Upon
reception of a packet, it forwards the packet in the direction mapping with the
label of the packet. It changes the label of the packet. (Lookup in
$\bigoh(1)$).

\subsubsection{The control plane}
There are two main techniques that can be used to maintain the forwarding table
in a network: distance vector routing and link state routing.

\subsubsection{Distance vector routing}
Allows a router to discover the destinations reachable inside the network as
well as the shortest path to reach each of these destinations. (\textit{Each
link has a associated cost}.)
\paragraph{Routing table contains}:
\begin{itemize}
    \item \texttt{R[d].link}: outgoing link used to forward packet to
    \texttt{d};
    \item \texttt{R[d].cost}: cost of shortest path to \texttt{d};
    \item \texttt{R[d].time}: timestamp of the last distance vector containing
    destination \texttt{d}.
\end{itemize}

When it boots, it sends a distance vector with its address at a distance 0.

\paragraph{Routing table update}
The router sends its \textbf{distance vector} regularly over every interface.
If a router receives a distance vector from link l, it only updates
the corresponding entry if either:
\begin{itemize}
    \item $\texttt{V[d].cost} + \texttt{l.cost} < \texttt{R[d].cost}$: new
    route smaller thant route already known;
    \item[OR]
    \item \texttt{R[d].link == l}: Update of the same route.
\end{itemize}

All routers send their distance vector every $N$ seconds. After $3N$ seconds,
the distance vector cost is set to $+\infty$ if there were no updates.

\subparagraph{Distance vector packet}
The distance vector to a neighbour isn't broadcasted (well, it is implicitly
when we send our own distance vector after having updated our forwarding table).

\paragraph{Count to infinity} If two routers send distance vectors at the same
time, the cost keeps increasing infinitely. Solution $16=\infty$
\paragraph{Split horizon} Do not send information to where you have learned it.
It is done by making specific distance vectors for each neighbor.
\subsubsection{Link state routing}
Exchange messages to allow each router to learn the entire network topology,
and so each router is then able to compute its routing table by using a
shortest path computation\footnote{By Dijkstra.}.

\textbf{Weight} (\textit{Usually symmetric, but it's not an assumption}) of a
link can be:
\begin{enumerate}
    \item Unit weight. (\textit{Shortest path = lowest intermediate routers});
    \item Proportional to the propagation delay;
    \item Inversely proportional to the bandwidth
    ($\frac{C}{\textnormal{bandwidth}}$ where $C$ is higher than the highest
    bandwidth in the network).
\end{enumerate}

When booting, they send HELLO messages to neighbours. HELLO is never fowarded
and it is sent every $N$ seconds. If no HELLO is received for $kN$ seconds, the
link is considered to have failed.

\paragraph{Static vs dynamic metric} Experience has shown that it is difficult
to tune the dynamic adjustments and ensure that no forwarding loops occur in
the network! \textit{Actually}, the link state routing protocol uses metrics
that are configured manually.

\paragraph{Network topology update}
The update is performed by \textbf{link-state packets} that contain:
\begin{itemize}
    \item \texttt{LSP.Router}: identification of the LSP sender;
    \item \texttt{LSP.age}: age (\textit{remaining lifetime}) of LSP;
    \item \texttt{LSP.seq}: sequence number of LSP;
    \item \texttt{LSP.Links[]}: links advertised in the LSP. Each directed
    link is represented by the following information:
    \begin{itemize}
         \item \texttt{LSP.Links[i].Id}: identification of the neighbour;
         \item \texttt{LSP.Links[i].cost}: cost of the link.
    \end{itemize}
\end{itemize}

\begin{figure}[ht]
    \centering
    \includegraphics[width=12cm]{lsp.png}
    \caption{LSP link example.}
\end{figure}

\paragraph{\textbf{Flooding} algorithm}
Is used to efficiently distribute the LSPs of each router.
To implement this, each router maintains a \textit{link state database} (LSDB)
that contains the most recent LSP for each router.

$\to$ Router forwards a LSP only if it is more recent that the actual LSP on
LSDB.

\paragraph{\textbf{Reliable flooding}} Router uses ack (and retransmissions) to
ensure that all link state packets are successfully transferred to all
neighbouring routers.

When a failure link has been detected, the router attached to it sends an LSP
without this link, so all routers update their LSDB.

\paragraph{Two-way connectivity} A link is considered failed when one of the
routers attached to it has detected it.

\subsection{Application}
There are two models to organise a networked application. Client-server and
peer-to-peer. To understand each other, the client and the server need to agree
on a protocol (format of message + organisation of the information flow).
\begin{itemize}
    \item \textbf{Big Endian}: send the most significant byte followed by the
    least significant byte (protocol chosen by the Internet).
    \item \textbf{Little Endian}: send the least significant byte followed by
    the most significant byte.
\end{itemize}

In the peer-to-peer model, the host acts like a client and a server.

\subsection{The network layer}
Most networks use \textbf{datagram organisation} and provide a simple service
which is called the \textcolor{red}{connectionless service}.

\paragraph{ } Virtual organisations have a connection system based on labels.

% Network:
% Datagram: connectionless-> most
% Virtual circuit: connection-> few
%
% Transport
% most on top on datagram so
% UDP: datagraph: connectionless
% TCP: connection oriented
%
%
\subsection{The transport layer}
This layer improves the service provided by the network to make it usable by
applications. We only consider datagram organisation and connectionless
service. It usually supports an unreliable connectionless service. It needs to
manage different issues from the network layer:
\begin{multicols}{2}
\begin{itemize}
    \item corrupt data;
    \item loose data;
    \item data not delivered in-order;
    \item upper bound on maximum length of the data;
    \item duplicate data.
\end{itemize}
\end{multicols}
The main reason for packet loss is the buffer usage (discard if buffer full).
The transport layer sends \textbf{segments} as PDU. Segments contain
\textbf{headers} with some control information and \textbf{payloads} from the
application layer.

\begin{center}
    \textit{When a segment is created, this segment is encapsulated by the
    network layer into a packet which contains the segment as its payload and a
    network header. The packet is then encapsulated in a frame to be
    transmitted in the datalink layer.}
\end{center}

\paragraph{ }
\textbf{Transport layer} service is, for the most part, on top of datagram
organisation and so based on connectionless network layer service.

To go from the network layer to the transport layer, we must add an
\textbf{error management} mechanism and a \textbf{multiplexing} technique.

\subparagraph{ }There are three different services on the transport layer:
connectionless service, connection-oriented service and request-response
service.

\subsubsection{Connectionless service}
Data is sent and the service guarantees it will arrive. Used for small SDU
transport.

\begin{description}
    \item[Reliable]: Guarantees data arrival. (\textit{Hard to implement}.)
    \item[Unreliable]: Imperfection:
        \begin{itemize}
            \item Only guarantees a majority arrives. (\textit{Usually, what's
            left over is due to buffer overflow.})
            \item Can duplicate packets on the network.
            \item Can deliver different SDUs.
            \item Has a limited data size.
        \end{itemize}
\end{description}

This service brings two new features compared to connectionless network layer
service: an error detection mechanism and a multiplexing technique that
permits differenciating applications on the host.

\subsubsection{Connection-oriented}
There are three phases to this: establishing a connection, transferring SDUs
(\textit{the connection is bidirectional}) and closing the connection.

\begin{itemize}
    \item \textbf{Reliability}: Is only guaranteed when the connection
    terminates with ``gracefully'', otherwise packet loss in possible.
\end{itemize}

\paragraph{\textbf{Connection}}

\paragraph{Refus:} Either by the receiver or the sender.
\begin{figure}[ht]
    \centering
    \includegraphics[width=8cm]{refus.png}
    \caption{Refused connection.}
\end{figure}

\subparagraph{Three-way handshake:} used against the naive approach that is back-and-forth connection. It requires three separate steps:
\begin{itemize}
     \item The first host sends the second host a SYN message with its own sequence number $x$, which the second host receives.
     \item The second host replies with a SYN-ACK message with its own sequence number $y$ and acknowledgment number $x+1$, which the first host receives.
     \item The first host replies with an ACK message with acknowledgment number $y+1$, which the second host receives and doesn't reply to.
\end{itemize}
\begin{figure}[ht]
    \centering
    \includegraphics[width=12cm]{threeway.png}
    \caption{Three-way handshake connection establishment.}
\end{figure}

To avoid \textbf{duplicate transport connection}, \textbf{transport clocks}
incremented every \textit{clock cycle} and after each connection establishment
are used. (\textit{Must continue to be incremented even if the transport entity
stops or reboots}.)

Implemented with a $k$ bit counter and its \textbf{clock cycle} is such that
$2^k \times cycle \gg MSL$.

\begin{figure}[ht]
    \centering
    \begin{tabular}{ccc}
    \includegraphics[width=5cm]{three1.png}
    &
    \includegraphics[width=5cm]{three2.png}
    &
    \includegraphics[width=5cm]{three3.png}
    \\
    \multicolumn{3}{c}{$\color{red} \bullet$ when a segment
    is lost and $\color{yellow} \bullet$ when it comes from another.}
\end{tabular}
\caption{Scenario where a transport clock is useful.}
\end{figure}

\paragraph{\textbf{Data transfer}}
\begin{itemize}
    \item \textsc{message-mode}: Messages are sent and received as is (rarely
    used).
    \item \textsc{stream-mode}: Here, byte fluxes are sent, and a delimiter
    needs to be specified to limit SDUs in the bytestream. (The provider
    assures that bytes arrive in the correct order.)
\end{itemize}

$\implies$  The \textbf{Stream-mode} is used for reliable transport protocols,
and the sequence number placed in the frame corresponds to the position of the
last payload byte in the bytestream.

\paragraph{Note:} Multiple differences compared to the \textbf{datalink layer}
to assure data is delivered.

\begin{enumerate}
    \item Compared to the datalink layer where the transmission delay is fixed
    when two hosts are connected, it is here \textbf{variable}.
    (\textit{Because the sent packets dont necessarily take the same path and
    can therefore be forced to wait by the router buffer.})
    \item Packets can arrive \textbf{out of sequence}, as opposed to the
    datalink layer. The network can duplicate packets in the transport layer.
    \item Transmission of \textbf{large SDUs}, larger than the maximum size of
    a packet in the network.
\end{enumerate}

\subparagraph{Solution}:
\begin{enumerate}
    \item To detect transmission errors, as is the case in the datalink layer,
    a CRC or checksum is used on \textbf{each} segment.
    \item To make the protocol reliable, a sequence number is used which
    corresponds to \textbf{the position of the first byte of the payload in the
    bytestream}, as well as ACK numbers.

    $\to$ Either 32 or 64 bits are necessary (to detect delays and because the
    numbers are used up faster) rather than 8 bits in the datalink layer
    protocol.
\end{enumerate}

\paragraph{Go-Back-N and selective repeat}
In the transport layer, \textbf{selective repeat} is preferred because this
layer doesn't guarantee sequential arrival, as opposed to the datalink layer.

\paragraph{Variable buffer}
In the transport layer, multiple concurrent applications can communicate, and
therefore the memory space accessible to each application can vary, which makes
the buffer size a variable.

The sender has a \textbf{swin} (\textit{size of its buffer}), and \textbf{rwin}
(\textit{size of the receiver's buffer}). It considers the minimum of those as
the window size.

\textbf{To avoid deadlock}, a persistent timer is used when the sender receives
a window size equal to 0. When the persistent timer expires, the last segment
is forcefully resent.

\paragraph{Excessive timer retransmission} can cause ambiguities if
you make the turn of sliding window\ldots\ To solve this issue, the transport
protocol requires the network layer to enforce a \textbf{Maximum Segment
Lifetime}.

\textsc{MSL} limits the maximun bandwidth of a transport because if it uses $n$
bits for sequence number, then it cannot send more than $2^n$ segments every
MSL.

\paragraph{\textbf{Connection release}}
Either \textbf{abruptly} or \textbf{gracefully}, that is, by closing the
connection both ways with a DR followed by an ACK.

\subsubsection{Request-response service}
It's a compromise between connectionless service and connection-oriented
service. It is used when a host needs to execute a procedure on another host
(\textbf{Remote Call Procedure}). Sends a small SDU but ensures a reliable
delivery.

\subsection{Naming and addressing}
The network and transport layer rely on addresses that are encoded as fixed
size bit strings. (\textit{For human-friendly use.}). At first, there was a
file (\texttt{hosts.txt}) that mapped the name of the host to their IP address.
Then, it was decided to organise the name as a tree structure. Each of the
top-level domains is managed by an organisation that decides how sub-domain
names can be registered.
\begin{figure}[ht]
    \centering
    \includegraphics[width=6cm]{internet.png}
    \caption{Internet tree structure.}
\end{figure}
This is a key component for the Domain Name Server (distributed database which
makes the mapping between name and IP address). The name server responsible for
a domain can answer 2 queries:
\begin{itemize}
	\item The IP address of any host residing inside its domain;
	\item Name server(s) that are responsible for any direct sub-domain of the
     domain.
\end{itemize}
\paragraph{Root name server} Name server that is responsible for the root of
the domain name hierarchy. A dozen of them exist and are synchronized.
\paragraph{DNS resolver} To avoid the client having to maintain a list of root
server IPs, only the resolver keeps them.
Clients contact local DNS resolvers if needed.
\subsubsection{Benefit}
Using names instead of addresses allows to:
\begin{itemize}
	\item Keep the same identifier (name) even if the server processes move to
     another physical layer. No need to inform the client about the new server
     IP address.
	\item If there are many concurrent clients on a server, a server can be
     added to reduce load. (Mapping a name to a set of addresses.)
\end{itemize}

\subsection{Sharing ressources}
\begin{itemize}
    \item \textbf{Link bandwidth} is the most important resource in a network.
    \item The \textbf{buffer} of the network node is another important resource.
    \item \textbf{Processing capacity} of nodes to analyse packets and see on
    forwarding table.
\end{itemize}

\subsubsection{Organisation to share bandwidth}
\begin{figure}[ht]
    \centering
    \begin{tabular}{ccccc}
    \includegraphics[width=2cm]{fullmesh.png} &
    \includegraphics[width=2cm]{bus.png} &
    \includegraphics[width=2cm]{star.png} &
    \includegraphics[width=2cm]{circle.png} &
    \includegraphics[width=2cm]{tree.png} \\
    Full mesh & Bus & Star & Ring & Tree
\end{tabular}
    \caption{Different organisations.}
\end{figure}

\begin{itemize}
    \item \textbf{A full mesh}
    The most efficient, but needs $\frac{n \times (n-1)}{2}$ links, and every
    host has to manage $n-1$ interfaces which quickly becomes impossible.

    \item \textbf{Bus organisation}
    The danger is that if the bus' cable is ruptured, the network is divided
    which can be hard to maintain.

    \item \textbf{Star organisation}
    The node at the center of the star is vital for the network, but it permits
    centralisation of control in a single point. (\textit{It's an excellent
    control point and a good observation point.})

    It's a lot easier to maintain than a bus organisation.

    \item \textbf{Ring organisation}
    A link shuts down the entire network, which is why a dual-ring is often
    used.

    \item \textbf{Tree organisation}
    Allows for connecting a large amount of clients with very low cost.
\end{itemize}

\subsubsection{Sharing bandwidth}

Bandwidth sharing in networks is \textbf{max-min fair}. (Can't allocate more
bandwidth to a flow without reducing the bandwidth of another).

Multiple algorithms are explained in the Medium Access Control section,
\sectionref{medium_access_control}.

\subsubsection{Network congestion}

\textbf{Congestion} when $\sum \textnormal{demand} > \textnormal{capacity}$.

\paragraph{Congestion collapse}
Arrives when the network is slightly congested, and thus transmission is slowed
down. If a protocol such as \textbf{selective repeat} is used, the sender might
thing the packet has been loss and therefore send back a packet which only adds
onto the congestion.

To regulate congestion, one solution is to know the actual congestion so that
the hosts can adjust their accessible bandwidth to reduce congestion.

\subparagraph{$\to$} As long as the buffer isn't full, demand is less than
capacity and therefore the buffer just serves \textbf{to smooth} demands over
time, otherwise the demand is going to be higher than the available bandwidth,
which in turn leads to congestion.


\paragraph{Discard mechanism}
Packets can be discarded when the buffer is full, or rather when its size goes
up dramatically in order to prevent congestion. However, discarding a packet is
the last resort.

\begin{enumerate}
    \item \textbf{Discard the ingoing (tail drop)}: the most common one, but
    it tends to add on to the existing congestion and real-time applications
    suffer from this.
    \item \textbf{Discard the next outgoing (drop from front)}: smarter than
    it might seem since this packet has been there for a long time and surely
    the loss (due to congestion) has been detected beforehand.
    \item \textbf{Random early discard}: discards a random packet.
    (\textit{Allows for discarding packets from different flows proportional to
    their bandwidth.})
\end{enumerate}

\subparagraph{Remark:} Discarding a packet isn't the smartest solution since
it boils down to discarding a packet which has used up resources, when
resources are exactly what is missing in the first place.

\paragraph{Forward Explicit Congestion Notification}
For datagram networks, a bit is used to show whether the packet has passed
through a congested area, and if it has, the receiver sends a packet to inform
the sender of the degree of congestion (which is equal to
$\frac{n_{\textnormal{congested}}}{n_{\textnormal{total}}}$).

\paragraph{Backward Explicit Congestion Notification}
Identical technique for virtual networks, except for the fact that
acknowledgment is shown.

\paragraph{Control packet}
Allow the network node to send a \textbf{control packet} to the source to
indicate the current congestion level. Their usage is mainly restricted to
small networks because:
\begin{enumerate}
    \item In large networks this packet increases the network load when the
    network is congested.
    \item Network nodes are optimized for forwarding packets, not creating them.
\end{enumerate}

\paragraph{Scheduler mechanism}
Allows to attribute a FIFO queue for each flow and a round-robin chooses the
next packet.


\subsubsection{Distributing the load across the network}

\paragraph{Virtual network}
Here, when a host wants to send out information, they have to specify its
destination and sometimes the necessary bandwidth. They then get a response
telling them whether they can connect. This technique is named
\textbf{connection admission control}. (\textit{An example of usage is in
telephones.})

\paragraph{Datagram network}
The virtual network technique can't be used here because the host doesn't need
autorisation to send out packets. See \sectionref{congestion} for various
techniques for reacting to congestion.

\paragraph{Shared popular file}
\begin{itemize}
    \item \textbf{Save on multi server}: One server name known by the client
    and the client sends a \textsc{query} to know the address of the server.
    (\textit{Automatically, if there is different server the address can change
    for different people.})
    \item \textbf{Using a popular bittorent service}: Split the file into
    different blocks and the client needs all blocks to have the file. There is
    a \textbf{metadata} file that contains where each block can be downloaded.
    (\textit{Most deployments of bittorents allow the clients to participate
     in the distribution of the blocks.})
\end{itemize}

\subsubsection{Medium Access Control (MAC)}
\label{sec:medium_access_control}
The first of the 3 resources that need to be shared inside a network is the
link bandwidth (\textit{the 2 others are router processing time and router
buffers}).

\textbf{Collisions} happen when two hosts try to send a frame simultaneously.
They are the principal source of errors in Local Area Networks.

\paragraph{ }
There are 2 types of MAC:
\begin{itemize}
    \item \textbf{Deterministic} or pessimistic MACs: They ensure that there
    will be \emph{no} collision. They are more appropriate in networks where
    the load is constant (e.g. telephone, radio).
    \item \textbf{Stochastic} or optimistic MACS: They try to minimize the
    number of collisions. They are more appropriate in networks where the load
    has an on-off behaviour (e.g. internet).
\end{itemize}

\paragraph{\textbf{Deterministic MAC}}
The deterministic MACs are the following (the first three are \emph{static}
allocation methods).
\begin{itemize}
     \item[-] \textbf{Frequency Division Multiplexing}:
     In a wireless medium, we can give a different frequency for devices.
     \item[-] \textbf{Wavelength Division Multiplexing}:
     In an optical medium, we can give a different wavelength for devices.
     \item[-] \textbf{Time Division Multiplexing}:
     In any medium, we can just divide the time in separate slots and assign
     the slots to devices statically or dynamically.
     \item[-] \textbf{IEEE 802.4 (Token Bus Network)}:
     Transmit a token in a network with a bus topology. To transmit data, wait
     for the token and then transmit the data instead of the token. Once the
     data is transmitted, put the token back in the network.
     \item[-] \textbf{IEEE 802.5 (Token Ring Network)}:
     Same as 802.4 but with a ring topology.
     \item[-] \textbf{Deterministic Medium Acces Control (DMAC) }: Example
     with the Token Ring
     \begin{itemize}
          \item Nodes have two modes: listen mode where they forward the signal
          they receive, inserting a one bit delay and transmit mode where they
          have the token and they transmit data.
          \item There is a special node called \textbf{monitor} which ensures
          that the token always travels on the token ring (it inserts a token
          size delay).
          \item Nodes stop transmitting when they get their own bit. the
          beginning of the token is the beginning of the data frame so the node
          can know with a bit if it's a token or data.
          \item A station can't retain the token for more than Token Holding
          Time.
     \end{itemize}
\end{itemize}

\paragraph{\textbf{Stochastics MAC}}

\begin{itemize}
    \item[-] \textbf{[slotted] ALOHA}: First approach is ALOHA. When no ack is
    received, we wait a \textcolor{red}{random time} instead of a fixed time to
    \textbf{avoid synchronisation} with other hosts.

    A simple improvement is \textbf{slotted} ALOHA which divides the times into
    slots of the same size as the time required to transmit one frame.
    Transmisssions are only allowed to start at the beginning of a time slot.
    This avoids collision on part of a frame since that requires the frame to
    be completely retransmitted anyway.

\item[-] \textbf{[non-]persistent Carrier Sense Multiple Access (CSMA)}:
An improvement to ALOHA is \textbf{persistent} CSMA which senses the network
before sending a frame but does not wait a random time.

To avoid synchronisation with CSMA, \textbf{non-persistent} CSMA waits a random
time before sensing the network. If it is not free, it will wait a random time
before sensing it again.

\item[More] improvements can be made but depend on the technology.

\item[-] \textbf{CSMA/CD for IEEE 802.3 (Ethernet)}: On ethernet, a host is
able to detect a collision while it is listening.

If $\tau$ is the diameter of the network (the largest time between 2 hosts), we
know that if a frame of length $2\tau$ has a collision, the sending host
\textbf{will sense it} (as any other host by the way). We therefore enforce
$2\tau$ as the minimum frame size.

$\to$ Since Ethernet doesn't have many transmission errors, CSMA/CD does
\textbf{not use acknowledgment} to avoid the collisions they would cause. Also,
because the sender knows when there is a collision, it knows when a
transmission error occurs.

The time waited by the hosts when a collision is detected is random to avoid
synchronisation, is a multiple of $2\tau$ for the same reason as slotted ALOHA
and is selected in a range multiplied by 2 after each collision. This is called
a \emph{binary exponential back-off}.

\item[-] \textbf{CSMA/CA for IEEE 802.11 (WiFi)}

\begin{description}
    \item[SIFS]: This is required to \textbf{switch} between upload and
    download in a router.
    \item[DIFS]: The channel must be idle DIFS microseconds before a device
    can transmit \textbf{after the previous frame was received correctly}.
    \item[EIFS]: the time the device must sense the channel idle \textbf{after
    the previous frame was corrupted} before transmitting.
    \item \[\mathrm{SIFS} < \mathrm{DIFS} < \mathrm{EIFS}.\]
\end{description}

There is a \textbf{random} exponential \textit{backoff timer} in addition to
\textsc{DIFS} or \textsc{EIFS} before sending a frame. (\textit{This timer is
frozen when the channel is busy!})

Another problem with wireless is the \textit{hidden station problem}, where you
don't receive the signal of another device (and therefore can't see when the
channel is busy).

To solve this, there are two control frames: \textsc{RTS} (ask for a delay
reservation) and \textsc{CTS} (to confirm reservation). (\textit{Small size to
minimize collisions.}) All hosts are informed of the reservation.

\begin{figure}[ht]
    \centering
    \includegraphics[width=6cm]{hiddenstation.png}
    \caption{Hidden station problem.}
\end{figure}

\end{itemize}

\subsubsection{Congestion control}
\label{sec:congestion}

To remove \strong{congestion collapse}, the hosts have to regulate their
transmission rate. (\emph{There are other mechanisms to regulate this such as
the one based around credits.})

\paragraph{Goals of congestion control} for a set of $i$ hosts:
\begin{itemize}
    \item Has to remove congestion, that is $\forall t \sum r_i(t) \leq R$.
    \item Efficiency, that is $\forall t \sum r_i(t) \approx R$.
    \item Fairness ($\to$ max-min fairnessis the ideal).
\end{itemize}

Such a mechanism can be implemented in the \textit{transport layer} or in the
the \textit{network layer}. (TCP/IP implements this in transport.)

\paragraph{Congestion control} is an algorithm that adjusts the rate of
congestion:
\begin{enumerate}
    \item \textbf{Multiplicative decreasing} if there is congestion:
    $\textnormal{rate} = \textnormal{rate} \cdot \beta, \quad \beta <1$.
    \item \textbf{Additive increasing} otherwise:  $\textnormal{rate} =
    \textnormal{rate} + \alpha, \quad \alpha>0$.
\end{enumerate}

\paragraph{Congestion control in window-based transport protocol}
Reducing the window size can reduce the congestion because a device cannot
send data faster than: \[\frac{\textnormal{Window}}{\textnormal{rtt}},
\textrm{ where window is current sending window}\]

\subparagraph{Limit window size}
\begin{itemize}
    \item[-] Review: \textit{swin} is the size of the sending window,
    \textit{rwin} is the size of the receiving window and now \textit{cwin} is
    the congestion window.
    \item[-] \textbf{Congestion window}: limits the sending window.
    \textit{swin} = $\min(swin,cwin)$
\end{itemize}

When starting, \textit{cwin} is set to one segment and it increases
(\textit{additive}) by one every RRT when there is no congestion, else it is
divided by two (\textit{multiplicative}).

\paragraph{Note: }
\begin{description}
    \item Congestion is detected when a packet is lost.
    \item Mild congestion: 3 duplicate acks ($\to$ Fast retransmit).
    \item Severe congestion: retransmission timer expires.
\end{description}

\subsection{The reference models}


\begin{table}[ht]
    \begin{tabular}{|c|c|c|c|}
        \hline
        Application & SDU     &                                        &  \\
        \hline
        Transport   & Segment & Connectionless unreliable              &
        connect to network  \\
                    &         & connection-oriented often reliable   & \\
        \hline
        Network     & Packet  & Unreliable                             &
        connect to network \\
        \hline
        Datalink    & Frame   & Reliable (if it uses ack and an error is
        detected) & directly connect to device \\
        \hline
        Physical    & Bit     & Unreliable                             &
        directly connect to device \\
        \hline
    \end{tabular}
    \caption{Reference models.}
\end{table}

\paragraph{TCP/IP reference model} The same but the physical layer and datalink
layer are combined into the link layer.

\paragraph{Model OSI} There is a session and a presentation layer before the
application layer. The session layer deals with the organisation and the
synchronization of the message/data exchanged with the presentation entity.
The presentation layer deals with representing the information.

\section{Protocols}

\subsection{Application layer}
An application uses connectionless or connection-oriented services and is
\textcolor{red}{identified} by a port number and a network address
(\textit{IPv4 or IPv6}). TCP is also called the byte-stream mode service and
UDP the datagram service.

\begin{description}
    \item IPv4: 32 bits wide.
    \item IPv6: 128 bits wide.
\end{description}

\subsection{DNS}

\begin{figure}[ht]
    \centering
    \includegraphics[width=10cm]{dnsheader.png}
    \caption{DNS header.}
\end{figure}
DNS usually runs above datagram service. The message is divided into five
parts. The first three are mandatory, the other two are optional. In the DNS
message header, there is an identifier to match requests to return inforrmation
on server authority (that is, who manages the server). A request is recursive
when the resolver recurses through the DNS hierarchy to retrieve its answer.
\paragraph{Resource Record}
\begin{itemize}
	\item[Time-to-Live] How long the client can keep the Resource Record inside its cache.
	\item[Type] Field in the record to specify either IPv4(A) or IPv6(AAAA).
\end{itemize}


DNS allows for obtaining the address which corresponds to a certain name, but
can also be reversed to obtain the name corresponding to a certain address.

\subsection{Electronic mail}

An \textbf{email system} is composed of:
\begin{itemize}
    \item A message format.
    \item Protocols: to exchange between host and server.
    \item Client software: to create/read mail.
    \item Software: to allow the server to efficiently exchange mail.
\end{itemize}

\begin{figure}[ht]
    \centering
    \includegraphics[width=10cm]{mail.png}
    \caption{Simplified mail architecture.}
\end{figure}

\subsubsection{Email messages}
Email messages are composed of two parts: a \textbf{header} (\textit{From*,
Date*, To*, Subject,  cc, bcc}) and a \textbf{body}. (* indicates mandatory
fields).

There is an empty line to separate header and body which contains only CR and
LF.

\subsubsection{Header field to support Multipurpose Internet Mail Extension
(MIME)}
New header lines support MIME\footnote{To add non-ASCII characters in mail
without breaking the email servers that were deployed at that time.}:
\begin{itemize}
    \item MIME-version: MIME version for ecncoding mail.
    \item Content-type: Datatype of the message:
        \begin{enumerate}
            \item multipart/mixed: the file contains independent parts (text,
            binary separated in the body by empty lines).
            \item multipart/alternative: same message but with different
            representations (e.g. text and HTML).
        \end{enumerate}
    \item Content-Transfer-Encoding: How the message is encoded and a second
    parameter explaining which string is used to delimit fields.
\end{itemize}
 A frequent encoding for email is \textbf{Base64} where sequences of bytes are
 encoded in groups of three bytes adb then each group is divided into six-bit
 fields that correspond to a character. (`=' is reserved for padding).

\subsubsection{The Simple Mail Transfer Protocol (SMTP)}

SMTP is a client/server protcolol and text-based protocol relies on the
connection-oriented service. (\textit{Server listens on port 25})
\paragraph{Note:} The DNS MX record of the DNS is used to find the destination
SMTP server.
Five agents are involved:
\begin{multicols}{2}
	\begin{itemize}
		\item Mail User Agent: email client.
		\item Mail Submission Agent: Process and forward.
		\item Mail Transmission Agent: Forward to other MTA or MDA.
		\item Mail Delivery Agent: Deliver to MUA destination.
	\end{itemize}
\end{multicols}
Each agent forwards to the next agent.
\paragraph{Working}:

\begin{tabular}{m{9cm}m{6cm}}
\begin{enumerate}
    \item The client opens transport connection with the server.
    \item Once connection is established, the client server exchanges
    \textit{greeting} messages (\textsc{EHLO}).
    \item After that, the email transfer phase can start: clients transfer one
    or more emails by indicating \textsc{MAIL FROM:} and \textsc{RCPT TO:}.
    \item At the end, the connection is closed (QUIT).
\end{enumerate}
&
\includegraphics[width=7cm]{stmp.png} \\
\multicolumn{2}{m{15cm}}{220: service ready, 221: connection close, 250:
action okay, 354: start mail, 3xx: need more information, 5xx: error.}
\end{tabular}

\subsubsection{The post office protocol (POP) }
POP runs above a connection-oriented service. (\textit{Server listens on port
110}), it allows a client to download all of his/her email from a server.

\paragraph{Working}:

\begin{tabular}{m{9cm}m{6cm}}
\begin{enumerate}
    \item Autorisation phase when the server verifies the client's credentials
    (\textit{username/password}).
    \item Transaction phase when the client downloads the message.
    \item Update phase that concludes the session.
\end{enumerate}
&
\includegraphics[width=7cm]{pop.png} \\
\multicolumn{2}{c}{+ok: accept, -ERR: error.}
\end{tabular}


\subsection{HTTP}
Uses hypertext links to link files together.

\paragraph{Three components of the \textbf{World Wide Web}}
\begin{enumerate}
    \item \textbf{Uniform Resource Identifiers}: A URI \strong{uniquely}
    identifies a resource on the World Wide Web.
    \item A standard document format: \textbf{HTML}, containing hypertext
    links.
    \item A standard protocol with efficient access to documents on the server:
    \textbf{HTTP} made of request(\textit{method, header, MIME document}) and
    responses(\textit{status line, header, MIME document}).
\end{enumerate}

\paragraph{\textbf{URI}} is divided up in three parts:
\begin{enumerate}
    \item A \textbf{protocol} for the application layer (\texttt{http://},
    \texttt{mailto:}, \texttt{ftp://}).
    \item \textbf{Authority} which is the IP address or DNS server where the
    file is located.
    \item A \textbf{path} to the file in UNIX format.
\end{enumerate}

\subparagraph{URI looks like this:}
\texttt{protocol://utilisateur@serveur:port/ressource}.

\paragraph{\textbf{HTML}}
Divided up in two parts: head and body.

\paragraph{\textbf{HTTP}}
Protocol is text-based and runs above a connection-oriented service. Contains:
\begin{enumerate}
    \item A \textbf{method} (\textit{GET, HEAD, POST}), a \textbf{URI} and a
    version of the HTTP.
    \item A \textbf{header} (specifies optional parameters).
    \item An optional MIME.
\end{enumerate}

\subparagraph{Persistent TCP connection}
An HTTP request has to be \strong{self-contained}. For this reason, HTTP1.0
would open a different connection for every HTTP request. HTTP1.1 introduced
\textbf{persistent TCP connections} via two new HTTP headers:

\begin{itemize}
    \item \textbf{Connection}: with \textit{Keep-alive or close} to specify
    what the client wants.
    \item \textbf{Keep-alive}: with the \textbf{maximum number} of requests
    that the server agrees to serve and the \textbf{timeout} after which the
    server will close an idle connection.
\end{itemize}

\paragraph{Preferences}
However, it is possible for servers to want to personalise responses according
to the client's preference. Three solutions come to mind:

\begin{enumerate}
    \item By forcing the client to authenticate themselves, (user/password is
    deprecated).
    \item By using HTTP headers such as \texttt{Accept-*}. In real use, this is
    however not used except for Accept-language by the browser.
    \item By using cookies in two headers: Cookie (\textit{request}) and
    Set-cookie (\textit{response}).
\end{enumerate}

\subsection{Remote procedure call}
Similar to calling a procedure inside code, except for the fact it happens on
the network.

\paragraph{Encoding data}
The first issue is encoding data. Two popular mechanisms exist: \textsc{xdr}
(\textit{more efficient}), \textsc{json} (\textit{more readable}),\ldots

\paragraph{Reaching the call}
The second issue is sending out data.

A simple mechanism is \textsc{json-rpc} which can be used both
connectionless or connection-oriented, containing:
\begin{itemize}
    \item a request: a \textbf{json-rpc} string indicating the version of the
    protocol, the name of a \textbf{method}, a structure containing the
    \textbf{params} and an \textbf{id}.
    \item a response: a \textbf{json-rpc} string indicating the version of the
    protocol, a \strong{result} or an error and an \textbf{id}.
\end{itemize}

\subsection{Internet transport protocols}
On the internet, the network layer provides an unreliable connectionless
service and an IP address to identify each host, with a maximum packet size
equal to $\SI{64}{\kilo\byte}$ of payload.

\subsection{The User Datagram Protocol (UDP)}
Unreliable connectionless transport service, runs on top of an unreliable
connectionless network service, which uses ports to allow communicating with
multiple applications and has the following characteristics:

\begin{itemize}
    \item SDUs have to be $< \SI{65467}{\byte}$.
    \item Does not guarantee the SDU is delivered.
    \item Can't deliver a corrupted SDU.
\end{itemize}

\paragraph{Header} = Source port (16), dest port (16), length (16) and checksum
(16)

\paragraph{Port}: 0 - 1023 (\textit{privileged}), 1024 - 49 151
(\textit{registered}), 49 152 - 65 535 (\textit{ephemere})

\paragraph{Usage}
\textbf{UDP} is used when delay must be minimised or losses can be recovered by
the application itself. (\textit{Or real-time applications like interactive
videos.})

\subsection{The transmission control protocol (TCP)}
Bi-directional bytestream of a connection-oriented transport service, runs on
top of an unreliable connectionless network service.

(Window-based transport protocol using Go-Back-N).

\paragraph{Header}:
\begin{itemize}
    \item Source port (16), Destination port (16).
    \item Sequence number per byte (32), ack number (32), window size (16)
        \begin{itemize}
            \item[$\to$] TO make a reliable data transfer.
        \end{itemize}
    \item Flag: \textit{SYN, FIN, RST, ACK}.
    \item checksum (16).
    \item Length in world (4).
\end{itemize}

\subsubsection{TCP connection etablishment}
Three-way handshake using a sequence number, ack number and SYN/ACK/RST flag.

\begin{figure}[!ht]
  \centering
  \includegraphics[width=12cm]{tcpconnect.jpg}
  \caption{TCP connection.}
  \label{fig:tcpconnect}
\end{figure}

While connection is being established, the client/server negotiates different
options:
\begin{enumerate}
  \item The \textbf{MSS}, which is the size of the largest acceptable payload
  (\textit{at least 536}).
  \item Window size.
  \item Usage of SACK (information in out-of-sequence segments).
  \item A number of options where the client and the server both communicate
  what they want and support.
\end{enumerate}

\paragraph{Options encoding} is done with \textsc{TLV}, Type-Length-Value.

\paragraph{Robustness principle}: SYN can contain unknown options without
crashing, however SYN+ACK can only contain that which we know.

\paragraph{Denial of service attacks}
A server maintains a TCB (transmission control block) which contains the state
of each connection's TCP entity. To avoid memory overflow, the TCB's size is
limited to 100.

\textbf{Denial of service attacks} try to render data on the network
unaccessible.

\textbf{SYN flood attack} merely has to flood the server with SYN to flood its
TCB, thus blocking everyone from establishing a connection with it.

\subparagraph{Solution:} Using client-side cookies to replace the server TCB.
However, this isn't entirely backward-compatible with the TCP option.

\subsubsection{TCP reliable data transfert}

For every TCP connection, a \textbf{TCB} (transmission control block) is
maintained with all necessary information to send and receive segments.

\paragraph{Segment transmission strategy}

Two radical solutions:
\begin{enumerate}
    \item Send when needed, but this could mean sending out a single byte of
    information, together with 20 header bytes, which is hardly ideal.
    \item Send whenever \textsc{mss} bytes of data have been filled.
\end{enumerate}

\subparagraph{Nagle's algorithm:} the packet is sent if it either fills up the
\textsc{mss}  or if an acknowledgment has just been received. (\textit{At least
every round trip time}).

\subsubsection{TCP window}

Negociating the window scaling factor ($0 \leq \textnormal{scaling}  \leq 14$)
is done at connection time, even if some implementations automatically adjust
its size.

\subsubsection{TCP retransmission}
\textbf{Go-Back-N} needs a good retransmission timer (\textsc{rtt} is a good
first guess even though it changes over time).
\begin{center}
\textsc{rtt} $\approx$ delay between the transmission of a segment and the
reception of an ack.
\end{center}

\paragraph{Measure RTT}
The issue is we don't know which segment the ack refers to (multiple identical
segments might have been sent).

To solve this, \textbf{timestamp options} are used:
\begin{itemize}
    \item \textbf{TS}: for example clock value.
    \item \textbf{TS} echo: last received \textsc{ts}.
\end{itemize}

Since there is no more ambiguity, the time value can be updated
($\textnormal{rtt} = \SI{3}{\second}$ at initialisation).

\begin{align*}
\textnormal{srtt} \textrm{ (smoothed rtt mcomputed) } &=  (\alpha \times
\textnormal{srtt}) + (1-\alpha) \times \textnormal{rtt} \\
\textnormal{rto} \textrm{ (retransmission timedout) } &= \min(60, \max(1, \beta
\times \textnormal{srtt})).
\end{align*}

where rtt is the measured \texttt{RTT} and $60, 1$ are minimal/maximal bounds.

\subparagraph{Jacobson's algorithm:}
However, in a real world environment, this doesn't work too well. To solve this
problem, Jacobson proposed the following idea: rto is always initialised to
\SI{3}{\second}.
\begin{multicols}{2}
\begin{itemize}
    \item[-] For the first \texttt{RTT} value
        \begin{align*}
            \textnormal{srtt} &= \textnormal{rtt} \\
            \textnormal{rttvar} &= \frac{\textnormal{var}}{2} \\
            \textnormal{rto} &= \textnormal{srtt} + 4 \cdot \textnormal{rttvar}.
       \end{align*}

    \item[-] Otherwise
        \begin{align*}
        \textnormal{rto} &= \textnormal{srtt} + 4 \cdot \textnormal{rttvar} \\
        \textnormal{rttvar} &= (1-\beta) \times \textnormal{rttvar} + \beta
        \times \abs{\textnormal{srtt} - \textnormal{rtt}} \\
        \textnormal{srtt} &= (\alpha \times \textnormal{srtt}) + (1-\alpha)
        \times \textnormal{rtt}).
   \end{align*}
\end{itemize}
\end{multicols}

\paragraph{$\alpha, \beta$ value: }  $\beta = \frac{1}{4}, \alpha =
\frac{1}{8}$.

\subsubsection{Advanced retransmission strategy}

\paragraph{}
\textsc{tcp} uses \textbf{Go-Back-N} by default, and when the same timer has
expired, it is doubled (called \textit{exponential backoff}) until it reaches
\SI{60}{\second} after which it is declared unreachable.

\paragraph{Piggybacking} is used by \textsc{TCP} when there is a bidirectional
data transfer, which is rather rare.

\paragraph{Delayed acknowledgment strategy}
It is also possible to have performance issues when sending acks when they
serve no real purpose. Typically, a \textbf{delayed acknowledgment strategy}
can be implemented which makes sure an ack is sent every time a timer runs out
(\textit{a typical delay value is one second}) or when an out-of-sequence
segment is received because in that case the ack is of utmost importance.

\paragraph{Out-of-sequence}
It's really easy for the receiver to add a buffer to recover the
out-of-sequence segments without having to notify the sender.

\paragraph{Fast retransmit}
Currently, the timer has to end before a lost segment can be sent back. Another
method is to detect the the error when three acks are receivent for the same
segment.

It does however require a \texttt{dubpacks} varialbe to be added to the
\textsc{tcb}.

\paragraph{Selective repeat}
One of the \textsc{tcp} options allows activating \textbf{SACKs} which allow us
to indicate which segment were received out-of-sequence. In this case, Fast
Retransmit is often used.

\subsubsection{TCP connection release}
Abru[t (\textsc{rst}) or gracefully (FIN, FIN+ACK).

\subsection{The stream control transmission protocol (SCTP)}
Alternative to \textsc{tcp} which offers the following characteristics:
\begin{enumerate}
    \item Efficiently supports \textbf{multihomed hosts}, that is multiple
    network interfaces. In \textsc{tcp}, there is a single IP address per
    interface, which means that if a user is connected via WiFi and the
    connection is dropped, mobile data can't recover instead.
    \item The possibility to send \textbf{messages} via bytestream.
    \item Partially-reliable, useful for timed delivery.
    \item A single real connnection with logical streams in order to avoid
    having to manage multiple connections.
\end{enumerate}

\subsubsection{Segment}
A segment is a header followed by a sequence of chunks.
\begin{figure}[ht]
    \centering
    \begin{tabular}{m{8cm}m{7cm}}
        \includegraphics[width=8cm]{sctpsegment.png} &
        \includegraphics[width=7cm]{sctpchunk.png}
    \end{tabular}
    \caption{SCTP segment/chunk format.}
\end{figure}

\paragraph{Header} = Source port number (16), dest port number (16),
verification tag (32) and checksum (32).

\paragraph{Chunk} = Type (8), flags (8), length (16) and value (32) to allow
for easy insertion of options. STCP, as opposed to \textsc{tcp}, is not limited
in its number of options. Its a good example of an easy to extend protocol.


\subsubsection{Connection etablishment}
Four-way handshake (to protect against ``Denial of service attacks'').

\begin{figure}[ht]
    \centering
    \includegraphics[width=6cm]{fourway.png}
    \caption{Four-way handshake.}
\end{figure}

\subsubsection{Reliable data transfer}

\begin{figure}[ht]
    \centering
    \begin{tabular}{m{8cm}m{8cm}}
    \includegraphics[width=8cm]{datachunk.png} &
    \includegraphics[width=8cm]{sackchunk.png}
\end{tabular}
    \caption{SCTP data/sack chunk format.}
\end{figure}

\paragraph{Data chunk}
Data is sent via \textbf{data chunks} which use a TNS as a sequence number
(increment by 1 for every data chunk). When a chunk is divided up into parts,
the \textsc{b} and \textsc{e} bits are used to specify the first and last chunk.

\paragraph{Sack chunk}
To guarantee arrival, a cumulative TNS ack is used (on chunck level and not
byte level). It also supplies information about the out-of-sequence chunks. Some
other differences include:
\begin{enumerate}
    \item Can supply information about the various ``holes'' in the reception
    buffer.
    \item Can give feedback about a duplicate chunk. (Points to a bad heuristic
    on the sender's behalf.)
\end{enumerate}

\subsubsection{Connection release}
Uses a three-way handshake to close the connection, or an \texttt{ABORT} chunk
to either refuse connection or close it immediately.

\subsection{UDP - TCP - SCTP}
\begin{table}[ht]
    \begin{tabular}{|c|c|c|c|c|c|}
    	   \hline
    	    Protocol& Reliablility&\# options&Mode&Establishment&Num seq\\
        \hline
        UDP & Unreliable   & x              & Stream-mode     &
        x                   & x \\
        \hline
        TCP & Reliable     & Option finie   & Stream-mode     & Three-way
        handshake & seq. = byte \\
            &              &                &  Single host
            &                     &  \\
        \hline
        SCTP & Reliable or & Option infinie & Message mode as & Four-way
        handshake & chunk \\
             & partially   & with chunk     & a stream mode
             &                    & \\
             &             &                & Multihomed host
             &                    & \\
        \hline
    \end{tabular}
    \caption{Difference between UDP, TCP and SCTP.}
\end{table}
\subsection{Congestion control}
The transport layer is where control happens.

\textsc{tcp} controls congestion by acting on the window size since a
connection can't send data faster than \[\frac{\textnormal{window}}{rtt}\quad
\textnormal{window} = \min(swin,rwin).\]

\paragraph{Congestion window}
\textbf{CWND} is stored in the \textsc{TCB} of every connection and the size of
the window is $\min(cwnd, rwin, swin)$.
\begin{itemize}
    \item[-] \textit{Additive increase}: every \textsc{rount trip time},
    \textsc{cwnd} is incremented MSS bytes (Congestion Avoidance Phase).
    \item[-] \textit{Multiplicative decrease}: when congestion is detected,
    \textsc{cwnd} is decreased by a certain factor.
\end{itemize}

\paragraph{Initial value of cwnd:}
\textbf{MSS} bytes so as to not cause congestion the network at its launch.
(This initial value has been rounded up to \SI{15}{\kilo\byte}.)


However, cwnd goes up slowly until it arrives to a value which uses the network
efficiently. To avoid this slow process, a \textbf{slow-start algorithm} is
used which doubles \textsc{cwnd} every RTT.

\paragraph{Detecting congestion} comes down to detecing packet loss.

\begin{itemize}
    \item \textit{mild congestion}: if fast retransmit is used (cwin divided
    by two).
    \item \textit{severe congestion}: when the transmission timer expires
    (sstresh is divied by 2 and $\textnormal{cwin} = \textnormal{MSS}$).
\end{itemize}

\begin{figure}[ht]
    \centering
    \begin{tabular}{m{8cm}m{8cm}}
    \includegraphics[width=8cm]{severecongestion.png} &
    \includegraphics[width=8cm]{midlecongestion.png} \\
    \multicolumn{1}{c}{Severe congestion.} & \multicolumn{1}{c}{Middle
    congestion.}
\end{tabular}
    \caption{Congestion window when congestion is detected.}
\end{figure}

\subsubsection{Controlling congestion without losing data}
The main idea here is to detect congestion before actual packet loss by using an
\textbf{Explicit Congestion Notification}. (When a packet passes through a
congested router, the \textbf{CE} bit is set to 1 and when the receiver gets
this information they transmit it back to the sender in order for the latter to
adjust their rate.)

\begin{enumerate}
    \item In order to use this solution, \textbf{another bit} (ECT, ECN-capable
    transport) is needed to specify whether the packet uses \textbf{ECN} or
    not. If nothing is specified, and the router cis congested, the ones that
    do not implement ECN are given priority.

        \begin{itemize}
            \item[$\to$] if the router is congested and the packet implements
            ECN, CE is set to 1, otherwise the packet is discarded.
        \end{itemize}
    \item If the protocol is reliable, the sender can be informed through an
    ack, either via a flag in the header or an option (TCP uses a flag, STCP
    uses an option).
        \begin{itemize}
            \item[$\to$] A variable keeps a value of 1 on the receiver end
            while received packets are marked as ``congested''.
        \end{itemize}
    \item Lastly, the sender and receiver know whether they are using a ECN via
    a TCP option during the connection three-way handshake.
\end{enumerate}

If the receiver detects congestion, the next sent packets will have this
information (to avoid it eing lost if the ack gets lost).

\paragraph{Router algorithm}
Two types of routers, either with a single FIFO or multiple FIFOs and a round
robin scheduler.

Instead of measuring the buffer's fullness at a given moment in time, an
average is taken. On top of that, every packet has a certain probability of
being marked ``congested'', and the higher the average fullness is, the higher
this probability becomes (Random Early Detection).

\paragraph{}
When multiple queues are present this probability is computed independently to
guarantee its correctness.

\subsubsection{Modeling TCP congestion control}
To show the factors that affect the performance of TCP, if $p$ is the segment
loss ratio then $\frac{1}{p}$ is the number of successfully transferred
segments.

\begin{figure}[ht]
    \centering
    \includegraphics[width=9cm]{modelingtcp.png}
    \caption{Evolution of the congestion window with regular losses.}
\end{figure}

\begin{description}
    \item The number of segments that are sent during a cycle is in yellow:
    $A = \frac{3 \times W^2}{8}$.
    \item But by definition, $A = \frac{1}{p}$ so: $W = \frac{k}{\sqrt{p}}$
    \item The throughput (\textit{bytes/sec}) is the number of segments
    transmitted divided by the duration of the cycle:
    \[\textnormal{Throughput} = \frac{A \times
    \textnormal{MSS}}{\textnormal{time}} \quad \approx \frac{k \times
    MSS}{\textnormal{rtt} \times \sqrt{p}}.\]
\end{description}

This is an important result which shows that:
\begin{itemize}
    \item TCP connections with \textbf{smaller rtt} have higher throughput than
    ones with higher rtt when losses occur (\textit{unfair}).
    \item TCP connections that use a \textbf{large MSS} can achieve a higher
    throughput than ones with shorter MSS (\textit{unfair}). Most hosts use the
    same $\textnormal{MSS} = \SI{1460}{\byte}$.
\end{itemize}

\paragraph{Conclusion:} $\textnormal{Throughput} <
\min(\frac{\textnormal{window}}{\textnormal{rtt}}, \frac{k \times
\textnormal{MSS}}{\textnormal{rtt} \times \sqrt{p}})$
Maximum throughput depends on the maximum window size and rtt if there are no
losses, otherwise it depends on the MSS, rtt and loss ratio.

\subsection{Network layer}
Three types of datalink layers:
\begin{enumerate}
    \item Directly a physical link from the physical layer.
    \item Local Area Network.
    \item Non-Broadcast Multi-Access (used to emulate a LAN with only unicast).
\end{enumerate}

\paragraph{LAN}
In a LAN, every host is identified by a datalink layer address (\textbf{MAC
address}).

A LAN supports broadcast and multicast datalink layer addresses. A frame sent
to the \textbf{multicast} address of the LAN is sent to all the participants in
the group. The \textbf{broadcast} address is used to send out messages to
everyone.

\subsubsection{IP version 6}
Soon, IPv4 won't be able to accommodate for all addresses anymore, which is why
another solution is needed.

\begin{tabular}{m{2.5cm}m{10cm}}
    Assumptions: &
    \begin{itemize}
        \item[-] Adress encoded on 128 bits.
        \item[-] IPv6 format header can easily be parsed by hardware device.
        \item[-] Able to configure IPv6 address automatically.
        \item[-] Security.
    \end{itemize}
\end{tabular}

\paragraph{IPv6 addressing architecture}
Scalability of a network layer protocol depends on its addressing architecture.

IPv6 supports \textbf{unicast, multicast and anycast}.

\paragraph{\textbf{Unicast}}:

\begin{tabular}{m{8cm}m{7cm}}
    \begin{itemize}
        \item Global routing prefix that is assigned to the \textbf{Internet
        Service Provider}.
        \item Subnet id identifies a customer of the ISP.
        \item Interface id.
    \end{itemize}
    &
    \includegraphics[width=8cm]{unicast.png}
\end{tabular}

\paragraph{ }
\textbf{Hierarchical} address allocation allows for minimising the amount of
routes known to the router. It therefore only knows the route for certain
address blocks (\textit{$2^{128}$ grouped in $2^{64}$ subnets!}).

Two types of address allocation:
\begin{itemize}
    \item \textit{provider independent} (PI): For companies connected to at
    least two ISPs. Address blocks are given out independently of the ISP.
    \item \textit{provider aggregatable} (PA): Depends on the ISP.
\end{itemize}

The drawback with \textbf{PA} addresses is that when your provider changes, all
the addresses you use in a PA address block also change.

\paragraph{Size IPv6 address}
\texttt{/32} = Internet Service Provider, \texttt{/48} = single company,
\texttt{/56} = small user site, \texttt{/64} = single user, \texttt{/128} =
rare. (No more than one host attached to a prefix.)

\paragraph{Usage of the IPv6 prefix}
\textbf{Longest prefix match} assures the route that has the best match with
the address is the one that is used.

Note:\texttt{::/0} matches with all and is therefore the \textit{default
route}.

\paragraph{\textbf{Unique Local Unicast}}
The ULA address is an \texttt{fc00::/7} and is similar to an IPv4 address. The
address isn't necessarily unique and a router doesn't forward a ULA.

As opposed to local unicast links, it doesn't necessarily need to have the same
link.

\paragraph{\textbf{Link Local Unicast}}
The address starts with \texttt{fe80::/64}, followed by 64 interface bits. Used
when two hosts on the same link (or LAN) want to exchange a packet. The router
can't forward a packet with a local unicast link. (\textit{Used when regular
IPv6 isn't an option, that is, an isolated LAN.})

\paragraph{\textbf{Multicast}}
Allows for sending a packet to all people in a same LAN group.

\begin{figure}[ht]
    \centering
    \includegraphics[width=8cm]{multicast.png}
    \caption{Multicast address.}
\end{figure}


\paragraph{\textbf{IPv6 packet format}}

\begin{figure}[ht]
    \centering
    \includegraphics[width=12cm]{ipv6format.png}
    \caption{IPv6 packet format.}
\end{figure}

\subparagraph{Header}(\SI{40}{\byte}) = Payload length (16 bits), Hop limit:
\textit{limit of the router that can forward the packet} (8 bits) and Next
header (8bits) + Version (4bits) + SRC (128b) + DST (128b) + Traffic (8bits
ECE/ECT).

\subparagraph{No checksum}
There is no checksum in the IPv6 packet header since there is already a
checksum on the \textbf{frames} at the datalink layer level. Adding a checksum
prevents memory corruption errors in a router but at a very high cost (which is
\emph{ridiculous}).

\subparagraph{Next header}
Indicates the next header, such as a transport layer header (\textit{UDP=17,
TCP=6, SCTP=132}) or an IPv6 option since the packets can have multiple
headers. An option also has the \textit{next header} field.

\subparagraph{Options}
Contain a next header, a length and an \textbf{opt field} which indicates what
the receiver should do in case of an unknown option:
\begin{itemize}
    \item[-] 00: continues without worrying about the option.
    \item[-] 01: discards the packet.
    \item[-] 10: returns a control packet.
\end{itemize}

\subparagraph{Fragment option}
In IPv6, fragmentation is done by the sender and not by the router. The router
discards the packet and send an error message to the sender if the packet is
too big.

A fragmented packet contains:
\begin{itemize}
    \item[-] the offset (13) of the fragmentation;
    \item[-] next header (8) (\textit{because it's an option});
    \item[-] a flag: 0 = last fragment;
    \item[-] a field to identify the \emph{original} packet. It must make sure
    not to use the same one for MSL seconds.
\end{itemize}

\subparagraph{Note:}
Packets can be sent from first to last or in reverse order. The second solution
allows for the receiver to know the size of the packet (and thus the necessary
buffer). When a fragment is received, a timer is started and if all fragments
haven't arrived by the end of this timer, the packet is considered lost.

\paragraph{Loose source} IPv6 option to determine which route to take (router
addresses are specified). Isn't used because of security concerns.

\subsubsection{ICMP version 6}
\textit{Internet Control Message Protocol} version 6 is used in IPv6 packets.
(next header = 58) to transmit messages to the sender. There are two types of
messages:

\begin{itemize}
    \item Error messages:
        \begin{enumerate}
            \item Destination unreachable (0: no route, 1: firewall refuse, 2:
            sender used link-local address to reach global unicast address, 3:
            address unreachable, 4: port unreachable);
            \item Packet too big;
            \item Time exceeded;
            \item Parameter problem.
        \end{enumerate}
    \item Information messages: when a host receives an \textit{Echo request},
        it has to respond with an \textit{Echo reply}.
\end{itemize}

\paragraph{PathMTUDiscovery}
Use of ICMP segment to discover the MTU of the path. Then adjust MSS to avoid
fragmentation.

\subsection{The IPv6 subnet}
Every interface on a device is identified by a unique MAC address\footnote{MAC
address blocks are allocated to manufacturers.}.

Thanks to this uniqueness, two hosts connected on a LAN have a unique address.

\paragraph{Broadcast:} the packet is delivered to all devices attacked to the
datalink network.

\paragraph{Multicast:} just like a broadcast, except the interface knows
whether a packet is destined to it or not (\textit{based on a logical address}).

Potentially, all IPv6 nodes are capable of capturing frames destined to
different multicast addresses.

\subsubsection{Interactions between IPv6 and datalink layer}

\paragraph{In an internetless LAN,} hosts take on an IP address thanks to
\textbf{link-local addresses}: with MAC addresses, \texttt{fe80::/64} is chosen.

\paragraph{If the LAN is connected to the internet,} the \textbf{Neighbor
Discovery Protocol} (part of ICMPv6) is used to determine the others' addresses.

\paragraph{Neighbor Discovery Protocol (NDP)}
Care must be taken since IPv6 addresses have been configured \textbf{manually}.

\begin{enumerate}
    \item Sends out a \textbf{ICMPv6 Neighbor Solicitation} (NS) with the IPv6
    address in question.
    \item The receiver responds with a \textbf{ICMPv6 Neighbor Advertisement}
    with its IPv6 and MAC addresses.
    \item As it is receiving the ICMPv6 NA, it stores the IPv6-MAC link in its
    \textbf{NDP table} (mapping between IP address and MAC).
\end{enumerate}

\subparagraph{Note:} ICMPv6 NS can also be used in unicast to know whether a
host is reachable. On top of that, the ICMPv6 NA is stored temporarily in the
cache and when it expires, the address has to be revalidated.

\paragraph{Duplicate Address Detection algorithm} makes it possible to not have
two identical IPv6 addresses. In order to do this, it sends an \textbf{ICMPv6
NS} in unicast to its address, and if it doesn't receive a response, it is
unique.

\paragraph{\textbf{Automatically configure IPv6 addresses}}
There are two protocols: SLAAC and DHCPv6.

\subparagraph{Stateless Address Auto-Configuration (SLAAC) }
\begin{enumerate}
    \item A device creates its \textbf{link-local address} with
    \texttt{fe80::/64} and \SI{64}{\bit}, part of which comes from the MAC.
    \item It verifies its uniqueness by sending out an \textbf{ICMPv6 NS} with
    its link-local in multicast.
    \item To know the \textit{prefix IPv6 subnet}, the router regularly sends
    ICMPv6 \textbf{Router Advertisements}, multicast on \textbf{ff02::1}
    \footnote{Corresponds to all reachable hosts.} containing:
        \begin{itemize}
            \item[-] \textit{Cur hop limit}: temporary maximum of the router
            (usually 64).
            \item[-] \textit{Router lifetime}: estimated time where the router
            is the default router.
            \item[-] \textit{Reachable time and Retrans timer}: used to
            configure the NDP protocol usage for the subnet hosts.
        \end{itemize}

        A lot of \textbf{options} can be added to Router Advertisement messages:
        \begin{itemize}
            \item \textbf{MTU} option: indicates the maximum size of what can
            be transmitted \textbf{without} fragmentation on the subnet.
            \item \textbf{Prefix} option: gives the prefix size and its length,
            and the estimated lifetime of the prefix.
        \end{itemize}
    \item IPv6 address = Prefix + 64 identification bits.
    \item Sends an \textbf{ICMPv6 NS} to its IPv6 to assure uniqueness.
\end{enumerate}

\subparagraph{Notes:} A device can send an ICMPv6 Router Solicitation to the
router via \textbf{ff01::2} \footnote{Corresponds to all reachable routers.} to
ask for an ICMPv6 RA if it considers the delay to be too long.

\subparagraph{Dynamic Host Configuration Protocol (DHCP)}
Utilise un serveur DHCP. Ce serveur a un ensemble d'addresse et écoute sur le port 67.
\begin{itemize}
	\item Quand un host se connect, il envoie un \textbf{DHCP request} en UDP à tout les serveurs DHCP
	\item Le serveur capture la request, et envoie un \textbf{DHCP reply} avec un addresse IP non assigné.
	Ce reply contient aussi le temps de vie de l'addresse pour forcer à renouveller son addresse.(Donc les addresse peuvent être réutilisé).
\end{itemize}

\paragraph{Multi router on subnet}
Grace au \textbf{ICMPv6 Redirect Message}, il est possible que les hosts
apprennent automatiquement les nouvelles routes:

\begin{itemize}
    \item[-] Lorsque un router doit forwarded back un packet, il envoit un
ICMPv6 RM qui indique que le packet doit être envoyé par un autre router.

    \item[-] Celui ci met à jour sa forwarding table (\textit{timer est souvent associé
    à ce nouveau router pour supprimer la router après un certain temps})
\end{itemize}


\subsection{Routing in IP networks}
Deux classe de protocol entre différents domaines pour echanger
efficacement de l'information.

Une grande différence entre \textbf{intra} et \textbf{inter} domaine est
la \textit{routing policies} utilisé par chaque domaine.

Dans un domaine toute les routers sont égaux et la meilleur route est choisi sur
différents critére: temps, nombre d'intermédiare et taille de bandwidth.

\subsection{Intradomain routing}
Echange des informations sur les destinations atteignables \textbf{dans} le domain.
\textit{RIP} est protocol de \textbf{distance vector} et \textit{OSPF}
utilise \textbf{link-state routing}.

\subsubsection{RIP}
Les routers echanges periodiquement des RIP messages (inside \textbf{UDP} segment).

Pour accélerer le processus quand un router boot, il peut envoyer une
\textbf{RIP request} en multicast à ff02::9
(\textit{pour demander les tables de routings}).

Tout ceux qui reçoivent la demande répond en envoyant sa routing table avec
une séquence de \textbf{RIP response}.

\paragraph{Note: } Les routers envoyaient périodiquement (30sec) une ou plusieurs RIP response
reprenant les distances vectors qui résume la routing table. Mais cela causait au routeur de se synchroniser et parfois d'être surcharger. Maintenant ils attendent un temps aléatoire entre 15 et 30.

\subsubsection{OSPF}
Avec le link-state routing, pour des grands réseaux c'est très couteux de stocker
l'ensemble du réseau en mémoire.

\begin{tabular}{m{10cm}m{5cm}}
    \begin{itemize}
        \item \textbf{Border router}: Router attaché à plusieurs area
        \item \textbf{Internal router }
        \item \textbf{Area}: Les routers connaissent la topologie de leur area
            et comment rejoindre la backbone area.
        \item \textbf{Backbone area}: C'est l'area qui regroupe les \textbf{border
            router} et ceux qui ne sont pas dans une area.
    \end{itemize}
    &
    \includegraphics[width=5cm]{area.png}
\end{tabular}

Notons que les border router nécessite une configuration manuelle.

\paragraph{Area}
Les routers échangent des link-state packet à tout ceux dans l'area.

\paragraph{Inter-area}
L'inter-area  routing  est  fait  en  échangeant  des  distance  vector
protocol pour prévenir  les routeur d'une area le cout pour atteindre
un routeur d'un autre domain.

\paragraph{LAN}
Quand  les  routeurs  boot  dans   un  LAN,  ils  élisent  un  routeurs
(\textbf{Designated  Routeur}) pour  ne pas  devoir échanger  des HELLO  packet
entre tout  les routeurs. Les  routeurs peuvent seulement  échanger des
HELLO avec le DR: c'est lui qui représente la LAN.

(\textit{La  topologie  de  la  LAN   apparait  comme  un  full-mesh  of
point-to-point link connected to the DR router})

\subsubsection{Shortest path}
Les protocoles intradomaines choississent les plus court chemin. Il peut arriver qu'il y ait deux
chemins qui aient le même coût. Dans ce cas deux approches:
\begin{itemize}
    \item On utilise toujours le même (\textit{en ne mettant que celui la dans la forwarding table}).
	\item On utilise les différents chemin:
	\begin{itemize}
		\item Via un Round Robin mais cela implique d'utiliser des chemins différent pour les
		paquets d'un même flow (TCP connection) et potentiellemnt augmenter les arrivées
        hors séquence.
        \begin{center}
            (\textit{hors-sequence $\to$ dupplicate ack $\to$ fast retransmit $\to$ congestion })
        \end{center}

		\item Utiliser un Round Robin par flow, via une hash value du tuple de paquet.
		   La même interface (chemin) sera utiliser pour les paquets du même flow.
            $$hash( NextHeader, IP_{src}, IP_{dst}, Port_{src}, Port_{dst} ) mod N$$
            N is the number or outgoing interface on the equal cost paths
	\end{itemize}
\end{itemize}

\subsection{Interdomain routing}
Echange de l'informations entre les domaines. Cette information est une information
agrégé des routers et on considère ici les domaines comme des boîtes noires.

A \textit{content-rich} sub domain is a domain that contains hosts that
mainly receive packets. (Google,\ldots)

\subsubsection{Connected link}
\begin{itemize}
    \item \textbf{Private peering link} permet de lier deux domaines. Pour des questions
de performances plusieurs lien physique sont établi entre les domaines.

     \item \textbf{Internet eXchange Point} Une solution moins couteuse est de les connecter
via un IXP.  The IXP contient une LAN où tout les routers sont connectés.
\end{itemize}

\begin{figure}[ht]
    \centering
    \includegraphics[width=8cm]{exchange.png}
    \caption{Internet eXchange Point}
\end{figure}

\subsubsection{Connection cost}
Le \textbf{coût} d'une route est très importante en interdomain
alors qu'en intra on préfére la performance.

\begin{figure}[ht]
    \centering
    \includegraphics[width=5cm]{peering.png}
    \caption{Peering relationships}
\end{figure}

Il existe deux types de relation entre domaine
\begin{itemize}
    \item \textbf{customer->provider}:
        Le customer paye pour que son domaine soit distribué dans Internet.
        (\textit{La relation inverse est que le provider partage les routes qu'il
        connait})
    \item \textbf{Shared cost}:
        Cela arrive quand on a des domaines de taille similaire.
            \textit{On ne  veut pas  qu'un packet  passant dans  un
    shared cost ne soit pas destiné au domaines après ce lien.}

        $\to$ Via un shared cost, un domaine n'annonce que ses routes internes et les liens avec
        les clients.
    \item \textbf{Sibling}:
        Ils échangenet les routes dans les deux directions (souvent c'est des routers
        de la même compagnie).
\end{itemize}

\subsubsection{Interdomain routing policie}
\begin{itemize}
    \item \textbf{Import filter}: spécifie pour chaque lien les routes acceptes du voisin.
        (\textit{les non-acceptables sont ignorés et jamais utilisé pour forward un packet})
    \item \textbf{Export filter}: spécifier pour chaque lien les routes qui peuvent
        être annoncer au voisin
    \item \textbf{Ranking algorithm}: utiliser pour choisir la meilleur route parmis
        celle que le domain à reçu vers ce préfix de destination.
\end{itemize}

\begin{figure}[ht]
    \centering
    \includegraphics[width=11cm]{policies.png}
    \caption{Import export policies example}
\end{figure}

\subsubsection{The Border Gateway Protocol (BGP)}

Dans BGP chaque domain est identifié par un unique \textit{Autonomous System} number (AS).

BGP n'envois pas sa routing table entière mais le fait de manière incrémentale, càd en
envoyant uniquement les routes qui ont changé.

De plus BGP utilise TCP pour garantir le bon délivrement des BGP messages.

\paragraph{BGP session}

\paragraph{Etablisment} doit être fait manuellement pour des raisons de sécurité.

\paragraph{Messages}

\begin{enumerate}
    \item \textit{OPEN}: Quand la connection est établie, ça initialise la session et
        négocie d'option
    \item \textit{NOTifICATION}: Pour cloturer la session
    \item \textit{UPDATE}: Averti du changement d'une route. C'est le message le plus important et pour que le protocol soit efficace, le message doit minimiser le nombre de bit envoyé.
    \item \textit{KEEP ALIVE}: Averti que rien n'a changé (pour être sur que le router
        est toujours en vie)
\end{enumerate}

\paragraph{BGP update} = \{IP prefixe retié\}, \{IP préfixe ajouté\}, \{AS-Path\}

\subparagraph{ } Une route qui est envoyé doit d'abord passer l'\textit{export filter},
de même qu'une route reçue doit passer l'\textit{import filter}. Un BGP routeur est composée de 4 structures de donnée:
\begin{itemize}
	\item Adj-RIB-In: Les routes que l'ont a apprises au routeur mais qui n'ont pas encore était filtré.
	\item Local-Routing-Information Base: Routes apprises après import filter.
	\item Fowarding Information Base: Mapping entre destination et meilleur route.
	\item Adj-RIB-Ouut: Route que le routeur apprend au voisin après application du export filter.

\end{itemize}
\paragraph{The BGP decision process}
En plus des import/export filter, il y a un algorithme qui choisit la meilleur route.
Ce choix est fait sur base des BGP attribut attaché à chaque route.

\paragraph{Local-pref}
Le premier attribut de cet algorithme est la local-preference, celui ci est attribué
selon l'import filter. (Highest value est préféré)

\subparagraph{Cheap link} On peut implémenté la préférence d'un lien peut couteux
plutôt qu'un autre via le local-pref en définissant les valeurs dans l'import filter.


\paragraph{Local-pref with customer->provider and shared cost}
\begin{enumerate}
    \item High local-pref pour les routes apprisent par le customer (\textit{provider->customer})
    \item Medium local-pref pour les shared-cose
    \item Low local-pref pour les routes apprisent par le provider (\textit{customer->provider})
\end{enumerate}


\paragraph{BGP convergence}
Certaines routing policies peuvent interférer entre elles et aboutir (en théorie),
à des ping-pong infini.

\paragraph{ } La convergence des BGP n'esy pas toujours garanti et verifier la
convergence global est un problème NP-complet.

\paragraph{Guideline to guarantee BGP convergence}
\begin{enumerate}
    \item Le graph est \textit{customer->provider} est acyclique
    \item AS préfére une route reçue d'un customer plutôt qu'une shared-cost
\end{enumerate}

\subsubsection{Structure global internet}
Les domains peuvent être divisé en 4 catégories par rapport à leurs role et leur position dans la topologie AS:
\begin{itemize}
	\item Tier-1: Le noyau de l'Internet, ce sont les domaines qui n'ont pas de provider.
	\item Tier-2: Customer de Tier-1, ils ont des customers plus petit et partage les coût avec d'autre T2
	\item Tier-3: Customer de Tier-1 et Tier-2 (stub domain ou petit ISP)
	\item Large content provider: Large datacenter, ils produisent une grandes parties des données échangées sur l'internet globale. Customer de T1 ou T2 mais souvent ils sont en shared-cost avec les T1 ou T2 (via IXP)

\end{itemize}

\subsection{Datalink layer technologie}

\subsubsection{The point to point protocol}
On se limite aux protocoles qui sont souvnet utilisés pour le \textbf{transport de paquets IP} entre hôtes et routeurs connectés par un lien point-to-point.

\paragraph{\textbf{Serial Line IP} (SLIP)}: C' est une technique de character
stuffing appliqué aux paquets IP utilisant deux caractères spéciaux:
\begin{itemize}
    \item[-] \textsc{END}: en début et fin de paquet
    \item[-] \textsc{ESC}: devant un END s'il est dans les informations à transmettre
\end{itemize}

\begin{tabular}{cm{11cm}}
    \underline{Limitations} de SLIP:
    &
    \begin{itemize}
     \item Supporte uniquement la transmission de paquets IP
     \item Configuration manuelle des addresses IP entre host et routeur.
    \end{itemize}
\end{tabular}

\subparagraph{Note:} SLIP était surtout utilisé sur des liens $\leq 20Kps$ sur lequel
    des techniques de \textbf{compressions} des headers TCP/IP étaient indispensable.
    (\textit{Les header prenaient déja bcp de temps})

    $\to$ On exploitait la redondance de segment consécutif pour compresser

\paragraph{\textbf{Point-to-Point protocol} (PPP)}: Supporte IP et d'autres protocoles de la couche réseaux sur différents types de lignes de transmission. Il utilise le bit stuffing ou le character stuffing selon l'environnement où le protocole est utilisé.

PPP est une famille de 3 protocoles qui s'utilisent ensemble:
\begin{enumerate}
    \item Le \textit{Point-to-Point Protocol}: Technique de framing pour transporter les paquets dans la couche réseau.
    \item Le \textit{Link Control Protocol}: Utilisé pour négocier les options et authentifier la session (\textit{username/password, \ldots})
    \item Le \textit{Network Control Protocol}: Spécifique à chaque protocole de la couche réseau, il permet de négocier les options spécifiques au protocole. (\textit{IPv4's NCP négocie
        l'addresse IPv4})
\end{enumerate}

\subparagraph{Frame}:
Les champs Address (8) et Control (8) sont présentes pour des raisons de compatibilité. Le champ Protocol (16 bits) contient l'ID du type de protocole utilisé pour le transport de la trame PPP.

PPP supporte des paquets de taille variable mais LCP peut négocier une taille de paquet maximum.

\begin{figure}[!ht]
    \centering
    \includegraphics[width=10cm]{ppp_frame_format.png}
    \caption{Format d'une trame PPP}
\end{figure}

\begin{center}
\textit{PPP a fournit un accès dial-up (ligne commuté) aux fournisseurs d'accès Internet (ISP).
    Cet accès à internet se faisait via l'Extensible Authentification Protocol (EAP) par
    mot de passe.
    Avec l'arrive de l'Asymmetric Digital Subscriber Lines (ADSL) et voulant réutiliser
    leur système d'authentification/facturation, IETF a dévellopé des spécifications
pour transporter des framesPPP sur d'autre réseaux que les point-to-point.}
\end{center}

\subsubsection{Ethernet}

\paragraph{Ethernet standard 1980}:
\begin{itemize}
    \item \textbf{Vitesse}  : 10 Mbps  (\textit{3Mbps en 1970})
    \item \textbf{Slot time}: 51.2ms (\textit{minimun frame size = 64bytes})
        \begin{itemize}
            \item[$\to$] Equivaut à deux fois le temps que prend un impulsion électronique
                pour parcourir une longeur maximun entre deux noeuds.

                Attendre ce slot time permet de détecter les collisions en CSMA/CD
        \end{itemize}
    \item \textbf{Frame format}:
        \begin{itemize}
            \item \textbf{Dest addresse} (48 bits): multicast, broadcast et unicast.

                Il est placé en début de frame pour que le recepteur sache rapidement si
                elle lui est destiné ou non
            \item \textbf{Source addresse} (48 bits): Uniquement unicast
            \item \textbf{Type} (16): type du paquet transporté sur la couche réseau , \textbf{CRC} (32)
            \item 46bytes $<$ \textbf{Payload} $<$ 1500 bytes
        \end{itemize}

        Note que le début d'une frame Ethernet commence par un préambule
        utilisé par la pysical layer pour synchroniser les horloges.
    \item \textbf{Adresses}:
        \begin{itemize}
            \item Addresse global \textbf{unique} (MAC addresse) sur 48bits (\textit{permet d'allouer des
                gros blocs d'addresse au fabriquant})
            \item Définition d'addresse broadcast and multicast
        \end{itemize}
\end{itemize}

\begin{figure}[!ht]
  \centering
  \includegraphics[width=9cm]{ethernet_address_format.png}
  \caption{MAC addresse format}
\end{figure}

\paragraph{ }Un réseau Ethernet fournit un \textbf{unreliable connectionless service}
(\textit{Avec unicast, multicast et broadcast}) tel que:
\begin{itemize}
    \item Very high probability of successful delivery ($\to$ utilisation de CSMA/CD)
    \item Transmission frame dans l'ordre ($\to$ topologie réseau ethernet = shared bus)
\end{itemize}

\begin{figure}[!ht]
    \centering
    \includegraphics[width=6cm]{impact_of_frame_length_in_ethernet.png}
    \caption{Pourcentage d'utilisation de la ligne en fonction de frame size}
\end{figure}

\begin{table}[ht]
    \centering
  \begin{tabular}{c|c|c|c|c|c}
  Nom & Vitesse & Longueur du cable & Type de câble & Répéteurs & Topologie réseau \\
  10Base5 & 10 Mbps & 500 mètres & Cable coaxial & Oui & Shared bus\\
  10Base2 & & 185 mètres & Cable coaxial & Shared bus\\
  10BaseF & & & Lien optique & \\
  10BaseT & & & Câble torsadé & En étoile\\
  \end{tabular}
  \caption{Différentes couches physiques définie pour Ethernet}
\end{table}


\paragraph{Introduction paires torsadées} amène à deux changements majeurs sur Ethernet:
\begin{itemize}
    \item La topologie physique du réseau (Start shaped)
    \item L'obligation d'instaurer des hubs pour créer cette topologie
\end{itemize}

\paragraph{Hub}:

\begin{tabular}{m{10cm}m{5cm}}
    Un hub est un relais sur la couche physique qui
    \begin{itemize}
        \item reçoit un signal électrique sur une interfaces
        \item le transmet sur toutes ses autres interfaces
        \item est capable de convertir des signaux électriques de deux
            couche physiques différentes (10BaseT$\to$10Base2)
    \end{itemize}
    &
	\includegraphics[width=5cm]{ethernet_hub.png}
\end{tabular}

Pour créer un réseau complexe avec des hubs, il faut faire attention à certains points:
\begin{enumerate}
    \item La topologie du réseau doit être un arbre
    \item Il peut exister des collisions $\to$ CSMA/CD
    \item L'étendue du réseau est limitée par le slot time
\end{enumerate}

\begin{figure}[!ht]
    \begin{center}
    \includegraphics[scale=0.3]{hierarchical_ethernet_network.png}
    \caption{Un réseau Ethernet hiérarchique composé de hubs}
    \end{center}
\end{figure}

\paragraph{Le Fast Ethernet} est une technologie LAN qui permet d'aller jusqu'à 100Mbps en utilisant la fibre optique. Deux contraintes:
\begin{itemize}
    \item Supporter les paires torsadées (\textit{en couche physique on aurait préféré les câbles coaxiaux mais c'est chiant pour la maintenance})
    \item Utiliser le même format de frame qu'en 10Mbps (parce qu'à la base, ce lien de 100Mbps servait de lien entre 2 liens 10 Mbps).

        $\to$ Pour préserver CSMA/CD: slot time = 5.12 microsecondes.
\end{itemize}

\paragraph{\textbf{Ethernet switches}}
Une autre solution pour améliorer les performances des LAN ethernets (\textit{que le fast ethernet}), est de rendre les hubs intelligents.

\paragraph{Switch} \textit{hubs intelligents} capable d'agir sur la couche datalink
et analyser l'addresse de destination de chaque frame et forwarder dans la direction
de la destination.

\begin{figure}[!ht]
    \centering
    \begin{tabular}{cc}
        \includegraphics[width=6cm]{ethernet_switches.png} &
        \includegraphics[width=6cm]{switch_mac_address_table.png}
    \end{tabular}
    \caption{Les switch Ethernet et table des addresses MAC}
\end{figure}

\subparagraph{Switch forwarding table}
\begin{itemize}
    \item La table de forwarding fait le lien entre MAC addresse et port où forwarder

    \item Utilisations des MAC addresse (48 bits), un LAN ethernet est \textit{plug and play}.
        (\textit{On se connecte et on peut directement échanger})

        \begin{itemize}
            \item[$\to$] Nécessite une configuration automatique des tables des switchs
        \end{itemize}

    \item Avec les switchs, un host ne reçoit que les frames qui lui sont destinés
        (\textit{unicast, multicast et broadcast}) ainsi que les frames dont le
        switch ne connait pas la destination

    \item[$\color{red}\bullet$]  \textcolor{red}{Attaque par dénide service}
        Taille des tables MAC est limité, un host peut envoyer plein de frame avec
        des addresses sources aléatoire et overflow le switch.
        \begin{itemize}
            \item [$\to$] Le switch devra broadcast toutes les frames reçue, et l'attaquant
                recevra donc tout les frames..
        \end{itemize}
\end{itemize}


Si  on  combine le  \textbf{MAC  address  learning} et  l'algorithme  de
forwarding,  on  peut gérer les réseaux en arbre qui n'ont pas de boucle
(pas de TTL, ni HopLimit)

(\textit{Cependant, un arbre est dangereux:  en cas de problèmes avec un lien le
réseau est scindé en deux.})


\paragraph{Spanning Tree Protocol}
Permet de réduire un réseau à un \textbf{Spanning Tree}.
\begin{itemize}
    \item Les switches échangent les BPDU pour créer le spanning,
        Les BPDU sont envoyé avec \textsc{ALL\_BRIDGES} comme destination multicast
    \item Le plus petit 64bits identifier (48 lower avec MAC addresse, 16 higher
        permette à l'admin d'influencer le réseau) est élu \textbf{root}, et les branches
        sont composés avec les chemins les plus court permettant à tout les
        switchs du réseau d'être atteint
\end{itemize}

\subparagraph{BPU} (:= <R, c, T, P>) contient:

\begin{itemize}
    \item L'identifiant du switch racine (R)
    \item Le coût du chemin le plus court entre le switch qui a envoyé le BPDU et la racine (c)
    \item L'identifiant du switch qui a envoyé le BPDU (T)
    \item Le numéro du port du switch qui a envoyé le BPDU (p)
\end{itemize}

\subparagraph{Priority vector} $V[q] = < T, c + cost(q), T, p, q>$

\subparagraph{Port}
Chaque port d'un switch est \textbf{root, désigné} ou \textbf{bloqué}.
Un port est \textbf{root} est celui qui est le plus proche du switch racine!

\begin{table}[ht]
    \begin{center}
    \begin{tabular}{|c|c|c|c|}
        \hline
        Etat du port & Réception de BPDU & Envoi de BPDU & Gestion de trames de données \\
        \hline
        Bloqué & OUI & NON & NON \\
        Racine & OUI & NON & OUI \\
        Désigné & OUI & OUI & OUI \\
        \hline
    \end{tabular}
    \end{center}
    \caption{Transmission des BPDU selon l'état de chaque port}
\end{table}

\subparagraph{Fonctionnement}

\begin{itemize}
    \item Chaque switch ecoute les BPDU sur ses ports
    \item Pour les BPDU reçu, il calcule le \textbf{priority vector} associé au port d'où vient le BPDU et stocké le \textbf{meilleur pour chaque port}
    \item Switch root est connu en regardant le plus petit identifiant stocké dans la
        table des priority vector
        \begin{enumerate}
            \item Le switch root à $<R,O,R,p>$
            \item Les autres switch au $<R, c, S, p>$ connaissent leur root port (celui avec le meilleur
                priority vector)
        \end{enumerate}
    \item Un port est designé si le BPDU du switch est meilleur que le priority vector
        du port sinon il est bloqué.
    \item[$\to$] R: root switch, S: identifiant switch, c: cost best priority vector, p: number port d'où vient le BPDU
\end{itemize}

\subparagraph{Quand la version est stable}
Le switch racine envoie régulièrement son propre BPDU qui est reçu sur le port Root des switchs directement connectés à la racine.

\begin{itemize}
    \item Les switches écoutent tout de même sur le port Bloqué, mais si la topologie du réseau est stable, normalement aucun BPDU ne doit arriver dessus.
    \item Lors du calcul du Spanning Tree, \textbf{aucune donnée} n'est transmise dans le réseau.

        Après le spanning tree, l'algorithme de MAC learning se lance.
\end{itemize}


\subparagraph{Recover failure}
Les switches, ports et liens peuvent foirer dans un réseau Ethernet avec des switches: quand ça arrive, il faut refaire le spanning tree. Pour les détecter, on envoie des BPDU régulièrement (le BPDU contient deux autres champs: son âge et l'age maximum).
\begin{itemize}
    \item Age=0 pour le root et il est incrémenté de 1 dés qu'il passe par un switch
    \item Les switch stock age et l'incrémente chaque seconde. Si il dépasse le
        max age, il y a donc une erreur et on refait STP
\end{itemize}

\paragraph{Les Virtual LANs} sont un ensemble de ports dans un ou plusieurs switches. Un switch peut gérer plusieurs LANs séparément et applique l'algorithme de MAC learning sur chacun des VLANs sans partager leurs informations respectives.

\begin{figure}[ht]
    \begin{center}
      \includegraphics[width=9cm]{virtual_local_area_networks.png}
      \caption{Trois Virtual Area Networks connectés par des switches}
    \end{center}
\end{figure}


Pour permettre au switch de distinguer les VLAN on met un identifiant à chaque VLAN qu'on met dans chaque header de frame échangée. Ce header est inséré directement après l'addresse MAC dans la frame Ethernet (avant EtherType).
\begin{figure}[!ht]
    \begin{center}
    \includegraphics[width=14cm]{header_vlan.png}
    \caption{Header VLAN (802.1q)}
    \end{center}
\end{figure}

\begin{itemize}
    \item Le Tag Protocol mis à 0x8100 permet au receveur de détecter la présence de ce header.
    \item Le Priority Code Point (PCP) permet de définir des priorités dans les trames.
    \item Le champ C permet la compatibilité entre Ethernet et les réseaux Token Ring.
    \item Le dernier champ est l'identifiant de la VLAN (0 = pas un VLAN; 4095 est réservé).
\end{itemize}

\subsubsection{802.11 Wireless networks}

Le spectre radio est une ressource limitée partagée entre tous.
Il est reglementé par les instituions internationales et les gouvernements pour éviter les interférences. La seule exception de non-règlementation sont les ondes ISM (Industrial, Scientific and Medical). On utilise la bande 2400-2500Ghz pour le WiFi comme pour les micro-ondes, le bluetooth,... (il peut donc y avoir des interférences).

\paragraph{Standard 802.11}
IEEE a créé ce standard pour les familles de réseaux sans fil WIFI:

\begin{center}
\begin{tabular}{|c|c|c|c|c|}
\hline
Standard & Fréquence & Débit typique & Bande passante max & Portée (m) intérieur/extérieur \\
\hline
802.11 & 2.4 Ghz & 0.9 Mbps & 2 Mbps & 20/100 \\
802.11a & 5 Ghz & 23 Mbps & 54 Mbps & 35/120 \\
802.11b & 2.4 Ghz & 4.3 Mbps & 11 Mbps & 38/140 \\
802.11g & 2.4 Ghz & 19 Mbps & 54 Mbps & 38/140 \\
802.11n & 2.4-5 Ghz & 74 Mbps & 150 Mbps & 70/250 \\
\hline
\end{tabular}
\end{center}

Le wifi reprend CSMA/CA et la même architecture/frame que Ethernet.

L'architecture des réseaux WiFi est différente des LAN. Il y a deux types de réseaux WiFi:
\begin{itemize}
    \item indépendant/adhoc: Quand on ne le relie pas à Internet (imprimante wifi)
    \item infrastructure: contiennent un ou plusieurs points d'accès attachés à un LAN (souvent Ethernet) qui lui est connecté à Internet.
\end{itemize}

Le payload maximum théorique en 802.11 est de 2324 bytes mais le standard le limite à 1500.

\begin{figure}[!ht]
    \centering
    \begin{tabular}{cc}
        \includegraphics[width =6cm]{wifi_infrastructure.png} &
         \includegraphics[width=6cm]{wifi_access_point.png}\\
         Réseau infrastructure 802.11 & Point d'accès WiFi
     \end{tabular}
\end{figure}

\begin{figure}[!ht]
    \centering
    \includegraphics[width=12cm]{wifi_data_frame_format.png}
    \caption{Format des trames de données en 802.11}
\end{figure}

Format des frames en 802.11:
\begin{itemize}
    \item Frame Control: indique le type de trame (data, RTS/CTS/ ack, management frames,...), si la trame est envoyée depuis un LAN,...
    \item Duration: permet de réserver un temps de transmission (temps de transmission de l'ack + SIFS pour les data ; 0 pour multi/broadcast).
    \item Sequence Control: contient un numéro de séquence incrémenté pour chaque trame de données.
    \end{itemize}

On se rend compte qu'il y a 3 champs d'addresse:
\begin{enumerate}
    \item L'addresse MAC du point d'accès
    \item L'addresse MAC de la source WiFi
    \item L'addresse de destination final sur le LAN.
\end{enumerate}

\paragraph{Unreliable connectionless service}
Malgré l'utilisation d'acquittements, la couche 802.11 ne fournit qu'un service unreliable connectionless comme Ethernet. Les acquittements sont utilisés pour minimiser la probabilité de duplication de trame, ils n'assurent pas la livraison des données (haute probabilité mais pas garantie).

\paragraph{Encapsulation}
L'encapsulation d'IP sur 802.11 ajoute 6 bytes au header 802.11, 4 bytes pour LLC/SNAP et 2 bytes de Ethernet Type (IP ou ARP).

\begin{figure}[!ht]
\begin{center}
    \includegraphics[width=12cm]{ip_over_wifi.png}
  \caption{Paquet IP encapsulé dans une trame WiFi 802.11}
\end{center}
\end{figure}

\section{Algorithmes}
   \subsection{MAC Address Learning par les switches}
	  \lstset{keepspaces=true, keywordstyle=\color{red!70}, commentstyle=\color{blue!60}}
	  \begin{center}
	  \framebox{\begin{minipage}{0.9\linewidth}
	     \lstinputlisting[language=Python]{mac_address_learning.py}
	  \end{minipage}}
	  \end{center}

    \subsection{Réception d'un distance vector}
	  \lstset{keepspaces=true, morekeywords={isin, each}, keywordstyle=\color{red!70}, commentstyle=\color{blue!60}, breaklines=true}
	  \begin{center}
	  \framebox{\begin{minipage}{\linewidth}
	     \lstinputlisting[language=Python]{reception_of_distance_vector.py}
	  \end{minipage}}
	  \end{center}

    \subsection{Aloha}
	  \lstset{keepspaces=true, keywordstyle=\color{red!70}, commentstyle=\color{blue!60}}
	  \begin{center}
	  \framebox{\begin{minipage}{0.9\linewidth}
	     \lstinputlisting[language=Python]{aloha.py}
	  \end{minipage}}
	  \end{center}

    \subsection{CSMA basique}
	  \lstset{keepspaces=true, keywordstyle=\color{red!70}, commentstyle=\color{blue!60}}
	  \begin{center}
	  \framebox{\begin{minipage}{0.9\linewidth}
	     \lstinputlisting[language=Python]{basic_CSMA.py}
	  \end{minipage}}
	  \end{center}

    \subsection{CSMA/CD solution finale}
	  \lstset{keepspaces=true, keywordstyle=\color{red!70}, commentstyle=\color{blue!60}}
	  \begin{center}
	  \framebox{\begin{minipage}{0.9\linewidth}
	     \lstinputlisting[language=Python]{csma_cd_final_solution.py}
	  \end{minipage}}
	  \end{center}

    \subsection{CSMA/CA côté sender}
	  \lstset{keepspaces=true, keywordstyle=\color{red!70}, commentstyle=\color{blue!60}}
	  \begin{center}
	  \framebox{\begin{minipage}{0.9\linewidth}
	     \lstinputlisting[language=Python]{csma_ca_sender.py}
	  \end{minipage}}
	  \end{center}

    \subsection{Window-based congestion control}
	  \lstset{keepspaces=true, keywordstyle=\color{red!70}, commentstyle=\color{blue!60}}
	  \begin{center}
	  \framebox{\begin{minipage}{0.9\linewidth}
	     \lstinputlisting[language=Python]{window_based_congestion_control.py}
	  \end{minipage}}
	  \end{center}

    \subsection{Additive increase/multiplicative decrase in TCP}
	  \lstset{keepspaces=true, keywordstyle=\color{red!70}, commentstyle=\color{blue!60}}
	  \begin{center}
	  \framebox{\begin{minipage}{0.9\linewidth}
	     \lstinputlisting[language=Python]{AIMD_in_tcp.py}
	  \end{minipage}}
	  \end{center}

\biblio

\subsection{Connection et deconnection TCP}

\begin{figure}[ht]
    \centering
    \includegraphics[width=10cm]{cofsm.png}
    \caption{Connection TCP finite state machine}
\end{figure}

\begin{figure}[ht]
    \centering
    \includegraphics[width=10cm]{decofsm.png}
    \caption{Deconnection TCP finite state machine. Timeout is 2*MSL.}
\end{figure}


\end{document}
