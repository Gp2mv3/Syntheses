\documentclass[fr]{../../../../../../eplexam}

\hypertitle{Analyse Numérique}{5}{INMA}{1170}{2013}{Janvier}{Majeure}
{Beno\^it Legat}
{Paul Van Dooren}

\paragraph{Ressource utile}
\url{http://www.forum-epl.be/viewtopic.php?t=12324}

\section{(4 pts)}
Démontrez le théorème de Laguerre:
\textit{
  Soit le polynôme à coefficients complexes
  $p(z) = z^n + a_{n-1}z^{n-1} + \cdots + a_0$.
  Soit $z_0$ tel que $p(z_0)p'(z_0) \neq 0$.
  Alors, il existe au moins une racine du polynôme dans le disque
  fermé $z_0$ et de rayon $R = n\frac{|p(z_0)}{p'(z_0)}$.
}

Donnez la démonstration seulement dans le cas de zéros distincts

\begin{solution}
  Voir cours.
\end{solution}

\section{(4 pts)}
Démontrez le théorème suivant (variant des notes de cours):
\textit{
  Si $A$ est une matrice strictement diagonalement dominante,
  la méthode de Jacobi est convergente.
}
(Rappel: pour la méthode de Jacobi, $M = D$ et $N = -(L+U)$.)

\begin{solution}
  Il faut prouver que $|\rho(-D^{-1}(L+U))| < 1$.
  On a
  \[ \det(\lambda I + D^{-1}(L+U)) = 0 \]
  ce qui est équivalent, car $\det(D^{-1}) \neq 0$ vu qu'elle est inversible, à
  \[ \det(\lambda D + (L+U)) = 0. \]
  Soit $B = \lambda D + (L+U)$
  Si $|\lambda| \geq 1$ et que $A$ est diagonale dominante,
  $|\lambda b_{ii}| \geq |b_{ii}| = |a_{ii}| > \sum_{j \neq i} |a_{ij}| = \sum_{j \neq i} |b_{ij}|$,
  pareil pour les colonnes.
  $B$ est donc strictement diagonale dominante.

  Du coup, il est absurde de penser que $B$ pourrait avoir un déterminant nul car ça reviendrait
  à penser que $B$ pourrait être singulière et donc qu'il existerait un $x$ tel que
  $Bx = 0$.
  Soit $k$ tel que $|x_k| \geq |x_i|$ $\forall i \neq k$, on aurait alors
  \[ \sum_{i \neq k} b_{ki} x_i + b_{kk} x_k = 0 \]
  alors que
  \begin{align*}
    |\sum_{i \neq k} b_{ki} x_i + x_k b_{kk}| & \leq |\sum_{i \neq k} b_{ki} x_i| - |x_k b_{kk}|
    & \leq \sum_{i \neq k} |b_{ki}| |x_i| - |x_k| |b_{kk}|\\
    & \leq |x_k| \sum_{i \neq k} |b_{ki}| - |x_k| |b_{kk}|\\
    & < |x_k| |b_{kk}| - |x_k| |b_{kk}|\\
    & = 0.
  \end{align*}
\end{solution}

\section{(4 pts)}
Démontrer le théorème suivant:
\textit{
  Si $f: \Rn \to \Rn$ a un point fixe $s \in \Rn$ et si $f$ satisfait
  la condition de Lipschitz
  \begin{align*}
    \|f(x) - f(s)\| & \leq L\|x-s\|, &
    \forall x \in B_\epsilon(s) & := \{x \in \Rn : \|x - s\| < \epsilon\}
  \end{align*}
  alors tous les itérés de $x_{k+1} = f(x_k)$ appartiennent à $B_\epsilon(s)$
  si $x_0 \in B_\epsilon(s)$ et convergent exponentiellement vers $s$,
  qui est le seul point fixe de $f$ dans $B_\epsilon(s)$.
}

\begin{solution}
  Voir cours.
\end{solution}

\section{(4 pts)}
Donnez la méthode de l'itération inverse pour une matrice $A$
diagonalisable et avec valeurs propres $\lambda_j$,
quand on utilise un shift $\mu$ pour calculer le vecteur propre
correspondant à la valeur propre $\lambda_J$ la plus proche de $\mu$.
Discutez de sa convergence quand $|\lambda_j - \mu|$ est
strictement plus petit que tout autre $|\lambda_j - \mu|$, $j \neq J$.

\begin{solution}
  Voir cours.
\end{solution}

\section{(4 pts)}
Définissez le concept de région de stabilité d'une méthode numérique
d'une équation différentielle ordinaire et expliquez le concept de
stabilité absolue.

\begin{solution}
  Voir cours.
\end{solution}

Essayez de limiter vos réponses à 2 pages par question (soyez concis).

\end{document}
