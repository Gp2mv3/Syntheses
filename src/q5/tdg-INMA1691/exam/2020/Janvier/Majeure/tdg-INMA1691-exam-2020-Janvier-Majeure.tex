\documentclass[fr]{../../../../../../eplexam}

\hypertitle{Théorie des graphes}{5}{INMA}{1691}{2020}{Janvier}{Majeure}
{Miguel De Le Court}
{Jean-Charles Delvenne et Vincent Blondel}

\section{Question 1}
Les Pokémon sont des créatures tantôt attachantes, tantôt belliqueuses, qui vivent à l'état sauvage mais se laissent volontiers apprivoiser par les dresseurs talentueux.

Sacha est un jeune dresseur originaire de Bourg-Palette qui parcourt Kanto, Johto, Hoenn, etc. par monts et par vaux, en compagnie de ses Pokémons préférés. Quand il rencontre un Pokémon sauvage, ou appartenant à d'autres dresseurs, il espère le vaincre en lançant un de ses propres Pokémons à l'assaut. En cas de victoire il a la joie de pouvoir l'inscrire dans son carnet de bord, le Pokédex, mais on considère qu'il ne l'apprivoise pas: l'équipe de Pokémons de Sacha ne change pas au cours de l'aventure.

Différents facteurs influencent l'issue du combat: le degré d'évolution de chaque Pokémon, le lien qu'il entretient avec son dresseur, le talent de ce dernier, etc. et aussi le type: ainsi Pikachu (Pokémon de type électrique) a de quoi terroriser Carapuce (type eau), lequel fera boire la tasse à Sabelette (type sol), qui ne fera qu'une bouchée de Pikachu. Il n'y a donc pas un Pokémon qui domine les autres. Étant donné deux espèces de Pokémon distinctes, on peut toujours déterminer laquelle domine l'autre.

Sacha souhaite composer son équipe de façon à toujours pouvoir dominer, par un Pokémon de son équipe, tout Pokémon qui aurait l'audace de se trouver sur son chemin. Éventuellement il opposera un Pokémon identique, par exemple son cher Pikachu face à un Pikachu de grand chemin. Pour savoir comment composer son équipe, il fait appel au célèbre professeur Chen:

``\textbf{Première question}, Sacha. Démontre que parmi tout ensemble $S$ de $n$ espèces de Pokémons, il en existe au moins une qui domine au moins la moitié des $n-1$ autres.''

``\textbf{Deuxième question}, Sacha. Démontre que parmi tout ensemble $S$ de $n$ espèces de Pokémons, il existe un sous-ensemble $T$ dominant de $\lceil\log_{2}n\rceil$ espèces (c'est-à-dire que pour toute espèce $s\in S\backslash T$ il y a au moins une espèce $t\in T$ qui domine $s$).''

``\textbf{Troisième question}, Sacha. Tu sais que pas moins de 893 espèces de Pokémon sont connues à ce jour. Selon le résultat ci-dessus, combien de Pokémons te suffit-il d'emporter?''

``\textbf{Quatrième question}, Sacha, un peu plus difficile. Montre que tu peux lister ces 893 Pokémons dans un ordre tel que chaque Pokémon domine le suivant dans la liste. Tu peux par exemple procéder par récurrence.''

Pauvre Sacha! Il a bien besoin de votre aide.
\nosolution


\newpage
\section{Question 2}
2030, une épidémie mortelle ravage la planète. Afin de sauver l'espèce humaine, Elon Musk décide d'en envoyer quelques représentants sur la planète Mars.

Vu le nombre de places limitées pour une telle expédition, une sélection a lieu.

Il est pragmatiquement décidé que les compétences professionnelles suivantes devront chacune pouvoir être assumées par une personne au minimum: médecin (A), mathématicien appliqué spécialise en optimisation (B), mathématicien appliqué spécialisé en machine learning (C), mathématicien appliqué spécialisé en théorie des graphes (D), et fermier (E).

Mais aussi, pour conserver l'art, il est décidé de sélectionner au moins un humain par compétence artistique: écrivain (a), musicien (b), clown (c), fidget spinner (d) et dabbeur (e).

Supposez dans un premier temps qu'une liste de candidats avec leurs compétences professionnelles et artistiques nous est donnée.

\begin{enumerate}
	\item Modélisez le problème de choisir un représentant pour chaque compétence comme un problème de couplage. Décrivez-en le graphe (nœuds et arêtes). Travaillez avec l'hypothèse qu'un humain ne peut exercer effectivement qu'une seule compétence à la fois parmi les 10.

Par exemple Joséphine est compétente comme médecin, fermière et musicienne, elle peut être choisie pour exercer une et une seule de ses compétences, la médecine par exemple.

	\item Comment ce problème de couplage peut-il être réduit à un problème de flot maximal?
	\item Dans le problème précédent, on utilisait un humain par compétence, on aimerait savoir s'il est possible d'utiliser moitié moins d'humains que de compétences en autorisant qu'un humain remplisse à la fois une compétence professionnelle et artistique (mais pas deux compétences professionnelles ou deux compétences artistiques).

Modélisez ce problème comme un problème de flot maximal.

Indice: chaque humain choisi doit assumer une compétence de chaque type dans une solution valide.
	\item Construisez le graphe et donnez une solution avec 5 humains si elle existe pour ce problème avec le groupe d'humains suivant. S'il n'y a pas de solution donnez en une preuve (à l'aide d'une coupe par exemple).
	\begin{itemize}
		\item Arthur: A,B,D, a,e
		\item Bob: A,B, a,b,e
		\item Charlotte: B,C, a,e
		\item Didier: C, b,c,d,e
		\item Elliot: C,E, a,e
		\item Félicie: D, a,b
		\item Gauthier: E, a,e
	\end{itemize}
\end{enumerate}
\nosolution


\newpage
\section{Question 3}
Vrai ou faux? Justifiez par la théorie des graphes. Vous pouvez utiliser des résultats du cours sans les démontrer (mais en y faisant référence clairement).

\begin{enumerate}
	\item Alexandre le Grand conquit l'empire perse dans son entièreté, province après province. Une fois une province conquise il n'y revenait jamais personnellement et se déplaçait avec le gros de ses troupes vers une province voisine à conquérir. Son génie stratégique lui fit judicieusement stationner des phalanges aguerries dans les provinces de Cappadoce, Babylonie et Bactriane, car le reste de l'empire se découpait alors en cinq parties séparées: ainsi tout mouvement séditieux dans l'une des parties devait affronter les soldats d'Alexandre pour faire la jonction avec une autre partie. Il acheva sa vie dans la dernière province conquise, terrassé par les fièvres à l'âge de 32 ans.
	\item Il existe un graphe planaire simple à 20 nœuds, tous de degré 3.
	\item Il existe un graphe simple à 7 nœuds, dont les degrés sont 6, 6, 5, 4, 3, 3, 1.
	\item Il existe un graphe planaire simple à 10 nœuds, tous de degré 5.
	\item Dans tout graphe connexe de 51 nœuds, de 30 arêtes de poids 1 et de 30 arêtes de poids 2 (60 arêtes en tout), les poids de deux arbres sous-tendants ne diffèrent que de 15 \% ou moins.
	\item Il existe un graphe simple à 20 nœuds, tous de degrés 11 et sans triangle (clique à 3 nœuds).
	\item Un graphe à cent nœuds dont tous les nœuds sont de degré 5 admet un coloriage propre de nœuds à 5 couleurs ou moins.
	\item Le graphe biparti complet à $n+m$ nœuds, noté $K_{n,m}$, est hamiltonien si et seulement si
\[ n=m>1. \]
	\item Bonus: Le graphe biparti complet $K_{n,n}$ possède $n^{2(n-1)}$ arbres sous-tendants.
\end{enumerate}
\nosolution

\end{document}
