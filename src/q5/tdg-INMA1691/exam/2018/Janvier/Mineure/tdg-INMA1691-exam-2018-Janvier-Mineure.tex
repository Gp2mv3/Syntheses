\documentclass[fr]{../../../../../../eplexam}

\hypertitle{Théorie et algorithmique des graphes}{5}{INMA}{1691}{2018}{Janvier}{Mineure}
{Gilles Peiffer}
{Vincent Blondel et Jean-Charles Delvenne}

Veuillez expliquer vos raisonnements avec clarté,
répondre à des questions différentes sur des feuilles \emph{différentes},
sans oublier d'apposer \emph{votre nom} sur chacune.

\section{Question 1}
Deux alpinistes Henri et Brieuc sont de part et d'autre
d'une chaîne de montagnes,
à l'altitude zéro, et souhaitent échanger leurs positions,
avec la fantaisie suivante: à tout moment
ils souhaitent être l'un et l'autre à la même altitude.
On modélise la chaîne de montagnes dans le plan discret
comme une suite d'\og escaliers \fg{} montants ou descendants.
Chaque marche a une coordonnée horizontale $x \in 0, 1, \ldots, N-1$
et une altitude $z \in \{\,0, 1, 2, \ldots\,\}$
(l'altitude ne descend donc jamais en dessous de zéro).
Deux marches successives ont toujours
une différence d'altitude de $1$ exactement.

À chaque temps $t = 0, 1, 2, \ldots$,
chaque alpiniste avance ou recule de $1$,
et donc monte ou descend de $1$:
$\abs{x_{\textnormal{Henri}}(t+1) - x_{\textnormal{Henri}}(t)} = 1$
et $\abs{z_{\textnormal{Henri}}(t+1) - z_{\textnormal{Henri}}(t)} = 1$,
et de même pour Brieuc.
Ils sont initialement en $x=0$ et $x=N-1$ respectivement.

Considérez le graphe dont les n\oe{}uds sont les couples (paires ordonnées)
de marches à la même altitude.
Deux n\oe{}uds $(u_1, u_2)$ et $(v_1, v_2)$ sont reliés
s'il est possible pour Henri et Brieuc
sur les marches $u_1$ et $u_2$ respectivement
de passer en une étape aux marches $v_1$ et $v_2$.

\begin{enumerate}
	\item Démontrez que chaque n\oe{}ud de ce graphe
	a un degré de $0$, $1$, $2$ ou $4$,
	et qu'il y a exactement quatre n\oe{}uds de degré $1$.
	Quels sont ces n\oe{}uds?
	\item Démontrez que dans tout graphe
	avec exactement deux n\oe{}uds de degré impair,
	ces deux n\oe{}uds font partie de la même composante connexe.
	\item Démontrez que Henri et Brieuc
	peuvent bien échanger leurs positions
	en restant à tout moment à la même hauteur l'un que l'autre.
\end{enumerate}

\nosolution

\end{document}
