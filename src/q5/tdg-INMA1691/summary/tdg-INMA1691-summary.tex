\documentclass[fr]{../../../eplsummary}

\usepackage{graphicx}
\usepackage{float}
\usepackage{../../../eplmath}
\usepackage{stmaryrd}
\usepackage{centernot}
\usepackage[french,ruled,vlined]{algorithm2e}

\newcommand{\contradiction}{\hspace*{\fill}$\lightning$}
\DeclareMathOperator{\val}{\mathrm{val}}
\DeclareMathOperator{\capa}{\mathrm{cap}}

\SetKwComment{Comment}{$\triangleright$\ }{}

% TODO rewrite the proof of Euler's Theorem
% TODO proof of Prim's Algorithm's correctness
% TODO add graphs of isomorphism property for complementary graphs of isomorphic graphs
% TODO add tikz drawings
% TODO complete proof of Theorem 5.3.8 (Whitney) in D. West Intro to Graph Theory
% TODO add missing theorems and proofs, comment out unneeded ones
% TODO add some Ramsey theory (e.g. Erdos-Szekeres general case)
% TODO add proof of Turàn's Theorem and some other stuff about cliques and independent sets

\hypertitle{Mathématiques discrètes I: Théorie et algorithmique des graphes}{5}{INMA}{1691}
{Gilles Peiffer}
{Raphaël Jungers}[
\paragraph{Remarque} Ce document reprend les notes prises
au cours de l'année 2018-2019.
Lors de cette année, le Jean-Charles Delvenne a été supléé par Raphaël Jungers.
Il est donc possible que certaines parties soient
absentes ou incomplètes (ou superflues)
par rapport aux années précédentes.
]

\section{Concepts fondamentaux}
\subsection{Introduction}
De façon informelle,
on peut dire qu'un graphe
est \og un réseau de sommets reliés par des arêtes\fg{}.
Un exemple d'un graphe serait donc par exemple
celui à la \figuref{graph}.
\begin{figure}[H]
\centering
\begin{tikzpicture}
	\node[draw, circle] at (-1,2)  (1) {};
	\node[draw, circle] at ( 1,2)  (2) {};
	\node[draw, circle] at (-1,0)  (3) {};
	\node[draw, circle] at ( 1,0)  (4) {};

	\draw[-] (2) edge node[anchor = north] {} (3);
	\draw[-] (2) edge node[anchor = north east] {} (4);
\end{tikzpicture}
\caption{Un graphe à 4 sommets et 2 arêtes.}
\label{fig:graph}
\end{figure}

Pourquoi est-il intéressant d'étudier la théorie des graphes?
Grâce aux résultats de celle-ci,
énormément de problèmes semblant initialement compliqués
sont réduits à des problèmes relativement simples.
En effet, un acronyme intéressant,
très souvent valable en théorie des graphes,
est \textsc{TONCAS}:
\og \emph{The Obvious Necessary Condition is Also Sufficient}\fg{},
c'est-à-dire que si une condition paraît intuitivement nécessaire,
il y a de bonnes chances qu'elle soit également suffisante.

\subsection{Définitions}
\begin{mydef}[Graphe]
	Un \emph{graphe} $G$ est un triplet ordonné $(V, E, \phi)$, où:
	\begin{itemize}
	\item $V$ est un ensemble dont les éléments sont appelés sommets ou n\oe{}uds;
	\item $E$ est un ensemble dont les éléments sont appelés arêtes;
	\item $\phi$ est une fonction, dite fonction d'incidence,
	qui associe à chaque arête un sommet ou une \emph{paire} de sommets.
	\end{itemize}
\end{mydef}
Prenons de nouveau un exemple de graphe (\figuref{graph_2}).
\begin{figure}[H]
\centering
\begin{tikzpicture}
	\node[draw, circle] at ( 0, 2.82)  (u) {$u$};
	\node[draw, circle] at ( 0, 0  )  (x) {$x$};
	\node[draw, circle] at (-4, 0  )  (y) {$y$};
	\node[draw, circle] at ( 2,-2  )  (v) {$v$};
	\node[draw, circle] at (-2,-2  )  (w) {$w$};

	\draw (u) -- (v) node [midway, right] {$e_1$};
	\draw (u) -- (w) node [midway, right] {$e_2$};
	\draw (v) -- (w) node [midway, above] {$e_3$};
	\draw (x) -- (v) node [midway, above] {$e_4$};
	\draw (x) -- (w) node [midway, right] {$e_5$};
	\draw (x) -- (u) node [midway, left] {$e_6$};
	\draw (x) -- (y) node [midway, above] {$e_7$};
\end{tikzpicture}
\caption{Un graphe à 5 sommets et 7 arêtes.}
\label{fig:graph_2}
\end{figure}

Pour ce graphe, on a
\begin{align*}
	V &= \{\,u, v, w, x, y\,\}\,,\\
	E &= \{\,e_1, e_2, \dots, e_7\,\}\,,\\
	\phi(e_1) &= (u,v)\,,\\
	\phi(e_2) &= (u,w)\,,\\
	&\vdotswithin{=}\\
	\phi(e_7) &= (x,y)\,.\\
\end{align*}

\begin{mydef}[Degré]
	Le degré d'un sommet
	est le nombre d'arêtes incidentes à celui-ci.
\end{mydef}

\begin{myexem}[Sous-graphe]
	Un exemple de sous-graphe du graphe en \figuref{graph_2}
	est le graphe en \figuref{subgraph}.
	\begin{figure}[H]
	\centering
	\begin{tikzpicture}
		\node[draw, circle] at ( 0, 0  )  (x) {$x$};
		\node[draw, circle] at (-4, 0  )  (y) {$y$};
		\node[draw, circle] at ( 2,-2  )  (v) {$v$};
		\node[draw, circle] at (-2,-2  )  (w) {$w$};
		\draw (v) -- (w) node [midway, above] {$e_3$};
		\draw (x) -- (w) node [midway, right] {$e_5$};
		\draw (x) -- (y) node [midway, above] {$e_7$};
	\end{tikzpicture}
	\caption{Un graphe chemin à 4 sommets et 3 arêtes,
	sous-graphe de celui à 5 sommets et 7 arêtes en \figuref{graph_2}.}
	\label{fig:subgraph}
	\end{figure}
	En général, un \emph{sous-graphe} du graphe $(V, E, \phi)$
	est un graphe $(V', E', \phi')$ avec:
	\begin{itemize}
	\item $V' \subseteq V$;
	\item $E' \subseteq E$;
	\item $\phi'$ est la restriction de $\phi$ à $E'$.
	\end{itemize}
\end{myexem}

\begin{mydef}[Graphe simple]
	Un graphe simple est un graphe sans boucle ni arête multiple.
\end{mydef}

Il faut noter qu'un graphe est une notion abstraite,
indépendante de sa représentation.
Le graphe $G$ aurait tout aussi bien pu être noté sous la forme
\begin{figure}[H]
\centering
\begin{tikzpicture}
	\node[draw, circle] at ( 2, 2)  (u) {$b$};
	\node[draw, circle] at ( 2,-2)  (x) {$d$};
	\node[draw, circle] at ( 4, 0)  (y) {$k$};
	\node[draw, circle] at (-2, 2)  (v) {$a$};
	\node[draw, circle] at (-2,-2)  (w) {$c$};

	\draw (u) -- (v) node [midway, above] {$e_1$};
	\draw (u) -- (w) node [near end, right] {$e_2$};
	\draw (v) -- (w) node [midway, left] {$e_3$};
	\draw (x) -- (v) node [near start, left] {$e_4$};
	\draw (x) -- (w) node [midway, above] {$e_5$};
	\draw (x) -- (u) node [midway, left] {$e_6$};
	\draw (x) -- (y) node [midway, left] {$e_7$};
\end{tikzpicture}
\caption{Le graphe de la \figuref{graph_2} représenté différemment.}
\label{fig:graph_2_iso}
\end{figure}

L'\emph{isomorphisme} est
\[
\begin{array}{cc}
	y & k \\
	x & d \\
	w & c \\
	u & b \\
	v & a
\end{array}
\]

\subsection{Isomorphisme de graphes}
\begin{mydef}[Isomorphisme de graphes]
	Deux graphes $(V, E, \phi)$ et $(V', E', \phi')$
	sont dits \emph{isomorphes} s'il existe
	des bijections $f \colon V \to V'$ et $g \colon E \to E'$ telles que:
	\[
	\phi(e) = (u,v) \iff \phi(g(e)) = \big(f(u),f(v)\big)\,.
	\]
	Deux graphes sont isomorphes
	s'il y a une bijection entre les n\oe{}uds et les arêtes.
\end{mydef}

\begin{mypropo}[La relation d'isomorphisme est une relation d'équivalence]
	Toute relation d'équivalence satisfait à 3 conditions:
	\begin{itemize}
		\item elle est réflexive;
		\item elle est symétrique;
		\item elle est transitive.
	\end{itemize}
	Prouvons que la relation d'isomorphisme de graphes
	est une relation d'équivalence.
\end{mypropo}
\begin{proof}
	Commençons par prouver la réflexivité.
	Cela revient à dire qu'il y a un isomorphisme
	entre le graphe et lui-même.
	Pour prouver cela,
	il suffit de remarquer
	qu'un isomorphisme satisfaisant cette propriété
	est la permutation identité,
	car elle satisfait $G_1 \cong G_2$.
	\bigbreak
	Prouvons maintenant la symétrie de l'isomorphisme.
	Soit $f \colon V(G_1) \to V(G_2)$,
	alors $f^{-1}$ est un isomorphisme de $G_2 \to G_1$.
	En effet, on sait que
	\[
	uv \in E(G_1) \iff f(u)f(v) \in E(G_2)\,.
	\]
	On sait alors que $\forall x,y \in V(G_2)$
	\[
	f^{-1}(x)f^{-1}(y) \in E(G_1) \iff xy \in E(G_2)\,.
	\]
	\bigbreak
	Finalement, prouvons la transitivité.
	Supposons que $f \colon V(F) \to V(G)$
	et $g \colon V(G) \to V(H)$.
	On a alors
	\begin{align*}
		uv \in E(F) &\iff f(u)f(v) \in E(G)\\
		\intertext{et}
		f(u)f(v) \in E(G) &\iff g\big(f(u)\big)g\big(f(v)\big) \in E(H)\,.
	\end{align*}
	Cela implique que $g \circ f$
	est une bijection,
	ce qui implique à son tour $F \cong H$.
	\bigbreak
	En combinant ces trois résultats,
	on remarque donc que l'isomorphisme de graphes
	est une relation d'équivalence.
\end{proof}

\begin{mypropo}[$G \cong H \iff \widebar{G} \cong \widebar{H}$]
	La preuve est immédiate en observant
	que toute non-arête de $G$ sera également
	une non-arête de $H$.
	Ce petit truc permet parfois
	de voir relativement facilement
	si deux graphes sont isomorphes,
	bien qu'en général,
	cela reste une tâche compliquée.
	% TODO add graphs
\end{mypropo}

\subsection{Théorème d'Euler}
Un des nombreux théorèmes dus à Leonhard Euler
est l'observation qu'un graphe connexe est eulérien
si et seulement si le degré de tous ses sommets est pair.
\begin{mytheo}[Théorème d'Euler]
	Un graphe connexe est eulérien
	si et seulement si le degré de chacun de ses sommets
	est un nombre pair.
\end{mytheo}
\begin{proof}
	Démontrons l'implication directe d'abord.
	Considérons le problème eulérien
	où en tout n\oe{}ud il y a
	autant d'arêtes entrantes que sortantes.
	(Il s'agit de la condition pour qu'un graphe soit eulérien.)
	Le degré du n\oe{}ud est
	la somme de ce nombre d'arêtes entrantes et sortantes.
	Il sera donc égal à $2n$,
	où $n$ est le nombre d'arêtes entrantes,
	ce qui est toujours un nombre pair.
	\bigbreak
	L'implication indirecte se démontre par construction,
	et se comprend intuitivement par construction.\footnote{C'est vague comme preuve,
	mais il y en a des plus complètes sur Internet,
	notamment dans le syllabus en ligne.}
\end{proof}

\subsection{Représentations d'un graphe}
Soit le graphe $G$ de la \figuref{mat}.
\begin{figure}[H]
	\centering
	\begin{tikzpicture}
	\node[draw, circle] at ( 0, 4)  (1)  {$1$};
	\node[draw, circle] at ( 2, 0)  (2)  {$2$};
	\node[draw, circle] at (-2, 0)  (3)  {$3$};

	\draw[-] (1) edge [loop above] node {$a$} (1);
	\draw[-] (1) edge [bend left] node [midway,right] {$b$} (2);
	\draw[-] (1) edge [bend right] node [midway,left] {$c$} (2);
	\draw[-] (2) edge node [midway,below] {$d$} (3);
	\draw[-] (1) edge node [midway,left] {$e$} (3);
	\end{tikzpicture}
	\caption{Un graphe.}
	\label{fig:mat}
\end{figure}
On peut représenter ce graphe de deux façons distinctes:
\begin{itemize}
	\item sa \emph{matrice d'incidence} et
	\item sa \emph{matrice d'adjacence}.
\end{itemize}
\begin{mydef}[Matrice d'incidence d'un graphe non orienté]
	La matrice d'incidence d'un graphe non orienté
	est la matrice rectangulaire $M$ de taille $n \times m$
	dont l'élément $m_{ij}$ est
	\begin{itemize}
		\item $1$ si le sommet $v_i$ est une extrémité de l'arête $e_j$;
		\item $2$ si l'arête $x_{j}$ est une boucle sur $v_{i}$;
		\item $0$ sinon.
	\end{itemize}
\end{mydef}
\begin{mydef}[Matrice d'adjacence]
	La matrice d'adjacence d'un graphe simple $G$ à $n$ sommets
	est la matrice $n \times n$ booléenne $A$
	dont l'élément $a_{ij}$ est
	\begin{itemize}
		\item $1$ si $v_i v_j \in E(G)$;
		\item $0$ sinon.
	\end{itemize}
\end{mydef}
\bigbreak
Pour le graphe $G$ de la \figuref{mat},
on a donc les représentations suivantes:
\[
A = \bordermatrix{  & 1 & 2 & 3 \cr
                  1 & 1 & 2 & 1 \cr
                  2 & 2 & 0 & 1 \cr
                  3 & 1 & 1 & 0 \cr}\,,
\quad
M = \bordermatrix{  & a & b & c & d & e\cr
                  1 & 2 & 1 & 1 & 0 & 1\cr
                  2 & 0 & 1 & 1 & 1 & 0\cr
                  3 & 0 & 0 & 0 & 1 & 1\cr}\,.
\]

\begin{mytheo}[Théorème des poignées de main]
	La somme des degrés des n\oe{}uds d'un graphe
	est deux fois le nombre d'arêtes.
	Mathématiquement,
	\[
	\sum_{v_i \in V} d(v_i) = 2 \abs{E}\,.
	\]
\end{mytheo}
\begin{proof}
	\begin{align*}
	\sum_{i}\sum_{j} M_{ij} &= \sum_{v_i \in V} d(v_i)\\
	&= \sum_j 2\\
	&= 2 \abs{E}\,.\qedhere
	\end{align*}
\end{proof}

\begin{mytheo}[Matrice d'adjacence et nombre de parcours]
	Soit $A$ la matrice d'adjacence d'un graphe.
	Alors l'élément $a^k_{ij}$ de $A^k$ ($k \ge 0$)
	est le nombre de parcours de longueur $k$
	de $v_i$ vers $v_j$.
\end{mytheo}
\begin{proof}
	Par induction.
	\begin{description}
		\item[Cas de base.]
		Pour $k = 1$, c'est vrai
		par la définition de la matrice d'adjacence.
		\item[Cas inductif.]
		Supposons la propriété vraie pour $k$.
		Soit
		\begin{itemize}
			\item $i$ et $j$ deux n\oe{}uds,
			\item $\pi^{k+1}_{ij}$ le nombre de parcours
			de $i$ vers $j$ de longueur $k+1$,
			\item  $\epsilon_{ij}$ le nombre d'arêtes
			de $i$ vers $j$ et
			\item $a^k_{ij}$ l'entrée en ligne $i$
			et colonne $j$ de $A^k$.
		\end{itemize}
		On a alors
		\begin{align*}
		\pi^{k+1}_{ij} &= \sum_{l \in V} \pi_{il}^{k} \epsilon_{lj}\\
		&= \sum_{l \in V} a^k_{il} a_{lj}\\
		&= a^{k+1}_{ij}\,.\qedhere
		\end{align*}
	\end{description}
\end{proof}

\begin{myexem}[Suites de de Bruijn]
\label{exem:code}
Un autre exemple d'utilisation de la matrice d'adjacence
est qu'elle permet par exemple de répondre à la question
\og Combien de mots binaires de longueur $k$ ne contiennent pas ``$111$''?\fg{}\,.

Prenons le graphe du problème (\figuref{code}),
où chaque n\oe{}ud est la terminaison actuelle du mot,
et où chaque arête indique la terminaison que l'on aurait
en rajoutant soit un $0$ soit un $1$.
\begin{figure}[H]
	\centering
	\begin{tikzpicture}
	\node[draw, circle] at (-2, -2)  (10)  {$10$};
	\node[draw, circle] at (-2,  2)  (00)  {$00$};
	\node[draw, circle] at ( 2,  2)  (01)  {$01$};
	\node[draw, circle] at ( 2, -2)  (11)  {$11$};

	\begin{scope}[every node/.style={fill=white,circle}]
		\path[->] (00) edge [loop above] node {$000$} (00);
		\path[->] (11) edge [loop below,red] node {$111$} (11);
		\path[->] (00) edge node {$001$} (01);
		\path[->] (10) edge node {$100$} (00);
		\path[->] (10) edge [bend left] node {$101$} (01);
		\path[->] (01) edge [bend left] node {$001$} (10);
		\path[->] (01) edge node {$011$} (11);
		\path[->] (11) edge node {$001$} (10);
	\end{scope}
	\end{tikzpicture}
	\caption{Graphe de l'Exemple~\ref{exem:code}.
	La seule arête qu'on ne compte pas est celle en rouge,
	car c'est ce cas-là qu'on veut éviter.}
	\label{fig:code}
\end{figure}
On construit la matrice d'adjacence (en ne comptant pas l'arête rouge):
\[
A = \begin{pmatrix}
1 & 1 & 0 & 0\\
0 & 0 & 1 & 1\\
1 & 1 & 0 & 0\\
0 & 0 & 1 & 0
\end{pmatrix}\,.
\]
Maintenant, en calculant les valeurs propres de la matrice $A$,
et en choisissant la plus grande ($\rho \approx 1.84$ en l'occurrence),
on peut trouver la capacité du code.
Pour une longueur $k$,
le nombre de mots ne contenant pas la séquence $111$ est $\approx \rho^k$.
\end{myexem}

\subsection{Graphe biparti}
\begin{mytheo}[Graphe biparti]
\label{theo:bipartite}
	Un graphe est biparti si et seulement si
	tous ses cycles sont de longueur paire.
\end{mytheo}
\begin{proof}
	\noindent
	\newline
	$\boxed{\implies}$
	\newline
	Soit un graphe $G$ biparti et un cycle $C = v_0, v_1, v_2, \dots, v_0$.
	Supposons $v_0 \in V_0$, $v_1 \in V_1$, $v_2 \in V_0$,\dots
	Tout cycle termine en $V_0$ et contient donc
	un nombre pair d'arêtes.

	\noindent
	\newline
	$\boxed{\impliedby}$
	\newline
	Soit $v_0$ un n\oe{}ud arbitraire.
	On définit
	\begin{align*}
		V_0 &= \set{v \suchthat d(v,v_0) \textnormal{ est pair}}\\
		V_1 &= \set{v \suchthat d(v,v_0) \textnormal{ est impair}}\,.
	\end{align*}
	Par contradiction.
	Supposons sans perte de généralité $\exists u,v \in V_0$.
	Soient $p,q$ les plus courts chemins
	de $v_0$ à $u$ et $v$ respectivement.
	\begin{figure}[H]
	\centering
	\begin{tikzpicture}
	\begin{scope}[every node/.style={fill=white,circle}]
		\node[draw, circle] at (-4, 0)  (v0)  {$v_0$};
		\node[draw, circle] at (-1, 0)  (w)  {$w$};
		\node[draw, circle] at ( 2, 1)  (u)  {$u$};
		\node[draw, circle] at ( 2,-1)  (v)  {$v$};

		\draw[-] (v0) edge [bend left]  node {$q_1$} (w);
		\draw[-] (v0) edge [bend right] node {$p_1$} (w);
		\draw[-] (w) edge [bend left]  node {$p_2$} (u);
		\draw[-] (w) edge [bend right] node {$q_2$} (v);
		\draw[-] (u) edge node {$e$} (v);
	\end{scope}
	\end{tikzpicture}
	\caption{Le graphe pour la preuve du Théorème~\ref{theo:bipartite}.}
	\label{fig:bipartite}
	\end{figure}
	Soit $w$ le dernier n\oe{}ud dans $p \cap q$.
	Notre \emph{claim} est que $p_2 e q_2$ est un cycle pair.
	Cependant,
	\begin{align*}
	\abs{p_1 p_2 e q_2 q_1} &= \abs{p_1} + \abs{p_2 e q_2} + \abs{q_1}\\
	d(v_0,u) + d(v_0,v) + 1 &= 2 \abs{p_1} + \abs{p_2 e q_2}\,.
	\end{align*}
	Comme le coté de gauche de cette équation est impair
	($u$ et $v$ sont à une distance paire de $v_0$ car ils sont dans $V_0$),
	et que $2 \abs{p_1}$ est évidemment pair,
	le cycle $p_2 e q_2$ doit être impair.
	C'est donc une contradiction,
	ce qui termine la preuve.
\end{proof}

\begin{mytheo}[Théorème des amis et des étrangers]
	Dans toute fête de six personnes,
	soit au moins trois d'entre eux sont (par paires)
	mutuellement des étrangers,
	soit au moins trois d'entre eux sont (par paires)
	mutuellement des amis\footnote{Ce théorème est un cas particulier
	du Théorème de Ramsey, qui a donné lieu
	à une nouvelle branche des mathématiques,
	la théorie de Ramsey.}.
\end{mytheo}
\begin{proof}
	Si on attribue à chaque personne un sommet dans un graphe,
	et qu'on trace une arête rouge
	si les deux sommets sont \og amis\fg{},
	et une arête bleue si ils sont \og inconnus\fg{},
	on peut raisonner sur ce problème
	comme un problème de théorie des graphes.

	Choisissons un sommet quelconque; appelons-le $P$.
	Il y a cinq arêtes issues de $P$.
	Elles sont de couleur rouge ou bleu.
	Le principe des tiroirs dit
	qu'au moins trois d'entre elles doivent être
	de la même couleur,
	car s'il y a moins de trois d'une couleur,
	disons rouge, alors il y en a au moins trois qui sont bleues.

	Soient $A$, $B$, $C$ les autres extrémités
	de ces trois arêtes,
	toutes de la même couleur, disons bleu.
	Si parmi $AB$, $BC$, $CA$, l'une est bleue,
	alors cette arête,
	avec les deux arêtes joignant $P$
	aux extrémités de cette arête forme un triangle bleu.
	Si aucune parmi $AB$, $BC$, $CA$ n'est bleue,
	alors les trois arêtes sont rouges
	et nous avons un triangle rouge, à savoir, $ABC$.
\end{proof}

\section{Arbres et distance}
\subsection{Propriétés de base}
\begin{mytheo}
	Le plus court parcours entre deux n\oe{}uds
	est toujours un chemin.
\end{mytheo}
\begin{proof}
	Soit $d'$ la longueur du plus court chemin
	et $d$ la longueur du plus court parcours $P$.
	Comme tout chemin est un parcours,
	on a nécessairement $d' \ge d$.

	Supposons, par l'absurde, $d' > d$.
	$P$ n'est donc pas un chemin et contient un ou plusieurs cycles.
	\[
	p = u \ldots \underbrace{v_i \ldots v_i}_{\textnormal{cycle $C$}} \ldots v\,.
	\]
	Si on retire le cycle $C$,
	on obtient le parcours $P' = u \ldots v_i \ldots v$.
	Comme on considère tous les poids positifs,
	on a deux cas possibles:
	\begin{enumerate}
		\item Le cycle $C$ possède au moins une arête
		de poids $w_i > 0$
		$\implies P'$ est plus court que $P$.
		\hspace*{\fill} $\lightning$
		\item Le cycle $C$ est de poids nul
		$\implies P'$ et $P$ ont la même longueur.
		On a donc deux sous-cas:
		\begin{enumerate}[label=(\roman*)]
			\item Il n'y a plus de cycle dans $P'$
			$\implies P'$ est un chemin
			de longueur $d = d'$.
			\hspace*{\fill} $\lightning$
			\item Il reste des cycles dans $P'$.
			On recommence l'argument
			en remplaçant $P$ par $P'$.
			On finira éventuellement sur une contradiction
			car le nombre de cycles dans $P$ est fini.
		\end{enumerate}
	\end{enumerate}
\end{proof}

\subsubsection{Algorithme de Dijkstra}
\begin{myexem}
	On cherche les distances à partir de $a$.
	\begin{figure}[H]
	\centering
	\begin{tikzpicture}
		\node[draw, circle] at (0,  2) (a) {$a$};
		\node[draw, circle] at (2,  1) (b) {$b$};
		\node[draw, circle] at (1, -2) (c) {$c$};
		\node[draw, circle] at (-1,-2) (d) {$d$};
		\node[draw, circle] at (-2, 1) (e) {$e$};
		\begin{scope}[every node/.style={fill=white,circle}]
			\draw[->] (a) edge node {$50$} (b);
			\draw[->] (c) edge node {$5$} (b);
			\draw[->] (c) edge node {$50$} (d);
			\draw[->] (e) edge node {$10$} (d);
			\draw[->] (a) edge node {$10$} (e);
			\draw[->] (d) edge node {$5$} (a);
			\draw[->] (a) edge node {$30$} (c);
			\draw[->] (d) edge node[near start] {$20$} (b);
		\end{scope}
	\end{tikzpicture}
	\caption{Un exemple de digraphe pondéré.}
	\label{fig:dijkstra_graph}
	\end{figure}

	Le graphe de la \figuref{dijkstra_graph} est défini par
	\begin{align*}
	V &= \{\,a,b,c,d,e\,\}\,,\\
	E &= \{\,ab, cb, cd, ed, ae, da, ac\,\}\,.
	\end{align*}

	\begin{table}[H]
		\centering
		\begin{tabular}{ccccccc}
			\hline
			$u'$ & $S$ & $\ell(a)$ & $\ell(b)$ & $\ell(c)$ & $\ell(d)$ & $\ell(e)$\\
			\hline
			$a$ & $\lbrace a \rbrace$ & $\boxed{0}$ & $\infty$ & $\infty$ & $\infty$ &$\infty$\\
			$e$ & $\lbrace a,e \rbrace$ && 50 & 30 & $\infty$ & $\boxed{10}$\\
			$d$ & $\lbrace a,e,d \rbrace$ && 50 & 30 & $\boxed{20}$ &\\
			$c$ & $\lbrace a,e,d,c \rbrace$ && 40 & $\boxed{30}$ & &\\
			$b$ & $\lbrace a,e,d,c,b \rbrace$ && $\boxed{35}$ & & &\\
			\hline
		\end{tabular}
		\caption{Résultats de l'Algorithme de Dijkstra
		pour le graphe de la \figuref{dijkstra_graph}.}
		\label{tab:dijkstra}
	\end{table}
\end{myexem}

\begin{myrem}
	Les dernières arêtes utilisées pour joindre un sommet
	forment les plus courts chemins.
\end{myrem}

En pseudocode,
l'Algorithme de Dijkstra peut être implémenté comme suit:

\begin{algorithm}[H]
\DontPrintSemicolon
\KwData{$G$, le graphe, et $s$, le n\oe{}ud source.}
\KwResult{Les plus courts chemins de $s$ vers tout autre n\oe{}ud du graphe.}
\Begin{
	crée l'ensemble des n\oe{}uds $\mathcal{Q}$\;
	dist[$s$] $\gets$ 0 \Comment*[r]{Initialisation}

	\ForEach{vertex $v$ in $G$}{
		dist[$v$] $\gets \infty$ \Comment*[r]{Distance inconnue de $s$ à $v$}
		prev[$v$] $\gets$ UNDEFINED \Comment*[r]{Prédécesseur de $v$ sur chemin optimal partant de $s$}
		ajoute $v$ à $\mathcal{Q}$ \Comment*[r]{Tous les n\oe{}uds sont initialement dans $\mathcal{Q}$}
	}
	dist[$s$] $\gets 0$\;
	\While{$\mathcal{Q}$ n'est pas vide}{
		$u \gets$ n\oe{}ud dans $\mathcal{Q}$ avec $\min$ dist[$u$] \Comment*[r]{Enlève et retourne le n\oe{}ud avec la distance minimale dans $\mathcal{Q}$}
		enlève $u$ de $\mathcal{Q}$\;
		\ForEach{voisin $v$ de $u$}{
			alt $\gets$ dist[$u$] $+$ length($u$, $v$)\;
			\If{alt $<$ dist[$v$]}{
				dist[$v$] $\gets$ alt\;
				prev[$v$] $\gets u$\;
			}
		}
	}
	\Return dist[], prev[]\;
}
\caption{Dijkstra($G$, $s$)\label{algo:dijkstra}}
\end{algorithm}

L'Algorithme~\ref{algo:dijkstra} a une complexité de $\bigoh(\abs{V}^2)$.
Il existe aussi des implémentations
avec des structures de données différentes\footnote{Les min-priority queues.},
permettant de réduire la complexité à $\bigoh(\abs{E} + \abs{V} \log \abs{V})$.

\begin{mytheo}[Correction de l'Algorithme de Dijkstra]
	L'Algorithme de Dijkstra est correct et efficace.
\end{mytheo}
\begin{proof}
Par induction.
On suppose les propriétés vérifiées à l'itération $i-1$
et on veut les prouver pour l'itération $i$.
Soit $S_i$ l'ensemble $S$ à l'itération $i$.
\noindent
\newline
\fbox{Cas de base ($i = 1$)}
\newline
À la fin de l'initialisation,
\begin{align}
S &= \{\,u_0\,\}\,, \nonumber\\
\ell(u_0) &= 0 = d(u_0,u_0)\,. \label{eq:dijkstra1}
\end{align}
Après la première mise à jour de $\ell$,
\begin{align}
\ell(v) &= \min(\ell(v), \underbrace{\ell(v_0)}_{0} {} + w(u_0 v))\,, \quad \forall v \in S \nonumber\\
\implies \ell(v) &= \infty \textnormal{ s'il n'y a pas d'arête } v_0v. \label{eq:dijkstra2}
\end{align}
\noindent
\newline
\fbox{Cas inductif}
\newline
$u_i$ est le n\oe{}ud qui vient d'être rajouté à l'itération précédente.
Il faut prouver qu'après avoir rajouté $u_i$ dans $S_i$,
$\ell(u_i) = d(u_0, u_i)$.

La condition~\eqref{eq:dijkstra1} est verifiée
par les autres éléments de $S_i$
par l'hypothèse d'induction.
Supposons par l'absurde que l'algorithme
n'ait pas trouvé le plus court chemin vers $u_i$
et que donc $\ell(u_i) > d(u_0, u_i)$.

Par la condition~\eqref{eq:dijkstra2} appliquée à l'itération $i-1$,
$\ell(u_i)$ est la longueur du plus court chemin
où tous les n\oe{}uds internes sont dans $S_{i-1}$.

Si $\ell(u_i) \ne d(u_0, u_i)$,
alors il y a un autre plus court chemin
qui passe par un n\oe{}ud de $\widebar{S_{i-1}}$.
Soit $v$ le premier n\oe{}ud de $\widebar{S_{i-1}}$ de ce chemin.
Comme $u \ldots v \ldots u_i$ est un plus court chemin,
$u_0 \ldots v$ en est aussi un.
On sait que $\ell(v) = d(u_0, v)$
car tous les n\oe{}uds internes de $u_0 \ldots v$
sont uniquement dans $S_{i-1}$
(par \eqref{eq:dijkstra2} à l'étape $i-1$).
De plus, on sait que les poids sont $\ge 0$.
\[
\implies \ell(v) = d(u_0, v) \le d(u_0, u_i)\,.
\]

Mais $u_i$ est tel que
\[
\ell(u_i) = \min\limits_{u \in \widebar{S_{i-1}}}(\ell(u))\,.
\]
On a donc $\ell(u_i) \le \ell(v) \le d(u_0, u_i)$.
\contradiction

Commençons par montrer que
\[
\ell(v) \ge d(u_0, v)\,, \quad \forall v \in S\,.
\]
Les seuls $\ell(v)$ qui changent à cette itération
sont égaux à
\[
\ell(v) = \underbrace{\ell(u_i)}_{d(u_0, u_i)} {} + w(u_i v)\,,
\]
qui ont la longueur du chemin $u_0 \ldots u_i v$.
Donc, $\ell(v) \ge d(u_0, v)$.
Montrons que $\ell(v)$ est la longueur du plus court chemin
dont tous les n\oe{}uds internes sont dans $S$.

Par \eqref{eq:dijkstra2} à l'étape $i-1$,
$\ell(v)\,,\ \forall v \notin S_{i-1}$
est longueur du plus court chemin de $u_0$ à $v$
dont tous les n\oe{}uds internes appartiennent à $S_{i-1}$.

Il faut juste considérer les chemins passant par $u_i$.
Le plus court de ces chemins est le chemin fait
du plus court chemin de $u_0$ à $u_i$ et de l'arête $u_i v$,
de longueur $d(u_0, u_i) + w(u_i v)$.
Le plus court chemin vers $v$ dont les n\oe{}uds internes
sont dans $S_i$ a donc soit
ses n\oe{}uds internes uniquement dans $S_{i-1}$ (longueur $\ell(v)$),
soit il passe par $u_i$ (longueur $\ell(u_i) + w(u_i v)$).
Cela correspond à la mise à jour de $\ell(v)$.
\[
\ell(v) = \min(\ell(v), \ell(u_i) + w(u_i v))\,.\qedhere
\]
\end{proof}

\begin{mypropo}[Complexité de l'Algorithme de Dijkstra] \leavevmode
	L'Algorithme de Dijkstra est en $\bigoh(\abs{V}^2)$
\end{mypropo}
\begin{proof}\leavevmode
	\begin{itemize}
		\item Après l'initialisation,
		$\abs{\widebar{S}} = \abs{V}-1$
		et à chaque passage de boucle décroit de $1$
		$\implies \abs{V}-1$ itérations.
		\item À chaque itération,
		l'opération \og $\forall v \notin S$\fg{} est
		$\bigoh(\abs{V} - \abs{S})$, tout comme
		l'opération \og trouver $v_{\textnormal{min}}$\fg{}.
		\item Le nombre total d'opérations est donc
		$2 (\abs{V} - 1) + 2 (\abs{V} - 2) + \cdots + 2
		\in \bigoh(\abs{V}^2)$.
	\end{itemize}
	L'algorithme est polynomial
	et donc \emph{relativement} efficace.
\end{proof}

\subsubsection{Anneaux et semi-anneaux}

\begin{mydef}[Semi-anneau]
	Un \emph{semi-anneau} est un ensemble
	muni de deux opérations ($\oplus,\otimes$)
	et ses propriétés.
\end{mydef}

\begin{mydef}[Anneau]
	Un \emph{anneau} est un ensemble
	muni de trois opérations ($\oplus,\otimes,\ominus$)
	et ses propriétés.
\end{mydef}

\begin{mydef}[Corps]
	Un \emph{corps} est un ensemble
	muni de quatre opérations ($\oplus,\otimes,\ominus,\oslash$)
	et ses propriétés.
\end{mydef}

\begin{myform}[Autre formule pour le plus court chemin]
	En prenant $A$ tel que $a_{ij} = w(i j)$
	et $w(i j) = \infty$ s'il n'y a pas d'arête $ij$.

	On a
	\begin{align*}
		A \oplus B &= \Big(a_{ik} \oplus b_{kj}\Big)_{ij}\,,\\
		A \otimes B &= \Big(\sum_k a_{ik} \otimes b_{kj}\Big)_{ij}\,.
	\end{align*}
	Le $\oplus$ définit la somme matricielle
	et le $\otimes$ définit le produit matriciel.
	Prenons $\oplus = \min$ et $\otimes = +$.
	Le neutre du minimum est $+\infty$
	et le neutre de l'addition est $0$.
	Définissons donc
	\[
	I =
	\begin{pmatrix}
	0 & +\infty & \cdots & +\infty\\
	+\infty & \ddots & \ddots & \vdots\\
	\vdots & \ddots & \ddots & +\infty\\
	+\infty & \cdots & +\infty & 0
	\end{pmatrix}\,.
	\]

	On définit alors
	\begin{align*}
		A \otimes A &= \Big(\min\limits_{k}\big(a_{ik} + a_{kj}\big)\Big)_{ij}\\
		&= \min\limits_{k}\big(w(i k) + w(k j)\big)\\
		&= \textnormal{plus court chemin de deux arêtes.}
	\end{align*}
	et en général, la matrice des plus courts chemins est donnée par
	\[
	(I \oplus A)^{\otimes n}\,.
	\]
	Pour calculer le plus court chemin,
	on peut définir alors l'itération:
	\begin{align*}
	M_0 &= I\,,\\
	M_1 &= I \oplus (M_0 \otimes A)\,,\\
	M_2 &= I \oplus (M_1 \otimes A)\,,\\
	&\vdotswithin{=}\\
	M_{k+1} &= I \oplus (M_k \otimes A)\,.
	\end{align*}
	On a fini de converger quand $M_k = M_{k+1}$.
	L'Algorithme de Dijkstra peut être vu
	comme une manière efficace d'implémenter cette itération.
\end{myform}

\subsection{Arbres sous-tendants et énumération}
\begin{mydef}[Arbre]
	Un \emph{arbre} est un graphe connexe sans cycle.
	Une \emph{forêt} est un graphe sans cycle.
\end{mydef}
\begin{mydef}[Sous-graphe sous-tendant]
	Un \emph{sous-graphe sous-tendant} d'un graphe $G$
	est un sous-graphe qui contient tous les sommets de $G$.
\end{mydef}

\begin{mytheo}
	\label{theo:spanning_tree}
	Tout graphe connexe contient un arbre sous-tendant.
\end{mytheo}
\begin{proof}
Soit $G$ un graphe connexe.
Parmi tous les sous-graphes sous-tendants connexes de $G$,
on choisit un sous-graphe $G'$,
minimal pour l'inclusion dans cet ensemble.
Pour que $G'$ soit un arbre,
il faut qu'il soit connexe et acyclique.
\begin{itemize}
	\item Il est connexe par construction.
	\item Supposons que $G'$ ait un cycle.
	Soit $e = uv$ une arête du cycle.
	Si on enlève $e$,
	le graphe est toujours connexe.
	Pour tout parcours $x \ldots uv \ldots y$,
	on peut remplacer $e$ par le reste du cycle.
	\contradiction
\end{itemize}
\end{proof}

\begin{mylem}[Feuilles d'un arbre]
	\label{lem:leaves}
	Tout arbre avec au moins deux sommets
	a aux moins deux feuilles (n\oe{}uds de degré un)
	et supprimer une feuille d'un arbre à $n$
	n\oe{}uds produit un arbre à $n-1$ n\oe{}uds.
\end{mylem}
\begin{proof}
	Dans un graphe acyclique avec au moins deux sommets,
	les extrémités d'un chemin maximal n'ont pas d'autres voisins
	que leurs voisins sur le chemin.
	Ces extrémités sont donc des feuilles.

	Soit $v$ une feuille d'un arbre $G = (V, E, \phi)$.
	Soit $G' = (V', E', \phi') = G - v$.
	Comme un sommet de degré un
	n'appartient à aucun chemin reliant deux autres sommets,
	pour tout $u,v \in V'$,
	tout chemin de $u$ à $v$ dans $G$ est aussi dans $G'$.
	$G'$ est donc connexe.

	Puisque supprimer un sommet ne peut pas créer de cycle,
	$G'$ est également acyclique.
\end{proof}

\subsubsection{Caractérisations des arbres}
\begin{mytheo}[Caractérisations des arbres]
	Soit $G$ un graphe à $n$ sommets et $m$ arêtes.
	Alors les conditions suivantes sont équivalentes.
	\begin{enumerate}[label=(\arabic*)]
		\item \label{conn_acy} $G$ est connexe et sans cycle.
		\item \label{acy_mn1} $G$ est sans cycle et $m = n - 1$.
		\item \label{conn_mn1} $G$ est connexe et $m = n-1$.
		\item \label{conn_del}$G$ est connexe
		et supprimer une arête quelconque déconnecte $G$.
		\item \label{acy_add} $G$ est sans cycle
		et ajouter une arête quelconque crée un et un seul cycle.
		\item \label{link} Deux n\oe{}uds de $G$ sont toujours reliés
		par un seul chemin.
	\end{enumerate}
	La dernière condition implique que $G$ est sans boucle
	(pour deux n\oe{}uds identiques).
\end{mytheo}
\begin{proof}
	\noindent
	\newline
	\newline
	\fbox{\ref{conn_acy} $\implies$ \ref{acy_mn1}}
	\newline
	Par récurrence sur $n$.
	Pour $n = 1$, $m = 0$.
	On a donc bien $m = n - 1$.

	On suppose \ref{acy_mn1} vrai pour $n-1$.
	Pour tout arbre de $n$ n\oe{}uds,
	on trouve une feuille $x$.
	On enlève $x$ et son arête
	et on obtient par le Lemme~\ref{lem:leaves}
	un arbre de $n-1$ n\oe{}uds
	et $n-2$ arêtes (par la récurrence).
	Tout arbre de $n$ n\oe{}uds a donc $n-1$ arêtes.

	\noindent
	\newline
	\fbox{\ref{acy_mn1} $\implies$ \ref{conn_mn1}}
	\newline
	Soient $G_1 = (V_1, E_1, \phi_1),
	\ldots, G_k = (V_k, E_k, \phi_k)$
	les $k$ composantes connexes de $G = (V, E, \phi)$
	avec $n_i = \abs{V_i}$ et $m_i = \abs{E_i}$
	pour $i = 1, \ldots, k$.

	On a
	\[
	\sum_{i=1}^{k} n_i = n
	\quad \textnormal{et} \quad
	\sum_{i=1}^{k} m_i = m\,.
	\]
	Chaque composante satisfait \ref{conn_acy},
	et donc $m_i = n_i - 1$ pour $i = 1, \ldots, k$.
	\begin{align*}
		\implies m &= \sum_{i=1}^{k} m_i \\
		&= \sum_{i=1}^{k} \big(n_i - 1\big) \\
		&= n - k\,.
	\end{align*}
	Or $m = n - 1$,
	ce qui implique $k = 1$.
	Le graphe est donc connexe.

	\noindent
	\newline
	\fbox{\ref{conn_mn1} $\implies$ \ref{conn_del}}
	\newline
	Supposons qu'en enlevant une arête $e$ à $G$,
	le graphe $G' = G - e$ reste connexe.
	Par le Théorème~\ref{theo:spanning_tree},
	$G'$ a un arbre sous-tendant $A$.
	Par \ref{conn_acy} $\implies$ \ref{acy_mn1},
	$A$ a $n - 1$ arêtes.
	$G'$ a au moins $n - 1$ arêtes et $G$ a au moins $n$ arêtes.
	Or, $G$ a $n - 1$ arêtes,
	ce qui implique que $G - e$ n'est pas connexe.
	\contradiction

	\noindent
	\newline
	\fbox{\ref{conn_del} $\implies$ \ref{acy_add}}
	\newline
	Supposons qu'il y ait un cycle dans $G$.
	On peut alors supprimer une arête de ce cycle,
	en gardant la propriété de convexité,
	ce qui contredit l'hypothèse.
	$G$ est donc acyclique.
	\contradiction

	Ajouter une arête quelconque crée un et un seul cycle.
	\begin{itemize}
		\item On montre qu'ajouter une arête $e$
		entre $u$ et $v$ crée un cycle.
		Par connexité,
		il existe un chemin $C$ entre $u$ et $v$ dans $G$.
		On a $u \ldots v e u$ qui est un cycle.
		\item C'est le seul cycle.
		Supposons qu'on ait obtenu au moins deux cycles
		$C_1$ et $C_2$ en ajoutant $e$.
		Si c'est le cas,
		on peut adjoindre $C_1 - e$ et $\widebar{C_2 - e}$
		pour former un parcours fermé $u \ldots v \ldots u$.
		\contradiction
	\end{itemize}

	\noindent
	\newline
	\fbox{\ref{acy_add} $\implies$ \ref{link}}
	\newline
	Supposons qu'il existe deux chemins
	$P_1 = u \ldots v$ et $P_2 = u \ldots v$,
	alors $u \ldots v \ldots u$ est un parcours fermé
	dont on peut extraire un cycle.
	Il y a donc au plus un chemin entre $u$ et $v$.
	\contradiction

	Prouvons maintenant l'existence de ce chemin.
	Si on ajoute une arête entre $u$ et $v$,
	alors par hypothèse on crée un cycle $C$.
	Cela implique que $C - e$ est un chemin entre $u$ et $v$.

	\noindent
	\newline
	\fbox{\ref{link} $\implies$ \ref{conn_acy}}
	\newline
	$G$ est d'office connexe.
	Il suffit donc de montrer qu'il est également acyclique.
	Supposons qu'il existe un cycle.
	Soient $x$ et $y$ deux n\oe{}uds de ce cycle.
	Le cycle donne deux chemins différents de $x$ à $y$.
	$G$ est donc sans cycle.
	\contradiction

\end{proof}

\subsubsection{Énumération des arbres sous-tendants}
\begin{myform}[Formule de Cayley]
	Soit $\tau(G)$ le nombre d’arbres sous-tendants de $G$
	et $e$ une arête quelconque de $G$
	qui n’est pas une boucle.
	Alors
	\[
	\tau(G) = \tau(G - e) + \tau(G \cdot e)\,.
	\]
	$G - e$ désigne le graphe avec l'arête $e$ enlevée.
	$G \cdot e$ désigne le graphe
	avec les n\oe{}uds reliés par $e$ fusionnés.
\end{myform}
\begin{proof}
	On divise les arbres sous-tendants de $G$ en deux catégories:
	\begin{enumerate}[label=(\alph*)]
		\item \label{subtract} ceux qui ne contiennent pas $e$;
		\item \label{contract} ceux qui contiennent $e$.
	\end{enumerate}
	On compte les arbres de chaque catégorie.
	\begin{itemize}
		\item Les arbres de \ref{subtract} sont en bijection
		avec les arbres sous-tendants de $G - e$.
		\item Les arbres de \ref{contract} sont en bijection
		avec les arbres sous-tendants de $G \cdot e$.
	\end{itemize}
	On a donc
	\[
	\tau(G) = \tau(G - e) + \tau(G \cdot e)\,.\qedhere
	\]
\end{proof}

\begin{mytheo}[Arbres sous-tendants de $K_n$]
	Le \emph{graphe complet à $n$ n\oe{}uds}, noté $K_n$,
	a $n^{n-2}$ arbres sous-tendants.
	C'est également un théorème de Cayley.
\end{mytheo}

\subsubsection{Arbre sous-tendant de poids minimum}
On a un algorithme naïf tournant en $\bigoh(n^{n-2})$ pour $n$ n\oe{}uds,
mais on peut faire beaucoup mieux.

\paragraph{Algorithme de Kruskal}
Le premier algorithme que l'on présente est celui de Kruskal.

\begin{algorithm}[H]
\DontPrintSemicolon
\KwData{$G$, une graphe pondéré à $n$ n\oe{}uds avec poids $\ge 0$.}
\KwResult{Le graphe formé des arêtes $T$
est un arbre sous-tendant de poids minimum.}
\Begin{
	sort($E$)\Comment*[r]{On trie les arêtes par poids croissant}
	$T \gets \emptyset$\;

	\While{$\abs{T} < n - 1$}{
		$e \gets$ l'arête pas encore considérée de moindre poids\;
		\If{$T \cup \{\,e\,\}$ acyclique}{
			$T \gets T \cup \{\,e\,\}$\;
		}
	}
	\Return $T$\;
}
\caption{Kruskal($G$)\label{algo:kruskal}}
\end{algorithm}

\begin{mytheo}[Correction de Kruskal]
	L'Algorithme de Kruskal est correct et efficace.
\end{mytheo}
\begin{proof}
	Soit $T = \{\,e_1, e_2, \ldots, e_m\,\}$
	l'ensemble des arêtes de l'arbre trouvé par Kruskal,
	avec les arêtes dans cet ordre.
	C'est bien un arbre sous-tendant
	car il a $n - 1$ arêtes et pas de cycle.

	On suppose que $T$ n'est pas optimal
	et que $T^*$ soit un arbre optimal.
	Soit $e_i$ la première arête choisie pour $T$
	mais absente dans $T^*$.

	$T^*$ est un arbre optimal
	qui contient $e_1, e_2, \ldots, e_{i-1}$ mais pas $e_i$.
	Il y a donc un seul cycle dans $T^* + e_i$.
	Il y a forcément une arête de ce cycle qui n'est pas dans $T$,
	sinon $T$ aurait un cycle.
	On l'appelle $e'$.

	Montrons que le sous-graphe
	avec les arêtes $e_1, e_2, \ldots, e_{i-1}, e'$
	n'a pas de cycle puisqu'il est inclus dans $T^*$.
	Au moment de choisir $e_i$,
	l'algorithme n'a pas choisi $e'$,
	qui pourtant ne crée pas de cycle (donc $w(e') \ge w(e_i)$).

	On considère donc l'arbre $T^* + e_i - e'$.
	C'est un arbre sous-tendant puisqu'il a $n - 1$ arêtes
	et pas de cycles.
	$T^* + e_i - e'$ est au moins aussi léger que $T^*$
	et contient $\{\,e_1, e_2, \ldots, e_{i-1}, e_i\,\}$.
	\contradiction

\end{proof}

C'est un algorithme glouton (\emph{greedy}),
c'est-à-dire qu'à chaque itération,
il vise l'optimum local.
\begin{mypropo}[Complexité de l'Algorithme de Kruskal]
	L'Algorithme de Kruskal requiert un temps de calcul
	de l'ordre de $\bigoh(m \log m)$ sur un graphe à $m$ arêtes.
	C'est le tri qui prend le plus de temps.
	Une implémentation efficace de la détection de cycles utilise
	une structure de données appelée \emph{ensembles disjoints}.
\end{mypropo}

\paragraph{Algorithme de Prim}
Un autre algorithme est celui de Prim.

\begin{algorithm}[H]
\DontPrintSemicolon
\KwData{$G$, une graphe pondéré à $n$ n\oe{}uds avec poids $\ge 0$.}
\KwResult{Le graphe formé des arêtes $T$
est un arbre sous-tendant de poids minimum.}
\Begin{
	$T \gets v$\Comment*[r]{$v$ est un n\oe{}ud de départ arbitraire}

	\While{$\abs{T} < n - 1$}{
		$e \gets$ l'arête de moindre poids incidente à exactement un n\oe{}ud de $T$\;
		$T \gets T \cup \{\,e\,\}$\;
	}
	\Return $T$\;
}
\caption{Prim($G$)\label{algo:prim}}
\end{algorithm}

\begin{mytheo}[Correction de Prim]
	L'Algorithme de Prim est correct et efficace.
	% TODO proof
\end{mytheo}

\begin{mypropo}[Complexité de l'Algorithme de Prim]
	L'Algorithme de Prim requiert un temps de calcul
	de l'ordre de $\bigoh(m + n \log n)$
	sur un graphe à $n$ n\oe{}uds et $m$ arêtes.
	Une implémentation efficace de la sélection d'arêtes utilise
	une structure de données appelée \emph{file de priorité}
	implémentée par un \og tas de Fibonacci\fg{}.
\end{mypropo}

\section{Couplages}
\subsection{Couplages et couvertures}
\subsubsection{Couplages maximums}
\begin{mydef}[Couplage]
	Un \emph{couplage} dans un graphe $G$ est un ensemble d'arêtes $M$
	tel que $M$ ne contient pas de boucles et
	que deux arêtes de $M$ n'ont jamais d'extrémité en commun.
	Les n\oe{}uds incidents aux arêtes d'un couplage $M$ sont dits
	\emph{saturés} par $M$; les autres sont \emph{insaturés}.
	Un \emph{couplage parfait} dans un graphe
	est un couplage qui sature tous les n\oe{}uds.
\end{mydef}
\begin{mydef}[Couplage maximal]
	Un \emph{couplage maximal} dans un graphe est un couplage
	qui ne peut pas être agrandi en ajoutant une arête.
	Un \emph{couplage maximum} est un couplage de taille maximum
	parmi tous les couplages du graphe.
\end{mydef}
\begin{myrem}
	Tout couplage maximum est maximal, mais pas inversement.
	Un couplage parfait, s'il existe, est un couplage maximum.
\end{myrem}
\begin{mydef}[Chemin $M$-alterné]
	Soit un couplage $M$.
	Un \emph{chemin $M$-alterné} est un chemin
	qui alterne entre les arêtes de $M$ et les arêtes hors $M$.
	Un chemin $M$-alterné dont les n\oe{}uds extrêmes
	sont insaturés par $M$
	est un \emph{chemin $M$-augmenté}.
\end{mydef}
\begin{mydef}[Différence symétrique]
	Si $G$ et $H$ sont des graphes avec un ensemble de n\oe{}uds $V$,
	alors la \emph{différence symétrique} $G \triangle H$
	est le graphe avec $V$ pour ensemble de n\oe{}uds
	dont les arêtes sont celles qui appartiennet soit à $G$, soit à $H$,
	mais pas aux deux.
	On utilise également cette définition pour les ensembles d'arêtes;
	en particulier, si $M$ et $M'$ sont des couplages,
	alors $M \triangle M' = (M \setminus M') \cup (M' \setminus M)$.
\end{mydef}
\begin{mylem}
	\label{lem:symdif}
	Toute composante de la différence symétrique
	de deux couplages est un chemin ou un cycle pair.
\end{mylem}
\begin{proof}
	Soient $M$ et $M'$ des couplages, et $F = M \triangle M'$.
	Comme $M$ et $M'$ sont des couplages,
	chaque n\oe{}ud a au plus une arête incidente appartenant à chaque.
	$F$ a donc au plus deux arêtes à chaque n\oe{}ud.
	Comme $\Delta(F) \le 2$,
	toute composante de $F$ est un chemin, ou un cycle.
	De plus, tout chemin ou cycle dans $F$ alterne entre
	les arêtes de $M \setminus M'$ et celles de $M' \setminus M$.
	Tout cycle est donc de longueur paire,
	avec un nombre d'arêtes dans $M$ égal au nombre d'arêtes dans $M'$.
\end{proof}
\begin{mytheo}[Berge, 1957]
	Un couplage $M$ dans un graphe $G$ est maximum dans $G$
	si et seulement si $G$ n'a pas de chemin $M$-augmenté.
\end{mytheo}
\begin{proof}
	On prouve la contraposée dans chaque sens:
	$G$ a un couplage plus grand que $M$ si et seulement si
	$G$ a un chemin $M$-augmenté.
	\noindent
	\newline
	\fbox{$\implies$}
	\newline
	Soit un chemin $M$-augmenté $P$.
	On peut remplacer les arêtes de $M$ dans $P$
	par les autres de $P$
	pour obtenir un nouveau couplage $M'$ avec une arête en plus.
	Lorsque $M$ est un couplage maximum,
	on n'a donc pas de chemin $M$-augmenté.
	\noindent
	\newline
	\fbox{$\impliedby$}
	\newline
	Soit $M'$ un couplage dans $G$ plus grand que $M$.
	On construit un chemin $M$-augmenté.
	Soit $F = M \triangle M'$.
	Par le Lemme~\ref{lem:symdif},
	$F$ consiste de chemins et de cycles pairs;
	les cycles pairs ont le même nombre d'arêtes dans $M$ que dans $M'$.
	Comme $\abs{M'} > \abs{M}$,
	$F$ doit avoir une composante avec plus d'arêtes dans $M'$
	que dans $M$.
	Une telle composante ne peut être qu'un chemin
	qui commence et termine avec une arête de $M'$.
	C'est donc un chemin $M$-augmenté dans $G$.
\end{proof}

\subsubsection{Condition de couplage de Hall}
Soit un graphe biparti avec bipartition $X, Y$.
Si un couplage $M$ sature $X$, alors pour tout $S \subseteq X$ il doit y avoir
au moins $\abs{S}$ n\oe{}uds ayant des voisins dans $S$,
parce que les n\oe{}uds couplés à $S$ doivent être choisis dans cet ensemble.
\begin{mynota}[Voisinage ouvert]
	On utilise $N(S)$ ou $N_G(S)$ pour dénoter
	les n\oe{}uds ayant un voisin dans l'ensemble $S$.
\end{mynota}
La condition nécessaire
\og Pour tout $S \subseteq X$, $\abs{N(S)} \ge \abs{S}$ \fg{}
est la \emph{Condition de Hall}.
Hall a prouvé que cette condition évidemment nécessaire
est également suffisante (\textsc{toncas}).
\begin{mytheo}[Théorème de Hall---P. Hall, 1935]
	\label{theo:hall}
	Un graphe biparti avec bipartition $X, Y$ a un couplage qui sature $X$
	si et seulement si $\abs{N(S)} \ge \abs{S}$ pour tout $S \subseteq X$.
\end{mytheo}
\begin{proof}
	La nécessité est triviale, comme dit ci-dessus:
	les $\abs{S}$ n\oe{}uds couplés à $S$ doivent appartenir à $N(S)$.
	Pour la suffisance,
	on prouve la contraposée.
	Si $M$ est un couplage maximum dans $G$,
	et $M$ ne sature pas $X$,
	alors on peut trouver un ensemble $S \subseteq X$
	tel que $\abs{N(S)} < \abs{S}$.
	Soit $u \in X$ un n\oe{}ud insaturé par $M$.
	Parmi tous les n\oe{}uds atteignables à partir de $u$
	au moyen de chemins $M$-alternés dans $G$,
	soit $S$ l'ensemble de ceux-ci dans $X$,
	et $T$ l'ensemble de ceux dans $Y$.
	On note que $u \in S$.

	On prétend que $M$ couple $T$ avec $S \setminus \{\,u\,\}$.
	Les chemins $M$-alternés à partir de $u$ atteignent $Y$
	le long des arêtes n'appartenant pas à $M$ et
	retournent vers $X$ le long des arêtes dans $M$.
	Ainsi, tout n\oe{}ud de $S \setminus \{\,u\,\}$
	est atteint par une arête dans $M$ à partir d'un n\oe{}ud dans $T$.
	Comme il n'y a pas de chemin $M$-augmenté,
	tout n\oe{}ud de $T$ est saturé;
	un chemin $M$-alterné atteignant $y \in T$
	s'étend par $M$ à un n\oe{}ud de $S$.
	Les arêtes de $M$ donnent alors une bijection
	de $T$ à $S \setminus \{\,u\,\}$,
	et on trouve $\abs{T} = \abs{S \setminus \{\,u\,\}}$.

	Le couplage entre $T$ et $S \setminus \{\,u\,\}$
	donne $T \subseteq N(S)$.
	En fait, on a même $T = N(S)$.
	Supposons que $y \in Y \setminus T$ ait un voisin $v \in S$.
	L'arête $vy$ ne peut être dans $M$,
	car $u$ est insaturé et que le reste de $S$ est couplé à $T$ par $M$.
	Ajouter $vy$ à un chemin $M$-alterné atteignant $v$
	donne un chemin $M$-alterné vers $y$.
	Ceci contredit $y \notin T$,
	et $vy$ ne peut donc pas exister.

	Avec $T = N(S)$,
	on a prouvé que $\abs{N(S)} = \abs{T} = \abs{S} - 1 < \abs{S}$,
	pour ce choix de $S$,
	ce qui termine la preuve de la contraposée.
\end{proof}
Lorsque les deux ensembles des la bipartition ont la même taille,
le Théorème de Hall est le Théorème du Mariage.

\begin{mydef}[Graphe $k$-régulier]
	Un graphe est dit \emph{$k$-régulier}
	si tous les n\oe{}uds sont de degré $k$.
\end{mydef}
\begin{mycorr}
	\label{corr:hall}
	Pour $k > 0$, tout graphe biparti $k$-régulier
	possède un couplage parfait.
\end{mycorr}
\begin{proof}
	Soit $G$ un graphe biparti $k$-régulier de bipartition $X, Y$.
	En comptant les arêtes par extrémités dans $X$ et dans $Y$,
	on trouve que $k \abs{X} = k \abs{Y}$, et donc $\abs{X} = \abs{Y}$.
	Il suffit alors de vérifier la Condition de Hall;
	un couplage qui sature $X$ saturera également $Y$ et est parfait.

	Considérons $S \subseteq X$.
	Soit $m$ le nombre d'arêtes de $S$ à $N(S)$.
	Comme $G$ est $k$-régulier, $m = k \abs{S}$.
	Ces $m$ arêtes sont incidentes à $N(S)$,
	donc $m \le k \abs{N(S)}$.
	On trouve donc $k \abs{S} \le k \abs{N(S)}$,
	et donc $\abs{S} \le \abs{N(S)}$ lorsque $k > 0$.
	Ayant choisi $S \subseteq X$ arbitrairement,
	on a établi la Condition de Hall.
\end{proof}

\subsubsection{Théorèmes min-max}
\begin{mydef}[Couverture de sommets]
	Une \emph{couverture de sommets} d'un graphe $G$
	est un ensemble $Q \subseteq V(G)$
	qui contient au moins une extrémité de chaque arête.
	Les n\oe{}uds dans $Q$ couvrent $E(G)$.
\end{mydef}
\begin{mytheo}[König, 1931; Egerváry, 1931]
	Si $G$ est un graphe biparti,
	alors la taille maximum d'un couplage dans $G$
	est égale à la taille minimum d'une couverture des arêtes de $G$.
\end{mytheo}
\begin{proof}
	Soit $G$ un graphe biparti de bipartition $X, Y$.
	Comme des n\oe{}uds distincts doivent être utilisés
	pour couvrir les arêtes d'un couplage,
	$\abs{Q} < \abs{M}$ quand $Q$ est une couverture des n\oe{}uds
	et $M$ est un couplage dans $G$.
	Étant donné une couverture minimale $Q$ de $G$,
	on construit un couplage de taille $\abs{Q}$
	pour montrer que l'égalité peut toujours être atteinte.

	On partitionne $Q$ en disant $R = Q \cap X$
	et $T = Q \cap Y$.
	Soient $H$ et $H'$ les sous-graphes de $G$
	induits par $R \cup (Y \setminus T)$
	et $T \cup (X \setminus R)$, respectivement.
	On utilise le Théorème de Hall afin de montrer
	que $H$ a un couplage qui sature $R$ vers $Y \setminus T$
	et $H'$ a un couplage qui sature $T$.
	Comme $H$ et $H'$ sont disjoints,
	les deux couplages ensemble
	forment un couplage de taille $\abs{Q}$ dans $G$.

	Comme $R \cup T$ est une couverture d'arêtes,
	$G$ n'a pas d'arête de $Y \setminus T$ vers $X \setminus R$.
	Pour tout $S \subseteq R$,
	on considère $N_H(S)$, qui est contenu dans $Y \setminus T$.
	Si $\abs{N_H(S)} < \abs{S}$,
	alors on peut substituer $N_H(S)$ pour $S$ dans $Q$
	pour obtenir une couverture de sommets plus petite,
	car $N_H(S)$ couvre toutes les arêtes incidentes à $S$
	qui ne sont pas couvertes par $T$.

	La minimalitee de $Q$ nous donne donc la Condition de Hall dans $H$,
	et donc $H$ a un couplage qui sature $R$.
	En appliquant le même argument à $H'$,
	on trouve un couplage qui sature $T$.
\end{proof}

\begin{myrem}
	Une \emph{relation min-max} est un théorème établissant l'égalité
	entre les réponses à un problème de minimisation
	et un problème de maximisation sur une classe d'instances.
	Le Théorème de König-Egerváry est une telle relation
	pour la couverture de sommets et le couplage dans les graphes bipartis.
\end{myrem}

\subsection{Algorithmes et applications}
\subsubsection{Algorithme hongrois}
\begin{algorithm}[H]
\DontPrintSemicolon
\KwData{Une graphe biparti $G$ de bipartition $X, Y$.}
\KwResult{Un couplage maximum dans $G$.}
\Begin{
	$M \gets$ couplage initial arbitraire\;
	$U \gets \{\,u \in X\,,\ \textnormal{u non incident à $M$}\,\}$\;
	\While{$U$ non vide}{
		$C \gets$ chemin $M$-augmenté à partir de $u \in U$\;
		\eIf{$C$ existe}{
			$M \gets M \triangle C$\;
		}{
			\Return $M$\;
		}
	}
	\Return $M$\;
}
\caption{Algorithme hongrois---Kuhn, 1955; Munkres, 1957 \label{algo:hung}}
\end{algorithm}

L'étape de mise à jour $M \leftarrow M \triangle C$
consiste à inverser les arêtes qui font partie du couplage
le long du chemin $M$-augmenté $C$ qui a été trouvé.

\begin{myprop}
	L'algorithme hongrois est de complexité $\bigoh(\abs{V}\abs{E})$.
	On peut arriver à une complexité $\bigoh\big(\sqrt{\abs{V}}\abs{E}\big)$
	en utilisant une stratégie plus intelligente
	pour trouver les chemins augmentés (Algorithme de Hopcroft-Karp (1973)).
\end{myprop}

\begin{myprop}
	L'algorithme hongrois peut être généralisé au problème pondéré:
	trouver le couplage de poids maximum.
	Pour les graphes non bipartis et non pondérés,
	l'Algorithme d'Edmonds (1965) permet de trouver le couplage maximum
	en $\bigoh(\abs{V}^2\abs{E})$.
	Micali et Vazirani (1980) ont diminué cette complexité
	en $\bigoh\big(\sqrt{\abs{V}}\abs{E}\big)$.
\end{myprop}

\section{Connectivité et chemins}
\subsection{Coupes et connectivité}
\subsubsection{Connectivité}
\begin{mydef}
	Une \emph{coupe de sommets} d'un graphe $G$
	est un ensemble $S \subseteq V(G)$ tel que
	$G \setminus S$ ait plus d'une composante.
	La \emph{connectivité} de $G$,
	notée $\kappa(G)$,
	est la taille minimum d'un ensemble de n\oe{}uds $S$
	tel que $G \setminus S$ soit déconnecté ou ait un seul n\oe{}ud.
	Un graphe $G$ est \emph{$k$-connexe} si
	sa connectivité est au moins $k$.
\end{mydef}

\begin{myexem}[Graphes de Harary]
	Étant donné $k < n$,
	on place $n$ n\oe{}uds autout d'un cercle, espacés uniformément.
	On suppose $k$ pair
	(pour $k$ impair, la construction est plus difficile),
	et on forme $H_{k, n}$ en rendant chaque n\oe{}ud adjacent
	aux $k/2$ n\oe{}uds les plus proches
	dans chaque direction autour du cercle.
\end{myexem}
\begin{mytheo}[Harary, 1962]
	$\kappa(H_{k, n}) = k$, et donc le nombre minimal d'arêtes
	dans un graphe $k$-connexe à $n$ n\oe{}uds est $\lceil kn/2 \rceil$.
\end{mytheo}
\begin{proof}
	Comme le graphe $G$ est $k$-connexe,
	le degré minimal dans le graphe, $\delta(G)$,
	est plus grand ou égal à $k$.
	La somme des degrés est donc plus grande ou égale à $kn$,
	ce qui implique que le nombre d'arêtes est plus grand ou égal à $kn/2$.
	Cette borne est trivialement atteinte dans les graphes de Harary.
\end{proof}

\subsubsection{Arête-connectivité}
\begin{mydef}
	Une \emph{coupe d'arêtes}\footnote{En fait,
	il s'agit d'un \emph{ensemble déconnectant},
	et une \emph{coupe d'arêtes} est un ensemble déconnectant minimal.}
	est un ensemble $F \subseteq E(G)$
	tel que $G \setminus F$ a plus d'une composante.
	Un graphe est \emph{$k$-arête-connexe}
	si toute coupe d'arêtes a au moins $k$ arêtes.
	L'\emph{arête-connectivité} de $G$,
	notée $\kappa'(G)$,
	est la taille minimale d'une coupe d'arêtes.
\end{mydef}

\begin{mytheo}[Whitney, 1932]
	Si $G$ est un graphe simple, alors
	\[
	\kappa(G) \le \kappa'(G) \le \delta(G)\,.
	\]
\end{mytheo}
\begin{proof}
	Les arêtes incidentes à un n\oe{}ud $v$ de degré minimum
	forment une coupe d'arêtes;
	on a donc $\kappa'(G) \le \delta(G)$.
	On doit alors montrer que $\kappa(G) \le \kappa'(G)$.

	On observe que $\kappa(G) \le n(G) - 1$.
	Considérons une coupe d'arêtes minimale $[S, \widebar{S}]$.
	Si tout n\oe{}ud de $S$
	est adjacent à tout n\oe{}ud de $\widebar{S}$, alors
	$\abs{[S, \widebar{S}]} = \abs{S} \abs{\widebar{S}}
	\ge n(G) - 1 \ge \kappa(G)$,
	et l'inégalité désirée est satisfaite.

	Sinon, on choisit $x \in S$ et $y \in \widebar{S}$
	avec $x \centernot \leftrightarrow y$.
	Soit $T$ l'ensemble des voisins de $x$ dans $\widebar{S}$
	et des n\oe{}uds de $S \setminus \{\,x\,\}$
	avec des voisins dans $\widebar{S}$.
	Tout chemin entre $x$ et $y$ passe par $T$,
	donc $T$ est une coupe de sommets.
	On a également que choisir les arêtes de $x$ à $T \cap \widebar{S}$
	et une arête de chaque n\oe{}ud de $T \cap S$ à $\widebar{S}$
	donne $\abs{T}$ arêtes distinctes de $[S, \widebar{S}]$.
	On a donc $\kappa'(G) = \abs{[S, \widebar{S}]}
	\ge \abs{T} \ge \kappa(G)$.
\end{proof}

\subsection{Graphes $k$-connexes}
\subsubsection{Graphes $2$-connexes}
\begin{mytheo}[Whitney, 1932]
	\label{theo:whitney}
	Un graphe $G$ ayant au moins trois n\oe{}uds
	est $2$-connexe si et seulement si
	pour toute paire $u, v \in V(G)$
	il existe des chemins entre $u$ et $v$
	dont les n\oe{}uds internes sont distincts dans $G$.
\end{mytheo}
\begin{proof}
	\emph{Suffisance.}
	Quand $G$ a des chemins entre $u$ et $v$
	dont les n\oe{}uds internes sont distincts,
	enlever un n\oe{}ud ne peut pas déconnecter $u$ et $v$.
	Comme cette condition est donnée pour toute paire $u, v$,
	enlever un n\oe{}ud quelconque
	ne peut pas déconnecter de paire de n\oe{}uds.
	On conclut que $G$ est $2$-connexe.

	\emph{Nécessité.}
	Supposons que $G$ soit $2$-connexe.
	On prouve par induction sur $d(u, v)$
	que $G$ a des chemins entre $u$ et $v$
	dont les n\oe{}uds internes sont distincts.

	Cas de base ($d(u, v) = 1$).
	Quand $d(u, v) = 1$,
	le graphe $G - uv$ est connexe,
	car $\kappa'(G) \ge \kappa(G) \ge 2$.
	Un chemin entre $u$ et $v$ dans $G - uv$
	a des n\oe{}uds internes distincts de ceux du chemin $u, v$
	formé par l'arête elle-même dans $G$.

	Cas inductif ($d(u, v) > 1$).
	Soit $k = d(u, v)$.
	Soit $w$ l'arête avant $v$ dans un plus court chemin entre $u$ et $v$;
	on a $d(u, w) = k - 1$.
	Par l'hypothèse d'induction,
	$G$ a des chemins entre $u$ et $w$
	dont les n\oe{}uds internes sont distincts, $P$ et $Q$.
	Si $v \in V(P) \cup V(Q)$,
	alors on trouve les chemins désirés dans le cycle $P \cup Q$.
	Supposons que ce ne soit pas le cas.

	Comme $G$ est $2$-connexe,
	$G - w$ est connexe et contient un chemin $R$ entre $u$ et $v$.
	Si $R$ évite $P$ ou $Q$,
	on a fini,
	mais il est possible
	que $R$ partage des arêtes internes avec $P$ et $Q$.
	Soit $z$ le dernier n\oe{}ud de $R$ avant $v$ appartenant à $P \cup Q$.
	Par symétrie, on peut assumer que $z \in P$.
	On combine le chemin entre $u$ et $z$ de $P$
	avec le chemin entre $z$ et $v$ de $R$,
	afin d'obtenir un chemin entre $u$ et $v$
	dont les n\oe{}uds internes sont distincts de $Q \cup wv$.
\end{proof}

\subsubsection{Graphes $k$-connexes et $k$-arête-connexes}
Le Théorème~\ref{theo:mengervertex} donne
une généralisation du Théorème~\ref{theo:whitney}.
\begin{mytheo}[Menger, 1927]
	\label{theo:mengervertex}
	Si $x$ et $y$ sont des sommets d'un graphe $G$,
	et que $xy \notin E(G)$,
	alors la taille minimum de la coupe $x,y$
	est égale au nombre maximal de chemins
	dont les n\oe{}uds internes sont distincts.
\end{mytheo}

\begin{mytheo}[Menger, 1927]
	\label{theo:mengeredge}
	Si $x$ et $y$ sont des n\oe{}uds distincts
	d'un graphe $G$ orienté ou non orienté,
	alors la taille minimum d'une coupe d'arêtes déconnectant $x$ et $y$
	est égale au nombre maximum de chemins dont les arêtes sont distinctes.
\end{mytheo}
Une généralisation du Théorème~\ref{theo:mengeredge} est donnée
par Ford-Fulkerson (Remarque~\ref{rem:mengerff}).

\subsection{Problèmes de flots}
\begin{mydef}
	Un \emph{réseau} est un graphe dirigé
	avec une \emph{capacité} non négative $c(e)$
	sur chaque arête $e$ et des ensembles de \emph{sommets source}
	et \emph{puits} $s$ et $t$.
	Un sommet est également appelé \emph{n\oe{}ud}.
	Un \emph{flot} $f$ assigne une valeur $f(e)$
	à chaque arête $e$.
	On écrit $f^+(v)$ pour le flot total des arêtes quittant $v$
	et $f^-(v)$ pour le flot total des arêtes rentrant dans $v$.
	Un flot est \emph{faisable}
	s'il satisfait les \emph{contraintes de capacité} $0 \le f(e) \le c(e)$
	pour chaque arête et
	les \emph{contraintes de conservation} $f^-(v) = f^+(v)$
	pour tout n\oe{}ud $v \notin \{\,s, t\,\}$.
\end{mydef}

\subsubsection{Flot maximum dans un réseau}

\begin{mydef}
	La \emph{valeur} $\val(f)$ d'un flot $f$
	est le flot net $f^+(t) - f^-(t)$ entrant dans les n\oe{}uds puits.
	Un \emph{flot maximum} est un flot faisable de valeur maximale.
\end{mydef}
On s'intéresse au problème du flot maximum,
c'est-à-dire trouver le flot dont la valeur est maximale.

\begin{mydef}
	Lorsque $f$ est un flot faisable dans un réseau $N$,
	un \emph{chemin $f$-augmentant} est un chemin d'un n\oe{}ud source
	à un n\oe{}ud puits $P$ dans le graphe sous-jacent $G$
	tel que pour tout $e \in E(P)$,
	\begin{itemize}
		\item si $P$ suit $e$ dans la direction avant,
		alors $f(e) < c(e)$;
		\item si $P$ suit $e$ dans la direction arrière,
		alors $f(e) > 0$.
	\end{itemize}
	Soit $\varepsilon(e) = c(e) - f(e)$ quand
	$e$ est suivi dans la direction avant dans $P$,
	et $\varepsilon(e) = f(e)$ si $e$ est suivi
	dans la direction arrière dans $P$.
	La \emph{tolérance} de $P$ est $\min_{e \in E(P)} \varepsilon(e)$.
\end{mydef}

\begin{mylem}
	Si $P$ est un chemin $f$-augmentant avec tolérance $z$,
	alors changer le flot de $+z$
	sur les arêtes suivies dans la direction avant par $P$,
	et de $-z$ sur les arêtes suivies dans la direction arrière par $P$
	produit un flot faisable $f'$ avec $\val(f') = \val(f) + z$.
\end{mylem}
\begin{proof}
	La définition de la tolérance assure que $0 \le f'(e) \le c(e)$
	pour toute arête $e$,
	donc les contraintes de capacité sont satisfaites.
	Pour les contraintes de conservation,
	on doit uniquement vérifier les arêtes de $P$,
	comme le flot n'a pas changé autre part.

	Les arêtes de $P$ incidentes à un n\oe{}ud interne $v$ de $P$
	arrivent dans un des quatre cas de figure ci-dessous.
	Dans tous les cas, le changement de flot sortant de $v$
	est le même que le changement de flot rentrant dans $v$,
	donc le flot net sortant de $v$ reste $0$ dans $f'$.
	Finalement, le flot net dans le puits $t$ augmente de $z$.
\end{proof}
\begin{figure}[H]
	\centering
	\begin{tikzpicture}
		\node at (-3, 0.5) (u1) {};
		\node[draw, circle] at (-2, 0.5)  (v1) {$v$};
		\node at (-1, 0.5)  (w1) {};
		\node at (1, 0.5)  (u2) {};
		\node[draw, circle] at (2, 0.5)  (v2) {$v$};
		\node at (3, 0.5)  (w2) {};
		\node at (-3, -0.5)  (u3) {};
		\node[draw, circle] at (-2, -0.5)  (v3) {$v$};
		\node at (-1, -0.5)  (w3) {};
		\node at (1, -0.5)  (u4) {};
		\node[draw, circle] at (2, -0.5)  (v4) {$v$};
		\node at (3, -0.5)  (w4) {};

		\draw[->] (u1) edge node[above] {$+$} (v1);
		\draw[->] (v1) edge node[above] {$+$} (w1);
		\draw[->] (u2) edge node[above] {$+$} (v2);
		\draw[->] (w2) edge node[above] {$-$} (v2);
		\draw[->] (v3) edge node[above] {$-$} (u3);
		\draw[->] (v3) edge node[above] {$+$} (w3);
		\draw[->] (v4) edge node[above] {$-$} (u4);
		\draw[->] (w4) edge node[above] {$-$} (v4);
	\end{tikzpicture}
\end{figure}

\begin{mydef}
	Dans un réseau, une \emph{coupe source/puits} $[S, T]$
	consiste des arêtes d'un \emph{ensemble source} $S$
	vers un \emph{ensemble puits} $T$,
	où $S$ et $T$ partitionnent
	l'ensemble des n\oe{}uds\footnote{C'est-à-dire
	$S = \widebar{T}$ et $T = \widebar{S}$.},
	avec $s_i \in S$ et $t_i \in T$.
	La \emph{capacité} de la coupe $[S, T]$,
	notée $\capa(S, T)$, est la somme des capacités des arêtes de $[S, T]$.
\end{mydef}

\begin{mylem}
	\label{lem:flowval}
	Si $U$ est un ensemble de n\oe{}uds dans un réseau,
	alors le flot net sortant de $U$ est la somme des flots nets
	sortant des n\oe{}uds de $U$.
	En particulier, si $f$ est un flot faisable
	et $[S, T]$ est un coupe source/puits,
	alors le flot net sortant de $S$ et le flot net entrant dans $T$
	sont égaux à $\val(f)$.
\end{mylem}
\begin{proof}
	Mathématiquement, on veut prouver
	\[
	f^+(U) - f^-(U) = \sum_{v \in U} [f^+(v) - f^-(v)]\,.
	\]

	On considère la contribution du flot $f(xy)$
	sur une arête $xy$ des deux cotés de la formule.
	Si $x,y \in U$, alors $f(xy)$ n'est pas compté à gauche,
	mais contribue positivement (par $f^+(x)$)
	et négativement (par $f^-(y)$) à droite.
	Si $x,y \notin U$, alors $f(xy)$ ne contribue à aucun des deux cotés.
	Si $xy \in [U, \widebar{U}]$, alors l'arête contribue négativement
	des deux cotés.
	En sommant sur toutes les arêtes, on a l'égalité demandée.

	Lorsque $[S, T]$ est un coupe source/puits
	et que $f$ est un flot faisable,
	le flot net des n\oe{}uds de $S$ vaut $f^+(s) - f^-(s)$,
	et le flot net des n\oe{}uds de $T$
	vaut $f^+(t) - f^-(t) = -\val(f)$.
	Le flot net à travers chaque coupe source/puits est donc égale
	au flot net sortant de $s$ et au flot net entrant dans $t$.
\end{proof}
\begin{mycorr}[Dualité faible]
	Si $f$ est un flot faisable et $[S, T]$ est une coupe source/puits,
	alors $\val(f) \le \capa(S, T)$.
\end{mycorr}
\begin{proof}
	Par le Lemme~\ref{lem:flowval},
	la valeur de $f$ est égale au flot net sortant de $S$.
	On a donc
	\[
	\val(f) = f^+(S) - f^-(S) \le f^+(S)\,,
	\]
	comme le flot entrant dans $S$ n'est pas négatif.
	Comme les contraintes de capacité demandent $f^+(S) \le \capa(S, T)$,
	on trouve $\val(f) \le \capa(S, T)$.
\end{proof}

\begin{algorithm}[H]
\DontPrintSemicolon
\KwData{Un flot faisable $f$ dans un réseau.}
\KwResult{Un chemin $f$-augmentant ou une coupe avec capacité $\val(f)$.}
\Begin{
	$R \gets \{\,s\,\}$\;
	$S \gets \emptyset$\;
	\While{true}{
		$v \gets R - S$\;
		\ForAll{arête sortante $vw$ avec $f(vw) < c(vw)$ et $w \notin R$}{
			$R \gets R \cup \{\,w\,\}$\;
		}
		\ForAll{arête entrante $uv$ avec $f(uv) > 0$ et $u \notin R$}{
			$R \gets R \cup \{\,u\,\}$\;
		}
		Annoter toute arête dans $R$ comme \og atteinte \fg{}\;
		Ajouter $v$ comme le sommet atteignant cette arête\;
		Après avoir exploré toutes les arêtes en $v$,
		ajouter $v$ à $S$\;
		\If{$t \in R$}{
			Tracer le chemin pour atteindre $t$\;
			\Return chemin $f$-augmentant construit\;
		}
		\If{$R = S$}{
			\Return $[S, \widebar{S}]$\;
		}
	}
}
\caption{Algorithme de Ford-Fulkerson\label{algo:fordfulkerson}}
\end{algorithm}

\begin{mytheo}[Théorème Max-flow Min-cut---Ford et Fulkerson, 1956]
	Dans tout réseau, la valeur maximum d'un flot faisable
	est égale à la capacité minimum d'une coupe source/puits.
\end{mytheo}
\begin{proof}
	Dans le problème max-flow,
	le flot nul ($f(e) = 0$ pour tout $e$)
	est toujours un flot faisable et nous donne un endroit où commencer.
	Étant donnés un flot faisable,
	on applique l'Algorithme de Ford-Fulkerson.
	Celui-ci ajoute itérativement des n\oe{}uds à $S$,
	et termine lorsque $t \in R$ ou $R = S$.

	Dans le premier cas, on a un chemin $f$-augmentant
	et on augmente la valeur du flot.
	On répète alors l'algorithme.
	Lorsque les capacités sont rationelles,
	toute augmentation augmente le flot d'un multiple de $1/a$,
	où $a$ est le plus petit commun multiple des dénominateurs,
	donc après un nombre fini d'augmentations,
	la capacité d'une certaine coupe
	est atteinte\footnote{Comme
	l'ensemble des rationnels $\mathbb{Q}$ est infini dénombrable.}.
	L'algorithme termine alors avec $R = S$.

	Lorsqu'on se termine de cette façon,
	on dit que $[S, T]$ est une coupe source/puits avec capacité $\val(f)$,
	où $T = \widebar{S}$ et $f$ est le flot actuel.
	C'est une coupe car $s \in S$ et $t \notin R = S$.
	Comme appliquer l'algorithme de Ford-Fulkerson au flot $f$
	n'introduit pas de n\oe{}ud de $T$ dans $R$,
	aucune arête de $S$ dans $T$ dépasse sa capacité,
	et aucune arête de $T$ dans $S$ a un flot non nul dans $f$.
	Par conséquent, $f^+(S) = \capa(S, T)$ et $f^-(S) = 0$.

	Comme le flot net sortant d'un quelconque ensemble contenant les sources mais pas les puits est $\val(f)$,
	nous avons prouvé
	\[
	\val(f) = f^+(S) - f^-(S) = f^+(S) = \capa(S, T)\,.\qedhere
	\]
\end{proof}

\subsubsection{Flots intégraux}

\begin{mycorr}[Théorème d'intégralité]
	Si toutes les capacités dans un réseau sont entières,
	alors il y a un flot maximum assignant un flot intégral à chaque arête.
	De plus, un certain flot maximum peut être partitionné
	en flots unitaires le long de chemins des sources aux puits.
\end{mycorr}

\begin{myrem}[Menger à partir de Max-flow Min-cut]
	\label{rem:mengerff}
	Lorsque $x$ et $y$ sont des sommets dans un graphe dirigé $D$,
	on peut voir $D$ comme un réseau avec source $x$ et puits $y$,
	et une capacité unitaire sur chaque arête
	(afin d'assurer que les unités de flot de $x$ à $y$ correspondent
	à des chemins de $x$ vers $y$ disjoints deux à deux dans $D$).
	Un flot de valeur $k$ donne donc $k$ de ces chemins.

	De façon similaire,
	toute partition source/puits $S, T$ définit un ensemble d'arêtes qui,
	si elles sont enlevées, déconnectent $x$ et $y$: l'ensemble $[S, T]$.
	Comme chaque capacité vaut $1$,
	la taille de cet ensemble est $\capa(S, T)$.

	Les chemins et la coupe d'arêtes que nous avons obtenus
	ne sont pas nécessairement optimaux,
	mais par le Théorème Max-flow Min-cut,
	on a
	\[
	\lambda'_D(x,y) \ge \max \val(f) = \min \capa(S, T) \ge \kappa_D'(x,y)\,,
	\]
	où $\lambda'_D(x,y)$ dénote
	le nombre maximal de chemins arête-disjoints entre $x$ et $y$ dans $D$
	et $\kappa'_D(x,y)$ dénote l'arête-connectivité de $x$ et $y$ dans $D$.
	Comme on a toujours $\kappa'(x,y) \ge \lambda'(x,y)$,
	l'égalité tient toujours.
\end{myrem}

\section{Coloriage de graphes}
\subsection{Coloriage de n\oe{}uds et bornes supérieures}
\subsubsection{Définitions et examples}
\begin{mydef}
	Un \emph{$k$-coloriage} d'un graphe $G$
	est une annotation $f \colon V(G) \to S$,
	où $\abs{S} = k$.
	On annote chaque sommet avec une certaine \emph{couleur},
	et on dit que les sommets d'une même couleur forment
	une \emph{classe de couleurs}.
	Un $k$-coloriage est \emph{propre} si des sommets adjacents
	ont des couleurs différentes.
	Un graphe est \emph{$k$-coloriable} s'il a un $k$-coloriage propre.
	Le \emph{nombre chromatique} $\chi(G)$ est le plus petit $k$
	tel que $G$ soit $k$-coloriable.
\end{mydef}

\begin{mydef}
	Un graphe $G$ est \emph{$k$-chromatique} si $\chi(G) = k$.
	Un $k$-coloriage propre d'un graphe $k$-chromatique
	est un \emph{coloriage optimal}.
\end{mydef}

\begin{mydef}
	Le \emph{nombre de clique} d'un graphe $G$, noté $\omega(G)$,
	est la taille maximum
	d'un ensemble de sommets adjacents deux à deux (clique) dans $G$.
\end{mydef}

\begin{mypropo}
	Pour tout graphe $G$, $\chi(G) \ge \omega(G)$
	et $\chi(G) \ge \frac{n(G)}{\alpha(G)}$.
\end{mypropo}
\begin{proof}
	La première borne est respectée car les sommets d'une clique
	requièrent des couleurs différentes.
	La deuxième borne tient car chaque classe de couleurs
	est un ensemble indépendant et a donc au plus $\alpha(G)$ sommets.
\end{proof}

\subsubsection{Bornes supérieures}
Le \emph{coloriage glouton} relatif à
un ordre des sommets $v_1, \ldots, v_n$ de $V(G)$ est obtenu
en coloriant les sommets dans l'ordre $v_1, \ldots, v_n$,
en assignant à $v_i$ la couleur avec l'indexe le plus petit
qui ne soit pas encore utilisée pour ses voisins $v_j$ avec $j < i$.

\begin{mypropo}
	$\chi(G) \le \Delta(G) + 1$.
\end{mypropo}
\begin{proof}
	Dans un ordre de sommets,
	tout sommet a au plus $\Delta(G)$ voisins précédents,
	donc le coloriage glouton ne peut pas être forcé
	d'utiliser plus de $\Delta(G) + 1$ couleurs.
	Ceci prouve par construction que $\chi(G) \le \Delta(G) + 1$.
\end{proof}

Une \emph{représentation intervalle} d'un graphe est une famille d'intervalles
assignées aux sommets telle que les sommets soient adjacents si et seulement si
les intervalles correspondants s'intersectent.
Un graphe ayant une telle représentation est un \emph{graphe intervalle}.

\begin{mypropo}
	Si $G$ est un graphe intervalle, alors $\chi(G) = \omega(G)$.
\end{mypropo}
\begin{proof}
	Trions les sommets par rapport à l'extrémité gauche
	des intervalles dans une représentation intervalle.
	Appliquons le coloriage glouton, et supposons que $x$ reçoive $k$,
	la couleur maximale assignée.
	Comme $x$ ne reçoit pas une couleur plus petite,
	l'extrémité gauche $a$ de son intervalle appartient aussi
	à des intervalles ayant déjà les couleurs $1$ à $k-1$.
	Ces intervalles partagent tous le point $a$,
	donc on a une $k$-clique consistant de $x$ et de ses voisins,
	avec des couleurs de $1$ à $k-1$.
	Par conséquent, $\omega(G) \ge k \ge \chi(G)$.
	Comme on a toujours $\chi(G) \ge \omega(G)$, le coloriage est optimal.
\end{proof}

\subsubsection{Théorème de Brooks}

\begin{mytheo}[Brooks, 1941]
	Si $G$ est un graphe connexe autre
	qu'un graphe complet ou un cycle impair\footnote{On parle ici
	des graphes $C_{2k+1}$, pour $k \ge 0$.
	Le graphe peut contenir des cycles impairs.},
	alors $\chi(G) \le \Delta(G)$.
\end{mytheo}
\begin{proof}
	Soit $G$ un graphe connexe, et soit $k = \Delta G$.
	Nous pouvons supposer que $k \ge 3$,
	comme $G$ est un graphe complet quand $k \le 1$
	et $G$ est un cycle impair ou un graphe biparti quand $k = 2$,
	impliquant que la borne tient dans ce cas.

	Nous essayons d'ordonner les sommets tel que chaque sommet
	ait au plus $k-1$ voisins le précédant.
	Le coloriage glouton pour un tel ordre de sommets donne la borne.

	Quand $G$ n'est pas $k$-régulier, nous pouvons choisir
	un sommet de degré plus petit que $k$ comme $v_n$.
	Comme $G$ est connexe, on peut construire un arbre sous-tendant
	à partir de $v_n$, en assignant les indices en ordre décroissant.
	Tout sommet autre que $v_n$ dans l'ordre résultant $v_1,\ldots,v_n$
	a un voisin d'indice plus élevé
	le long du chemin vers $v_n$ dans l'arbre.
	Par conséquent, tout sommet a au plus $k-1$ voisins
	d'indice moins élevé et le coloriage glouton
	utilise au plus $k$ couleurs.

	Dans le cas restant, $G$ est $k$-régulier.
	Supposons d'abord que $G - x$ ne soit pas connexe,
	et soit $G'$ un sous-graphe consistant
	d'une composante de $G - x$
	ensemble avec ses arêtes vers $x$.
	Le degré de $x$ dans $G'$ est plus petit que $k$,
	donc la méthode ci-dessus donne un $k$-coloriage propre de $G$.

	On peut donc supposer que $G$ soit $2$-connexe.
	Dans tout ordre de sommets,
	le dernier sommet a $k$ voisins trouvés précédemment.
	L'idée du coloriage glouton pourrait toujours fonctionner
	si on fait en sorte que deux voisins de $v_n$ aient la même couleur.

	En particulier, supposons qu'un certain sommet $v_n$
	ait des voisins $v_1, v_2$ tels que $v_1 \centernot \leftrightarrow v_2$
	et que $G - \{\,v_1, v_2\,\}$ soit connexe.
	Dans ce cas, on indexe les sommets
	d'un arbre sous-tendant $G - \{\,v_1, v_2\,\}$
	avec $3, \ldots, n$. de telle sorte que les indexes augmentent
	le long des chemins vers la racine $v_n$.
	Comme avant, chaque sommet $v_n$ a
	au plus $k-1$ voisins d'indexe plus bas.
	Le coloriage glouton utilise au plus $k-1$ couleurs
	sur les voisins de $v_n$,
	comme $v_1$ et $v_2$ reçoivent la même couleur.

	Il suffit donc de montrer que tout graphe $2$-connexe $k$-régulier
	avec $k \ge 3$ a un tel triplet $v_1, v_2, v_n$.
	Choisissons un sommet $x$.
	Si $\kappa(G - x) \ge 2$,
	laissons $v_1$ être $x$ et $v_2$ être un sommet
	à une distance $2$ de $x$.
	Un tel sommet $v_2$ existe car $G$ est régulier
	et n'est pas un graphe complet;
	soit $v_n$ un voisin commun à $v_1$ et $v_2$.

	Si $\kappa(G - x) = 1$, laissons $v_n = x$.
	Comme $G$ n'a pas de sommet qui déconnecte le graphe,
	$x$ a un voisin dans chaque block feuille de $G - x$.
	Les voisins $v_1, v_2$ de $x$ dans deux de ces blocks
	ne sont pas adjacents.
	Aussi, $G - \{\,x, v_1, v_2\,\}$ est connexe,
	puisque les blocks n'ont pas de sommet déconnectant $G$.
	Comme $k \ge 3$, le sommet $x$ a un autre voisin,
	et $G - \{\,v_1, v_2\,\}$ est connexe.
\end{proof}

\subsection{Structure des graphes $k$-chromatiques}
\subsubsection{Théorème de Turán}
\begin{mytheo}[Turán, 1941]
	Tout graphe $G$ ayant $n$ sommets, et ne contenant pas de clique
	de taille plus grande que $r$
	(c'est-à-dire ne contenant pas le graphe $K_{r+1}$)
	possède au plus le nombre suivant d'arêtes:
	\[
	\frac{r-1}{r} \frac{n^2}{2} = \left( 1 - \frac{1}{r} \right) \frac{n^2}{2}\,.
	\]
	Cette borne est atteinte par le graphe de Turán $T_{n, r}$.
\end{mytheo}

\subsection{Aspects énumératifs}
\subsubsection{Compter les coloriages propres}
Pour un graphe $G$, le polynôme chromatique du graphe $G$, $\chi(G; k)$,
est le nombre de coloriages propres différents à $k$ couleurs.
On a donc $\chi(G) = \min \{\,k: \chi(G; k) > 0\,\}$.

\begin{mytheo}[Récurrence chromatique]
	Si $G$ est un graphe simple et $e \in E(G)$,
	alors $\chi(G; k) = \chi(G - e; k) + \chi(G \cdot e; k)$.
\end{mytheo}
\begin{proof}
	Tout $k$-coloriage propre de $G$
	est un $k$-coloriage propre de $G - e$.
	Un $k$-coloriage propre de $G - e$ est un $k$-coloriage propre de $G$
	si et seulement si il donne des couleurs distinctes
	aux extrémités $u$ et $v$ de $e$.
	Ainsi, on peut compter le nombre de $k$-coloriages propres de $G$
	en soustrayant de $\chi(G - e; k)$
	le nombre de $k$-coloriages propres de $G - e$
	qui donnent la même couleur à $u$ et $v$.

	Les coloriages de $G - e$ dans lesquels $u$ et $v$ ont la même couleur
	correspondent directement aux $k$-coloriages propres de $G \cdot e$,
	dans lesquels la couleur du sommet contracté
	est la couleur commune de $u$ et $v$.
	Un tel coloriage colorie proprement toutes les arêtes de $G \cdot e$
	si et seulement s'il colorie proprement
	toutes les arêtes de $G$ autres que $e$.
\end{proof}

\begin{mytheo}[Whitney, 1933]
	Le polynôme chromatique $\chi(G; k)$ d'un graphe simple $G$
	a un degré $n(G)$, avec des coefficients entiers
	alternant en signe et commençant avec $1, -e(G),\ldots$
\end{mytheo}
\begin{proof}
	On utilise l'induction sur $e(G)$.
	Le théorème est trivialement vrai lorsque $e(G) = 0$,
	où $\chi(\widebar{K_n}; k) = k^n$.
	On raisonne ensuite par induction sur le nombre d'arêtes.
	% TODO complete, p.222
\end{proof}

\begin{table}[H]
	\centering
	\begin{tabular}{lc}
		\hline
		\hline
		\multicolumn{1}{c}{Graphe} & Polynôme chromatique \\
		\hline
		\\
		Graphe complet $K_n$ & $\chi(K_n; k) = k (k-1) \cdots (k - (n-1))$ \\
		\\
		Graphe cycle $C_n$ & $\chi(C_n; k) = (-1)^n (k-1) + (k-1)^n$ \\
		\\
		Arbre à $n$ sommets & $\chi(G; k) = k (k-1)^{n-1}$ \\
		\hline
		\hline
	\end{tabular}
	\caption{Polynômes chromatiques pour quelques graphes.}
\end{table}

\section{Graphes planaires}
\subsection{Enchâssement et formule d'Euler}
\subsubsection{Dessins dans le plan}

\begin{mydef}
	Un graphe est \emph{planaire} s'il est possible
	de le dessiner dans le plan sans intersections.
	Un tel dessin est un \emph{enchâssement planaire} de $G$.
	Un \emph{graphe plan} est un enchâssement planaire particulier
	d'un graphe planaire.
\end{mydef}

\begin{mypropo}
	$K_5$ et $K_{3,3}$ ne peuvent pas être planaires.
\end{mypropo}
\begin{proof}
	Considérons un dessin de $K_5$ ou $K_{3,3}$ dans le plan.
	Soit $C$ un cycle sous-tendant.
	Si le dessin n'a pas d'arêtes s'intersectant,
	alors $C$ est dessiné comme une courbe fermée.
	Les cordes de $C$ doivent être dessinées à l'intérieur
	ou à l'extérieur de cette courbe.
	Deux cordes sont en conflit
	si leurs extrémités dans $C$ arrivent en ordre alternant.
	Lorsque deux cordes sont en conflit,
	on ne peut en dessiner qu'une à l'intérieur et une à l'extérieur de $C$.

	Un $6$-cycle dans $K_{3,3}$ a
	trois paires de cordes en conflit deux à deux.
	On peut en mettre au plus une à l'intérieur et une à l'extérieur,
	et ils n'est donc pas possible de compléter l'enchâssement.
	Lorsque $C$ est un $5$-cycle dans $K_5$,
	au plus deux cordes peuvent aller à l'intérieur ou à l'extérieur.
	Comme il y a cinq cordes, il n'est de nouveau pas possible
	de compléter l'enchâssement.
	Par conséquent, aucun des ces graphes n'est planaire.
\end{proof}

\subsubsection{Graphes duaux}
\begin{mydef}
	Le \emph{graphe dual} $G^*$ d'un graphe plan $G$
	est un graphe plan dont les sommets correspondent aux faces de $G$.
	Les arêtes de $G^*$ correspondent aux arêtes de $G$ comme suit:
	si $e$ est une arête de $G$ avec la face $X$ d'un coté
	et la face $Y$ de l'autre,
	alors les extrémités de l'arête duale $e^* \in E(G^*)$
	sont les sommets $x,y$ de $G^*$
	qui représentent les faces $X$ et $Y$ de $G$.
	L'ordre dans le plan des arêtes incidentes à $x \in V(G)$
	est l'ordre des arêtes longeant la face $X$ de $G$
	le long d'une promenade sur son bord.
\end{mydef}

\begin{mydef}
	La \emph{longueur}\footnote{Également appelée \emph{degré}.}
	d'une face dans un graphe plan $G$
	est la longueur totale d'une promenade fermée dans $G$
	longeant la face.
\end{mydef}

\begin{mypropo}
	\label{propo:facesum}
	Si $l(F_i)$ dénote la longueur de la face $F_i$ dans un graphe plan $G$,
	alors $2e(G) = \sum l(F_i)$.
\end{mypropo}
\begin{proof}
	Les longueurs de face sont les degrés des sommets du dual.
	Comme $e(G) = e(G^*)$,
	dire que $2e(G) = \sum l(F_i)$ est donc équivalent à la formule
	$2e(G) = \sum d_{G^*}(x)$ pour $G^*$.\footnote{Les deux sommes
	comptent chaque arête en double.}
\end{proof}

\subsubsection{Formule d'Euler}
\begin{mytheo}[Euler, 1758]
	Si un graphe plan connexe a exactement
	$n$ sommets, $e$ arêtes et $f$ faces, alors $n - e + f = 2$.
\end{mytheo}
\begin{proof}
	Par induction sur $n$.
	Si $n = 1$, le graphe $G$ est un \og bouquet \fg{} de boucles,
	chacune une boucle fermée dans l'enchâssement.
	Si $e = 0$, $f = 1$ et la formule est juste.
	Chaque boucle ajoutée passe à travers une face
	et coupe celle-ci en deux.
	Ceci augmente $e$ de $1$ et $f$ de $1$,
	et la formule tient pour $n=1$, peu importe le nombre d'arêtes.

	Si $n > 1$, comme $G$ est connexe,
	on peut trouver une arête qui ne soit pas une boucle.
	Lorsqu'on contracte cette arête,
	on obtient un graphe plan $G'$ avec $n'$ sommets,
	$e'$ arêtes et $f'$ faces.
	La contraction ne change pas le nombre de faces
	(on a simplement raccourci les bords),
	mais elle réduit le nombre de sommets et d'arêtes de $1$,
	donc $n' = n-1$, $e' = e-1$ et $f' = f$.
	Par l'hypothèse d'induction, on trouve
	\[
	n - e + f = n' + 1 - (e' + 1) + f' = n' - e' + f' = 2\,.\qedhere
	\]
\end{proof}

\begin{mytheo}
	Si $G$ est un graphe planaire simple avec au moins trois sommets,
	alors $e(G) \le 3 n(G) - 6$.
	Si en plus $G$ ne contient pas de triangles,
	alors $e(G) \le 2 n(G) - 4$.
\end{mytheo}
\begin{proof}
	Il est suffisant de ne s'occuper que des graphes connexes;
	sinon, on pourrait ajouter des arêtes.
	La formule d'Euler mettra en relation $e(G)$ et $n(G)$
	si on peut disposer de $f$.

	La Proposition~\ref{propo:facesum}
	donne une inégalité entre $e$ et $f$.
	Tout bord de face dans un graphe simple contient au moins trois arêtes
	(si $n(G) \ge 3$).
	En laissant $\{\,f_i\,\}$ être la liste des longueurs de face,
	ceci donne $2e = \sum f_i \ge 3f$.
	Substituant ceci dans $n - e + f = 2$ donne $e \le 3n - 6$.

	Lorsque $G$ ne contient pas de trinagles,
	les faces ont une longueur d'au moins $4$.
	Dans ce cas,
	$2e = \sum f_i \ge 4f$, et on obtient $e \le 2n - 4$.
\end{proof}
On note que les conditions ci-dessus sont nécessaires.

\begin{mytheo}
	\label{theo:mindeg}
	Dans tout graphe planaire simple $G$,
	il y a un sommet de degré plus petit ou égal à $5$.
\end{mytheo}
\begin{proof}
	On montre la proposition équivalente
	que le degré moyen est plus petit que $6$.
	Si $n(G) < 3$, alors $d(v) \le n(G) - 1 < 2 < 5$
	pour tout $v \in V(G)$.
	Si $n(G) \ge 3$, on a $e(G) \le 3 n(G) - 6$.
	Le degré moyen est donné par
	\[
	\frac{\sum_{v \in V(G)} d(v)}{n(G)} = \frac{6 n(G)-12}{n(G)} < 6\,.\qedhere
	\]
\end{proof}

\begin{myrem}
	Un graphe est planaire si et seulement s'il est représentable
	sur la sphère sans croisement d'arêtes.
\end{myrem}

\begin{myexem}[Polyèdres réguliers]
	De façon informelle,
	on s'imagine un polyèdre régulier comme un solide
	dont la frontière consiste en polygones réguliers de la même longueur,
	avec le même nombre de faces se joignant en chaque sommet.
	Lorsqu'on étire le polyèdre pour former une sphère,
	et qu'on projette ensuite le dessin
	dans le plan par projection stéréographique,
	on obtient un graphe plan régulier avec des faces de la même longueur.
	Le dual est donc aussi un graphe régulier.

	Soit $G$ un graphe plan avec $n$ sommets, $e$ arêtes et $f$ faces.
	Supposons que $G$ soit régulier de degré $k$,
	et que toutes les faces aient une longueur $l$.
	La formule pour la somme des degrés pour $G$ et $G^*$
	donne alors $kn = 2e = lf$.
	En substituant pour $n$ et $f$ dans la Formule d'Euler,
	on obtient $e(2/k - 1 + 2/l) = 2$.
	Comme $2$ et $e$ sont positifs,
	l'autre facteur doit l'être également,
	ce qui donne $2/k + 2/l > 1$,
	et donc $2l + 2k > kl$.
	Cette inégalité est équivalente à $(k-2)(l-2) < 4$.

	Comme le dual d'un graphe $2$-régulier n'est pas simple,
	on requiert que $k, l \ge 3$.
	On a également que $(k-2)(l-2) < 4$ requiert $k,l \le 5$.
	On a donc seulement cinq paires d'entiers satisfaisant ces équations,
	et donc cinq solides platoniciens.
\end{myexem}

\subsection{Caractérisation des graphes planaires}
\begin{mytheo}[Kuratowski, 1930]
	Un graphe est planaire si et seulement s'il
	ne contient pas une subdivision
	(remplacement de chaque arête par un chemin)
	de $K_5$ ou $K_{3, 3}$ comme sous-graphe.
\end{mytheo}

\subsection{Paramètres de planarité}
\subsubsection{Coloriage des graphes planaires}
\begin{mytheo}[Théorème des cinq couleurs---Heawood, 1890]
	Tout graphe planaire est $5$-coloriable.
\end{mytheo}
\begin{proof}
	On travaille par induction sur $n(G)$.

	Il est trivial que pour $n(G) \le 5$,
	tous les graphes $G$ sont $5$-coloriables.

	Pour $n(G) > 5$,
	on utilise le Théorème~\ref{theo:mindeg}.
	Celle-ci implique que $G$ a un sommet de degré au plus $5$.
	Par l'hypothèse d'induction,
	$G - v$ est $5$-coloriable.
	Soit $f \colon V(G - v) \to [5]$
	un coloriage propre de $G - v$.
	Si $G$ n'est pas $5$-coloriable,
	alors $f$ assigne chaque couleur à un voisin de $v$, et donc $d(v) = 5$.
	Soient $v_1, v_2, v_3, v_4, v_5$ les voisins de $v$
	dans l'ordre horloger autour de $v$.
	On nomme les couleurs de sorte à ce que $f(v_i) = i$.

	Soit $G_{i,j}$ le sous-graphe de $G - v$
	induit par les sommets de couleurs $i$ et $j$.
	Échanger les deux couleurs sur une composante quelconque de $G_{i,j}$
	donne un autre $5$-coloriage de $G - v$.
	Si la composante de $G_{i,j}$ contenant $v_i$ ne contient pas $v_j$,
	alors on peut échanger les couleurs dessus pour enlever $i$ de $N(v)$.
	Donnant maintenant la couleur $i$ à $v$
	produit un $5$-coloriage propre de $G$.
	$G$ est donc $5$-coloriable à moins que pour tout choix de $i$ et $j$,
	la composante $G_{i,j}$ contenant $v_i$ contient aussi $v_j$.
	Soit $P_{i,j}$ un chemin dans $G_{i,j}$ de $v_i$ à $v_j$.

	Considérons le cycle $C$ complété avec $P_{1, 3}$ par $v$;
	ceci sépare $v_2$ de $v_4$.
	Le chemin $P_{2,4}$ doit croiser $C$.
	Comme $G$ est planaire,
	les chemins ne peuvent se croiser qu'en des sommets partagés.
	Les sommets de $P_{1,3}$ ont tous une couleur $1$ ou $3$,
	alors que les sommets de $P_{2,4}$ ont tous une couleur $2$ ou $4$,
	et ils n'ont donc pas de sommet en commun.

	Par cette contradiction, $G$ est $5$-coloriable.
\end{proof}

\begin{mytheo}[Théorème des quatre couleurs---Appel \& Haken, 1976]
	Tout graphe $G$ planaire est $4$-coloriable.
\end{mytheo}

\section{Arêtes et cycles}
\subsection{Coloriage d'arêtes}
\subsubsection{Coloriages d'arête}
\begin{mydef}
	Un \emph{$k$-coloriage d'arêtes} de $G$
	est une annotation $f \colon E(G) \to S$,
	où $\abs{S} = k$.
	Les annotations sont des \emph{couleurs};
	les arêtes d'une couleur forment une \emph{classe de couleur}.
	Un $k$-coloriage d'arêtes est \emph{propre}
	si les arêtes incidentes ont des couleurs différentes.
	Un graphe est \emph{$k$-arête-coloriable}
	s'il a un $k$-coloriage d'arêtes propre.
	L'\emph{indice chromatique} $\chi'(G)$ d'un graphe sans boucle $G$
	est le plus petit $k$ tel que $G$ soit $k$-arête-coloriable.
\end{mydef}

\begin{mytheo}
	Pour tout graphe $G$, $\chi'(G) \ge \Delta(G)$.
\end{mytheo}
\begin{proof}
	Les arêtes ayant un sommet en commum doivent être de couleur différente.
\end{proof}

\begin{mytheo}[König, 1916]
	Si $G$ est biparti, alors $\chi'(G) = \Delta(G)$.
\end{mytheo}
\begin{proof}
	Par le Corollaire~\ref{corr:hall},
	on peut déduire que tout graphe régulier biparti $H$
	peut être $\Delta(H)$-arête-colorié.
	Il suffit donc de montrer que pour tout graphe $G$
	avec un degré maximal $k$,
	il y a un graphe biparti $k$-régulier $H$ contenant $G$.

	Pour construire un tel graphe,
	on ajoute d'abord des sommets à l'ensemble biparti de $G$ le plus petit,
	afin d'égaliser les tailles.
	Si le graphe résultant $G'$ n'est pas régulier,
	alors chaque ensemble biparti a un sommet de degré plus petit que $k$.
	On ajoute alors une arête entre ces deux sommets.
	On continue d'ajouter de telles arêtes
	tant que le graphe n'est pas $k$-régulier;
	le graphe résultant est $H$.
\end{proof}

\begin{mytheo}[Vizing, 1964--1965 \& Gupta, 1966]
	Si $G$ est un graphe simple,
	alors $\chi'(G) \le \Delta(G) + 1$.
\end{mytheo}
\begin{proof}
	Soit $f$ un $\Delta(G)+1$-arête-coloriage propre
	d'un sous-graphe $G'$ de $G$.
	Si $G' \ne G$, alors une certaine arête $uv$ n'est pas coloriée par $f$.
	Après avoir recolorié potentiellement quelques arêtes,
	ont peut étendre le coloriage pour inclure $uv$;
	on parle d'\emph{augmentation}.
	Après $e(G)$ augmentations,
	on obtient un $\Delta(G)+1$-arête-coloriage propre de $G$.

	Comme le nombre de couleurs excède $\Delta(G)$,
	chaque sommet a une couleur
	qui n'apparaît pas sur ses arêtes incidentes.
	Soit $a_0$ une couleur manquante en $u$.
	On génère un liste de voisins de $u$
	et la liste de couleurs correspondante.
	On commence avec $v_0 = v$.

	Soit $a_1$ une couleur manquante en $v_0$.
	On peut supposer que $a_1$ apparaît en $u$
	sur une certaine arête $uv_1$;
	sinon on utiliserait $a_1$ sur $uv_0$.

	Soit $a_2$ une couleur manquante en $v_1$.
	On peut supposer que $a_2$ apparaît en $u$
	sur une certaine arête $uv_2$;
	sinon on remplacerait la couleur $a_1$ avec $a_2$ sur $uv_1$
	et puis on utiliserait $a_1$ sur $uv_0$ pour augmenter le coloriage.

	Ayant sélectionné $uv_{i-1}$ avec couleur $a_{i-1}$,
	soit $a_i$ la couleur manquante en $v_{i-1}$.
	Si $a_i$ manque en $u$, alors on utilise $a_i$ sur $uv_{i-1}$
	et on décale la couleur $a_j$ de $uv_j$ à $uv_{j-1}$
	pour $1 \le j \le i-1$ afin de compléter l'augmentation.
	On appelle ceci le \emph{décalage vers le bas à partir de $i$}.
	Si $a_i$ apparaît en $u$ (sur une certain arête $uv_i$),
	alors le procédé continue.

	Comme on a seulement $\Delta(G) + 1$ couleurs parmi lesquelles choisir,
	la liste de couleur sélectionnées se répète éventuellement
	(ou on termine de compléter l'augmentation par décalage vers le bas).
	Soit $l$ le plus petit indice tel que la couleur manquante en $v_l$
	soit dans la liste $a_1, \ldots, a_l$;
	soit cette couleur $a_k$.
	Au lieu d'étendre la liste,
	on utilise la répétition pour appliquer l'augmentation
	dans l'une des différentes façons.

	La couleur $a_k$ manquante en $v_l$ est aussi manquante en $v_{k-1}$
	et apparaît sur $uv_k$.
	Si $a_0$ n'apparaît pas en $v_l$, alors on décale vers le bas
	à partir de $v_l$ et on utilise la couleur $a_0$ sur $uv_l$
	pour compléter l'augmentation.
	On peut donc supposer que $a_0$ apparaît en $v_l$.

	Soit $p$ le chemin alternant maximal
	des arêtes coloriées $a_0$ et $a_k$ commençant en $v_l$
	le long de la couleur $a_0$.
	Il n'y a qu'un seul de ces chemins,
	car chaque sommet a au plus une arête incidente de chaque couleur
	(on ignore les arêtes pas encore coloriées).
	Pour compléter l'augmentation,
	on interchange les couleurs $a_0$ et $a_k$ sur $P$
	et on décale vers le bas à partir d'un voisin approprié de $u$,
	dépendant de vers où va $P$.

	Si $P$ atteint $v_k$,
	alors il arrive en $v_k$ le long d'une arête de couleur $a_0$,
	suit $v_k u$ de couleur $a_k$ et s'arrête en $u$,
	qui n'a pas la couleur $a_0$.
	Dans ce cas, on décale vers le bas à partir de $v_k$
	et on échange les couleurs sur $P$.

	Si $P$ atteint $v_{k-1}$, alors il l'atteint par la couleur $a_0$
	et s'arrête là, car $a_k$ n'apparaît pas en $v_{k-1}$.
	Dans ce cas, on décale vers le bas à partir de $v_{k-1}$,
	on donne la couleur $a_0$ à $uv_{k-1}$
	et on échange les couleurs sur $P$.

	Si $P$ n'atteint ni $v_k$ ni $v_{k-1}$,
	alors il s'arrête à un certain sommet
	hors de $\{\,u, v_l, v_k, v_{k-1}\,\}$.
	Dans ce cas, on décale vers le bas à partir de $v_l$,
	on donne la couleur $a_0$ à $u v_l$ et on échange les couleurs sur $P$.

	Dans tous les cas,
	les échanges décrits donnent un $\Delta(G) + 1$-arête-coloriage propre
	de $G' + uv$, et on a donc complété l'augmentation désirée.
\end{proof}

\subsection{Cycles hamiltoniens}
\begin{mydef}
	Un \emph{graphe hamiltonien} est un graphe
	avec un cycle sous-tendant,
	également appelé \emph{cycle hamiltonien}.
\end{mydef}

\subsubsection{Conditions nécessaires}
\begin{mypropo}
	Si $G$ a un cycle hamiltonien,
	alors pour tout ensemble non-vide $S \subseteq V$,
	le graphe $G \setminus S$ a au plus $\abs{S}$ composantes.
\end{mypropo}
\begin{proof}
	Lorsqu'on quitte une composante de $G \setminus S$,
	un cycle hamiltonien ne peut aller que dans $S$,
	et les arrivées dans $S$ doivent utiliser des n\oe{}uds distincts.
	$S$ doit donc avoir au moins autant de n\oe{}uds
	que $G \setminus S$ a de composantes.
\end{proof}

\subsubsection{Conditions suffisantes}
\begin{mytheo}[Dirac, 1952]
	Si $G$ est un graphe simple avec au moins trois n\oe{}uds
	et $\delta(G) \ge n(G)/2$,
	alors $G$ est hamiltonien.
\end{mytheo}
\begin{proof}
	La condition $n(G) \ge 3$ est nécessaire car $K_2$
	satisfait $\delta(K_2) = n(K_2)/2$,
	mais n'est pas hamiltonien.

	La preuve est par contradiction et extrémalité.
	S'il existe un graphe non hamiltonien
	satisfaisant les hypothèses,
	alors ajouter des arêtes ne peut réduire le degré minimum.
	On peut donc s'occuper uniquement des graphes non hamiltoniens
	maximaux avec degré minimal au moins $n/2$,
	ou \og maximaux \fg{} signifie qu'ajouter des arêtes
	joignant des n\oe{}uds non adjacents crée un cycle sous-tendant.

	Lorsque $u \centernot \leftrightarrow v$ dans $G$,
	la maximalité de $G$ implique que $G$ a un chemin sous-tendant
	$v_1, \ldots, v_n$ de $u = v_1$ à $v = v_n$,
	car tout cycle sous-tendant dans $G + uv$
	contient la nouvelle arête $uv$.
	Afin de prouver le théorème,
	il suffit de changer quelque chose dans ce cycle
	pour éviter d'utiliser l'arête $uv$;
	on construit ainsi un cycle sous-tendant dans $G$.

	Si un voisin de $u$ suit directement un voisin de $v$ sur le chemin,
	tel que $u \leftrightarrow v_{i+1}$ et $v \leftrightarrow v_i$,
	alors $(u, v_{i+1}, v_{i+2}, \ldots, v, v_i, v_{i-1}, \ldots, v_2)$
	est un cycle sous-tendant.

	Afin de prouver qu'un tel cycle existe,
	on montre qu'il y a un indice commun dans les ensembles $S$ et $T$
	definis comme $S = \{\,i: u \leftrightarrow v_{i+1}\,\}$
	et $T = \{\,i: v \leftrightarrow v_i\,\}$.
	En sommant la taille de ces ensembles,
	on trouve
	\[
	\abs{S \cup T} + \abs{S \cap T} = \abs{S} + \abs{T} = d(u) + d(v) \ge n\,.
	\]
	Ni $S$ ni $T$ contient l'indice $n$.
	On a donc $\abs{S \cup T} < n$, et donc $\abs{S \cap T} \ge 1$.
	On a établi une contradiction
	en trouvant un cycle sous-tendant dans $G$;
	il n'y a donc pas de graphe maximal non hamiltonien
	satisfaisant les hypothèses.
\end{proof}

\begin{mydef}
	Un \emph{chemin hamiltonien} est un chemin sous-tendant.
\end{mydef}

\begin{myprob}[Problème du voyageur de commerce, \textsc{tsp}]
	Dans un graphe pondéré, trouver le parcours fermé le plus court
	qui passe par tous les n\oe{}uds au moins une fois.
\end{myprob}
\begin{myprob}[Problème du postier chinois]
	Dans un graphe pondéré, trouver le parcours fermé le plus court
	qui passe par toutes les arêtes au moins une fois.
\end{myprob}

\section{Optimisation et complexité}
\subsection{Preuves de NP-complétude}
Une \emph{transformation} d'un problème A vers un problème B
est une procédure qui convertit les instances de A en instances de B
telle que la réponse à A sur l'instance initiale
est déterminée par la réponse à B sur l'instance transformée.
Si on a un algorithme efficace (en temps polynomial) pour B,
alors on a un algorithme efficace pour transformer A en B,
alors on a un algorithme efficace pour A.
On dit que A \emph{se réduit à} ou \emph{se transforme en} B.

\begin{mydef}
	Un problème A est dans P si et seulement s'il
	existe une machine de Turing et un polynôme $p(x)$
	tels que la machine de Turing termine en temps borné par $p(n)$ pour A,
	où $n$ est la taille de l'instance et renvoie \og OUI \fg{}
	si l'instance est positive et \og NON \fg{} sinon.

	Un problème A est dans NP s'il existe une machine de Turing
	qui prend deux entrées $m$ et $n$ et un polynôme $p(x)$
	qui borne sa complexité tels que la machine de Turing termine
	en temps borné par $p(\abs{m})$ et tel que la machine renvoie OUI
	sur l'entrée $(m, n)$ si et seulement si $m$ est une instance positive.
	Un problème A est NP-complet si tout problème dans NP
	peut être réduit à une instance de A.
\end{mydef}

Si A est NP-dur et se réduit à B par un algorithme en temps polynomial,
alors B est aussi NP-dur (un algorithme polynomial pour B
donne un algorithme polynomial pour A et donc pour tout NP).
Si B est aussi dans NP,
alors on dit que B est NP-complet par \emph{réduction à partir de} ou
\emph{par transformation de} A.
Ceci est vrai car les polynomes sont fermés sous multiplication
et sous composition.

\begin{mytheo}[Cook, 1971]
	SATISFAISABILITÉ est un problème NP-complet initial.
\end{mytheo}

\begin{mydef}[$3$-SATISFAISABILITÉ]
	Un ensemble de variables logiques $U = \{\,u_i\,\}$
	et un ensemble $C = \{\,c_i\,\}$ de clauses.
	Chaque clause consiste de trois littéraux
	étant des variables ou leur négation.

	La question qu'on se pose est de savoir
	si on peut assigner à chaque variable une valeur TRUE ou FALSE
	tel que chaque clause soit satisfaite.
\end{mydef}

\begin{mytheo}[Karp, 1972]
	$3$-SAT est NP-complet.
\end{mytheo}

\begin{mytheo}[Stockmeyer, 1973]
	$k$-COLORABILITÉ est un problème NP-complet.
\end{mytheo}

\begin{mytheo}[Karp, 1972]
	CLIQUE et ENSEMBLE INDÉPENDANT sont NP-complets.
\end{mytheo}

\begin{mytheo}
	$k$-ARÊTE-COLORABILITÉ est NP-complet.
\end{mytheo}

\end{document}
