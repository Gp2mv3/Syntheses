\section{Questions sur le chapitre ``Introduction et rappels''}
\subsection{Est-ce que la chaleur échangée entre un système et son extérieur est une variable d'état du système. Justifiez votre réponse.}
Une \textbf{variable d'état} caractérise un état et non une évolution entre deux états. La chaleur échangée $Q$ entre un système et son extérieur décrit le passage d'un état à un autre et ne peux pas être mesurée instantanément. Les variables d'état d'un système thermodynamique sont par exemple la pression $p$, le volume $V$ ou encore la température $T$. 

\subsection{Donnez la définition d'un processus \textit{adiabatique}. Donner un exemple d'un processus adiabatique réversible ainsi qu'un exemple d'un processus adiabatique irréversible.}
Une transformation est dite \textbf{adiabatique} si elle est effectué sans échange de chaleur ($Q = 0$). Cela n'implique pas pour autant que la température du système reste constante. La variation d'énergie mécanique est le seul échange avec l'extérieur du système qui modifie alors les variables d'états ($T$, $V$ et $p$).

Le concept de \textbf{réversibilité} est idéal. L'expression énergétique du second principe de la thermodynamique, aussi appelé principe d'évolution, s'écrit 
\begin{equation} Q + W_f = \int_1^2T dS \end{equation}
où $dS$ est la différentielle de la fonction d'état entropie et $W_f$ le terme dissipatif. Dans la réalité ce terme ne sera jamais nul. 
\begin{itemize}
	\item Un processus \textbf{réversible} est un processus où $W_f = 0$. Un processus adiabatique réversible est donc aussi isentropique car $\delta Q/T = dS = 0$. Un exemple d'un tel processus se trouve dans le cycle de Carnot qui comprend une détente ainsi qu'une compression adiabatique réversible. Il est à noter que le cycle de Carnot est un cycle théorique.
	\item Un processus \textbf{irréversible} est un processus où $W_f \neq 0$. On retrouve une détente adiabatique irréversible dans les cycle frigorifique. 
\end{itemize}

\subsection{Donnez la définition mathématique ainsi que la signification physique des chaleurs massiques $c_p$ et $c_v$.}
La capacité thermique massique $c$ correspond à la quantité d'énergie à apporter par échange thermique pour élever d''un kelvin la température de l'unité de masse d'une substance. C'est donc une grandeur intensive exprimé en joules par kilogramme-kelvin \si{\joule\per\kelvin\per\kilo\gram}. Si on note $U$ l'énergie interne, $H$ l'enthalpie et $m$ la masse d'un corps on a les capacités thermiques massiques :
\begin{itemize}
	\item à pression constante \begin{equation} c_p = \frac{1}{m}\left(\frac{\partial H}{\partial T}\right)_p \end{equation}
	\item à volume constant \begin{equation} c_v = \frac{1}{m}\left(\frac{\partial U}{\partial T}\right)_V \end{equation}
\end{itemize}

\subsection{Donnez la définition de l'enthalpie libre $F$ de Helmholtz ainsi que de l'enthalpie libre $G$ de Gibbs. Écriver l'équation de Gibbs sous la forme de différentielle de $F$ ainsi que de $G$.}
\paragraph{L'énergie libre $F$ de Helmholtz} est une fonction d'état extensive dont la variation permet d'obtenir le travail utile susceptible d'être fourni par un système fermé, à température constante, au cours d'une transformation réversible. 
Considérons un système thermodynamique (fermé) évoluant d'un état 1 à un état 2, transformation que l'on suppose totalement réversible ($W_f = 0$) et à température constante $T$. Le premier principe de la thermodynamique s'exprime par :
\begin{equation} \Delta U + \Delta K + g\Delta z = Q + \underline{W_e} \end{equation}
où $\Delta U$ désigne la variation de l'énergie interne, $\Delta K$ la variation de l'énergie cinétique, $g\Delta z$ la variation de l'énergie potentielle, $Q$ l'échange de chaleur et $\underline{W_e}$ le travail des forces extérieures. On introduit $W = -W_e$ qui correspond au travail récupérable par le milieu extérieur et on suppose $\Delta K = g\Delta z = 0$. On obtient alors
\begin{equation} W = Q - \Delta U \label{eq:Q1ePrincipe}\end{equation}
Le 2\ieme principe de la thermodynamique s'exprime par:
\begin{equation} \Delta S = \frac{Q}{T} + W_f \qquad\Leftrightarrow\qquad Q = T\Delta S - TW_f \label{eq:Q2ePrincipe}\end{equation}
où $\Delta S$ est la variation d'entropie du système. Si on remplace \ref{eq:Q2ePrincipe} dans \ref{eq:Q1ePrincipe}, nous trouvons
\begin{equation} W = T\Delta S - TW_f - \Delta U = T(S_2 - S_1) - TW_f + U_1 - U_2 \end{equation}
Il est évident, d'après la relation obtenue, que $W$ sera maximum si la transformation est totalement réversible ($W_f = 0$) :
\begin{equation} W_\text{max} = (U_1 - TS_1)-(U_2-TS_2)\end{equation}
On définit alors 
\begin{equation} F = U - TS \end{equation} 
Ce qui nous donne au final 
\begin{equation} W_\text{max} = -\Delta F \qquad\text{où}\qquad \underbrace{\Delta F}_\text{travail utile} = \underbrace{\Delta U}_\text{énergie totale} - \underbrace{T\Delta S}_\text{énergie inutilisable}\end{equation}

\paragraph{Forme différentielle de $F$}
\begin{equation} \left.\begin{array}{ll} dF &= d(U-TS) = dU - TdS - SdT \\ dU &= -pdV + TdS \end{array}\right\} \Rightarrow dF = -pdV - SdT\end{equation}

\paragraph{L'enthalpie libre $G$ de Gibbs} est une fonction d'état extensive dont la variation permet d'obtenir l'enthalpie utile susceptible d'être fournie par un système fermé, à température et à pression constante, au cours d'une transformation réversible. On considère le même système que précedemment avec la condition supplémentaire que la pression est constante. Par la définition de l'enthalpie :
\begin{equation} H \triangleq U + pV\end{equation}
De la même manière que pour l'énergie libre de Helmholtz, on trouve :
\begin{equation} W = Q - \Delta H \label{H1ePrincipe}\end{equation}
où $W$ correspond à nouveau au travail récupérable par le milieu extérieur. Si on remplace \ref{eq:Q2ePrincipe} dans \ref{H1ePrincipe} et qu'on prend on compte le fait que le processus est réversible, on obtient :
\begin{equation} W_\text{max} = (H_1 - TS_1) - (H_2 - TS_2) \end{equation}
On définit alors
\begin{equation} G = H - TS \end{equation}

\paragraph{Forme différentielle de G}
\begin{equation} \left.\begin{array}{ll} dG &= d(H-TS) = dH - TdS - SdT \\ dH &= Vdp + TdS \end{array}\right\} \Rightarrow dG = Vdp - SdT \label{eq:gibbs-duhem}\end{equation}

\subsection{Dérivez l'équation $\alpha = p\beta K$.}
Les variables d'état $p$, $v$ et $T$ d'un fluide sont liées par l'équation d'état, que l'on peut écrire sous forme du système différentiel :
\begin{equation}\begin{pmatrix} dp \\ dv \\ dT \end{pmatrix} = \begin{pmatrix} 0 & \left(\frac{\partial p}{\partial v}\right)_T & \left(\frac{\partial p}{\partial T}\right)_v \\ \left(\frac{\partial v}{\partial p}\right)_T & 0 & \left(\frac{\partial v}{\partial T}\right)_p \\ \left(\frac{\partial T}{\partial p}\right)_v & \left(\frac{\partial T}{\partial v}\right)_p & 0 \end{pmatrix}\begin{pmatrix} dp \\ dv \\ dT \label{eq:matetat}\end{pmatrix}\end{equation}
On définit les coefficients suivants :
\begin{itemize}
	\item le coefficient de dilatation isobare $\alpha \triangleq \frac{1}{v}\left(\frac{\partial v}{\partial T}\right)_p$
	\item le coefficient de dilatation isochore $\beta \triangleq \frac{1}{p}\left(\frac{\partial p}{\partial T}\right)_v$
	\item le coefficient de compressibilité isotherme $K \triangleq -\frac{1}{v}\left(\frac{\partial v}{\partial p}\right)_T$
\end{itemize}
Le système \ref{eq:matetat} peut se réécrire sous la forme :
\begin{equation}\begin{pmatrix} dp \\ dv \\ dT \end{pmatrix} = \begin{pmatrix} 0 & -\frac{1}{Kv} & \beta p \\ -Kv & 0 & \alpha v \\ \frac{1}{\beta p} & \frac{1}{\alpha v} & 0 \end{pmatrix}\begin{pmatrix} dp \\ dv \\ dT \end{pmatrix} \label{eq:fe_matrix}\end{equation}
On trouve l'équation recherchée à l'aide des variations de deux des trois variables d'état: 
\begin{equation} dp = -\frac{1}{Kv}dv + \beta p dT \label{eq:dp}\end{equation}
\begin{equation} dv = -Kv dp +  \alpha v dT \label{eq:dv}\end{equation}
En remplacant le terme $dv$ dans l'équation \ref{eq:dp} par l'équation \ref{eq:dv} on obtient :
\begin{equation} dp = dp - \frac{\alpha}{K}dT + \beta p dT \qquad\Rightarrow\qquad \alpha = p\beta K\end{equation}
Cette équation nous permet de conclure que l'équation d'état d'une substance est déterminée par la connaissance de deux des trois coefficients caractéristiques $\alpha$, $\beta$ ou $K$.

\subsection{Démontrez que les expressions \ref{eq:q6_1} sont valables pour toutes les espèces.}
\begin{equation} \left(\frac{\partial T}{\partial p}\right)_S = \left(\frac{\partial v}{\partial S}\right)_p, \qquad \left(\frac{\partial S}{\partial p}\right)_T = - \left(\frac{\partial v}{\partial T}\right)_p \label{eq:q6_1}\end{equation}
Toutes les fonctions d'état peuvent être exprimées en fonction de deux des trois variables d'état $p$, $v$ et $T$. Il en va de même pour l'entropie pour laquelle le système différentiel s'écrit :
\begin{equation}\begin{pmatrix}TdS\\TdS\\TdS\end{pmatrix} = \begin{pmatrix}0 & l_T & c_v \\ h_T & 0 & c_p \\ h_v & l_p & 0\end{pmatrix}\begin{pmatrix}dp\\dv\\dT\end{pmatrix} \end{equation}
On introduit ainsi quatre coefficients calorifiques $l_T$, $h_T$, $h_v$ et $l_p$ qui peuvent s'exprimer en fonction des coefficients calorifiques $c_p$ et $c_v$, et des coefficients de dilatation $\alpha$ et $\beta$ :
\begin{equation} h_v = \frac{c_v}{\beta p}, \qquad l_p = \frac{c_p}{\alpha v}, \qquad h_T = \frac{c_v-c_p}{\beta p}, \qquad l_T = \frac{c_p-c_v}{\alpha v}.\end{equation}
La substitution des expressions dans le système différentiel fournit la matrice $S'$ des dérivées partielles de l'entropie :
\begin{equation} S' = \begin{pmatrix} 0 & \frac{c_p-c_v}{\alpha vT} & \frac{c_v}{T} \\ \frac{c_v-c_p}{\beta pT} & 0 & \frac{c_p}{T} \\ \frac{c_v}{\beta pT} & \frac{c_p}{\alpha vT} & 0 \end{pmatrix} \end{equation}
Si on prend la deuxième ligne de ce système, on a :
\begin{equation} dS = \frac{c_v-c_p}{\beta pT}dp + \frac{c_p}{T}dT \qquad \Leftrightarrow \qquad \left(\frac{\partial T}{\partial p}\right)_S = \frac{c_p-c_v}{c_p\beta p} \end{equation}
Si on prend la troisième ligne de ce système, on a :
\begin{equation} dS = \frac{c_v}{\beta pT}dp + \frac{c_p}{\alpha vT}dv \qquad \Leftrightarrow \qquad \left(\frac{\partial v}{\partial S}\right)_p = \frac{\alpha vT}{c_p} \end{equation}
De la matrice \ref{eq:fe_matrix} on a :
\begin{equation} dv = -Kv dp + \alpha v dT \qquad \Leftrightarrow \qquad \left(\frac{\partial v}{\partial T}\right)_p = \alpha v \end{equation}
Avec la relation $c_p-c_v = \alpha\beta pvT$, on trouve finalement :
\begin{equation} \left(\frac{\partial T}{\partial p}\right)_S = \frac{c_p-c_v}{c_p\beta p} = \frac{\alpha vT}{c_p} = \left(\frac{\partial v}{\partial S}\right)_p \end{equation}
\begin{equation} \left(\frac{\partial S}{\partial p}\right)_T = \frac{c_v-c_p}{\beta pT} = \alpha v = \left(\frac{\partial v}{\partial T}\right)_p \end{equation}

\subsection{Dérivez l'équation $c_p-c_v = \alpha\beta pvT$.\label{q:1_7}}
On part des équations de Gibbs (énergie libre de Helmholtz et enthalpie libre de Gibbs) :
\begin{equation} \begin{cases} F &= U-TS \\ G &= H-TS \end{cases} \qquad \Rightarrow \qquad \begin{cases} dF &= -pdv - SdT \\ dG &= vdp - SdT \end{cases} \end{equation}
Ces différentielles correspondent aux expressions des dérivées partielles suivantes :
\begin{equation} \left(\frac{\partial F}{\partial v}\right)_T = -p \qquad \left(\frac{\partial F}{\partial T}\right)_v = -S \qquad \left(\frac{\partial G}{\partial p}\right)_T = v \qquad \left(\frac{\partial G}{\partial T}\right)_p = -S \end{equation}
Soit un potentiel thermodynamique $\Phi$ étant présumé au moins deux fois dérivable par rapport à chacune de ses variables. Le \textbf{théorème de Schwarz}\footnote{\url{https://fr.wikipedia.org/wiki/Th\%C3\%A9or\%C3\%A8me_de_Schwarz}} implique que pour tout couple de variable $x_i$ et $x_j$ :
\begin{equation} \left(\frac{\partial}{\partial x_i}\left(\frac{\partial \Phi}{\partial x_j}\right)_{x_{k\neq j}}\right)_{x_{k\neq i}} = \left(\frac{\partial}{\partial x_j}\left(\frac{\partial \Phi}{\partial x_i}\right)_{x_{k\neq i}}\right)_{x_{k\neq j}} \end{equation}
Si on applique le théorème de Schwarz, on obtient :
\begin{equation} \frac{\partial^2F}{\partial T\partial v} = -\left(\frac{\partial p}{\partial T}\right)_v = -\beta p , \qquad \frac{\partial^2F}{\partial v\partial T} = -\left(\frac{\partial S}{\partial v}\right)_T = -\frac{c_p-c_v}{\alpha vT} \end{equation}
\begin{equation} \frac{\partial^2G}{\partial T\partial p} = \left(\frac{\partial v}{\partial T}\right)_p = \alpha v , \qquad \frac{\partial^2G}{\partial p\partial T} = -\left(\frac{\partial S}{\partial p}\right)_T = -\frac{c_v-c_p}{\beta vT} \end{equation}
Ce qui entraîne la propriété liant entre elles les chaleurs massique $c_p$ et $c_v$ :
\begin{equation} c_p-c_v = \alpha\beta pvT \label{eq:relation_chaleur}\end{equation}

\subsection{Dérivez l'équation \ref{eq:q8}}
\begin{equation} l_T = \frac{c_p-c_v}{\alpha v} \label{eq:q8}\end{equation}
Toutes les fonctions d'état peuvent être exprimées en fonction de deux des trois variables d'état $p$, $v$ et $T$. Il en va de même pour l'entropie pour laquelle le système différentiel s'écrit :
\begin{equation}\begin{pmatrix}TdS\\TdS\\TdS\end{pmatrix} = \begin{pmatrix}0 & l_T & c_v \\ h_T & 0 & c_p \\ h_v & l_p & 0\end{pmatrix}\begin{pmatrix}dp\\dv\\dT\end{pmatrix} \end{equation}
En remplacant la variation de température $dT$ dans la première et seconde équations du système par 
\begin{equation} dT = \frac{1}{\beta p}dp + \frac{1}{\alpha v}dv \end{equation}
on obtient :
\begin{equation}\begin{pmatrix}TdS\\TdS\\TdS\end{pmatrix} = \begin{pmatrix}\frac{c_v}{\beta p} & l_T + \frac{c_v}{\alpha v} & 0\\ h_T + \frac{c_p}{\beta p} & \frac{c_p}{\alpha v} & 0 \\ h_v & l_p & 0\end{pmatrix}\begin{pmatrix}dp\\dv\\dT\end{pmatrix} \end{equation}
L'identification des trois équations de ce système permet d'exprimer les quatres nouveaux coefficients calorifiques en fonction des seuls coefficients calorifiques $c_p$ et $c_v$, et des coefficients de dilatation $\alpha$ et $\beta$ :
\begin{equation} h_v = \frac{c_v}{\beta p}, \qquad l_p = \frac{c_p}{\alpha v}, \qquad h_T = \frac{c_v-c_p}{\beta p}, \qquad l_T = \frac{c_p-c_v}{\alpha v}.\end{equation}
