\documentclass[fr]{../../../eplnotes}
\usepackage{../../../eplunits}
\numberwithin{equation}{section}

\hypertitle{thermo-MECA1855}{5}{MECA}{1855}
{Adrien Couplet}
{Miltiadis Papalexandris et Yann Bartosiewicz}[
\paragraph{Remarque} Ce document a pour objectif de rassembler et de répondre en détails aux questions de théorie du cours en vue
de l’examen. Celles-ci ont soit été fournies par M. Papalexandris, soit proviennent d'anciens examens. Le document suppose une compréhension préalable de la matière et ne fait office ni de synthèse, ni de syllabus.]
\newpage

\section{Questions sur le Chapitre ``Introduction et rappels''}
\subsection{Est-ce que la chaleur échangée entre un système et son extérieur est une variable d'état du système. Justifier votre réponse.}
Une \textbf{variable d'état} caractérise un état et non une évolution entre deux états. La chaleur échangée $Q$ entre un système et son extérieur décrit le passage d'un état à un autre et ne peux pas être mesurée instantanément. Les variables d'état d'un système thermodynamique sont par exemple la pression $p$, le volume $V$ ou encore la température $T$. 

\subsection{Donner la définition d'un processus \textit{adiabatique}. Donner un exemple d'un processus adiabatique réversible ainsi qu'un exemple d'un processus adiabatique irréversible.}
Une transformation est dite \textbf{adiabatique} si elle est effectué sans échange de chaleur ($Q = 0$). Cela n'implique pas pour autant que la température du système reste constante. La variation d'énergie mécanique est le seul échange avec l'extérieur du système qui modifie alors les variables d'états ($T$, $V$ et $p$).

Le concept de \textbf{réversibilité} est idéal. L'expression énergétique du second principe de la thermodynamique, aussi appelé principe d'évolution, s'écrit 
\begin{equation} Q + W_f = \int_1^2T dS \end{equation}
où $dS$ est la différentielle de la fonction d'état entropie et $W_f$ le terme dissipatif. Dans la réalité ce terme ne sera jamais nul. 
\begin{itemize}
	\item Un processus \textbf{réversible} est un processus où $W_f = 0$. Un processus adiabatique réversible est donc aussi isentropique car $\delta Q/T = dS = 0$. Un exemple d'un tel processus se trouve dans le cycle de Carnot qui comprend une détente ainsi qu'une compression adiabatique réversible. Il est à noter que le cycle de Carnot est un cycle théorique.
	\item Un processus \textbf{irréversible} est un processus où $W_f \neq 0$. On retrouve une détente adiabatique irréversible dans les cycle frigorifique. 
\end{itemize}

\subsection{Donner la définition mathématique ainsi que la signification physique des chaleurs massiques $c_p$ et $c_v$}
La capacité thermique massique $c$ correspond à la quantité d'énergie à apporter par échange thermique pour élever d''un kelvin la température de l'unité de masse d'une substance. C'est donc une grandeur intensive exprimé en joules par kilogramme-kelvin \si{\joule\per\kelvin\per\kilo\gram}. Si on note $U$ l'énergie interne, $H$ l'enthalpie et $m$ la masse d'un corps on a les capacités thermiques massiques :
\begin{itemize}
	\item à pression constante \begin{equation} c_p = \frac{1}{m}\left(\frac{\partial H}{\partial T}\right)_p \end{equation}
	\item à volume constant \begin{equation} c_v = \frac{1}{m}\left(\frac{\partial U}{\partial T}\right)_V \end{equation}
\end{itemize}

\subsection{Donner la définition de l'enthalpie libre $F$ de Helmholtz ainsi que de l'enthalpie libre $G$ de Gibbs. Écriver l'équation de Gibbs sous la forme de différentielle de $F$ ainsi que de $G$.}
\paragraph{L'énergie libre $F$ de Helmholtz} est une fonction d'état extensive dont la variation permet d'obtenir le travail utile susceptible d'être fourni par un système fermé, à température constante, au cours d'une transformation réversible. 
Considérons un système thermodynamique (fermé) évoluant d'un état 1 à un état 2, transformation que l'on suppose totalement réversible ($W_f = 0$) et à température constante $T$. Le premier principe de la thermodynamique s'exprime par :
\begin{equation} \Delta U + \Delta K + g\Delta z = Q + \underline{W_e} \end{equation}
où $\Delta U$ désigne la variation de l'énergie interne, $\Delta K$ la variation de l'énergie cinétique, $g\Delta z$ la variation de l'énergie potentielle, $Q$ l'échange de chaleur et $\underline{W_e}$ le travail des forces extérieures. On introduit $W = -W_e$ qui correspond au travail récupérable par le milieu extérieur et on suppose $\Delta K = g\Delta z = 0$. On obtient alors
\begin{equation} W = Q - \Delta U \label{eq:Q1ePrincipe}\end{equation}
Le 2\ieme principe de la thermodynamique s'exprime par:
\begin{equation} \Delta S = \frac{Q}{T} + W_f \qquad\Leftrightarrow\qquad Q = T\Delta S - TW_f \label{eq:Q2ePrincipe}\end{equation}
où $\Delta S$ est la variation d'entropie du système. Si on remplace (\ref{eq:Q2ePrincipe}) dans (\ref{eq:Q1ePrincipe}), nous trouvons
\begin{equation} W = T\Delta S - TW_f - \Delta U = T(S_2 - S_1) - TW_f + U_1 - U_2 \end{equation}
Il est évident, d'après la relation obtenue, que $W$ sera maximum si la transformation est totalement réversible ($W_f = 0$) :
\begin{equation} W_\text{max} = (U_1 - TS_1)-(U_2-TS_2)\end{equation}
On définit alors 
\begin{equation} F = U - TS \end{equation} 
Ce qui nous donne au final 
\begin{equation} W_\text{max} = -\Delta F \qquad\text{où}\qquad \underbrace{\Delta F}_\text{travail utile} = \underbrace{\Delta U}_\text{énergie totale} - \underbrace{T\Delta S}_\text{énergie inutilisable}\end{equation}

\paragraph{Forme différentielle de $F$}
\begin{equation} \left.\begin{array}{ll} dF &= d(U-TS) = dU - TdS - SdT \\ dU &= -pdV + TdS \end{array}\right\} \Rightarrow dF = -pdV - SdT\end{equation}

\paragraph{L'enthalpie libre $G$ de Gibbs} est une fonction d'état extensive dont la variation permet d'obtenir l'enthalpie utile susceptible d'être fournie par un système fermé, à température et à pression constante, au cours d'une transformation réversible. On considère le même système que précedemment avec la condition supplémentaire que la pression est constante. Par la définition de l'enthalpie :
\begin{equation} H \triangleq U + pV\end{equation}
De la même manière que pour l'énergie libre de Helmholtz, on trouve :
\begin{equation} W = Q - \Delta H \label{H1ePrincipe}\end{equation}
où $W$ correspond à nouveau au travail récupérable par le milieu extérieur. Si on remplace (\ref{eq:Q2ePrincipe}) dans (\ref{H1ePrincipe}) et qu'on prend on compte le fait que le processus est réversible, on obtient :
\begin{equation} W_\text{max} = (H_1 - TS_1) - (H_2 - TS_2) \end{equation}
On définit alors
\begin{equation} G = H - TS \end{equation}

\paragraph{Forme différentielle de G}
\begin{equation} \left.\begin{array}{ll} dG &= d(H-TS) = dH - TdS - SdT \\ dH &= Vdp + TdS \end{array}\right\} \Rightarrow dG = Vdp - SdT\end{equation}
\end{document}
