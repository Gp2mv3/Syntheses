\documentclass[fr]{../../../../../../eplexam}
\usepackage{diagbox,makecell}

\hypertitle{Probabilité et statistiques}{5}{FSAB}{1105}{2017}{Août}
{Martin Braquet\and Etudiants Bac 3 de 2016}
{Anouar El Ghouch et Rainer von Sachs}

\section{4 points}

\begin{enumerate}
	\item (1 point) The life span (in hundred of hours) of a light bulb follows a uniform distribution on [0,5].
	
	What is the probability that the average life span of a stock of 50 bulbs is smaller than 280 hours?
	\item (1.5 points) The remote of your television uses 2 batteries that work independtly. When both batteries are empty, the remote stops working.\\
	The life span (in hundreds of hours) is a r.v characterized by the following density function :
	\[ f(y) = \begin{cases} 
	\frac{3}{(1+y)^4} & y > 0 \\
	0 & \mathrm{otherwise}
	\end{cases}
	\]
	What is the probability that the remote is operational for more than 100 hours?
	\item (1.5 points) The university has to solve a very complex serie of equations. It uses then its 2 super computers named A and B in order to solve them.\\
	Probability that at least 1 of the 2 computers solves the equation = 0.85.\\
	Probability that both solve it = 0.25.\\
	Probability that A solves it = 0.7.\\
	What is the expected number of trials before B solves the equations for the first time (while A does not) ?
\end{enumerate}

\begin{solution}
\begin{enumerate}
	\item U $\sim$ [0,5] $\sim$ N($\frac{5}{2}$,$\frac{25}{12}$). On a que $\mu = \frac{a+b}{2} = \frac{0+5}{2}$; $\sigma^2 = \frac{(b-a)^2}{12}=\frac{25}{12}$.\\
	$$P(\overline{X}<2.8) = P\left(Z<\frac{(2.8-2.5)\sqrt{50}}{\sqrt{\frac{25}{12}}}\right) = P(Z<1.4697) = 1 - P(Z>1.47) = 1-0.0708 = 0.9292$$
	\item La probabilité qu'une pile soit plate avant 100 heures est donnée par la formule suivante:\\
	$$P(Y<1)=\int_{0}^{1} \frac{3}{(1+y)^4} dy = 0.875$$
	\\La télécommande ne fonctionne plus quand les deux piles sont plates : \\
	$P(Y_1<1,Y_2<1)= 0.875\cdot0.875 =0.765625$ (les deux événements sont indépendants)\\ 
	On a donc que $$P(Y_1>1,Y_2>1) = 1 - P(Y_1<1,Y_2<1) = 0.234375$$
	\item S = résoudre une équation. On sait que 
	$$P(B\cap \overline{A})=0.15$$
	A compléter.
\end{enumerate}

\end{solution}

\section{6 points}
2 cars arrive at a highway tollbooth and choose independently among 1 of the 3 toll gates : A, B or C.

We define $X_1$, $X_2$ and $X_3$ as being the number of cars passing gates A, B and C respectively.

\begin{enumerate}
	\item (1 point) Give the joint probability function of the variables $X_1$ and $X_2$. Detail your reasoning.
	\item (1 point) What is the probability that exactly 1 car passes through gate A given that at least 1 car has passed gate C?
	\item (1.5 points) Give the covariance between $X_1$ and $X_2$. Does it sound coherent to obtain a positive/negative/zero covariance? Comment. \\
	
	Let $Y_1$ and $Y_2$ denote the waiting times (in minutes) for the first and the second car respectively between the arrival of a car in a toll queue and the moment that car passes the gate. We suppose that $Y_1$ and $Y_2$ are independent and follow an exponential distribution with a mean of 5 minutes.
	\item (1.5 points) What is the probability that the total waiting time of 1 car is smaller than 10 minutes?
	\item (1 point) Compute the density function for the average time of the 2 cars. Justify your answer.
\end{enumerate}

\begin{solution}
\begin{enumerate}
	\item 3 tunnels et 2 voitures. La probabilité de rentrer dans un des tunnels est de $\frac{1}{3}$, on a ici une distribution Binomiale $X_i \sim Bi(2,\frac{1}{3})$.
	
	On établit les 9 possibilités: (A,A), (A,B),..., (C,C). On construit le tableau en associant chaque case à un certain nombre des 9 cas possibles:
	
	\begin{center}
		\begin{tabular}{|c|ccc|}
			\hline
			\diaghead{xxxxxxxxxx}{  $X_2$  }{  $X_1$  }	& 0 & 1 & 2  \\
			\hline
			0 & 1/9 & 2/9 & 1/9  \\
			1& 2/9 & 2/9 & 0  \\
			2& 1/9 & 0 & 0 \\
			\hline
		\end{tabular} 
	\end{center}
 
	\item $$P(X_1=1 | X_3\geq1) = \frac{P(X_1=1\cap X_3\geq1)}{P(X_3\geq1)}=\frac{2/9}{5/9}=2/5 $$
	\item $$\sigma_{X_1,X_2}=E[X_1 X_2]-E[X_1]E[X_2]=\frac{2}{9}-\frac{4}{9}\frac{4}{9}=0.0247$$
	La covariance est quasi nulle car les valeurs sont bien dispersées dans le plan $(X_1,X_2)$.
	\item $Y_1\sim \mathrm{Expo}(5)$, $$P(Y_1<10) = \int_{0}^{10} \frac{1}{5}e^{\frac{-y_1}{5}}\mathrm{d}y_1 = (e^2-1)e^{-2} = 0.864665 $$
	\item Une somme de variables suivant une exponentielle de paramètre 5 suit une
	$$Y_1+Y_2\sim\mathrm{Gamma}(2,5)$$
	Et la moitié d'une VA suivant une gamma suit une 
	$$\frac{Y_1+Y_2}{2}\sim \mathrm{Gamma}(2,5/2)$$
\end{enumerate}

\end{solution}
 
\section{5 points}

$X_1, X_2, ..., X_n$ are independent and identically distributed random variable with 

\begin{equation*}
f(x) = 
\frac{9x}{\theta ^2}e^{-3x/\theta}\:I(x>0)
\end{equation*}

(where $\theta >0$ is unknown)\\

Hint : $\int_0^{+\infty} x^{n}e^{-kx}dx = \frac{n!}{k^{n+1}}$ when $n$ is a positive integer and $k > 0$.

\begin{enumerate}
	\item (1 point) Determine $\hat{\theta}$ the MLE of $\theta$.
	\item (1.5 points) MSE of $\hat{\theta}$? Is it consistent?
	\item (1 points) Show that $\frac{6\bar{X}}{\theta}$ is a pivotal quantity.
	\item (1.5 points) Give a CI 95\% for $\theta$ based on $X_1, X_2, ..., X_n $ if $n = 10$ and $\overline{X}=5$.
\end{enumerate}

\begin{solution}
\begin{enumerate}
	\item $$L_n(\theta) = \frac{9^n}{\theta^{2n}} \left(\prod_i X_i\right) \exp\left({-\frac{3}{\theta}\sum_i X_i}\right)$$
	$$LL_n(\theta) = n \ln(\frac{9}{\theta^2})+\ln\left(\prod_i X_i\right) - \frac{3}{\theta}\sum_i X_i$$
	$$\frac{d}{d\theta} LL_n(\hat{\theta}) = 0 = -\frac{2n}{\hat{\theta}} + \frac{3}{\hat{\theta}^2}\sum_i X_i $$
	$$\hat{\theta} = \frac{3}{2} \frac{1}{n} \sum_i X_i = \frac{3}{2}\overline{X}$$
	Attention, il faut vérifier que la dérivée seconde soit négative aussi pour que ce soit bien un maximum et pas un point selle ou un minimum.
	$$\frac{d^2}{d\theta^2} LL_n(\hat{\theta}) = \frac{2n}{\hat{\theta}^2} - \frac{6}{\hat{\theta}^3}\sum_i X_i<0 \quad\Longrightarrow\quad \hat{\theta}<3\overline{X}\quad \mathrm{OK} $$
		
	\item E(X) = $\frac{2\theta}{3}$ et V(X) = $\frac{2\theta^2}{9}$. \\
	$$MSE(\hat{\theta}) = Bias(\hat{\theta})+V(\hat{\theta}) = 0 + \frac{9}{4}\cdot\frac{2\theta^2}{9n} = \frac{\theta^2}{2n}$$
	L'estimateur est consistent car Bias($\hat{\theta}$) = 0 et  $\lim_{n \rightarrow \infty} MSE = 0$.
	\item On cherche la fonction de densité de $g(X)= Y =\frac{6\bar{X}}{\theta}$
	$$m_X(t)=E[e^{tx}]=\frac{9}{\theta^2}\frac{1}{(\frac{3}{\theta}-t)^2}$$
	$$m_{\sum X_i}(t)=E[e^{t\sum X_i}]=\frac{9^n}{\theta^{2n}}\frac{1}{(\frac{3}{\theta}-t)^{2n}}$$
	$$m_{\sum Y}(t)=E[e^{\frac{6t}{n\theta}\sum X_i}]=m_{\sum X_i}(\frac{6t}{n\theta})=(1-\frac{2t}{n})^{-2n}$$
	Donc 
	$$Y\sim \mathrm{Gamma}(2n,\frac{2}{n})$$
	C'est un quantité pivot car elle ne dépend plus de $\theta$.
	\item On sait que $a$ et $b$ tels que 
	$P(a<Y<b)=0.95$
	répondent aux deux équations:
	\[
		\int_0^a\frac{y^{2n-1}e^{-ny/2}}{(2/n)^{2n}\Gamma(2n)}=0.025
	\]
		\[
	\int_b^{+\infty}\frac{y^{2n-1}e^{-ny/2}}{(2/n)^{2n}\Gamma(2n)}=0.025
	\]
  On déduit ainsi le CI $$P(a<Y<b)=P(a<\frac{6\overline{X}}{\theta}<b)=P(\frac{6\overline{X}}{b}<\theta<\frac{6\overline{X}}{a})$$
  \[
  	CI_{\theta}=\left[ \frac{6\overline{X}}{b};\frac{6\overline{X}}{a} \right]
  \]
\end{enumerate}

\end{solution}

\section{5 points}
$X \in \{60,62,64, 66\}$ is the dose of the drug and $Y_j$ is the response of patient number $j$.

\begin{center}
	\begin{tabular}{|c|c|c|c|c|}
		\hline
		& $X=60$ & $X=62$ & $X=64$ & $X=66$ \\
		\hline
		$Y_j$ & 115 & 125 & 145 & 140 \\
		& 125 & 140 & 135 & 135 \\
		& 120 & 130 & & 130\\
		& & 125 & & \\
		\hline
		$\sum _j Y_j$ & 360 & 520 & 280 & 405 \\
		\hline
		$\sum _j Y_j ^2$ & 43250 & 67750 & 39250 & 54725\\
		\hline
	\end{tabular} 
\end{center}

Significance level of $\alpha = 0.05$.

\begin{enumerate}
	\item (2 points) Given that $X$ is a categorical variable, can we say that the average of the $4^{th}$ group ($X=66$) is significantly larger than the average of the second group ($X=62$) ?\\
	Give the details of your statistical approach using (i) the critical value approach, (ii) the p-value approach and (iii) name the underlying (model) assumptions you use.
	\item (2 points) Still considering the dose $X$ as a categorical variable ; can we say that the mean is not the same for each of the 4 groups? Again, give the details of your statistical approach using (i) the critical value approach, (ii) the p-value approach and (iii) name the underlying (model) assumptions you use.
	\item (1 point) In this type of problem can we go beyond the categorical analysis, that is, let $X$ be a continuous r.v? If yes, propose an approach with a formal statistical procedure to show that the effect of the drug on the patient increases with the dose.\\ You are supposed to derive the statistical procedure in detail without performing the numerical calculations that are necessary for the final answer.
\end{enumerate}

\begin{solution}
\begin{enumerate}
	\item 
	On fait les hypothèses que les données sont issues de distributions normales ayant la même variance.
	
	On peut ainsi écrire que sous $H_0$,
	\[
		T=\frac{\bar{Y_4}-\bar{Y_2}}{S_p\sqrt{1/n_2+1/n_4}}\sim t_{n_2+n_4-2}
	\]
	où
	\[
		s_2^2=\frac{1}{3}\sum_i(Y_i-\bar{Y})^2=50
	\]
	\[
		s_4^2=\frac{1}{2}\sum_i(Y_i-\bar{Y})^2=25
	\]
	\[
	s_p^2=\frac{3s_2^2+2s_4^2}{5}=40
	\]
	On écrit les hypothèses: $H_0: \mu_4=\mu_2$ et $H_1: \mu_4>\mu_2$. On peut rejeter $H_0$ avec une certitude de $1-\alpha$ si la statistique de test
	\[
		t=\frac{\bar{y}_4-\bar{y}_2}{s_p\sqrt{1/n_2+1/n_4}}>t_{n_2+n_4-2,\alpha}
	\]
	car $P(T(y_2,y_4)>t_{n_2+n_4-2,\alpha}|H_0)=\alpha$. 
	
	Pour cette expérience, on a $t=1.035<t_{n_2+n_4-2,\alpha}=2.015$ et on ne peut donc pas établir de conclusion, le test est non-significatif.
	
	Par l'approche de la p-valeur, on a
	\[
		\mathrm{p-valeur}=P(t_{n_2+n_4-2}>t)=P(t_5>1.035)\in \{0.1;0.5\}
	\]
	Elle est supérieure à $\alpha$, on ne peut donc pas rejeter l'hypothèse nulle.
	\item 
	On peut rejeter $H_0: \mu_i=\mu_j$ avec $i,j=1,\ldots,4$ et $i\neq j$ avec une certitude de $1-\alpha$ si la statistique de test
	\[
	t=\left|\frac{\bar{y}_i-\bar{y}_j}{s_p\sqrt{1/n_i+1/n_j}}\right|>t_{n_i+n_j-2,\alpha/2}
	\]
	Si cette inégalité est vérifiée pour un des couples $(i,j)$, alors on peut dire que leur moyenne est différente.
	On a 
	\[
		s_1^2=25\qquad s_2^2=50\qquad s_3^2=50\qquad s_4^2=25
	\]
	\[
	\bar{y}_i=120\qquad \bar{y}_i=130\qquad \bar{y}_i=140\qquad \bar{y}_i=135
	\]
	On obtient ainsi les relations
	\[
		t_{12}=2.07<t_{5,\alpha/2}=2.571\qquad t_{13}=3.79>t_{3,\alpha/2}=3.182
	\] 
	\[
	t_{14}=3.67>t_{4,\alpha/2}=2.776\qquad t_{23}=1.63<t_{4,\alpha/2}=2.776
	\] 
	\[
	t_{24}=1.04<t_{5,\alpha/2}=2.571\qquad t_{34}=1.90<t_{3,\alpha/2}=3.182
	\] 
	On peut donc conclure qu'il y a une différence significative entre les moyennes des groupes 1 et 3 ainsi que 1 et 4.
	
	Les p-valeurs valent
	\[
	\mathrm{p-valeur}_{12}\in\{ 0.05;0.1 \}>\alpha\qquad \mathrm{p-valeur}_{13}\in\{ 0.02;0.05 \}<\alpha
	\] 
	\[
	\mathrm{p-valeur}_{14}\in\{ 0.02;0.05 \}<\alpha\qquad \mathrm{p-valeur}_{23}\in\{ 0.1;0.2 \}>\alpha
	\] 
	\[
	\mathrm{p-valeur}_{24}\in\{ 0.1;0.5 \}>\alpha\qquad \mathrm{p-valeur}_{34}\in\{ 0.1;0.2 \}>\alpha
	\] 
	
	\item 
	On calcule la moyenne et la variance des échantillons venant des groupes pour lesquels $X=x$ et $X=x+dx$. On réalise ensuite le test statistique 
	\[
		t=\frac{\bar{y}_{x+dx}-\bar{y}_{x}}{s_p\sqrt{1/n_{x}+1/n_{x+dx}}}>t_{n_x+n_{x+dx}-2,\alpha}
	\]
	Si cette condition est vérifiée pour tout $X$ dans l'intervalle de recherche, alors on peut établir que la réponse du patient augmente avec la dose de drogue.
	
\end{enumerate}

\end{solution}


\end{document}