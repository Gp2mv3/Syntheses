\documentclass[en]{../../../../../../eplexam}

\usepackage{../../../../../../eplunits}
\usepackage{../../../../stat-FSAB1105}

\hypertitle{Probabilité et Statistiques}{5}{FSAB}{1105}{2016}{Janvier}
{Cynthia Laureys}
{Anouar El Ghouch et Rainer von Sachs}

L'examen était en anglais et consistait en 5 questions dont la répartition des points était, respectivement: /3 , /7, /4, /2, /4, le tout /20.

%%%%%%%%%%
% QUESTION 1
%%%%%%%%%% 
\section{(/3)}
Soient deux variables aléatoires indépendantes X et Y qui suivent respectivement une loi
\begin{itemize}
    \item $X \sim \gammad(\alpha_1,\beta)$
    \item $Y \sim \gammad(\alpha_2,\beta)$
\end{itemize}
\begin{enumerate}
    \item \begin{enumerate}
    \item Quelle est la joint density function de $(U,V)=(X+Y,\frac{X}{X+Y})$? Est-ce que $U$ et $V$ sont indépendants? Justifier.
    \item Quelle est la distribution (laquelle?) de $U$ ? Quel est son nom? Quels sont ses paramètres? Justifier.
\end{enumerate}
\item Si $X$ est continue avec comme cumulative distribution function $F$, quelle est la distribution de $Y=-\ln{F(x)}$? Quel est son nom? Quels sont ses paramètres? Justifier.
\end{enumerate}

\begin{solution}
\begin{enumerate}
\item \begin{enumerate}
\item On sait que $X, Y \sim_{ind} Gamma(\alpha_i,\beta)$ avec $i$ valant respectivement 1 et 2. En outre, on connait $\textbf{U}=g(\textbf{X})=\begin{cases} U=X+Y \\V=\frac{X}{X+Y}\end{cases}$. Pour déterminer la distribution de $\textbf{U}=(U,V)$, il s'agit donc ici d'appliquer une \textit{One-to-one transformation} définie comme suit:
$$  f_\textbf{U}(\textbf{u})=\left.\frac{f_\textbf{X}(\textbf{x})}{|Jac_g(\textbf{x})|} \right|_{\textbf{x}=g^{-1}(\textbf{u})} \textrm { avec } U_i=g_i(X,Y) $$
\begin{itemize}
\item Déterminons la \textit{Continuous Joint Distribution} $f_\textbf{X}(\textbf{x})$ de $\textbf{X}=(X,Y)$. Puisque $X$ et $Y$ sont indépendantes, cette distribution peut s'écrire comme le produit de leur distribution respective:
$$X \Perp Y \Leftrightarrow f(x,y)=f_X(x)f_Y(y)$$
\begin{align*}
	f_\textbf{X}(\textbf{x})&=f_X(x)f_Y(y)\\
	&= \dfrac{x^{\alpha_1-1}e^{-x/\beta}}{\beta^{\alpha_1} \Gamma(\alpha_1)}\dfrac{y^{\alpha_2-1}e^{-y/\beta}}{\beta^{\alpha_2} \Gamma(\alpha_2)} && \textrm{ pour } x,y \geq 0\\
	&= \dfrac{x^{\alpha_1-1}y^{\alpha_2-1}e^{-(x+y)/\beta}}{\beta^{\alpha_1}\beta^{\alpha_2} \Gamma(\alpha_1)\Gamma(\alpha_2)}&& \textrm{ pour } x,y \geq 0\\
\end{align*}
\item Évaluons la valeur absolue du Jacobien $|Jac_g(\textbf{x})|$:
\begin{align*}
    |Jac_g(\textbf{x})|&= \begin{vmatrix}
    \frac{\partial g_1}{\partial x} &\frac{\partial g_1}{\partial y} \\ \frac{\partial g_2}{\partial x} & \frac{\partial g_2}{\partial y} \end{vmatrix} =\begin{vmatrix} 1 & 1 \\ \frac{y}{(x+y)^2} & -\frac{x}{(x+y)^2} \end{vmatrix}\\
    &= \left| \frac{\partial g_1}{\partial x}\frac{\partial g_2}{\partial y}-\frac{\partial g_1}{\partial y}\frac{\partial g_2}{\partial x} \right|= \left| - \frac{x+y}{(x+y)^2}\right|= \left|- \frac{1}{x+y}\right|= \frac{1}{x+y} && \textrm{ car } x,y \geq 0
\end{align*}
\item Calculons $g^{-1}(\textbf{U})$:
$$g(\textbf{X})=\begin{cases} U=X+Y \\V=\frac{X}{X+Y}\end{cases}\Leftrightarrow g^{-1}(\textbf{U})=\begin{cases} X=UV \\Y=U-UV\end{cases} $$
\end{itemize}   
On obtient par conséquent:
\begin{align*}
    f_\textbf{U}(\textbf{u})&=\left.\frac{f_\textbf{X}(\textbf{x})}{|Jac_g(\textbf{x})|} \right|_{\textbf{x}=g^{-1}(\textbf{u})} && \textrm { pour } u,v \geq 0\\
    &=\left. (x+y) \dfrac{x^{\alpha_1-1}y^{\alpha_2-1}e^{-(x+y)/\beta}}{\beta^{\alpha_1}\beta^{\alpha_2} \Gamma(\alpha_1)\Gamma(\alpha_2)} \right|_{\textbf{x}=g^{-1}(\textbf{u})} && \textrm { pour } u,v \geq 0\\
    &= u \dfrac{(uv)^{\alpha_1-1}(u-uv)^{\alpha_2-1}e^{-u/\beta}}{\beta^{\alpha_1}\beta^{\alpha_2} \Gamma(\alpha_1)\Gamma(\alpha_2)} && \textrm { pour } u,v \geq 0\\\end{align*}  
Pour prouver l'indépendance de $U$ et $V$, on peut utiliser le fait que si la \textit{Continuous Joint Distribution} peut s'écrire comme le produit de deux fonctions, l'une dépendant uniquement de $U$ et l'autre de $V$ alors elles seront indépendantes. Ce ne sera le cas que si les domaines de définition de $U$ et $V$ sont indépendants ce qui est bien le cas ici. On montre donc:
$$U \Perp V \Leftrightarrow f(u,v)=f_U(u)f_V(v)$$ où $f(u,v)=\frac{1}{\beta^{\alpha_1}\beta^{\alpha_2} \Gamma(\alpha_1)\Gamma(\alpha_2)} \left[u^{\alpha_1-\alpha_2-1}e^{-u/\beta}\right]\left[v^{\alpha_1-1}(1-v)^{\alpha_2-1} \right]$
\item On sait que $U=X+Y$ avec $X, Y \sim_{ind} Gamma(\alpha_i,\beta)$ pour $i$ valant respectivement 1 et 2. Il s'agit donc d'une somme de deux distributions Gamma indépendantes. On va donc se servir de la propriété de la M.G.F. suivante:
$$m_{\sum^n_{i=1} Y_i}(t)=\prod^n_{i=1} m_{Y_i}(t) \textrm{ avec }  Y_i \textrm{ mutuellement indépendants}$$ On connait les M.G.F. de $X$ et $Y$: $m_{X_i}(t)=(1-\beta t)^{-\alpha_i}$. On peut donc déterminer la M.G.F. de $U$:
\begin{align*}
	m_U(t)&=m_{\sum^2_{i=1} X_i}(t)=\prod^2_{i=1} m_{X_i}(t)\\
	&= m_{X}(t)m_{Y}(t)=(1-\beta t)^{-\alpha_1}(1-\beta t)^{-\alpha_2}=(1-\beta t)^{-(\alpha_1+\alpha_2)}
\end{align*}
L'expression de la M.G.F. nous permet d'identifier la distribution de $U$, en effet on reconnait la M.G.F. d'une distribution Gamma. En conclusion, $U\sim Gamma(\alpha_1+\alpha_2, \beta)$.
\end{enumerate}
\item Déterminer la distribution de $Y$ se fera en deux étapes. Il faut trouver la distribution de $U=F(X)$ et puis lui appliquer une \textit{Strictly Monotonic Transformation}.
\begin{itemize}
\item Commençons par déterminer la distribution de $U$. Puisque $F(x)$ est une \textit{Continuous Cumulative Distribution Function}, il s'agit d'une fonction strictement croissante. On peut donc appliquer une \textit{Strictly Monotonic Transformation} sur $X$:
\begin{align*}
	f_U(u)&=\left.\frac{f_X(x)}{|g'(x)|} \right|_{x=g^{-1}(u)} && \textrm{ pour } u \in g(I)\\
	\intertext{Dans ce cas-ci, la transformation appliquée est $g(x)=F(x)$. Par les propriétés d'une \textit{Continuous Cumulative Distribution Function}, on sait que $g'(x)=F'(x)=f_X(x)$ et que $g(I)=F(I)\in [0,1]$.}\\
	f_U(u)&=\left.\frac{f_X(x)}{|F'(x)|} \right|_{x=F^{-1}(u)} && \textrm{ pour } u \in F(I)\\
	&=\left.\frac{f_X(x)}{|f_X(x)|} \right|_{x=F^{-1}(u)} && \textrm{ pour } u \in [0,1]\\
	\intertext{Une \textit{Continuous Probability Density Function} étant toujours définie positive:}
	&=\left.\frac{f_X(x)}{f_X(x)} \right|_{x=F^{-1}(u)}=1 && \textrm{ pour } u \in [0,1]\\
\end{align*}
On reconnait une distribution Uniforme: $U \sim U(0,1)$. 
\item Maintenant que nous connaissons la distribution de $U$, nous pouvons lui appliquer à nouveau une \textit{Strictly Monotonic Transformation} puisque la fonction $g(U)=-\ln{U}$ est strictement décroissante:
\begin{align*}
	f_Y(y)&=\left.\frac{f_U(u)}{|g'(u)|} \right|_{u=g^{-1}(y)} && \textrm{ pour } y \in g(I)\\
	\intertext{Nous pouvons calculer que $g'(u)=-\frac{1}{u}$. De plus, nous savons que $f_U(u)=1$ $\forall u \in [0,1]$.}
	&=\left. u \right|_{u=g^{-1}(y)} && \textrm{ pour } y \geq 0\\
	\intertext{Si $g(u)=-\ln{u}$, alors $g^{-1}(y)=\exp{(-y)}$.}
	&=\exp{(-y)} && \textrm{ pour } y \geq 0
\end{align*}
\end{itemize}
On reconnait une distribution Gamma telle que $Y\sim Gamma(1,1)$ ou plus spécifiquement une distribution Exponentielle: $Y \sim Expo(1)$.
\end{enumerate}
\end{solution}

%%%%%%%%%%
% QUESTION 2
%%%%%%%%%% 
\section{(/7)}
Soient $X_1,\ldots,X_n$, des valeurs aléatoires dont la probability density function est
\begin{equation}
  f(x)=\frac{\alpha}{\beta}x^{\alpha-1}\exp{(\frac{-x^\alpha}{\beta})} \text{   avec }x>0
\end{equation}
$\alpha,\beta>0$ et $\alpha$ connu et $\beta$ inconnu
\begin{enumerate}
    \item Soit $\pi \in (0,1)$ donné, trouver $q_\pi$, le $\pi$-quantile de $X$ en fonction de $\alpha, \beta$
    \item Quelle est la distribution de $X^\alpha$ ? Quel est son nom, ses paramètres? Justifier?
    \item Quel est $\widehat{\beta}$, l'estimateur de $\beta$ ? Que vaut $\mse(\widehat{\beta})$? Est que $\widehat{\beta}$ est ``consistent''? En déduire $\widehat{q}_\pi$, un estimateur ``consistent'' pour $q_\pi$. Justifier
    \item Utiliser la mgf pour montrer que $n\widehat{\beta}\sim \gammad(n,\beta)$. En déduire la distribution de $\dfrac{2n\widehat{\beta}}{\beta}$. Justifier
    \item On veut tester $H_0$  : $\beta\leq\beta_0$ VS $H_1$ : $\beta>\beta_0$. Suggérez un test statistic \footnote{``Suggest a statistical test''} and dérivez-en la $p$-value. Justifiez chaque étape
    \item Donnez un intervalle de confiance de \SI{95}{\%} pour $q_\pi$
    \item Quelle est la distribution asymptotique de $\widehat{q}_\pi$ ? Donnez l'intervalle de confiance asymptotique de $q_\pi$
\end{enumerate}
\nosolution

%%%%%%%%%%
% QUESTION 3
%%%%%%%%%% 
\section{(/4)}
Soit une distribution jointe:
\begin{equation}
    f(x,y)=\dfrac{3y}{4} \text{ ($0\leq y \leq x \leq 2$)}
\end{equation}
\begin{enumerate}
    \item $F(Y|X=x)$ ?
    \item $P(Y>1/2|X=3/2)$ et $V(Y|X=3/2)$ ?
    \item $P(X+Y<2)$ ?
    \item $V(X-2Y)$ ?
\end{enumerate}
\nosolution

%%%%%%%%%%
% QUESTION 4
%%%%%%%%%% 

Les deux questions suivantes sont faites de tête, une aide à la restitution est la bienvenue!

\section{(/2)}
Des composants électroniques sont placés en parralèle. Il y en a $n_1$ qui suivent une distribution exponentielle (en nombre d'années) $\sim \expo(\beta_1)$ et $n_2$ dont la distribution est une exponentielle (en nombre d'années) $\sim \expo(\beta_2)$ avec $n_1+n_2=n$ et $n>2$. Sachant que le système ne fonctionne plus si TOUS les composants ne fonctionnent plus, quelle est la probabilité que le système dans sa globalité tiendra plus de 10 ans? Exprimer votre réponse en terme $\beta_1$, $\beta_2$, $n_1$ et $n_2$.
\nosolution

%%%%%%%%%%
% QUESTION 5
%%%%%%%%%% 
\section{(/4)}
Une population de composants a une probabilité de dysfonctionnement de \SI{8}{\%}.
\begin{enumerate}
    \item Si $n=12$, quelle est la probabilité que plus de 2 composants aient un défaut?
    \item si $n=200$, quell est la probabilité que moins de 20 composants aient un défaut?
    \item On teste les composants l'un a à la suite de l'autre. Quelle est la probabilité que l'on doive en tester plus de 3 avant de trouver le premier dysfonctionnant?
    \item On teste les composants l'un a à la suite de l'autre. Quelle est la probabilité que l'on doive en tester plus de 5 avant d'en trouver 3 dysfonctionnant?
\end{enumerate}
\nosolution

\end{document}
