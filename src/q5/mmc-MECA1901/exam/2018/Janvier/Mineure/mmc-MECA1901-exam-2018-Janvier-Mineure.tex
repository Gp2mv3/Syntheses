\documentclass[fr]{../../../../../../eplexam}
\usepackage{bm}

\hypertitle{Mécanique des milieux continus}{5}{MECA}{1901}{2018}{Janvier}{Mineure}
{Martin Braquet \and Olivier Leblanc}
{Issam Doghri et Philippe Chatelain}

\section{Théorie}

\subsection{Partie I. Doghri}

\begin{enumerate}
    \item Soit $\bm{A}$ un tenseur d'ordre 2 anti-symétrique et $\bm{u}$ un vecteur, tous les deux quelconques, mais non nuls.
    Répondez vrai ou faux aux 3 affirmations ci-dessous et justifiez pour chacune:
    \begin{itemize}
     \item $ A_{ij}A_{jk}$ est symétrique
     \item $A_{ij}A_{ji}>0$
     \item $u_iA_{ij}u_j>0$
    \end{itemize}
   
    \item Prouver que le tenseur de Green-Lagrange représente une variation de longueur dans la description lagrangienne.
    
\end{enumerate}

\end{document}

