\documentclass[en]{../../../../../../eplexam}

\usepackage{../../../mmc-MECA1901-exam}

\hypertitle{Mécanique des milieux continus}{5}{MECA}{1901}{2014}{Juin}
{Vincent Schellekens\and Antoine de Comité\and Aurélien Pignolet\and Mamadou Segpa\and Philippe Greiner}
{Philippe Chatelain et Issam Doghri}

\section{}
Interpréter le terme $D_{11}$ du tenseur des taux de déformation. Utiliser un dessin pour votre explication.

\begin{solution}
Rappelons d'abord la définition de $D_{11}$ en coordonnées cartésiennes :

\begin{equation}
D_{11}= \PDeriv{v_1}{x_1}
\end{equation}

Il s'agit du \textit{taux de dilatation linéaire} associé à la direction $\Base{1}$ et comporte un seul terme mesurant la variation de la vitesse en $\Base{1}$ selon $\Base{1}$.
\end{solution}


\section{}
Qu'est-ce que la trace d’un tenseur? Illustrez son utilisation dans deux concepts vus au cours (en cinématique, en dynamique,...).

\nosolution

\section{}
Illustrez le principe d’admissibilité (thermodynamique) dans son utilisation pour établir deux lois de comportement (en mécanique des fluides, élasticité, transfert de chaleur,...).

\begin{solution}
\mypar{La loi de Fourier (en thermoélasticité)}
Le principe d'admissiblité permet d'établir une loi de constituation  pour le flux de chaleur $\textbf{q}$. Pour respecter l'admissibilité, on doit respecter l'inégalité de Clausius-Duhem qui se simplifie en :

\begin{equation}
0 \geq \frac{1}{T}\textbf{q} \cdot \nabla T
\end{equation}

La façon la plus simple de garantir sytèmatiquement cette inégalité, pour toutes les valeurs de $\nabla T$, compte tenu du fait que $T > 0$, est de choisir $\textbf{q}$ comme suit (c'est la loi de Fourier pour le flux de chaleur) :

\begin{equation}
\textbf{q} = - k \nabla T
\end{equation}

Ou $k$ est une constante positive appelée la \textit{conductivité thermique}.

\mypar{Mécanique des fluides (écoulement incompressible et indilatable)}
Il ne faut pas oublier qu'on a $\nabla \cdot \textbf{v} = \mathrm{Tr}\uuline{D} = 0$. L'inégalité de Clauisius-Duhem est :
\begin{equation}
\rho T \PTDeriv{S}{t} - \rho \PTDeriv{e}{t} \geq \frac{1}{T}\textbf{q} \cdot \nabla T - \uuline{\sigma}:\uuline{D}
\end{equation}

Pour un fluide, on a (rappel $\uuline{I}:\uuline{A} = \mathrm{Tr}\uuline{A}$) :
\begin{equation}
e = \Integr{}{}{c_v}{T} \rightarrow \PTDeriv{e}{t} = c_v \PTDeriv{T}{t}
\end{equation}
\begin{equation}
S = \Integr{}{}{\frac{c_v}{T}}{T} \rightarrow \PTDeriv{S}{t} = \frac{c_v}{T} \PTDeriv{T}{t}
\end{equation}
\begin{equation}
\uuline{\sigma} = 2 \mu \uuline{D} - p\uuline{I} \rightarrow \uuline{\sigma}:\uuline{D} =  (2 \mu \uuline{D} - p\uuline{I}):\uuline{D} = 2\mu \uuline{D}:\uuline{D}
\end{equation}

Les deux termes du membre de gauche s'annulent donc mutuellement. En notation indicielle, on a finalement :
\begin{equation}
2 \mu (D_{ij}D_{ij}) + \frac{k}{T}\PDeriv{T}{x_i}\PDeriv{T}{x_i} \geq 0
\end{equation}

Pour garantir cette inégalité pour toutes les valeurs de $D_{ij}$ et $\PDeriv{T}{x_i}$ on a nécessairement : $k > 0$ (voir plus haut) et $\mu \geq 0$. Dans ce cas, l'admissibilité est garantie.
\end{solution}


\section{}
Enoncez le principe de Saint-Venant, et illustrez son utilité dans la résolution de problèmes en mécanique du solide.

\begin{solution}
Le principe de Saint-Venant nous dit qu'on peut remplacer la distribution des forces de contact imposées à une interface par une autre distribution statiquement équivalente (i.e. la somme (intégrale) est la même?) sans influencer de manière significative la solution à distance suffisamment grande de l'interface.

Par exemple, pour calculer la charge sur une poutre encastrée, on peut remplacer la distribution de forces de contact (à priori compliquée) par une force distribuée continuement sur toute la surface de l'extrémité de la poutre.
\end{solution}

\end{document}
