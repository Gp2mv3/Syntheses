\documentclass[fr]{../../../eplsummary}

\usepackage{csquotes} % environnement enquote

\hypertitle{Théorie des organisations}{5}{ECGE}{1317}
{Florian Thuin}
{Matthieu de Nanteuil}

\part{Syllabus et cours}

\section{Fiche introductive}

\subsection{Définitions}

\begin{description}
    \item[Organisation] Une configuration de personnes et de ressources
        créée pour coordonner une série d'activités professionnelles,
        dans le but d'atteindre des objectifs ou de produire des
        résultats ; ensemble de conduites \enquote{rationnelles}
        engagées dans des rapports mutuels, pour lequel il n'existe pas
        de théorie générale permettant de prédire le sens de cette
        combinaison.
    \item[Subjectivité] Rapport à l'individu en tant que le sujet
        éprouvant des expériences personnelles (par opposition aux
        objets), être vivant capable d'un mode de pensée autonome.
    \item[S'organiser] Déployer un principe de rationalité dans les
        relations humaines pour éviter le désordre.
    \item[Lumières]
    \item[Entreprise duale] Idée à la fin du XX\up{ème} siècle que la
        performance d'une entreprise peut se faire contre le bien-être
        des salariés.
    \item[Equivoque] Sujet à plusieurs interprétations différentes.
    \item[Utilitarisme]
    \item[Paradigme] Choix de problèmes à étudier et des techniques
        propres à leur étude. Un nouveau paradigme traduit l'émergence
        d'une nouvelle vision de la réalité.
    \item[Welfarisme] Un comportement est utile/rationnel s'il permet
        d'accéder à un niveau supérieur de bien-être (théorie du
        bien-être).
\end{description}
\bigskip

\subsection{Penseurs}

\begin{description}
    \item[\bsc{Hegel}] Approche mettant l'accent sur l'avènement d'un mode de
        pensée autonome grâce à des institutions qui conditionnent et
        garantissent une éthique universelle. Mise en lumière des liens
        entre travail, reconnaissance et citoyenneté. Développement
        d'une philosophie idéaliste basé sur la séparation entre
        \textit{travail de l'Esprit} et \textit{travail productif}. Met
        en avant l'unité substantielle entre le peuple et l'Etat et la
        nécessité d'une pondération entre ces deux parties pour éviter
        le despotisme ou l'anarchisme. L'être humain peut domestiquer la
        nature (mais il ne peut pas s'en émanciper), il peut devenir
        citoyen (mais pas s'émanciper des rapports de pouvoir). 
    \item[\bsc{Platon}] L'organisation s'oppose au monde des idées, elle
        n'est pas d'intérêt. 
    \item[\bsc{Kant}] Il est choqué par la situation d'hétéronomie dans
        l'autonomie et pensait que l'organisation ne méritait donc
        aucune attention particulière. \enquote{L'organisation n'est pas
        la vie}. Approche mettant l'accent sur l'avènement d'un mode de
        pensée autonome grâce à des institutions qui conditionnent et
        garantissent une éthique universelle .
    \item[Max \bsc{Weber}] A opéré la distinction entre rationnalité formelle
        et substantielle. Il s'intéresse à la bureaucratie car c'est
        l'acceptation de la domination par la raison. Cette
        \enquote{domination légale} l'inquiète car elle entraîne un
        nivellement des qualifications, la ploutocratisation et
        l'obéissance impersionnelle ; cependant il n'y apporte pas de
        solution.
    \item[Renaud \bsc{Sainsaulieu}] Montre l'existence de \textit{mondes
        sociaux} avec la mise en avant de \enquote{entreprise duale},
        \enquote{entreprise bureaucratique}, \enquote{entreprise
        communauté}, \enquote{entreprise en crise}. Il s'intéresse à
        l'entreprise privée comme institution (qui subit les attentes de
        transformation sociale et qui a une autonomie dans le
        changement, autrement dit un système social ouvert). Il met en
        avant la recherche de la performance sociale comme
        \enquote{renouveau de l'entreprise}. 
    \item[Hannah \bsc{Arendt}] Pense que la citoyenneté ne peut pas s'acquérir que
        par le travail car la persistence du chômage dans les sociétés
        modernes et l'existence de travails indignes supposent
        l'exclusion de certains citoyens de la \enquote{société de
        travailleurs}.
    \item[J. Habermas] Critique acerbe de l'Etat-providence.
    \item[\bsc{Marx}] \OE uvre pour l'abolition du régime de propriété privée,
        la révolution effective des rapports sociaux. Nie la possibilité
        d'une reconnaissance à partir du travail dans l'économie
        capitaliste.
    \item[A. \bsc{Honneth}] Souhaite une reconnaissance des travailleurs, pas
        seulement par la politique ou l'économie mais également pour
        l'image de soi. Chacun doit avoir sa place dans la société et
        toute organisation doit réaliser un traitement politique des
        revendications si elle souhaite créer une identité forte. Pense
        que l'Etat ne devrait pas être le lieu unique de la vie
        politique mais uniquement coordonner les orientations produites
        à l'extérieur de lui. 
\end{description}

\subsection{Contenu}

Toute organisation peut être décrite par :

\begin{enumerate}
    \item Un ou plusieurs buts
        \subitem Ceux-ci sont soit autodéterminés (une organisation
        humaine choisit ses buts ; parfois les buts liés aux
        organisations ne sont pas choisis par tous les membres, mais par
        une partie de ceux-ci voire même par une autre organisation)
        soit fixé pour une raison de survie (organisation d'une
        fourmilière). La finalité d'une organisation peut être
        économique (marchande\footnote{entreprise capitaliste} ou non
        marchande\footnote{entreprise publique, d'intérêt général}),
        politique, culturelle, sociale ou un mix de plusieurs de ces
        possibilités. Cette variété des buts n'a pas de conséquences
        directes sur l'organisation du travail, on peut voir des
        bureaucraties dans un contexte de marché et des formes
        entrepreneuriales dans les milieux associatifs (\enquote{la
        finalité ne détermine pas la structure de l'organisation}).
    \item Trois dimensions constitutives, notions de base
        \begin{itemize}
            \item \textbf{Rationalité} : L'organisation est une
                    affirmation très ancienne d'un principe de
                    rationalité : le but est de sortir du chaos, de
                    mettre de l'ordre. Si la philosophie a pour thème
                    principal la raison, l'organisation devrait y être
                    une question centrale. On distingue deux
                    rationalités (complémentaires en théorie, pas
                    toujours dans la pratique) qui ont deux rapports
                    distincts à la performance (on ne connait pas de
                    combinaison optimale des deux).

                    \subitem \textbf{Rationalité formelle} : On met
                    l'accent sur la cohérence des procédures de calcul
                    coût/bénéfice (organisation marchande) et sur les
                    règles de droit définies \textit{a priori}
                    (administration, bureaucratique). Univoque, durable
                    et contraignant. \textit{La performance comme
                    efficacité.}

                    \subitem \textbf{Rationalité substantielle} : On met
                    l'accent sur la coopération, les relations, les
                    interactions, l'ordre moral. Equivoque, rationnel,
                    cohérent mais hors des règles
                    économiques\footnote{mais pouvant dégager une
                    valeur} et de droit. \textit{La performance comme
                    visibilité.}
                \item \textbf{Travail} : L'organisation a pour but de
                    produire des biens et des services au sein d'une
                    collectivité. Le travail n'est pas uniquement lié à
                    une survie matérielle, mais également à la
                    politique, la famille, la culture, etc.\footnote{On
                    notera que l'amour ou la guerre ne sont pas
                    étrangers à certaines formes de travail, les tâches
                    ménagères ou l'armée en tant qu'emploi le prouvent}.
                \item \textbf{Subjectivité} : L'organisation coordonne
                    des actions et des liens, pas seulement des
                    ressources. L'organisation est façonnée par les
                    engagements réciproques et les décisions
                    individuelles et collectives de ses membres
                    (l'organisation comme \enquote{expérience
                    intersubjective}).
        \end{itemize}
\end{enumerate}
\bigskip

Une organisation est performante si :

\begin{enumerate}
    \item Elle atteint un haut degré d'efficacité (elle est conforme aux
        règles de calcul et de droit permettant d'être plus efficace que
        ses concurrentes - rationalité formelle).
    \item Elle rend visible les interactions sociales et modélise ces
        interactions (santé, bien-être, conditions de travail - ne peut
        pas se comparer directement par rapport à des concurrents).
\end{enumerate}

Un problème d'organisation concerne au moins une des 3 dimensions constitutives
de l'organisation mais souvent une articulation entre plusieurs de ces
dimensions. Il ne se limite pas à la rationalité formelle car ce n'est
pas un problème \textit{technique}.

\subsection{La cité grecque}

Dans la Grèce antique, le travail se passe en dehors de la Cité, il y a
une exclusion mutuelle entre le citoyen et le
travailleur\footnote{les tâches de production et de reproduction
empêchent le statut de citoyens : les femmes, les esclaves et les
artisans n'ont pas la possibilité d'avoir ce statut.}. Le travail
est considéré comme nécessaire, mais peu valorisé car considéré comme
empêchant de réfléchir à l'excellence éthique (car il faut être au-delà
de la nécessité pour être libre). \newline

Les grecs pensaient que le travail productif était la suite de
l'animalité et le mettaient complètement de côté pour mettre en avant le
politique. A l'inverse, l'économie politique moderne a tendance à
survaloriser la production, ce qui a pour effet de transformer le
politique en simple \textit{gestion administrative des inégalités}. Avec
l'Etat social (ou Etat-providence), on tente d'arriver à une
articulation entre ces deux situations : chacun possède les mêmes
droits malgré les inégalités sociales qui peuvent exister, la
participation à la production crée une \textit{identité sociale}. Le
travail devient un point de passage vers la citoyenneté (entrainant des
contraintes, donnant un statut et assurant la socialisation). \newline

C'est donc une forme d'organisation qui laisse une grande place à la
rationalité substantielle pour les citoyens. \newline

\subsection{L'abbaye médiévale}

Organisation à finalité religieuse qui intègre des préoccupations
économiques avec une hiérarchie sociale précise soumise à des lois
métaphysiques et religieuses. On y distingue deux types de but : \newline

\begin{enumerate}
    \item Les buts de mission : la finalité de l'organisation
    \item Les buts de système : la gestion et la hiérarchie
\end{enumerate}
\bigskip

A la différence de la Cité grecque, le travail est ici à l'intérieur de
l'abbaye.

\section{Fiche I}

\subsection{Définitions}

\begin{description}
    \item[Taylorisme] Analyse scientifique de l'organisation et de la
        division du travail.
    \item[Lumières] Mouvement de pensée défendant l'émancipation de la
        condition humaine pour contrer les despotismes et l'Inquisition
        (idéal de la raison autonome, promotion de la science et du
        libre-arbitre, matérialisme/hédonisme/utilitarisme, foi dans le
        progrès). Le mouvement anglais acceptait l'existence de Dieu,
        alors que le mouvement français était athéiste. Cette
        philosophie transforme l'accumulation de richesses en objectif
        moral légitime (richesse = succès = bien-être).
    \item[Utilitarisme] Au départ, philosophie morale qui défend l'idée
        d'une société juste est une société heureuse (welfariste,
        individualiste, calculatrice). L'utilitarisme
        vise à analyser les mécanismes à l'origine du bien-être d'une
        collectivité donnée (\textit{conséquentialiste}, ne propose pas
        de principes \textit{a priori}).
        \subitem Considère que chacun peut être défini en fonction de
        ses préférences (surtout matérielles), que le collectif est un
        agrégat d'actions individuelles et que les actions individuelles
        sont des parties d'équation. La maximisation des utilités
        individuelles entraîne la maximisation de l'utilité globale.
    \item[Positivisme scientifique] Idée que la science produira un ensemble de
        savoirs objectifs sur le monde, le conduisant à un avenir
        radieux. C'est une sorte de nouvelle religion qui remplacerait
        les précédentes.
\end{description}

\subsection{Penseurs}

\begin{description}
    \item[Adam Smith] Membre des Lumières anglaises. Impulseur de
        l'utilitarisme. Vision que la richesse est un bien désirable,
        que la valeur des choses correspond au temps de travail
        nécessaire pour les réaliser, nécessité de la division du
        travail, le désir d'enrichissement conduit à la satisfaction
        globale.
    \item[D. Hume] Membre des Lumières anglaises. La raison ne
        s'applique qu'aux faits, pas aux valeurs. C'est un sceptique
        radical.
    \item[J-S Mill] Défend une conception socialiste de l'utilitarisme
        dans laquelle l'Etat a un rôle à jouer pour assurer l'optimum
        global escompté en faisant des redistributions publiques si
        nécessaire.
    \item[Max Weber] Sociologue. Met en avant le fait que l'utilitarisme
        n'est pas adapté à l'entièreté de l'économie car l'économie
        n'est pas seulement le marché, mais également la redistribution
        publique et la réciprocité (organisation économique et citoyenne
        avec lien social fort). Il définit le concept univoque de
        rationalité formelle.
    \item[John Locke] Le travail fait un médiateur essentiel de
        l'identité humaine car il permet d'accéder à la propriété.
    \item[Emile Durkheim] Fondateur de la sociologie française. Critique
        la division du travail : à la fois la société industrielle
        pousse à l'anomie\footnote{caractéristique de l'état d'une
            société dont les normes réglant la conduite de l'humain et
        assurant l'ordre social apparaissent inefficientes} mais la
        division du travail permet l'intégration progressive des groupes
        sociaux et des complémentarités entre les personnes. Il jette
        les bases du passage d'une \textit{solidarité mécanique} à une
        \textit{solidarité organique} et de l'Etat-Providence.
    \item[Georges Friedmann] Critique de Durkheim : les différences
        entre ce qu'il avait prévu et la réalité sont importantes, il
        faut rajouter deux types de solidarité.
        \subitem \textbf{La solidarité d'entreprise}
        \subitem \textbf{La solidarité ouvrière}
    \item[Everett Hughes] Contre le courant fonctionnaliste. Il
        s'intéresse aux dirty jobs (les professions moins nobles) pour
        savoir s'ils arrivent à avoir un sens et à donner une identité
        professionnelle.
\end{description}

\subsection{Contenu}

La rationalité poussée à l'extrême et couplée à une réduction du travail
et une abolition du moral peut conduire à des organisations extrêmes de
type \textit{génocidaire} ou \textit{totalitaire}. \newline

Le projet des Lumières soutient la doctrine utilitariste à la fin du
18\up{ème} siècle qui décrit les modifications des types d'organisation
(apparition de la fabrique). \newline

L'utilitarisme se base sur des calculs d'utilité, autrement dit
nécessite des règles de calcul et exclut les formes de motivation
non-calculables comme les affects, les traditions, la culture, les
valeurs,\ldots Il exclut également le rôle de l'Etat dans l'économie.
\newline

L'économie de marché radicalise les présupposés utilitaristes dans 3
directions :

\begin{enumerate}
    \item \textbf{Formalisme} : La rationalité formelle est mise en
        avant. Dans l'économie de marché, on se focalise sur des
        arbitrages coûts/bénéfices en fonction des préférences. Le
        bien-être est alors uniquement matériel et issu d'un calcul
        objectif.
    \item \textbf{Egoïsme} : L'économie de marché se base sur la
        propriété privée et les calculs d'utilité. Les ressources étant
        limitées, on recherche la satisfaction d'un intérêt personnel et
        cela peut se faire au détriment des autres. C'est ici que se
        base la principale critique de Marx.
    \item \textbf{Optimisation marchande} : L'économie de marchés
        considère que le seul mode de coordination et d'agrégation des
        utilités individuelles qui garantisse un optimum  est le marché. 
\end{enumerate}

\bigskip
\begin{tabular}{|p{0.3\linewidth}|p{0.3\linewidth}|p{0.3\linewidth}|}
    \hline
    \textbf{Utilitarisme originel} (Lumières anglaises) &
    \textbf{Utilitarisme élargi} (Economie de marché) &
    \textbf{Caractéristiques de la rationalité} \\
    \hline
    \enquote{welfarisme} & formalisme économique &
    \multirow{3}{*}{\parbox{\linewidth}{Rationalité formelle et
    optimisatrice}} \\
    \cline{1-2}
    individualisme & égoïsme & \\ \cline{1-2}
    calcul d'agrégation & optimisation marchande (offre/demande/prix) &
    \\
    \hline
\end{tabular}
\bigskip

On notera que la seule rationalité qui existe est la rationalité
utilitaire-formelle, le concept de rationalité substantielle est encore
méconnu. Le travail (le temps) devient un bien mesurable comme un autre,
ce qui provoque l'abstraction de l'expérience des travailleurs. C'est la
base de l'engrenage qui consistera à rationaliser les moyens disponibles
pour la production dans le but de créer une plus-value sur les produits.

\subsubsection{Taylorisme - Organisation Scientifique du Travail}

Le Taylorisme est à la base du passage de la gestion de l'organisation
du travail en véritable matière scientifique. C'est un mouvement impulsé
par Fréderic Winslow Taylor qui consiste en un ensemble de méthodes et
de principes d'organisation de la production, rassemblés sous le terme
d'O.S.T.\footnote{Organisation scientifique du travail}. Jusqu'au
Taylorisme, l'organisation est considérée comme une \enquote{boîte
noire}, mais en se basant sur le projet d'Adam Smith au niveau du
marché, Taylor va étendre les calculs aux rapports sociaux de
production. Taylor pense que la croissance de la productivité assure aux
ouvriers un meilleur salaire, réconciliant ainsi travailleurs et
patrons\footnote{Il ira plus loin en déclarant que la volonté de ne pas
atteindre le maximum de la production est équivalent à un vol du
peuple}. \newline

Taylor ne considère pas que le marché soit suffisant pour assurer
l'optimalité, c'est pourquoi il faut étudier l'organisation du travail.
L'organisation scientifique du travail, c'est la norme scientifique
objective d'organisation des moyens de production. \newline

Les 3 principes de l'O.S.T. :

\begin{enumerate}
    \item La lutte contre la \enquote{flânerie} et les savoirs de
        métier. 
        \subitem $\Rightarrow$ Suppression des corporations, remplacement des
        travailleurs qualifiés par des travailleurs non qualifiés
        entraînant la baisse du pouvoir des syndicats
    \item Le contrôle des temps productifs et la parcellisation des
        tâches.
        \subitem $\Rightarrow$ Optimisation du temps de travail grâce à
        la standardisation des tâches (descriptibles, impersonnelles,
        divisibles), un commandement de
        l'apprentissage et un pouvoir légitime.
    \item La division du travail entre concepteurs et exécutants.
        \subitem $\Rightarrow$ Transformation du rapport de forces en
        science, la hiérarchie étant des \enquote{concepteurs} et les
        ouvriers des \enquote{exécutants}.
\end{enumerate}

\bigskip
\noindent
\begin{tabular}{|p{0.25\linewidth}|p{0.25\linewidth}|p{0.25\linewidth}|p{0.25\linewidth}|}
    \hline
    \textbf{Utilitarisme originel} (Lumières anglaises) &
    \textbf{Utilitarisme élargi} (Economie de marché) &
    \textbf{Utilitarisme organisationnel} (Taylorisme) &
    \textbf{Caractéristique(s) de la rationalité} \\
    \hline
    \enquote{welfarisme} & formalisme économique & formalisme
    organisationnel & \multirow{3}{*}{\parbox{\linewidth}{Rationalité
    formelle et optimisatrice}} \\
    \cline{1-3}
    individualisme & égoïsme & intérêt \enquote{bien compris} & \\
    \cline{1-3}
    calcul d'agrégation & optimisation marchande (offre/demande/prix) &
    organisation scientifique du travail & \\
    \hline
\end{tabular}
\bigskip

\subsubsection{Fordisme}

Le Fordisme va apparaître bien plus tard (vers 1930) mais va se baser
sur le taylorisme pour organiser la production de masse. Cependant, la
crise de 1929 et la croissance de la classe ouvrière entraîne un
problème de consommation de la production. Pour résoudre ce problème,
Ford va introduire un concept qui va permettre la consommation de masse
: le \enquote{Five Dollar Day}. Concrètement, les coûts de production
baissent (et donc le salaire des ouvriers aussi) et c'est cela qui
freine la consommation, alors Ford va leur donner un salaire avantageux
pour leur permettre de consommer. L'idée de Ford est également de
moraliser la consommation (pour que les ouvriers consomment bien les
biens qu'ils produisent). Cette période marque également le début de la
sécurité sociale, toujours pour permettre une régulation de la
consommation (diminution des risques de ne pas être payé : ça évite que
les gens épargnent trop et donc ça leur permet de consommer).
\subsubsection{La critique marxienne}

Marx critique la division du travail en 3 niveaux :

\begin{enumerate}
    \item \textbf{Le niveau philosophique} : pour Marx, le travail est
        l'expérience humaine par excellence, et enlever au travailleur
        la possibilité de le gérer et de choisir sa production est une
        formé d'\textit{aliénation} (l'homme est forcé à se vendre, il
        devient une machine). L'homme est indivisible, ce n'est pas un
        sujet comptable.
    \item \textbf{Le niveau économique} : le capitalisme et la division du
        travail entraîne un rapport d'exploitation car le but est de
        retirer du travail des autres une \textit{plus-value}. Dès lors,
        ça veut dire que tous les travailleurs sont sous-rémunérés par
        rapport à la valeur réelle de leur production.
    \item \textbf{Le niveau sociologique} : \textit{la lutte des
        classes}, une lutte inégale et permanente à cause de la
        domination sociale des capitalistes sur les prolétaires.
\end{enumerate}

Cette critique a comme limite que sa vision de l'organisation du travail
est assez stéréotypée ; qu'au final Marx accepte le bien-être matériel
comme une préférence ; elle ne critique que l'organisation marchande
capitaliste (pas la non-marchande par ex.).

\subsubsection{La critique fonctionnaliste ou structuro-fonctionnaliste}

Critique faite par Mintzberg notamment, sur les manquements et
l'inadaptation fonctionnelle (la sous-efficacité) du taylorisme. Il
existe en effet des \enquote{effets de système} qui ne sont pas
l'addition de comportements individuels.

\subsubsection{La critique structuraliste}

Critique faite par Foucault. Il transpose l'aliénation de Marx pour
faire la critique d'une société où il faut voir sans être vu, une
théorie du contrôle social. Ce ne sont plus les élites qui surveillent,
mais le collectif au sein même des organisations, au nom de l'efficacité.

\subsubsection{Le néo-utilitarisme}

Pour répondre aux critiques du manque de réalisme de l'utilitarisme,
deux directions ont été choisies :

\begin{enumerate}
    \item Il y a une reconnaissance des interactions mais toujours avec
        la possibilité de calcul (donc toujours critiqué par les
        fonctionnalistes) qui auparavant n'étaient pas théorisés
        (\enquote{calculs de transactions inter- ou
        intra-organisationnels}). \bsc{Williamson}
    \item Il y a un retour en arrière sur le lien entre les préférences
        et le bien-être matériel. On ne s'intéresse plus qu'aux
        préférences (radicalisation de l'individualisme, retour à la
        conception pré-taylorienne des organisations). \bsc{Coleman}
\end{enumerate}

Max \bsc{Weber} critique la confusion entre les moyens et les fins : ce
n'est pas parce qu'on veut arriver à un optimum qu'on doit tout calculer
sans prendre en compte les interactions (\enquote{rationalité
instrumentale} ou \enquote{autorité rationnelle-légale}). 

\section{Fiche II}

\subsection{Définitions}

\begin{description}
    \item[Fonctionnalisme] ou structuro-fonctionalisme : les structures
        organisationnelles sont un ensemble d'agencements collectifs
        momentanés en adaptation continue (pas de notion historique).
    \item[Structuraliste] ou post-structuraliste : explication des
        formes d'organisations via une analyse des processus
        inconscients ou invisibles (via l'histoire collective).
\end{description}

\subsection{Penseurs}

\begin{description}
    \item[Elton Mayo] Expérimentateur dont l'expérience à la Western
        Electric ont donné naissance à l'Ecole des relations humaines
        (qui consiste à mettre en avant la nécessite d'une bonne
        ambiance de travail, la coopération,\ldots)
    \item[Renaud Sainsaulieu] Uniquement utiliser l'utilité comme
        facteur de gestion est inefficace, il faut prendre en compte le
        facteur humain. Ce n'est pas irrationnel, il y a juste deux
        formes de rationalité : celle basée sur le calcul et celle basée
        sur l'interaction. 
    \item[Paul Lawrence] Structuro-fonctionaliste
    \item[Jay Lorsch] Structuro-fonctionaliste
    \item[Henry Mintzberg] Structuro-fonctionaliste. Pense que la
        domination n'a pas de statut scientifique, car le pouvoir a
        toujours une fonction particulière. Son analyse montre qu'un
        diagnostic \enquote{d'efficacité} des organisations peut être fait à travers l'analyse
        des rapports entre coordination et pouvoir. Le pouvoir
        correspond à la possibilité de permettre à l'organisation de
        s'adapter, en effet il pense qu'on peut sauver les organisations
        via l'adaptation à l'environnement.
    \item[Michel Foucault] Structuraliste. Pense que la domination
        existe au-delà des formes de coordination. Montre l'influence
        des transformations socio-politiques sur les structures
        organisationnelles. Il pense que le calcul rationnel cache la
        montée d'une société disciplinaire qui amène à une surveillance
        généralisée dans laquelle la subjectivité disparaitrait
        (\enquote{sujet assujetti}). Il pense que le pouvoir passe
        toujours par une forme de domination, qu'il \enquote{imprime sa
        marque sur nos propres corps} ; dès lors son souhait et que les
        sujets se sauvent par l'émancipation vis-à-vis de cette
        domination.
\end{description}

\subsection{Contenu}

\subsubsection{Paradoxe organisationnel}

Une organisation ne peut pas nier qu'elle doit respecter des règles
globales d'utilité (efficacité) pour assurer son existence. Mais se
baser sur les comportements individuels en supposant qu'ils sont
rationnels pour faire des calculs amènent généralement à l'inverse du
but : \textbf{la recherche d'efficacité collective passe par des dimensions
non-utilitaires}.

\subsubsection{Structuro-fonctionalisme}

Paul \bsc{Lawrence}, Jay \bsc{Lorsch}, Renaud \bsc{Sainsaulieu}, Henry
\bsc{Mintzberg}, Talcott \bsc{Parsons}.
\newline

La rationalité n'est plus une question de choix individuels, elle décrit
la capacité d'adaptation d'un collectif. Une organisation répond à des
stimulis, les différences de stimuli vont crééer différentes fonctions
et des structures d'intégration complexes. \newline

\bigskip
\begin{tabular}{p{0.25\linewidth}c|m{0.3\linewidth}|m{0.3\linewidth}|}
    & & \multicolumn{2}{c|}{Types de structures organisationnelles}\\
    & & \textbf{Structure simple} & \textbf{Structure complexe} \\ \cline{3-4}
    \multirow{2}{*}{\parbox[t]{\linewidth}{Degré de pression à la différenciation émanant de
    l'environnement}} &
    \parbox[t]{2mm}{\rotatebox[origin=c]{90}{\textbf{Fort}}} & Organisation hiérarchique et rigide & Organisation
    matricielle, participative et négociée \\ \cline{2-4}
    & \parbox[t]{2mm}{\rotatebox[origin=c]{90}{\textbf{Faible}}} & Organisation pyramidale et formaliste & Organisation
    \enquote{utopique}, idéologique \\ \cline{3-4}
\end{tabular}
\bigskip

Les 4 dimensions spécifiques de Mintzberg :

\begin{enumerate}
    \item Mise à jour progressive de la question de pouvoir ;
    \item Importance accordée à la coordination des activités
        individuelles ;
    \item Analyse du rapport entre coordination et pouvoir ;
    \item Analyse de plusieurs configurations organisationnelles.
\end{enumerate}
\bigskip

La réflexion éthique se fait sur l'adaptation à un environnement. Un
environnement possède 3 caractéristiques, il est :

\begin{enumerate}
    \item \textbf{Déterministe} : il inscrit les organisations dans une
        dépendance vis-à-vis d'un environnement donné et accepte toutes
        les finalités (l'enrichissement illimité est possible) ;
    \item \textbf{Pluriel} : il est possible d'avoir autre chose que des
        entreprises privées (reconnaissance de l'existence de structures
        administratives et des associations) ;
    \item \textbf{Exogène} : l'entreprise dépend de cet environnement,
        l'environnement n'est pas affecté par les entreprises.
\end{enumerate}
\bigskip

Les structuro-fonctionnalistes font sortir de l'ombre les rapports de
pouvoir internes, attribuant un contenu \textit{substantiel} aux
relations de pouvoir (\enquote{interdépendances politiques}). La
fonction des rapports de pouvoir est de contribuer à l'adaptation de
l'organisation à l'environnement (pas d'idée de domination), ils
remplissent un \textbf{objectif fonctionnel}. \newline

Le point de rupture entre structuro-fonctionalistes et tayloristes est
la \textbf{coordination}. La coordination correspond aux dispositifs
permettant de combiner des actions indépendantes et séparées (règles,
outils, procédures de concertation, réunions,\ldots). Permettre aux
exécutants de participer ou d'avoir des initiatives ou une réflexion à
propos des choix est l'inverse de ce que voulait le taylorisme (car il
voulait supprimer les corporations de métier qui avait justement ce
pouvoir de coordination). Taylor ne nie pas la coordination mais dit
qu'elle ne peut être que l'\oe uvre des ingénieurs-concepteurs par les
procédures de calcul, c'est une dérivée de la division du travail.  \newline

Mintzberg distingue 3 sources principales de pouvoir (de coordination) :

\begin{enumerate}
    \item \textbf{Les relations interpersonnelles} : présentes à tous
        les niveaux de l'organisation, elles sont complémentaires à la
        division du travail ;
    \item \textbf{La supervision hiérarchique} plutôt que le
        \textit{commandement}, c'est le début du management ;
    \item \textbf{La standardisation} sous 3 formes :
        \begin{enumerate}
            \item \textbf{La standardisation des procédures et des
                résultats} dans la même idée que Taylor ;
            \item \textbf{La standardisation des qualifications} :
                utilisation de la formation pour les ouvriers ;
            \item \textbf{La standardisation des modèles culturels} : la
                standardisation n'est possible que si les membres se
                reconnaissent dans un \textit{modèle culturel}.
        \end{enumerate}
\end{enumerate}
\bigskip

Mintzberg considère que le pouvoir réside dans la faculté des individus
ou des groupes à peser sur les décisions en mobilisant des ressources de
coordination. Mintzberg distingue 2 formes de pouvoir :

\begin{enumerate}
    \item \textbf{Le pouvoir comme accès à la décision} : possédé par
        les dirigeants, les actionnaires,\ldots dans le cadre des
        décisions générales ; parfois les employés dans les décisions
        techniques locales.
    \item \textbf{Le pouvoir comme possibilité d'influencer la décision}
        de manière formelle ou informelle. \enquote{Pouvoir d'influence}
\end{enumerate}
\bigskip

Mintzberg distingue 3 grandes familles d'acteurs :

\begin{enumerate}
    \item \textbf{Les acteurs externes} : actionnaires ou tutelle
        administrative (pas de pouvoir opérationnel) ;
    \item \textbf{Les acteurs internes liés à la décision formelle} :
        ligne hiérarchique ;
    \item \textbf{Les acteurs internes susceptibles d'influencer la
        décision} : possédant le pouvoir opérationnel, ils sont en 3
        groupes :
        \begin{enumerate}
            \item \textbf{les analystes de la technostructure} :
                financier, GRH, comptable ;
            \item \textbf{services logistiques} : nettoyage,
                restauration, sécurité (aujourd'hui souvent
                externalisés).
            \item \textbf{centre opérationnel} : les membres en fin de
                chaîne opérationnel, selon leurs fonctions et leurs
                qualifications.
        \end{enumerate}
\end{enumerate}

\bigskip
\noindent
\begin{tabular}{|p{0.25\linewidth}|p{0.25\linewidth}|p{0.25\linewidth}|p{0.25\linewidth}|}
    \hline
    \textbf{Utilitarisme originel} & \enquote{welfarisme} & individualisme &
    calcul d'agrégation \\
    \hline
    \textbf{Utilitarisme élargi} & formalisme économique & égoïsme & optimisation
    marchande (offre/demande/prix)\\
    \hline
    \textbf{Utilitarisme \mbox{organisationnel} (taylorisme)} & formalisme
    organisationnel & intérêt \enquote{bien compris} & organisation
    scientifique du travail \\
    \hline
    \textbf{Structuro-fonctionnalisme} & capacité collective d'adaptation à des
    environnements multiples mais exogènes & interdépendances et
    dynamiques collectives & pouvoir de décision et coordination
    (influences formelles et informelles sur la décision \\
    \hline
    \textbf{Caractéristique(s) de la rationalité} &
    \multicolumn{3}{c|}{Rationalité substantielle et adaptative
        (\enquote{quasi-optimisatrice})} \\
    \hline
\end{tabular}
\bigskip

\bsc{Mintzberg} distingue 5 grandes configurations (structures idéales
par rapport à différents environnements qui correspondent à une division
pouvoir/coordination différente) organisationnelles :

\begin{enumerate}
    \item \textbf{La configuration taylorienne ou bureaucratique} :
        forte division du travail, coordination par standardisation des
        procédures, pouvoir aux analystes de la technostructure.
    \item \textbf{La configuration entrepreunariale} : sous l'autorité
        d'un leader, peu de division du travail car peu d'employés,
        hiérarchie faible. Parfois présente au sein d'une configuration
        plus grande pour booster la créativité.
    \item \textbf{La configuration professionnelle} : grande firme
        industrielle ; division du travail et hiérachie importante mais
        standardisation par qualifications (engagement cognitif)
    \item \textbf{La configuration missionnaire} : ONG, start-up (forte
        présence d'un modèle culturel)
    \item \textbf{La configuration adhocratique} : temporaire pour un ou
        plusieurs projets, division du travail évolutive.
    \item \textbf{L'\enquote{organisation flexible}} : taylorienne et
        adhocratique.
\end{enumerate}
\bigskip

Le travail est une dimension relativisée au bénéfice des
\textit{fonctions et des rôles} : la coordination plutôt que la division
du travail. La subjectivité est réduite à être des \enquote{membres de
l'organisation} (ça oublie le fait qu'un même homme peut également
appartenir à une famille, des associations, etc.), les hommes sont des
entités fonctionnelles. Pour les structuro-fonctionnalistes, le sujet
est parfaitement socialisé, au point de se fondre dans la multiplicité
des exigences fonctionnelles que lui imposent son organisation (on
ignore les sentiments d'appartenance et les sentiments de plaisir ou de
souffrance) : il n'y a pas \textit{d'expérience vécue}.

\subsubsection{Structuralisme}

Michel \bsc{Foucault} met en avant le \textit{panoptique}, l'idée de la
surveillance généralisée par le collectif (effacement d'une figure
personnalisée de l'autorité - voir sans être vu), à l'opposé du pouvoir
qui devait se montrer auparavant. Le pouvoir est vue comme un pouvoir
disciplinaire/normatif. \newline

Les 3 principes de la surveillance généralisée selon Foucault :

\begin{enumerate}
    \item \textbf{Le recours à l'emploi du temps}
    \item \textbf{L'articulation \enquote{corps-geste-objet}} : le corps
        est vu comme un appareil de production comme un autre
    \item \textbf{L'utilisation exhaustive} : chacun doit utiliser tout
        ce qu'il possède (capacités, compétences,\ldots)
\end{enumerate}

\bigskip
\noindent
\begin{tabular}{|p{0.25\linewidth}|p{0.25\linewidth}|p{0.25\linewidth}|p{0.25\linewidth}|}
    \hline
    & \textbf{Rationalité} & \textbf{Subjectivité} & \textbf{Travail} \\
    \hline
    \textbf{Utilitarisme originel (Lumières anglaises)} & \enquote{welfarisme} & individualisme &
    calcul d'agrégation \\
    \hline
    \textbf{Utilitarisme élargi (Economie de marché)} & formalisme économique & égoïsme & optimisation
    marchande (offre/demande/prix)\\
    \hline
    \textbf{Utilitarisme \mbox{organisationnel} (taylorisme)} & formalisme
    organisationnel & intérêt \enquote{bien compris} & organisation
    scientifique du travail (lutte contre la flânerie, contrôle des
    temps, scission concepteurs/exécutants) \\
    \hline
    Caractéristique(s) de la rationalité &
    \multicolumn{3}{c|}{Rationalité formelle et optimisatrice} \\
    \hline
    \textbf{Structuro-fonctionnalisme} & capacité collective d'adaptation à des
    environnements multiples mais exogènes & interdépendances et
    dynamiques collectives & pouvoir de décision et coordination
    (influences formelles et informelles sur la décision \\
    \hline
    Caractéristique(s) de la rationalité &
    \multicolumn{3}{c|}{Rationalité substantielle et adaptative
        (\enquote{quasi-optimisatrice})} \\
    \hline
    \textbf{Critique structuraliste} & Dilution du pouvoir dans des
    mécanismes anonymes & Corps individuel rendu constamment visible et
    utile & Rapport de surveillance généralisée (emploi du temps,
    articulation corps/geste/objet, principe d'exhaustivité) \\
    \hline
    Caractéristique(s) de la rationalité &
    \multicolumn{3}{c|}{\parbox{0.65\linewidth}{Rationalité substantielle à
        \enquote{deux faces} (efficacité optimale et assujettissement)}} \\
    \hline
\end{tabular}
\bigskip

La critique structuraliste est une méthode systématique visant à montrer
la logique sourde de la surveillance généralisée. \bsc{Foucault}
préferera parler de \enquote{gouvernementalité} que de pouvoir, car on
peut résister à un gouvernement, mais on ne peut pas résister à un
pouvoir (on est absorbé par lui et ça se transforme en rapport de
forces). Le but est de dire aux sujets/membres qu'ils doivent sortir de
ce rapport de domination, ou du moins en diminuer les effets.\newline

\bsc{Foucault} sépare \enquote{subjectivité} (autonome et rationnelle,
se situer dans le monde en tant que personne) et
\enquote{subjectivation} (apprentissage de moyens de se jouer des
structures sans s'en échapper complètement).


\section{Fiche III}

\subsection{Définitions}

\begin{description}
    \item[Stratégie] Réactions au jour le jour qui forment des
        \textit{régularités} analysables empiriquement \textit{a
        posteriori}. 
    \item[Coordination] Relations générales entre entités prédéterminées
        par une structure d'interdépendance donnée.
    \item[Coopération] activité stratégique des acteurs en situation
        concrète. Celle-ci n'est pas dictée par une organisation, elle
        précède d'une décision individuelle ou collective.
\end{description}

\subsection{Penseurs}

\begin{description}
    \item[Michel Crozier] assiste à une consolidation de la
        bureaucratie comme phénomène de société (ce qui rend l'idée de
        Weber de la \enquote{bureaucratie nocive} est fausse). Il dit
        qu'elle peut être bonne si elle n'est pas trop centralisée et
        qu'elle s'appuie sur les points :
        \begin{itemize}
            \item stabilité
            \item régularité
            \item protection des individus
        \end{itemize}
        Il y a également dans ses pensées une certaine tolérance pour
        les gaspillages et les déviances. En plus des points
        précédents, une organisation doit être capable de concilier :
        \begin{itemize}
            \item liberté des individus
            \item développement de coopérations transversales
            \item standardisation à grande échelle
        \end{itemize}
        La \textit{réalité d'une organisation} est vue comme la façon
        dont les acteurs agissent ou n'agissent pas. 
    \item[Erhard Friedberg] a aidé Crozier à faire ses travaux. 
\end{description}

\subsection{Contenu}

Pour H. \bsc{Mintzberg}, l'analyse des structures du système
(\enquote{l'environnement}) permet aux organisations l'adaptation
permanentes de l'organsation à son environnement. Pour M.
\bsc{Foucault}, cette analyse permet de montrer les formes modernes de
domination et de contrôle dans les organisations. \newline

\subsubsection{Le paradigme stratégique}

Michel \bsc{Crozier}, Erhard \bsc{Friedberg}, James \bsc{March}, Herbert
\bsc{Simon}. \newline

La naissance de ce paradigme intervient dans un contexte d'essouflement
progressif du modèle d'organisation de masse. C'est avant tout une
critique du paradigme structurel, selon les analystes stratégiques, le
paradigme structurel a 3 gros défauts :

\begin{enumerate}
    \item Il donne naissance à une organisation dans laquelle les
        acteurs sont réduits à de simples positions fonctionnelles
        (\bsc{Mintzberg} - enfermement dans des positions \textit{a
        priori}) ou à des formes de subjectivation qui n'ont pas
        d'incidence directe sur le fonctionnement des structures
        (\bsc{Foucault}). Le pouvoir est conçu comme un attribut de la
        structure, non comme le résultat d'une relation ou d'une
        stratégie.
    \item Il a une conception passive de l'environnement (dépendance
        extrême de l'organisation vis-à-vis de celui-ci). Il est
        possible d'avoir de l'influence sur son environnement.
    \item Il suppose un état d'optimalité (une situation
        organisationnelle idéale), une sorte de rationalité illimitée.
        \bsc{Mintzberg} suppose l'adaptation optimale (rationalité
        \enquote{quasi-optimisatrice}). Les utilitaristes et les
        structuralistes (\bsc{Foucault}) sont d'accord avec cette idée.
\end{enumerate}

\subsubsection{Le concept de \enquote{rationalité limitée}}

La rationalité (individuelle ou collective) des acteurs présente selon
Foucault et les structuralistes 3 caractéristiques :

\begin{enumerate}
    \item l'information est parfaite
    \item les préférences sont stables, cohérentes et hiérarchisées
    \item les solutions alternatives sont toutes analysées
        (\enquote{raisonnement synoptique})
\end{enumerate}
\bigskip

On se rend compte que cette vision est irréaliste : le point de vue d'un
décideur est toujours limité, contextualisé. Il faut donc faire place
à une \enquote{rationalité limitée} (aussi bien d'un point de vue formel
que substantiel) qui ne prétend pas tout maîtriser.
La \textbf{rationalité} devient un ensemble de comportement permettant d'arriver
à une \textbf{solution satisfaisante} dans un \textbf{contexte donné}.
\newline

La prise en compte du contexte a 3 conséquences (conception pragmatique
de la raison) :

\begin{enumerate}
    \item l'accès à l'information est limité
    \item Les préférences des acteurs sont variables
    \item le raisonnement est sous contraintes
\end{enumerate}

\subsubsection{La conception de l'intéret}

L'intérêt est ici entendu sous la notion d'\textit{intérêt politique}.

Le résultat d'un comportement global est guidé par l'intérêt qui
représente les motifs de l'action. Cet intérêt ne peut donc plus être
calculé, ce qui a 2 conséquences : \newline

\begin{enumerate}
    \item impossibilité de la formalisation exhaustive des échanges
    \item l'importance du pouvoir comme finalité (calcul politique en
        vue d'accroître son pouvoir) :
        \begin{enumerate}
            \item le pouvoir comme désier d'influencer autrui
            \item le pouvoir comme volonté d'accroître son autonomie
        \end{enumerate}
\end{enumerate}

\bigskip
\noindent
\begin{tabular}{|p{0.25\linewidth}|p{0.25\linewidth}|p{0.25\linewidth}|p{0.25\linewidth}|}
    \hline
    & \textbf{Rationalité} & \textbf{Subjectivité} & \textbf{Travail} \\
    \hline
    \textbf{Utilitarisme originel (Lumières anglaises)} & \enquote{welfarisme} & individualisme &
    calcul d'agrégation \\
    \hline
    \textbf{Utilitarisme élargi (Economie de marché)} & formalisme économique & égoïsme & optimisation
    marchande (offre/demande/prix)\\
    \hline
    \textbf{Utilitarisme \mbox{organisationnel} (taylorisme)} & formalisme
    organisationnel & intérêt \enquote{bien compris} & organisation
    scientifique du travail (lutte contre la flânerie, contrôle des
    temps, scission concepteurs/exécutants) \\
    \hline
    Caractéristique(s) de la rationalité &
    \multicolumn{3}{c|}{Rationalité formelle et optimisatrice} \\
    \hline
    \textbf{Structuro-fonctionnalisme} & capacité collective d'adaptation à des
    environnements multiples mais exogènes & interdépendances et
    dynamiques collectives & pouvoir de décision et coordination
    (influences formelles et informelles sur la décision) \\
    \hline
    Caractéristique(s) de la rationalité &
    \multicolumn{3}{c|}{Rationalité substantielle et adaptative
        (\enquote{quasi-optimisatrice})} \\
    \hline
    \textbf{Critique structuraliste} & Dilution du pouvoir dans des
    mécanismes anonymes & Corps individuel rendu constamment visible et
    utile & Rapport de surveillance généralisée (emploi du temps,
    articulation corps/geste/objet, principe d'exhaustivité) \\
    \hline
    Caractéristique(s) de la rationalité &
    \multicolumn{3}{c|}{\parbox{0.65\linewidth}{Rationalité substantielle à
        \enquote{deux faces} (efficacité optimale et assujettissement)}} \\
    \hline
    \textbf{Analyse stratégique} & relations de coopération et de
    conflit sous contrainte (\enquote{systèmes d'action concrets}) &
    comportements rationnels des acteurs fondés sur une conception
    élargie de l'intérêt & pouvoir d'influence à tous les niveaux de
    l'organisation (relation imprévisible, déséquilibrée et non
    transitive) \\
    \hline
    Caractéristique(s) de la rationalité & 
    \multicolumn{3}{c|}{\parbox{0.65\linewidth}{Rationalité
        substantielle et stratégique (\enquote{limitée})}} \\
    \hline
\end{tabular}
\bigskip

\subsubsection{Le \enquote{triangle opératoire}}

Le triangle opératoire correspond à 3 grandes séparations de l'analyse
stratégique (qui consiste à étudier la dynamique des relations de
pouvoir) :

\begin{enumerate}
    \item \textbf{Les ressources} :
        \begin{itemize}
            \item capital économique
            \item informationnelles
            \item techniques
            \item professionnelles
            \item identitaires
        \end{itemize}
    \item \textbf{Les stratégies}
        \begin{itemize}
            \item Le comportement humain est toujours actif (même la
                passivité est une forme d'actiivté)
            \item Le comportement humain est toujours dirigé vers
                l'intérêt de manière empirique
            \item Le comportement est offensif (nouvelles opportunités)
                et défensif (maintien de sa marge de liberté).
        \end{itemize}
    \item \textbf{Le pouvoir} d'influence comme relation (pas comme
        attribut donné) permettant une relation favorable dans les
        limites de celui-ci, qui a plusieurs caractéristiques :
    \begin{itemize}
        \item ses conséquences sont imprévisibles et parfois
            dysfonctionnelles ;
        \item non-transitive (le pouvoir ne se transfère pas)
        \item réciproque mais déséquilibrée
    \end{itemize}
\end{enumerate}
\bigskip

On passe d'une logique de coordination à une \textbf{logique de
coopération} qui n'est pas désintéressée. En effet, la coopération est
le fruit d'un calcul politique pour Crozier et Friedberg. On coopère
soit pour influencer, soit pour se protéger de l'influence d'autrui.
\newline

\bsc{Grosjean} et \bsc{Lacoste} ont fait des études sur les milieux
hospitaliers, ils mettent en avant qu'une bonne communication est
isssue à la fois d'une bonne coordination et également d'une bonne
coopération. \newline

\subsubsection{La notion de régulation}

La notion de régulation est l'idée que les acteurs ne sont pas seulement
guidés par le pouvoir mais par la création de règles communes. Il y a
donc des règles qui viennent s'ajouter volontairement à l'ensemble des
relations stratégiques. \newline

J-D. \bsc{Reynaud} dit que les acteurs ne jouent pas \textit{dans} le
système mais \textit{avec} le système. L'idée est que les groupes de
personnes sont capables de structurer leurs relations à travers des
règles collectives (compromis entre \enquote{autonomie} et
\enquote{contrôle}) qui vont guider les choix individuels. L'ambition
des acteurs va être à la fois de tirer profit de l'échange mais
également de participer à la définition des règles de l'échange
(\enquote{régulation conjointe}).\newline

\part{La rationalité du mal et Si c'est un homme}

\begin{description}
    \item[Lévi] pense que le souci constant d'efficacité entraîne le
        fait que la responsabilité morale devienne une responsabilité
        technique (c'est la rationalité/modernité du mal). Il met en
        avant l'importance de l'éthique dans les organisations.
        \textit{Survivant des camps de concentration}
    \item[Bauman] Selon lui, la division du travail abolit le sens
        moral (les gens ne réfléchissent plus à la finalité de leurs
        actes quand ils entreprennent des petites actions).
        \textit{Sociologue}
\end{description}

Il est important de noter que c'est dans le cadre d'une bureaucratie
engagée dans une action logique et rationnelle que le projet d'une
société \enquote{idéale}, racialement purifiée a pu prendre place.
\newline

Dans cette bureaucratie, on peut analyser la division du travail et la
parcellisation des tâches pour élaborer l'extermination du
\enquote{peuple} juif dont voici les étapes (\enquote{solution finale}) :

\begin{enumerate}
    \item Propagande du pouvoir contre les juifs (stéréotypes) ;
    \item Désocialisation et écartement des juifs (interdiction de fréquenter certains
        lieux et port de l'étoile jaune de David) ;
    \item Déportation des juifs dans des ghettos proches dans les
        grandes villes (entrainant la désolidarisation des juifs entre
        eux car chacun lutte pour sa survie) ;
    \item Déportation des juifs depuis les ghettos vers des camps de
        concentration (de travaux forcés) dans lesquels un tri est fait
        entre les personnes aptes à travailler et les autres (on
        supprime la subjectivité des personnes en les rasant, prenant
        tout objet personnel, uniforme de détenus, chacun a un numéro
        qui sera la seule manière dont il sera identifié) ;
    \item Proposition à certains juifs de dominer les autres en
        devenant \enquote{kapo}, leur permettant un traitement de faveur
        en échange de leur aide à l'encadrement des prisonniers ;
    \item Application d'une charge importante de travail et d'un rythme
        ininterrompu d'action, empêchant les communications entre les
        personnes, poussant à l'individualisme (pour assurer sa survie)
        et empêchant toute réflexion (élimination systématique des
        travailleurs qui ne sont plus aptes à produire);
    \item Application de la \enquote{solution finale}, l'extermination
        dans des chambres à gaz, la crémation des morts pour éliminer
        totalement les traces de ce peuple (lorsqu'il y a trop de
        personnes qui sont mortes, on les enterre dans des fosses
        communes).
\end{enumerate}
\bigskip

La force de ce découpage des actions est qu'elles peuvent être réalisées
par des personnes différentes qui n'ont pas forcément de contact entre
elles : aucune n'a l'impression d'être coupable du sort des personnes
qu'elles gèrent et la lutte pour la survie laisse peu de place
aux choix pour les victimes (soit elles coopèrent, soit elles meurent). 

\end{document}
