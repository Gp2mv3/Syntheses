\documentclass[fr]{../../../eplsummary}

\usepackage{csquotes} % environnement enquote
\usepackage{slashbox} % commande \backslashbox

\hypertitle{Théorie des organisations}{5}{ECGE}{1317}
{Florian Thuin}
{Matthieu de Nanteuil}

\section{Fiche introductive}

\subsection{Définitions}

\begin{description}
    \item[Organisation] Une configuration de personnes et de ressources
        créée pour coordonner une série d'activités professionnelles,
        dans le but d'atteindre des objectifs ou de produire des
        résultats ; ensemble de conduites \enquote{rationnelles}
        engagées dans des rapports mutuels, pour lequel il n'existe pas
        de théorie générale permettant de prédire le sens de cette
        combinaison.
    \item[Subjectivité] Rapport à l'individu en tant que le sujet
        éprouvant des expériences personnelles (par opposition aux
        objets), être vivant capable d'un mode de pensée autonome.
    \item[S'organiser] Déployer un principe de rationalité dans les
        relations humaines pour éviter le désordre.
    \item[Rationalité]
    \item[Hétéronomie] 
    \item[Lumières]
    \item[Entreprise duale] Idée à la fin du XX\up{ème} siècle que la
        performance d'une entreprise peut se faire contre le bien-être
        des salariés.
    \item[Equivoque] Sujet à plusieurs interprétations différentes.
    \item[Utilitarisme]
    \item[Paradigme] Choix de problèmes à étudier et des techniques
        propres à leur étude. Un nouveau paradigme traduit l'émergence
        d'une nouvelle vision de la réalité.
\end{description}
\bigskip

\subsection{Penseurs}

\begin{description}
    \item[Charles Taylor] 
    \item[Hegel] Approche mettant l'accent sur l'avènement d'un mode de
        pensée autonome grâce à des institutions qui conditionnent et
        garantissent une éthique universelle. Mise en lumière des liens
        entre travail, reconnaissance et citoyenneté. Développement
        d'une philosophie idéaliste basé sur la séparation entre
        \textit{travail de l'Esprit} et \textit{travail productif}. Met
        en avant l'unité substantielle entre le peuple et l'Etat et la
        nécessité d'une pondération entre ces deux parties pour éviter
        le despotisme ou l'anarchisme. L'être humain peut domestiquer la
        nature (mais il ne peut pas s'en émanciper), il peut devenir
        citoyen (mais pas s'émanciper des rapports de pouvoir). 
    \item[Platon] L'organisation s'oppose au monde des idées, elle
        n'est pas d'intérêt. 
    \item[Kant] Il est choqué par la situation d'hétéronomie dans
        l'autonomie et pensait que l'organisation ne méritait donc
        aucune attention particulière. \enquote{L'organisation n'est pas
        la vie}. Approche mettant l'accent sur l'avènement d'un mode de
        pensée autonome grâce à des institutions qui conditionnent et
        garantissent une éthique universelle .
    \item[Adam Smith]
    \item[Frederic Winslow Taylor]
    \item[Max Weber] A opéré la distinction entre rationnalité formelle
        et substantielle.
    \item[Renaud Sainsaulieu] Montre l'existence de \textit{mondes
        sociaux} avec la mise en avant de \enquote{entreprise duale},
        \enquote{entreprise bureaucratique}, \enquote{entreprise
        communauté}, \enquote{entreprise en crise}.
    \item[H. Arendt] Pense que la citoyenneté ne peut pas s'acquérir que
        par le travail car la persistence du chômage dans les sociétés
        modernes et l'existence de travails indignes supposent
        l'exclusion de certains citoyens de la \enquote{société de
        travailleurs}.
    \item[J. Habermas] Critique acerbe de l'Etat-providence.
    \item[Marx] \OE uvre pour l'abolition du régime de propriété privée,
        la révolution effective des rapports sociaux. Nie la possibilité
        d'une reconnaissance à partir du travail dans l'économie
        capitaliste.
    \item[A. Honneth] Souhaite une reconnaissance des travailleurs, pas
        seulement par la politique ou l'économie mais également pour
        l'image de soi. Chacun doit avoir sa place dans la société et
        toute organisation doit réaliser un traitement politique des
        revendications si elle souhaite créer une identité forte. Pense
        que l'Etat ne devrait pas être le lieu unique de la vie
        politique mais uniquement coordonner les orientations produites
        à l'extérieur de lui. 
\end{description}

\subsection{Contenu}

Toute organisation peut être décrite par :

\begin{enumerate}
    \item Un ou plusieurs buts
        \subitem Ceux-ci sont soit autodéterminés (une organisation
        humaine choisit ses buts ; parfois les buts liés aux
        organisations ne sont pas choisis par tous les membres, mais par
        une partie de ceux-ci voire même par une autre organisation)
        soit fixé pour une raison de survie (organisation d'une
        fourmilière). La finalité d'une organisation peut être
        économique (marchande\footnote{entreprise capitaliste} ou non
        marchande\footnote{entreprise publique, d'intérêt général}),
        politique, culturelle, sociale ou un mix de plusieurs de ces
        possibilités. Cette variété des buts n'a pas de conséquences
        directes sur l'organisation du travail, on peut voir des
        bureaucraties dans un contexte de marché et des formes
        entrepreneuriales dans les milieux associatifs (\enquote{la
        finalité ne détermine pas la structure de l'organisation}).
    \item Trois dimensions constitutives, notions de base
        \begin{itemize}
            \item \textbf{Rationalité} : L'organisation est une
                    affirmation très ancienne d'un principe de
                    rationalité : le but est de sortir du chaos, de
                    mettre de l'ordre. Si la philosophie a pour thème
                    principal la raison, l'organisation devrait y être
                    une question centrale. On distingue deux
                    rationalités (complémentaires en théorie, pas
                    toujours dans la pratique) qui ont deux rapports
                    distincts à la performance (on ne connait pas de
                    combinaison optimale des deux).

                    \subitem \textbf{Rationalité formelle} : On met
                    l'accent sur la cohérence des procédures de calcul
                    coût/bénéfice (organisation marchande) et sur les
                    règles de droit définies \textit{a priori}
                    (administration, bureaucratique). Univoque, durable
                    et contraignant.

                    \subitem \textbf{Rationalité substantielle} : On met
                    l'accent sur la coopération, les relations, les
                    interactions, l'ordre moral. Equivoque, rationnel,
                    cohérent mais hors des règles
                    économiques\footnote{mais pouvant dégager une
                    valeur} et de droit.
                \item \textbf{Travail} : L'organisation a pour but de
                    produire des biens et des services au sein d'une
                    collectivité. Le travail n'est pas uniquement lié à
                    une survie matérielle, mais également à la
                    politique, la famille, la culture, etc.\footnote{On
                    notera que l'amour ou la guerre ne sont pas
                    étrangers à certaines formes de travail, les tâches
                    ménagères ou l'armée en tant qu'emploi le prouvent}.
                \item \textbf{Subjectivité} : L'organisation coordonne
                    des actions et des liens, pas seulement des
                    ressources. L'organisation est façonnée par les
                    engagements réciproques et les décisions
                    individuelles et collectives de ses membres
                    (l'organisation comme \enquote{expérience
                    intersubjective}).
        \end{itemize}
\end{enumerate}
\bigskip

Une organisation est performante si :

\begin{enumerate}
    \item Elle atteint un haut degré d'efficacité (elle est conforme aux
        règles de calcul et de droit permettant d'être plus efficace que
        ses concurrentes - rationalité formelle).
    \item Elle rend visible les interactions sociales et modélise ces
        interactions (santé, bien-être, conditions de travail - ne peut
        pas se comparer directement par rapport à des concurrents).
\end{enumerate}

Un problème d'organisation concerne au moins une des 3 dimensions constitutives
de l'organisation mais souvent une articulation entre plusieurs de ces
dimensions. Il ne se limite pas à la rationalité formelle car ce n'est
pas un problème \textit{technique}.

\subsection{La cité grecque}

Dans la Grèce antique, le travail se passe en dehors de la Cité, il y a
une exclusion mutuelle entre le citoyen et le
travailleur\footnote{les tâches de production et de reproduction
empêchent le statut de citoyens : les femmes, les esclaves et les
artisans n'ont pas la possibilité d'avoir ce statut.}. Le travail
est considéré comme nécessaire, mais peu valorisé car considéré comme
empêchant de réfléchir à l'excellence éthique (car il faut être au-delà
de la nécessité pour être libre). \newline

Les grecs pensaient que le travail productif était la suite de
l'animalité et le mettaient complètement de côté pour mettre en avant le
politique. A l'inverse, l'économie politique moderne a tendance à
survaloriser la production, ce qui a pour effet de transformer le
politique en simple \textit{gestion administrative des inégalités}. Avec
l'Etat social (ou Etat-providence), on tente d'arriver à une
articulation entre ces deux situations : chacun possède les mêmes
droits malgré les inégalités sociales qui peuvent exister, la
participation à la production crée une \textit{identité sociale}. Le
travail devient un point de passage vers la citoyenneté (entrainant des
contraintes, donnant un statut et assurant la socialisation).

\subsection{L'abbaye médiévale}

Organisation à finalité religieuse qui intègre des préoccupations
économiques avec une hiérarchie sociale précise soumise à des lois
métaphysiques et religieuses.

\section{Fiche I}

\subsection{Définitions}

\begin{description}
    \item[Taylorisme] Analyse scientifique de l'organisation et de la
        division du travail.
    \item[Lumières] Mouvement de pensée défendant l'émancipation de la
        condition humaine pour contrer les despotismes et l'Inquisition
        (idéal de la raison autonome, promotion de la science et du
        libre-arbitre, matérialisme/hédonisme/utilitarisme, foi dans le
        progrès). Le mouvement anglais acceptait l'existence de Dieu,
        alors que le mouvement français était athéiste. Cette
        philosophie transforme l'accumulation de richesses en objectif
        moral légitimei (richesse = succès = bien-être).
    \item[Utilitarisme] Au départ, philosophie morale qui défend l'idée
        d'une société juste est une société heureuse (welfariste,
        individualiste, calculatrice). L'utilitarisme
        vise à analyser les mécanismes à l'origine du bien-être d'une
        collectivité donnée (\textit{conséquentialiste}, ne propose pas
        de principes \textit{a priori}).
        \subitem Considère que chacun peut être défini en fonction de
        ses préférences (surtout matérielles), que le collectif est un
        agrégat d'actions individuelles et que les actions individuelles
        sont des parties d'équation. La maximisation des utilités
        individuelles entraîne la maximisation de l'utilité globale.
    \item[Positivisme scientifique] Idée que la science produira un ensemble de
        savoirs objectifs sur le monde, le conduisant à un avenir
        radieux. C'est une sorte de nouvelle religion qui remplacerait
        les précédentes.
\end{description}

\subsection{Penseurs}

\begin{description}
    \item[Adam Smith] Membre des Lumières anglaises. Impulseur de
        l'utilitarisme. Vision que la richesse est un bien désirable,
        que la valeur des choses correspond au temps de travail
        nécessaire pour les réaliser, nécessité de la division du
        travail, le désir d'enrichissement conduit à la satisfaction
        globale.
    \item[D. Hume] Membre des Lumières anglaises. La raison ne
        s'applique qu'aux faits, pas aux valeurs.
    \item[J-S Mill] Défend une conception socialiste de l'utilitarisme
        dans laquelle l'Etat a un rôle à jouer pour assurer l'optimum
        global escompté en faisant des redistributions publiques si
        nécessaire.
    \item[Max Weber] Sociologue. Met en avant le fait que l'utilitarisme
        n'est pas adapté à l'entièreté de l'économie car l'économie
        n'est pas seulement le marché, mais également la redistribution
        publique et la réciprocité (organisation économique et citoyenne
        avec lien social fort). Il définit le concept univoque de
        rationalité formelle.
\end{description}

\subsection{Contenu}

La rationalité poussée à l'extrême et couplée à une réduction du travail
et une abolition du moral peut conduire à des organisations extrêmes de
type \textit{génocidaire} ou \textit{totalitaire}. \newline

Le projet des Lumières soutient la doctrine utilitariste à la fin du
18\up{ème} siècle qui décrit les modifications des types d'organisation
(apparition de la fabrique). \newline

L'utilitarisme se base sur des calculs d'utilité, autrement dit
nécessite des règles de calcul et exclut les formes de motivation
non-calculables comme les affects, les traditions, la culture, les
valeurs,\ldots Il exclut également le rôle de l'Etat dans l'économie.
\newline

L'économie de marché radicalise les présupposés utilitaristes dans 3
directions :

\begin{enumerate}
    \item \textbf{Formalisme} : La rationalité formelle est mise en
        avant. Dans l'économie de marché, on se focalise sur des
        arbitrages coûts/bénéfices en fonction des préférences. Le
        bien-être est alors uniquement matériel et issu d'un calcul
        objectif.
    \item \textbf{Egoïsme} : L'économie de marché se base sur la
        propriété privée et les calculs d'utilité. Les ressources étant
        limitées, on recherche la satisfaction d'un intérêt personnel et
        cela peut se faire au détriment des autres. C'est ici que se
        base la principale critique de Marx.
    \item \textbf{Optimisation marchande} : L'économie de marchés
        considère que le seul mode de coordination et d'agrégation des
        utilités individuelles qui garantisse un optimum  est le marché. 
\end{enumerate}

\bigskip
\begin{tabular}{|p{0.3\linewidth}|p{0.3\linewidth}|p{0.3\linewidth}|}
    \hline
    \textbf{Utilitarisme originel} (Lumières anglaises) &
    \textbf{Utilitarisme élargi} (Economie de marché) &
    \textbf{Caractéristiques de la rationalité} \\
    \hline
    \enquote{welfarisme} & formalisme économique &
    \multirow{3}{*}{\parbox{\linewidth}{Rationalité formelle et
    optimisatrice}} \\
    \cline{1-2}
    individualisme & égoïsme & \\ \cline{1-2}
    calcul d'agrégation & optimisation marchande (offre/demande/prix) &
    \\
    \hline
\end{tabular}
\bigskip

On notera que la seule rationalité qui existe est la rationalité
utilitaire-formelle, le concept de rationalité substantielle est encore
méconnu. Le travail (le temps) devient un bien mesurable comme un autre,
ce qui provoque l'abstraction de l'expérience des travailleurs. C'est la
base de l'engrenage qui consistera à rationaliser les moyens disponibles
pour la production dans le but de créer une plus-value sur les produits.

\subsection{Taylorisme - Organisation Scientifique du Travail}

Le Taylorisme est à la base du passage de la gestion de l'organisation
du travail en véritable matière scientifique. C'est un mouvement impulsé
par Fréderic Winslow Taylor qui consiste en un ensemble de méthodes et
de principes d'organisation de la production, rassemblés sous le terme
d'O.S.T.\footnote{Organisation scientifique du travail}. Jusqu'au
Taylorisme, l'organisation est considérée comme une \enquote{boîte
noire}, mais en se basant sur le projet d'Adam Smith au niveau du
marché, Taylor va étendre les calculs aux rapports sociaux de
production. Taylor pense que la croissance de la productivité assure aux
ouvriers un meilleur salaire, réconciliant ainsi travailleurs et
patrons\footnote{Il ira plus loin en déclarant que la volonté de ne pas
atteindre le maximum de la production est équivalent à un vol du
peuple}. \newline

Taylor ne considère pas que le marché soit suffisant pour assurer
l'optimalité, c'est pourquoi il faut étudier l'organisation du travail.
L'organisation scientifique du travail, c'est la norme scientifique
objective d'organisation des moyens de production. \newline

Les 3 principes de l'O.S.T. :

\begin{enumerate}
    \item La lutte contre la \enquote{flânerie} et les savoirs de
        métier. 
        \subitem $\Rightarrow$ Suppression des corporations, remplacement des
        travailleurs qualifiés par des travailleurs non qualifiés
        entraînant la baisse du pouvoir des syndicats
    \item Le contrôle des temps productifs et la parcellisation des
        tâches.
        \subitem $\Rightarrow$ Optimisation du temps de travail grâce à
        la standardisation des tâches (descriptibles, impersonnelles,
        divisibles), un commandement de
        l'apprentissage et un pouvoir légitime.
    \item La division du travail entre concepteurs et exécutants.
        \subitem $\Rightarrow$ Transformation du rapport de forces en
        science, la hiérarchie étant des \enquote{concepteurs} et les
        ouvriers des \enquote{exécutants}.
\end{enumerate}

\bigskip
\noindent
\begin{tabular}{|p{0.25\linewidth}|p{0.25\linewidth}|p{0.25\linewidth}|p{0.25\linewidth}|}
    \hline
    \textbf{Utilitarisme originel} (Lumières anglaises) &
    \textbf{Utilitarisme élargi} (Economie de marché) &
    \textbf{Utilitarisme organisationnel} (Taylorisme) &
    \textbf{Caractéristique(s) de la rationalité} \\
    \hline
    \enquote{welfarisme} & formalisme économique & formalisme
    organisationnel & \multirow{3}{*}{\parbox{\linewidth}{Rationalité
    formelle et optimisatrice}} \\
    \cline{1-3}
    individualisme & égoïsme & intérêt \enquote{bien compris} & \\
    \cline{1-3}
    calcul d'agrégation & optimisation marchande (offre/demande/prix) &
    organisation scientifique du travail & \\
    \hline
\end{tabular}
\bigskip

\subsection{La critique marxiste}

Marx critique la division du travail en 3 niveaux :

\begin{enumerate}
    \item Le niveau philosophique : pour Marx, le travail est
        l'expérience humaine par excellence, et enlever au travailleur
        la possibilité de le gérer et de choisir sa production est une
        formé d'\textit{aliénation}. L'homme est indivisible, ce n'est
        pas un sujet comptable.
    \item Le niveau économique : le capitalisme et la division du
        travail entraîne un rapport d'exploitation car le but est de
        retirer du travail des autres une \textit{plus-value}. Dès lors,
        ça veut dire que tous les travailleurs sont sous-rémunérés par
        rapport à la valeur réelle de leur production.
    \item Le niveau sociologique : la lutte des classes, une lutte
        inégale et permanente à cause de la domination sociale des capitalistes sur
        les prolétaires.
\end{enumerate}

Cette critique a comme limite que sa vision de l'organisation du travail
est assez stéréotypée ; qu'au final Marx accepte le bien-être matériel
comme une préférence ; elle ne critique que l'organisation marchande
capitaliste (pas la non-marchande par ex.).

\end{document}
