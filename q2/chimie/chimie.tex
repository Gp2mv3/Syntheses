%%% chimie.tex --- 


%% Author: gp2mv3
%% Version: $Id: chimie.tex,v 0.0 2012/02/03 15:03:20 gp2mv3 Exp$


\documentclass[11pt,a4paper,french]{article}
\usepackage[utf8]{inputenc}
\usepackage[T1]{fontenc}
\usepackage{lmodern}
\usepackage[french]{babel}
\usepackage[version=3]{mhchem}
%%\usepackage[debugshow,final]{graphics}

%%\revision$Header: chimie.tex,v 0.0 2012/02/03 15:03:20 gp2mv3 Exp$

\pagestyle{plain}

\title{Synthèse de Chimie Q2}
\author{Beno\^it Legat}
\date{\today}

\begin{document}

\newcommand\sorb{\mathrm{s}}
\newcommand\porb{\mathrm{p}}
\newcommand\dorb{\mathrm{d}}

\maketitle
\part{Atomistique}

\subsection{\'Energie d'ionisation.}
\begin{tabular}{ll}
	$E_I$ & toujours endothermique\\
	1$^{\mathrm{st}}$ $E_A$ & g\'en\'eralement exothermique\\
	2$^{\mathrm{nd}}$ $E_A$ & toujours endothermique
\end{tabular}

\subsection{Electrons de coeur.}
Les \'electrons de coeur sont tous les \'electrons sauf les \'electrons de valences.

\begin{description}
	\item[Exemple]
		$\ce{Zn}$ de composition $[\ce{Ar}]3\dorb^{10}4\sorb^{2}$
		a 28 \'electrons de coeur car $\ce{Ar}$ a 18 \'electrons
		et la derni\'ere couche est $n=4$,
		donc $3\dorb^{10}$ n'y est pas et peut \^etre compt\'e.
		On peut aussi dire qu'il en a 18 en prenant une valence de 12.
\end{description}

\section{Th\'eorie de la liaison de valence.}
\subsection{Liaison $\sigma$.}
Toutes les liaisons covalentes {\it simples} sont des liaisons $\sigma$.

Elles ont une sym\'etrie cylindrique.
Elle n'ont aucun plan nodal cotenant l'axe internucl\'eaire.

\subsection{Hybridation.}

La promotion d'un \'electron est possile si le changement global, en tenant compte de toutes les contributions \`a l'\'energie et particuli\`erement de l'augementation du nombre de liaisons qui peuvent ainsi se former, va vers une diminution de l'\'energie.

Chaque fois qu'un atome d'une mol\'ecule a une figure de r\'epulsion t\'etra\'edrique, nous disons qu'il est hybrid\'e $\sorb\porb^3$.

\end{document}
