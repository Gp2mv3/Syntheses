%%% chimie.tex --- 


%% Author: gp2mv3
%% Version: $Id: chimie.tex,v 0.0 2012/02/03 15:03:20 gp2mv3 Exp$


\documentclass[11pt,a4paper,french]{article}
\usepackage[utf8]{inputenc}
\usepackage[T1]{fontenc}
\usepackage{lmodern}
\usepackage[french]{babel}
\usepackage[version=3]{mhchem}
%%\usepackage[debugshow,final]{graphics}

%%\revision$Header: chimie.tex,v 0.0 2012/02/03 15:03:20 gp2mv3 Exp$

\pagestyle{plain}

\title{Synthèse de Chimie Q2}
\author{Beno\^it Legat}
\date{\today}

\begin{document}

\newcommand\sorb{\mathrm{s}}
\newcommand\porb{\mathrm{p}}
\newcommand\dorb{\mathrm{d}}

\maketitle
\part{Atomistique}

\subsection{\'Energie d'ionisation.}
\begin{tabular}{ll}
	$E_I$ & toujours endothermique\\
	1$^{\mathrm{st}}$ $E_A$ & g\'en\'eralement exothermique\\
	2$^{\mathrm{nd}}$ $E_A$ & toujours endothermique
\end{tabular}

\subsection{Electrons de coeur.}
Les \'electrons de coeur sont tous les \'electrons sauf les \'electrons de valences.

\begin{description}
	\item[Exemple]
		$\ce{Zn}$ de composition $[\ce{Ar}]3\dorb^{10}4\sorb^{2}$
		a 28 \'electrons de coeur car $\ce{Ar}$ a 18 \'electrons
		et la derni\'ere couche est $n=4$,
		donc $3\dorb^{10}$ n'y est pas et peut \^etre compt\'e.
		On peut aussi dire qu'il en a 18 en prenant une valence de 12.
\end{description}

\section{Th\'eorie de la liaison de valence.}
\subsection{Liaison $\sigma$.}
Toutes les liaisons covalentes {\it simples} sont des liaisons $\sigma$.

Elles ont une sym\'etrie cylindrique.
Elle n'ont aucun plan nodal cotenant l'axe internucl\'eaire.

\subsection{D\'etermination de type de liaison.}

\begin{tabular}{lll}
	Liaison & Nombre de liaisons $\sigma$ & Nombre de liaisons $\pi$\\
	 Simple & 1 & 0\\
	 Double & 1 & 1\\
	 Triple & 1 & 2\\
\end{tabular}

\subsection{Hybridation.}

La promotion des \'electrons se produira si elle conduit globalement \`a une diminution de l'\'energie en permettant la formation d'un plus grand nombre de liaisons. On construit des orbitales hybrides sur un atime pour reproduire la figure de r\'epulstion caract\'eristique de la forme exp\'erimentale de la mol\'ecule.

Chaque fois qu'un atome d'une mol\'ecule a une figure de r\'epulsion t\'etra\'edrique, nous disons qu'il est hybrid\'e $\sorb\porb^3$.

On adopte un sch\'ema d'hybridation qui corresponde \`a la figure de r\'epulsion de la mol\'ecule. L'extension de la couche de valence n\'ecessite l'utilisation des orbitales $\dorb$.

Les liaisons multiples se forment losqu'un atome forme une liaison $\sigma$ en utilisant une orbitale hybride $\sorb\porb$ ou $\sorb\porb^2$ et une ou plusieurs liaison $\pi$ en utilisant les orbitales $\porb$ non hybrid\'ees.
Le recouvrement lat\'eral qui donne une liaison $\pi$ rend la mol\'ecule r\'esistante \`a la torsion, donne des liaisons plus faibles que les liaisons $\sigma$ et emp\^eche les atomes ayant de grands rayons de former des liaisons multiples.

\subsection{D\'etermination de l'hybridation.}
Un site de r\'epulsion est soit une pair non liante, soit une liaison simple, soit une liaison multiple.

\begin{tabular}{lll}
	Sites de r\'epulsion & Figure de r\'epulsion & Hybridation\\
					   2 & Lin\'eaire & $\sorb\porb$\\
					   3 & Triangulaire & $\sorb\porb^2$\\
					   4 & T\'etra\'edrique & $\sorb\porb^3$\\
					   5 & Bipyramide trigonale & $\sorb\porb^3\dorb$\\
					   6 & Octa\'edrique & $\sorb\porb^3\dorb^2$\\
\end{tabular}

D'autres combinaisons $\sorb$, $\porb$ et $\dorb$ peuvent donner naissance \`a des formes identiques ou diff\'erentes, mais ces combinaisons sont moins courantes.

\end{document}
