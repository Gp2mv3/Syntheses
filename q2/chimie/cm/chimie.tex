\documentclass[11pt,a4paper,french]{article}
\usepackage[utf8]{inputenc}
\usepackage[T1]{fontenc}
\usepackage{lmodern}
\usepackage{multirow}
\usepackage[french]{babel}
\usepackage[version=3]{mhchem}
\usepackage[Grey,squaren]{SIunits}
%%\usepackage[debugshow,final]{graphics}

\pagestyle{plain}

\title{Synthèse de Chimie Q2}
\author{Benoît Legat \and Lucas Nyssens}
\date{\today}

\begin{document}

\newcommand\sorb{\mathrm{s}}
\newcommand\porb{\mathrm{p}}
\newcommand\dorb{\mathrm{d}}
\newcommand\gaz{_{(g)}}
\newcommand\solid{_{(s)}}
\newcommand\liquid{_{(l)}}
\newcommand\debye{\mathrm{D}}
\newcommand\calo{\mathrm{cal}}
\newcommand\atm{\mathrm{atm}}
\newcommand\ccal{C_\mathrm{cal}}

\maketitle

\section{Notations}

De nombreuses notations seront utilisées au cours de cette synthèse.
Pour des raisons de concision, leur signification ne sera pas rappelée à chaque fois mais elle sera indiqué dans la figure \ref{fig:notations}.
\begin{figure}[h!]
	\begin{center}
		\begin{tabular}{ll}
			$\nu$ & La fréquence d'un rayonnement électromagnétique\\
			$\lambda$ & La longeur d'onde d'un rayonnement électromagnétique\\
			$c$ & La vitesse d'un rayonnement électromagnétique\\
			$c_0$ & La vitesse de la lumière\\
			$h$ & La constante de Planck\\
			$E_A$ & \'Electroaffinité d'un atome, voir section~\ref{sec:electro} page~\pageref{sec:electro}\\
			$E_I$ & \'Energie d'ionisation d'un atome, voir section~\ref{sec:ioni} page~\pageref{sec:ioni}\\
			$E_l$ & \'Energie de liaison, voir section~\ref{sec:E_l} page~\pageref{sec:E_l}\\ % FIXME: E_l ou E_D ?
			$E_N$ & \'Energie de neutralisation entre deux atomes\\
			$N_A$ & Nombre d'avogadro\\
			%$r$ & Le rayon d'un atome\\
			$C_s$ & Capacité calorifique spécifique d'un corps, voir section~\ref{sec:C_s} page~\pageref{sec:C_s}\\
			$\ccal$ & Capacité calorifique d'un corps, voir section~\ref{sec:C_cal} page~\pageref{sec:C_cal}\\
			$\Delta H$ & Variation d'enthalpie, voir section~\ref{sec:DH} page~\pageref{sec:DH}
		\end{tabular}
		\label{fig:notations}
		\caption{Répertoire des notations}
	\end{center}
\end{figure}

\part{Atomistique}

\section{L'atome}

%\subsection{\'Electrons de coeur.}
%Les électrons de coeur sont tous les électrons sauf les électrons de valences.

%\paragraph{Exemple}
%$\ce{Zn}$ de composition $[\ce{Ar}]3\dorb^{10}4\sorb^{2}$
%a 28 électrons de coeur car $\ce{Ar}$ a 18 électrons
%et la derniére couche est $n=4$,
%donc $3\dorb^{10}$ n'y est pas et peut être compté.
%On peut aussi dire qu'il en a 18 en prenant une valence de 12.

\subsection{Rayonnement électromagnétique}

On sait que
\[ c = \lambda \nu \]

Si le rayonnement est dans le vide
\[ c = c_0 \]

Un rayonnement électromagnétique ne peut transmettre de l'énergie que par {\it quanta}, qui est l'énergie transmise par un photon et qui vaut
\[ E = h\nu \]

L'énergie transmise par électron passant d'une couche à une distance $n_i$ à un couche à une distance $n_f$ vaut
\[ \Delta E = h\nu = \frac{hc}{\lambda} = R_y \left( \frac{1}{n_f^2} - \frac{1}{n_i^2} \right) \]
où $R_y = hc R_\infty = \unit{2,178 . 10^{-18}}{\joule}$ est la constante de Rydberg.

On a aussi, la longueur d'onde de Broglie
\[ \lambda = \frac{h}{m\nu} \]

\subsection{Nombres quantiques}

Un électron dans un atome voyage toujours sur orbitale.
Deux électrons ne peuvent pas être tous les deux sur la même orbitale.
Les orbitales peuvent être décrites par 4 paramètres.

\paragraph{Le nombre quantique principal}
L'électron se déplace dans son orbite le long d'une sinusoïde, $n$ détermine le nombre de période qu'il fait par tour.
Il que la circonférence de l'orbite soit un multiple de $\lambda$. D'où
\[ n\lambda = 2\pi r \]

Par la longueur d'onde de Broglie
\[ m \nu r = n \frac{h}{2\pi} \]

$n$ détermine donc aussi le rayon de l'orbitale. On voit que $n$ est proportionnel au rayon de l'orbitale.

$n$ peut prendre les valeurs $1, 2, \ldots$.

$n$ détermine la couche de l'orbitale.

La valeur de $n$ peut aussi être représenté par la période correspondante

\begin{center}
	\begin{tabular}{c|ccccc}
		$n$ & 1 & 2 & 3 & 4 & \multirow{2}{*}{$\cdots$}\\
		\cline{1-5}
		Période & K & L & M & N
	\end{tabular}
\end{center}

\paragraph{Le nombre quantique secondaire}
$l$ détermine la forme de l'orbitale.
Il y a exactement $n$ formes possible pour une orbitale de nombre quantique $n$.

{\bf La numérotation se fait}, comme en informatique, {\bf à partir de $0$}.

Pour une orbitale de nombre quantique $n$, $l$ peut donc prendre les valeurs $0, 1, \ldots, n-1$

$l$ détermine la sous-couche de l'orbitale.

La valeur de $l$ peut aussi être représenté par une lettre minuscule

\begin{center}
	\begin{tabular}{c|cccccc}
		$l$ & 0 & 1 & 2 & 3 & 4 & \multirow{2}{*}{$\cdots$}\\
		\cline{1-6}
		Lettre & s & p & d & f & g\\
		Origine & {\bf s}harp & {\bf p}rincipal & {\bf d}iffuse & {\bf f}ondamental
	\end{tabular}
\end{center}

\paragraph{Le nombre quantique magnétique}
$m_l$ détermine l'orientation relative de l'orbitale par rapport aux autres orbitales de sa sous-couche.
Il y a exactement $2l + 1$ orientation possible pour une orbitale de nombre quantique $l$.

Pour une orbitale de nombre quantique $l$, $m_l$ peut donc prendre les valeurs $-l, \ldots, -1,  0, 1, \ldots, l$

\paragraph{Le nombre quantique de spin}
$m_s$ détermine le sens dans lequel l'électron se déplace.

$m_s$ peut prendre les valeurs $-\frac{1}{2}$ et $\frac{1}{2}$.

La valeur de $m_s$ peut aussi être représenté par l'orientation du moment magnétique de l'électron dans l'orbitale correspondante

\begin{center}
	\begin{tabular}{c|cc}
		$m_s$ & $-\frac{1}{2}$ & $\frac{1}{2}$\\
		\hline
		Moment magnétique & up & down
	\end{tabular}
\end{center}

% E ^
%  n v
%  r v
%  Z ^
%  Ecrantage v
% s > p > d > f

\subsection{\'Energie d'ionisation}
\label{sec:ioni}
L'ionisation est une réaction qui consiste à enlever un électron d'un atome.

L'électron doit être pris dans l'orbitale au plus grand $n$, puis au plus grand $l$ puis commencer par les électrons appariés.

Avec ionisation, le rayon de l'atome diminue.

\subsection{\'Electroaffinité}
\label{sec:electro}
L'électroaffinité est une réaction consiste à ajouter un électron d'un atome.

Avec l'ionisation, le rayon de l'atome augmente.

\begin{center}
	\begin{tabular}{ll}
		$E_I$ & toujours endothermique\\
		1$^{\mathrm{ère}}$ $E_A$ & généralement exothermique\\
		2$^{\mathrm{ème}}$ $E_A$ & toujours endothermique
	\end{tabular}
\end{center}

\section{Liaisons intramoléculaires}

\subsection{Liaisons métalliques}
\paragraph{Caractéristiques}
\begin{itemize}
	\item entre métaux
	\item faible énergie d'ionisation
	\item faible électronégativité
	\item peu d'électrons de valence et faiblement liés au noyau
	\item délocalisation des électrons sous forme de ``gaz'', ``mer'' d'électrons libres.
	\item liaison forte et isotrope (toutes les directions)
	\item conducteurs d'électricité
	\item déformables, plastiques
\end{itemize}

\subsection{Liaisons ioniques}
\paragraph{Caractéristiques}
\begin{itemize}
	\item transfert d'électrons du cation vers l'anion
	\item liaison forte et isotrope (toutes les directions)
	\item rigidité et fragilité des solides ioniques
\end{itemize}

\paragraph{Calcul de l'énergie de liaison ionique}
\[ \ce{N + X -> N+X-} \]
\begin{center}
	\begin{tabular}{llll}
		\'Energie d'ionisation & $\ce{M -> M+ + e-}$ & $\Delta E = E_I > 0$ & endo\\
		\'Electroaffinité & $\ce{X + e- -> X-}$ & $\Delta E = E_A < 0$ & exo\\
		\'Energie de neutralisation & $\ce{M+ + X- -> M+X-}$ & $\Delta E = E_N < 0$ & exo
	\end{tabular}
\end{center}
\[ E_{\mathrm{liaison}} = E_A + E_N + E_I \]
Attention !
Pour $E_N$, certaines tables donnent son opposé, $-E_N$. Vérifiez bien que la plupart soient négatifs !

\paragraph{Calcul de l'énergie de neutralisation $E_N$}

\subparagraph{Pour les gaz}
\[ E_N = \frac{-N_A}{4\pi\epsilon_0}\times\frac{|z_1z_2|e^2}{d} \]
où $z_1$ et $z_2$ sont les valences des atomes, $e$ leur charge et $d = r_{\mathrm{anion}} + r_{\mathrm{cation}}$ la distance entre leurs centres atomiques.
\subparagraph{Pour les solides (et liquides ???)}
Interaction entre un grand nombre de cations/anions.
\[ E_N = \frac{-N_A}{4\pi\epsilon_0}\times\frac{|z_1z_2|e^2}{d} \left(1-\frac{d^*}{d}\right)A \]
où $d^* = \unit{34,5}{\pico\meter}$, $A$ est la constante de Madelung qui varie selon les composés et $d$ est comme pour les gaz.

\subsection{Liaison covalente}
\paragraph{Caractéristiques}
\begin{itemize}
	\item Partage d'électrons de valence
		\begin{center}
			\begin{tabular}{ll}
				Covalence classique & Chaque atome donne un électron célibataire\\
				Covalence dative & Un atome donne une paire d'électron non-liants
			\end{tabular}
		\end{center}
	\item[\bf Règle de l'octet]
		Tendance des atomes à compléter leur dernière couche électronique pour avoir la configuration électronique d'un gaz noble.
	\item[\bf Structure de résonance] Si plusieurs configuration de Lewis sont équivalentes, il y a délocalisation des doubles liaisons (e.g. benzène). Des liaisons intermédiaires sont alors formées (entre double et simple).
	\item La liaison peut être de deux types
		\begin{description}
			\item[Covalente pure] Doublets répartis symétriquement, moment dipolaire nul
			\item[Covalente polarisée] Formation de charges partielles: $\delta^+$ et $\delta^-$ et d'un moment dipolaire $\mu$
		\end{description}
	\item $\mu = \delta . l$ où $l$ est la distance entre les deux noyaux.
		$\vec{\mu}$ est orienté de la charge positive à la charge négative.
		$\mu[\debye]$ peut être exprimé en Debye, $\unit{4,8}{\debye} = \unit{1}{e^-} . \unit{100}{\pico\meter}$.
	\item Soit $r = \frac{\mu_{\mathrm{exp}}}{\mu_{\mathrm{théo}}}$ où $\mu_{\mathrm{théo}} = q_e l_{\mathrm{connu}} = \unit{6,1}{\debye}$
		\begin{center}
			\begin{tabular}{ll}
				Si $r \to 1$ & liaison ionique\\
				Si $r \ll 1$ & liaison covalente
			\end{tabular}
		\end{center}
\end{itemize}
\paragraph{\'Electronégativité}
L'électronégativité est le pouvoir attracteur d'un atome sur un doublet dans une molécule.
C'est une échelle relative
\begin{center}
	\begin{tabular}{ll}
		Si $\Delta \chi < 1,5$ & liaison covalente\\
		Si $\Delta \chi > 2$ & liaison ionique
	\end{tabular}
\end{center}

\paragraph{Modèle VSEPR}
\subparagraph{Site de répulsion}
Un site de répulsion est soit une liaison (simple ou multiple), soit un doublet libre.

\subparagraph{Principe}
Pour minimiser l'énergie électrostatique (due aux répulsion mutuelles de chaque site de répulsion), on éloigne au maximum chaque site.

\subparagraph{Cas}
La géométrie d'une molécule peut être déterminée directement à partir des sites de répulsions
\begin{center}
	\begin{tabular}{ccc}
		2 sites & 3 sites & 4 sites\\
		Linéaire & Triangulaire & Tétraédrique\\
		$180\degree$ & $120\degree$ & $109,5\degree$\\
	\end{tabular}
\end{center}

\subparagraph{Exceptions}
Toutes les liaisons n'entrainent pas une répulsion de la même importance, c'est pourquoi dans la pratique, les angles peuvent varier quelque peu des angles théoriques voici les liaisons par ordre croissant d'importance de répulsion
\begin{itemize}
	\item Les liaisons polarisées positivement
	\item Les liaisons simples
	\item Les liaisons polarisées négativement
	\item Les liaisons doubles
	\item Les liaisons triples
	\item les paires libres
\end{itemize}

\paragraph{Molécules polaires}
Une molécule est polaire ssi la résultante de ses moments dipôlaires est non nulle. Pour cela, il faut faire une somme vectorielle pondérée des moments dipôlaires. La résultante dépend donc de la géométrie de la molécule. Une molécule non polaire est dite apolaire.

\paragraph{Théorie de la liaison de valence}
Formation des liaisons (covalentes) par fusion d'orbitales atomiques.
\begin{description}
	\item[Liaisons sigma $\sigma$]
		\begin{itemize}
			\item Recouvrement axial d'orbitales
			\item Symétrie cylindrique
		\end{itemize}
	\item[Liaisons pi $\pi$]
		\begin{itemize}
			\item Recouvrement latéral d'orbitales
			\item Symétrie avec plan nodal
		\end{itemize}
\end{description}

\begin{center}
	\begin{tabular}{ll}
		Liaison simple & $1\sigma$ sauf $p-p$ où c'est $\pi$\\ % TODO : DAFUQ ????
		Liaison double & $1\sigma + 1\pi$\\
		Liaison triple & $1\sigma + 2\pi$
	\end{tabular}
\end{center}

\subsection{Hybridation}

Pour augmenter le nombre de liaisons possibles (trop d'électrons opposés) on fait une promotion
\begin{center}
	\begin{tabular}{ll}
		$\ce{C}$ & $[\ce{He}]2\sorb^22\porb^2$\\
		$\downarrow$ &  promotion\\
		$\ce{C}$ & $[\ce{He}]2\sorb^12\porb^3$
	\end{tabular}
\end{center}

Pour avoir 4 liaisons identiques $\sigma$, % TODO : toujours sigma ?
il faut qu'elles aient le même niveau énergétique.
On va créer une orbitale intermédiaire $\sorb\porb^3$ dont le niveau énergétique est plus élevé que celui de $\porb$ et plus bas que celui de $\sorb$.

\begin{center}
	\begin{tabular}{ll}
		$\downarrow$ &  hybridation\\
		$\ce{C}$ & $[\ce{He}]\sorb\porb^3$
	\end{tabular}
\end{center}

\paragraph{Détermination de l'hybridation}
On sait déterminer l'hybridation directement à partir du nombre de sites de répulsion

\begin{center}
	\begin{tabular}{lll}
		Sites de répulsion & Figure de répulsion & Hybridation\\
		2 & Linéaire & $\sorb\porb$\\
		3 & Triangulaire & $\sorb\porb^2$\\
		4 & Tétraédrique & $\sorb\porb^3$\\
		5 & Bipyramide trigonale & $\sorb\porb^3\dorb$\\
		6 & Octaédrique & $\sorb\porb^3\dorb^2$\\
	\end{tabular}
\end{center}

D'autres combinaisons $\sorb$, $\porb$ et $\dorb$ peuvent donner naissance à des formes identiques ou différentes, mais ces combinaisons sont moins courantes.

\subsection{\'Energie de liaison}
\label{sec:E_l}

\paragraph{Définition}
\'Energie à fournir pour casser la liaison.

\paragraph{Obtention}
Voir table.

\paragraph{Influence}
\begin{itemize}
	\item Les doublets libres provoquent une répulsion électrostatiques qui diminue $E_l$
	\item Une augmentation du rayon atomique augmente $E_l$
	\item Les liaisons $\pi$ sont plus réactives que les liaisons $\sigma$
		\[ E_l(\sigma + \pi) <  E_l(2\sigma) \]
	\item Plus $E_l$ est important, plus la molécule est stable, moins elle est réactionnelle
\end{itemize}

\section{Théorie de la liaison de valence.}
\subsection{Liaison $\sigma$.}
Toutes les liaisons covalentes {\it simples} sont des liaisons $\sigma$.

Elles ont une symétrie cylindrique.
Elle n'ont aucun plan nodal cotenant l'axe internucléaire.

\subsection{Détermination de type de liaison.}

\begin{center}
	\begin{tabular}{lll}
		Liaison & Nombre de liaisons $\sigma$ & Nombre de liaisons $\pi$\\
		Simple & 1 & 0\\
		Double & 1 & 1\\
		Triple & 1 & 2\\
	\end{tabular}
\end{center}

\subsection{Hybridation.}

La promotion des électrons se produira si elle conduit globalement à une diminution de l'énergie en permettant la formation d'un plus grand nombre de liaisons. On construit des orbitales hybrides sur un atime pour reproduire la figure de répulstion caractéristique de la forme expérimentale de la molécule.

Chaque fois qu'un atome d'une molécule a une figure de répulsion tétraédrique, nous disons qu'il est hybridé $\sorb\porb^3$.

On adopte un schéma d'hybridation qui corresponde à la figure de répulsion de la molécule. L'extension de la couche de valence nécessite l'utilisation des orbitales $\dorb$.

Les liaisons multiples se forment losqu'un atome forme une liaison $\sigma$ en utilisant une orbitale hybride $\sorb\porb$ ou $\sorb\porb^2$ et une ou plusieurs liaison $\pi$ en utilisant les orbitales $\porb$ non hybridées.
Le recouvrement latéral qui donne une liaison $\pi$ rend la molécule résistante à la torsion, donne des liaisons plus faibles que les liaisons $\sigma$ et empêche les atomes ayant de grands rayons de former des liaisons multiples.

%TODO: to here

\paragraph{Théorie des orbitales moléculaires}
% TODO

\section{Liaisons intermoléculaires}
Les liaisons intermoléculaires sont à l'origine de la cohésion du matériau. Plus il y en a et plus elles sont fortes, plus les températures de fusion ($t\degree_{\mathrm{fus}}$) et d'ébulition ($t\degree_{\mathrm{éb}}$) sont importantes.
Voici les différentes liaisons intermoléculaires, par ordre décroissant d'intensité
\begin{itemize}
	\item Liaison hydrogène (ponts $\ce{H}$)
	\item Forces de Van der Waals
		\begin{center}
			\begin{tabular}{ll}
				Keesom & dipôle permanent-permanent\\
				Debye & dipôle permanet-induit\\
				London & dipôle induit-induit
			\end{tabular}
		\end{center}
\end{itemize}

\subsection{Forces de Keesom}
\[ E_p = \frac{-\mu_1^2 \mu_2^2}{r^6} \]

\subsection{Forces de Debye}
\[ E_p = \frac{-\mu_1^2 \alpha_2}{r^6} \]
où $\alpha$ est la fonction de polarisabilité

\paragraph{Origine}
Modification de la répartition éléctronique dans une molécule apolaire induite par une molécule polaire.

\subsection{Forces de London}
\[ E_p = \frac{-\alpha_1 \alpha_2}{r^6} \]
où $\alpha$ est la fonction de polarisabilité

\paragraph{Origine}
Fluctuation rapide des répartitions électroniques dans les molécules apolaires induisent des dipôles instantannés.

\subsection{Influences}

\begin{itemize}
	\item Intensité du moment dipôlaire ($\mu$)
		\begin{center}
			\begin{tabular}{ll}
				Différence d'électronégativité & $+ \Rightarrow +$\\
				Taille des molécules & $+ \Rightarrow +$
			\end{tabular}
		\end{center}
	\item Polarisabilité ($\alpha$)
		\begin{center}
			\begin{tabular}{ll}
				Nombre d'électrons  & $+ \Rightarrow +$\\
				Rayon atomique & $+ \Rightarrow +$
			\end{tabular}
		\end{center}
	\item Taille des molécules ($+ \Rightarrow +$) % ??? + => + quoi ?
\end{itemize}

\subsection{Liaisons hydrogènes}
\paragraph{Origine}
Possible qu'entre $\ce{H}$ et $\ce{F}$, $\ce{O}$, $\ce{N}$, des atomes fortement éléctronégatifs.
Les atomes $\ce{H}$, partiellement chargé $\delta^+$ de petites tailles se rapprochent des atomes ($\ce{F}$, $\ce{O}$, $\ce{N}$) $\delta^-$. Il y a formation de ponts $\ce{H}$.

\section{Chimie du solide}
Dans cette section, nous utiliserons $a$ pour désigner la longueur du côté de la maille.

\subsection{Classification}

\paragraph{Solides amorphes}
Pas d'ordre à grande distance.

\paragraph{Solides cristallins}
Répétition à longue distance d'un motif élémentaire
\begin{center}
	\begin{tabular}{ll}
		Métalliques & Liaisons métalliques\\
		Ionique & Liaisons ioniques\\
		Covalents & Liaisons covalentes ou iono-covalentes\\
		Moléculaires & Liaisons intermoléculaires, il y a différentes molécules
	\end{tabular}
\end{center}

\paragraph{Caractéristique}
Reproduction de la maille élémentaire.
Plus petite cellule présentant tous les éléments symétriques du réseau.

\subsection{Cristaux métalliques}

\paragraph{Caractéristiques}
\begin{itemize}
	\item Atomes de taille semblable (billes)
	\item Liaisons métalliques fortes et isotropes
\end{itemize}

\paragraph{Masse volumique}
Lorqu'on connait le nombre d'atomes par maille $n_{\mathrm{at}}$ et la masse molaire de l'atome $M$,
on peut calculer la masse volumique $\rho$ par la formule suivante
\[ \rho = \frac{n_{\mathrm{at}}M}{a^3N_A} \]

\paragraph{Structures}
Il y a trois types de structures pour les cristaux métalliques
\begin{center}
	\begin{tabular}{|p{2cm}|l|l|l|}
		\hline
		Structure & Hexagonale compacte & Cubique centrée & cubique centrée\\
		\hline
		Raccourcis & HC & FCC & BCC\\
		\hline
		Empilement & A-B-A & A-B-C-A-B-C\\
		\hline
		Coordinence & 12 ($6 + 2\times3)$ & 12 ($6 + 2\times3)$ & 8 ($2 \times 4$)\\
		\hline
		Atomes par mailles & 6 & 4 & 2\\
		\hline
		$a$ & & $2\sqrt{2}r$ & $\frac{4}{\sqrt{3}}r$\\
		\hline
		Sites tétraédriques & & 8 & 12\\
		\hline
		Sites octaédriques & & 4 & 6\\
		\hline
	\end{tabular}
\end{center}

\subsection{Cristaux ioniques}

\paragraph{Caractéristiques}
\begin{itemize}
	\item Structures plus complexes
	\item Mailles électriquement neutres
	\item Taille et valence des ions différents
	\item Soit $\rho = \frac{r_\mathrm{cation}}{r_\mathrm{anion}}$
\end{itemize}

\paragraph{Structures}
Il existe deux types de structure pour les cristaux ioniques, on sait la déterminer en fonction de $\rho$ défini précédemment.

\begin{center}
	\begin{tabular}{|l|l|l|}
		\hline
		Structure & Sel gemme & Chlorure césium\\
		\hline
		Exemple & $\ce{NaCl}$ & $\ce{CsCl}$\\
		\hline
		Condition & $\rho < 0,7$ & $\rho > 0,7$\\
		\hline
		Agencement des anions & FCC & BCC\\
		\hline
		Position des cations & Dans les sites octaédriques & Au centre du cube\\
		\hline
		Coordinence & (6, 6) & (8, 8)\\
		\hline
		$a$ & $2 (r_\mathrm{anion} + r_\mathrm{cation})$ & $\frac{2}{\sqrt{3}} (r_\mathrm{anion} + r_\mathrm{cation})$\\
		\hline
	\end{tabular}
\end{center}
Une coordinence $(x, x)$ signifie que chaque anion a $x$ cations comme proches voisins et réciproquement.

\subsection{Cristaux covalents}

\paragraph{Caractéristiques}
Structure déterminée par l'orientation des liaisons.

\section{Bandes d'énergie électroniques}

Lors de la formation de molécules, il y a création d'orbitales moléculaires (1 liante et 1 anti-liante).

Avec les solides, c'est la même chose, sauf qu'il y a des orbitales avec $N$ atomes, donc qu'il y a création de $N$ orbitales moléculaires.
A chaque orbitale, on associe son niveau énergétique (bande).

Comme $N$ est très grand, beaucoup de niveaux sont proches et il y a formation de bandes $\pm$ continues séparées par des bandes (niveaux énergétiques) interdites.

La dernière bande est la bande de conduction. L'avant-dernière est la bande de valence.

\begin{itemize}
	\item Si la bande de valence est partiellement remplie, il y a des mouvements possibles d'électrons entre les états voisins, c'est un {\em conducteur}.
	\item Si la bande de valence est remplie, soit $E_g$, l'énergie nécessaire à fournir à un électron pour l'exciter et le faire passer de la bande de valence à celle de conduction.
		\begin{itemize}
			\item Si $E_g > \unit{2}{\electronvolt}$, alors la demande en énergie est beaucoup trop importante, donc il n'y aura pas (presque) d'électrons excités. C'est un {\em isolant}.
			\item Si $E_g < \unit{2}{\electronvolt}$, alors l'électron requière peu d'énergie pour passer à la bande de conduction. C'est un {\em semi-conducteur}.
		\end{itemize}
\end{itemize}

\paragraph{Dans un semi-conducteur}
Lorsque les électrons sont excités, il y a création de trous $h+$ dans la bande de valence. %FIXME: h+

Sous l'effet d'un champs électrique, les électrons et $h+$ seront accélérés, d'où création d'un courant électrique.

\paragraph{Dopage $N$ des semi-conducteurs}
On ajoute (artificiellement) des atomes pentavalents, donneurs d'électrons pour augmenter d'effet d'un champs électrique externe (conduction via les électrons).

\paragraph{Dopage $P$ des semi-conducteurs}
On a joute des atomes trivalents, accepteur d'électrons, conducteurs via $h+$.

\section{Thermochimie}

\paragraph{Première loi de la thermodynamique}
L'énergie se transforme, elle n'est ni créée, ni détruite.

L'énergie est la capacité de fournir de la chaleur ou un travail. Son unité est le Joule $[\joule]$ ou la Calorie [$\calo$].
\begin{center}
	\begin{tabular}{ll}
		$\unit{1}{\calo}$ & Quantité  d'énergie nécessaire à augmenter $\unit{1}{\gram}$ d'eau de $\unit{1}{\celsius}$\\
		$\unit{1}{\joule}$ & \'Energie nécessaire pour excercer $\unit{1}{\newton}$ sur $\unit{1}{\meter}$
	\end{tabular}
\end{center}
\[ \unit{1}{\calo} = \unit{4,18}{\joule} \]

\subparagraph{Système} Milieu réactionnel.
\subparagraph{Environnement} Extérieur du système.
\subparagraph{Système ouvert} Système effectuant des échanges de matière et d'énergie avec l'environnement.
\subparagraph{Système fermé} Système effectuant uniquement des échanges d'énergie avec l'environnement.
\subparagraph{Système isolé} Système n'effectuant aucun échange avec l'environnement.
\subparagraph{Capacité calorifique d'un corps} \label{sec:C_cal}
La capacité calorifique d'un corps est une grandeur permettant de quantifier la possibilité qu'a un corps d'absorber ou restituer de l'énergie par échange thermique au cours d'une transformation pendant laquelle sa température varie.
\[ q = \ccal \Delta T \]
où $q[\joule]$ est la chaleur fournie, $m[\gram]$ est la masse du corps et $\Delta T[\celsius]$ est la variation de température. %FIXME: sure ? gram, not kilogram ?

\subparagraph{Capacité calorifique spécifique d'un corps} \label{sec:C_s}
La capacité calorifique spécifique d'un corps est sa capacité calorifique par unité de masse $[\kilogram]$
\begin{eqnarray*}
	q &=& m C_s \Delta T\\
	C_s &=& \frac{\ccal}{m}
\end{eqnarray*}

Il y a aussi la capacité calorifique molaire $C_m = \frac{\ccal}{n}$.

\subparagraph{Enthalpie} \label{sec:DH}
Chaleur absorbée à pression constante par une réaction chimique.

L'enthalpie est mesurée par mole.
Il y a aussi l'enthalpie spécifique qui est par unité de masse et la densité d'enthalpie qui est par unité de volume.

Il faut faire attention néanmoins, dire qu'une réaction a un $\Delta H = \unit{x}{\kilo\joule\per\mole}$ est déconseillé car ce n'est pas très clair à quel composé le ``$\per\mole$'' fait référence.
Il vaut mieux écrire $\Delta H = \unit{x}{\kilo\joule}$.
On comprendra que c'est l'énergie absorbée par réaction, c'est à dire, par mole d'un composé ayant un coefficient stoechiométrique 1 dans la réaction.

\begin{figure}[h!]
	\begin{center}
		\begin{tabular}{|c|cc|c|}
			\cline{1-1} \cline{4-4}
			& \multicolumn{2}{c|}{Gaz} & \\
			\cline{2-3}
			\multicolumn{1}{|c}{} & Condensation & \multicolumn{1}{c}{$\uparrow \Delta H > 0$} & \\
			\multicolumn{1}{|c}{Déposition} & $\downarrow \Delta H < 0$ & \multicolumn{1}{c}{\'Evaporation} & $\uparrow$\\
			\cline{2-3}
			$\Delta H < 0$ & \multicolumn{2}{c|}{Liquide} & Sublimation\\
			\cline{2-3}
			\multicolumn{1}{|c}{$\downarrow$} & Cristallisation & \multicolumn{1}{c}{$\uparrow \Delta H > 0$} & $\Delta H > 0$\\
			\multicolumn{1}{|c}{} & $\downarrow \Delta H < 0$ & \multicolumn{1}{c}{Fusion} & \\
			\cline{2-3}
			& \multicolumn{2}{c|}{Solide} &\\
			\cline{1-1} \cline{4-4}
		\end{tabular}
	\end{center}
	\label{fig:state}
	\caption{Changements d'état de la matière}
\end{figure}

Après avoir atteint la température d'ébullition (resp. de fusion), il faut encore fournir un certaine quantité d'énergie pour le passage de l'état liquide (resp. solide) à l'état gazeux (resp. liquide).
Ce sont les enthalpies de vaporisation (resp. de fusion).

Ces enthalpies doivent être prise en compte (avec le signe opposé) pour la condensation et la cristallisation également.

Pour faire passer de l'eau à $\unit{25}{\celsius}$ liquide à de l'eau à $\unit{100}{\celsius}$ gazeuse, il faut fournir $\Delta H = \unit{44}{\kilo\joule\per\mole}$.

%TODO: trouver une place pour le texte d'ici
La combustion est une réaction exothermique.

\begin{figure}[h!]
	\begin{center}
		\begin{tabular}{|lllll|}
			\hline
			\multirow{2}{*}{Condition} & normale & \multirow{2}{*}{de température et de pression} & \multirow{2}{*}{$\unit{1}{\atm}$} & $\unit{0}{\celsius}$\\
			& standard & & & $\unit{25}{\celsius}$\\
			\hline
		\end{tabular}
	\end{center}
	\label{fig:cntp}
	\caption{Conditions normales et standards}
\end{figure}
%TODO: à ici

\paragraph{Deuxième principe de la thermochimie}
Une équation chimique peut être inversée à condition de changer le signe de l'enthalpie.

\paragraph{Loi de HESS}
L'enthalpie d'une réaction globale est la somme des enthalpies des réactions intermédiaires possibles (même si ces étapes réactionnelles sont théoriques).

\begin{figure}[h!]
	\begin{center}
		\begin{tabular}{|ll|}
			\hline
			Enthalpie standard & $\kilo\joule\per\mole$\\
			Enthalpie spécifique & $\kilo\joule\per\gram$\\
			Densité d'enthalpie & $\kilo\joule\per\liter$\\
			\hline
		\end{tabular}
	\end{center}
	\label{fig:enthunit}
	\caption{Unité des différents types d'enthalpie}
\end{figure}

\paragraph{\'Energie d'atomisation}
A partir d'un corps pur simple stable (trouvé dans la nature), elle correspond à l'énergie qu'il faut pour le rendre réactif.

\subparagraph{Exemples}
Voici deux exemples d'énergie d'atomisation
\begin{center}
	\begin{tabular}{ll}
		$\ce{1/2O2\gaz} \ce{-> O\gaz}$ & \'Energie des liaisons\\
		$\ce{Na}_{(s)} \ce{-> Na\gaz}$ & \'Energie de sublimation
	\end{tabular}
\end{center}

\paragraph{\'Energie de formation}
\subparagraph{Postulat}
Les énergies de formation des molécules (corps simples) trouvés dans la nature sont nulles ($\ce{Cu}_{(s)}$, $\ce{O2\gaz}$, \ldots)

\`A partir de molécules, corps purs simples, trouvés dans la nature, correspon à l'énergie nécessaire pour en obtenir des molécules plus complexes connues, $\ce{H4}$, $\ce{C6H6}$, $\ce{CO2}$, \ldots {\it à l'état après formation}.

\subparagraph{Exemples}
\begin{eqnarray*}
	\ce{C}_{(s)} \ce{+ 2H2\gaz} & \ce{->} & \ce{CH4}\\ % FIXME: state of CH4 ?
	\ce{C}_{(s)} \ce{+ O_2 \gaz} & \ce{->} & \ce{CO2} % FIXME: state of CO2 ?
\end{eqnarray*}

\paragraph{\'Energie de formation de liaisons}
\'Energie nécessaire à faire passer une molécule complexe en un ensemble d'atomes réactifs. % FIXME: Réactif ? Pas stables ?

C'est la somme des liaisons covalentes présentes dans la molécule complexe. Correspond aussi à la somme de l'énergie de formation de la molécule et de l'énergie d'atomisation des corps purs simples obtenus.

\subparagraph{Exemples}
Méthane ($\ce{CH4}$)
\begin{figure}[h!]
	\begin{center}
		\begin{tabular}{clc|cl|}
			\cline{4-5}
			& & $\ce{2H\gaz + O\gaz}$ & \multirow{4}{*}{$\updownarrow$} & \'Energie de\\
			\cline{3-3}
			& & \multicolumn{1}{|c}{$\updownarrow$ \'Energie d'atomisation} & & formation de\\
			\cline{1-3}
			\multicolumn{1}{|l}{\'Energie} & \multicolumn{1}{c|}{} & $\ce{H2\gaz + 1/2O2\gaz}$ & & liaisons (2\\
			\multicolumn{1}{|l}{de} & \multicolumn{1}{c|}{$\updownarrow$} & $\ce{H2O\gaz}$ & & liaisons $\ce{O-H}$)\\
			\cline{4-5}
			\multicolumn{1}{|l}{formation} & \multicolumn{1}{c|}{} & \multicolumn{1}{c}{$\ce{H2O\liquid}$} & & \multicolumn{1}{l}{}\\
			\cline{1-2}
		\end{tabular}
	\end{center}
	\label{fig:state}
	\caption{\'Energie de formation de l'eau}
\end{figure}

\end{document}
