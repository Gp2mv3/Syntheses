\documentclass[11pt,a4paper]{article} % french
\usepackage[utf8]{inputenc}
\usepackage{amsfonts}
%\usepackage[french]{babel}

\newtheorem{defin}{Definition}[section]
\newtheorem{nota}[defin]{Notation}
\newtheorem{prop}[defin]{Propriete}

\title{Mathematique Q2}
\author{Benoit Legat \and Nicolas Cognaux}

\begin{document}
\maketitle

\newcommand\dist{\mathrm{dist}}
\newcommand\Ker{\mathrm{Ker}}
\renewcommand\Im{\mathrm{Im}}

\section{Espaces euclidiens}
\begin{defin}[Espace euclidien]
	Un espace euclidien $E$ est un espace vectoriel réel muni d'une fonction
	$(-|-) : E \times E \to \mathbb{R}$ qui est
	\begin{description}
		\item[Bilineaire]
			$\forall x, y, z \in E, \alpha, \beta \in \mathbb{R}$
			\begin{eqnarray*}
				(\alpha x + \beta y | z) & = & \alpha (x | z) + \beta (y | z)\\
				(x | \alpha y + \beta z) & = & \alpha (x | y) + \beta (x | z)
			\end{eqnarray*}
		\item[Symetrique]
			$\forall x,y \in E$
			$$(x|y) = (y|x)$$
		\item[Defini positif]
			$\forall x \in E \setminus \{0\}$
			$$(x|x) > 0$$
	\end{description}
	C'est le produit scalaire.
\end{defin}

\begin{nota}
	Soient $E$ un espace euclidien, $V \subseteq E$ et $x,y \in E$.
	\begin{eqnarray*}
		||x|| & \stackrel{\Delta}{=} & \sqrt{(x|x)}\\
		\dist(x, y) & \stackrel{\Delta}{=} & ||x - y||\\
		x \perp y & \stackrel{\Delta}{\Leftrightarrow} & (x|y) = 0\\
		V^{\perp} & \stackrel{\Delta}{=} & \left\{z \in E | z \perp v, \forall v \in V\right\}
	\end{eqnarray*}
\end{nota}

\begin{prop}
	Soient $E$ un espace euclidien, $V \subseteq E$ et $x,y \in E$.
	\begin{itemize}
		\item $||x|| \geq 0 \land dist(x, y) \geq 0$
		\item $(x \neq 0 \Rightarrow ||x|| > 0) \land (x \neq y \Rightarrow \dist(x, y) > 0)$
		\item $||\alpha x|| = \alpha||x||$
		\item $|(x | y)| \leq ||x||\times||y||$ (Inegalite de Cauchy)
		\item $||x + y|| \leq ||x|| + ||y||$ (Inegalite triangulaire)
		\item $V \cap V^{\perp} = \{0\}$
		\item $E^{\perp} = \{0\} \land \{0\}^{\perp} = E$
		\item $V^{\perp}$ est un sev de $E$
	\end{itemize}
\end{prop}

\begin{defin}
	Soient $E$ un espace euclidien, $V$ un sev de $E$ et $x \in E$.
	La projection orthogonale de $x$ sur $V$ est un vecteur $P_V(x)$ tel que
	\begin{enumerate}
		\item $P_V(x) \in V$
		\item $x - P_V(x) \in V^{\perp}$
	\end{enumerate}
\end{defin}

\begin{prop}
	Soient $E$ un espace euclidien, $V$ un sev de $E$ et $x \in E$.
	\begin{itemize}
		\item $P_V(x)$ existe et est unique.
		\item $y \neq P_V(x) \Rightarrow \dist(x, P_V(x)) < \dist(x, y)$.
		\item $P_V : E \to E$ est une application lineaire.
		\item $\Ker P_V = V^{\perp} \land \Im P_V = V$.
	\end{itemize}
\end{prop}

%% Ayant raté la deuxieme heure de cours je sais pas trop où il s'arrete, je commence directement au deuxieme cours. %%

\subsection{Existence de la projection hortogonale}
\begin{defin}
  Soient $E$ un espace Euclidien et $x_1, x_2,... ,x_n \in E$,
  \begin{enumerate}
  \item La famille $x_1, x_2,... ,x_n$ est orthogonale si:
    \begin{itemize}
    \item $x \neq 0, \forall x \in \_1,x_2,...,x_n$
    \item $(x_i|x_j) = 0 \forall i, j \in [1, n]$
    \end{itemize}

  \item La famille $x_1, x_2,... ,x_n$ est orthonormée si:
    \begin{itemize}
    \item $||x_i|| = 1$ $\forall i \in [1, n]$
    \item $(x_i|x_j) = 0$ $\forall i, j \in [1, n]$ %% BUG dans mes notes ?
    \end{itemize}
  \end{enumerate}
\end{defin}
Remarque: Une famille orthonormée est d'office orthogonale.

\begin{prop}
  Soient: $E$ un espace vectoriel, $v$ un sous-espace vectoriel de $E$, $u_1, u_2,..., u_n$ une base orthonormée de $V$.
  $\forall x \in E, P_v(x)$ existe et: $P_v(x) = x(u_1|u_1)+x(u_2|u_2)+...+x(u_n|u_n)$
  Commentaires:
  \begin{itemize}
    \item Si on a une base orthonormée, alors $P_v(x)$ existe, maintenant, a-t-on une base orthonormée ? (A prouver)
    \item L'hypothèse (base: $u_1, u_2,..., u_n$) doit être une base de $V$, $E$ est de dimension trop importante.
  \end{itemize}
  
  
  
\end{prop}


\end{document}
