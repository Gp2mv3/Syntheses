\documentclass[11pt,a4paper]{article} % french
\usepackage[utf8]{inputenc}
\usepackage{amsfonts}
\usepackage{amsmath}
%\usepackage{proof}
%\usepackage[french]{babel}

\newtheorem{defin}{Definition}[section]
\newtheorem{nota}[defin]{Notation}
\newtheorem{prop}[defin]{Propriete}

\title{Mathematique Q2}
\author{Benoit Legat}

\begin{document}
\maketitle

\newcommand\dist{\mathrm{dist}}
\newcommand\Ker{\mathrm{Ker}}
\renewcommand\Im{\mathrm{Im}}

\begin{enumerate}
	\item
		\begin{enumerate}
			\item On a
				\begin{eqnarray*}
					(1-x^2|1-x^2) & = & \int_0^1 (1-x^2)(1-x^2) dx\\
					              & = & \int_0^1 1 - 2x^2 + x^4 dx\\
					              & = & \left[x - \frac{2}{3}x^3 + \frac{1}{5}x^5\right]_0^1\\
					              & = & 1 - \frac{2}{3} + \frac{1}{5}\\
					              & = & \frac{8}{15}
				\end{eqnarray*}
				Des lors
				\begin{eqnarray*}
					||1-x^2|| & = & \sqrt{(1-x^2|1-x^2)}\\
					          & = & \sqrt{\frac{8}{15}}\\
					          & = & \frac{2\sqrt{30}}{15}
				\end{eqnarray*}
			\item On a
				\begin{eqnarray*}
					\left(\exp\left(\frac{x}{2}\right)\left|\exp\left(\frac{x}{2}\right)\right.\right)
					& = & \int_0^1 \exp\left(\frac{x}{2}\right)\exp\left(\frac{x}{2}\right) dx\\
					& = & \int_0^1 \exp(x) dx\\
					& = & \left[\exp(x)\right]_0^1\\
					& = & e - 1
				\end{eqnarray*}
				Des lors
				\begin{eqnarray*}
					\left|\left|\exp\left(\frac{x}{2}\right)\right|\right|
					& = & \sqrt{\left(\exp\left(\frac{x}{2}\right)\left|\exp\left(\frac{x}{2}\right)\right.\right)}\\
					& = & \sqrt{e - 1}
				\end{eqnarray*}
		\end{enumerate}
	\item
	\item
		Posons $x = \begin{pmatrix}1\\2\\3\end{pmatrix}$.
		Il y a 3 manieres de resoudre ce probleme, les voici
		\begin{description}
			\item[Geometrique]
				Profitons du fait qu'on travaille dans $\mathbb{R}^3$ et que le produit scalaire choisi est le produit scalaire canonique.
				Soit $n$ la normale a $\Pi$ passant par $x$. On a alors que $P_V(x) \in \Pi \cap n$. Calculons $n$
				\begin{eqnarray*}
					n & \equiv &
						\left\{
							\begin{array}{rcl}
								x & = & k + 1\\
								y & = & k + 2\\
								z & = & k + 3
							\end{array}
						\right.\\
					& \equiv &
						\left\{
							\begin{array}{rcl}
								x - y & = & -1\\
								x - z & = & -2
							\end{array}
						\right.
				\end{eqnarray*}
				On peut a present calculer $\Pi \cap n$
				\begin{eqnarray*}
					\Pi \cap n & \equiv &
						\left\{
							\begin{array}{rcl}
								x + y + z & = & 0\\
								x - y & = & -1\\
								x - z & = & -2
							\end{array}
						\right.\\
					& = & \left\{\begin{pmatrix}-1\\0\\1\end{pmatrix}\right\}
				\end{eqnarray*}
				D'ou
				\[
					P_V(x) = \begin{pmatrix}-1\\0\\1\end{pmatrix}
				\]
			\item[Definition]
				On doit respecter
				\begin{eqnarray}
					P_V(x) & \in & \Pi\label{eq_direct}\\
					x - P_V(x) & \in & \Pi^{\perp}\label{eq_perp}
				\end{eqnarray}
				On remarque que
				\begin{eqnarray*}
					\Pi & = & <\begin{pmatrix}-1\\1\\0\end{pmatrix}, \begin{pmatrix}-1\\0\\1\end{pmatrix}>\\
					\Pi^{\perp} & = & <\begin{pmatrix}1\\1\\1\end{pmatrix}>
				\end{eqnarray*}
				On sait des lors par \eqref{eq_direct} que $\exists a, b \in \mathbb{R}$ tel que
				\[
					P_V(x) = a\begin{pmatrix}-1\\1\\0\end{pmatrix} + b\begin{pmatrix}-1\\0\\1\end{pmatrix}
				\]
				\eqref{eq_perp} nous dit alors que $\exists c \in \mathbb{R}$ tel que
				\[
					\begin{pmatrix}1\\2\\3\end{pmatrix} - a\begin{pmatrix}-1\\1\\0\end{pmatrix} - b\begin{pmatrix}-1\\0\\1\end{pmatrix}
					= c\begin{pmatrix}1\\1\\1\end{pmatrix}
				\]
				C'est a dire
				\[
					\begin{pmatrix}-1&-1&1\\1&0&1\\0&1&1\end{pmatrix}\begin{pmatrix}a\\b\\c\end{pmatrix}
					= \begin{pmatrix}1\\2\\3\end{pmatrix}
				\]
				D'ou $\begin{pmatrix}a\\b\\c\end{pmatrix} = \begin{pmatrix}0\\1\\2\end{pmatrix}$ ce qui nous donne
				\[
					P_V(x) = \begin{pmatrix}-1\\0\\1\end{pmatrix}
				\]
			\item[Gram-Schmidt]
		\end{description}
	\item
	\item
	\item
		%\begin{proof}
		\begin{description}
			\item[$\Rightarrow$]
				On a successivement
				\begin{eqnarray*}
					||x + y||^2 & = & ||x||^2 + ||x||^2\\
					(x+y|x+y)   & = & (x|x) + (y|y)\\
					(x|x+y) + (y|x+y) & = & (x|x) + (y|y)\\
					(x|x) + (x|y) + (y|x) + (y|y) & = & (x|x) + (y|y)\\
					2(x|y) & = & 0
				\end{eqnarray*}
				D'ou $(x|y) = 0$.
			\item[$\Leftarrow$]
				On calcule
				\begin{eqnarray*}
					||x + y||^2 & = & (x+y|x+y)\\
					& = & (x|x+y) + (y|x+y)\\
					& = & (x|x) + (x|y) + (y|x) + (y|y)\\
					& = & (x|x) + 2(x|y) + (y|y)
				\end{eqnarray*}
				Comme $(x|y) = 0$
				\begin{eqnarray*}
					||x + y||^2 & = & (x|x) + 2(x|y) + (y|y)\\
					& = & (x|x) + (y|y)
				\end{eqnarray*}
		\end{description}
		%\end{proof}
\end{enumerate}

\end{document}
